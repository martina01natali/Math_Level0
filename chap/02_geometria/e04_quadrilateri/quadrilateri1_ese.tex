% Copyright (c) 2015 Daniele Masini - d.masini.it@gmail.com
% Copyright (c) 2016 Daniele Zambelli - daniele.zambelli@gmail.com

\section{Esercizi}

\subsection{Esercizi riepilogativi}

\begin{esercizio}
\label{ese:4.1}
Quali tra le seguenti sono proprietà del parallelogrammo?
\begin{enumeratea}
\item Ciascuna diagonale lo divide in due triangoli 
uguali\hfill\boxV\quad\boxF
\item Gli angoli opposti sono uguali\hfill\boxV\quad\boxF
\item Tutti i lati sono uguali\hfill\boxV\quad\boxF
\item Gli angoli sulla base sono uguali\hfill\boxV\quad\boxF
\item Le diagonali sono perpendicolari\hfill\boxV\quad\boxF
\item Gli angoli sono tutti congruenti\hfill\boxV\quad\boxF
\item Le diagonali sono anche bisettrici\hfill\boxV\quad\boxF
\end{enumeratea}
\end{esercizio}

\begin{esercizio}
\label{ese:4.2}
Vero o Falso?
\begin{enumeratea}
\item Un quadrilatero che ha i lati consecutivi a due a due 
congruenti è un deltoide\hfill\boxV\quad\boxF
\item Un quadrilatero che ha una sola coppia di lati opposti uguali è 
un trapezio\hfill\boxV\quad\boxF
\item Il trapezio scaleno ha tutti i lati diversi tra di loro per 
lunghezza\hfill\boxV\quad\boxF
\item Gli angoli adiacenti alla base maggiore di un trapezio 
rettangolo sono uno retto e uno acuto\hfill\boxV\quad\boxF
\item Un trapezio scaleno può avere due angoli opposti 
ottusi\hfill\boxV\quad\boxF
\item In un trapezio isoscele gli angoli adiacenti alla base minore 
sono ottusi\hfill\boxV\quad\boxF
\item In un trapezio isoscele sono congruenti le proiezioni dei lati 
obliqui sulla base maggiore\tab\hfill\boxV\quad\boxF
\item Le diagonali di un deltoide si incontrano nel loro punto medio 
comune\hfill\boxV\quad\boxF
\item Nel parallelogramma gli angoli adiacenti allo stesso lato sono 
supplementari\hfill\boxV\quad\boxF
\item Nel parallelogramma una delle due diagonali lo divide in due 
triangoli isosceli\tab\tab\tab\hfill\boxV\quad\boxF
\item Se le diagonali di un quadrilatero si dividono a metà allora è 
un parallelogramma\tab\tab\hfill\boxV\quad\boxF
\item Le diagonali del rombo sono anche 
bisettrici\hfill\boxV\quad\boxF
\item Se le diagonali di un parallelogramma sono uguali il 
parallelogramma è un quadrato\tab\tab\hfill\boxV\quad\boxF
\item Un parallelogramma che ha un angolo retto è un 
rettangolo\hfill\boxV\quad\boxF
\item Un parallelogramma che ha due lati consecutivi congruenti è un 
quadrato\hfill\boxV\quad\boxF
\item Un quadrilatero con due lati opposti congruenti è un 
trapezio\hfill\boxV\quad\boxF
\item Il rombo è anche un rettangolo\hfill\boxV\quad\boxF
\item Il rombo è anche quadrato\hfill\boxV\quad\boxF
\item Il rettangolo è anche parallelogrammo\hfill\boxV\quad\boxF
\item Il quadrato è anche rombo\hfill\boxV\quad\boxF
\item Il trapezio è anche parallelogrammo\hfill\boxV\quad\boxF
\item Alcuni rettangoli sono anche rombi\hfill\boxV\quad\boxF
\end{enumeratea}
\end{esercizio}

\begin{multicols}{2}

\subsubsection*{Dimostra le seguenti proprietà}

\begin{esercizio}
	\label{ese:4.10}
	In un triangolo $ABC$ prolunga la mediana $AM$ di un segmento $MD$ 
	congruente ad $AM$. Dimostra che il quadrilatero $ABCD$ è un 
	parallelogramma.
\end{esercizio}

\begin{esercizio}
	\label{ese:4.11}
	Sia $ABCD$ un parallelogramma, siano $M$, $N$, $O$ e $P$ i punti medi 
	dei lati. Dimostra che $MNOP$ è un parallelogramma.
\end{esercizio}

\begin{esercizio}
	\label{ese:4.13}
	Nel parallelogramma $ABCD$ si prendono sui lati opposti $AB$ e $CD$ i 
	punti $E$ ed $F$ tali che $AE$ sia congruente a $CF$. Dimostra che 
	anche $AECF$ è un parallelogramma.
\end{esercizio}

\begin{esercizio}
	\label{ese:4.14}
	Di un triangolo $ABC$ prolunga i lati $AB$ e $CB$ rispettivamente di 
	due segmenti $BD$ e $BE$ tali che $AB\cong BD$ e $CB\cong BE$. 
	Dimostra che $ACDE$ è un parallelogramma.
\end{esercizio}

\begin{esercizio}
	\label{ese:4.17}
	Dato un parallelogramma $ABCD$ prolunga il lati nel seguente modo: 
	$CD$ di un segmento $DE$, $DA$ di un segmento $DF$, $AB$ di un 
	segmento $BG$, $BC$ di un segmento $CH$. Dimostra che se $DE\cong 
	AF\cong BG\cong CH$ allora $EFGH$ è anche un parallelogramma.
\end{esercizio}

\begin{esercizio}
	\label{ese:4.18}
	Dato un segmento $AB$, sia $M$ il suo punto medio. Traccia 
	rispettivamente da $A$ e da $B$ le rette $r$ ed $s$ parallele tra 
	loro. Dal punto $M$ traccia una trasversale $t$ alle due rette che 
	incontra $r$ in $C$ ed $s$ in $D$. Dimostra che $CADB$ è un 
	parallelogramma.
\end{esercizio}

\begin{esercizio}
	\label{ese:4.19}
	Dimostra che in un parallelogramma $ABCD$ i due vertici opposti $A$ e 
	$C$ sono equidistanti dalla diagonale $BD$.
\end{esercizio}

\begin{esercizio}
	\label{ese:4.50}
	Che tipo di quadrilatero si ottiene congiungendo i punti medi dei 
	lati di un rombo?
\end{esercizio}

\begin{esercizio}
	\label{ese:4.51}
	Che tipo di quadrilatero si ottiene congiungendo i punti medi dei 
	lati di un rettangolo?
\end{esercizio}

\begin{esercizio}
\label{ese:4.8}
Se un trapezio ha tre lati congruenti, le diagonali sono bisettrici 
degli angoli adiacenti alla base maggiore.
\end{esercizio}

\begin{esercizio}
\label{ese:4.9}
Dimostra che un rombo è diviso da una sua diagonale in due triangoli 
isosceli congruenti.
\end{esercizio}

\begin{esercizio}
\label{ese:4.21}
Nel parallelogramma $ABCD$ sia $M$ il punto medio di $AB$ ed $N$ il 
punto medio di $DC$. Sia $P$ il punto di intersezione di $AN$ con 
$DM$ e $Q$ il punto di intersezione di $CM$ con $BN$. Dimostra che 
$PNAM$ è un rombo.
\end{esercizio}

\begin{esercizio}
\label{ese:4.22}
Dimostra che se un rombo ha le diagonali congruenti allora è un 
quadrato.
\end{esercizio}

\begin{esercizio}
\label{ese:4.23}
Dimostra che congiungendo i punti medi dei lati di un rettangolo si 
ottiene un rombo.
\end{esercizio}

\begin{esercizio}
\label{ese:4.27}
In un trapezio $ABCD$ la diagonale $AC$ è congruente alla base 
maggiore $AB$. Sia $M$ il punto medio del lato obliquo $BC$. Prolunga 
$AM$ di un segmento $ME$ congruente ad $AM$. Dimostra che $ABEC$ è un 
rombo.
\end{esercizio}

\begin{esercizio}
\label{ese:4.28}
Nel trapezio isoscele $ABCD$ con la base maggiore doppia della base 
minore, unisci il punto medio $M$ di $AB$ con gli estremi della base 
$DC$. Dimostra che $AMCD$ è un parallelogramma.
\end{esercizio}

\begin{esercizio}
\label{ese:4.29}
Nel trapezio isoscele $ABCD$ i punti $M$ e $N$ sono rispettivamente i 
punti medi delle basi $AB$ e $DC$. Dimostra che $MNCB$ è un trapezio 
rettangolo.
\end{esercizio}

\begin{esercizio}
\label{ese:4.30}
Siano $M$ e $N$ i punti medi dei lati obliqui di un trapezio isoscele 
$ABCD$. Dimostra che $BCMN$ è un trapezio isoscele.
\end{esercizio}

\begin{esercizio}
\label{ese:4.32}
Dimostra che le proiezioni dei lati obliqui di un trapezio isoscele 
sulla base maggiore sono congruenti.
\end{esercizio}

\begin{esercizio}
\label{ese:4.33}
Nel triangolo isoscele $ABC$, di base $BC$, traccia le bisettrici 
agli angoli adiacenti alla base. Detti $D$ ed $E$ i punti di incontro 
di dette bisettrici rispettivamente con $AC$ e $AB$, dimostra che 
$EBCD$ è un trapezio isoscele.
\end{esercizio}

\begin{esercizio}
\label{ese:4.41}
Dato un quadrato $ABCD$ di centro $O$, siano $H$ e $K$ due punti 
sulla diagonale $AC$ simmetrici rispetto ad $O$. Dimostra che il 
quadrilatero $BHDK$ è un rombo. 
\end{esercizio}

\begin{esercizio}
\label{ese:4.48}
Le diagonali di un trapezio isoscele dividono il trapezio in quattro 
triangoli, dei quali due triangoli sono isosceli e aventi gli angoli 
ordinatamente congruenti, mentre gli altri due triangoli sono 
congruenti.
\end{esercizio}

\begin{esercizio}
\label{ese:4.59}
Sia $AD$ bisettrice del triangolo $ABC$. Da $D$ traccia le parallele 
ai lati $AB$ e $AC$, detto $E$ il punto di intersezione del lato $AC$ 
con la parallela ad $AB$ ed $F$ il punto di intersezione del lato 
$AB$ con la parallela ad $AC$, dimostra che $AEDF$ è un rombo.
\end{esercizio}

\end{multicols}

\noindent\begin{minipage}{0.6\textwidth}\parindent15pt
\begin{esercizio}[Prove invalsi 2003]
\label{ese:4.60}
Il quadrilatero nella figura a fianco è simmetrico rispetto alla 
retta $AC$.
Sapendo che $B\widehat{A}C = 30\grado$ e $C\widehat{D}A = 70\grado$, 
quanto vale $B\widehat{C}D$?
\begin{enumeratea}
\item $140\grado$;
\item $150\grado$;
\item $160\grado$;
\item $165\grado$;
\item Le informazioni sono insufficienti.
\end{enumeratea}
\end{esercizio}
\end{minipage}\hfil
\begin{minipage}{0.4\textwidth}
	\centering\input{\folder lbr/fig019_ese.pgf}
\end{minipage}

\begin{esercizio}[Prove invalsi 2003]
\label{ese:4.61}
Quale fra le seguenti proprietà è falsa per tutti i parallelogrammi?
\begin{enumeratea}
\item I lati opposti sono uguali.
\item Gli angoli adiacenti sono supplementari.
\item Gli angoli opposti sono supplementari.
\item I lati opposti sono paralleli.
\item Le diagonali si dimezzano scambievolmente.
\end{enumeratea}
\end{esercizio}

\begin{esercizio}[Prove invalsi 2004]
\label{ese:4.62}
Quale tra le seguenti affermazioni riferite ad un parallelogramma 
qualsiasi è FALSA?
\begin{enumeratea}
\item I lati opposti sono paralleli.
\item Le diagonali sono uguali.
\item Gli angoli opposti sono uguali.
\item Ogni diagonale divide il parallelogramma in due triangoli 
uguali.
\end{enumeratea}
\end{esercizio}

\begin{esercizio}[Prove invalsi 2005]
\label{ese:4.63}
Quale tra le seguenti affermazioni relative ad un rombo è FALSA?
\begin{multicols}{2}
\begin{enumeratea}
\item Non ha i lati opposti paralleli.
\item Ha tutti i lati uguali.
\item Ha gli angoli opposti uguali.
\item Ha le diagonali perpendicolari.
\end{enumeratea}
\end{multicols}
\end{esercizio}

\begin{esercizio}[Prove invalsi 2005]
\label{ese:4.64}
Quale fra le seguenti condizioni è sufficiente affinché un 
quadrilatero sia un rettangolo?
\begin{enumeratea}
\item I lati opposti siano uguali e un angolo sia retto.
\item Le diagonali si dividano a metà.
\item I lati opposti siano paralleli.
\item Le diagonali siano uguali e un angolo sia retto.
\end{enumeratea}
\end{esercizio}

\begin{esercizio}[Prove invalsi 2006]
\label{ese:4.65}
Quale fra le seguenti affermazioni è vera?
Il quadrilatero avente i vertici nei punti medi dei lati di \ldots{}
\begin{enumeratea}
\item \ldots{} un rettangolo qualsiasi è sempre un quadrato.
\item \ldots{} un trapezio isoscele qualsiasi è un rettangolo.
\item \ldots{} un quadrilatero qualsiasi è un parallelogramma.
\item \ldots{} un quadrato è un rombo, ma non un quadrato.
\end{enumeratea}
\end{esercizio}

\begin{esercizio}[Prove invalsi 2007]
\label{ese:4.66}
Quale fra le seguenti affermazioni è falsa?
\begin{enumeratea}
\item Ogni rettangolo è anche un rombo.
\item Ogni rettangolo è anche un parallelogramma.
\item Ogni quadrato è anche un rombo.
\item Ogni rettangolo ha le diagonali uguali.
\end{enumeratea}
\end{esercizio}

\begin{esercizio}[Prove invalsi 2007]
\label{ese:4.67}
\`E dato un quadrilatero con le diagonali perpendicolari che si 
dimezzano scambievolmente.\\
Alberto afferma: <<Di sicuro si tratta di un quadrato.>>\\
Barbara afferma: <<Non è detto che sia un quadrato, ma di sicuro è un 
rombo.>>\\
Carla afferma: <<Non è detto che sia un quadrato, ma di sicuro è un 
rettangolo.>>\\
Daniele afferma: <<Si tratta certamente di un quadrilatero a forma di 
aquilone.>>\\
Chi ha ragione?
\begin{multicols}{4}
\begin{enumeratea}
\item Alberto;
\item Barbara;
\item Carla;
\item Daniele.
\end{enumeratea}
\end{multicols}
\end{esercizio}

\subsection{Risposte}

\begingroup
\hypersetup{linkcolor=black}

\paragraph{\ref{ese:4.1}.}
a)~V,\quad b)~V,\quad c)~F,\quad d)~F,\quad e)~F,\quad f)~F,\quad 
g)~F.

\paragraph{\ref{ese:4.2}.}
a)~F,\quad b)~F,\quad c)~V,\quad d)~V,\quad e)~F,\quad f)~V,\quad 
g)~V,\quad h)~F,\quad i)~V,\quad j)~F,\quad k)~F,\quad l)~V,\quad 
m)~F,\quad n)~V,\quad o)~V,\quad p)~F,\quad q)~F,\quad r)~F,\quad 
s)~V,\quad t)~V,\quad u)~F,\quad v)~V.

\paragraph{\ref{ese:4.60}.}
c.

\paragraph{\ref{ese:4.61}.}
c.

\paragraph{\ref{ese:4.62}.}
c.

\paragraph{\ref{ese:4.63}.}
a.

\paragraph{\ref{ese:4.64}.}
a.

\paragraph{\ref{ese:4.65}.}
c.

\paragraph{\ref{ese:4.66}.}
a.

\paragraph{\ref{ese:4.67}.}
b.

\endgroup
