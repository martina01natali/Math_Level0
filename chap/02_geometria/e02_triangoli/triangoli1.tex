a% Copyright (c) 2015 Daniele Masini - d.masini.it@gmail.com

\chapter{Congruenza nei 
triangoli}\label{chap:congruenza_nei_triangoli}

\includegraphics[width=0.95\textwidth]{\folder img/triangle_shapes.jpg}
  \begin{center}
    {\large ``Triangle Shapes''}\par
    Foto di maxtodorov\par
    \url{http://www.flickr.com/photos/maxtodorov/3066505212/}\par
    Licenza: Creative Commons Attribution\par
  \end{center}
\newpage

\section{Definizioni relative ai 
triangoli}\label{sect:definizioni_triangoli}

Definiamo gli elementi principali di un triangolo
\begin{definizione}~
\begin{itemize*}
\item Un \emph{triangolo} è un poligono di tre lati.
\item Si chiamano \emph{vertici} gli estremi dei lati.
\item Un vertice si dice \emph{opposto a un lato} se non appartiene a 
quel lato.
\item Si chiamano \emph{angoli interni} del triangolo i tre angoli 
formati dai lati.
\item Un angolo interno si dice \emph{angolo compreso tra due lati} 
quando i lati dell'angolo contengono dei lati del triangolo.
\item Un angolo interno si dice \emph{angolo adiacente a un lato} del 
triangolo quando uno dei suoi lati contiene quel lato del triangolo.
\item Un angolo si dice \emph{angolo esterno} al triangolo se è un 
angolo adiacente a un angolo interno.
\item Si dice \emph{bisettrice} relativa a un vertice, il segmento di 
bisettrice dell'angolo al vertice che ha per estremi il vertice 
stesso e il punto in cui essa incontra il lato opposto.
\item Si dice \emph{mediana} relativa a un lato il segmento che ha 
per estremi il punto medio del lato e il vertice opposto a quel lato.
\item Si dice \emph{altezza} di un triangolo relativa a un suo lato 
il segmento di perpendicolare che ha per estremi il vertice opposto 
al lato e il punto di intersezione della perpendicolare con la retta 
contenente il lato. 
\item Si dice \emph{asse} di un triangolo, relativo a un suo lato, la 
perpendicolare al lato condotta nel suo punto medio.
\end{itemize*}
\end{definizione}

Nel triangolo (a) della figura seguente, $A$, $B$ e $C$ sono i 
vertici del triangolo, il vertice $A$ è opposto al lato $a$, l'angolo 
$\alpha$ è interno al triangolo ed è compreso tra i lati $AB$ e $AC$, 
mentre l'angolo $\beta$ è esterno. Nel triangolo (b) $AL$ è la 
bisettrice dell'angolo nel vertice $A$, $AH$ è altezza relativa alla 
base $BC$, $AM$ è la mediana relativa al lato $BC$ e la retta $r$ è 
l'asse di $BC$.


\begin{inaccessibleblock}[Figura: TODO]
 \begin{figure}[htb]
\centering\input{\folder lbr/fig001_triangolo.pgf}
\caption{Triangolo. Vertici, angoli, bisettrice, mediana, asse.}
\label{fig:triangolo1}
\end{figure}
\end{inaccessibleblock}

I triangoli possono essere classificati rispetto ai lati
\begin{definizione}~
\begin{itemize*}
\item un triangolo si dice \emph{equilatero} se ha i tre lati 
congruenti;
\item un triangolo si dice \emph{isoscele} se ha (almeno) due lati 
congruenti;
\item un triangolo si dice \emph{scaleno} se ha i lati a due a due 
non congruenti.
\end{itemize*}
\end{definizione}


\begin{inaccessibleblock}[Figura: TODO]
 \begin{figure}[htb]
\centering\input{\folder lbr/fig002_triangolo.pgf}
\caption{Classificazione di un triangolo rispetto ai 
lati}\label{fig:class_triangolo_lati}
\end{figure}
\end{inaccessibleblock}

\noindent o rispetto agli angoli
\begin{definizione}~
\begin{itemize*}
\item un triangolo si dice \emph{rettangolo} se ha un angolo interno 
retto; in un triangolo rettangolo si chiama \emph{ipotenusa} il lato 
che si oppone all'angolo retto e si chiamano \emph{cateti} i lati 
adiacenti all'angolo retto;
\item un triangolo si dice \emph{ottusangolo} se ha un angolo interno 
ottuso;
\item un triangolo si dice \emph{acutangolo} se ha tutti gli angoli 
interni acuti.
\end{itemize*}
\end{definizione}


\begin{inaccessibleblock}[Figura: TODO]
\begin{figure}[htb]
\centering\input{\folder lbr/fig003_triangolo.pgf}
\caption{Classificazione di un triangolo rispetto agli angoli}
\label{fig:class_triangolo_angoli}
\end{figure}
\end{inaccessibleblock}

Abbiamo già costruito un triangolo equilatero 
(vedi \ref{proc:fonda_equilatero}).

\begin{comment}
 
Con riga e compasso.

\begin{procedura}[Triangolo equilatero]\label{proc:fonda_equilatero}
  Dati due punti A e B, si deve costruire un punto C in modo che ABC sia un 
triangolo equilatero:
  \begin{enumerate} [nosep]
    \item 
    Traccia i punti A e B.
    \item 
    Traccia la circonferenza di centro A e passante per B.
    \item 
    Traccia la circonferenza di centro B e passante per A.
    \item 
    Individua un punto C di intersezione delle due circonferenze.
    \item 
    Il poligono ABC è il triangolo richiesto.    
  \end{enumerate}
\end{procedura}

Con la geometria interattiva.

\lstinputlisting[firstline=2]{\folder src/11triequi.py} %, lastline=5]

\end{comment}

Vediamo ora come costruire triangoli isosceli. Se è data la base.

Con riga e compasso.

\begin{procedura}
  Costruzione di un triangolo isoscele di base assegnata:
  \begin{enumerate} [nosep]
    \item 
    Traccia un segmento di estremi A e B, base del triangolo da costruire.
    \item 
    Costruisci l'asse del segmento AB.
    \item 
    Prendi un punto sull'asse e denominalo C.
    \item 
    I segmenti AC e BC hanno la stessa lunghezza e quindi il triangolo ABC è 
isoscele.
  \end{enumerate}
\textit{ Con questa procedura, quanti triangoli isosceli di base assegnata AB 
puoi costruire?}  
\end{procedura}
 
Con la geometria interattiva.

\lstinputlisting[firstline=2]{\folder src/01triiso_base.py} %, lastline=5]

Se è dato il lato obliquo.

Con riga e compasso.

 \begin{procedura}
   Costruzione di un triangolo isoscele di lato obliquo assegnato:
   \begin{enumerate} [nosep]
     \item 
     Traccia un segmento di estremi A e B, lato obliquo del triangolo da 
costruire.  
     \item 
     Traccia una circonferenza puntando il compasso in B, con apertura AB.  
     \item 
     Scegli un punto qualsiasi sulla circonferenza: denominalo C; il triangolo 
ABC è isoscele sulla base AC.  
   \end{enumerate}
   \textit{ Con questa procedura, quanti triangoli isosceli di lati assegnati 
congruenti ad AB puoi costruire?}  
 \end{procedura}
 
Con la geometria interattiva.

\lstinputlisting[firstline=2]{\folder src/02triiso_lato.py} %, lastline=5]

\section{Criteri di congruenza dei triangoli}
\label{sect:primo_secondo_criterio_di_congruenza_triangoli}

Ricordiamo che due figure piane si dicono \emph{congruenti} se sono 
sovrapponibili, cioè se è possibile spostare una sull'altra, senza 
deformarle, in modo che coincidano perfettamente. 

In particolare, due triangoli sono sovrapponibili se hanno 
``ordinatamente'' congruenti i tre lati e i tre angoli. Con il 
termine ordinatamente intendiamo che, a partire da una coppia di 
vertici (il primo di un triangolo ed il secondo dell'altro) procedendo 
lungo il contorno in senso orario, oppure antiorario, incontriamo lati 
tra loro congruenti e vertici di angoli tra loro congruenti. Nel caso 
dei triangoli, questo succede esattamente quando angoli congruenti nei 
due triangoli sono compresi tra coppie di lati congruenti o, in 
maniera equivalente, quando sono opposti a lati congruenti.

I criteri di congruenza dei triangoli ci dicono che è sufficiente 
conoscere la congruenza di solo alcuni elementi dei due triangoli, 
generalmente tre elementi di un triangolo congruenti a tre elementi 
dell'altro triangolo, per poter affermare che i due triangoli sono 
tra loro congruenti, e quindi dedurne la congruenza degli altri 
elementi.

Un modo tradizionale di presentare l'argomento, dovuto allo stesso 
Euclide, è quello di ``dimostrare'' i primi due criteri di congruenza 
dei triangoli facendo uso della definizione stessa di congruenza come 
``uguaglianza per sovrapposizione'', e di utilizzarli successivamente 
per la verifica di altre proprietà.

Secondo il matematico tedesco Hilbert, il primo criterio di 
congruenza è invece un assioma e il secondo criterio può essere 
dimostrato per assurdo attraverso il primo. 

Di seguito presenteremo solo gli enunciati dei tre criteri di congruenza.

\begin{teorema}[1\textsuperscript{o} Criterio di congruenza dei 
triangoli]
Due triangoli sono congruenti se hanno congruenti due lati e l'angolo 
tra essi compreso.
\end{teorema}

% figura ()

\begin{inaccessibleblock}[Figura: TODO]
 \begin{figure}[htb]
\centering\input{\folder lbr/fig004_1crit_cong_tri.pgf}
\end{figure}
\end{inaccessibleblock}

\noindent Ipotesi: $AC\cong A'C'$, $BC\cong B'C'$, $\gamma \cong 
\gamma'$.\tab Tesi:  $ABC \cong A'B'C'$.

\begin{teorema}[2\textsuperscript{o} Criterio di congruenza dei 
triangoli]
Due triangoli sono congruenti se hanno congruenti due angoli e il 
lato tra essi compreso.
\end{teorema}


\begin{inaccessibleblock}[Figura: TODO]
 \begin{figure}[htb]
\centering\input{\folder lbr/fig005_2crit_cong_tri.pgf}
\end{figure}
\end{inaccessibleblock}

\noindent Ipotesi: $AB\cong A'B'$, $\alpha\cong \alpha'$, $\beta 
\cong \beta'$.\tab Tesi:  $ABC \cong A'B'C'$.

\begin{teorema}[3\textsuperscript{o} criterio di congruenza dei 
  triangoli]
  Due triangoli sono congruenti se hanno congruenti le tre coppie di 
  lati.
\end{teorema}


\begin{inaccessibleblock}[Figura: TODO]
  \begin{figure}[htb]
    \centering\input{\folder lbr/fig011_3crit_cong_tri.pgf}
  \end{figure}
\end{inaccessibleblock}

\noindent Ipotesi: $AB\cong A'B'$, $BC\cong B'C'$, $AC\cong 
A'C'$.\tab Tesi: $ABC\cong A'B'C'$.


\begin{exrig}
\begin{esempio}\label{esempio:2.1}
Si considerino due rette incidenti, $r$ ed $s$, ed il loro punto in 
comune $P$. Sulle semirette opposte di origine $P$ si prendano punti 
equidistanti da $P$, come in figura, in maniera tale che $AP\cong 
PB$, $CP\cong PD$. Dimostra che, unendo i quattro punti in modo da 
costruire un quadrilatero, i quattro triangoli che si vengono a 
formare sono a due a due congruenti: $ACP\cong BDP$, $ADP\cong BPC$.

Realizziamo il disegno (figura~\ref{fig:esempio2.1}) ed esplicitiamo 
ipotesi e tesi.


\begin{inaccessibleblock}[Figura: TODO]
 \begin{figure}[htb]
\centering\input{\folder lbr/fig006_esempio.pgf}
\caption{Esempio~\ref{esempio:2.1}}\label{fig:esempio2.1}
\end{figure}
\end{inaccessibleblock}

\noindent Ipotesi: $r\cap s=P$, $AP\cong PB$, $CP\cong PD$.\tab\tab 
Tesi: $ACP\cong BDP$, $ADP\cong BPC$.

\begin{proof}
I triangoli $ACP$ e $BPD$ hanno: $AP\cong PB$ per ipotesi, $CP\cong 
PD$ per ipotesi, $A\widehat{P}C\cong B\widehat{P}D$ perché opposti al 
vertice. Pertanto sono congruenti per il 1\textsuperscript{o} 
criterio di congruenza dei triangoli.

Analogamente, i triangoli $ADP$ e $BPC$ hanno: 
\ldots\ldots\ldots\ldots\ldots\ldots\ldots
\end{proof}
\end{esempio}

\begin{esempio}\label{esempio:2.2}
Si considerino un segmento $AB$ ed il suo punto medio $M$. Si tracci 
una generica retta $r$ passante per $M$ e distinta dalla retta per 
$AB$. Si traccino inoltre due semirette di origine rispettivamente 
$A$ e $B$, situate nei due semipiani opposti rispetto alla retta per 
$AB$, che intersechino la retta $r$ rispettivamente in $C$ e in $D$ e 
che formino con la retta per $AB$ due angoli congruenti (vedi 
figura~\ref{fig:esempio2.2}). Detti $C$ e $D$ i rispettivi punti di 
intersezione delle due semirette con la retta $r$, dimostra che i 
triangoli $AMC$ e $BMD$ sono congruenti.


\begin{inaccessibleblock}[Figura: TODO]
 \begin{figure}[htb]
\centering\input{\folder lbr/fig007_esempio.pgf}
\caption{Esempio~\ref{esempio:2.2}}\label{fig:esempio2.2}
\end{figure}
\end{inaccessibleblock}

\noindent Ipotesi: $AM\cong MB$, $M\widehat{A}C\cong 
M\widehat{B}D$.\tab Tesi: $AMC\cong BMD$.

\begin{proof}
I segmenti $AM$ e $MB$ sono congruenti in quanto $M$ è il punto medio 
di $AB$, gli angoli di vertice $M$ sono congruenti perché opposti al 
vertice, gli angoli di vertici $A$ e $B$ sono congruenti per 
costruzione. Allora i triangoli $AMC$ e $BMD$ sono congruenti per il 
2\textsuperscript{o} criterio di congruenza dei triangoli.
\end{proof}
\end{esempio}
\end{exrig}

\section{Teoremi del triangolo 
isoscele}\label{sect:teoremi_triangolo_isoscele}

Il \emph{triangolo isoscele} ha almeno due lati congruenti, 
l'eventuale lato non congruente si chiama \emph{base}, i due lati 
congruenti si dicono \emph{lati obliqui}.

Il \emph{triangolo equilatero} è un caso particolare di triangolo 
isoscele: si dice che \emph{il triangolo equilatero è isoscele 
rispetto a qualsiasi lato preso come base}.

\begin{teorema}[del triangolo isoscele {[}teorema diretto{]}]
In un triangolo isoscele gli angoli alla base sono congruenti.
\end{teorema}


\begin{inaccessibleblock}[Figura: TODO]
 \begin{figure}[htb]
\centering\input{\folder lbr/fig008_tri_isoscele_dir.pgf}
\end{figure}
\end{inaccessibleblock}

\noindent Ipotesi: $AC\cong BC$.\tab Tesi: $\alpha\cong \beta$.

\begin{proof}
Tracciamo la bisettrice $CK$ dell'angolo in $C$.
I triangoli $ACK$ e $BCK$ sono congruenti per il primo criterio, 
infatti hanno:
\begin{itemize*}
\item $AC\cong CB$ per ipotesi;
\item $CK$ lato in comune;
\item $A\widehat{C}K\cong B\widehat{C}K$ perché $CK$ è la bisettrice 
dell'angolo in $C$.
\end{itemize*}
Pertanto, essendo congruenti, i due triangoli hanno tutti gli 
elementi congruenti, in particolare l'angolo $\alpha$ (in $A$) è 
congruente all'angolo $\beta$ (in $B$).
\end{proof}

Il teorema precedente è invertibile, nel senso che è valido anche il 
teorema inverso, quello che si ottiene scambiando tra loro ipotesi e 
tesi.

\begin{teorema}[del triangolo isoscele {[}teorema inverso{]}]
Se un triangolo ha due angoli congruenti, allora è isoscele (rispetto 
al lato compreso tra gli angoli congruenti preso come base).
\end{teorema}


\begin{inaccessibleblock}[Figura: TODO]
 \begin{figure}[htb]
\centering\input{\folder lbr/fig009_tri_isoscele_inv.pgf}
\end{figure}
\end{inaccessibleblock}

\noindent Ipotesi: $\alpha\cong \beta$.\tab Tesi: $AC\cong BC$.

\begin{proof}
Procediamo per passi, realizzando una costruzione che ci permetta di 
confrontare coppie di triangoli congruenti. Prolunghiamo i lati $AC$ 
e $BC$ dalla parte di $A$ e di $B$ rispettivamente, e sui 
prolungamenti prendiamo due punti $D$ ed $E$ in maniera tale che 
risulti $AD\cong BE$.

Osserviamo che i triangoli $ADB$ e $BAE$ risultano congruenti per il 
1\textsuperscript{o} criterio, avendo in comune il lato $AB$ ed 
essendo $AD\cong BE$ per costruzione e $D\widehat{A}B\cong 
A\widehat{B}E$ perché adiacenti agli angoli $C\widehat{A}B$ e 
$C\widehat{B}A$ congruenti per ipotesi. Pertanto, tutti gli elementi 
dei due triangoli $ADB$ e $AEB$ sono ordinatamente congruenti, in 
particolare $DB\cong AE$, $A\widehat{D}B\cong B\widehat{E}A$ e 
$A\widehat{B}D\cong B\widehat{A}E$.

I triangoli $CDB$ e $CAE$ risultano dunque congruenti per il 
2\textsuperscript{o} criterio poiché hanno $DB\cong AE$, 
$C\widehat{D}B\cong C\widehat{E}A$ per quanto appena dimostrato e 
$C\widehat{D}B\cong C\widehat{A}E$ perché somma di angoli 
rispettivamente congruenti: $C\widehat{B}D\cong C\widehat{B}A + 
A\widehat{B}D$ e $C\widehat{A}E\cong C\widehat{A}B + B\widehat{A}E$.

Pertanto, i restanti elementi dei due triangoli risultano 
ordinatamente congruenti, in particolare $CB\cong CA$, che è la tesi 
che volevamo dimostrare.
\end{proof}


Dai due teoremi precedenti seguono importanti proprietà, che qui 
riportiamo come corollari.

\begin{corollario}
Un triangolo equilatero è anche equiangolo.
\end{corollario}

\begin{proof}
Poiché un triangolo equilatero è isoscele rispetto a qualsiasi lato 
preso come base, la tesi segue dal teorema diretto del triangolo 
isoscele.
\end{proof}

\begin{corollario}
Se un triangolo è equiangolo allora è equilatero.
\end{corollario}

\begin{proof}
Possiamo confrontare gli angoli a due a due; risulteranno i lati 
congruenti a due a due in base al teorema inverso del triangolo 
isoscele.
\end{proof}

\begin{corollario}
Un triangolo scaleno non ha angoli congruenti.
\end{corollario}

\begin{proof}
Se per assurdo un triangolo scaleno avesse due angoli congruenti, 
allora risulterebbe isoscele, in base al teorema inverso del 
triangolo isoscele.
\end{proof}

\begin{corollario}
Se un triangolo non ha angoli congruenti allora è scaleno.
\end{corollario}

\begin{proof}
Se un triangolo non ha angoli tra loro congruenti non può essere 
isoscele.
\end{proof}

\begin{proposizione}[Proprietà del triangolo isoscele]
In ogni triangolo isoscele, la mediana relativa alla base è anche 
altezza e bisettrice.
\end{proposizione}
Nella figura, $CJ$ è per ipotesi la bisettrice dell'angolo al vertice 
$\gamma_1$ del triangolo $ABC$, $FK$ è la mediana relativa alla base 
$DE$ del triangolo $DEF$, $IL$ è l'altezza relativa alla base $GH$ 
del triangolo $GHI$.


\begin{inaccessibleblock}[Figura: TODO]
 \begin{figure}[htb]
\centering\input{\folder lbr/fig010_tri_isoscele_prop.pgf}
\end{figure}
\end{inaccessibleblock}

Dividiamo l'enunciato in tre parti:
\begin{enumeratea}
\item In un triangolo isoscele la bisettrice dell'angolo al vertice è 
anche altezza e mediana relativa alla base.
\item In un triangolo isoscele la mediana relativa alla base è anche 
bisettrice dell'angolo al vertice e altezza relativa alla base.
\item In un triangolo isoscele l'altezza relativa alla base è anche 
bisettrice dell'angolo al vertice e mediana relativa alla base.
\end{enumeratea}

Per ciascuna di esse scriviamo ipotesi e tesi.
\begin{enumeratea}
\item In $ABC$:  Ipotesi: $AC\cong CB$, $\alpha_1\cong \beta_1$, 
$A\widehat{C}J\cong B\widehat{C}J$. Tesi: $CJ\perp AB$, $AJ\cong JB$.
\item In $DEF$:  \,Ipotesi: $DF\cong FE$, $\alpha_2\cong \beta_2$, 
$DK\cong KE$.\tab Tesi: $FK\perp DE$, $D\widehat{F}K\cong 
E\widehat{F}K$.
\item In $GHI$:  \,Ipotesi: $IG\cong IH$, $\alpha_3\cong \beta_3$, 
$IL\perp GH$.\tab Tesi: $GL\cong LH$, $G\widehat{I}L\cong 
H\widehat{I}L$.
\end{enumeratea}

\begin{proof}
Avviamo la dimostrazione delle prime due parti, che lasciamo 
completare al lettore e rimandiamo al prossimo capitolo la 
dimostrazione della terza.

\begin{enumeratea}
\item I triangoli $AJC$ e $CJB$ sono congruenti per il 
2\textsuperscript{o} criterio. Infatti hanno 
\ldots\ldots\ldots\ldots{}\\
Dunque $AJ\cong JB$ e $A\widehat{J}C\cong C\widehat{J}B$ che 
risultano pertanto retti in quanto adiacenti.  
\item I triangoli $DKF$ e $FKE$ sono congruenti per il 
1\textsuperscript{o} criterio. Infatti hanno 
\ldots\ldots\ldots\ldots{}\\
Dunque $D\widehat{F}K\cong E\widehat{F}K$ e $F\widehat{K}D\cong 
F\widehat{K}E$ che risultano pertanto retti in quanto adiacenti.
\end{enumeratea}
\end{proof}

% 
% \newpage
% 
% \input{./chap/02_esercizi}
% 
% \cleardoublepage
