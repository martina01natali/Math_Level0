% (c) 2015 Daniele Zambelli daniele.zambelli@gmail.com

% (c) 2014 Daniele Zambelli - daniele.zambelli@gmail.com
% 
% Tutti i grafici per il capitolo relativo alle parabole
%

\input{\magdir definizioni_tikz}

\newcommand{\puntia}{% 
    % Alcuni punti di una parabola.
    \disegno{
    \rcom{-5}{+5}{-4}{13}{gray!50, very thin, step=1}
    \foreach \pi in {
    % (-4, 22), 
    (-3, 13), (-2, 6), (-1, 1), 
    (0, -2), (1, -3), (2, -2), (3, 1), (4, 6), (5, 13)}
    \filldraw [Maroon!50!black] \pi circle (1.5pt);
    }
}

\newcommand{\puntib}{% 
    % Altri punti di una parabola.
    \disegno{
    \rcom{-5}{+5}{-4}{13}{gray!50, very thin, step=1}
    \foreach \pi in {
    (-3, 13), (-2, 6), (-1, 1), (0, -2), (1, -3), (2, -2), (3, 1), 
    (4, 6), (5, 13),
    (-2.5, 9.25), (-1.5, 3.25), (-0.5, -0.75), (0.5, -2.75), 
    (1.5, -2.75), (2.5, -0.75), (3.5, 3.25), (4.5, 9.25) 
    }
    \filldraw [Maroon!50!black] \pi circle (1.5pt);
    }
}

\newcommand{\graficotrinomio}{% 
    % Grafico di un trinomio di secondo grado.
    \disegno{
    \rcom{-5}{+5}{-4}{13}{gray!50, very thin, step=1}
    \tkzInit[xmin=-5.3,xmax=+5.3,ymin=-4.3,ymax=+13.3]
    \tkzFct[domain=-5:+6, ultra thick, color=Maroon!50!black]{x*x-2*x-2}
    }
}

\newcommand{\coefficientea}{% 
    % Parabole con diversi coefficienti del termine di 2° grado.
    \disegno{
    \rcom{-10}{+10}{-10}{10}{gray!50, very thin, step=1}
    \begin{scope}[domain=-10:+10, ultra thick, color=Maroon!50!black]
    \tkzInit[xmin=-10.3,xmax=+10.3,ymin=-10.3,ymax=+10.3]
    \tkzFct{2*x*x+x+2}
    \tkzFct{0.5*x*x+x+2}
    \tkzFct{0.1*x*x+x+2}
    \tkzFct{0.01*x*x+x+2}
    \begin{scope}[color=Green!50!black]
    \tkzFct{-0.01*x*x+x+2}
    \tkzFct{-0.1*x*x+x+2}
    \tkzFct{-0.5*x*x+x+2}
    \tkzFct{-2*x*x+x+2}
    \end{scope}
    \end{scope}
    \begin{scope}[color=black]
    \draw (-1.5, 9.5) node {a} (-4, 8.5) node {b} (-9, 2) node {c} 
        (-9, -5.5) node {d} (-8, -7.5) node {e} (-5.6, -8.5) node {f}
        (-3.5, -9.5) node {g} (-1.5, -9.5) node {h};
    \end{scope}
    }
}

\newcommand{\quadrati}{% 
    % Numeri quadrati
    \disegno{
    \draw[gray!50, very thin, step=1] (-1, 1) grid (27, 7); % Griglia
    \draw (-.5, 0) node{0};
    \foreach \n in {1, ..., 4} {
        \pgfmathparse{\n * \n / 2 + \n * 3 / 2}
        \xdef\corner{\pgfmathresult}
        \fill[fill=green,fill opacity=0.3] (\corner, \n + 1) --
                                            (\corner - \n, \n + 1) --
                                            (\corner - \n, \n) --
                                            (\corner - 1, \n) --
                                            (\corner - 1, 1) --
                                            (\corner, 1) -- cycle; 
        \foreach \j in {1, ..., \n} {
            \foreach \i in {1, ..., \n} { 
                \fill[blue!40!white] (\n * \n / 2 + \n / 2 -.5 + \i, \j+.5) 
                                    circle (.4);
            }
        }
    }
    \foreach \n in {1, ..., 6} {
        \pgfmathparse{int(\n * \n)}
        \xdef\square{\pgfmathresult}
        \draw (\n * \n / 2 + \n  , 0) node{\square};
    }
    \draw (.5, -1) node{1} (2.6  , -1) node{3} (5.9  , -1) node{5}
        (10, -1.3) node{\ldots} (15, -1.3) node{\ldots}
        (21, -1.3) node{\ldots} (26, -1.3) node{\ldots};
    }
}

\newcommand{\disrapido}{% 
    % Metodo rapido per il disegno di parabole.
    \disegno{
    \tkzInit[xmin=-5.5,xmax=+5.5,ymin=-5.5,ymax=+15.5]

    \clip (-5.3, -1.3) rectangle (5.7, 13.8);
    \rcom{-5}{+5}{-1}{13}{gray!50, very thin, step=1}

    \coordinate (a0) at (0, 0); \coordinate (a0p) at (1, 0); 
    \coordinate (a1) at (1, 1); \coordinate (a1p) at (2, 1);
    \coordinate (a2) at (2, 4); \coordinate (a2p) at (3, 4);
    \coordinate (a3) at (3, 9);
    \coordinate (i0m) at (-1, 0);
    \coordinate (i1) at (-1, 1); \coordinate (i1m) at (-2, 1);
    \coordinate (i2) at (-2, 4); \coordinate (i2m) at (-3, 4);
    \coordinate (i3) at (-3, 9);

    \begin{scope}[red!50!black]
    % Senza etichetta
    \foreach \p in {a0, a1, a2, a3, i1, i2, i3}
    \filldraw (\p) circle (1.5pt);

    \tkzFct[domain=-5:+5, ultra thick]{x*x}

    \draw (i3) 
    .. controls +(-.5, 0) and +(-.5, 0) .. ++(0, -1)
    .. controls +(-.5, 0) and +(-.5, 0) .. ++(0, -1)
    .. controls +(-.5, 0) and +(-.5, 0) .. ++(0, -1)
    .. controls +(-.5, 0) and +(-.5, 0) .. ++(0, -1)
    .. controls +(-.5, 0) and +(-.5, 0) .. ++(0, -1)
    .. controls +(0, -.5) and +(0, -.5) .. ++(+1, 0)
    .. controls +(-.5, 0) and +(-.5, 0) .. ++(0, -1)
    .. controls +(-.5, 0) and +(-.5, 0) .. ++(0, -1)
    .. controls +(-.5, 0) and +(-.5, 0) .. ++(0, -1)
    .. controls +(0, -.5) and +(0, -.5) .. ++(+1, 0)
    .. controls +(-.5, 0) and +(-.5, 0) .. ++(0, -1)
    .. controls +(0, -.5) and +(0, -.5) .. ++(+1, 0)
    
    .. controls +(0, -.5) and +(0, -.5) .. ++(+1, 0)
    .. controls +(+.5, 0) and +(+.5, 0) .. ++(0, +1)
    .. controls +(0, -.5) and +(0, -.5) .. ++(+1, 0)
    .. controls +(+.5, 0) and +(+.5, 0) .. ++(0, +1)
    .. controls +(+.5, 0) and +(+.5, 0) .. ++(0, +1)
    .. controls +(+.5, 0) and +(+.5, 0) .. ++(0, +1)
    .. controls +(0, -.5) and +(0, -.5) .. ++(+1, 0)
    .. controls +(+.5, 0) and +(+.5, 0) .. ++(0, +1)
    .. controls +(+.5, 0) and +(+.5, 0) .. ++(0, +1)
    .. controls +(+.5, 0) and +(+.5, 0) .. ++(0, +1)
    .. controls +(+.5, 0) and +(+.5, 0) .. ++(0, +1)
    .. controls +(+.5, 0) and +(+.5, 0) .. ++(0, +1);

    \draw node [below, scale=.7, yshift=-7pt] at ($ (a0)!.5!(i0m) $) {$-1$};
    \draw node [left, scale=.7, xshift=+10pt] at ($ (i1)!.5!(i0m) $) {$+1$};
    \draw node [below, scale=.7, yshift=-7pt] at ($ (i2m)!.5!(i2m) $) {$-1$};
    \draw node [left, scale=.7, xshift=+10pt] at ($ (i2)!.5!(i1m) $) {$+3$};
    \draw node [below, scale=.7, yshift=-7pt] at ($ (i1m)!.5!(i1m) $) {$-1$};
    \draw node [left, scale=.7, xshift=+10pt] at ($ (i3)!.5!(i2m) $) {$+5$};

    \draw node [below, scale=.7, yshift=-7pt] at ($ (a0)!.5!(a0p) $) {$+1$};
    \draw node [left, scale=.7, xshift=+9pt] at ($ (a1)!.5!(a0p) $) {$+1$};
    \draw node [below, scale=.7, yshift=-7pt] at ($ (a2p)!.5!(a2p) $) {$+1$};
    \draw node [left, scale=.7, xshift=+9pt] at ($ (a2)!.5!(a1p) $) {$+3$};
    \draw node [below, scale=.7, yshift=-7pt] at ($ (a1p)!.5!(a1p) $) {$+1$};
    \draw node [left, scale=.7, xshift=+9pt] at ($ (a3)!.5!(a2p) $) {$+5$};

    \end{scope}
    }
}

\newcommand{\piancart}{% 
    % Piano cartesiano  per il disegno di alcune parabole
    \disegno{
    \rcom{-10}{+10}{-10}{+10}{gray!50, very thin, step=1}

    % \tkzInit[xmin=-10.5,xmax=+10.5,ymin=-10.5,ymax=+10.5]
    % \tkzFct[domain=-10:+10, ultra thick, color=Maroon!50!black]{-x*x}
    % \tkzFct[domain=-10:+10, ultra thick, color=Maroon!50!black]{-0.5*x*x+2*x+8}
    % \tkzFct[domain=-10:+10, ultra thick, color=Maroon!50!black]{2*x*x+24*x+64}
    }
}

\newcommand{\parabolaerette}{% 
    % Una parabola e tre rette nel piano cartesiano.
    \disegno{
    \rcom{-7}{+5}{-7}{+10}{gray!50, very thin, step=1}

    \tkzInit[xmin=-7.3,xmax=+5.7,ymin=-7.5,ymax=+10.5]

    \begin{scope}[color=Red!50!black]
    \tkzFct[domain=-10:+10, ultra thick]{-x*x+4}
    \node at (2.5, -5.3) {p};
    \end{scope}
    \begin{scope}[color=Blue!50!black]
    \tkzFct[domain=-10:+10, ultra thick]{2*x+9}
    \tkzFct[domain=-10:+10, ultra thick]{2*x+5}
    \tkzFct[domain=-10:+10, ultra thick]{2*x+1}
    \node at (-6.2, -2.3) {r};
    \node at (-5.5, -5.3) {t};
    \node at (-4.2, -6.5) {s};
    \end{scope}
    }
}

\newcommand{\parabolaetangenti}{% 
    % Parabola e tangenti.
    \disegno{
    \rcom{-5}{+10}{-6}{+11}{gray!50, very thin, step=1}
    \tkzInit[xmin=-5.3,xmax=+10.3,ymin=-6.3,ymax=+11.3]
    \tkzFct[domain=-10:+10.3, ultra thick, color=Blue!50!black]{-1./4.*x*x+2*x+1}
    \tkzFct[domain=-10:+10.3, ultra thick, color=Green!50!black]{x+2}
    \tkzFct[domain=-10:+10.3, ultra thick, color=Green!50!black]{3*x+2}
    \tkzFct[domain=-10:+10.3, ultra thick, color=Red!50!black]{-2.*x+17}

    \begin{scope}[color=Green!50!black]
    \coordinate (a) at (0, +2);
    \filldraw  (a) circle (1.5pt); 
    \node at (a) [xshift=-9pt] {$A$};
    \end{scope}

    \begin{scope}[color=Red!50!black]
    \coordinate (b) at (+8, +1);
    \filldraw (b) circle (1.5pt); 
    \node at (b) [xshift=+7pt] {$B$};
    \end{scope}

    \begin{scope}[color=Blue!50!black]
    \coordinate (b) at (+5, -2);
    \filldraw (b) circle (1.5pt); 
    \node at (b) [xshift=+7pt] {$C$};
    \filldraw (+2, +4) circle (1.5pt); 
    \filldraw (-2, -4) circle (1.5pt); 
    \end{scope}
    }
}

\newcommand{\intersezioniparabole}{% 
    % Intersezione tra due parabole.
    \disegno{
    \rcom{-3}{+5}{-6}{+10}{gray!50, very thin, step=1}
    \tkzInit[xmin=-3.3,xmax=+5.3,ymin=-6.3,ymax=+10.3]
    \tkzFct[domain=-3.3:+5.3, ultra thick, color=Blue!50!black]{-x*x+2*x+8}
    \tkzFct[domain=-3.3:+5.3, ultra thick, color=Red!50!black]{5./4.*x*x-19./4.*x-1}
    \filldraw (-1, +5) circle (1.5pt) node [xshift=-9pt] {$I_0$};
    \filldraw (+4, 0) circle (1.5pt) 
        node [xshift=9pt, yshift=6pt] {$I_1$};
    }
}

\newcommand{\parabolapertrepunti}{% 
    % Parabola per tre punti.
    \disegno{
    \rcom{-5}{+5}{-9}{+10}{gray!50, very thin, step=1}
    \tkzInit[xmin=-5.3,xmax=+5.3,ymin=-9.3,ymax=+10.3]
    \tkzFct[domain=-10:+10, ultra thick, color=Blue!50!black]{1./2.*x*x+2*x-7}
    \filldraw (-4, -7) circle (1.5pt) node [xshift=-9pt] {$P_0$};
    \filldraw (+2, -1) circle (1.5pt) node [xshift=9pt] {$P_1$};
    \filldraw (+4, +9) circle (1.5pt) node [xshift=9pt] {$P_2$};
    }
}

\newcommand{\parabolaverticepunto}{% 
    % Equazione della parabola dati vertice e punto.
    \disegno{
    \rcom{-1}{+10}{-6}{+8}{gray!50, very thin, step=1}
    \tkzInit[xmin=-1.3,xmax=+10.3,ymin=-6.3,ymax=+8.3]
    \tkzFct[domain=-10:+10, ultra thick, color=Blue!50!black]{-1./3.*x*x+2*x+4}
    \filldraw (+3, +7) circle (1.5pt) node [yshift=9pt] {$V$};
    \filldraw (+9, -5) circle (1.5pt) node [xshift=9pt] {$P$};
    }
}


\chapter{La circonferenza nel piano cartesiano}

\section{TODO}

\section{Circonferenza con il centro nell'origine}
\label{sec:circ_circcentroorigine}

% \subsection{Circonferenza come luogo geometrico}
% \label{subsec:circ_luogo}
% 
% \subsection{Equazione della circonferenza}
% \label{subsec:circ_equazione}

Ci sono varie definizioni della curva che è sempre stata ritenuta un esempio di 
perfezione, la circonferenza, vediamone alcune.

\begin{definizione}%[intrinseca]
 La circonferenza è una linea del piano che ha sempre la stessa curvatura.
\end{definizione}

\begin{definizione}%[non standard]
 La circonferenza è un poligono regolare con infiniti lati.
\end{definizione}

\begin{definizione}%[luogo di punti]
 La circonferenza l'insieme dei punti del piano equidistanti da un punto detto 
centro.
\end{definizione}

A seconda del problema che vogliamo risolvere può essere più comodo utilizzare 
una o un'altra delle definizioni precedenti. In questo capitolo volgiamo 
studiare la circonferenza nel piano cartesiano e useremo l'ultima definizione.


\begin{figure}[h]
\centering
\begin{minipage}[]{.48\textwidth}
% Possiamo ottenere una una situazione particolarmente semplice da descrivere 
con 
% un'equazione scegliendo come centro della circonferenza, l'origine delle 
% coordinate.

 Se prendiamo come centro della circonferenza l'origine delle coordinate, 
otteniamo una situazione particolarmente semplice da descrivere con 
un'equazione.

In questo caso infatti la relazione del teorema di Pitagora lega i tre lati del 
triangolo: \(x,~y \text{ e } r\):
\[x^2 + y^2 = r^2\]
che è l'equazione della circonferenza perché tutti e solo i punti 
della circonferenza sono soluzioni di questa equazione.
\end{minipage}
\hfill
\begin{minipage}[]{.48\textwidth}
\begin{center}
\begin{inaccessibleblock}[Una circonferenza con centro nell'origine degli assi.]
  \circonfO
  \caption{Circonf. con centro nell'origine.} \label{fig:circonfO}
\end{inaccessibleblock}
\end{center}
\end{minipage}
\end{figure}

\begin{esempio}
Calcola l'equazione della circonferenza con centro nell'origine e passante per 
il punto \(P\punto{4}{6}\).

L'equazione sarà del tipo: \(x^2 + y^2 = r^2\) l'unico parametro da individuare 
è il raggio che è la distanza di un punto qualsiasi della circonferenza dal 
centro. L'esercizio ci dà un punto della circonferenza e quindi possiamo usarlo 
per trovare il raggio:
\[r = \sqrt{x_P^2 + y_P^2} = \sqrt{4^2 + 6^2} = \sqrt{16 + 36} = \sqrt{52}\]
l'equazione della circonferenza è allora:
\[x^2 + y^2 = 52\]
\end{esempio}

\begin{esempio}
Calcola le intersezioni tra la circonferenza  \(x^2 + y^2 = 25\) e la 
retta di equazione \(x=-4\).

\noindent\begin{minipage}{.48\textwidth}
La circonferenza ha centro nell'origine e ha \(r^2 = 25\) quindi \(r=5\). 
Disegniamo quindi la circonferenza con centro nell'origine e raggio~5,
poi disegniamo anche la retta formata da tutti i punti che hanno ascissa~\(-4\).

Le intersezioni si ottengono risolvendo il sistema:

\(\sistema{x=-4 \\ x^2 + y^2 = 25}\)

Con la sostituzione otteniamo l'equazione risolvente: 

\(\tonda{-4}^2 + y^2 = 25 \sRarrow 16 + y^2 = 25 \sRarrow\)

\(y^2 = 9 \sRarrow y= \mp 3\)

Le intersezioni tra la retta e la circonferenza sono dunque: 
\[p_0 \punto{-4}{-3} \text { e } p_1 \punto{-4}{+3}\]
\end{minipage}
\hfill
\begin{minipage}{.48\textwidth}
\begin{center}
\begin{inaccessibleblock}[Intersezioni tra una circonferenza e una retta.]
  \circonfretta
\end{inaccessibleblock}
\end{center}
\end{minipage}
\end{esempio}

% \begin{wrapfloat}{figure}{r}{0pt}
% \includegraphics[scale=0.35]{img/fig000_.png}
% \caption{...}
% \label{fig:...}
% \end{wrapfloat}
% 
% \begin{center} \input{\folder lbr/fig000_.pgf} \end{center}

\section{Circonferenza traslata}
\label{sec:circ_circtraslata}

Fin'ora abbiamo trattato circonferenze con il centro nell'origine degli assi, 
vogliamo ora generalizzare l'equazione in modo da ottenere l'equazione di una 
generica circonferenza del piano.

Consideriamo una circonferenza con centro nell'origine:
\[x^2 + y^2 = 52\]
e una generica traslazione:
\[\sistema{x' = x + \alpha \\ y' = y + \beta}\]
Riscriviamo le equazioni della traslazione esplicitando~\(x\) e~\(y\):
\[\sistema{x = x'- \alpha \\ y = y' - \beta}\]

Per traslare la circonferenza, operiamo la sostituzione di variabili
indicata dalla traslazione:
\[\tonda{x'-\alpha}^2 + \tonda{y'-\beta}^2 = r^2\]
questa è l'equazione della circonferenza traslata. Si può osservare che il 
centro della circonferenza traslata è: 
\[C'\punto{\alpha}{\beta}\]
Dato che ci riferiamo sempre allo stesso sistema di riferimento, semplifichiamo 
la scrittura eliminando gli apici ed evidenziando così che quella ottenuta è 
l'equazione di un'altra circonferenza dello stesso piano:
\[\tonda{x-\alpha}^2 + \tonda{y-\beta}^2 = r^2\]
Ora possiamo svolgere i calcoli e riscrivere l'equazione in un altro modo:
\[x^2 -2 \alpha x + \alpha^2 + y^2 -2 \beta y + \beta^2 = r^2\]
\[x^2 + y^2 -2 \alpha x -2 \beta y - r^2 + \alpha^2 + \beta^2 = 0\]
Possiamo osservare che essendo \(\alpha\) un numero, anche \(-2\alpha\) è un 
numero e anche \(-2\beta\) e anche \(- r^2 + \alpha^2 + \beta^2\). 
L'equazione di una circonferenza con centro in un punto qualsiasi del piano 
sarà del tipo:
\[x^2 + y^2 +a x +b y +c = 0\]
dove:
\[\sistema{a = -2\alpha \\ b = -2\beta \\ c = -r^2 +\alpha^2 +\beta^2}\]
In quest'ultimo sistema si possono esplicitare le coordinate del 
centro (\(\alpha\) e \(\beta\)) e il raggio (\(r\)):
\[\sistema{\alpha = -\dfrac{a}{2} \\ 
           \beta = -\dfrac{b}{2} \\ 
           r = \sqrt{-c +\alpha^2 + \beta^2}}\]
In questo modo possiamo calcolare le coordinate del centro e il raggio della 
circonferenza partendo dai coefficienti dell'equazione scritta in forma 
polinomiale.

\noindent\begin{minipage}{.48\textwidth}
TODO
\end{minipage}
\hfill
\begin{minipage}{.48\textwidth}
\begin{center}
\begin{inaccessibleblock}[Circonferenza con il centro in un punto qualsiasi 
del  piano.]
%   \circonfretta TODO
\end{inaccessibleblock}
\end{center}
\end{minipage}

\begin{esempio}
Calcola l'equazione polinomiale della circonferenza di cento \(C\punto{-4}{3}\) 
e di raggio \(r=6\).

Possiamo usare l'equazione in forma canonica:
\[\tonda{x+4}^2 + \tonda{y-3}^2 = 6^2\]
e svolgere i calcoli:
\[x^2 +8x+16 +y^2 -6y +9 -36 = 0\]
da cui si ottiene:
\[x^2 +y^2 +8x -6y -11 = 0\]

Oppure possiamo partire dal significato dei coefficienti illustrato sopra:
\[\sistema{a = -2(-4) \\ b = -2(+3) \\ c = -(+6)^2 +(-4)^2 +(+3)^2}\]
da cui si ottiene:
\[x^2 +y^2 +8x -6y -11 = 0\]
\end{esempio}

\begin{esempio}
Calcola le coordinate del centro e il raggio della circonferenza:
\(x^2 +y^2 -6x + 10y -11 = 0\)

Usando il sistema precedente otteniamo:
\[\sistema{\alpha = -\dfrac{-6}{2} = +3\\ 
           \beta = -\dfrac{10}{2} = -5 \\ 
           r = \sqrt{-(-11) +(+3)^2 + (-5)^2} = \sqrt{45}= 
               \sqrt{9 \cdot 5} = 3\sqrt{5}}\]
Questa circonferenza ha centro \(C\punto{+3}{-5}\) e raggio \(r=3\sqrt{5}\)
\end{esempio}

\begin{osservazione}
La presenza di una radice quadrata nel calcolo del raggio della circonferenza, 
dovrebbe farci scattare un campanello di allarme: siamo sicuri di poter 
calcolare questa radice? siamo sicuri che il radicando sia positivo?

Il fatto che \(\alpha\) e \(\beta\) siano elevati al quadrato ci assicura che 
questi due addendi siano positivi, ma che dire di \(-c\)? Se \(c\) è negativo 
possiamo essere sicuri che il radicando sia positivo, ma se \(c\) è positivo e 
abbastanza grande, il radicando può essere negativo e in questo caso il raggio 
non potrà essere un numero reale.
\end{osservazione}

Vediamo un esempio.

\begin{esempio}
Calcola le coordinate del centro e il raggio della circonferenza:
\(x^2 +y^2 +2x -4y +9 = 0\)

Usando il sistema precedente otteniamo:
\[\sistema{\alpha = -\dfrac{+2}{2} = -1\\ 
           \beta = -\dfrac{-4}{2} = +2 \\ 
           r = \sqrt{-(+9) +(-1)^2 + (+2)^2} = \sqrt{-4}}\]
Questa circonferenza ha centro \(C\punto{-1}{+2}\) ma il suo raggio non è un 
numero reale. È una circonferenza immaginaria!
\end{esempio}

\section{Circonferenze e rette}
\label{sec:circ_circrette}

Se consideriamo le posizioni reciproche di una circonferenza e di una retta, 
possiamo avere uno di questi tre casi:

\begin{description} %[noitemsep]
 \item [retta secante] 
quando retta e circonferenza hanno due punti distinti in comune. La retta avrà 
una distanza dal centro della circonferenza minore del raggio.
 \item [retta tangente]
quando retta e circonferenza si intersecano in due punti infinitamente vicini. 
La retta avrà una distanza dal centro della circonferenza uguale al raggio e 
sarà quindi perpendicolare al segmento che unisce il centro con il punto di 
tangenza.
 \item [retta è esterna]
quando retta e circonferenza non hanno punti reali in comune. La retta avrà una 
distanza dal centro della circonferenza maggiore del raggio. 
\end{description}

\noindent\begin{minipage}{.48\textwidth}
TODO
\end{minipage}
\hfill
\begin{minipage}{.48\textwidth}
\begin{center}
\begin{inaccessibleblock}[Circonferenza con una retta secante, una retta 
tangente e una retta esterna.]
%   \circonfretta TODO
\end{inaccessibleblock}
\end{center}
\end{minipage}


\begin{esempio}
 Senza disegnarle, scopri se la retta e la la circonferenza sono secanti, 
tangenti o si intersecano. TODO
\end{esempio}

\begin{esempio}
 Calcola l'equazione della retta tangente alla circonferenza nel suo punto di 
coordinate TODO
\end{esempio}

\begin{esempio}
 Calcola le equazioni delle rette tangenti alla circonferenza tracciate dal 
punto TODO
\end{esempio}

\section{Posizioni reciproche tra circonferenze}
\label{sec:circ_posizionireciproche}

Se vogliamo vedere tutte le posizioni reciproche di due circonferenza  
di raggio \(r_1\) e \(r_2\), possiamo partire dal posizionarle in modo che 
abbiano lo stesso centro, poi muovere una delle due e osservare che cosa 
succede (nel prossimo elenco la distanza tra i centri delle due circonferenze 
viene indicata con \(\abs{C_1 - C_0}\)).

\begin{description} %[noitemsep]
 \item [Concentriche]
I due centri coincidono: \(\abs{C_1 - C_0} = 0\).
Se i due raggi sono diversi non avranno punti in comune, altrimenti saranno 
coincidenti.
 \item [Una interna all'altra]
In questo caso \(\abs{C_1 - C_0} < r_1 - r_0\).
 \item [Una è tangente interna all'altra]
In questo caso \(\abs{C_1 - C_0} = r_1 - r_0\).
 \item [Secanti]
In questo caso \(r_1 - r_0 < \abs{C_1 - C_0} < r_1 + r_0\).
 \item [Tangenti esterne]
In questo caso \(\abs{C_1 - C_0} = r_1 + r_0\).
 \item [Esterne]
In questo caso \(\abs{C_1 - C_0} > r_1 + r_0\).
\end{description}

\begin{esempio}
Trova qual è la posizione reciproca delle due circonferenze TODO
\end{esempio}

\begin{esempio}
Calcola i punti di intersezione delle due circonferenze TODO
\end{esempio}

% \subsection{Fasci di circonferenze}
% \label{subsec:circ_fasci}

% \section{Curve deducibili dall'equazione della circonferenza}
% \label{sec:circ_curve_deducibili}


