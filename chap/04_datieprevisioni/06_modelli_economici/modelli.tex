% (c) 2015 Daniele Zambelli daniele.zambelli@gmail.com

\chapter{Modelli Economici}

\section{TODO}

% \section{Gli elementi di base dell'economia}
% \label{sec:modec_elementi_di_base}
\begin{comment}
 
Introduzione
  Micro economia
  Macro economia
  Variabili discrete
I mercati
  bene
  mercato
  consumatore
  produttore
  concorrenza perfetta
  monopolio
  oligopolio
  concorrenza monopolistica
Funzione di domanda e offerta
  leggi della domanda
    domanda
    offerta
    d=f(p) p>0 decrescente
      modello lineare
      modello parabolico
      modello esponenziale
      modello iperbolico
    esempi
    elasticità della domanda
      totalmente elastica
      elastica
      unitaria
      rigida
      totalmente rigida
  leggi dell'offerta
    r=f(p) p>0 crescente
    coefficiente di elasticità dell'offerta (positivo)
      elastica
      unitaria
      rigida
Legge di mercato
  Regime di concorrenza perfetta
    condizioni
      molti consumatori
      molti produttori
      libertà di acquisto e di vendita
      ogni operatore può entrare o uscire dal mercato 
      c'è trasparenza
      non ci sono coalizioni
      prodotti a larga diffusione
    prezzo di equilibrio
  Cambiamento del prezzo di equilibrio
    a parità di domanda
    a parità di offerta
    con cambiamento sia di domanda sia di offerta
Funzione di utilità
  paniere di consumo
    relazione tra panieri
      preferito
      indifferente
    proprietà
      riflessiva
      transitiva
      completezza
      continuità
      non sazietà
      stretta convessità
    funzione di utilità
      curva di livello o curve di indifferenza
      saggio marginale di sostituzione
        caratteristiche delle curve di indifferenza
          decrescenti
          concavità verso l'alto
          non si intersecano
          curve più alte->soddisfazione più elevata
        casi estremi
          sostituti perfetti
          complementari perfetti

\end{comment}

\section{Economia}
\label{sec:modelli_economia}

Per ``economia'' – dal greco %οἶκος
(\emph{oikos}), "casa" inteso anche come "beni di famiglia", e %νόμος 
(\emph{nomos}), "norma" o "legge" – si intende sia l'organizzazione 
dell'utilizzo di risorse scarse (limitate o finite) quando attuata al 
fine di soddisfare al meglio bisogni individuali o collettivi, sia 
un sistema di interazioni che garantisce un tale tipo di organizzazione, 
sistema detto anche \emph{sistema economico}.
% <ref>«\emph{Economics is the 
% science which studies human behaviour as a relationship between ends and 
% scarce 
% means which have alternative uses}»; la definizione, del~1932, è tratta da 
(\emph{Lionel Robbins, Essay on the Nature and Significance of Economic 
Science, Macmillan, London, 1945}
\url{http://mises.org/books/robbinsessay2.pdf})

Per l'economista e politico francese Raymond Barre
(\emph{Raymond Barre, Economie politique, Presses universitaires de France, 
1959}): 
\begin{quote}
 L'economia 
è la scienza della gestione delle risorse scarse. Essa prende in esame le forme 
assunte dal comportamento umano nella gestione di tali risorse; analizza e 
spiega le modalità secondo le quali un individuo o una società destinano mezzi 
limitati alla soddisfazione di esigenze molteplici ed illimitate.
\end{quote}


Per l'economista inglese Alfred Marshall:
(\emph{Alfred Marshall, Principi di Economia, 1890})
\begin{quotation}
 L'economia è uno studio del genere umano negli affari ordinari della vita. 
\end{quotation}

I soggetti che creano tali sistemi di organizzazione possono essere persone, 
organizzazioni o istituzioni. Normalmente si considerano i 
soggetti (detti anche "agenti" o "attori" o "operatori" economici) come attivi 
nell'ambito di un dato territorio; peraltro si tiene conto anche delle 
interazioni con altri soggetti attivi fuori dal territorio.

\section{Il sistema economico}
\label{sec:modelli_sistemaeconomico}

Il sistema economico, secondo la visione dell'economia di mercato della 
moderna società occidentale, è la rete di interdipendenze ed interconnessioni 
tra operatori o soggetti economici che svolgono le attività di produzione, 
consumo, scambio, lavoro, risparmio e investimento per soddisfare i bisogni 
individuali e realizzare il massimo profitto, ottimizzando l'uso delle risorse, 
evitando sprechi e aumentando la produttività individuale anche diminuendo il 
costo del lavoro.

\subsection{Componenti o sottosistemi}

I componenti o sottosistemi del sistema economico sono:

\begin{description}
 \item [Sistema di produzione], attraverso la produzione promuove e 
determina l'offerta di beni e servizi sotto 
continua spinta all'investimento per produrre innovazione 
(aziende e imprese).
 \item [Sistema dei consumatori], promuove e determina 
attraverso il consumo la domanda e offerta di beni e 
servizi (es. famiglie e in parte anche imprese).
 \item [Sistema creditizio-finanziario], da esso i 
precedenti sottosistemi afferiscono fondi di liquidità (capitali) e 
strumenti finanziari per 
promuovere e raggiungere i loro obiettivi (produzione e/o consumo) 
(banche e istituti di intermediazione finanziaria).
 \item [Mercato], è l'ambiente di interazione dei precedenti sottosistemi 
dove avviene lo scambio di beni, servizi e denaro tipicamente regolati 
dalla legge della domanda e dell'offerta.
 \item [Stato], alimenta il sistema economico attraverso la spesa 
pubblica (offerta di servizi pubblici a fronte di prelievo fiscale), 
regolandolo anche attraverso interventi mirati di politica economica 
(politica di bilancio e politica monetaria).
\end{description}

Il livello di sviluppo ed efficienza di tali sottosistemi e del relativo 
sistema economico riflette il livello di sviluppo della società stessa e varia 
in funzione delle epoche storiche o della parte del mondo o Stato considerato. 
Storicamente si passa da economie prettamente agricole ad economie 
agricole-industriali fino ad arrivare a economie 
agricole-industriali-terziarie, mentre attualmente e geograficamente si 
classifica l'efficienza dei sistemi economici con le denominazioni di primo 
mondo, secondo mondo, terzo mondo e quarto mondo. Il processo di 
globalizzazione sta gradualmente portando ad una progressiva 
omogeneizzazione dei vari sistemi economici mondiali grazie all'interdipendenza 
a livello internazionale dei vari mercati nazionali (internazionalizzazione).

\subsection{Operatori economici e loro funzioni}

% File:Trieste Assicurazioni Generali 04032007 01.jpg Le Assicurazioni 
% Generali a Trieste
Il sistema economico può definirsi, altresì, come l'ambiente o l'insieme delle 
attività promosse dagli operatori economici per le suddette finalità. Gli 
operatori economici svolgono una o più delle seguenti funzioni:
% <ref>La classificazione e le definizioni che seguono sono quelle usate in 
% ambito internazionale (secondo gli standard SNA delle Organizzazione delle 
% Nazioni Unite e Sistema europeo dei conti nazionali e regionali, SEC, da esso 
% derivato, dell'Unione europea) per l'analisi della struttura complessiva di un 
% sistema economico, di suoi aspetti specifici (ruolo dello Stato, sottosistemi 
% regionali ecc.), della sua evoluzione nel tempo, delle relazioni con altri 
% sistemi economici; in particolare, sono tratte da:
% * {{en}} Eurostat, 
% [http://circa.europa.eu/irc/dsis/nfaccount/info/data/esa95/en/titelen.htm 
% European system of accounts ESA 1995];
% * ISTAT, \emph{I conti degli italiani}, Bologna, Il Mulino, 2001.</ref>

\begin{itemize} [noitemsep]
\item produzione di beni e servizi;
\item consumo di beni e servizi;
\item intermediazione finanziaria;
\item accumulazione di ricchezza;
\item redistribuzione del reddito e della ricchezza;
\item assicurazione.
\end{itemize}

\subsubsection{Classificazione degli operatori}
Gli operatori economici vengono classificati secondo le funzioni svolte. Si 
hanno:

\begin{itemize} [noitemsep]
\item le famiglie, che consumano beni e servizi prodotti (prodotti nel 
territorio considerato, o importati, a cura di altri operatori, dal "resto del 
mondo"), ma possono anche produrre e accumulare (Impresa|imprese 
individuali, Azienda|aziende familiari);
\item le società che svolgono attività finalizzate al 
conseguimento di utili ed all'accumulazione:
  \begin{itemize} [noitemsep]
  \item le società di intermediazione finanziaria (in primo luogo le banche; 
  in Italia vi sono poi le Società di Intermediazione Mobiliare (SIM), le 
  Società di gestione del risparmio (SGR), le SICAV ecc.);
  \item le società di assicurazione;
  \item le società (dalle grandi società per azioni alle piccole società di 
  persone) che producono altri beni e servizi;
  \end{itemize}
\item la pubblica amministrazione, in tutte le sue articolazioni, che 
contribuisce al consumo (cosiddetti consumi collettivi), produce 
prevalentemente servizi non destinati alla vendita (istruzione, ordine 
pubblico, 
Ministero della difesa ecc.) e redistribuisce il reddito e la ricchezza 
tra gli operatori del sistema;
\item altre organizzazioni senza finalità di lucro, che erogano servizi a 
beneficio 
delle famiglie (partiti, sindacati dei 
lavoratori, organizzazioni religiose, associazioni culturali ricreative e 
sportive, enti di beneficenza ed assistenza).
\item Professionisti (avvocati, commercialisti, farmacisti...) che offrono 
servizi 
regolati da ordini professionali.
\end{itemize}

\subsubsection{Le operazioni economiche}

File:Euro coins and banknotes.jpg Monete e banconote in euro.
Gli operatori interagiscono ponendo in essere operazioni economiche che possono 
essere:
* operazioni su beni e servizi: sono sia quelle che danno origine a beni e 
servizi mediante la produzione o l'importazione, sia quelle che ad essi 
danno destinazione (consumi intermedi o finali, 
investimenti, esportazioni);
* operazioni finanziarie: consistono nell'acquisizione o cessione di 
attività finanziarie (acquisto di azioni o altri 
titoli, apertura di depositi, erogazione di prestiti ecc.);
* operazioni di distribuzione e redistribuzione del reddito e della ricchezza: 
fanno sì che il valore aggiunto generato dall'attività produttiva venga sia 
distribuito fra i fattori della produzione (percezione del profitto e 
del reddito da lavoro autonomo, distribuzione di redditi da 
capitale da parte delle società, pagamento di Redditi di lavoro 
redditi da lavoro dipendente), sia redistribuito tra gli operatori 
(riscossione di imposte e tasse, erogazione di contributi).
Vi sono poi altre operazioni quali gli ammortamenti o lo 
scambio di attività non finanziarie non prodotte (terreni, brevetti, licenze).

Tutte le operazioni indicate costituiscono \emph{flussi}; vengono pertanto 
misurate tenendo conto delle variazioni (creazione, trasformazione, scambio, 
trasferimento di valore) che intervengono in un dato periodo di tempo. Ad 
esempio, si misura l'insieme delle vendite effettuate da una società, oppure 
l'insieme delle imposte percepite dalla pubblica amministrazione, nel corso di 
un anno.

Le operazioni possono avere o non avere una contropartita. Nel primo caso (ad 
esempio, la vendita di un bene), ad un flusso di denaro o in natura corrisponde 
un flusso di beni o servizi di pari valore; nel secondo caso (ad esempio, 
l'erogazione delle pensioni) non vi è una diretta contropartita e si parla 
di operazioni unilaterali o trasferimenti.

\subsection{I settori economici}

Le diverse attività di produzione di beni e servizi vengono ripartite in 
settori economici.

Al livello più generale si usa la tradizionale distinzione tra:
* settore primario, che comprende l'agricoltura, la selvicoltura, 
la pesca, lo sfruttamento delle cave e delle miniere;
* settore secondario, che comprende l'industria in senso stretto, 
l'edilizia e l'artigianato;
* settore terziario, che produce e fornisce servizi.
Vengono attualmente utilizzate, tuttavia, classificazioni più articolate:
* l'ESCAP delle Organizzazione delle Nazioni Unite 
propone una classificazione che individua 20 settori 
economici;
\url{http://www.unescap.org/publications/accsectors.asp|titolo=Sett
ori economici ESCAP}
* la Divisione Statistica delle Organizzazione delle Nazioni 
Unite usa l'ISIC (International Standard Industrial Classification of All 
Economic Activities), che individua 21 settori (detti "sezioni");
* l'Eurostat, organo statistico della Commissione europea, usa la 
classificazione NACE, derivata dall'ISIC;
* in Italia, l'ISTAT adotta la classificazione ATECO, traduzione 
italiana del NACE.

\subsection{La ricchezza di un sistema economico}

File:Palazzo della banca d'italia (firenze) 03.JPG Palazzo della 
Banca d'Italia, Firenze
Gli operatori che svolgono la funzione di accumulazione danno luogo a 
variazioni delle attività del sistema. Altre variazioni possono 
manifestarsi indipendentemente dalla loro volontà (incendi, catastrofi 
naturali, ecc.).

Le attività si dividono in non finanziarie e finanziarie. Tra le 
prime rientrano:
* attività fisse materiali: terreni, abitazioni, macchine e impianti, mezzi 
di trasporto, giacimenti minerari ecc.;
* attività fisse immateriali: opere artistiche, software, brevetti ecc.;
* scorte di materie prime, prodotti in corso di lavorazione, prodotti 
finiti;
* oggetti di valore: pietre e metalli preziosi, oggetti 
di antiquariato ecc.
Tra le attività finanziarie vi sono monete, depositi, azioni ed altri titoli 
ecc.

La misurazione delle attività ad una certa data consente di determinare la 
ricchezza, a quella data, di un sistema economico (si tratta di uno 
\emph{stock}, non di un \emph{flusso}).

\section{Tipi di sistemi economici}
Si possono individuare diversi tipi di sistemi economici, sulla base della 
presenza di tutti, o solo di alcuni, degli operatori sopra indicati, della 
maggiore importanza di alcuni rispetto ad altri, di diverse modalità di 
esplicazione delle loro funzioni, di diverse regole per l'esecuzione delle 
operazioni. Su tali aspetti influiscono 
le Istituzione|istituzioni politiche e sociali, le 
Tecnologia|tecnologie disponibili, aspetti culturali e 
ideologia|ideologici.

Nel corso della storia si sono susseguiti diversi sistemi economici, mentre 
altri sono stati solo ideati e mai realizzati.

\subsection{Sistemi economici nella storia}

\subsubsection{Antichità}

File:Port of Piraeus Panoramic View.JPG Il porto del Pireo oggi
Vi è stata una grande varietà di sistemi economici nell'antichità. In generale 
si può dire che, per millenni, hanno dominato l'agricoltura, finalizzata 
prevalentemente all'autoconsumo, ed il commercio lungo vie d'acqua anche con 
terre lontane. Si faceva inoltre ampio ricorso alla schiavitù.

I Sumeri erano divisi in varie città-stato indipendenti, spesso in 
conflitto tra loro per il controllo di canali che 
delimitavano i territori e consentivano di irrigare i terreni drenando le acque 
in eccesso e trasportandole alle zone più lontane. Nelle città avevano grande 
importanza i templi, sia come luoghi di culto che come sedi di 
raccolta e di redistribuzione delle eccedenze agricole.

Presso i Babilonesi il re era anche il maggiore 
proprietario terriero e le sue terre erano coltivate dagli schiavi. Il codice 
di Hammurabi ci rivela che vi erano tre classi sociali: uomini liberi, che 
potevano essere proprietari terrieri ma anche medici, commercianti o artigiani; 
uomini semiliberi, senza possedimenti, e schiavi. Erano anche stati definiti 
contratti per molte operazioni economiche: baratto, 
compravendita, prestito, donazione, deposito, pegno, 
assunzione di lavoratori al momento del raccolto.

File:Poikile quadriportico Villa Adriana.jpg Villa Adriana a 
Tivoli
In Grecia coesistevano diversi sistemi economici. A Sparta la 
popolazione era divisa in tre gruppi: gli spartiati erano i soli cittadini a 
pieno titolo ed erano tenuti a curare l'addestramento militare ed a dotarsi di 
armi pesanti; i perieci erano liberi, curavano il commercio e l'artigianato, ma 
erano obbligati a pagare tributi senza godere di alcun diritto politico; gli 
iloti erano schiavi di proprietà dello Stato, come la terra. Lo Stato affidava 
agli spartiati sia appezzamenti di terra, sia iloti per lavorarla. L'economia 
spartana aveva quindi come fulcro la coltivazione di terre conquistate grazie 
alla guerra. Storia di Atene, invece, cercò la propria espansione 
economica nel commercio marittimo, soprattutto con Pisistrato, che favorì 
la crescita di una classe di commercianti, e con Pericle, che usò i tributi 
per collegare la città al porto del Pireo e per incrementare la flotta 
mercantile.

Roma privilegiò l'espansione territoriale, quindi l'agricoltura, fin 
dall'origine. Si possono distinguere due fasi: all'inizio prevalevano i piccoli 
e medi proprietari terrieri, che costituivano anche il nerbo dell'esercito; 
successivamente prevalse il latifondo e si dovette creare un esercito di 
mercenari. Il cambiamento fu indotto dalla crisi economica successiva alla 
seconda guerra punica, che rovinò molti proprietari terrieri; ne seguirono 
anche la crisi della repubblica e, dopo lotte interne durate due secoli, la 
nascita dell'impero. Il latifondo dette gradualmente vita all'"economia delle 
ville romane", centri di produzione agricola sempre più ampi e 
sontuosi.

Sia ad Atene che a Roma venne dato grande impulso alle opere pubbliche.

\subsubsection{Medioevo}

File:Rolandfealty.jpg Carlo Magno investe Rolando e gli consegna Durlindana
Si distinguono due fasi principali: Alto e Basso Medioevo.

Nell'Alto Medioevo si diffuse in un primo tempo l'economia curtense. 
Derivate dalle ville romane, le corti 
costituivano unità produttive autosufficienti, in cui il commercio aveva un 
ruolo limitato e gli scambi avvenivano spesso in natura. Si 
distinguevano in esse una \emph{pars dominica}, gestita direttamente dal 
"signore", ed una \emph{pars massaricia}, gestita da contadini, liberi o 
asserviti, che avevano comunque l'obbligo di versare al signore un terzo del 
prodotto e di svolgere alcune giornate lavorative gratuite sulla \emph{pars 
dominica} (corvée).

Con l'affermazione dell'Impero Carolingio, l'economia curtense si trasformò 
in economia feudale. In un primo tempo le terre appartenevano 
all'imperatore, che ne assegnava in comodato parti, dette feudi, 
a persone di sua fiducia dette vassalli. Questi ne curavano 
l'amministrazione e potevano a loro volta assegnarne parti ai valvassori; i 
vassalli riuscirono presto ad ottenere anche il diritto di trasmettere il feudo 
ai loro eredi.

Vi erano poi i servi della gleba, che erano obbligati a 
coltivare le terre padronali, dalle quali non potevano allontanarsi per 
trasferirsi altrove; potevano coltivare nel tempo libero le terre dette 
"servili", riconoscendo peraltro un'imposta detta decima al clero.

Nel Basso Medioevo si ebbero graduali ma significativi progressi sia 
nell'agricoltura che nei commerci. Nell'Europa settentrionale iniziarono a 
diffondersi la rotazione triennale e l'uso dell'aratro pesante, 
che consentirono aumenti delle rese agricole e, con ciò, la disponibilità di 
maggiori eccedenze da dedicare al commercio. Lo sviluppo del commercio favorì, 
a sua volta, la nascita e la crescente importanza delle città.

In Italia le città acquisirono importanza tale da costituirsi in comuni 
(trasformatisi poi in signorie) e, in qualche caso, in 
repubbliche marinare. Tra le città italiane più importanti si possono ricordare:

* Storia di Venezia, che aveva acquistato, con la diplomazia e con 
la guerra, il dominio di quei pochi territori dell'entroterra necessari ai 
traffici e utili per l'incremento delle entrate governative, ma curava 
soprattutto l'espansione commerciale via mare;
* Storia di Milano, che curava soprattutto l'agricoltura e 
l'allevamento del bestiame, le lavorazioni artigianali dei metalli e dei 
tessuti (sotto gli Sforza si svilupparono la coltivazione del gelso e la 
lavorazione della seta) ed il commercio interno, grazie anche ad una rete di 
canali che penetravano dentro la città; perseguì quindi l'espansione 
territoriale, ottenendo sotto i Visconti il controllo di buona parte 
dell'Italia 
centrosettentrionale;
* Storia di Firenze, che sviluppò notevolmente, fin dal XII 
secolo, sia l'artigianato che il commercio internazionale, tanto da essere 
definita la \emph{Wall street} del medioevo. I traffici internazionali si 
giovavano della valle dell'Arno e della Via Francigena che, 
collegando Roma e Canterbury, costituiva una delle più importanti vie 
di comunicazione europee in epoca medioevale. I mercanti fiorentini si 
inserirono presto nel circuito degli scambi europei: importavano l'allume 
dal Levante e panni semilavorati dalle Fiandre e 
dalla Francia; raffinavano quindi i tessuti ottenendone preziose stoffe che 
esportavano con notevoli guadagni. L'esigenza di mezzi di pagamento idonei al 
commercio internazionale favorì, a sua volta, una forte crescita del 
sistema bancario (i Medici erano banchieri). Nel corso del XV 
secolo Firenze da sola aveva un reddito superiore a quello dell'intera 
Inghilterra, grazie alle industrie e alle grandi banche fiorentine, circa 
ottanta tra sedi e filiali, queste ultime sparse in buona parte d'Europa.

Nel resto d'Europa si formarono invece fin dal XIII secolo i primi 
Stati nazionali, che furono poi i protagonisti dell'età 
moderna.

\subsubsection{Età moderna}

File:Columbus Taking Possession.jpg Cristoforo Colombo 
sbarcato nel Nuovo Mondo
L'età moderna è caratterizzata, in estrema sintesi, 
dall'espansione territoriale nelle regioni rese accessibili dalle scoperte 
geografiche, dallo sviluppo del commercio marittimo internazionale, dalla 
progressiva affermazione degli Stati nazionali come Stati 
assoluti, dall'affermazione di una aristocrazia fondiaria e di un ceto 
borghese dedito al commercio ed alla finanza.

L'Impero portoghese privilegiò la ricerca di rotte per raggiungere 
l'India, da cui provenivano le spezie importate in Europa, con l'obiettivo 
commerciale di scavalcare l'intermediazione araba ed il monopolio commerciale 
di Storia di Venezia. L'Impero spagnolo preferì invece la 
conquista territoriale e lo sfruttamento agricolo e minerario dell'America 
meridionale.

L'Inghilterra e i Paesi Bassi riuscirono poi a conquistare gradualmente 
le basi portoghesi dal Capo di Buona Speranza all'Oceano pacifico, 
affermandosi a loro volta come potenze commerciali. Nel XVII secolo, 
Amsterdam divenne il porto più importante del mondo e un centro di finanza 
internazionale. Successivamente, le guerre contro l'Inghilterra e la 
Francia indebolirono i Paesi Bassi a favore dell'Inghilterra. Qui 
la Gloriosa rivoluzione portò ad una forma di monarchia costituzionale 
basata sull'equilibrio tra il sovrano, i proprietari terrieri e la borghesia, 
nella quale venivano disciplinati i modi di finanziamento dello Stato sia 
attraverso i tributi (che dovevano essere approvati dal Parlamento), sia 
attraverso il debito pubblico (la Banca d'Inghilterra, una delle prime 
banche centrali, venne fondata nel 1694).

I Paesi Bassi, poi imitati dall'Inghilterra, furono anche la culla 
della prima rivoluzione agricola. 
Nei Paesi Bassi l'agricoltura veniva finalizzata prevalentemente alle esigenze 
del commercio (lino per le tele, coloranti per il panno, ecc.), mentre 
l'Inghilterra dette grande impulso alla coltivazione dei cereali, 
all'allevamento del bestiame ed alla produzione della lana e della seta.

\subsubsection{Età contemporanea}

File:12 (236012210).jpg Una catena di montaggio
L'età contemporanea inizia, da un punto di vista 
economico, con la rivoluzione industriale: un processo di evoluzione che da 
un'economia 
agricola-artigianale-commerciale 
portò ad un'economia industriale moderna, caratterizzata dall'uso 
generalizzato di macchine azionate da energia meccanica e 
dall'utilizzo di nuove fonti energetiche inanimate (in 
primo luogo i combustibili fossili).

Ne sono seguiti il progressivo declino dell'agricoltura (il numero degli 
occupati nel settore agricolo iniziò a diminuire costantemente dopo la Grande 
depressione del 1873-1895, detta \emph{Long Depression}) e, con esso, quello 
dell'aristocrazia, la crescente importanza della borghesia produttiva, lo 
sviluppo sostenuto delle città, l'estensione della produzione per il mercato e 
la tendenziale scomparsa di quella per l'autoconsumo, la nascita di un mercato 
del lavoro.

Attraverso grandi momenti di crisi economica (la \emph{Long Depression} e il 
crollo di Wall Street del 1929) e politica (la 
Prima guerra mondiale, la Rivoluzione russa, la Repubblica di 
Weimar), si sono affermati nel XX secolo tre diversi sistemi economici:
* l'economia di mercato: è basata sull'interazione degli operatori 
economici privati, con un ruolo limitato dello Stato (ordine pubblico, 
difesa, giustizia, istruzione, costruzione di infrastrutture);
* l'economia pianificata: in essa la gestione delle dinamiche del sistema 
economico compete allo Stato, che elabora piani di breve-media durata che 
stabiliscono gli obiettivi e regolano conseguentemente l'impiego delle risorse;
* l'economia mista: accanto all'interazione degli operatori privati, lo 
interviene direttamente nel 
funzionamento del sistema economico, a sostegno della produzione e 
dell'occupazione, utilizzando la spesa pubblica ed 
avvalendosi di politiche fiscali e monetarie.

Nelle economie moderne il motore della crescita economica spesso è stato 
rappresentato dall'innovazione 
tecnologica: questa componente è stata infatti in grado di generare un 
effetto a catena/valanga sulle altre variabili 
macroeconomiche con conseguenziale aumento dei 
consumi, della produttività (PIL) e dell'occupazione. 
Fondamentale per la creazione di innovazione sotto forma di ricerca e 
sviluppo è ed è stato anche l'accesso al credito degli istituti di 
credito da parte delle imprese per la promozione dei loro 
investimenti, cioè l'interazione forte tra i sottosistemi di 
produzione e consumo e il sistema creditizio-finanziario all'interno del 
sistema economico stesso.

\subsection{Sistemi economici ideati e mai realizzati pienamente}

Aspetti economici possono ravvisarsi in molte utopie. Nel XX secolo 
vi sono stati, peraltro, sistemi economici "ideali" che sono stati assunti come 
obiettivo da partiti politici:
* il comunismo, caratterizzato dall'abolizione della proprietà privata, 
dalla proprietà collettiva dei mezzi di produzione ed ispirato al motto "da 
ciascuno secondo le sue capacità, a ciascuno secondo le sue necessità";
* la socializzazione proposta dal 
fascismo, basata anch'essa sulla proprietà collettiva dei mezzi di 
produzione, ma nell'ambito dello Stato corporativo.

Oltre questi sistemi economici ne esiste un altro, diverso da essi perché 
apolitico: è il Venus Project, ideato da Jacque 
Fresco, basato sull'abbondanza delle risorse attraverso l'utilizzo della 
tecnologia odierna.

Un altro sistema economico apolitico è quello fondato sul modello di 
Ayres-Warr (base della green economy), simile alla teoria 
"dell'astronave" ove la terra è considerata un sistema chiuso, come una 
grande nave, la cui somma delle risorse non è infinita e in cui occorre quindi 
fare attenzione al rapporto tra lo sfruttamento delle risorse del territorio e 
le esigenze dell'umanità. In questo modello il saldo 
entropico viene escluso dalle convenzionali esternalità negative 
dell'economia neoclassica, perché fondate sulla fisica newtoniana.

\section{Studio dei sistemi economici}


File:GDP PPP per capita 2007 IMF.pngStati per PIL (PPA) pro 
Paesi in base al PIL (PPA) pro capite del 2007
L'Economia politica studia i sistemi economici per individuarne le leggi di 
funzionamento. L'economia politica in senso moderno nasce quando si afferma 
la separazione tra etica e politica e ci si pone espressamente il problema 
della potenza economica degli Stati. Per lungo tempo tale disciplina 
si è occupata prevalentemente di sistemi economici nazionali;
% <ref>Cfr. il 
% Mercantilismo, l'Indagine sulla natura e le cause della ricchezza delle 
% nazioni} di Adam Smith, il \emph{Sistema nazionale di economia politica} 
% di Friedrich List ecc.</ref> 
i suoi concetti e metodi si sono tuttavia 
progressivamente estesi allo studio sia di sistemi sociali di ogni genere 
(economia aziendale), sia di singoli settori economici (economia 
agraria, economia industriale ecc.).

La Statistica economica ha invece come obiettivo la misurazione degli 
aspetti quantitativi di un'economia, dalla misura di grandezze semplici e di 
loro aggregati, all'analisi della dinamica e alle previsioni economiche, alla 
stima e alla verifica di modelli di comportamenti economici. Ad esempio, lo 
stato di un'economia nazionale viene rilevato mediante la contabilità economica 
nazionale (in Europa si usa il sistema di conti detto Sec95).

La Storia economica tenta di ricostruire il funzionamento di sistemi 
economici del passato, avvalendosi sia dei concetti dell'economia politica 
che dei metodi della statistica economica.

A partire dalla conoscenza o analisi del sistema economico è possibile agire 
sul sistema economico stesso con misure o interventi di politica economica 
mirati a stimolarne la stabilità o la crescita economica.

La Filosofia dell'economia è una branca della filosofia che studia le 
questioni relative all'economia o, in alternativa, il settore dell'economia che 
si occupa delle proprie fondamenta e del proprio \emph{status} di scienza 
umana<ref>{{cita libro|nome=D. | cognome=Wade Hands|capitolo = philosophy and 
economics | titolo = The New Palgrave Dictionary of Economics |ed = 
2|anno=2008 | editore = | città=}}</ref>.

\begin{comment}
 


\section{Note}
<references />

== Bibliografia ==
* {{cita libro|autore=Pierluigi Ciocca|titolo=Il tempo dell'economia. 
Strutture, fatti, interpreti del Novecento|editore=Bollati 
Boringhieri|anno=2004, ISBN 978-88-339-1559-3}}
* {{cita libro|autore=Sidney Pollard|titolo=L'economia internazionale dal 
1945 ad oggi|editore=Editori Laterza|anno=1999 |ISBN= 88-420-5791-6}}
* {{cita libro|autore=André Gauthier|titolo=L'economia mondiale dal 1945 
a oggi|editore=Il Mulino|anno=1998 |ISBN= 88-15-06381-1}}

== Voci correlate ==
{{div col|cols=4}}
* Autarchia
* Capitalismo
* Commercio
* Dirigismo
* Economia aziendale
* Economia del benessere
* Economia del lavoro
* Economia della conoscenza
* Economia dell'informazione
* Economia dello sviluppo
* Economia d'Italia
* Economia di mercato
* Economia e politica agraria
* Economia finanziaria
* Economia industriale
* Economia internazionale
* Economia keynesiana
* Economia mista
* Economia mondiale
* Economia monetaria
* Economia neoclassica
* Economia pianificata
* Economia politica
* Economisti classici
* Finanza
* Fisiocrazia
* Globalizzazione
* Glossario economico
* Liberismo
* Macroeconomia
* Marginalismo
* Mercantilismo
* Mercato
* Microeconomia
* Monetarismo
* Politica
* Politica economica
* Settore economico
* Statalismo
* Statistica economica
* Storia del pensiero economico
* Storia economica
{{div col end}}

== Altri progetti ==
{{interprogetto|etichetta=economia|wikt=economia|s=Categoria:Testi di 
economia|s_preposizione=di|q|commons=Category:Economics}}

== Collegamenti esterni ==
* {{Thesaurus BNCF}}
* {{cita web|http://www.treccani.it/enciclopedia/tag/economia/|Lemma Economia 
sull’Enciclopedia Treccani online}}
* {{cita web|http://www.sapere.it/enciclopedia/econom%C3%ACa.html|Lemma 
Economia nell’Enciclopedia Sapere online}}

{{Scienze sociali}}

{{Controllo di autorità}}
{{Portale|economia}}

Categoria:Economia| Economia



\end{comment}









































