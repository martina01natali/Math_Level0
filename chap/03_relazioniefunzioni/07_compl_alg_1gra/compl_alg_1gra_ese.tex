% (c) 2012-2013 Claudio Carboncini - claudio.carboncini@gmail.com
% (c) 2012-2014 Dimitrios Vrettos - d.vrettos@gmail.com
% (c) 2015 Daniele Zambelli daniele.zambelli@gmail.com

\section{Esercizi}

\subsection{Esercizi dei singoli paragrafi}

% \subsubsection*{20.1 - Equazioni di grado superiore al primo riducibili al 
%                 primo grado}
\subsubsection*{\numnameref{sec:compl1_eqgradosup}}

\begin{esercizio}[\Ast]
\label{ese:20.1}
Risolvere le seguenti equazioni riconducendole a equazioni di primo grado.
\begin{multicols}{2}
\begin{enumeratea}
 \item $x^{2}+2x=0$ \hfill $\left[0;~-2\right]$
 \item $x^{2}+2x-9x-18=0$ \hfill $\left[-2;~+9\right]$
 \item $2x^{2}-2x-4=0$ \hfill $\left[2;~-1\right]$
 \item $4x^{2}+16x+16=0$ \hfill $\left[-2\right]$
 \item $x^{2}-3x-10=0$ \hfill $\left[5;~-2\right]$
 \item $x^{2}+4x-12=0$ \hfill $\left[2;~-6\right]$
 \item $3x^{2}-6x-9=0$ \hfill $\left[3;~-1\right]$
 \item $x^{2}+5x-14=0$ \hfill $\left[2;~-7\right]$
 \item $-3x^{2}-9x+30=0$ \hfill $\left[2;~-5\right]$
 \item $7x^{2}+14x-168=0$ \hfill $\left[4;~-6\right]$
 \item $x^{4}-16x^{2}=0$ \hfill $\left[-4;~0;~+4\right]$
 \item $2x^{3}+2x^{2}-20x+16=0$ \hfill $\left[-4;~+1;~+2\right]$
 \item $-2x^{3}+6x+4=0$ \hfill $\left[2;~-1\right]$
 \item $-2x^{3}+10x^{2}+82x-90=0$ \hfill $\left[1;~+9;~-5\right]$
\end{enumeratea}
\end{multicols}
\end{esercizio}

\begin{esercizio}[\Ast]
\label{ese:20.6}
Risolvere le seguenti equazioni riconducendole a equazioni di primo grado.
\begin{multicols}{2}
\begin{enumeratea}
 \item $-x^{6}+7x^{5}-10x^{4}=0$ \hfill $\left[0;~+2;~+5\right]$
 \item $x^{3}-3x^{2}-13x+15=0$ \hfill $\left[-3;~+1;~+5\right]$
 \item $x^{2}+10x-24=0$ \hfill $\left[-12;~2\right]$
 \item $2x^{3}-2x^{2}-24x=0$ \hfill $\left[-3;~0;~+4\right]$
 \item $x^{4}-5x^{2}+4=0$ \hfill $\left[-2;~-1;~+1;~+2\right]$
 \item $-x^{3}-5x^{2}-x-5=0$ \hfill $\left[-5\right]$
 \item $-4x^{4}-28x^{3}+32x^{2}=0$ \hfill $\left[0;~+1;~-8\right]$
 \item $5x^{3}+5x^{2}-80x-80=0$ \hfill $\left[-1;~+4;~-4\right]$
 \item $-3x^{3}+18x^{2}+3x-18=0$ \hfill $\left[1;~-1;~+6\right]$
 \item $4x^{3}+8x^{2}-16x-32=0$ \hfill $\left[2;~-2\right]$
 \item $x^{3}+11x^{2}+26x+16=0$ \hfill $\left[-1;~-2;~-8\right]$
 \item $2x^{3}+6x^{2}-32x-96=0$ \hfill $\left[4;~-4;~-3\right]$
 \item $2x^{3}+16x^{2}-2x-16=0$ \hfill $\left[1;~-1;~-8\right]$
 \item $-2x^{3}+14x^{2}-8x+56=0$ \hfill $\left[7\right]$
\end{enumeratea}
\end{multicols}
\end{esercizio}

\begin{esercizio}[\Ast]
\label{ese:20.9}
Risolvere le seguenti equazioni riconducendole a equazioni di primo grado.
\begin{enumeratea}
 \item $-{\dfrac{3}{2}}x^{2}+\dfrac{3}{2}x+63=0$ \hfill $\left[7;~-6\right]$
 \item $\dfrac{7}{2}x^{2}+7x-168=0$ \hfill $\left[6;~-8\right]$
 \item $\dfrac{3}{4}x^{3}-\dfrac{3}{4}x=0$ \hfill $\left[0;~+1;~-1\right]$
 \item $-{\dfrac{6}{5}}x^{3}-\dfrac{6}{5}x^{2}+\dfrac{54}{5}x+\dfrac{54}{5}=0$ 
  \hfill $\left[-1;~+3;~-3\right]$
 \item $2x^{3}+12x^{2}+18x+108=0$ \hfill $\left[-6\right]$
 \item $x^{4}-10x^{3}+35x^{2}-50x+24=0$ \hfill $\left[1;~+2;~+3;~+4\right]$
 \item $-2x^{3}-12x^{2}+18x+28=0$ \hfill $\left[-1;~+2;~-7\right]$
 \item $(x^{2}-6x+8)(x^{5}-3x^{4}+2x^{3})=0$ 
  \hfill $\left[0;~+1;~+2;~+4\right]$
 \item $\left(25-4x^{2}\right)^{4}\left(3x-2\right)^{2}=0$ 
  \hfill $\left[\frac{5}{2};~-\frac{5}{2};~\frac{2}{3}\right]$
 \item $(x-4)^{3}\left(2x^{3}-4x^{2}-8x+16\right)^{9}=0$ 
  \hfill $\left[4;~+2;~-2\right]$
 \item $(x^{3}-x)(x^{5}-9x^{3})(x^{2}+25)=0$ 
  \hfill $\left[0;~+1;~-1;~+3;~-3\right]$
 \item $x^{5}+3x^{4}-11x^{3}-27x^{2}+10x+24=0$ 
  \hfill $\left[1;~-1;~-2;~+3;~-4\right]$
 \item $2x^{2}-x-1=0$ \hfill $\left[1;~-\frac{1}{2}\right]$
 \item $3x^{2}+5x-2=0$ \hfill $\left[-2;~\frac{1}{3}\right]$
 \item $-5x^{4}+125x^{2}+10x^{3}-10x-120=0$ \hfill $\left[1;~-1;~-4;~+6\right]$
 \item $\dfrac{7}{6}x^{4}-\dfrac{161}{6}x^{2}-21x+\dfrac{140}{3}=0$ 
  \hfill $\left[1;~-2;~+5;~-4\right]$
 \item $6x^{2}+x-2=0$ \hfill $\left[\frac{1}{2};~-\frac{2}{3}\right]$
 \item $2x^{3}-x^{2}-2x+1=0$ \hfill $\left[1;~-1;~\frac{1}{2}\right]$
 \item $3x^{3}-x^{2}-8x-4=0$ \hfill $\left[-1;~2;~-\frac{2}{3}\right]$
 \item $8x^{3}+6x^{2}-5x-3=0$ 
  \hfill $\left[-1;~-\frac{1}{2};~\frac{3}{4}\right]$
 \item $6x^{3}+x^{2}-10x+3=0$ 
  \hfill $\left[1;~\frac{1}{3};~-\frac{3}{2}\right]$
 \item $4x^{4}-8x^{3}-13x^{2}+2x+3=0$ 
  \hfill $\left[3;~-1;~\frac{1}{2};~-\frac{1}{2}\right]$
 \item $8x^{4}-10x^{3}-29x^{2}+40x-12=0$ 
  \hfill $\left[2;~-2;~\frac{3}{4};~\frac{1}{2}\right]$
 \item $-12x^{3}+68x^{2}-41x+5=0$ 
  \hfill $\left[5;~\frac{1}{2};~\frac{1}{6}\right]$
 \item $(x^{4}+3x^{3}-3x^{2}-11x-6)(4x^{6}-216x^{3}+2916)=0$ 
  \hfill $\left[-1;~+2;~+3;~-3\right]$
\end{enumeratea}
\end{esercizio}

% \subsubsection*{20.2 - Equazioni numeriche frazionarie}
\subsubsection*{\numnameref{sec:compl1_eqfratte}}

\begin{esercizio}[\Ast]
\label{ese:20.15}
Risolvi le seguenti equazioni frazionarie.
\begin{multicols}{2}
\begin{enumeratea}
 \item $\dfrac{2}{x+1}=\dfrac{1}{x+2}$ \hfill $\left[-3\right]$
 \item $\dfrac{1}{x-1}=2$ \hfill $\left[\frac{3}{2}\right]$
 \item $1-\dfrac{1}{x+1}=0$ \hfill $\left[0\right]$
 \item $\dfrac{2x-4}{x-2}=0$ \hfill $\left[\emptyset\right]$
 \item $\dfrac{x}{x+1}-\dfrac{1}{x-1}=1$ \hfill $\left[0\right]$
 \item $\dfrac{1}{x-3}=\dfrac{x}{3-x}$ \hfill $\left[-1\right]$
 \item $\dfrac{x-1}{x^{2}-4}=-{\dfrac{5}{x+2}}$ 
  \hfill $\left[\frac{11}{6}\right]$
 \item $\dfrac{3}{x+1}=\dfrac{2}{x+1}$ \hfill $\left[\emptyset\right]$
 \item $\dfrac{1}{3-x}-\dfrac{4}{2x-6}=0$ \hfill $\left[\emptyset\right]$
 \item $\dfrac{x^{2}-1}{x-1}-1=2x+1$ \hfill $\left[-1\right]$
 \item $\dfrac{x}{x^{2}-4}=\dfrac{1}{x+2}$ \hfill $\left[\emptyset\right]$
 \item $\dfrac{1}{x}-\dfrac{3}{x^{2}}=\dfrac{2-2x}{x^{3}}$ 
  \hfill $\left[2;~-1\right]$
 \item $\dfrac{x-2}{x-1}=\dfrac{x-1}{x-2}$ \hfill $\left[\frac{3}{2}\right]$
 \item $\dfrac{x+3}{x+1}=x+3$ \hfill $\left[0;~-3\right]$
 \item $\dfrac{3x+1}{3x^{2}+x}=1$ \hfill $\left[1\right]$
 \item $\dfrac{6+x}{x-3}=\dfrac{x^{2}}{x-3}$ \hfill $\left[-2\right]$
 \item $\dfrac{1}{x-2}+\dfrac{2}{x+1}=\dfrac{3}{x^{2}-x-2}$ 
  \hfill $\left[\emptyset\right]$
 \item $\dfrac{5}{x-2}-\dfrac{6}{x+1}=\dfrac{3x-1}{x^{2}-x-2}$
  \hfill $\left[\frac{9}{2}\right]$
 \item $\dfrac{1}{1-x}-\dfrac{x}{x-1}=0$
  \hfill $\left[-1\right]$
%  \item $\dfrac{x+1}{x-1}-\dfrac{x}{1+x}=0$
%   \hfill $\left[-{\frac{1}{3}}\right]$
%  \item $\dfrac{2x+1}{2x-1}+\dfrac{4x^{2}+1}{4x^{2}-1}=2$
%   \hfill $\left[-1\right]$
%  \item $\dfrac{1}{x-1}+\dfrac{2}{x}+\dfrac{1}{x^{2}-x}=0$
%   \hfill $\left[\frac{1}{3}\right]$
%  \item $\dfrac{x-1}{x^{2}-2x+1}=\dfrac{2}{2-2x}$
%   \hfill $\left[\emptyset\right]$
%  \item $4-x^{2}=\dfrac{x^{2}+5x+6}{x+2}-1$
%   \hfill $\left[1;~-2\right]$
\end{enumeratea}
\end{multicols}
\end{esercizio}

\begin{esercizio}[\Ast]
\label{ese:20.21}
Risolvi le seguenti equazioni frazionarie.
\begin{multicols}{2}
\begin{enumeratea}
 \item $\dfrac{5}{5x+1}+\dfrac{2}{2x-1}=\dfrac{1}{1-2x}$
  \hfill $\left[\frac{2}{25}\right]$
 \item $\dfrac{1}{x-2}+\dfrac{2}{x+1}=\dfrac{3}{x^{2}-x-2}$
  \hfill $\left[\emptyset\right]$
 \item $\dfrac{30}{x^{2}-25}+\dfrac{3}{5-x}=0$
  \hfill $\left[\emptyset\right]$
 \item $1+\dfrac{x-1}{x+1}=\dfrac{1}{x-2}+\dfrac{1-x^{2}}{x^{2}-x-2}$
  \hfill $\left[-{\frac{1}{3}}\right]$
 \item $-{\dfrac{3x}{6-2x}}+\dfrac{5x}{10-5x}=\dfrac{1-x}{4-2x}$
  \hfill $\left[\frac{3}{4}\right]$
 \item $\dfrac{1}{x+3}-\dfrac{1}{2-x}=\dfrac{x+3}{x^{2}+x-6}$
  \hfill $\left[\emptyset\right]$
 \item $\dfrac{1+2x}{1-2x}+\dfrac{1-2x}{1+2x}=\dfrac{6-8x^{2}}{1-4x^{2}}$
  \hfill $\left[\emptyset\right]$
 \item $\dfrac{3x}{x-2}+\dfrac{6x}{x^{2}-4x+4}=\dfrac{3x^{2}}{(x-2)^{2}}$
  \hfill $\left[\insR-\{2\}\right]$
 \item $(4x+6)\left(\dfrac{4}{x+1}-\dfrac{1}{x-1}\right)=0$
  \hfill $\left[-{\frac{3}{2}};~\frac{5}{3}\right]$
 \item $\dfrac{2x+1}{x+3}+\dfrac{1}{x-4}=\dfrac{4x-9}{x^{2}-x-12}$
  \hfill $\left[1\right]$
 \item $\dfrac{1}{x-1}-\dfrac{1}{x}=\dfrac{(x+1)^{2}}{2(x^{2}-1)}+1$
  \hfill $\left[-{\frac{2}{3}}\right]$
 \item $\dfrac{x^{2}-1}{x+1}-\dfrac{1}{x+2}=\dfrac{x+1}{x+2}-x$
  \hfill $\left[1\right]$
 \item $\dfrac{1}{x+3}-\dfrac{2}{x+2}=\dfrac{3x-6}{x^{2}+5x+6}$
  \hfill $\left[\frac{1}{2}\right]$
 \item $\dfrac{2x-3}{x+2}+\dfrac{1}{x-4}=\dfrac{2}{x^{2}-2x-8}$
  \hfill $\left[2;~3\right]$
 \item $\dfrac{x-1}{x+2}-\dfrac{x+2}{x-1}=\dfrac{1}{x^{2}+x-2}$
  \hfill $\left[-\frac{2}{3}\right]$
%  \item $\dfrac{3}{x-1}+\dfrac{1}{x+1}=\dfrac{12-x}{x^{2}-1}$
%   \hfill $\left[2\right]$
%  \item $\dfrac{x}{2x+1}+\dfrac{x+1}{2(x+2)}=\dfrac{x-1}{2x^{2}+5x+2}$
%   \hfill $\left[\emptyset\right]$
%  \item $\dfrac{3x+1}{x^{2}-9}+\dfrac{2}{3x^{2}-9x}=\dfrac{3}{x+3}$
%   \hfill $\left[-{\frac{3}{16}}\right]$
%  \item $\dfrac{1}{x^{2}-3x+2}+\dfrac{2}{x-1}=0$
%   \hfill $\left[\frac{3}{2}\right]$
%  \item $\dfrac{x+2}{(x-3)^{2}}-\dfrac{1}{x-3}=\dfrac{4}{9-3x}$
%   \hfill $\left[-{\frac{3}{4}}\right]$
\end{enumeratea}
\end{multicols}
\end{esercizio}

\begin{esercizio}[\Ast]
\label{ese:20.28}
Risolvi le seguenti equazioni frazionarie.
\begin{enumeratea}
 \item $\dfrac{18x^{2}-9x-45}{4-36x^{2}}-\dfrac{6x+1}{9x-3}+
        \dfrac{21x-1}{18x+6}=0$ \hfill $\left[\frac{7}{3}\right]$
 \item $\dfrac{5x}{3x^{2}-18x+15}-\dfrac{2}{3x-3}=\dfrac{5}{18x-90}$
  \hfill $\left[-5\right]$
 \item $(x-4)(x+3)=\dfrac{(x-4)(x+3)}{x-2}$
  \hfill $\left[4;~-3;~3\right]$
 \item $\left(\dfrac{1}{x+5}-\dfrac{1}{5}\right):\left(\dfrac{1}{x-5}+
        \dfrac{1}{5}\right)+\dfrac{x^{2}}{x^{2}-5x}=0$
  \hfill $\left[\frac{5}{3}\right]$
 \item $\dfrac{1}{3x+2}-\dfrac{3}{2-x}=\dfrac{10x+4}{3x^{2}-4x-4}$
  \hfill $\left[\insR-\left\{-{\frac{2}{3}};~2\right\}\right]$
 \item $\dfrac{1}{2} \left(x-\dfrac{1}{x}\right)-
        2\left(1-\dfrac{1}{x}\right)=\dfrac{x^{2}-1}{x}$
  \hfill $\left[-5;~+1\right]$
 \item $\dfrac{3(2x-3)}{x^{3}+27}+\dfrac{1}{x+3}=\dfrac{x}{x^{2}-3x+9}$
  \hfill $\left[\insR-\{-3\}\right]$
 \item $\dfrac{2x-1}{3x^{2}-75}-\dfrac{3-x}{x+5}+\dfrac{x-3}{10-2x}=
        \dfrac{7}{25-x^{2}}$
  \hfill $\left[\frac{35}{3}\right]$
 \item $\left(40-10x^{2}\right)^{3} \left(\dfrac{3x-1}{x+2}-
        \dfrac{3x}{x+1}\right)=0$
  \hfill $\left[2;~-\frac{1}{4};~-2\right]$
 \item $\dfrac{1+2x}{x^{2}+2x}+\dfrac{x^{3}-6x+1}{x^{2}-4}=
        \dfrac{x^{2}-2x}{x-2}+\dfrac{1}{x^{2}-2x}$
  \hfill $\left[-\frac{4}{3}\right]$
%  \item $\left(1-\dfrac{1}{2}x\right):\left(1+\dfrac{1}{2}x\right)=
%         \dfrac{2x+1}{6x+3}-\dfrac{1}{2}x+\dfrac{x^{2}}{2x+4}$
%   \hfill $\left[4\right]$
%  \item $\dfrac{3x-1}{1-2x}+\dfrac{x}{2x-1}-\dfrac{x^{3}-8}{x^{2}-4}:
%         \dfrac{x^{2}+2x+4}{x^{2}+2x+1}=
%         \dfrac{2-3x}{2x-6}\cdot {\dfrac{x^{2}-9}{4-9x^{2}}}-\dfrac{6x+7}{6}$
%   \hfill $\left[-{\frac{26}{25}}\right]$
%  \item $\dfrac{2x}{6x-3}+\dfrac{x}{4-8x}+\left(\dfrac{1}{2x+1}-
%         \dfrac{1}{2x-1}\right)\cdot 
%         {\dfrac{2x\left(x^{2}-1\right)}{8x^{2}-4x}}=
%         \dfrac{x^{2}(5x-3)}{3(2x+1)(2x-1)^{2}}$
%   \hfill $\left[\frac{12}{5}\right]$
%  \item $\dfrac{3x^{2}-2x+3}{x^{2}-3x}+\dfrac{x+2}{3-x}=
%         \left(\dfrac{x+1}{x}-1\right)\left(\dfrac{x^{2}}{x^{3}-27}+
%         \dfrac{x}{x-3}\right):\dfrac{3x}{3x^{3}-81}+\dfrac{x^{2}-x+2}{3-x}$
%   \hfill $\left[-30\right]$
\end{enumeratea}
\end{esercizio}

\begin{esercizio}
\label{ese:20.29}
$\left(2x-4x^{2}+7\right)^{6}=-{\dfrac{1}{\left(x^{2}-5x+7\right)^{4}}}$ 
Osservando i due membri dell'equazione, senza svolgere i calcoli, puoi subito 
affermare che non esiste alcun numero reale che rende vera l'uguaglianza?
  \hfill $\left[\right]$
\end{esercizio}

\begin{esercizio}[\Ast]
\label{ese:20.30}
Quale numero occorre aggiungere a numeratore e denominatore della frazione 
tre settimi perché essa raddoppi di valore?
  \hfill $\left[21\right]$
\end{esercizio}

\begin{esercizio}[\Ast]
\label{ese:20.31}
Quale numero occorre aggiungere a numeratore e denominatore della frazione 
due settimi perché essa triplichi di valore?
  \hfill $\left[28\right]$
\end{esercizio}

\begin{esercizio}
\label{ese:20.32}
Due amici A e B partono con le loro automobili nello stesso istante da due 
località diverse; A fa un viaggio di $100\unit{Km}$ a una certa velocità, 
B fa un viaggio di~$132\unit{Km}$ ad una velocità che supera quella dell'amico 
di~$20\unit{Km/h}$
I due amici arrivano nello stesso istante all'appuntamento. 
Qual è la velocità di A?
\begin{center}
 \input{\folder lbr/fig000_eser.pgf}
\end{center}
Traccia di soluzione:
\begin{itemize}
 \item se A e B partono insieme e arrivano insieme significa che hanno 
 impiegato lo stesso tempo per fare il proprio viaggio;
 \item il tempo è dato dal rapporto tra lo spazio percorso e la velocità;
 \item la velocità di A è l'incognita del problema: la indichiamo con~$x$
 \item l'equazione risolvente è~$\dfrac{110}{x}=\dfrac{132}{x+20}$
\end{itemize}
Prosegui nella risoluzione.
\end{esercizio}

\begin{esercizio}
\label{ese:20.33}
Per percorrere~$480\unit{Km}$ un treno impiega~$3$ ore di più di quanto 
impiegherebbe un aereo a percorrere~$1920\unit{Km}$
L'aereo viaggia ad una velocità media che è~$8$ volte quella del treno. 
Qual è la velocità del treno?
\end{esercizio}

% \subsubsection*{20.3 - Equazioni letterali}
\subsubsection*{\numnameref{sec:compl1_eqletterali}}

\begin{esercizio}[\Ast]
\label{ese:20.34}
Risolvi e discuti le seguenti equazioni letterali nell'incognita~$x$
\begin{enumeratea}
 \item $1+2x=a+1-2x$
\hfill $\left[\forall a\in \insR \rightarrow \left\{\frac{a}{4}\right\}\right]$
 \item $2x-\dfrac{7}{2}=ax-5$
\hfill $\left[a=2 \rightarrow \emptyset; \quad a \neq~2 \rightarrow 
              \left\{\frac{3}{2(a-2)}\right\}\right]$
 \item $b^{2}x=2b+bx$
\hfill $\left[b=0 \rightarrow \insR; \quad 
              b=1\rightarrow\emptyset; \quad 
              b\neq~0\wedge b\neq~1\rightarrow \left\{\frac{2}{b-1}\right\}
        \right]$
 \item $ax+2=x+3$
\hfill $\left[a=1\rightarrow \emptyset; \quad 
              a\neq~1\rightarrow \left\{\frac{1}{a-1}\right\}\right]$
 \item $k(x+2)=k+2$
\hfill $\left[k=0 \rightarrow \emptyset; \quad k\neq~0 \rightarrow 
\left\{\frac{2-k}{k}\right\}\right]$
 \item $(b+1)(x+1)=0$
\hfill $\left[b=-1 \rightarrow \insR; \quad b\neq -1 \rightarrow 
\left\{-1\right\}\right]$
 \item $k^{2}x+2k=x+2$
\hfill $\left[k=1 \rightarrow \insR; \quad k=-1 \rightarrow 
\emptyset; \quad k\neq~1\wedge k\neq -1 \rightarrow 
\left\{-{\frac{2}{k+1}}\right\}\right]$
 \item $(a-1)(x+1)=x+1$
\hfill $\left[a=2 \rightarrow \insR; \quad a\neq~2 \rightarrow 
\left\{-1\right\}\right]$
 \item $ax+x-2a^{2}-2ax=0$
  \hfill $\left[\right]$
 \item $3ax-2a=x\cdot (1-2a)+a\cdot (x-1)$
  \hfill $\left[\right]$
%  \item $x (3-5a)+2 (a-1)=(a-1) (a+1)$
%   \hfill $\left[\right]$
%  \item $x+2a\cdot (x-2a)+1=0$
%   \hfill $\left[\right]$
%  \item $(a-1)(x+1)=a-1$
%   \hfill $\left[a=1 \rightarrow \insR; \quad a\neq~1 \rightarrow \{0\}\right]$
%  \item $2k(x+1)-2=k(x+2)$
%   \hfill $\left[k=0 \rightarrow \emptyset; \quad 
%   k\neq~0 \rightarrow \left\{\frac{2}{k}\right\}\right]$
%  \item $a(a-1)x=2a(x-5)$
%   \hfill $\left[a=0 \rightarrow \insR; \quad 
%        a=3\rightarrow \emptyset; \quad 
%        a\neq~0 \wedge a\neq~3 \rightarrow \left\{\frac{10}{3-a}\right\}\right]$
%  \item $3ax+a=2a^{2}-3a$
%   \hfill $\left[a=0 \rightarrow \insR; \quad 
%           a\neq~0 \rightarrow \left\{\frac{2}{3} (a-2)\right\}\right]$
\end{enumeratea}
\end{esercizio}

\begin{esercizio}[\Ast]
\label{ese:20.38}
Risolvi e discuti le seguenti equazioni letterali nell'incognita~$x$
\begin{enumeratea}
 \item $3x-a=a(x-3)+6$
\hfill $\left[a=3 \rightarrow \insR; \quad a\neq~3 \rightarrow \{2\}\right]$
 \item $2+2x=3ax+a-a^{2}x$
\hfill $\left[a=2 \rightarrow \insR; \quad 
a=1 \rightarrow \emptyset; \quad 
a\neq~2\wedge a\neq~1 \rightarrow \left\{\frac{1}{a-1}\right\}\right]$
 \item $x(a^{2}-4)=a+2$
\hfill $\left[a=2 \rightarrow \emptyset; \quad 
a=-2 \rightarrow \insR; \quad 
a\neq -2\wedge a\neq~2 \rightarrow \left\{\frac{1}{a-2}\right\}\right]$
 \item $(x-m)(x+m)=(x+1)(x-1)$
\hfill $\left[m=1\vee m=-1 \rightarrow \insR; \quad 
m\neq~1\wedge m\neq -1 \rightarrow \emptyset\right]$
 \item $(a-2)^{2}x+(a-2)x+a-2=0$
\hfill $\left[a=2 \rightarrow \insR; \quad 
a=1 \rightarrow \emptyset; \quad 
a\neq~1\wedge a\neq~2 \rightarrow \left\{\frac{1}{1-a}\right\}\right]$
 \item $\left(9a^{2}-4\right)x=2(x+1)$
\hfill $\left[3a^{2}-2=0 \rightarrow \emptyset; \quad 
3a^{2}-2\neq~0 \rightarrow \left\{\frac{2}{3(3a^{2}-2)}\right\}\right]$
 \item $(a-1)x=a^{2}-1$
\hfill $\left[a=1 \rightarrow \insR; \quad 
a\neq~1 \rightarrow \{a+1\}\right]$
 \item $(a+2)x=a^{2}+a-1$
\hfill $\left[a=-2 \rightarrow \emptyset; \quad 
a\neq -2 \rightarrow \left\{\frac{a^{2}+a-1}{a+2}\right\}\right]$
 \item $a(x-1)^{2}=a(x^{2}-1)+2a$
\hfill $\left[a=0 \rightarrow \insR; \quad a\neq~0 \rightarrow \{0\}\right]$
%  \item $a^{3}x-a^{2}-4ax+4=0$
% \hfill $\left[a=-2\vee a=2 \rightarrow \insR; \quad 
% a=0 \rightarrow \emptyset; \quad 
% a\neq -2\wedge a\neq~0\wedge a\neq~2 \rightarrow 
%   \left\{\frac{1}{a}\right\}\right]$
 \item $bx\left(b^{2}+1\right)-(bx-1)\left(b^{2}-1\right)=2b^{2}$
\hfill $\left[b=0 \rightarrow \emptyset; \quad 
b\neq~0 \rightarrow \left\{\frac{1+b^{2}}{2b}\right\}\right]$
%  \item $a(a-5)x+a(a+1)=-6(x-1)$
% \hfill $\left[a=2 \rightarrow \insR; \quad 
% a=3 \rightarrow \emptyset; \quad 
% a\neq~2\wedge a\neq~3 \rightarrow \left\{\frac{a+3}{3-a}\right\}\right]$
%  \item $(x+a)^{2}-(x-a)^{2}+(a-4)(a+4)=a^{2}$
% \hfill $\left[a=0 \rightarrow \emptyset; \quad 
% a\neq~0 \rightarrow \left\{\frac{4}{a}\right\}\right]$
%  \item $b(b+3)+x\left(6-b^{2}\right)=bx$
% \hfill $\left[b=-3 \rightarrow \insR; \quad 
% b=2 \rightarrow \emptyset; \quad 
% b\neq -3\wedge b\neq~2 \rightarrow \left\{\frac{b}{b-2}\right\}\right]$
\end{enumeratea}
\end{esercizio}

% \begin{esercizio}[\Ast]
% \label{ese:20.41}
% Risolvi e discuti le seguenti equazioni nell'incognita~$x$ con due parametri.
% \begin{multicols}{2}
% \begin{enumeratea}
%  \item $(m+1)(n-2)x=0$
%  \item $m(x-1)=n$
%  \item $(a+1)(b+1)x=0$
%  \item $(m+n)(x-1)=m-n$
% \end{enumeratea}
% \end{multicols}
% \end{esercizio}
% 
% \begin{esercizio}[\Ast]
% \label{ese:20.42}
% Risolvi e discuti le seguenti equazioni nell'incognita~$x$ con due parametri.
% \begin{multicols}{2}
% \begin{enumeratea}
%  \item $x(2a-1)+2b(x-2)=-4a-x$
%  \item $ax-3+b=2(x+b)$
%  \item $(a+1)x=b+1$
%  \item $(a+b)(x-2)+3a-2b=2b(x-1)$
% \end{enumeratea}
% \end{multicols}
% \end{esercizio}
% 
% \begin{esercizio}[\Ast]
% \label{ese:20.43}
% Risolvi e discuti le seguenti equazioni nell'incognita~$x$ con due parametri.
% \begin{enumeratea}
%  \item $x(x+2)+3ax=b+x^{2}$
%  \item $(x-a)^{2}+b(2b+1)=(x-2a)^{2}+b-3a^{2}$
% \end{enumeratea}
% \end{esercizio}
% 
% \begin{esercizio}[\Ast]
% \label{ese:20.44}
% Risolvi e discuti le seguenti equazioni che presentano il parametro al denominatore.
% \begin{multicols}{2}
% \begin{enumeratea}
%  \item $\dfrac{x+2}{6a}+\dfrac{x-1}{2a^{2}}=\dfrac{1}{3a}$
%  \item $\dfrac{x-1}{b}+\dfrac{2x+3}{4b}=\dfrac{x}{4}$
%  \item $\dfrac{2x-1}{3a}+\dfrac{x}{3}=\dfrac{2}{a}$
%  \item $\dfrac{x}{a}+\dfrac{2x}{2-a}=\dfrac{a-x+2}{2a-a^{2}}$
% \end{enumeratea}
% \end{multicols}
% \end{esercizio}
% 
% \begin{esercizio}[\Ast]
% \label{ese:20.45}
% Risolvi e discuti le seguenti equazioni che presentano il parametro al denominatore.
% \begin{multicols}{2}
% \begin{enumeratea}
%  \item $\dfrac{x}{a-1}+8=4a-\dfrac{x}{a-3}$
%  \item $\dfrac{x-1}{a-1}+\dfrac{x+a}{a}=\dfrac{a-1}{a}$
%  \item $\dfrac{a^{2}-9}{a+2}x=a-3$
%  \item $\dfrac{x+2}{a^{2}-2a}+\dfrac{x}{a^{2}+2a}+\dfrac{1}{a}=\dfrac{2}{a^{2}-4}$
% \end{enumeratea}
% \end{multicols}
% \end{esercizio}
% 
% \begin{esercizio}[\Ast]
% \label{ese:20.46}
% Risolvi e discuti le seguenti equazioni che presentano il parametro al denominatore.
% \begin{enumeratea}
%  \item $\dfrac{x+1}{a^{2}+2a+1}+\dfrac{2x+1}{a^{2}-a-2}-\dfrac{2x}{(a+1)(a-2)}+\dfrac{1}{a-2}=0$
%  \item $\dfrac{x+1}{a-5}+\dfrac{2x-1}{a-2}=\dfrac{2}{a^{2}-7a+10}$
%  \item $\dfrac{x+2}{b-2}+\dfrac{2}{b^{2}-4b+4}+\left(\dfrac{1}{b-2}+\dfrac{x}{b-1}\right)\cdot (b-1)=0$
%  \item $\dfrac{3+b^{3}x}{7b^{2}-b^{3}}+\dfrac{(2b^{2}+b)x+1}{b(b-7)}=\dfrac{3b^{2}x+1}{b^{2}}-2x$
%  \item $\dfrac{x-2}{t^{2}+3t}+\dfrac{x-1}{t+3}=\dfrac{x-2}{t^{2}}+\dfrac{1}{t+3}$
%  \item $\dfrac{x}{2a}+\dfrac{x+1}{1-2a}=\dfrac{1}{a}$
% \end{enumeratea}
% \end{esercizio}
% 
% \begin{esercizio}[\Ast]
% \label{ese:20.47}
% Risolvi e discuti le seguenti equazioni parametriche frazionarie.
% \begin{multicols}{2}
% \begin{enumeratea}
%  \item $\dfrac{t-1}{x-2}=2t$
%  \item $\dfrac{x+m}{x+1}=1$
%  \item $\dfrac{3}{x+1}=2a-1$
%  \item $\dfrac{2a-x}{x-3}-\dfrac{ax+2}{9-3x}=0$
% \end{enumeratea}
% \end{multicols}
% \end{esercizio}
% 
% \begin{esercizio}[\Ast]
% \label{ese:20.48}
% Risolvi e discuti le seguenti equazioni parametriche frazionarie.
% \begin{multicols}{2}
% \begin{enumeratea}
%  \item $\dfrac{k-1}{x}=\dfrac{2}{k+1}$
%  \item $\dfrac{k}{x+1}=\dfrac{2k}{x-1}$
%  \item $\dfrac{a-1}{x+3}-\dfrac{a}{2-x}=\dfrac{ax+3a}{x^{2}+x-6}$
%  \item $\dfrac{a}{x}=\dfrac{1}{a}$
% \end{enumeratea}
% \end{multicols}
% \end{esercizio}
% 
% \begin{esercizio}[\Ast]
% \label{ese:20.49}
% Risolvi e discuti le seguenti equazioni parametriche frazionarie.
% \begin{enumeratea}
%  \item $\dfrac{x-a}{x^{2}-1}-\dfrac{x+3a}{2x-x^{2}-1}=\dfrac{x+5}{x+1}-2\dfrac{x}{(x-1)^{2}}-1$
%  \item $\dfrac{3}{1+3x}+\dfrac{a}{3x-1}=\dfrac{a-5x}{1-9x^{2}}$
%  \item $\dfrac{2a}{x^{2}-x-2}+\dfrac{1}{3x^{2}+2x-1}=\dfrac{6a^{2}-13a-4}{3x^{3}-4x^{2}-5x+2}$
%  \item $\dfrac{a+1}{x+1}-\dfrac{2a}{x-2}=\dfrac{3-5a}{x^{2}-x-2}$
% \end{enumeratea}
% \end{esercizio}
% 
% \begin{esercizio}[\Ast]
% \label{ese:20.50}
% Risolvi e discuti le seguenti equazioni parametriche frazionarie.
% \begin{multicols}{2}
% \begin{enumeratea}
%  \item $\dfrac{a}{x+a}=1+a$
%  \item $\dfrac{x}{x-a}+\dfrac{1}{x+a}=1$
%  \item $\dfrac{x+a}{x-a}=\dfrac{x-a}{x+a}$
%  \item $\dfrac{2}{1-ax}+\dfrac{1}{2+ax}=0$
% \end{enumeratea}
% \end{multicols}
% \end{esercizio}
% 
% \begin{esercizio}[\Ast]
% \label{ese:20.51}
% Risolvi e discuti le seguenti equazioni parametriche frazionarie.
% \begin{multicols}{2}
% \begin{enumeratea}
%  \item $\dfrac{2}{x-2}+\dfrac{a+1}{a-1}=0$
%  \item $\dfrac{1}{x+t}-\dfrac{1}{t+1}=\dfrac{tx}{tx+x+t^{2}+t}$
%  \item $\dfrac{tx}{x-2}+\dfrac{t^{2}}{t+1}-\dfrac{t}{x-2}=0$
%  \item $\dfrac{2x+1}{2x-1}=\dfrac{2a-1}{a+1}$
% \end{enumeratea}
% \end{multicols}
% \end{esercizio}
% 
% \begin{esercizio}
% \label{ese:20.52}
% Risolvi e discuti le seguenti equazioni parametriche frazionarie.
% \begin{multicols}{2}
% \begin{enumeratea}
%  \item $\dfrac{a}{x+1}=\dfrac{3}{x-2}$
%  \item $\dfrac{x}{x+1}+\dfrac{x}{x-1}=\dfrac{bx}{1-x^{2}}+\dfrac{a+2x^{2}}{x^{2}-1}$
%  \item $\dfrac{2x+1}{x}+\dfrac{2x^{2}-3b^{2}}{bx-x^{2}}=\dfrac{1}{x-b}$
%  \item $\dfrac{x-1}{x+a}=2+\dfrac{1-x}{x-a}$
% \end{enumeratea}
% \end{multicols}
% \end{esercizio}
% 
% \paragraph{20.41.}
% a)~$m=-1\vee n=2 \rightarrow \insR; \quad m\neq -1\wedge n\neq~2 \rightarrow \{0\}$
% \protect\\
% b)~$m=0\wedge n\neq~0 \rightarrow \emptyset; \quad m=0\wedge n=0 \rightarrow \insR; \quad m\neq~0 \rightarrow \left\{\frac{m+n}{m}\right\}$
% \protect\\
% c)~$a=-1\vee b=-1 \rightarrow \insR; \quad a\neq -1\wedge b\neq -1 \rightarrow \{0\}$
% \protect\\ d)~$m=n=0 \rightarrow \insR; \quad m=-n\neq~0 \rightarrow \emptyset; \quad m\neq -n \rightarrow \left\{\frac{2m}{m+n}\right\}$
% 
% \paragraph{20.42.}
% a)~$a=b=0 \rightarrow \insR; \quad a=-b\neq~0 \rightarrow \emptyset; \quad a\neq -b \rightarrow \left\{\frac{2(b-a)}{a+b}\right\}$
% \protect\\ b)~$a=2\wedge b=-3 \rightarrow \insR; \quad a=2\wedge b\neq -3 \rightarrow \emptyset; \quad a\neq~2 \rightarrow \left\{\frac{b+3}{a-2}\right\}$
% \protect\\ c)~$a=-1\wedge b=-1 \rightarrow \insR; \quad a=-1\wedge b\neq -1 \rightarrow \emptyset; \quad a\neq -1 \rightarrow \left\{\frac{b+1}{a+1}\right\}$
% \protect\\ d)~$a=b=0 \rightarrow \insR; \quad a=b\neq~0 \rightarrow \emptyset; \quad a\neq b \rightarrow \left\{\frac{2b-a}{a-b}\right\}$
% 
% \paragraph{20.43.}
% a)~$a=-{\frac{2}{3}}\wedge b=0 \rightarrow \insR; \quad a=-{\frac{2}{3}}\wedge b\neq~0 \rightarrow \emptyset; \quad a\neq -{\frac{2}{3}} \rightarrow \left\{\frac{b}{2+3a}\right\}$
% \protect\\ b)~$a=0\wedge b=0 \rightarrow \insR; \quad a=0\wedge b\neq~0 \rightarrow \emptyset; \quad a\neq~0 \rightarrow \left\{-{\frac{b^{2}}{a}}\right\}$
% 
% \paragraph{20.44.}
% a)~$a=0 \rightarrow$ assurdo; $a=-3 \rightarrow \emptyset; \quad a\neq~0\wedge a\neq -3 \rightarrow \left\{\frac{3}{a+3}\right\}$
% \protect\\ b)~$b=0 \rightarrow$ assurdo; $b=6 \rightarrow \emptyset; \quad b\neq~0\wedge b\neq~6 \rightarrow \left\{\frac{1}{6-b}\right\}$
% \protect\\ c)~$a=0 \rightarrow$ assurdo; $a=-2 \rightarrow \emptyset; \quad a\neq~0\wedge a\neq -2 \rightarrow \left\{\frac{7}{2+a}\right\}$
% \protect\\ d)~$a=0\vee a=2 \rightarrow$ assurdo; $a=-3 \rightarrow \emptyset; \quad a\neq~0\wedge a\neq~2\wedge a\neq -3 \rightarrow \left\{\frac{a+2}{a+3}\right\}$
% 
% \paragraph{20.45.}
% a)~$a=1\vee a=3 \rightarrow$ assurdo; $a\neq~1\wedge a\neq~3 \rightarrow \{2(a-1)(a-3)\}$,
% \protect\\ b)~$a=0\vee a=1 \rightarrow$ assurdo; $a=\frac{1}{2} \rightarrow \emptyset; \quad a\neq~0\wedge a\neq \frac{1}{2}\wedge a\neq~1 \rightarrow \left\{\frac{1}{2a-1}\right\}$
% \protect\\ c)~$a=-2 \rightarrow$ assurdo; $a=-3 \rightarrow \emptyset; \quad a=3 \rightarrow \insR; \quad a\neq -3\wedge a\neq -2\wedge a\neq~3 \rightarrow \left\{\frac{a+2}{a+3}\right\}$
% \protect\\ d)~$a=0\vee a=-2\vee a=2 \rightarrow$ assurdo; $a\neq~0\wedge a\neq -2\wedge a\neq~2 \rightarrow \left\{-{\frac{a}{2}}\right\}$
% 
% \paragraph{20.46.}
% a)~$a=2\vee a=-1 \rightarrow$ assurdo; $a\neq~2\wedge a\neq -1 \rightarrow \left\{\frac{a(a+4)}{2-a}\right\}$
% \protect\\ b)~$a=5\vee a=2 \rightarrow$ assurdo; $a=4 \rightarrow \emptyset; \quad a\neq~5\wedge a\neq~2\wedge a\neq~4 \rightarrow \left\{\frac{1}{3(4-a)}\right\}$
% \protect\\ c)~$b=2\vee b=1 \rightarrow$ assurdo; $b\neq~2\wedge b\neq~1 \rightarrow \left\{\frac{b}{2-b}\right\}$
% \protect\\ d)~$b=0\vee b=7\rightarrow$ assurdo; $b\neq~0\wedge b\neq~7 \rightarrow \left\{-{\frac{1}{2b^{2}}}\right\}$
% \protect\\ e)~$t=0\vee t=-3 \rightarrow$ assurdo; $t^{2}=3 \rightarrow \insR; \quad t\neq~0\wedge t\neq -3\wedge t^{2}\neq~3 \rightarrow \{2\}$
% \protect\\ f)~$a=0\vee a=\frac{1}{2} \rightarrow$ assurdo; $a\neq~0\wedge a\neq \frac{1}{2} \rightarrow \{2-6a\}$
% 
% \paragraph{20.47.}
% a)~$t=0\vee t=1 \rightarrow \emptyset; \quad t\neq~0\wedge t\neq~1 \rightarrow \left\{\frac{5t-1}{2t}\right\}$
% \quad b)~$m=1 \rightarrow \insR-\{-1\}; \quad m\neq~1 \rightarrow \emptyset$
% \quad c)~$a=\frac{1}{2} \rightarrow \emptyset; \quad a\neq \frac{1}{2} \rightarrow \left\{-{\frac{2(a-2)}{2a-1}}\right\}$
% \protect\\ d)~$a=3\vee a=\frac{7}{9} \rightarrow \emptyset; \quad a\neq~3\wedge a\neq \frac{7}{9} \rightarrow \left\{\frac{2(3a+1)}{3-a}\right\}$
% 
% \paragraph{20.48.}
% a)~$k=-1 \rightarrow$ assurdo; $k=1 \rightarrow \emptyset; \quad k\neq~1\wedge k\neq -1 \rightarrow \left\{-{\frac{\left(k^2-1\right)}{2}}\right\}$
% \protect\\ b)~$k=0 \rightarrow \insR-\{1,-1\}; \quad k\neq~0 \rightarrow \{-3\}$, c)~$a=1 \rightarrow \insR-\{-3,2\}; \quad a\neq~1 \rightarrow \emptyset$
% \protect\\ d)~$a=0 \rightarrow$ assurdo; $a\neq~0 \rightarrow \left\{a^{2}\right\}$
% 
% \paragraph{20.49.}
% a)~$a=-5\vee a=-1\vee a=7 \rightarrow \emptyset; \quad a\neq -5\wedge a\neq -1\wedge a\neq~7 \rightarrow \left\{\frac{-2(a-1)}{a+5}\right\}$
% \quad b)~$a=-{\frac{4}{3}}\vee a=\frac{5}{9}\vee a=\frac{13}{3} \rightarrow \emptyset; \quad a\neq -{\frac{4}{3}}\wedge a\neq \frac{5}{9}\wedge a\neq \frac{13}{3} \rightarrow \left\{\frac{3-2a}{4+3a}\right\}$
% \protect\\ c)~$a=-{\frac{1}{6}} \rightarrow \insR-\left\{-1,2,\frac{1}{3}\right\}; \quad a=\frac{7}{3}\vee a=4\vee a=1 \rightarrow \emptyset; \quad a\neq -{\frac{1}{6}}\wedge a\neq \frac{7}{3}\wedge a\neq~4\wedge a\neq~1 \rightarrow \{a-2\}$
% \quad d)~$a=1\vee a=-3\vee a=3 \rightarrow \emptyset; \quad a\neq -3\wedge a\neq~1\wedge a\neq~3 \rightarrow \left\{\frac{5-a}{1-a}\right\}$
% 
% \paragraph{20.50.}
% a)~$a=-1\vee a=0 \rightarrow \emptyset; \quad a\neq -1\wedge a\neq~0 \rightarrow \left\{-{\frac{\ a^{2}}{1+a}}\right\}$
% \protect\\ b)~$a=-1\vee a=0 \rightarrow \emptyset; \quad a\neq -1\wedge a\neq~0 \rightarrow \left\{-{\frac{a(a-1)}{a+1}}\right\}$
% \quad c)~$a=0 \rightarrow \insR-\{0\}; \quad a\neq~0 \rightarrow \{0\}$
% \quad d)~$a=0 \rightarrow \emptyset; \quad a\neq~0 \rightarrow \left\{-{\frac{5}{a}}\right\}$
% 
% \paragraph{20.51.}
% a)~$a=1 \rightarrow$ assurdo; $a=-1 \rightarrow \emptyset; \quad a\neq~1\wedge a\neq -1 \rightarrow \left\{\frac{4}{a+1}\right\}$
% \protect\\ b)~$t=-1 \rightarrow$ assurdo; $t\neq -1 \rightarrow \left\{\frac{1}{t+1}\right\}$
% \protect\\ c)~$t=-1 \rightarrow$ assurdo; $t=0 \rightarrow \insR-\{2\}; \quad t=-{\frac{1}{2}} \rightarrow \emptyset; \quad t\neq -\frac{1}{2}\wedge t\neq -1\wedge t\neq~0 \rightarrow \left\{\frac{3t+1}{2t+1}\right\}$
% \quad d)~$a=-1 \rightarrow$ assurdo; $a=2 \rightarrow \emptyset; \quad a\neq -1\wedge a\neq~2 \rightarrow \left\{\frac{3a}{2(a-2)}\right\}$

% \subsubsection*{20.4 - Equazioni letterali e formule inverse}
\subsubsection*{\numnameref{sec:compl1_formuleinverse}}

\begin{esercizio}
\label{ese:20.53}
Interesse~$I$ maturato da un capitale~$C$, al tasso di interesse annuo~$i$, 
per un numero di anni~$t$:
\begin{equation*}
  I=C\cdot i\cdot t
\end{equation*}

Ricava le formule per calcolare:~$C=\ldots\ldots\ldots\ldots$\,, $\quad 
i=\ldots\ldots\ldots\ldots$\,, $\quad t =\ldots\ldots\ldots\ldots$\,.

Se il capitale è~$12.000$ €, il tasso di interesse~$3,5\%$, il tempo è di~$6$ 
anni, calcola~$I$
\end{esercizio}

\begin{esercizio}
\label{ese:20.54}
Conversione da gradi Celsius $C$ a gradi Fahrenheit $F$:
\begin{equation*}
  C=\frac{5(F-32)}{9}
\end{equation*}

Ricava la formula per calcolare\, $F=\ldots\ldots\ldots\ldots$\,.

Calcola il valore di~$C$ quando~$F$ vale~$106$ e il valore di~$F$ 
quando~$C$ vale~$12$
\end{esercizio}

\begin{esercizio}
\label{ese:20.55}
Valore attuale~$V_a$ di una rendita che vale~$V_n$ dopo~$n$ anni, 
anticipata di~$t$ anni al tasso di interesse~$i$:
\begin{equation*}
  V_{a}=V_{n}\cdot (1-i\cdot t)
\end{equation*}

Ricava le formule per calcolare:~$V_n=\ldots\ldots\ldots\ldots$\,, $\quad 
i=\ldots\ldots\ldots\ldots$\,, $\quad t =\ldots\ldots\ldots\ldots$\,.

Se il valore attuale è~$120.000$ €, il tasso di interesse il~$2\%$, 
calcola il valore della rendita dopo~$20$ anni.
\end{esercizio}

\begin{esercizio}
\label{ese:20.56}
Sconto semplice~$S$, per un montante~$M$, al tasso di interesse~$i$, per un 
tempo~$t$ in anni:
\begin{equation*}
  S=\frac{M\cdot i\cdot t}{1+i\cdot t}
\end{equation*}

Ricava le formule per calcolare:~$M=\ldots\ldots\ldots\ldots$\,, $\quad 
i=\ldots\ldots\ldots\ldots$\,.

Se lo sconto semplice è~$12.000$ €, il tempo è~$12$ anni, il tasso di 
interesse il~$4,5\%$, calcola il montante.
\end{esercizio}

\begin{esercizio}
\label{ese:20.57}
Superficie~$S$ di un trapezio di base maggiore~$B$, base minore~$b$, 
altezza~$h$:
\begin{equation*}
  S=\frac{1}{2}\cdot (B+b)\cdot h
\end{equation*}

Ricava le formule per calcolare:~$B=\ldots\ldots\ldots\ldots$\,, $\quad 
b=\ldots\ldots\ldots\ldots$\,, $\quad h =\ldots\ldots\ldots\ldots$\,.

Se la base maggiore è~$12\unit{cm}$, la base minore~$8\unit{cm}$, la 
superficie~$12\unit{cm^2}$, calcola l'altezza del trapezio.
\end{esercizio}

\begin{esercizio}
\label{ese:20.58}
Superficie laterale~$S_l$ di un tronco di piramide con perimetro della base 
maggiore~$2p$, perimetro della base minore~$2p'$, apotema~$a$
(attenzione~$2p$ e~$2p'$ sono da considerare come un'unica incognita):
\begin{equation*}
  S_{l}=\frac{(2p+2p')\cdot a}{2}
\end{equation*}

Ricava le formule per calcolare:~$2p=\ldots\ldots\ldots\ldots$\,, 
$\quad~2p'=\ldots\ldots\ldots\ldots$\,, $\quad a =\ldots\ldots\ldots$\,.

Se la superficie laterale vale~$144\unit{cm^2}$, il perimetro della base 
minore~$12\unit{cm}$ e il perimetro della base maggiore~$14\unit{cm}$, 
calcola l'apotema.
\end{esercizio}

\begin{esercizio}
\label{ese:20.59}
Volume~$V$ del segmento sferico a una base di raggio~$r$ e altezza~$h$
\begin{equation*}
  V=\pi \cdot h^{2}\cdot \left(r-\frac{h}{3}\right)
\end{equation*}

Ricava la formula per calcolare~$r=\ldots\ldots\ldots\ldots$\,.

Se il volume misura~$200\unit{cm^3}$ e l'altezza~$10\unit{cm}$, calcola 
la misura del raggio.
\end{esercizio}

\begin{esercizio}
\label{ese:20.60}
Superficie totale~$S$ del cono di raggio di base~$r$ e apotema~$a$:
\begin{equation*}
  S=\pi \cdot r\cdot (r+a)
\end{equation*}

Ricava la formula per calcolare~$a=\ldots\ldots\ldots\ldots$\,.

Se la superficie totale è~$98\unit{cm^2}$ e il raggio~$6\unit{cm}$, calcola 
la misura dell'apotema.
\end{esercizio}

\begin{esercizio}
\label{ese:20.61}
Velocità~$v$ nel moto rettilineo uniforme con velocità iniziale~$v_0$, 
accelerazione costante~$a$ dopo un tempo~$t$:
\begin{equation*}
  v=v_{0}+a\cdot t
\end{equation*}

Ricava le formule per calcolare:~$v_0=\ldots\ldots\ldots\ldots$\,, $\quad 
a=\ldots\ldots\ldots\ldots$\,, $\quad t =\ldots\ldots\ldots\ldots$\,.

Se un corpo è passato in~$10$ secondi dalla velocità~$10\unit{m/s}$ alla 
velocità~$24\unit{m/s}$ qual è stata la sua accelerazione?
\end{esercizio}

\begin{esercizio}
\label{ese:20.62}
Spazio percorso~$s$ nel moto rettilineo uniformemente accelerato in un 
intervallo di tempo~$t$, per un corpo che ha posizione iniziale
$s_0$, velocità iniziale~$v_0$ e accelerazione~$a$:
\begin{equation*}
  s=s_{0}+v_{0}\cdot t+\dfrac{1}{2}\cdot a\cdot t^{2}
\end{equation*}

Ricava le formule per calcolare:~$v_0=\ldots\ldots\ldots\ldots$\,, $\quad 
a=\ldots\ldots\ldots\ldots$\,.

Se un corpo ha percorso~$100\unit{m}$, partendo dalla posizione iniziale~$0$, 
accelerazione~$3\unit{m/s^2}$, in~$10$ secondi, qual'era la sua velocità 
iniziale?
\end{esercizio}

\begin{esercizio}
\label{ese:20.63}
Formula di Bernoulli relativa al moto di un fluido:
\begin{equation*}
  p+\rho \cdot g\cdot h+\dfrac{1}{2}\rho \cdot v^{2}=k
\end{equation*}

Ricava le formule per calcolare:~$h=\ldots\ldots\ldots\ldots$\,, $\quad 
\rho=\ldots\ldots\ldots\ldots\quad$\,.
\end{esercizio}

\begin{esercizio}
\label{ese:20.64}
Legge di Gay-Lussac per i gas:
\begin{equation*}
  V=V_{0}\cdot (1+\alpha \cdot t)
\end{equation*}

Ricava le formule per calcolare:~$V_0=\ldots\ldots\ldots\ldots$\,, $\quad 
t=\ldots\ldots\ldots\ldots$\,.
\end{esercizio}

\begin{esercizio}
\label{ese:20.65}
Equazione di stato dei gas perfetti:
\begin{equation*}
  pV=nRT
\end{equation*}

Ricava le formule per calcolare:~$V=\ldots\ldots\ldots\ldots$\,, $\quad 
t=\ldots\ldots\ldots\ldots$\,.
\end{esercizio}

\begin{esercizio}
\label{ese:20.66}
Rendimento del ciclo di Carnot:
\begin{equation*}
  \eta =1-\dfrac{T_{1}}{T_{2}}
\end{equation*}

Ricava le formule per calcolare:~$T_1=\ldots\ldots\ldots\ldots$\,, $\quad 
T_2=\ldots\ldots\ldots\ldots$\,.
\end{esercizio}

\begin{esercizio}
\label{ese:20.67}
Legge di Stevino:
\begin{equation*}
  P_{B}=P_{A}+\rho \cdot g\cdot (z_{A}-z_{B})
\end{equation*}

Ricava le formule per calcolare:~$\rho=\ldots\ldots\ldots\ldots$\,, $\quad 
z_A=\ldots\ldots\ldots\ldots$\,, $\quad z_B =\ldots\ldots\ldots\ldots$\,.
\end{esercizio}

\begin{esercizio}
\label{ese:20.68}
Risolvi le seguenti equazioni rispetto alla lettera richiesta.
\begin{multicols}{2}
\TabPositions{2.5cm}
\begin{enumeratea}
 \item $y=\dfrac{2-a}{x}$\tab$x=\ldots,\,a=\ldots$
 \item $y=2-\dfrac{a}{x}$\tab$x=\ldots,\,a=\ldots$
 \item $y=\dfrac{2}{x}-a$\tab$x=\ldots,\,a=\ldots$
 \item $y=-{\dfrac{2-a}{x}}$\tab$x=\ldots,\,a=\ldots$
 \item $\dfrac{2x+1}{2x-1}=\dfrac{2k-1}{k+1}$\tab$k=\ldots$
 \item $(m-1)x=m-3$\tab$m=\ldots$
 \item $\dfrac{2}{x+2}+\dfrac{a-1}{a+1}=0$\tab$a=\ldots$
 \item $(a+1)(b-1)x=0$\tab$b=\ldots$
\end{enumeratea}
\end{multicols}
\end{esercizio}

\begin{esercizio}[\Ast]
\label{ese:20.70}
Risolvi le seguenti equazioni rispetto alla lettera richiesta.
\TabPositions{5cm}
\begin{enumeratea}
 \item $\dfrac{x}{a+b}+\dfrac{x-b}{a-b}=
        \dfrac{b}{a^{2}-b^{2}}$\tab$a=\ldots;\quad x=\ldots$
  \hfill $\left[a=\frac{b(b+1)}{2x-b}; \quad x=\frac{b(a+b+1)}{2a}\right]$
 \item $\dfrac{2x}{a+b}+\dfrac{bx}{a^{2}-b^{2}}-\dfrac{1}{a-b}=0$\tab$a=
        \ldots; \quad b=\ldots$
  \hfill $\left[a=\frac{b(x+1)}{2x-1}; \quad b=\frac{a(2x-1)}{x+1}\right]$
\end{enumeratea}
\end{esercizio}
% 
% %%%%%%%%%%%%%%%%%%%%%%%%%%%%%%%%%%%%%%%%%%%%%%%%%%%%%%%%%%
% % \subsubsection*{22.3 - Sistemi fratti}
% \subsubsection*{\numnameref{sec:compl1_sistemifratti}}
% 
% \begin{esercizio}[\Ast]
%  \label{ese:22.44}
% Verifica l'insieme soluzione dei seguenti sistemi.
% \begin{multicols}{2}
% \begin{enumeratea}
% \item $\longarray\left\{\begin{array}{l}
% \dfrac{4y+x}{5x}=1\\
% \dfrac{x+y}{2x-y}=2\end{array}\right.$
%  \hfill $\left[indeterminato\right]$
% \item $\longarray\left\{\begin{array}{l}
% y=\dfrac{4x-9}{12}\\
% {\dfrac{y+2}{y-1}+\dfrac{1+2x}{1-x}+1=0}\end{array}\right.$ 
%  \hfill $\left[(3;~3)\right]$
% \item $\longarray\left\{\begin{array}{l}
% 2+3\dfrac{y}{x}=\dfrac{1}{x}\\
% 3\dfrac{x}{y}-1=\dfrac{-2}{y} \end{array}\right.$
%  \hfill $\left[(-\frac{5}{11};~\frac{7}{11}\right]$
% \item $\longarray\left\{\begin{array}{l}
% \dfrac{y}{2x-1}=-1\\
% 2\dfrac{x}{y-1}=1\end{array}\right.$
%  \hfill $\left[impossibile\right]$
% \item $\longarray\left\{\begin{array}{l}
% 3\dfrac{x}{y}-\dfrac{7}{y}=1\\
% 2\dfrac{y}{x}+\dfrac{5}{x}=1\end{array}\right.$
%  \hfill $\left[\left(\frac{9}{5};-\frac{8}{5}\right)\right]$
% \item $\longarray\left\{\begin{array}{l}
% 2\dfrac{x}{3y}-\dfrac{1}{3y}=1\\
% \dfrac{3}{y+2x}=-1\end{array}\right.$
%  \hfill $\left[(-1;~-1)\right]$
% \item $\longarray\left\{\begin{array}{l}
% \dfrac{x}{9y}=-{\dfrac{1}{2}}+\dfrac{1}{3y}\\
% 9\dfrac{y}{2x}-1-\dfrac{3}{x}=0\end{array}\right.$
%  \hfill $\left[impossibile\right]$
% \item $\longarray\left\{\begin{array}{l}
% \dfrac{x}{2-\dfrac{y}{2}-2}=1\\
% \dfrac{x-y}{x+\dfrac{3}{2}y-1}=1\end{array}\right.$
%  \hfill $\left[\left(-\frac{1}{5};~\frac{2}{5}\right)\right]$
% \item $\longarray\left\{\begin{array}{l}
% \dfrac{\dfrac{x}{2}+\dfrac{2y}{3}-\dfrac{1}{6}}{x+y-2}=6\\
% x+y=1\end{array}\right.$
%  \hfill $\left[(39;~-38)\right]$
% \item $\longarray\left\{\begin{array}{l}
% \dfrac{x-2y}{4}=\dfrac{\dfrac{x-y}{2}+2x}{4}\\
% \dfrac{x}{\dfrac{y}{3}+1}=1\end{array}\right.$
%  \hfill $\left[\left(\frac{3}{4};~-\frac{3}{4}\right)\right]$
% \item $\longarray\left\{\begin{array}{l}
% \dfrac{x+3y-1}{x-y}=\dfrac{1}{y-x} \\
% x=2y-10 \end{array}\right.$
%  \hfill $\left[(-6;~2)\right]$
% \item $\longarray\left\{\begin{array}{l}
% \dfrac{2}{x-2}-\dfrac{3}{y+3}=1\\
% \dfrac{5}{y+3}=\dfrac{6}{2-x}-4\end{array}\right.$
%  \hfill $\left[(-2;~-5)\right]$
% \end{enumeratea}
% \end{multicols}
% \end{esercizio}
% 
% \begin{esercizio}[\Ast]
%  \label{ese:22.47}
% Verifica l'insieme soluzione dei seguenti sistemi.
% \begin{enumeratea}
% \item $\longarray\left\{\begin{array}{l}
% y-\dfrac{x}{3}+\dfrac{3}{4}=0\\
% \dfrac{2x+1}{1-x}+\dfrac{2+y}{y-1}=-1\end{array}\right.$
%  \hfill $\left[\left(-\frac{9}{8};~-\frac{9}{8}\right)\right]$
% \item $\longarray\left\{\begin{array}{l}
% x+y=2\\
% y\left(\dfrac{x}{y}+3\right)=4\end{array}\right.$
%  \hfill $\left[(1;~1)\right]$
% \item $\longarray\left\{\begin{array}{l}
% \dfrac{x}{3}-\dfrac{y}{2}=0\\
% \dfrac{y(y-x-1)}{y+1}+x-y+1=\dfrac{1}{2}\end{array}\right.$
%  \hfill $\left[impossibile\right]$
% \item $\longarray\left\{\begin{array}{l}
% \dfrac{3x-7y+1}{4x^{2}-9y^{2}}=\dfrac{4}{18y^{2}-8x^{2}}\\
% \dfrac{4(1-3x)^{2}}{2}-y=\dfrac{(12x-5)(6x-y)}{4}+3xy+2\end{array}\right.$
%  \hfill $\left[\left(-\frac{3}{17};~\frac{6}{17}\right)\right]$
% \item $\longarray\left\{\begin{array}{l}
% \dfrac{2x-3y}{x-2y}-\dfrac{3y-1}{x+5y}=
% \dfrac{2(x^{2}+2xy)-(3y-2)^{2}}{x^{2}+3xy-10y^{2}}\\
% x+y=-19\end{array}\right.$
%  \hfill $\left[(-18;~-1)\right]$
% \item $\longarray\left\{\begin{array}{l}
% \dfrac{x-3}{x-3y+1}+\dfrac{xy-y}{x-3y-1}=
% \dfrac{x^{2}-3xy+x^{2}y-3xy^{2}+3y^{2}}{x^{2}+9y^{2}-6xy-1}\\
% \dfrac{x-3}{5y-1}-\dfrac{y-3}{1+5y}=
% \dfrac{x+5y^{2}-5xy+2}{1-25y^{2}}\end{array}\right.$
%  \hfill $\left[\left(\frac{7}{4};~\frac{1}{2}\right)\right]$
% \item $\longarray\left\{\begin{array}{l}
% \dfrac{x-2y}{x^{2}-xy-2y^{2}}-\dfrac{1}{y}=2\\
% \dfrac{4}{y}-\dfrac{5}{x+y}=-9\end{array}\right.$
%  \hfill $\left[(2;~-1)\right]$
% \item $\longarray\left\{\begin{array}{l}
% {2x-y-11=0}\\
% {\dfrac{y+1}{x-1}+\dfrac{3-y}{5x-5}-\dfrac{2}{3}=0}\end{array}\right.$
%  \hfill $\left[\right]$
% \end{enumeratea}
% \end{esercizio}
% 
% \begin{esercizio}
%  \label{ese:22.49}
% Verifica l'insieme soluzione dei seguenti sistemi.
% \begin{multicols}{2}
% \begin{enumeratea}
% \item $\longarray\left\{\begin{array}{l}
% {\dfrac{x+1}{x}=\dfrac{y+2}{y-2}}\\
% {\dfrac{3x-1}{3x-2}=\dfrac{1+y}{y-2}}\end{array}\right.$
% \item $\longarray\left\{\begin{array}{l}
% {\dfrac{2}{5x-y}=\dfrac{-3}{5y-x}}\\
% {\dfrac{1}{4x-3y}=\dfrac{2x+y-1}{3y-4x}}\end{array}\right.$
% \item $\longarray\left\{\begin{array}{l}
% {\dfrac{\sqrt{3}}{x-\sqrt{2}}+\dfrac{2\sqrt{2}}{y-\sqrt{3}}=0}\\
% {\dfrac{1}{x-\sqrt{3}}-\dfrac{\sqrt{6}}{2\left(y+2\sqrt{2}\right)}=0}
% \end{array}\right.$
% \item $\longarray\left\{\begin{array}{l}
% {\dfrac{x-y+1}{x+y-1}=2}\\
% {\dfrac{x+y+1}{x-y-1}=-2}\end{array}\right.$
% \item $\longarray\left\{\begin{array}{l}
% {\dfrac{2}{x-2}=\dfrac{3}{y-3}}\\
% {\dfrac{1}{y+3}=\dfrac{-1}{2-x}}\end{array}\right.$
% \end{enumeratea}
% \end{multicols}
% \end{esercizio}
% 
% %%%%%%%%%%%%%%
% 
% % \subsubsection*{22.4 - Sistemi letterali}
% \subsubsection*{\numnameref{sec:compl1_sistemiletterali}}
% 
% \begin{esercizio}[\Ast]
%  \label{ese:22.50}
% Risolvere e discutere il seguente sistema. 
% Per quali valori di~$a$ la coppia soluzione è formata da numeri reali positivi?
% 
% $\left\{\begin{array}{l}
% {x+ay=2a}\\
% \dfrac{x}{2a}+y=\dfrac{3}{2}
% \end{array}\right.$
%  \hfill $\left[a>0\right]$
% \end{esercizio}
% 
% \begin{esercizio}
%  \label{ese:22.51}
% Perché se il seguente sistema è determinato la coppia soluzione è accettabile?
% 
% $\left\{\begin{array}{l}
% 3x-2y=0\\
% \dfrac{2x-y}{x+1}=\dfrac{1}{a}
% \end{array}\right.$
% \end{esercizio}
% 
% 
% \begin{esercizio}
%  \label{ese:22.52}
% Nel seguente sistema è vero che la coppia soluzione è formata da numeri 
% reali positivi se~$a>2$?
% $\left\{\begin{array}{l}
%  \dfrac{a-x}{a^{2}}+a+\dfrac{y-2a}{a+1}=-1\\
%  2y=x
% \end{array}\right.$
% \end{esercizio}
% 
% 
% \begin{esercizio}
%  \label{ese:22.53}
% Spiegate perché non esiste alcun valore di~$a$ per cui la
% coppia~$(0;2)$ appartenga a~$\IS$ del sistema:
% 
% $\left\{\begin{array}{l}
% 3x-2y=0\\
% \dfrac{2x-y}{x+1}=\dfrac{1}{a}
% \end{array}\right.$
% \end{esercizio}
% 
% \begin{esercizio}[\Ast]
%  \label{ese:22.54}
% Nel seguente sistema determinate i valori da attribuire al
% parametro~$a$ affinché la coppia soluzione accettabile sia formata da
% numeri reali positivi.
% 
% $\left\{\begin{array}{l}
% \dfrac{y}{x}-\dfrac{y-a}{3}=\dfrac{1-y}{3}\\
% a(x+2)+y=1
% \end{array}\right.$
%  \hfill $\left[-\frac{1}{2}<a<\frac{1}{2}\right]$
% \end{esercizio}
% 
% \begin{esercizio}[\Ast]
%  \label{ese:22.55}
% Risolvere i seguenti sistemi.
%  \begin{enumeratea}
% \item $\left\{\begin{array}{l}
% x+ay=2a\\
% \dfrac{x}{2a}+y=\dfrac{3}{2}\end{array}\right.$ 
%  \hfill $\left[a\neq~0\rightarrow (a;1)\right]$
% \item $\left\{\begin{array}{l}
% \dfrac{x^{3}-8}{x-2}=x^{2}-3x+y-2\\
% \dfrac{x^{2}-4xy+3y^{2}}{3y-x}=k\end{array}\right.$
% 
%  \hfill $\left[k~\neq~14 \quad \vee \quad k~\neq~\frac{6}{7} \rightarrow 
%    \left(\frac{k-6}{4}; \frac{5k-6}{4}\right); \quad 
%  k=14 \quad \vee \quad k=\frac{6}{7} \rightarrow \text{impossibile}\right]$
% \item $\left\{\begin{array}{l}
% kx-y=2\\
% x+6ky=0\end{array}\right.$
%  \hfill $\left[\forall k \rightarrow 
%   \left(\frac{12k}{6k^{2}+1};~\frac{2}{6k^{2}+1}\right)\right]$
% \item $\left\{\begin{array}{l}
% kx-8y=4\\
% 2x-4ky=3\end{array}\right.$
%  \hfill $\left[k\neq -2 \vee k\neq~2 \rightarrow 
% \left(\frac{4k-6}{k^{2}-4}; \frac{8-3k}{4(k^{2}-4)}\right);~  
% k=-2 \vee k=2 \rightarrow \text{impossibile}\right]$
%  \end{enumeratea}
% \end{esercizio}
% 
% \begin{esercizio}[\Ast]
%  \label{ese:22.56}
% Risolvere i seguenti sistemi.
%  \begin{enumeratea}
% \item $\left\{\begin{array}{l}
% 4x-k^{2}y=k\\
% kx-4ky=-3k\end{array}\right.$
% 
% \hfill $\left[ \begin{array}{r} 
% k\neq -4 \vee k\neq~4 \vee k\neq~0 \rightarrow 
%    \left(\frac{3k^{2}+4k}{16-k^{2}};~\frac{k+12}{16-k^{2}}\right) \\
% k=-4 \vee k=4 \rightarrow \text{impossibile} \\
% k=0 \rightarrow \text{ indeterminato con soluzioni tipo}~ 
%    (0;t) \wedge t \in \insR 
% \end{array} \right]$
%    
% \item $\left\{\begin{array}{l}
% kx-4ky=-6\\
% kx-k^{2}y=0\end{array}\right.$
%  \hfill $\left[k\neq~0 \vee k\neq~4 \rightarrow 
% \left(\frac{6}{4-k}; \frac{6}{k(4-k)}\right); \quad
% k=0\vee k=4 \rightarrow \text{ impossibile}\right]$
% \item $\left\{\begin{array}{l}
% (k-1)x+(1-k)y=0\\
% (2-2k)x+y=-1\end{array}\right.$
% 
% \hfill $\left[ \begin{array}{r} 
% k\neq~1 \vee k\neq \frac{3}{2} \rightarrow 
%    \left(\frac{1}{2k-3};~\frac{1}{2k-3}\right) \\
% k=\frac{3}{2} \rightarrow \text{ impossibile}; \quad \\
% k=1 \rightarrow \text{ indeterminato con soluzioni tipo }
% (t;-1) \wedge t \in \insR
% \end{array} \right]$
%  \end{enumeratea}
% \end{esercizio}

% % \subsubsection*{22.5 - Sistemi lineari di tre equazioni in tre incognite}
% \subsubsection*{\numnameref{sec:compl1_sistemitreeq}}
% 
% \begin{esercizio}[\Ast]
%  \label{ese:22.57}
%  Determinare la terna di soluzione dei seguenti sistemi.
% \begin{multicols}{2}
% \begin{enumeratea}
% \item $\left\{\begin{array}{l}
% x-2y+z=1 \\
% x-y=2 \\
% x+3y-2z=0 \end{array}\right.$
% \hfill $\left[(0;~-2;~3)\right]$
% \item $\left\{\begin{array}{l}
% x+y+z=4 \\
% x-3y+6z=1\\
% x-y-z=2 \end{array}\right.$
% \hfill $\left[\left(3;~\frac{8}{9};~\frac{1}{9}\right)\right]$
% \item $\left\{\begin{array}{l}
% x+2y-3z=6-3y\\
% 2x-y+4z=x\\
% 3x-z=y+2\end{array}\right.$
% \hfill $\left[(1;~1;~0)\right]$
% \item $\left\{\begin{array}{l}
% 2x-y+3z=1 \\
% x-2y+z=5\\
% x+2z=3 \end{array}\right.$
% \hfill $\left[(-21;~-7;~12)\right]$
% \item $\left\{\begin{array}{l}
% x+2y-z=1 \\
% y-4z=0\\
% x-2y+z=2 \end{array}\right.$
% \hfill $\left[\left(\frac{3}{2};~-\frac{2}{7};~-\frac{1}{14}\right)\right]$
% \item $\left\{\begin{array}{l}
% x-3y+6z=1 \\
% x+y+z=5 \\
% x+2z=3 \end{array}\right.$
% \hfill $\left[(-5;~6;~4)\right]$
% \item $\left\{\begin{array}{l}
% x-4y+6z=2 \\x+4y-z=2\\x+3y-2z=2 
% \end{array}\right.$
% \hfill $\left[(2;~0;~0)\right]$
% \item $\left\{\begin{array}{l}
% 4x-y-2z=1 \\3x+2y-z=4\\x+y+2z=4 
% \end{array}\right.$
% \hfill $\left[(1;~1;~1)\right]$
% \item $\left\{\begin{array}{l}
% x-3y=3 \\x+y+z=-1\\2x-z=0 
% \end{array}\right.$
% \hfill $\left[(0;~-1;~0)\right]$
% \item $\left\{\begin{array}{l}
% 2x-y+3z=1 \\x-6y+8z=2\\3x-4y+8z=2 
% \end{array}\right.$
% \hfill $\left[\left(\frac{2}{3};~-\frac{2}{3};~-\frac{1}{3}\right)\right]$
% \item $\left\{\begin{array}{l}
% 4x-6y-7z=-1 \\x+y-z=1\\3x+2y+6z=1 
% \end{array}\right.$
% \hfill $\left[\left(\frac{9}{31};~\frac{17}{31};~-\frac{5}{31}\right)\right]$
% \item $\left\{\begin{array}{l}
% 4x-3y+z=4\\x+4y-3z=2 \\y-7z=0 
% \end{array}\right.$
% \hfill $\left[\left(\frac{7}{6};~\frac{7}{30};~\frac{1}{30}\right)\right]$
% \item $\left\{\begin{array}{l}
% 3x-6y+2z=1 \\x-4y+6z=5\\x-y+4z=10 
% \end{array}\right.$
% \hfill $\left[(5;~3;~2)\right]$
% \item $\left\{\begin{array}{l}
% 4x-y-7z=-12 \\x+3y+z=-4\\2x-y+6z=5 
% \end{array}\right.$
% \hfill $\left[\left(-{\frac{60}{43}};~-\frac{53}{43};~\frac{47}{43}\right)
%         \right]$
% \item $\left\{\begin{array}{l}
% 2x+y-5z=2 \\x+y-7z=-2\\x+y+2z=1 
% \end{array}\right.$
% \hfill $\left[\left(\frac{10}{3};~-3;~\frac{1}{3}\right)\right]$
% \item $\left\{\begin{array}{l}
% 3x-y+z=-1\\x-y-z=3 \\x+y+2z=1 
% \end{array}\right.$
% \hfill $\left[(6;~11;~-8)\right]$
% \item $\left\{\begin{array}{l}
% x-4y+2z=7 \\-3x-2y+3z=0 \\x-2y+z=1 
% \end{array}\right.$
% \hfill $\left[\left(-5;~-\frac{33}{4};~-\frac{21}{2}\right)\right]$
% \item $\left\{\begin{array}{l}
% -2x-2y+3z=4 \\2x-y+3z=0\\2x+y=1 
% \end{array}\right.$
% \hfill $\left[\left(-{\frac{5}{2}};~6;~\frac{11}{3}\right)\right]$
% \end{enumeratea}
% \end{multicols}
% \end{esercizio}
% 
% \begin{esercizio}
%  \label{ese:22.60}
% Quale condizione deve soddisfare il parametro~$a$ affinché il sistema seguente 
% non sia privo di
% significato? Determina la terna soluzione assegnando ad~$a$ il valore~2.
%  \[\left\{\begin{array}{l}x+y+z=\frac{a^{2}+1}{a}\\ay-z=a^{2} \\
%  y+ax=a+1+a^{2}z\end{array}\right.\]
% \end{esercizio}
% 
% \begin{esercizio}
%  \label{ese:22.61}
% Determina il dominio del sistema e stabilisci se la terna soluzione è 
% accettabile:
% \[\longarray\left\{\begin{array}{l}\frac{5}{1-x}+\frac{3}{y+2}=
% \frac{2x}{xy-2+2x-y}\\
% \frac{x+1-3(y-1)}{\text{xyz}}=\frac{1}{xy}-\frac{2}{yz}-\frac{3}{xz}\\
% x+2y+z=0\end{array}\right.\]
% \end{esercizio}
% 
% \begin{esercizio}
%  \label{ese:22.62}
% Verifica se il sistema è indeterminato:
% \[\left\{\begin{array}{l}x+y=1 \\y-z=5
% \\x+z+2=0 \end{array}\right.\]
% \end{esercizio}
% 
% \begin{esercizio}
%  \label{ese:22.63}
% Determina il volume del parallelepipedo retto avente
% per base un rettangolo, sapendo che le dimensioni della base e
% l'altezza hanno come misura (rispetto al~$\unit{cm}$) i valori
% di~$x, y, z$ ottenuti risolvendo il sistema:
% \[\left\{\begin{array}{l}3x+1=2y+3z \\6x+y+2z=7
% \\9(x-1)+3y+4z=0 \end{array}\right.\]
% \end{esercizio}
% 
% %%%%%%%%%%%%%%%%%%%%%%%%%%%%%%%%%%%%%%%%%%%%

% \subsubsection*{22.6 - Sistemi da risolvere con sostituzioni delle variabili}
\subsubsection*{\numnameref{sec:compl1_sistemisotituzionevariabili}}

\begin{esercizio}[\Ast]
 \label{ese:22.64}
 Risolvi i seguenti sistemi per mezzo di opportune sostituzioni delle variabili.

\begin{enumeratea}
\item $\longarray\left\{\begin{array}{l}
\dfrac{1}{2x}+\dfrac{1}{y}=-4\\\dfrac{2}{3x}+\dfrac{2}{y}=1
\end{array}\right. 
\qquad \text{sostituire} \quad u=\frac{1}{x} \quad v=\frac{1}{y}$
\hfill $\left[\left(-{\frac{1}{27}};~\frac{2}{19}\right)\right]$

% \item $\left\{\begin{array}{l}
% x^{2}+y^{2}=13\\x^{2}-y^{2}=5 
% \end{array}\right. 
% \qquad \text{sostituire} u=x^{2} \quad v=y^{2}$
% \hfill $\left[(3;2);~(-3;2);~(3;-2);~(-3;-2)\right]$

\item $\longarray\left\{\begin{array}{l}
\dfrac{1}{x+y}+\dfrac{2}{x-y}=1\\\dfrac{3}{x+y}-\dfrac{5}{x-y}=2
\end{array}\right. 
\quad \text{sostituire} \quad u=\frac{1}{x+y} \quad v=\ldots$
\hfill $\left[\left(\frac{55}{9};~-\frac{44}{9}\right)\right]$

\end{enumeratea}
\end{esercizio}

\begin{esercizio}[\Ast]
 \label{ese:22.65}
 Risolvi i seguenti sistemi per mezzo di opportune sostituzioni delle variabili.
\begin{multicols}{2}
\begin{enumeratea}
{\longarray
\item $\left\{\begin{array}{l}
\dfrac{5}{2x}-\dfrac{2}{y}=2\\\dfrac{1}{x}+\dfrac{2}{y}=1
\end{array}\right.$
\hfill $\left[\left(\frac{7}{6};~14\right)\right]$
\item $\left\{\begin{array}{l}
\dfrac{1}{x}+\dfrac{2}{y}=3\\\dfrac{1}{x}+\dfrac{3}{y}=4
\end{array}\right.$
\hfill $\left[\left(1;~1\right)\right]$
\item $\left\{\begin{array}{l}
\dfrac{2}{x}+\dfrac{4}{y}=-3\\\dfrac{2}{x}-\dfrac{3}{y}=4 
\end{array}\right.$
\hfill $\left[\left(2;~-1\right)\right]$
\item $\left\{\begin{array}{l}
\dfrac{1}{x+1}-\dfrac{2}{y-1}=2\\\dfrac{2}{x+1}-\dfrac{1}{y-1}=3
\end{array}\right.$
\hfill $\left[\left(-{\frac{1}{4}};~-2\right)\right]$}
\end{enumeratea}
\end{multicols}
\end{esercizio}

\begin{esercizio}[\Ast]
 \label{ese:22.66}
 Risolvi i seguenti sistemi per mezzo di opportune sostituzioni delle variabili.
\begin{multicols}{2}
\begin{enumeratea}
\item $\longarray\left\{\begin{array}{l}
\dfrac{1}{x}-\dfrac{3}{y}+\dfrac{2}{z}=3 \\
\dfrac{2}{x}-\dfrac{3}{y}+\dfrac{2}{z}=4 \\
\dfrac{2}{x}+\dfrac{4}{y}-\dfrac{1}{z}=-3
\end{array}\right.$
\hfill $\left[\left(1;~-\frac{5}{8};~-\frac{5}{7}\right)\right]$
\item $\left\{\begin{array}{l}
x^{3}+y^{3}=9 \\2x^{3}-y^{3}=-6 
\end{array}\right.$
\hfill $\left[(1;~2)\right]$
\item $\left\{\begin{array}{l}
x^{2}+y^{2}=-1\\x^{2}-3y^{2}=12
\end{array}\right.$
\hfill $\left[\emptyset\right]$
\end{enumeratea}
\end{multicols}
\end{esercizio}

% \begin{esercizio}[\Ast]
%  \label{ese:22.66}
% $\longarray\left\{\begin{array}{l}
% \dfrac{4}{x^{2}}-\dfrac{2}{y^{2}}-\dfrac{2}{z^{2}}=0\\
% \dfrac{1}{x^{2}}+\dfrac{1}{z^{2}}=2 \\
% \dfrac{2}{y^{2}}-\dfrac{2}{z^{2}}=0
% \end{array}\right.$
% \hfill $\left[\begin{array}{r}
% (+1;~+1;~+1); \quad (-1;~+1;~+1) \\
% (+1;~-1;~+1); \quad (+1;~+1;~-1) \\
% (-1;~-1;~+1); \quad (-1;~+1;~-1) \\
% (+1;~-1;~-1); \quad (-1;~-1;~-1)
% \end{array}\right]$
% \end{esercizio}
% 
% % \subsubsection*{21.4 - Disequazioni polinomiali di grado superiore al primo}
% \subsubsection*{\numnameref{sec:compl1_disequazionipolinomialiefratte}}
% 
% \begin{esercizio}
%  \label{ese:21.42}
% Risolvi le seguenti disequazioni.
% \begin{multicols}{2}
% \begin{enumeratea}
% \item $(x+3)\cdot \left(\frac{1}{5}x+\frac{3}{2}\right) < 0$
% \item $\left(-{\frac{6}{11}}+2x\right)\cdot\left(-x+\frac{9}{2}\right) \le 0$
% \item $(x-3)\cdot (2x-9)\cdot (4-5x) > 0$
% \item $\left(x+\frac{3}{2}\right)\cdot \left(5x+\frac{1}{5}\right) < 0$
% \item $\left(-{\frac{1}{10}}x+2\right)\cdot \left(-3x+9\right) \ge 0$
% \item $(4x+3)\cdot (-2x-9)\cdot (-4x+5) > 0$
% \end{enumeratea}
% \end{multicols}
% \end{esercizio}
% 
% \begin{esercizio}[\Ast]
%  \label{ese:21.44}
% Trovare l'Insieme Soluzione delle seguenti disequazioni.
% \begin{multicols}{2}
%  \begin{enumeratea}
% \item $(x+2)(3-x)\le~0$ \hfill $\left[x\le -2\vee x\ge~3\right]$
% \item $x(x-2)>0$ \hfill $\left[x<0\vee x>2\right]$
% \item $(3x+2)(2-3x)<0$ \hfill $\left[x<-{\frac{2}{3}}\vee x>\frac{2}{3}\right]$
% \item $(-2x+5)(-3x+7)<0$ \hfill $\left[\right]$
% \end{enumeratea}
% \end{multicols}
% \end{esercizio}
% 
% \begin{esercizio}[\Ast]
%  \label{ese:21.45}
% Trovare l'Insieme Soluzione delle seguenti disequazioni.
%  \begin{enumeratea}
% \item $-3x(2-x)(3-x)\ge~0$ \hfill $\left[x\ge~0\vee~2\le x\le~3\right]$
% \item $(x+1)(1-x)\left(\frac{1}{2}x-2\right)\ge~0$ 
%  \hfill $\left[x\le -1\vee~1\le x\le4\right]$
% \item $(x-1)(x-2)(x-3)(x-4)<0$ \hfill $\left[1<x<2\vee~3<x<4\right]$
% \item $x^{2}-16\le~0$ \hfill $\left[-4\le x\le~4\right]$
% \item $4x^{2}-2x<0$ \hfill $\left[0<x<\frac{1}{2}\right]$
% \end{enumeratea}
% \end{esercizio}
% 
% \begin{esercizio}[\Ast]
%  \label{ese:21.47}
% Trovare l'Insieme Soluzione delle seguenti disequazioni.
% \begin{multicols}{2}
%  \begin{enumeratea}
%  \item $x^{4}-81\ge~0$ \hfill $\left[x\le -3\vee x\ge~3\right]$
% \item $x^{2}+17x+16\le~0$ \hfill $\left[-16\le x\le -1\right]$
% \item $16-x^{4}\le~0$ \hfill $\left[x\le -2\vee x\ge~2\right]$
% \item $x^{2}+2x+1<0$ \hfill $\left[\emptyset\right]$
%  \item $x^{2}+6x+9\ge~0$ \hfill $\left[\insR\right]$
% \item $x^{2}-5x+6<0$ \hfill $\left[2<x<3\right]$
% \item $x^{2}+3x-4\le~0$ \hfill $\left[-4\le x\le~1\right]$
% \item $x^{3}>x^{2}$ \hfill $\left[x>1\right]$
% \end{enumeratea}
% \end{multicols}
% \end{esercizio}
% 
% \begin{esercizio}[\Ast]
%  \label{ese:21.48}
% Trovare l'Insieme Soluzione delle seguenti disequazioni.
%  \begin{enumeratea}
% \item $x^{2}(2x^{2}-x)-(2x^{2}-x)<0$ 
%  \hfill $\left[-1<x<0\vee \frac{1}{2}<x<1\right]$
% \item $x^{2}-2x+1+x(x^{2}-2x+1)<0$ \hfill $\left[x<-1\right]$
% \item $x^{3}-2x^{2}-x+2\ge~0$ \hfill $\left[-1\le x\le~1\vee x\ge~2\right]$
% \item $x^{4}+4x^{3}+3x^{2}>0$ \hfill $\left[x<-3\vee x>-1\wedge x\neq~0\right]$
% \item $(6x^{2}-24x)(x^{2}-6x+9)<0$ \hfill $\left[0<x<4\wedge x\neq~3\right]$
% \item $(x^{3}-8)(x+2)<(2-x)(x^{3}+8)$ \hfill $\left[-2<x<2\right]$
% \item $(2a+1)(a^{4}-2a^{2}+1)<0$ \hfill $\left[a<-{\frac{1}{2}}\wedge a\neq -1\right]$
% \item $x^{3}-6x^{2}+11>1-3x$ \hfill $\left[-1<x<2\vee x>5\right]$
% \item $x^{6}-x^{2}+x^{5}-6x^{4}-x+6<0$ \hfill $\left[-3<x<-1\vee~1<x<2\right]$
% \end{enumeratea}
% \end{esercizio}
% 
% \begin{esercizio}[\Ast]
%  \label{ese:21.50}
%  Determinare i valori che attribuiti alla variabile~$y$ rendono positivi
% entrambi i polinomi
% seguenti:~$p_{1}=y^{4}-13y^{2}+36;\quad p_{2}=y^{3}-y^{2}-4y+4$ 
% \hfill $\left[-2<y<1~\vee~y>3\right]$
% \end{esercizio}
% 
% \begin{esercizio}[\Ast]
%  \label{ese:21.51}
%  Determinare i valori di~$a$ che rendono~$p=a^{2}+1$ minore di~5.
% \hfill $\left[-2<a<2\right]$
% \end{esercizio}
% 
% \begin{esercizio}[\Ast]
%  \label{ese:21.52}
%  Determina~$\IS$ dei seguenti sistemi di disequazioni.
%  \begin{multicols}{3}
%  \begin{enumeratea}
%  \item $\left\{\begin{array}{l}
% x^{2}-9\ge~0\\
% x^{2}-7x+10<0
% \end{array}\right.$
% \hfill $\left[3\le x<5\right]$
% \item $\left\{\begin{array}{l}
% x^{2}+3x-18\ge~0\\
% 12x^{2}+12x+3>0
% \end{array}\right.$
% \hfill $\left[x \le -6~\vee~x\ge~3\right]$
% \item $\left\{\begin{array}{l}
% 16x^{4}-1<0 \\
% 16x^{3}+8x^{2}\ge~0 \end{array}\right.$
% \hfill $\left[-{\frac{1}{2}}<x<\frac{1}{2}\right]$
%  \end{enumeratea}
%  \end{multicols}
% \end{esercizio}
% 
% \begin{esercizio}[\Ast]
%  \label{ese:21.53}
%  Determina~$\IS$ dei seguenti sistemi di disequazioni.
%  \begin{multicols}{2}
%  \begin{enumeratea}
%  \item $\left\{\begin{array}{l}
% 49a^{2}-1\ge~0\\
% 9a^{2}<1\\
% 1-a>0
% \end{array}\right.$
% \hfill $\left[-{\frac{1}{3}}<a\le -{\frac{1}{7}}~\vee~
% \frac{1}{7}\le a<\frac{1}{3}\right]$
% \item $\left\{\begin{array}{l}
% 2x^{2}-13x+6<0\\
% (2x^{2}-5x-3)(1-3x)>0\\
% x^{2}+7>1
% \end{array}\right.$
% \hfill $\left[\frac{1}{2}<x<3\right]$
%  \end{enumeratea}
%  \end{multicols}
% \end{esercizio}
% 
% \begin{esercizio}
% \label{ese:21.54}
% Studia il segno della frazione
% \[f=\dfrac{x^{3}+11x^{2}+35x+25}{x^{2}-25}\]
% \emph{Traccia di svolgimento}. Scomponi in fattori numeratore e denominatore, 
% otterrai
% \[ f=\frac{(x+5)^{2}(x+1)}{(x+5)(x-5)}\]
% Poniamo le~$\CE$ e semplifica la frazione: \dotfill
% 
% Studia separatamente il segno di tutti i fattori che vi compaiono. 
% Verifica che la tabella dei segni sia:
% \begin{center}
% \input{\folder lbr/fig031_seg.pgf}
% \end{center}
% La frazione assegnata, con la~$\CE: x\neq -5\text{ e }x\neq~5$, si annulla 
% se~$x=-1$
% è positiva nell'insieme~$A^{+}=\left\{x\in \insR/-5<x<-1\vee x>5\right\}$, 
% è negativa in
% $A^{-}=\left\{x\in\insR/x<-5\vee -1<x<5\right\}$
% \end{esercizio}
% 
% \begin{esercizio}[\Ast]
% \label{ese:21.55}
% Determinate~$\IS$ delle seguenti disequazioni fratte.
% \begin{enumeratea}
% \spazielenx
% \item $\dfrac{x-2}{3x-9}>0$
% \hfill $\left[x<2~\vee~x>3\right]$
% \item $\dfrac{3x+12}{(x-4)(6-3x)}\geqslant~0$
% \hfill $\left[x\le -4~\vee~2<x<4\right]$
% \item $\dfrac{x+2}{x-1}<2$
% \hfill $\left[x<1~\vee~x>4\right]$
% \item $\dfrac{4-3x}{6-5x}\geqslant -3$
% \hfill $\left[x<\frac{6}{5}~\vee~x\ge\frac{11}{9}\right]$
% % \end{enumeratea}
% % \end{multicols}
% % \end{esercizio}
% % 
% % \begin{esercizio}[\Ast]
% % \label{ese:21.56}
% % Determinate~$\IS$ delle seguenti disequazioni fratte.
% % \begin{multicols}{2}
% % \begin{enumeratea}
% % \spazielenx
%  \item $\dfrac{x+8}{x-2}\ge~0$
% \hfill $\left[x\le -8\vee x>2\right]$
% \item $\dfrac{3x+4}{x^{2}+1}\ge~2$
% \hfill $\left[-{\frac{1}{2}}\le x\le~2\right]$
% \item $\dfrac{4}{x+4}+\dfrac{2}{x-3}\leqslant~0$
% \hfill $\left[x<-4\vee\frac{2}{3}\le x<3\right]$
% \item $\dfrac{7}{x+3}-\dfrac{6}{x+9}\geqslant~0$
% \hfill $\left[-45\le x<-9\vee x>-3\right]$
%  \item $\dfrac{3}{2-x}\leqslant \dfrac{1}{x-4}$
% \hfill $\left[2<x\le \frac{7}{2}\vee x>4\right]$
% \item $\dfrac{2}{4x-16}<\dfrac{2-6x}{x^{2}-8x+16}$
% \hfill $\left[x<\frac{8}{13}\right]$
% \item $\dfrac{x-3}{x^{2}-4x+4}-1<\dfrac{3x-3}{6-3x}$
% \hfill $\left[x<2\vee~2<x<\frac{5}{2}\right]$
% \item $\dfrac{2}{x-2}>\dfrac{2x-2}{(x-2)(x+3)}$
% \hfill $\left[x<-3\vee x>2\right]$
%  \item $\dfrac{5}{2x+6}\geqslant \dfrac{5x+4}{x^{2}+6x+9}$
% \hfill $\left[x\le \frac{7}{5}\wedge x\neq-3\right]$
% \item $\dfrac{x}{x+1}-\dfrac{1}{x^{3}+1}\le~0$
% \hfill $\left[-1<x\le~1\right]$
% \item $\dfrac{(x+3)(10x-5)}{x-2}<0$
% \hfill $\left[x<-3\vee\frac{1}{2}<x<2\right]$
% \item $\dfrac{4-3x}{x-2}<\dfrac{3x+1}{x-2}$
% \hfill $\left[x<\frac{1}{2}\vee x>2\right]$
%  \item $\dfrac{5x-4}{3x-12}\ge \dfrac{x-4}{4-x}$
% \hfill $\left[x\le~2\vee x>4\right]$
% \item $\dfrac{2-x}{5x-15}\le \dfrac{5x-1}{2x-6}$
% \hfill $\left[x\le \frac{1}{3}\vee x>3\right]$
% \item $\dfrac{(3x-12)(6-x)}{(24-8x)(36-18x)}\leqslant~0$
% \hfill $\left[x<2\vee~3<x\le~4\vee x\ge~6\right]$
% \item $\dfrac{(x-2)(5-2x)}{(5x-15)(24-6x)}\geqslant~0$
% \hfill $\left[x\le~2\vee \frac{5}{2}\le x<3\vee x>4\right]$
% \end{enumeratea}
% \end{esercizio}
% 
% \begin{esercizio}[\Ast]
% \label{ese:21.60}
% Determinate~$\IS$ delle seguenti disequazioni fratte.
% \begin{enumeratea}
% \spazielenx
% \item $\dfrac{(x-2)(x+4)(x+1)}{(x-1)(3x-9)(10-2x)}\leqslant~0$
% \hfill $\left[x\le -4\vee -1\le x<1\vee~2\le x<3\vee x>5\right]$
% \item $\dfrac{(5-x)(3x+6)(x+3)}{(4-2x)(x-6)x}\leqslant~0$
% \hfill $\left[-3\le x\le -2\vee~0<x<2\vee~5\le x<6\right]$
% \item $\dfrac{(x-5)(3x-6)(x-3)}{(4-2x)(x+6)x}\leqslant~0$
% \hfill $\left[x<-6\vee~0<x\le3\vee x\ge~5\wedge x \neq~2\right]$
% \item $\dfrac{(x-3)(x+2)(15+5x)}{x^{2}-5x+4}\geqslant~0$
% \hfill $\left[-3\le x\le -2\vee~1<x\le~3\vee x>4\right]$
% \item $\dfrac{\left(x-4\right)^{2}(x+3)}{x^{2}+5x+6}\geqslant~0$
% \hfill $\left[x>-2\right]$
% \item $\dfrac{x}{1-x^{2}}>\dfrac{1}{2x+2}-\dfrac{2}{4x-4}$
% \hfill $\left[x<-1\right]$
% \item $\dfrac{3-x}{x-2}<\dfrac{x-1}{x+3}+\dfrac{2}{x^{2}+x-6}$
% \hfill $\left[x<-3\vee -1<x<2\vee x>\frac{5}{2}\right]$
% \item $\dfrac{2}{x+2}-\dfrac{1}{x+1}\ge \dfrac{3}{2x+2}$
% \hfill $\left[x\le -6\vee -2<x<-1\right]$
%  \item $\dfrac{3}{2x-1}\le \dfrac{2x^{2}}{2x^{2}-x}-\dfrac{x+1}{x}$
% \hfill $\left[x<0\vee\frac{1}{4}\le x<\frac{1}{2}\right]$
% \item $\dfrac{2x^{2}}{2x^{2}-x}>1$
% \hfill $\left[x<\frac{1}{2}\wedge x\neq~0\right]$
% \item $\dfrac{2x}{2x-1}+\dfrac{x+2}{2x+1}>\dfrac{3}{2}$
% \hfill $\left[-\frac{1}{2}<x<\frac{1}{10}\vee x>\frac{1}{2}\right]$
% \item $\dfrac{x^{2}-5x+6}{x^{2}-7x+12}\le~1$
% \hfill $\left[x<4\wedge x\neq~3\right]$
%  \item $\dfrac{\dfrac{2}{x+1}}{x^{2}-1}<0$
% \hfill $\left[x<-1\vee -1<x<1\right]$
% \item $\dfrac{x}{x+1}-\dfrac{4-x}{x+2}\ge \dfrac{2x+1}{x^{2}+3x+2}$
% \hfill $\left[x<-2\vee x\ge \frac{5}{2}\right]$
% \item $\dfrac{3}{2x^{2}-4x-6}-\dfrac{x-2}{3x+3}<\dfrac{x-1}{2x-6}$
% \hfill $\left[x<-1\vee~0<x<2\vee x>3\right]$
% \item $\dfrac{1}{2-2x}\cdot \left(\dfrac{x(x-2)}{x-1}-
%        \dfrac{3}{3-3x}\right)>-1$
% \hfill $\left[\insR-\{1\}\right]$
%  \item $-{\dfrac{2}{27-3x^{2}}}-\dfrac{x+1}{2x-6}+\dfrac{3-2x}{6x-18}<
%         -{\dfrac{3}{x^{2}-9}}+4\dfrac{x-3}{18-2x^{2}}$
% \hfill $\left[x<-3\vee x>3\right]$
% \item $\dfrac{2}{x^{2}-3x+2}-\dfrac{x}{x-2}<
%        \dfrac{x-1}{x-1}-\dfrac{1}{3x-x^{2}-2}+\dfrac{2-x}{4x-4}$
% \hfill $\left[x<0\vee~1<x<\frac{12}{7}\vee x>2\right]$
% \item $\dfrac{(x-2)(x+4)(x^{2}+5x+6)}
%              {(x^{2}-9)(-4-7x^{2})(x^{2}-6x+8)(x^{2}+4)}<0$
% \hfill $\left[x<-4\vee -2<x<3\vee x>4 \wedge x\neq2\right]$
% \end{enumeratea}
% \end{esercizio}
% 
% \begin{esercizio}
% \label{ese:21.65}
% Dopo aver ridotto ai minimi termini la frazione
% $f=\dfrac{3x^{4}-2x^{3}+3x^{2}-2x}{6x^{2}-x-7}$, completa
% 
%  \begin{enumeratea}
%  \item $f>0$ per~$x<-1$ oppure \dotfill
%  \item $f=0$ per \dotfill
%  \item $f<0$ per \dotfill
%  \end{enumeratea}
% \end{esercizio}
% 
% \begin{esercizio}
% \label{ese:21.66}
% Determinate il segno delle frazioni, dopo averle ridotte ai minimi termini.
% \[f_{1}=\dfrac{1-a^{2}}{2+3a};\quad 
% f_{2}=\dfrac{a^{3}-5a^{2}-3+7a}{9-6a+a^{2}};\quad 
% f_{3}=\dfrac{11m-m^{2}+26a}{(39-3m)(m^{2}+4m+4)}\]
% \end{esercizio}
% 
% \begin{esercizio}[\Ast]
% \label{ese:21.67}
% Determinate~$\IS$ delle seguenti disequazioni fratte.
% 
% \begin{enumeratea}{\longarray
% \item $\left\{\begin{array}{l}
% \dfrac{2-x}{3x^{2}+x}\ge~0\\
% x^{2}-x-6\ge~0\\
% x^{2}-4\le~0
% \end{array}\right.$
% \hfill $\left[\left\{x\in\insR/x=-2\right\}\right]$
% \item $\left\{\begin{array}{l}
% \dfrac{x^{2}-4x+4}{9-x^{2}}>0\\
% x^{2}-3x\le~0
% \end{array}\right.$
% \hfill $\left[\left\{x\in \insR/0\le x<3 \wedge x\neq~2\right\}\right]$
% \item $\left\{\begin{array}{l}
% \dfrac{1}{x-2}+\dfrac{3}{x+2}<0\\
% \dfrac{2-x}{5x-15}\le\dfrac{5x-1}{2x-6}
% \end{array}\right.$
% \hfill $\left[x<-2\right]$
% \item $\left\{\begin{array}{l}
% \dfrac{4}{8-4x}-\dfrac{6}{2x-4}<0\\
% \dfrac{x}{x-2}-\dfrac{6}{x^{3}-8}>1
% \end{array}\right.$
% \hfill $\left[x>2\right]$
% \item $\left\{\begin{array}{l}
% \left(1+\dfrac{2}{x-2}\right)\left(1-\dfrac{2}{x-2}\right)<\dfrac{x-4}{2-x}\\
% \left(\dfrac{2-x}{x^{2}-6x+9}+\dfrac{2+x}{x^{2}-9}\right)\cdot{\dfrac{x^{3}-27}{2x}}>0
% \end{array}\right.$
% \hfill $\left[1<x<3\wedge x\neq~2\right]$
%  \item $\left\{\begin{array}{l}
% \left(1-\dfrac{1}{x}\right)+3\left(\dfrac{2}{x}+1\right)>\dfrac{13}{2}\\
% \dfrac{7+x}{2x}>\dfrac{2-x}{1-2x}
% \end{array}\right.$
% \hfill $\left[0<x<\frac{7}{17}\vee\frac{1}{2}<x<2\right]$
% \item $\left\{\begin{array}{l}
% \dfrac{x^{2}-2x-3}{2x^{2}-x-1}\ge~0\\
% \dfrac{4x-1-3x^{2}}{x^{2}-4}\le~0
% \end{array}\right.$
% \hfill $\left[x<-2\vee \frac{1}{3}\le x<1\vee x\ge~3\right]$
% \item $\left\{\begin{array}{l}
% x^{2}-3x+2\le0\\
% \dfrac{6}{2+x}-\dfrac{x+2}{x-2}>\dfrac{x^{2}}{4-x^{2}}
% \end{array}\right.$
% \hfill $\left[1\le x<2\right]$
% \item $\left\{\begin{array}{l}
% x^{2}+1\le -2x\\
% 3x-1<2\left(x-\dfrac{1}{2}\right)
% \end{array}\right.$
% \hfill $\left[\emptyset\right]$}
% \end{enumeratea}
% \end{esercizio}
% 
% \begin{esercizio}
% \label{ese:21.69}
% Motivare la verità o la falsità delle seguenti
% proposizioni riferite alle frazioni.
% \begin{multicols}{3}
% \noindent\[f_{1}=\frac{a^{3}-81a}{81-a^{2}},\]
% \[f_{2}=\frac{7a^{2}+7}{3+3a^{4}+6a^{2}},\]
% \[f_{3}=\frac{20a-50a^{2}-2}{4a-20a^{2}},\]
% \[f_{4}=\frac{a^{4}}{2a^{4}+a^{2}},\]
% \[f_{5}=\frac{1-4a^{2}}{2-8a+8a^{2}},\]
% \[f_{6}=\frac{2a^{2}+a^{3}+a}{2a^{2}-a^{3}-a}.\]
% \end{multicols}
% \begin{enumeratea}
% \TabPositions{11cm}
% \item $f_{1}$ per qualunque valore positivo della variabile è negativa \tab\boxV\quad\boxF
% \item $f_{2}$ è definita per qualunque valore attribuito alla variabile \tab\boxV\quad\boxF
% \item $f_{3}$ è positiva nell'insieme~$\IS=\left\{a\in \insR/a<0\vee a>\frac{1}{5}\right\}$ \tab\boxV\quad\boxF
% \item $f_{4}$ è positiva per qualunque valore reale attribuito alla variabile \tab\boxV\quad\boxF
% \item nell'intervallo~${[}-\frac{1}{2},\frac{1}{2}{[}$, $f_{5}$ non si annulla \tab\boxV\quad\boxF
% \item $f_{6}$ è negativa per qualunque valore dell'insieme~$K=\insR-\{-1,0,1\}$ \tab\boxV\quad\boxF
% \end{enumeratea}
% \end{esercizio}
% 
% 
% %%%%%%%%%%%%%%%%%%%%%%%%%%%%%%%%%%%%%%%%%%%%%%%%%%%%%%%%%%%%%%%%%%%%%%%%%%%


%%%%%%%%%%%%%%%%%%%%%%%%%%%%%%%%%%%%%%%%%%%%%%%%%%%%%%%%%%%%%%%%%%%%%%%%%%%
% 
% \subsection{Risposte}
% \begin{multicols}{2}
% \paragraph{22.8.} a)~$(1;0)$, b)~$(-2;-2)$, c)~$(0;1)$, d)~$(0;1)$
% 
% \paragraph{22.9.} a)~$(4;5)$, b)~indeterminato, c)~$(1;-1)$, d)~$(-4;2)$
% 
% \paragraph{22.10.} a)~indeterminato, b)~$(4;5)$, c)~impossibile, d)~indeterminato.
% 
% \paragraph{22.11.} a)~$(-66;-12)$, b)~$(2;3)$, d)~$(0;0)$
% 
% \paragraph{22.12.} a)~$\left(-{\frac{9}{8}};-\frac{9}{8}\right)$, b)~$\left(\frac{28}{17};\frac{6}{17}\right)$, d)~$(1;-3)$
% 
% \paragraph{22.13.} a)~$\left(-4;-{\frac{3}{2}}\right)$, c)~$\left(\frac{1}{6};\frac{35}{24}\right)$
% 
% \paragraph{22.16.} a)~$(0;0)$, b)~$(2;-1)$, c)~$(2;1)$, \protect\\d)~$(-1;-3)$
% 
% \paragraph{22.17.} a)~impossibile, c)~impossibile, d)~indeterminato.
% 
% \paragraph{22.18.} a)~$\left(\frac{2}{3};0\right)$, b)~$\left(-{\frac{1}{5}};\frac{1}{5}\right)$, c)~$(3;4)$, d)~$(0;1)$
% 
% \paragraph{22.20.} a)~$(0;0)$, b)~$(0;1)$, c)~$\left(\frac{1}{2};0\right)$, d)~$\left(-{\frac{1}{2}};\frac{1}{2}\right)$
% 
% \paragraph{22.21.} a)~$(1;1)$, b)~$(1;1)$, c)~$\left(\frac{35}{12};\frac{19}{12}\right)$, \protect\\d)~$\left(-{\frac{12}{11}};\frac{7}{11}\right)$
% 
% \paragraph{22.22.} a)~$(-11;-31)$, b)~$(1;1)$, c)~$(1;2)$
% 
% \paragraph{22.26.} a)~$(2;0)$, b)~$\left(\frac{13}{12};\frac{5}{12}\right)$, c)~$\left(-{\frac{12}{11}};\frac{7}{11}\right)$, d)~$\left(0;-\frac{1}{2}\right)$
% 
% \paragraph{22.27.} a)~$(21,-12)$, b)~$\left(-{\frac{240}{19}};\frac{350}{19}\right)$, c)~$\left(\frac{34}{37};\frac{16}{37}\right)$, d)~$\left(1;\frac{7}{3}\right)$
% 
% \paragraph{22.28.} a)~$\left(-1;0\right)$, b)~$\left(\frac{1}{2};-1\right)$, c)~$\left(\frac{3}{10};\frac{7}{10}\right)$, d)~$\left(-{\frac{1}{4}};\frac{1}{6}\right)$
% 
% \paragraph{22.29.} a)~impossibile, b)~indeterminato, \protect\\ c)~$(a;2a)$, d)~$(2k;-k)$
% 
% \paragraph{22.39.} a)~rette parallele, sistema impossibile, b)~$(-3;-8)$, c)~rette identiche, indeterminato, d)~$(2;2)$
% 
% \paragraph{22.40.} a)~$\left(-1;0\right)$, b)~$(2;0)$, c)~$\left(-1;0\right)$, d)~rette parallele, impossibile.
% 
% \paragraph{22.41.} a)~$\left(0;-\frac{1}{2}\right)$, b)~$(1;1)$, c)~$\left(\frac{3}{4};\frac{1}{2}\right)$
% 
% \subsection{Risposte}
%  \paragraph{21.10.} a)~$x<\frac{3}{2}$,\quad b)~$x>\frac{3}{2}$,\quad
% c)~$x\le \frac{4}{3}$,\quad d)~$x\ge -{\frac{4}{5}}$,\quad
% e)~$\insR$,\quad f)~$\emptyset $,\quad
% g)~$x<3$,\quad \protect\\ h)~$x\ge -3$
% 
% \paragraph{21.11.} a)~$x\le~1$,\quad b)~$x\le~0$,\quad
% c)~$x\le~5$,\quad d)~$\emptyset $,\quad
% e)~$\insR$,\quad f)~$\insR$,\quad
% g)~$\insR $,\quad h)~$\emptyset $
% 
% \paragraph{21.12.} a)~$\emptyset $,\quad b)~$\insR$,\quad
% c)~$\emptyset $,\quad d)~$x\le -{\frac{10}{3}}$,\quad
% e)~$x<0$,\quad f)~$x\ge~0$,\quad
% g)~$x\le \frac{5}{3}$,\quad h)~$x\le -{\frac{8}{3}}$
% 
% \paragraph{21.13.} a)~$x\ge~0$,\quad b)~$x\le -{\frac{3}{4}}$,\quad
% c)~$x\le~0$,\quad d)~$x\le -{\frac{1}{2}}$,\quad
% e)~$x\ge -{\frac{1}{6}}$,\quad f)~$x\ge -{\frac{27}{2}}$,\quad
% \protect\\g)~$x>-{\frac{27}{5}}$,\quad h)~$\insR$
% 
% \paragraph{21.14.} a)~$x<-{\frac{3}{4}}$,\quad b)~$\insR$,\quad
% c)~$x\ge -{\frac{13}{6}}$,\quad d)~$x>\frac{3}{2}$,\quad
% e)~$x>1$,\quad f)~$x\ge~0$,\quad\protect\\
% g)~$\{x\in\insR/x<1\}=(-\infty,1)$,\quad h)~$x<\frac{13}{2}$
% 
% \paragraph{21.15.} a)~$\insR$,\quad b)~$x>-{\frac{10}{111}}$,\quad
% c)~$\emptyset $,\quad d)~$\insR$
% 
% \paragraph{21.16.} $x>5$
% 
% \paragraph{21.17.} $x\le -2/3$
% 
% \paragraph{21.18.} Massimo~$294\unit{km}$
% 
% \paragraph{21.20.} Meno di~3 minuti
% 
% \paragraph{21.21.} 14
% 
% \paragraph{21.23.} $x>11$
% 
% \paragraph{21.24.} Almeno~9
% 
% \paragraph{21.26.} Più di~$300\unit{km}$
% 
% \paragraph{21.28.} $x>310\unit{cm}$
% 
% \paragraph{21.29.} $\frac{7}{5}\unit{cm}<x<\frac{17}{5}\unit{cm}$
% 
% \paragraph{21.30.} $0^{\circ}<\alpha<45^{\circ}$
% 
% \paragraph{21.31.} $h\le \frac{150}{7}m$
% 
% \paragraph{21.32.} Il lato minore tra~$10\unit{m}$ e~$100\unit{m}$, il lato maggiore tra~$20\unit{m}$ e~$200\unit{m}$
% 
% \end{multicols}
