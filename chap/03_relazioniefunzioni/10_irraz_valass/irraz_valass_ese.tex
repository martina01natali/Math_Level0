% (c) 2015 Daniele Zambelli daniele.zambelli@gmail.com

\section{TODO}

\section{Esercizi}

\subsection{Esercizi dei singoli paragrafi}

% \subsubsection*{\numnameref{sec:01_}}

% \begin{esercizio}
% \label{ese:D.19}
% testo esercizio
% \end{esercizio}

\begin{esercizio}\label{ese:03.1}
Risolvi le seguenti equazioni:
\begin{enumeratea}
\item $9\,{x}^{8}-108\,{x}^{4}+360=0$ 
\hfill [$\not\exists x \in \mathbb{R}$]
\item $9\,{x}^{8}-72\,{x}^{4}+144=324$ 
\hfill [$(x=-\sqrt [4]{10})\vee (x=\sqrt [4]{10})$]
\item $-9=-64\, \left( 5\,x-2 \right) ^{6}$ 
\hfill [$(x={\frac {\sqrt [3]{3}}{10}}+{\frac {2}{5}})\vee (x=-{\frac {\sqrt 
[3]{3}}{10}}+{\frac {2}{5}})$]
\item $729\, \left( x+1 \right) ^{6}+2=0$ 
\hfill [$\not\exists x \in \mathbb{R}$]
\item $243\,{x}^{5}=-9$ 
\hfill [$x=-{\frac {{3}^{{\frac {2}{5}}}}{3}}$]
\item $0= \left( 9\,x-7 \right) ^{5}-9$ 
\hfill [$x={\frac {{3}^{{\frac {2}{5}}}}{9}}+{\frac {7}{9}}$]
\item $6\,{x}^{10}-96\,{x}^{5}+234=0$ 
\hfill [$(x=\sqrt [5]{3})\vee (x=\sqrt [5]{13})$]
\item $-8\,{x}^{8}-392=112\,{x}^{4}-72$ 
\hfill [$\not\exists x \in \mathbb{R}$]
\item $78=-3\,{x}^{10}+6\,{x}^{5}$ 
\hfill [$\not\exists x \in \mathbb{R}$]
\item $-70\,{x}^{4}+175=-7\,{x}^{8}-448$ 
\hfill [$\not\exists x \in \mathbb{R}$]
\item $- \left( 5\,x-9 \right) ^{7}=-9$ 
\hfill [$x={\frac {{3}^{{\frac {2}{7}}}}{5}}+{\frac {9}{5}}$]
\item $-7\,{x}^{2}=-x \left( {x}^{3}-10\,{x}^{2}+70 \right) $ 
\hfill [$(x=0)\vee (x=10)\vee (x=\sqrt {7})\vee (x=-\sqrt {7})$]
\item $-8\,{x}^{10}-32\,{x}^{5}=160$ 
\hfill [$\not\exists x \in \mathbb{R}$]
\item $-7\,{x}^{10}+140\,{x}^{5}-700=0$ 
\hfill [$x=\sqrt [5]{10}$]
\item $-56\,{x}^{4}-679=7\,{x}^{8}$ 
\hfill [$\not\exists x \in \mathbb{R}$]
\item ${x}^{4}-13\,{x}^{3}-13\,x+40=-41\,{x}^{2}$ 
\hfill [$(x=5)\vee (x=8)$]
\item $1024\,{x}^{5}-8=0$ 
\hfill [$x={\frac {{2}^{{\frac {3}{5}}}}{4}}$]
\item $16\,{x}^{5}+36={x}^{10}+64$ 
\hfill [$(x=\sqrt [5]{2})\vee (x=\sqrt [5]{14})$]
\item $x \left( {x}^{3}+6 \right) =-6\,{x}^{3}-9\,{x}^{2}-8$ 
\hfill [$(x=-4)\vee (x=-2)$]
\item ${x}^{4}-72\,x+120=2\,{x}^{2} \left( 6\,x-13 \right) $ 
\hfill [$(x=2)\vee (x=10)$]
\item $0=-6\,{x}^{8}-24\,{x}^{4}+462$ 
\hfill [$(x=-\sqrt [4]{7})\vee (x=\sqrt [4]{7})$]
\item $ \left( x+9 \right) ^{6}-4=0$ 
\hfill [$(x=\sqrt [3]{2}-9)\vee (x=-\sqrt [3]{2}-9)$]
\item $128\,{x}^{3}+800=-8\,{x}^{6}$ 
\hfill [$\not\exists x \in \mathbb{R}$]
\item $5\,{x}^{8}-40=0$ 
\hfill [$(x=-{2}^{{\frac {3}{8}}})\vee (x={2}^{{\frac {3}{8}}})$]
\item $0=2\,{x}^{10}-8\,{x}^{5}+8$ 
\hfill [$x=\sqrt [5]{2}$]
\item $108\,{x}^{4}=9\,{x}^{8}+324$ 
\hfill [$(x=-\sqrt [4]{6})\vee (x=\sqrt [4]{6})$]
\end{enumeratea}
\end{esercizio}

\begin{esercizio}\label{ese:03.1}
Risolvi le seguenti equazioni:
\begin{enumeratea}
\item $-{\frac { \left( x+15 \right) ^{5}}{59049}}= \left( 4\,x-5 \right) ^{5}$ 
\hfill [$x={\frac {30}{37}}$]
\item $-{\frac {175}{27}}=-{\frac {3\,{x}^{10}}{7}}-{\frac 
{7\,{x}^{5}}{5}}-{\frac {343}{300}}$ 
\hfill [$(x={\frac {\sqrt [5]{54810000}}{30}})\vee (x=-{\frac {\sqrt 
[5]{134190000}}{30}})$]
\item ${\frac {{x}^{10}}{5}}=-{\frac {3\,{x}^{5}}{4}}-{\frac {65}{64}}$ 
\hfill [$\not\exists x \in \mathbb{R}$]
\item ${x}^{3}-{\frac {5}{28}}={\frac {7\,{x}^{6}}{5}}$ 
\hfill [$x={\frac {\sqrt [3]{980}}{14}}$]
\item $-{\frac {1}{2}}=-{\frac {{x}^{10}}{2}}-{\frac {{x}^{5}}{7}}-{\frac 
{1}{98}}$ 
\hfill [$(x={\frac {\sqrt [5]{14406}}{7}})\vee (x=-{\frac {\sqrt 
[5]{19208}}{7}})$]
\item $-{\frac { \left( 25\,x-21 \right) ^{5}}{52521875}}={\frac {1024\, \left( 
6\,x+1 \right) ^{5}}{243}}$ 
\hfill [$x=-{\frac {77}{915}}$]
\item ${\frac {7\,{x}^{8}}{10}}=-{\frac {{x}^{4}}{9}}+{\frac {1615}{1134}}$ 
\hfill [$(x=-{\frac {{7}^{{\frac {3}{4}}}\sqrt {3}\sqrt [4]{85}}{21}})\vee 
(x={\frac {{7}^{{\frac {3}{4}}}\sqrt {3}\sqrt [4]{85}}{21}})$]
\item ${\frac { \left( 9\,x-32 \right) ^{4}}{20736}}=-2$ 
\hfill [$\not\exists x \in \mathbb{R}$]
\item ${x}^{4}+{\frac {62\,{x}^{2}}{7}}-{\frac {88\,x}{7}}-{\frac 
{80}{7}}=-11\,{x}^{3}$ 
\hfill [$(x=-10)\vee (x=-1)\vee (x={\frac {2 \sqrt {14}}{7}})\vee (x=-{\frac {2 
\sqrt {14}}{7}})$]
\item ${\frac {x \left( 4\,{x}^{3}+31\,x+63 \right) }{4}}=7\,{x}^{3}+{\frac 
{45}{2}}$ 
\hfill [$(x=5)\vee (x=-{\frac {3}{2}})\vee (x=2)\vee (x={\frac {3}{2}})$]
\item $-{\frac {4}{5}}={\frac {9\,{x}^{6}}{5}}$ 
\hfill [$\not\exists x \in \mathbb{R}$]
\item $0={\frac {{x}^{10}}{2}}-{\frac {{x}^{5}}{2}}+{\frac {401}{8}}$ 
\hfill [$\not\exists x \in \mathbb{R}$]
\item $0={\frac { \left( 35\,x-18 \right) ^{5}}{24300000}}$ 
\hfill [$x={\frac {18}{35}}$]
\item ${x}^{4}-{\frac {601\,{x}^{2}}{10}}-{\frac {2\,x}{5}}+6=-4\,{x}^{3}$ 
\hfill [$(x=-10)\vee (x=6)\vee (x={\frac {\sqrt {10}}{10}})\vee (x=-{\frac 
{\sqrt {10}}{10}})$]
\item $-90=-{\frac {x \left( 3\,{x}^{3}+45\,{x}^{2}+157\,x-75 \right) }{3}}$ 
\hfill [$(x=-9)\vee (x=-6)\vee (x={\frac {\sqrt {15}}{3}})\vee (x=-{\frac {\sqrt 
{15}}{3}})$]
\item $0={\frac { \left( 35\,x-16 \right) ^{7}}{163840000000}}-{\frac {1}{2}}$ 
\hfill [$x={\frac {4 {2}^{6/7}}{7}}+{\frac {16}{35}}$]
\item ${\frac {2}{9}}=-{\frac { \left( 9\,x-56 \right) ^{8}}{110075314176}}$ 
\hfill [$\not\exists x \in \mathbb{R}$]
\item $-{\frac {512\, \left( 25\,x-14 \right) ^{9}}{78815638671875}}+4=0$ 
\hfill [$x={\frac {7 {2}^{2/9}}{10}}+{\frac {14}{25}}$]
\item ${\frac {5\,{x}^{8}}{7}}+{\frac {3\,{x}^{4}}{2}}=-{\frac {91}{80}}$ 
\hfill [$\not\exists x \in \mathbb{R}$]
\item ${\frac { \left( 81\,x+80 \right) ^{7}}{10030613004288}}=- \left( x-1 
\right) ^{7}$ 
\hfill [$x=-{\frac {8}{153}}$]
\item $0=-{\frac { \left( 9\,x-5 \right) ^{4}}{6561}}$ 
\hfill [$x={\frac {5}{9}}$]
\item ${\frac { \left( 9\,x-10 \right) ^{3}}{3375}}=- \left( 9\,x+7 \right) 
^{3}$ 
\hfill [$x=-{\frac {95}{144}}$]
\item ${\frac {3\,{x}^{3}}{7}}={\frac {4\,{x}^{6}}{5}}+{\frac {24545}{784}}$ 
\hfill [$\not\exists x \in \mathbb{R}$]
\item ${\frac {1}{486}}=-6\,{x}^{10}-{\frac {2\,{x}^{5}}{9}}$ 
\hfill [$x=-{\frac {{12}^{{\frac {2}{5}}}}{6}}$]
\item $-{\frac { \left( 3\,x-2 \right)  \left( 4\,{x}^{2}-x-2 \right) 
}{3}}=-{x}^{4}$ 
\hfill [$(x={\frac {\sqrt {3}}{3}})\vee (x=-{\frac {\sqrt {3}}{3}})\vee (x=2)$]
\item ${\frac {4\,{x}^{3}}{3}}-{\frac {16}{81}}={\frac {9\,{x}^{6}}{4}}-{\frac 
{100}{9}}$ 
\hfill [$(x={\frac {\sqrt [3]{68}}{3}})\vee (x=-{\frac {\sqrt [3]{52}}{3}})$]
\end{enumeratea}
\end{esercizio}

\begin{esercizio}\label{ese:03.1}
Risolvi le seguenti disequazioni:
\begin{enumeratea}
\item $1953125\, \left( x-1 \right) ^{9}+512\, \left( 2\,x-1 \right) ^{9}>0$ 
\hfill [${\frac {7}{9}}<x$]
\item $-168\geq 3\,{x}^{6}+54\,{x}^{3}$ 
\hfill [$x\leq -{2}^{{\frac {2}{3}}} \wedge -\sqrt [3]{14}\leq x$]
\item $4\,{x}^{6}-48\,{x}^{3}+144\leq 324$ 
\hfill [$x\leq \sqrt [3]{15} \wedge -\sqrt [3]{3}\leq x$]
\item $-8\, \left( 3\,x-1 \right) ^{3}> \left( 7\,x+4 \right) ^{3}$ 
\hfill [$x<-{\frac {2}{13}}$]
\item $0\geq  \left( 10\,x-7 \right) ^{5}-1$ 
\hfill [$x\leq {\frac {4}{5}}$]
\item $-{x}^{3}\geq -1$ 
\hfill [$x\leq 1$]
\item $-5\,{x}^{9}+40\geq 0$ 
\hfill [$x\leq \sqrt [3]{2}$]
\item $9\,{x}^{10}+900\geq 180\,{x}^{5}$ 
\hfill [$\forall x \in \mathbb{R}$]
\item ${x}^{3} \left( x+8 \right) >-15\,{x}^{2}$ 
\hfill [$(x<-5)\vee ((x<0) \wedge (-3<x))\vee (0<x)$]
\item $3\,{x}^{8}-147<-6\,{x}^{4}-3$ 
\hfill [$x<\sqrt [4]{6} \wedge -\sqrt [4]{6}<x$]
\item $14\,{x}^{3}<-{x}^{4}-46\,{x}^{2}+28\,x+96$ 
\hfill [$((x<-6) \wedge (-8<x))\vee ((x<\sqrt {2}) \wedge (-\sqrt {2}<x))$]
\item $-10\,{x}^{6}\geq 90$ 
\hfill [$\not\exists x \in \mathbb{R}$]
\item $126\,{x}^{3}+133<7\,{x}^{6}$ 
\hfill [$(x<-1)\vee (\sqrt [3]{19}<x)$]
\item $-64\, \left( 2\,x+1 \right) ^{3}> \left( 2\,x-3 \right) ^{3}$ 
\hfill [$x<-{\frac {1}{10}}$]
\item $162>-2\,{x}^{8}+36\,{x}^{4}$ 
\hfill [$(x<-\sqrt {3})\vee ((x<\sqrt {3}) \wedge (-\sqrt {3}<x))\vee (\sqrt 
{3}<x)$]
\item ${x}^{10}+81>-18\,{x}^{5}$ 
\hfill [$\not\exists x \in \mathbb{R}$]
\item $ \left( 8\,x+7 \right) ^{4}+6\leq 0$ 
\hfill [$\not\exists x \in \mathbb{R}$]
\item $-1000000000\, \left( x+1 \right) ^{9}+ \left( 9\,x-8 \right) ^{9}>0$ 
\hfill [$x<-18$]
\item $-390625\, \left( 2\,x-1 \right) ^{8}\geq - \left( 2\,x-5 \right) ^{8}$ 
\hfill [$x\leq {\frac {5}{6}} \wedge 0\leq x$]
\item $-8\,{x}^{10}<16\,{x}^{5}+8$ 
\hfill [$(x<-1)\vee (-1<x)$]
\item $- \left( x-9 \right) ^{7}<-128\, \left( 4\,x-5 \right) ^{7}$ 
\hfill [$x<{\frac {1}{7}}$]
\item $-42\,{x}^{5}+700>-7\,{x}^{10}-63$ 
\hfill [$\forall x \in \mathbb{R}$]
\item $6\,{x}^{4}<30$ 
\hfill [$x<\sqrt [4]{5} \wedge -\sqrt [4]{5}<x$]
\item $-60\,{x}^{4}+490>-10\,{x}^{8}-90$ 
\hfill [$\forall x \in \mathbb{R}$]
\item $- \left( 8\,x+5 \right) ^{7}\geq -3$ 
\hfill [$x\leq -{\frac {5}{8}}+{\frac {\sqrt [7]{3}}{8}}$]
\item $-7\,{x}^{10}-28\geq -28\,{x}^{5}$ 
\hfill [$x=\sqrt [5]{2}$]
\end{enumeratea}
\end{esercizio}

\begin{esercizio}\label{ese:03.1}
Risolvi le seguenti disequazioni:
\begin{enumeratea}
\item $0\geq -{\frac {7\,{x}^{8}}{5}}$ 
\hfill [$\forall x \in \mathbb{R}$]
\item ${\frac {5\,{x}^{3}}{3}}+{\frac {25}{36}}\geq -{x}^{6}$ 
\hfill [$\forall x \in \mathbb{R}$]
\item ${\frac {x \left( 7\,{x}^{3}+70\,{x}^{2}+60\,x-30 \right) }{7}}<{\frac 
{27}{7}}$ 
\hfill [$((x<-1) \wedge (-9<x))\vee ((x<{\frac {\sqrt {21}}{7}}) \wedge (-{\frac 
{\sqrt {21}}{7}}<x))$]
\item ${\frac {3}{14}}\leq -{\frac {2\,{x}^{8}}{7}}$ 
\hfill [$\not\exists x \in \mathbb{R}$]
\item $ \left( 3\,x-2 \right) ^{4}-{\frac { \left( 3\,x+10 \right) 
^{4}}{50625}}\geq 0$ 
\hfill [$(x\leq {\frac {5}{12}})\vee ({\frac {20}{21}}\leq x)$]
\item ${x}^{2} \left( {x}^{2}+83 \right) \geq 19\,{x}^{3}-133\,x+630$ 
\hfill [$(x\leq -\sqrt {7})\vee ((x\leq 9) \wedge (\sqrt {7}\leq x))\vee (10\leq 
x)$]
\item $729\,{x}^{6}+3\leq 0$ 
\hfill [$\not\exists x \in \mathbb{R}$]
\item $-3\,{x}^{2} \left( 2\,x+5 \right) <-{x}^{4}-48\,x-56$ 
\hfill [$((x<-1) \wedge (-2 \sqrt {2}<x))\vee ((x<7) \wedge (2 \sqrt {2}<x))$]
\item ${\frac {8}{9}}\geq - \left( 5\,x+3 \right) ^{9}$ 
\hfill [$-{\frac {\sqrt [3]{2}{3}^{{\frac {7}{9}}}}{15}}-{\frac {3}{5}}\leq x$]
\item $-55\,{x}^{2}>-{x}^{4}-3\,{x}^{3}+3\,x-54$ 
\hfill [$(x<-9)\vee ((x<1) \wedge (-1<x))\vee (6<x)$]
\item $-{\frac {7\,{x}^{4}}{9}}-{\frac {7}{12}}>0$ 
\hfill [$\not\exists x \in \mathbb{R}$]
\item $-{\frac {10}{3}}\geq {\frac {5\,{x}^{6}}{4}}$ 
\hfill [$\not\exists x \in \mathbb{R}$]
\item $0\leq -{\frac { \left( 5\,x+42 \right) ^{8}}{656100000000}}-{\frac 
{1}{8}}$ 
\hfill [$\not\exists x \in \mathbb{R}$]
\item ${\frac {2}{7}}\geq -{x}^{6}$ 
\hfill [$\forall x \in \mathbb{R}$]
\item $-6\,{x}^{3}-47\,{x}^{2}+42\,x+280\geq -{x}^{4}$ 
\hfill [$(x\leq -4)\vee ((x\leq \sqrt {7}) \wedge (-\sqrt {7}\leq x))\vee 
(10\leq x)$]
\item $0>-{x}^{10}-1$ 
\hfill [$\forall x \in \mathbb{R}$]
\item ${\frac {5\,{x}^{6}}{3}}\geq -{\frac {9\,{x}^{3}}{8}}-{\frac 
{1971}{1280}}$ 
\hfill [$\forall x \in \mathbb{R}$]
\item $1\leq {\frac { \left( 21\,x-1 \right) ^{3}}{343}}$ 
\hfill [${\frac {8}{21}}\leq x$]
\item $0\geq -{\frac { \left( 35\,x-36 \right) ^{6}}{5489031744}}-{\frac { 
\left( 50\,x-27 \right) ^{6}}{8303765625}}$ 
\hfill [$\forall x \in \mathbb{R}$]
\item ${\frac {5}{98}}\leq {\frac {5\,{x}^{6}}{2}}-{\frac 
{4\,{x}^{3}}{5}}+{\frac {8}{125}}$ 
\hfill [$(x\leq {\frac {\sqrt [3]{735}}{35}})\vee ({\frac {\sqrt 
[3]{12985}}{35}}\leq x)$]
\item $9\,{x}^{8}\leq -{\frac {4\,{x}^{4}}{3}}-{\frac {241}{324}}$ 
\hfill [$\not\exists x \in \mathbb{R}$]
\item ${\frac {21}{50}}\geq -{\frac {7\,{x}^{9}}{10}}$ 
\hfill [$-{\frac {\sqrt [9]{3}{5}^{{\frac {8}{9}}}}{5}}\leq x$]
\item ${x}^{8}-1\leq -{\frac {6\,{x}^{4}}{7}}-{\frac {9}{49}}$ 
\hfill [$x\leq {\frac {{7}^{{\frac {3}{4}}}\sqrt {2}}{7}} \wedge -{\frac 
{{7}^{{\frac {3}{4}}}\sqrt {2}}{7}}\leq x$]
\item $-{\frac {72}{49}}\geq {\frac {2\,{x}^{6}}{9}}$ 
\hfill [$\not\exists x \in \mathbb{R}$]
\item ${\frac {4}{5}}\geq -{\frac { \left( x+3 \right) ^{9}}{19683}}$ 
\hfill [$-{\frac {3 {2}^{2/9}{5}^{{\frac {8}{9}}}}{5}}-3\leq x$]
\item ${\frac { \left( 8\,{x}^{2}-5 \right)  \left( {x}^{2}+2 \right) }{8}}\leq 
-{\frac {3\,x \left( 8\,{x}^{2}-5 \right) }{8}}$ 
\hfill [$((x\leq -1) \wedge (-2\leq x))\vee ((x\leq {\frac {\sqrt {10}}{4}}) 
\wedge (-{\frac {\sqrt {10}}{4}}\leq x))$]
\end{enumeratea}
\end{esercizio}




\subsection{Esercizi}

Calcola il valore dei seguenti valori assoluti:
\begin{enumerate}
\item $|-2|=\dots$
\item $|+7|=\dots$
\item $|3-7|=\dots$
\item $|5-\frac{1}{3}|=\dots$
\item $|\sqrt{2}-\sqrt{6}|=\dots$
\item $|\sqrt{3}-2|=\dots$
\end{enumerate}

Completa come nell'esempio:
$$
|2x|=
        \left\lbrace 
        \begin{array}{lcl}
        2x & \text{se}& x\geq 0\\
        -2x & \text{se}& x< 0\\
        \end{array}
        \right.
$$

\begin{enumerate}
        \item 
        $$
        |4x|=
        \left\lbrace 
        \begin{array}{lcl}
        \dots & \text{se}& x\geq\dots\\
        \dots & \text{se}& x<\dots\\
        \end{array}
        \right.
        $$
        \item 
        $$
        |-2x|=
        \left\lbrace 
        \begin{array}{lcl}
        \dots & \text{se}& x\geq\dots\\
        \dots & \text{se}& x<\dots\\
        \end{array}
        \right.
        $$
        \item 
        $$
        |x-3|=
        \left\lbrace 
        \begin{array}{lcl}
        x-3 & \text{se}& x\geq\dots\\
        \dots & \text{se}& x<\dots\\
        \end{array}
        \right.
        $$
        \item 
        $$
        |x^2-2x|=
        \left\lbrace 
        \begin{array}{lcl}
        \dots & \text{se}& x\leq\dots \vee x\geq\dots\\
        \dots & \text{se}& \dots <x<\dots\\
        \end{array}
        \right.
        $$
        \item 
        $$
        |7-2x|=
        \left\lbrace 
        \begin{array}{lcl}
        \dots & \text{se}& x\geq\dots\\
        \dots & \text{se}& x<\dots\\
        \end{array}
        \right.
        $$
        \item 
        $$
        |x^2-6x+8|=
        \left\lbrace 
        \begin{array}{lcl}
        \dots & \text{se}& x\leq\dots \vee x\geq\dots\\
        \dots & \text{se}& \dots <x<\dots\\
        \end{array}
        \right.
        $$
        \item 
        $$
        |\frac{x-1}{2x}|=
        \left\lbrace 
        \begin{array}{lcl}
        \frac{x-1}{2x} & \text{se}& x\leq\dots \vee x\geq\dots\\
        \dots & \text{se}& \dots <x<\dots\\
        \end{array}
        \right.
        $$
\end{enumerate}
Interpretazione di grafici:
\begin{enumerate}
\item Quale delle seguenti figure rappresenta il grafico della funzione 
$y=|x-4|$?

\begin{figure}[h]
\begin{inaccessibleblock}[TODO]
\centering
\includegraphics[width=0.9\linewidth]{img/imm6} %[scale=0.35]{img/fig001.png}
\end{inaccessibleblock}
% \caption{Retta}
\label{fig:abs_imm6}
\end{figure}
% \begin{figure}[h]
%         \centering
%         \includegraphics[width=0.9\linewidth]{imm6}
%         %\caption{}
%         \label{fig:imm1}
% \end{figure}
\end{enumerate}
Traccia il grafico delle seguenti funzioni come nell'esempio:
$$y=|x^2-4|$$
\begin{tabular}{cc}
\begin{tabular}{|c|c|}
        \hline
        x & y \\
        \hline
        0 & 4 \\
        \hline  
        -1 & 3 \\
        \hline
        1 & 3 \\
        \hline
        -2 & 0 \\
        \hline
        2 & 0 \\
        \hline
        -3 & 5 \\
        \hline
        3 & 5 \\
        \hline                                                  
\end{tabular}   
&

\begin{inaccessibleblock}[TODO]
\centering
\includegraphics[width=0.5\linewidth]{img/imm7} %[scale=0.35]{img/fig001.png}
\end{inaccessibleblock}
%       \includegraphics[width=0.5\linewidth]{imm7}
\\
\end{tabular}

\begin{enumerate}
        \item $y=|2x-1|$
        \item $y=|x+2|$
        \item $y=|x-1|$
        \item $y=2|x-1|$
        \item $y=\frac{|x-2|}{2}$
        \item $y=|3x|$
        \item $y=|x^2-3x+2|$
        \item $y=|x^2-1|$
\end{enumerate}
Equazioni del tipo $|P(x)|=k$\\
Esempi:
\begin{enumerate}
\item[a)] $|x^2-3|=0$, ricordando che $|x^2-3|=0$ se e solo se $x^2-3=0$, 
l'equazione ha come soluzioni $x=\pm\sqrt{3}$.
\item[b)] $|x^2-3|=-2$, impossibile perché il valore assoluto di un'espressione 
algebrica è sempre un numero non negativo.
\item[c)] $|2x-3|=2$, l'equazione equivale a:
$$2x-3=2 \vee 2x-3=-2$$
e quindi
$$x=\frac{5}{2}\vee x=\frac{1}{2}$$
\end{enumerate}
Risolvi le seguenti equazioni:

\begin{enumerate}
        \item $|x-3|=2$ \hfill $\left[ 1, 5\right] $
        \item $|x+1|=3$ \hfill $\left[ -4, 2\right] $
        \item $|x^2-6x+8|=0$ \hfill $\left[ 2, 4\right] $
        \item $\left| \frac{x-1}{2x}\right| =\frac{1}{4}$ \hfill $\left[ 
\frac{2}{3}, 2\right] $
                \item $|x^2-9|=-3$ \hfill $\left[impossibile \right] $
                \item $|x^4-x^2|=0$ \hfill $\left[ 0, \pm 1\right] $
                \item $|4x+3|=2$ \hfill $\left[ -\frac{3}{2}, -\frac{1}{4} 
\right] $
                \item $|x^2-6x+4|=4$ \hfill $\left[ 0, 2, 4 , 6 \right] $
                \item $|x^2-2x|=1$ \hfill $\left[ 1, 1+\sqrt{2} \right] $
                \item $|2x^3+6x-5|=-2$ \hfill $\left[ impossibile \right] $
                \item $\left| \frac{x^2-3x}{x+2}\right| =1$ \hfill $\left[ 2\pm 
\sqrt{6} \right] $
                \item $|x+3|=2$ \hfill $\left[ -1, -5 \right] $
\end{enumerate}



Equazioni del tipo $|A(x)|=|B(x)|$\\
Esempi:
\begin{enumerate}
        \item[a)] $|x^2-4|=|x-2|$, l'equazione equivale a
        $$x^2-4=x-2 \vee x^2-4=-(x-2)$$
        cioè:
        $$x^2-x-2=0 \vee x^2+x+6=0$$la prima equazione ha soluzioni $[-1, 2]$, 
la seconda $[-3, 2]$, pertanto le soluzioni dell'equazione di partenza sono 
$S=\left\lbrace -3, -1, 2\right\rbrace $.
        
        \item[b)] $|x^2-3|=-2$, impossibile perché il valore assoluto di 
un'espressione algebrica è sempre un numero non negativo.
        \item[c)] $|2x-3|=2$, l'equazione equivale a:
        $$2x-3=2 \vee 2x-3=-2$$
        e quindi
        $$x=\frac{5}{2}\vee x=\frac{1}{2}$$
\end{enumerate}

Risolvi le seguenti equazioni:

\begin{enumerate}
\item $\left| x-1\right| =\left| 2x-3\right| $ \hfill $\left[ \frac{4}{3}, 
2\right] $
\item $\left| x+1\right| =\left| 2x-1\right| $ \hfill $\left[ 0, 2\right] $
\item $\left| x^2-2x+3 \right| =\left| x-4 \right| $ \hfill $\left[ 
\frac{1\pm\sqrt{5}}{2}\right] $
\item $\left| x^2-x-5\right| =\left| x-2\right| $ \hfill $\left[ -1, 3, \pm 
\sqrt{7}\right] $
\item $\left| x+3\right| =\left| x\right| $ \hfill $\left[ -\frac{3}{2} \right] 
$
\item $\left| x^2-5x \right| =\left| x^2+2x \right| $ \hfill $\left[ 0, 
\frac{3}{2} \right] $
\item $\left| 3x+5\right| =\left| 2x+3\right| $ \hfill $\left[-2, -\frac{8}{5} 
\right] $
\item $\left| x+1\right| =\left| 2x\right| $ \hfill $\left[ -\frac{1}{3}, 1 
\right] $
\item $\left| x^3-6x\right| =\left| x^3-2x\right| $ \hfill $\left[ 0, \pm 2 
\right] $
\item $\left|\frac{x^2+1}{x}\right| =\left| 2x\right| $ \hfill $\left[ \pm 1 
\right] $
\item $\left| x^2-3x-10\right| =\left| x^2-4\right| $ \hfill $\left[-2, 
\frac{7}{2} \right] $
\end{enumerate}

Equazioni del tipo $|A(x)|=B(x)$\\
Esempi:
\begin{enumerate}
        \item[a)] $|x-3|=2x+2$, l'equazione equivale a risolvere:
        $$
        \left\lbrace 
        \begin{array}{l}
        x-3\geq 0 \\
        x-3=2x+2\\
        \end{array}
        \right.
        \vee
        \left\lbrace 
        \begin{array}{l}
        x-3< 0 \\
        -(x-3)=2x+2\\
        \end{array}
        \right.
        $$      

        cioè:
        $$
        \left\lbrace 
        \begin{array}{l}
        x\geq 3 \\
        x=-52\\
        \end{array}
        \right.
        \vee
        \left\lbrace 
        \begin{array}{l}
        x< 3 \\
        x=\frac{1}{3}\\
        \end{array}
        \right.
        $$
il primo sistema non ammette soluzione, pertanto la soluzione dell'equazione di 
partenza è $    x=\frac{1}{3}$.
\end{enumerate}

Risolvi le seguenti equazioni:

\begin{enumerate}
\item $\left| x+3 \right| =5x-2 $ \hfill $\left[ \frac{5}{4}\right] $
\item $\left| x-1 \right| =2x $ \hfill $\left[ \frac{1}{3}\right] $
\item $\left| x-4 \right| =-6+2x $ \hfill $\left[ \frac{10}{3}\right] $
\item $\left| x+1 \right| =\frac{1}{2}x^2-3 $ \hfill $\left[ -1-\sqrt{5}\right] 
$
\item $x^2-2= \left| x \right| $ \hfill $\left[ -2, 2\right] $
\item $2\left| x+1 \right| =x^2-2 $ \hfill $\left[1+\sqrt{5}, -2\right] $
\item $\left| x-7 \right| =x-8 $ \hfill $\left[ impossibile\right] $
\item $\left| \frac{x^2+1}{x} \right| =2x $ \hfill $\left[ 1 \right] $
\item $\left| x^2-4x-12 \right| =x^2 $ \hfill $\left[ -3, 1\pm \sqrt{7}\right] $
\item $\left| x^2-3x+2 \right| =-4+2x $ \hfill $\left[ 2, 3\right] $
\item $\left| x^2-4 \right| -x=8 $ \hfill $\left[ -3, 4\right] $
\end{enumerate}


Disequazioni:\\
Esempi:
\begin{enumerate}
        \item[a)] $|x-2|\geq -2$, il valore assoluto di un numero é sempre 
positivo o nullo, perciò la disequazione è verificata per ogni $x\in 
\mathbb{R}$.

        \item[b)] $|5x-2|\leq 0$, il valore assoluto di un numero é sempre 
positivo o nullo, perciò la disequazione è verificata se e solo se $5x-2=0$ 
quindi $x=\frac{2}{5}$.
        \item[c)] $|x-3|>5$, la disequazione è soddisfatta se $x-3<-5 \vee 
x-3>5$ quindi quando $x<-2 \vee x>8$.
        \item[d)] $|x-4|\leq 5$, la disequazione è equivalente a $-5\leq x-4 
\leq 5$ e quindi $-1\leq x \leq 9$.
\end{enumerate}

Risolvi le seguenti disequazioni:

\begin{enumerate}
\item $\left| 5x-2\right| \geq -2 $ \hfill $\left[ \dots \right] $
\item $\left| 8x+2\right| < 0 $ \hfill $\left[ \dots \right] $
\item $\left| x-5\right| \geq 3 $ \hfill $\left[ x\leq 2 \vee x\geq 8 \right] $
\item $\left| x-3\right| >0 $ \hfill $\left[ x\neq 3 \right] $
\item $\left| 2x-5\right| \leq 7 $ \hfill $\left[ -1\leq x \leq 6 \right] $
\item $\left| x^2+3x\right| >4 $ \hfill $\left[ x<-4 \vee x>1 \right] $
\item $\left| x^2-3x+2\right| >0 $ \hfill $\left[ x\neq 1 \wedge x\neq 2 \right] 
$
\item $\left| x^2-4x+4\right| \leq 0 $ \hfill $\left[ x=2 \right] $
\item $-\left| 2x-5\right| <3 $ \hfill $\left[ \forall x \in \mathbb{R} \right] 
$
\item $\left| x^4+16\right| \leq 0 $ \hfill $\left[ impossibile \right] $
\item $\left| 3x+2\right| \geq 5 $ \hfill $\left[ x\leq -\frac{7}{3} \vee x\geq 
1 \right] $
\item $\left| x^2-4\right| <-4 $ \hfill $\left[ impossibile \right] $
\item $-\left| x^3+2\right| <3 $ \hfill $\left[ \forall x \in \mathbb{R} \right] 
$
\item $\left| \frac{x^2-4}{x}\right| <3 $ \hfill $\left[ -4<x<-1 \vee 1<x<4 
\right] $
\item $\left| \frac{x-5}{x+3}\right| >\frac{1}{2} $ \hfill $\left[x<-3 \vee 
-3<x<\frac{7}{3} \vee x>13 \right] $
\item $\left| \frac{x-2}{x-4}\right| >1 $ \hfill $\left[x>3 \wedge x \neq 4 
\right] $
\end{enumerate}








\subsection{Esercizi riepilogativi}

\begin{esercizio}
\label{ese:D.19}
testo esercizio
\end{esercizio}

\begin{esercizio}\label{ese:03.1}
Consegna:
 \begin{enumeratea}
  \item  
 \end{enumeratea}
\end{esercizio}
