% (c) 2012 Claudio Carboncini - claudio.carboncini@gmail.com
% (c) 2012 -2014 Dimitrios Vrettos - d.vrettos@gmail.com
% (c) 2014 Daniele Zambelli - daniele.zambelli@gmail.com

\section{Esercizi}

\subsection{Esercizi dei singoli paragrafi}

%\subsubsection*{13.2 - Identità ed equazioni}
\subsubsection*{\numnameref{sec:13_definizioni}}

\begin{esercizio}
\label{ese:13.1}
Risolvi in~$\insZ$ la seguente equazione:~$-x+3=-1$

\emph{Suggerimento}. Lo schema operativo è: entra~$x$, cambia il segno in~$-x$, 
aggiunge~$3$, si ottiene~$-1$
Ora ricostruisci il cammino inverso: da~$-1$ togli~$3$ ottieni \ldots cambia 
segno ottieni come soluzione~$x = \ldots$
\end{esercizio}

%\subsubsection*{13.3 - Risoluzione di equazioni numeriche intere di primo 
% grado}
\subsubsection*{\numnameref{sec:13_principi}}

\begin{esercizio}
\label{ese:13.2}
Risolvi le seguenti equazioni applicando il~1° principio di equivalenza.
\begin{multicols}{3}
\begin{enumeratea}
% \spazielenx
 \item $x+2=7$
 \item $2+x=3$
 \item $16+x=26$
 \item $x-1=1$
 \item $3+x=-5$
 \item $12+x=-22$
 \item $3x=2x-1$
 \item $8x=7x+4$
 \item $2x=x-1$
 \item $5x=4x+2$
 \item $3x=2x-3$
 \item $3x=2x-2$
 \item $7+x=0$
 \item $7=-x$
 \item $-7=x$
 \item $1+x=0$
 \item $1-x=0$
 \item $0=2-x$
 \item $3x-1=2x-3$
 \item $7x-2x-2=4x-1$
 \item $-5x+2=-6x+6$
 \item $-2+5x=8+4x$
 \item $7x+1=6x+2$
 \item $-1-5x=3-6x$
\end{enumeratea}
\end{multicols}
\end{esercizio}

%%%%%%%%%%%%%%%%%%%%%%%%%%%%%%%%%%%%%%%%%%%%%%%%%%%%%%%

\begin{esercizio}
\label{ese:13.6}
Risolvi le seguenti equazioni applicando il~2° principio di equivalenza.
\begin{multicols}{3}
\begin{enumeratea}
\spazielenx
 \item $2x=8$
 \item $2x=3$
 \item $6x=24$
 \item $0x=1$
 \item $\dfrac{1}{3}x=-1$
 \item $\dfrac{1}{2}x=\dfrac{1}{4}$
 \item $\dfrac{3}{2}x=12$
 \item $2x=-2$
 \item $3x=\dfrac{1}{6}$
 \item $\dfrac{1}{2}x=4$
 \item $\dfrac{3}{4}x=\dfrac{12}{15}$
 \item $2x=\dfrac{1}{2}$
 \item $3x=6$
 \item $\dfrac{1}{3}x=\dfrac{1}{3}$
 \item $\dfrac{2}{5}x=\dfrac{10}{25}$
 \item $-{\dfrac{1}{2}}x=-{\dfrac{1}{2}}$
 \item $0,1x=1$
 \item $0,1x=10$
 \item $0,1x=0,5$
 \item $-0,2x=5$
\end{enumeratea}
\end{multicols}
\end{esercizio}

%%%%%%%%%%%%%%%%%%%%%%%%%%%%%%%%%%%%%%%%%%%%%%%%%%%%%%%%%%%%5

\begin{esercizio}
\label{ese:13.9}
Risolvi le seguenti equazioni applicando entrambi i principi.
\begin{multicols}{3}
\begin{enumeratea}
\spazielenx
 \item $2x+1=7$
 \item $3-2x=3$
 \item $6x-12=24$
 \item $3x+3=4$
 \item $5-x=1$
 \item $7x-2=5$
 \item $2x+8=8-x$
 \item $2x-3=3-2x$
 \item $6x+24=3x+12$
 \item $2+8x=6-2x$
 \item $6x-6=5-x$
 \item $-3x+12=3x+18$
 \item $3-2x=8+2x$
 \item $\dfrac{2}{3}x-3=\dfrac{1}{3}x+1$
 \item $\dfrac{6}{5}x=\dfrac{24}{5}-x$
 \item $3x-2x+1=2+3x-1$
 \item $\dfrac{2}{5}x-\dfrac{3}{2}=\dfrac{3}{2}x+\dfrac{1}{10}$
 \item $\dfrac{5}{6}x+\dfrac{3}{2}=\dfrac{25}{3}-\dfrac{10}{2}x$
\end{enumeratea}
\end{multicols}
\end{esercizio}

%%%%%%%%%%%%%%%%%%%%%%%%%%%%%%%%%%%%%%%%%%%%%%%%%%%%%%%%%%%%%%%%%%%%%%%%

\begin{esercizio}
\label{ese:13.12}
Risolvi l'equazione~$10x+4=-2\cdot (x+5)-x$ seguendo la traccia:
\begin{enumerate}
\spazielenx
 \item svolgi i calcoli al primo e al secondo membro: \dotfill;
 \item somma i monomi simili in ciascun membro dell'equazione: \dotfill;
 \item applica il primo principio d'equivalenza per lasciare in un membro solo 
monomi con l'incognita e nell'altro membro solo numeri: \dotfill;
 \item somma i termini del primo membro e somma i termini del secondo membro: 
\dotfill;
 \item applica il secondo principio d'equivalenza dividendo ambo i membri per il 
coefficiente dell'incognita: \dotfill in forma canonica: \dotfill;
 \item scrivi l'Insieme Soluzione:~$\IS = \ldots \ldots \ldots$
\end{enumerate}
\end{esercizio}

\begin{esercizio}
\label{ese:13.13}
Risolvi, seguendo la traccia, l'equazione~$x-(3x+5)=(4x+8)-4\cdot (x+1)$:
\begin{enumerate}
\spazielenx
 \item svolgi i calcoli: \dotfill;
 \item somma i monomi simili: \dotfill;
 \item porta al primo membro i monomi con la~$x$ e al secondo quelli 
senza:~$\dotfill$
 \item somma i monomi simili al primo membro e al secondo membro:~$\dotfill$
 \item dividi ambo i membri per il coefficiente dell'incognita:~$\dotfill$
 \item l'insieme soluzione è:~$\dotfill$
\end{enumerate}
\end{esercizio}

%%%%%%%%%%%%%%%%%%%%%%%%%%%%%%%%%%%%%%%%%%%%%%%%%%%%%%%%%%%%%%

\begin{esercizio}[\Ast]
\label{ese:13.14}
Risolvi le seguenti equazioni con le regole pratiche indicate.
 \begin{enumeratea}
 \item $3(x-1)+2(x-2)+1=2x$ \hfill $\left[2\right]$
 \item $x-(2x+2)=3x-(x+2)-1$ \hfill $\left[\frac{1}{3}\right]$
 \item $-2(x+1)-3(x-2)=6x+2$ \hfill $\left[\frac{2}{11}\right]$
 \item $x+2-3(x+2)=x-2$ \hfill $\left[-\frac{2}{3}\right]$
 \item $2(1-x)-(x+2)=4x-3(2-x)$ \hfill $\left[\frac{3}{5}\right]$
 \item $(x+2)^{2}=x^{2}-4x+4$ \hfill $\left[0\right]$
 \item $5(3x-1)-7(2x-4)=28$ \hfill $\left[5\right]$
 \item $(x+1)(x-1)+2x=5+x(2+x)$ \hfill $\left[Impossibile\right]$
 \item $2x+(x+2)(x-2)+5=(x+1)^{2}$ \hfill $\left[Indeterminata\right]$
 \item $4(x-2)+3(x+2)=2(x-1)-(x+1)$ \hfill $\left[-\frac{1}{6}\right]$
 \item $(x+2)(x+3)-(x+3)^{2}=(x+1)(x-1)-x(x+1)$ 
  \hfill $\left[Impossibile\right]$
 \item $x^{3}+6x^{2}+(x+2)^{3}+11x+(x+2)^{2}=(x+3)\left(2x^{2}+7x\right)$ 
  \hfill $\left[-2\right]$
 \item $(x+2)^{3}-(x-1)^{3}=9(x+1)^{2}-9x$ \hfill $\left[Indeterminata\right]$
 \item $(x+1)^{2}+2x+2(x-1)=(x+2)^{2}$ \hfill $\left[\frac{5}{2}\right]$
 \item $2(x-2)(x+3)-3(x+1)(x-4)=-9(x-2)^{2}+\left(8x^{2}-25x+36\right)$ 
  \hfill $\left[Indeterminata\right]$
 \item $(2x-3)^{2}-4x(2-5x)-4=-8x(x+4)$ \hfill $\left[\right]$
 \item $(x-1)\left(x^{2}+x+1\right)-3x^{2}=(x-1)^{3}+1$ \hfill $\left[\right]$
 \item $(2x-1)\left(4x^{2}+2x+1\right)=(2x-1)^{3}-12x^{2}$ 
  \hfill $\left[\right]$
 \end{enumeratea}
\end{esercizio}

%%%%%%%%%%%%%%%%%%%%%%%%%%%%%%%%%%%%%%%%%%%%%%%%%%%%%%%%%%%%%%%%%%%%%%%

%\subsubsection*{13.4 - Equazioni a coefficienti frazionari}
\subsubsection*{\numnameref{sec:13_coefffraz}}

\begin{esercizio}
\label{ese:13.19}
Risolvi l'equazione~$\dfrac{3\cdot (x-11)}{4}=\dfrac{3\cdot 
(x+1)}{5}-\dfrac{1}{10}$
\begin{enumerate}
\spazielenx
 \item calcola~$\mcm(4,5,10) = \ldots \ldots$
 \item moltiplica ambo i membri per \dotfill e ottieni: \dotfill;
 \item \dotfill
\end{enumerate}
\end{esercizio}

%%%%%%%%%%%%%%%%%%%%%%%%%%%%%%%%%%%%%%%%%%%%%%%%%%%%%%%%%%%%%%%%%%%%%%%
% \begin{esercizio}
% \label{ese:13.20}
% Risolvi le seguenti equazioni nell'insieme a fianco indicato.
% \begin{multicols}{3}
% \begin{enumeratea}
% \spazielenx
%  \item $x+7=8,\, \insN$
%  \item $4+x=2,\, \insZ$
%  \item $x-3=4,\, \insN$
%  \item $x=0,\,\insN$
%  \item $x+1=0,\, \insZ$
%  \item $5x=0,\, \insZ$
% \end{enumeratea}
% \end{multicols}
% \end{esercizio}
% 
% % \newpage
% \begin{esercizio}
% \label{ese:13.21}
% Risolvi le seguenti equazioni nell'insieme a fianco indicato.
% \begin{multicols}{3}
% \begin{enumeratea}
% \spazielenx
%  \item $\dfrac{x}{4}=0,\, \insQ$
%  \item $-x=0,\, \insZ$
%  \item $7+x=0,\, \insZ$
%  \item $-2x=0,\, \insZ$
%  \item $-x-1=0,\, \insZ$
%  \item $\dfrac{-x}{4}=0,\, \insQ$
% \end{enumeratea}
% \end{multicols}
% \end{esercizio}
% 
% \begin{esercizio}
% \label{ese:13.22}
% Risolvi le seguenti equazioni nell'insieme a fianco indicato.
% \begin{multicols}{3}
% \begin{enumeratea}
% \spazielenx
%  \item $x-\dfrac{2}{3}=0,\, \insQ$
%  \item $\dfrac{x}{-3}=0,\, \insZ$
%  \item $2(x-1)=0,\, \insZ$
%  \item $-3x=1,\, \insQ$
%  \item $3x=-1,\, \insQ$
%  \item $\dfrac{x}{3}=1,\, \insQ$
% \end{enumeratea}
% \end{multicols}
% \end{esercizio}
% 
% \begin{esercizio}
% \label{ese:13.23}
% Risolvi le seguenti equazioni nell'insieme a fianco indicato.
% \begin{multicols}{3}
% \begin{enumeratea}
% \spazielenx
%  \item $\dfrac{x}{3}=2,\, \insQ$
%  \item $\dfrac{x}{3}=-2,\, \insQ$
%  \item $0x=0,\, \insQ$
%  \item $0x=5,\, \insQ$
%  \item $0x=-5,\, \insQ$
%  \item $\dfrac{x}{1}=0,\, \insQ$
% \end{enumeratea}
% \end{multicols}
% \end{esercizio}
% 
% \begin{esercizio}
% \label{ese:13.24}
% Risolvi le seguenti equazioni nell'insieme a fianco indicato.
% \begin{multicols}{3}
% \begin{enumeratea}
% \spazielenx
%  \item $\dfrac{x}{1}=1,\, \insQ$
%  \item $-x=10,\, \insZ$
%  \item $\dfrac{x}{-1}=-1,\, \insZ$
%  \item $3x=3,\, \insN$
%  \item $-5x=2,\, \insZ$
%  \item $3x+2=0,\, \insQ$
% \end{enumeratea}
% \end{multicols}
% \end{esercizio}

% \begin{esercizio}
% \label{ese:13.25}
% Risolvi le seguenti equazioni nell'insieme~$\insQ$
% \begin{multicols}{3}
% \begin{enumeratea}
% \spazielenx
%  \item $3x=\dfrac{1}{3}$
%  \item $-3x=-{\dfrac{1}{3}}$
%  \item $x+2=0$
%  \item $4x-4=0$
%  \item $4x-0=1$
%  \item $2x+3=x+3$
% \end{enumeratea}
% \end{multicols}
% \end{esercizio}
% 
% \begin{esercizio}
% \label{ese:13.26}
% Risolvi le seguenti equazioni nell'insieme~$\insQ$
% \begin{multicols}{3}
% \begin{enumeratea}
% \spazielenx
%  \item $4x-4=1$
%  \item $4x-1=1$
%  \item $4x-1=0$
%  \item $3x=12-x$
%  \item $4x-8=3x$
%  \item $-x-2=-2x-3$
% \end{enumeratea}
% \end{multicols}
% \end{esercizio}
% 
% \begin{esercizio}
% \label{ese:13.27}
% Risolvi le seguenti equazioni nell'insieme~$\insQ$
% \begin{multicols}{3}
% \begin{enumeratea}
% \spazielenx
%  \item $-3(x-2)=3$
%  \item $x+2=2x+3$
%  \item $-x+2=2x+3$
%  \item $3(x-2)=0$
%  \item $3(x-2)=1$
%  \item $3(x-2)=3$
% \end{enumeratea}
% \end{multicols}
% \end{esercizio}
% 
% \begin{esercizio}
% \label{ese:13.28}
% Risolvi le seguenti equazioni nell'insieme~$\insQ$
% \begin{multicols}{3}
% \begin{enumeratea}
% \spazielenx
%  \item $0(x-2)=1$
%  \item $0(x-2)=0$
%  \item $12+x=-9x$
%  \item $40x+3=30x-100$
%  \item $4x+8x=12x-8$
%  \item $-2-3x=-2x-4$
% \end{enumeratea}
% \end{multicols}
% \end{esercizio}

% \newpage
\begin{esercizio}
\label{ese:13.29}
Risolvi le seguenti equazioni.
\begin{multicols}{3}
\begin{enumeratea}
\spazielenx
 \item $2x+2=2x+3$
 \item $\dfrac{x+2}{2}=\dfrac{x+1}{2}$
 \item $\dfrac{2x+1}{2}=x+1$
 \item $\dfrac{x}{2}+\dfrac{1}{4}=3x-\dfrac{1}{2}$
 \item $\pi x=0$
 \item $2\pi x=\pi$
 \item $0,12x=0,1$
 \item $-{\dfrac{1}{2}}x-0,3=-{\dfrac{2}{5}}x-{\dfrac{3}{20}}$
 \item $892x-892=892x-892$
 \item $892x-892=893x-892$
 \item $348x-347=340x-347$
%  \item $340x+740=8942+340x$
 \item $2x+3=2x+4$
 \item $2x+3=2x+3$
 \item $2(x+3)=2x+5$
 \item $2(x+4)=2x+8$
 \item $3x+6=6x+6$
 \item $-2x+3=-2x+4$
 \item $\dfrac{x}{2}+\dfrac{1}{4}=\dfrac{x}{4}-\dfrac{1}{2}$
 \item $\dfrac{x}{2}+\dfrac{1}{4}=\dfrac{x}{2}-\dfrac{1}{2}$
 \item $\dfrac{x}{2}+\dfrac{1}{4}=3\dfrac{x}{2}-\dfrac{1}{2}$
 \item $\dfrac{x}{200}+\dfrac{1}{100}=\dfrac{1}{200}$
%  \item $1000x-100=2000x-200$
%  \item $100x-1000=-1000x+100$
\end{enumeratea}
\end{multicols}
\end{esercizio}

\begin{esercizio}[\Ast]
\label{ese:13.33}
Risolvi le seguenti equazioni.
\begin{multicols}{2}
\begin{enumeratea}
\spazielenx
 \item $x-5(1-x)=5+5x$ \hfill $\left[10\right]$
 \item $2(x-5)-(1-x)=3x$ \hfill $\left[Impossibile\right]$
 \item $3(2+x)=5(1+x)-3(2-x)$ \hfill $\left[\frac{7}{5}\right]$
 \item $4(x-2)-3(x+2)=2(x-1)$ \hfill $\left[-12\right]$
 \item $\dfrac{x+1000}{3}+\dfrac{x+1000}{4}=1$ 
  \hfill $\left[-\frac{6988}{7}\right]$
 \item $\dfrac{x-4}{5}\;=\;\dfrac{2x+1}{3}$ \hfill $\left[-\frac{17}{7}\right]$
 \item $\dfrac{x+1}{2}+\dfrac{x-1}{5}=\dfrac{1}{10}$ 
  \hfill $\left[-{\frac{2}{7}}\right]$
 \item $\dfrac{x}{3}-\dfrac{1}{2}\;=\;\dfrac{x}{4}-\dfrac{x}{6}$ 
  \hfill $\left[2\right]$
 \item $8x-\dfrac{x}{6}=2x+11$ \hfill $\left[\frac{66}{35}\right]$
 \item $3(x-1)-\dfrac{1}{7}=4(x-2)+1$ \hfill $\left[\frac{27}{7}\right]$
 \item $537x+537\dfrac{x}{4}-\dfrac{537x}{7}=0$ \hfill $\left[0\right]$
 \item $\dfrac{2x+3}{5}=x-1$ \hfill $\left[\frac{8}{3}\right]$
 \item $\dfrac{x}{2}-\dfrac{x}{6}-1=\dfrac{x}{3}$ 
  \hfill $\left[Impossibile\right]$
 \item $\dfrac{4-x}{5}+\dfrac{3-4x}{2}=3$ 
  \hfill $\left[-{\frac{7}{22}}\right]$
 \item $\dfrac{x+3}{2}=3x-2$ 
  \hfill $\left[\frac{7}{5}\right]$
 \item $\dfrac{x+0,25}{5}=1,75-0,\overline{{3}}x$ 
  \hfill $\left[\frac{51}{16}\right]$
 \item $3(x-2)-4(5-x)=3x\left(1-\dfrac{1}{3}\right)$ 
  \hfill $\left[\frac{26}{5}\right]$
%  \item $4(2x-1)+5=1-2(-3x-6)$ 
%   \hfill $\left[6\right]$
\end{enumeratea}
\end{multicols}
\end{esercizio}

\begin{esercizio}[\Ast]
\label{ese:13.36}
Risolvi le seguenti.
\begin{multicols}{2}
\begin{enumeratea}
\spazielenx
 \item $\dfrac{3}{2}(x+1)-\dfrac{1}{3}(1-x)=x+2$ 
  \hfill $\left[1\right]$
 \item $\dfrac{1}{2}(x+5)-x=\dfrac{1}{2}(3-x)$ 
  \hfill $\left[Impossibile\right]$
 \item $(x+3)^{2}\;=\;(x-2)(x+2)+\dfrac{1}{3}x$ 
  \hfill $\left[-{\frac{39}{17}}\right]$
 \item $\dfrac{(x+1)^{2}}{4}-\dfrac{2+3x}{2}\;=\;\dfrac{(x-1)^{2}}{4}$ 
  \hfill $\left[-2\right]$
%  \item $2\left(x-\dfrac{1}{3}\right)+x\;=\;3x-2$ 
%   \hfill $\left[Impossibile\right]$
 \item $\dfrac{3}{2}x+\dfrac{x}{4}=5\left(\dfrac{2}{3}x-\dfrac{1}{2}\right)-x$
  \hfill $\left[\frac{30}{7}\right]$
 \item $(2x-3)(5+x)+\dfrac{1}{4}=2(x-1)^{2}-\dfrac{1}{2}$ 
  \hfill $\left[\frac{65}{44}\right]$
 \item $(x-2)(x+5)+\dfrac{1}{4}=x^{2}-\dfrac{1}{2}$ 
  \hfill $\left[\frac{37}{12}\right]$
 \item $\left(x-\dfrac{1}{2}\right)\left(x-\dfrac{1}{2}\right)=
        x^{2}+\dfrac{1}{2}$ \hfill $\left[-{\frac{1}{4}}\right]$
%  \item $(x+1)^{2}=(x-1)^{2}$ \hfill $\left[\right]$
 \item $\dfrac{(1-x)^{2}}{2}-\dfrac{x^{2}-1}{2}=1$ \hfill $\left[0\right]$
 \item $\dfrac{(x+1)^{2}}{3}=\dfrac{1}{3}(x^{2}-1)$ \hfill $\left[-1\right]$
\end{enumeratea}
\end{multicols}
\end{esercizio}

\begin{esercizio}[\Ast]
\label{ese:13.38}
Risolvi le seguenti equazioni.
\begin{enumeratea}
\spazielenx
 \item $4(x+1)-3x(1-x)=(x+1)(x-1)+4+2x^{2}$ \hfill $\left[-1\right]$
 \item $\dfrac{1-x}{3}\cdot (x+1)=1-x^{2}+\dfrac{2}{3}\left(x^{2}-1\right)$ 
  \hfill $\left[Indeterminata\right]$
 \item $(x+1)^{2}=x^{2}-1$ \hfill $\left[-1\right]$
 \item $(x+1)^{3}=(x+2)^{3}-3x(x+3)$ \hfill $\left[Impossibile\right]$
 \item $\dfrac{1}{3}x\left(\dfrac{1}{3}x-1\right)+
        \dfrac{5}{3}x\left(1+\dfrac{1}{3}x\right)=\dfrac{2}{3}x(x+3)$ 
         \hfill $\left[0\right]$
 \item $\dfrac{1}{2}\left(3x+\dfrac{1}{3}\right)-
        (1-x)+2\left(\dfrac{1}{3}x-1\right)=-{\dfrac{3}{2}}x+1$ 
         \hfill $\left[\frac{23}{28}\right]$
 \item $3+2x-\dfrac{1}{2}\left(\dfrac{x}{2}+1\right)-\dfrac{3}{4}x=
        \dfrac{3}{4}x+\dfrac{x+3}{2}$ \hfill $\left[4\right]$
 \item $\dfrac{1}{2}\left[\dfrac{x+2}{2}-\left(x+\dfrac{1}{2}\right)+
        \dfrac{x+1}{2}\right]+\dfrac{1}{4}x=
        \dfrac{x-2}{4}-\left(x+\dfrac{2-x}{3}\right)$ 
  \hfill $\left[-{\frac{5}{2}}\right]$
 \item $2\left(x-\dfrac{1}{2}\right)^{2}+\left(x+\dfrac{1}{2}\right)^{2}
        =(x+1)(3x-1)-5x-\dfrac{1}{2}$ \hfill $\left[-{\frac{9}{8}}\right]$
 \item $\dfrac{2\left(x-1\right)}{3}+\dfrac{x+1}{5}-\dfrac{3}{5}=
        \dfrac{x-1}{5}+\dfrac{7}{15}x$ \hfill $\left[\frac{13}{3}\right]$
 \item $\dfrac{1}{2}(x-2)-\left(\dfrac{x+1}{2}-\dfrac{1+x}{2}\right)=
        \dfrac{1}{2}-\dfrac{2-x}{6}+\dfrac{1+x}{3}$ 
  \hfill $\left[Impossibile\right]$
 \item $-\left(\dfrac{1}{2}x+3\right)-\dfrac{1}{2}\left(x+\dfrac{5}{2}\right)+
        \dfrac{3}{4}(4x+1)=\dfrac{1}{2}(x-1)$ \hfill $\left[2\right]$
 \item $\dfrac{(x+1)(x-1)}{9}-\dfrac{3x-3}{6}=
        \dfrac{(x-1)^{2}}{9}-\dfrac{2-2x}{6}$ \hfill $\left[1\right]$
 \item $\left(x-\dfrac{1}{2}\right)^{3}-
        \left(x+\dfrac{1}{2}\right)^{2}-x(x+1)(x-1)=\dfrac{-5}{2}x(x+1)$ 
  \hfill $\left[\frac{3}{26}\right]$
 \item $\dfrac{1}{2}\left(3x-\dfrac{1}{3}\right)-
        \dfrac{1}{3}(1+x)(-1+x)+3\left(\dfrac{1}{3}x-1\right)^{2}=
        \dfrac{2}{3}x$ \hfill $\left[\frac{19}{7}\right]$
 \item $(x-2)(x-3)-6=(x+2)^2 +5$ \hfill $\left[-1\right]$
 \item $(x-3)(x-4)-\dfrac{1}{3}(1-3x)(2-x)=
        \dfrac{1}{3}x-5\left(\dfrac{2x-9}{6}\right)$ 
  \hfill $\left[\frac{23}{20}\right]$
 \item $\dfrac{2w-1}{3}+\dfrac{w-5}{4}=\dfrac{w+1}{3}-4$ 
  \hfill $\left[-\frac{25}{7}\right]$
\end{enumeratea}
\end{esercizio}

\begin{esercizio}[\Ast]
\label{ese:13.41}
Risolvi le seguenti equazioni.
\begin{enumeratea}
\spazielenx
 \item $(2x-5)^2 +2(x-3)=(4x-2)(x+3)-28x+25$
  \hfill $\left[Indeterminata\right]$
 \item $\dfrac{(x-3)(x+3)+(x-2)(2-x)-3(x-2)}{\dfrac{1}{3}-3}=
        \dfrac{\dfrac{2}{3}x+\dfrac{1}{2}x}{2}$
  \hfill $\left[\frac{63}{23}\right]$
 \item $2\left(\dfrac{1}{2}x-1\right)^{2}-\dfrac{(x+2)(x-2)}{2}+2x=
        x+\dfrac{1}{2}$
  \hfill $\left[\frac{7}{2}\right]$
 \item $\left(0,\overline{{1}}x-10\right)^{2}+0,1(x-0,2)+
        \left(\dfrac{1}{3}x+0,3\right)^{2}=\dfrac{10}{81}x^{2}+0,07$
  \hfill $\left[\frac{9000}{173}\right]$
 \item $5x+\dfrac{1}{6}-\left(\dfrac{2x+1}{2}\right)^{2}+
        \left(\dfrac{3x-1}{3}\right)^{2}+\dfrac{1}{3}x+(2x-1)(2x+1)=
        (2x+1)^{2}+\dfrac{1}{36}$
  \hfill $\left[-6\right]$
%  \item \begin{multline*}
%   \left(1+\dfrac{1}{2}x\right)^{3}-2\left(\dfrac{1}{2}x-2\right)^{2}+
%   \left(\dfrac{3x-1}{3}\right)^{2}-\left(1-\dfrac{1}{3}x\right)x+\dfrac{1}{3}x=
%   {\dfrac{1}{3}}(2x+1)^{2} \\
%   +\dfrac{1}{4}x^{2}-\dfrac{5}{9}+\dfrac{1}{2}x\left(\dfrac{1}{2}x+1\right)
%   \left(\dfrac{1}{2}x-1\right)
%  \end{multline*}
%   \hfill $\left[2\right]$
 \item $\left(\dfrac{1}{2}x+\dfrac{1}{3}\right)\left(\dfrac{1}{2}x-
        \dfrac{1}{3}\right)+\left(\dfrac{1}{2}+\dfrac{1}{3}\right)x=
        \left(\dfrac{1}{2}x+1\right)^{2}$
  \hfill $\left[-{\frac{20}{3}}\right]$
 \item $\dfrac{3}{20}+\dfrac{6x+8}{10}-\dfrac{2x-1}{12}+\dfrac{2x-3}{6}=
        \dfrac{x-2}{4}$
  \hfill $\left[-2\right]$
 \item $\dfrac{x^{3}-1}{18}+\dfrac{(x+2)^{3}}{9}=
        \dfrac{(x+1)^{3}}{4}-\dfrac{x^{3}+x^{2}-4}{12}$
  \hfill $\left[-{\frac{3}{7}}\right]$
 \item $\dfrac{2}{3}x+\dfrac{5x-1}{3}+\dfrac{(x-3)^{2}}{6}+
        \dfrac{1}{3}(x+2)(x-2)=\dfrac{1}{2}(x-1)^{2}$
  \hfill $\left[\frac{2}{7}\right]$
 \item $\dfrac{5}{12}x-12+\dfrac{x-6}{2}-\dfrac{x-24}{3}=
        \dfrac{x+4}{4}-\left(\dfrac{5}{6}x-6\right)$
  \hfill $\left[12\right]$
 \item $x+\dfrac{1}{2}=\dfrac{x+3}{3}-1$
  \hfill $\left[...\right]$
 \item $\dfrac{2}{3}x+\dfrac{1}{2}=\dfrac{1}{6}x+\dfrac{1}{2}x$
  \hfill $\left[...\right]$
 \item $\dfrac{3}{2}=2x-\left[\dfrac{x-1}{3}-
        \left(\dfrac{2x+1}{2}-5x\right)-\dfrac{2-x}{3}\right]$
  \hfill $\left[...\right]$
 \item $\dfrac{x+5}{3}+3+\dfrac{2\cdot \left(x-1\right)}{3}=x+4$
 \item $\dfrac{1}{5}x-1+\dfrac{2}{3}x-2=\dfrac{10}{15}+\dfrac{3}{5}x$
  \hfill $\left[...\right]$
 \item $\dfrac{1}{2}(x-2)^{2}-\dfrac{8x^{2}-25x+36}{18}+\dfrac{1}{9}(x-2)(x+3)=
        \dfrac{1}{6}(x+1)(x-4)$
  \hfill $\left[...\right]$
 \item $\left(1-\dfrac{x+\dfrac{1}{2}}{1-\dfrac{1}{2}}\right)
        \left(1+\dfrac{\dfrac{1}{2}x+1}{\dfrac{1}{2}-1}\right)+
        \left(\dfrac{\dfrac{1}{2}x+1}{\dfrac{1}{2}+1}-1\right)\cdot 
        {\dfrac{\dfrac{1}{2}+x}{\dfrac{1}{2}-1}}-
        \dfrac{x\left(\dfrac{1}{2}x+1\right)}{\dfrac{1}{2}+1}=x^{2}$
  \hfill $\left[-{\frac{1}{5}}\right]$
\end{enumeratea}
\end{esercizio}

\begin{esercizio}
\label{ese:13.44}
Per una sola delle seguenti equazioni, definite in~$\insZ$, l'insieme soluzione 
è vuoto. Per quale?
\[\boxA\quad x=x+1\qquad\boxB\quad x+1=0\qquad\boxC\quad x-1=+1\qquad\boxD\quad 
x+1=1\]
\end{esercizio}

\begin{esercizio}
\label{ese:13.45}
Una sola delle seguenti equazioni è di primo grado nella sola incognita~$x$ 
Quale?
\[\boxA\quad x+y=5\qquad\boxB\quad x^{2}+1=45\qquad\boxC\quad 
x-\dfrac{7}{89}=+1\qquad\boxD\quad x+x^{2}=1\]
\end{esercizio}

\begin{esercizio}
\label{ese:13.46}
Tra le seguenti una sola equazione non è equivalente alle altre. Quale?
\[\boxA\quad \dfrac{1}{2}x-1=3x\qquad\boxB\quad~6x=x-2\qquad\boxC\quad 
x-2x=3x\qquad\boxD\quad~3x=\dfrac{1}{2}(x-2)\]
\end{esercizio}

\begin{esercizio}
\label{ese:13.47}
Da~$8x=2$ si ottiene:
\[\boxA\quad x=-6\qquad\boxB\quad x=4\qquad\boxC\quad 
x=\dfrac{1}{4}\qquad\boxD\quad x=-{\dfrac{1}{4}}\]
\end{esercizio}

\begin{esercizio}
\label{ese:13.48}
Da~$-9x=0$ si ottiene:
\[\boxA\quad x=9\qquad\boxB\quad x=-{\dfrac{1}{9}}\qquad\boxC\quad 
x=0\qquad\boxD\quad x=\dfrac{1}{9}\]
\end{esercizio}

\begin{esercizio}
\label{ese:13.49}
L'insieme soluzione dell'equazione~$2\cdot \left(x+1\right)=5\cdot 
\left(x-1\right)-11$ è:
\[\boxA\quad \IS=\Bigl\{-6\Bigr\}\qquad\boxB\quad 
\IS=\Bigl\{6\Bigr\}\qquad\boxC\quad 
\IS=\left\{\dfrac{11}{3}\right\}\qquad\boxD\quad 
\IS=\left\{\dfrac{1}{6}\right\}\]
\end{esercizio}

\begin{esercizio}
\label{ese:13.50}
Per ogni equazione, individua quali tra gli elementi dell'insieme indicato a 
fianco sono soluzioni:
\begin{enumeratea}
\spazielenx
 \item $\dfrac{x+5}{2}+\dfrac{1}{5}=0$, $\qquad 
Q=\left\{1,-5,\,7,-\dfrac{27}{5}\right\}$
 \item $x-\dfrac{3}{4}x=4$, $\qquad Q=\Bigl\{1,-1,\,0,\,16\Bigr\}$
 \item $x(x+1)+4=5-2x+x^{2}$,$\qquad 
Q=\left\{-9,\,3,\,\dfrac{1}{3},-\dfrac{1}{3}\right\}$
\end{enumeratea}
\end{esercizio}
