%%%%%%%%%%%%%%%%%%%%%%%%%%%%%%%%%%%%%%%%%%%%%%%%%%%%%%%%%%%%%%%%%%%%
%        Matematica dolce
%
% Copyright 2016 Daniele Zambelli
%
%------------------------------
% Matematica dolce per i licei Linguistico e economico sociale, volume 1
%
% m_d_2_les.tex
%------------------------------
%
% This work may be distributed and/or modified under the
% conditions of the LaTeX Project Public License, either version 1.3
% of this license or (at your option) any later version.
% The latest version of this license is in
%   http://www.latex-project.org/lppl.txt
% and version 1.3 or later is part of all distributions of LaTeX
% version 2005/12/01 or later.
%
% This work has the LPPL maintenance status `maintained'.
% 
% The Current Maintainer of this work is 
% Dimitrios Vrettos - d.vrettos@gmail.com
%
% This work consists of the files:
%  -  m_d_2_les.tex (this file)
%  -  createpdf.py
%  -  readme.md
%  -  the part of code of the files under chap/, img/  and lbr/ directories
%%%%%%%%%%%%%%%%%%%%%%%%%%%%%%%%%%%%%%%%%%%%%%%%%%%%%%%%%%%%%%%%%%%%

%========================
% Lettura preambolo
%========================
%=============================%
% VARIABILI MATEMATICA DOLCE  %
%=============================%
%
%% Nomi di autori, collaboratori, etc
%%
\newcommand{\coord}{Daniele~Zambelli}

\newcommand{\autori}{
Leonardo~Aldegheri,
Elisabetta~Campana, 
Luciana~Formenti, 
Michele~Perini,
Maria~Antonietta~Pollini, 
Nicola~Sansonetto, 
Andrea~Sellaroli,
Daniele~Zambelli
}
\newcommand{\colab}{
Alberto~Bicego, 
Alberto~Filippini
}
\newcommand{\texcol}{
Claudio~Carboncini, 
Silvia~Cibola, 
Tiziana~Manca,
Daniele~Zambelli
}
%%%%%%% EOF

%%%%%%%%%%%%%%%%%%%%%%%%%%%%%%%%%%%%%%%%%%%%%%%%%%%%%%%%%%%%%%%%%%%%
%%        Matematica dolce
%%%%%%%%%%%%%%%%%%%%%%%%%%%%%%%%%%%%%%%%%%%%%%%%%%%%%%%%%%%%%%%%%%%%
%% Copyright 2016 Daniele Zambelli
%------------------------------
%% intestazioni.tex
%------------------------------
%
% This work may be distributed and/or modified under the
% conditions of the LaTeX Project Public License, either version 1.3
% of this license or (at your option) any later version.
% The latest version of this license is in
%   http://www.latex-project.org/lppl.txt
% and version 1.3 or later is part of all distributions of LaTeX
% version 2005/12/01 or later.
%
% This work has the LPPL maintenance status `maintained'.
% 
% The Current Maintainer of this work is 
% Dimitrios Vrettos - d.vrettos@gmail.com
%
% This work consists of the files:
%  -  algebra1_light_1ed.tex (this file)
%  -  Makefile
%  -  README
%  -  the part of code of the files under chap/, img/  and lbr/ directories
%%%%%%%%%%%%%%%%%%%%%%%%%%%%%%%%%%%%%%%%%%%%%%%%%%%%%%%%%%%%%%%%%%%%
\documentclass[10pt,a4paper,openright]{matc3mem}
% \documentclass[10pt,a4paper,openright,gray]{matc3mem}
%========================
%  Lingua & codifica
%========================
\usepackage[T1]{fontenc} 
\usepackage{textcomp} 	
\usepackage[utf8x]{inputenc}
\usepackage[italian]{babel}
%========================
% Font
%========================
\renewcommand{\rmdefault}{ppl}
\usepackage{mathpazo}
\usepackage[scaled=.95]{helvet}
\usepackage{eulervm}
%========================
% Tipografia
%========================
\usepackage[bindingoffset=6mm]{geometry}
\usepackage{multicol}
\usepackage{multirow}
\usepackage{indentfirst}
\usepackage{emptypage}
\usepackage{nonumonpart}
\usepackage{enumitem}
\usepackage{tabto}
\usepackage{microtype}
%========================
% Tabelle aggiunto da claudio
%========================
\usepackage{threeparttable}
%========================
% Matematica I
%========================
%\usepackage{amsmath} % aggiunto da daniele
\usepackage{amssymb}
\usepackage{amsthm}
\usepackage{cancel}
\usepackage[]{units}
%========================
% Grafica
%========================
\usepackage[pdftex]{graphicx}
\usepackage{rotating}
\usepackage{shadow}
\usepackage{fancybox}
\usepackage{empheq}
\usepackage{framed}
\usepackage{wrapfig}
%========================
% pgf & TikZ
%========================
\usepackage{tikz}
\usepackage{pgfplots}
\usepgfplotslibrary{patchplots}
\pgfplotsset{compat=1.8}
\usepackage{tkz-euclide}
\usepackage{tkz-fct}
\usepackage{circuitikz}
\usepackage{tikz-qtree} % aggiunto da daniele
\usetkzobj{all}
%========================
% Librerie TikZ
%========================
\usetikzlibrary{arrows,%
                arrows.meta,
                through,
		automata,%
		backgrounds,%
		calc,%
		decorations.markings,%
		decorations.shapes,%
		decorations.text,% 
		decorations.pathreplacing,%
		fit,%
		matrix,%
		mindmap,%
		patterns,%
		positioning,%
		intersections,%aggiunto da claudio
		shapes,%
		shapes.geometric}
%========================
% Simboli
%========================
\usepackage{marvosym}
\usepackage{eurosym}
\usepackage{pifont}
% \usepackage{hiero}
%========================
% Matematica II
%========================
\usepackage{matc3}
%========================
% Collegamenti
%========================
\usepackage[colorlinks, hypertexnames=false]{hyperref}
\usepackage{bookmark}
%========================
% Personalizzazioni
%========================
%=============================%
% VARIABILI MATEMATICA DOLCE  %
%=============================%
%
%% Nomi di autori, collaboratori, etc
%%
\newcommand{\coord}{Daniele~Zambelli}

\newcommand{\autori}{
Leonardo~Aldegheri,
Elisabetta~Campana, 
Luciana~Formenti, 
Michele~Perini,
Maria~Antonietta~Pollini, 
Nicola~Sansonetto, 
Andrea~Sellaroli,
Daniele~Zambelli
}
\newcommand{\colab}{
Alberto~Bicego, 
Alberto~Filippini
}
\newcommand{\texcol}{
Claudio~Carboncini, 
Silvia~Cibola, 
Tiziana~Manca,
Daniele~Zambelli
}
%%%%%%% EOF

% \graphicspath{{img/}}
\setsecnumdepth{subsection} 
\maxtocdepth{subsection}
\setlength{\cftpartnumwidth}{2.25em}
\setlength{\shadeboxsep}{5pt} 
\setlength{\shadeboxrule}{.4pt} 
\setlength{\shadedtextwidth}{\textwidth}
\addtolength{\shadedtextwidth}{-2\shadeboxsep}
\addtolength{\shadedtextwidth}{-2\shadeboxrule}
\setlength{\shadeleftshift}{0pt}
\setlength{\shaderightshift}{0pt}
\linespread{1.05}
\captionnamefont{\small\scshape}
\captiontitlefont{\small}
\newcommand{\mail}[1]{\href{mailto:#1}{\texttt{#1}}}
\definecolor{grigio80}{gray}{0.8}
\definecolor{grigio70}{gray}{0.7}
\hypersetup{%
  pdffitwindow=true,%
  linkcolor=RoyalBlue,%	
%   linkcolor=Black,%	
  linktocpage=true,%
  filecolor=black,%
  urlcolor=RoyalBlue,%
%   urlcolor=Black,%
  plainpages=false,%
  pdftitle={\pdftitolo, \edizione \tipo},%
  pdfauthor=Dimitrios Vrettos,%
  pdfdisplaydoctitle=true%
}
\bookmarksetup{startatroot}
%======================== o l'una o l'altra
% Personalizzazione per matematica dolce
% per: postulato, proposizione, parte, capitolo, numnameref, tggg
% inacessibleblock
%%%%%%%%%%%%%
% Claudoio Carboncini
%%%

\theoremstyle{plain}
% le seguenti due righe ora danno un errore penso vadano semplicemente tolte:
% \newshadetheorem{postulato}{\thmcolor{Postulato}}[chapter]
% \newshadetheorem{proposizione}{\thmcolor{Proposizione}}[chapter]

%%%%%%%%%%%%%
% Daniele Zambelli
%%%

% per 05_02_tarta
\usepackage{listings}             % Include the listings-package
%
% per mantenere allineati i riferimenti presenti
% negli esercizi e nelle soluzioni.
\newcommand{\numnameref}[1]{\ref{#1} \nameref{#1}}
%
% per disegnare il simbolo >>>
% in 05_02_tartaruga.
\newcommand{\tggg}[0]{\textgreater\textgreater\textgreater}

% Inizializza la ``variabile globale'' folder
\newcommand{\folder}{./}

% Crea una nuova parte
\newcommand{\parte}[2]{
  \renewcommand{\folder}{#1}
  \graphicspath{\folder}
  \include{\folder #2}
}

% Crea un nuovo capitolo
\newcommand{\capitolo}[2]{
  \renewcommand{\folder}{#1}
  \graphicspath{{\folder}}
  \include{\folder #2}
  \newpage
  \include{\folder #2_ese}
  \cleardoublepage
}

% Per contrassegnare e sostituire i blocchi inacessibili ai ciechi.
\newenvironment{inaccessibleblock}[1][]{}{}

%========================
% Caratteri sans serif
%========================
% \renewcommand{\familydefault}{\sfdefault}
% %========================
% % Documento
% %========================
% \begin{document}
% \frontmatter
% 
\newcommand*{\frntspz}{%
  \begingroup\newlength{\drop}
  \drop=0.15\textheight
  \vspace{\drop}
  \centering
    \fontsize{16pt}{0in}%
    \selectfont\MakeUppercase\serie\\[0.5\drop]
    \fontsize{26pt}{0pt}%
    \selectfont\MakeUppercase\titolo\par
  \vspace{\drop}
    {\LARGE\descr}\par
  \vspace{2.5\drop}
    \large\editore
  \vskip2mm
    \large\Edizione\ - \anno\par
  \vspace{\drop}
  \endgroup}

\pdfbookmark{frontespizio}{frontespizio}
\pagenumbering{gobble}
% \begin{titlingpage}
 \frntspz
% \end{titlingpage}

% % Copyright (c) 2015 Daniele Zambelli - daniele.zambelli@gmail.com

\pagenumbering{roman}
\thispagestyle{empty}
\pdfbookmark{colophon}{colophon}
{\setlength{\parindent}{0em}\small{
\begin{center}
{\large{\serie – \titolo}}

Copyright {\textcopyright} {\anno} \editore
\end{center}

\begin{wrapfloat}{figure}{I}{0pt}
\includegraphics[width=0.2\columnwidth]{img/by-sa.png}
\end{wrapfloat}

Questo libro, eccetto dove diversamente specificato, è rilasciato nei termini 
della licenza Creative Commons Attribuzione – Condividi allo stesso modo 3.0 
Italia (CC BY-SA 3.0) il cui testo integrale è disponibile al 
sito~\url{http://creativecommons.org/licenses/by-sa/3.0/it/legalcode}.

Tu sei libero:
di riprodurre, distribuire, comunicare al pubblico, esporre in pubblico, 
rappresentare, eseguire e recitare quest'opera, di modificare quest'opera, 
alle seguenti condizioni:

\emph{Attribuzione} --- Devi attribuire la paternità dell'opera nei modi 
indicati dall'autore o da chi ti ha dato l'opera in licenza e in modo tale 
da non suggerire che essi avallino te o il modo in cui tu usi l'opera.

\emph{Condividi allo stesso modo} --- Se alteri o trasformi quest'opera, 
o se la usi per crearne un'altra, puoi distribuire l'opera risultante solo 
con una licenza identica o equivalente a questa.

Per maggiori informazioni su questo particolare regime di diritto d'autore si 
legga il materiale informativo pubblicato su~\url{http://www.copyleft-italia.it}.

\mcpar{Coordinatori del Progetto} \coord .

\mcpar{Autori} \autori.

\mcpar{Hanno Collaborato} \colab.

\mcpar{Progettazione e Implementazione in \LaTeX} {Dimitrios Vrettos}.

\mcpar{Collaboratori} {\texcol}.

\mcpar{Collaborazione, commenti e suggerimenti} Se vuoi contribuire anche tu alla stesura e aggiornamento 
del manuale Matematica \(C^3\) - Algebra 1 o se  vuoi inviare i tuoi commenti e/o suggerimenti scrivi 
a~\mail{daniele.zambelli@istruzione.it}.

\vspace{2ex}
 Versione del documento: {\docvers} del {\oggi}.

 Stampa \edizione : \mese\ \anno.

 ISBN \mcisbn

\vspace{2ex}
 {\scshape{Dati tecnici per l'adozione del libro a scuola}}

 Titolo: \serie, \titolo\ -\edizione.

 Codice ISBN: \mcisbn 

 Editore: \href{http://www.matematicamente.it}{\editore}. 

 Anno di edizione: \anno.

 Prezzo pdf: \officialeuro\ 0,00.

 Formato: ebook (\scshape{pdf}).
}}
% \cleardoublepage

% \include{chap/00_intestazioni/indice}
% \pagestyle{matc3page}
\chapter*{Prefazione}
\addcontentsline{toc}{chapter}{Prefazione}
\markboth{Prefazione}{Prefazione}

Nei convegni di ``Analisi non standard , per le scuole superiori'' è sorta 
l'esigenza di un testo che accompagni l'insegnamento dell'analisi a partire dai 
numeri iperreali.

L'Analisi Non Standard unisce gli aspetti semplici e intuitivi del 
calcolo infinitesimale dei primi due secoli di analisi al rigore dato da 
Abraham Robinson all'insieme dei numeri iperreali.
Permette di iniziare a parlare di argomenti di analisi prime dell'ultimo anno.
Si incominciano a diffondere esperienze che anticipano la trattazione di 
derivate e integrali, l'uso dei numeri iperreali può dare a queste esperienze 
un supporto rigoroso.

L'introduzione dei numeri iperreali e del calcolo infinitesimale nella terza 
superiore può portare alcuni vantaggi didattici:
\begin{itemize} [nosep]
\item permette una maggiore gradualità nell'apprendimento di alcuni fondamenti 
dell'analisi che, se presentati solo all'ultimo anno, 
rischiano di rimanere più superficiali;
\item permette una maggiore integrazione con il corso di fisica, poiché gli 
studenti sono già in grado di utilizzare il concetto di derivata.
\item Concede più tempo in quinta per gli approfondimenti, poiché i concetti di
limite e di derivata in un punto sono già stati affrontati in terza e in quarta.
\end{itemize}

Questo manuale raccoglie il lavoro e l'esperienza di alcuni insegnanti. È 
ancora molto giovane e ha bisogno, per crescere, del contributo di altri 
insegnanti che hanno voglia di sperimentare e di condividere le loro esperienze.

Possiamo considerare questa come una versione \emph{beta} già utilizzabile, tenendo
presente che vi possono essere parti da migliorare e completare. 

Per crescere ha bisogno di:
\begin{itemize} [nosep]
\item osservazioni,
\item critiche,
\item correzioni,
\item aggiunte.
\end{itemize}

Ma un testo vive se è usato e se è in grado di crescere e adattarsi alle 
diverse situazioni. Per questo motivo abbiamo adottato una licenza CC BY-SA che 
permette di:
\begin{itemize} [nosep]
\item scaricare,
\item condividere-duplicare,
\item modificare e ripubblicare,
\item vendere.
\end{itemize}

Con l'unico obbligo che venga mantenuta questa licenza, cioè che anche le 
opere derivate da questa rimangano libere.


Buon divertimento con la matematica!

\begin{flushright}
Bruno Stecca e Daniele Zambelli
\end{flushright}

% \cleardoublepage

% \mainmatter
% %..................................................
% %%
% %% Primo tema: Aritmetica e algebra
% %% --------------------------------
% \parte{chap/01_aritmeticaealgebra/}{part_01_d}
% \capitolo{chap/01_aritmeticaealgebra/01_naturali/}{naturali}
% \capitolo{chap/01_aritmeticaealgebra/02_interi/}{interi}
% \capitolo{chap/01_aritmeticaealgebra/03_razionali/}{razionali}
% \capitolo{chap/01_aritmeticaealgebra/04_altrebasi/}{altrebasi}
% \capitolo{chap/01_aritmeticaealgebra/05_calcololetterale/}{calcololetterale}
% %..................................................
% %%
% %% Secondo tema: Geometria
% %%
% \parte{chap/02_geometria/}{part_02_d}
% \capitolo{chap/02_geometria/01_pianocartesiano/}{pianocartesiano}
% %..................................................
% %%
% %% Terzo tema: Relazioni e funzioni
% %%
% \parte{chap/03_relazioniefunzioni/}{part_03_d}
% \capitolo{chap/03_relazioniefunzioni/01_insiemi/}{insiemi}
% \capitolo{chap/03_relazioniefunzioni/02_equazioni/}{equazioni}
% \capitolo{chap/03_relazioniefunzioni/03_problemi/}{problemi}
% %..................................................
% %%
% %% Quarto tema: Dati e previsioni
% %%
% \parte{chap/04_datieprevisioni/}{part_04_d}
% \capitolo{chap/04_datieprevisioni/01_statistica/}{statistica}
% %..................................................
% %%
% %% Quinto tema: Elementi di informatica
% %%
% \parte{chap/05_informatica/}{part_05_d}
% \capitolo{chap/05_informatica/01_fogliodicalcolo/}{fogliodicalcolo}
% \capitolo{chap/05_informatica/02_tartaruga/}{tartaruga}
% %..................................................
% %%
% %% Azzeramento numerazione capitoli
% %%
% \renewcommand{\thechapter}{\Alph{chapter}}
% \setcounter{chapter}{0}
% 
% \end{document}


%%%%%%%%%%%%%
% Claudoio Carboncini
%%%

\theoremstyle{plain}
% le seguenti due righe ora danno un errore penso vadano semplicemente tolte:
% \newshadetheorem{postulato}{\thmcolor{Postulato}}[chapter]
% \newshadetheorem{proposizione}{\thmcolor{Proposizione}}[chapter]

%%%%%%%%%%%%%
% Daniele Zambelli
%%%

% per 05_02_tarta
\usepackage{listings}             % Include the listings-package
%
% per mantenere allineati i riferimenti presenti
% negli esercizi e nelle soluzioni.
\newcommand{\numnameref}[1]{\ref{#1} \nameref{#1}}
%
% per disegnare il simbolo >>>
% in 05_02_tartaruga.
\newcommand{\tggg}[0]{\textgreater\textgreater\textgreater}

% Inizializza la ``variabile globale'' folder
\newcommand{\folder}{./}

% Crea una nuova parte
\newcommand{\parte}[2]{
  \renewcommand{\folder}{#1}
  \graphicspath{\folder}
  \include{\folder #2}
}

% Crea un nuovo capitolo
\newcommand{\capitolo}[2]{
  \renewcommand{\folder}{#1}
  \graphicspath{{\folder}}
  \include{\folder #2}
  \newpage
  \include{\folder #2_ese}
  \cleardoublepage
}

% Per contrassegnare e sostituire i blocchi inacessibili ai ciechi.
\newenvironment{inaccessibleblock}[1][]{}{}

%========================
% Caratteri sans serif
%========================
\renewcommand{\familydefault}{\sfdefault}
%========================
% Documento
%========================
\begin{document}
\frontmatter
\intestazione{chap/00_intestazioni/les/}{frontespizio}
\intestazione{chap/00_intestazioni/les/}{colophon}
\intestazione{chap/00_intestazioni/les/}{indice}
\intestazione{chap/00_intestazioni/les/}{prefazione}
\mainmatter
%..................................................
%%
%% Primo tema: Aritmetica e algebra
%%
\parte{chap/01_aritmeticaealgebra/}{part_01_2_d}
\capitolo{chap/01_aritmeticaealgebra/05_reali2/}{reali2}
\capitolo{chap/01_aritmeticaealgebra/06_radicali/}{radicali}
%%%
%%% Secondo tema: Geometria
%%%
\parte{chap/02_geometria/}{part_02_2_d}
\capitolo{chap/02_geometria/e03_parallelismo/}{parallelismo}
\capitolo{chap/02_geometria/e04_quadrilateri/}{quadrilateri}
\capitolo{chap/02_geometria/e07_equiestensione/}{equiestensione}
\capitolo{chap/02_geometria/02_retta/}{retta}
\capitolo{chap/02_geometria/03_trasformazioni/}{trasformazioni}
%..................................................
%%%
%%% Terzo tema: Relazioni e funzioni
%%%
\parte{chap/03_relazioniefunzioni/}{part_03_2_d}
\capitolo{chap/03_relazioniefunzioni/05_disequazioni/}{disequazioni}
\capitolo{chap/03_relazioniefunzioni/06_sistemi/}{sistemi}
\capitolo{chap/03_relazioniefunzioni/04_relazioni/}{relazioni}
%%%
%%% Quarto tema: Dati e previsioni
%%%
\parte{chap/04_datieprevisioni/}{part_04_2_d}
\capitolo{chap/04_datieprevisioni/02_probabilita/}{probabilita}
%..................................................
%%%
%%% Quinto tema: Elementi di informatica
%%%
% \parte{chap/05_informatica/}{part_05_2_d}
% \capitolo{chap/05_informatica/01_geointerattiva/}{geointerattiva}
%..................................................
%%
%% Azzeramento numerazione capitoli
%%
\renewcommand{\thechapter}{\Alph{chapter}}
\setcounter{chapter}{0}

\end{document}
