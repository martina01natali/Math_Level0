\pagestyle{matc3page}
\chapter*{Prefazione}
\addcontentsline{toc}{chapter}{Prefazione}
\markboth{Prefazione}{Prefazione}

Nei convegni di ``Analisi non standard , per le scuole superiori'' è sorta 
l'esigenza di un testo che accompagni l'insegnamento dell'analisi a partire dai 
numeri iperreali.

L'Analisi Non Standard unisce gli aspetti semplici e intuitivi del 
calcolo infinitesimale dei primi due secoli di analisi al rigore dato da 
Abraham Robinson all'insieme dei numeri iperreali.
Permette di iniziare a parlare di argomenti di analisi prime dell'ultimo anno.
Si incominciano a diffondere esperienze che anticipano la trattazione di 
derivate e integrali, l'uso dei numeri iperreali può dare a queste esperienze 
un supporto rigoroso.

L'introduzione dei numeri iperreali e del calcolo infinitesimale nella terza 
superiore può portare alcuni vantaggi didattici:
\begin{itemize} [nosep]
\item permette una maggiore gradualità nell'apprendimento di alcuni fondamenti 
dell'analisi che, se presentati solo all'ultimo anno, 
rischiano di rimanere più superficiali;
\item permette una maggiore integrazione con il corso di fisica, poiché gli 
studenti sono già in grado di utilizzare il concetto di derivata.
\item Concede più tempo in quinta per gli approfondimenti, poiché i concetti di
limite e di derivata in un punto sono già stati affrontati in terza e in quarta.
\end{itemize}

Questo manuale raccoglie il lavoro e l'esperienza di alcuni insegnanti. È 
ancora molto giovane e ha bisogno, per crescere, del contributo di altri 
insegnanti che hanno voglia di sperimentare e di condividere le loro esperienze.

Possiamo considerare questa come una versione \emph{beta} già utilizzabile, tenendo
presente che vi possono essere parti da migliorare e completare. 

Per crescere ha bisogno di:
\begin{itemize} [nosep]
\item osservazioni,
\item critiche,
\item correzioni,
\item aggiunte.
\end{itemize}

Ma un testo vive se è usato e se è in grado di crescere e adattarsi alle 
diverse situazioni. Per questo motivo abbiamo adottato una licenza CC BY-SA che 
permette di:
\begin{itemize} [nosep]
\item scaricare,
\item condividere-duplicare,
\item modificare e ripubblicare,
\item vendere.
\end{itemize}

Con l'unico obbligo che venga mantenuta questa licenza, cioè che anche le 
opere derivate da questa rimangano libere.


Buon divertimento con la matematica!

\begin{flushright}
Bruno Stecca e Daniele Zambelli
\end{flushright}

% \cleardoublepage
