%%%%%%%%%%%%%%%%%%%%%%%%%%%%%%%%%%%%%%%%%%%%%%%%%%%%%%%%%%%%%%%%%%%%
%        Matematica dolce
%
% Copyright 2016 Daniele Zambelli
%
%------------------------------
% Matematica dolce per i licei Linguistico e economico sociale, volume 1
%
% m_d_les_5.tex
%------------------------------
%
% This work may be distributed and/or modified under the
% conditions of the LaTeX Project Public License, either version 1.3
% of this license or (at your option) any later version.
% The latest version of this license is in
%   http://www.latex-project.org/lppl.txt
% and version 1.3 or later is part of all distributions of LaTeX
% version 2005/12/01 or later.
%
% This work has the LPPL maintenance status `maintained'.
% 
% The Current Maintainer of this work is 
% Daniele Zambelli - daniele.zambelli@gmail.com
%
% This work consists of the files:
%  -  m_d_les_5.tex (this file)
%  -  createpdf.py
%  -  readme.md
%  -  the part of code of the files under chap/, img/  and lbr/ directories
%%%%%%%%%%%%%%%%%%%%%%%%%%%%%%%%%%%%%%%%%%%%%%%%%%%%%%%%%%%%%%%%%%%%

%========================
% Definizione delle directory
%========================
\newcommand{\basedir}{matematicadolce/}
\newcommand{\depdir}{\basedir deposito/}
\newcommand{\magdir}{\depdir magazzino/}
\newcommand{\matdir}{\depdir materiali/}

\input{\magdir variabili}

% %=============================%
% VARIABILI MATEMATICA DOLCE  %
%=============================%
%
%% Nomi di autori, collaboratori, etc
%%
\newcommand{\coord}{Daniele~Zambelli}

\newcommand{\autori}{
Leonardo~Aldegheri,
Elisabetta~Campana, 
Luciana~Formenti, 
Michele~Perini,
Maria~Antonietta~Pollini, 
Nicola~Sansonetto, 
Andrea~Sellaroli,
Daniele~Zambelli
}
\newcommand{\colab}{
Alberto~Bicego, 
Alberto~Filippini
}
\newcommand{\texcol}{
Claudio~Carboncini, 
Silvia~Cibola, 
Tiziana~Manca,
Daniele~Zambelli
}
%%%%%%% EOF


%========================
% Variabili per questo volume
%========================
\newcommand{\editore}{Matematicamente.it}
\newcommand{\serie}{Matematica $C^3$}
\newcommand{\titolo}{Matematica dolce 5}        
\newcommand{\pdftitolo}{Matematica C3 - Matematica dolce 5}
\newcommand{\docvers}{\texttt{0.0.9}}
\newcommand{\edizione}{ 2016}
\newcommand{\Edizione}{2016 Edizione}
\newcommand{\tipo}{ (versione completa a colori)}
\newcommand{\descr}{Testo per il secondo biennio \protect\\ 
                    della Scuola Secondaria di $II$ grado}
\newcommand{\oggi}{27 giugno 2016}
\newcommand{\mese}{giugno}
\newcommand{\anno}{2016}
\newcommand{\mcisbn}{}

%========================
% Lettura preambolo
%========================
\input{\magdir packages}
\input{\magdir definizioni}

%========================
% Caratteri sans serif
%========================
% \renewcommand{\familydefault}{\sfdefault}
%========================
% Documento
%========================
\begin{document}
\frontmatter
\intestazione{\matdir 00_intestazioni/les/}{frontespizio}
\intestazione{\matdir 00_intestazioni/les/}{colophon}
\intestazione{\matdir 00_intestazioni/les/}{indice}
\intestazione{\matdir 00_intestazioni/les/}{prefazione}
\mainmatter
%..................................................
%% --------------------------------
%% Capitoli
%% --------------------------------
%....... Proposta di Luciana ......................
% 
% Argomenti proposti da Luciana in sintesi
% Funzione esponenziale
% Funzione logaritmica
% Elementi di Topologia
% Limiti
% Funzioni economiche
% Continuità
% Derivate
% Ottimizzazione e studio di funzione
% Integrali

%..................................................
\parte{\matdir 01_aritmeticaealgebra/}{part_01_d}
\capitolo{\matdir 03_relazioniefunzioni/12_funzioni/}{funzioni}
\capitolo{\matdir 01_aritmeticaealgebra/10_insiemi_numerici/}
         {insiemi_numerici}
\capitolo{\matdir 03_relazioniefunzioni/14_differenziazione/}
         {differenziazione}
\capitolo{\matdir 03_relazioniefunzioni/15_funzionicontinue/}
         {funzionicontinue}
\capitolo{\matdir 03_relazioniefunzioni/16_studiofunzioni/}
         {studiofunzioni}
\capitolo{\matdir 04_datieprevisioni/07_modellieconomici/}
         {modellieconomici}
\capitolo{\matdir 03_relazioniefunzioni/17_integrazione/}
         {integrazione}
%..................................................
%%
%% Azzeramento numerazione capitoli
%%
\renewcommand{\thechapter}{\Alph{chapter}}
\setcounter{chapter}{0}

\end{document}
