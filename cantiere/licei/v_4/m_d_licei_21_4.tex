%%%%%%%%%%%%%%%%%%%%%%%%%%%%%%%%%%%%%%%%%%%%%%%%%%%%%%%%%%%%%%%%%%%%
%        Matematica dolce
%
% Copyright 2016-22 Daniele Zambelli
%
%------------------------------
% Matematica dolce per i licei, volume 4
%
% Compilation:
% pdflatex --shell-escape m_d_licei_21_4.tex
%------------------------------
%
% This work may be distributed and/or modified under the
% conditions of the LaTeX Project Public License, either version 1.3
% of this license or (at your option) any later version.
% The latest version of this license is in
%   http://www.latex-project.org/lppl.txt
% and version 1.3 or later is part of all distributions of LaTeX
% version 2005/12/01 or later.
%
% This work has the LPPL maintenance status `maintained'.
% 
% The Current Maintainer of this work is 
% Daniele Zambelli - daniele.zambelli@gmail.com
%
% This work consists of the files:
%  -  m_d_licei_21_4.tex (this file)
%  -  createpdf.py
%  -  readme.md
%  -  the part of code of the files under chap/, img/  and lbr/ directories
%%%%%%%%%%%%%%%%%%%%%%%%%%%%%%%%%%%%%%%%%%%%%%%%%%%%%%%%%%%%%%%%%%%%

%========================
% Definizione delle directory
%========================
% \newcommand{\basedir}{matematicadolce/}
\newcommand{\depdir}{deposito/}
\newcommand{\magdir}{\depdir magazzino/}
\newcommand{\matdir}{\depdir materiali/}

%========================
% Classe del documento
%========================
\documentclass[10pt,a4paper,openright]{\magdir matsweetmem}
% Per la versione in scala di grigio commentare la precedente e
% decommentare la riga seguente
% \documentclass[10pt,a4paper,openright,gray]{\magdir matsweetmem}

%========================
% Variabili del progetto
%========================
\input{\magdir variabili}

%========================
% Variabili per questo volume
%========================
\newcommand{\editore}{Matematicamente.it}
\newcommand{\serie}{Matematica $C^3$}
\newcommand{\titolo}{Matematica dolce 4 - licei}
\newcommand{\pdftitolo}{Matematica dolce 4 - licei}
\newcommand{\docvers}{\texttt{8.0.0}}
\newcommand{\edizione}{2021}
\newcommand{\Edizione}{Edizione}
\newcommand{\tipo}{}% (versione completa a colori)}
\newcommand{\descr}{Testo per il secondo biennio \protect\\ 
                    della Scuola Secondaria di $II$ grado \\ \null 
                    licei\\[2em]
                    \emph{Versione accessibile}}
\newcommand{\oggi}{27 agosto 2021}
\newcommand{\mese}{giugno}
\newcommand{\anno}{2021}
\newcommand{\mcisbn}{9788899988036}

%========================
% Lettura preambolo
%========================
\input{\magdir packages}
\input{\magdir definizioni}
\input{\magdir definizioni_tikz}

%========================
% Con o senza linguaggio di programmazione
%========================
\newif\ifcoding
\codingtrue     % commentare questa linea se NON si vuole il coding
% \codingfalse    % commentare questa linea se si vogliono le parti di coding

%========================
% Documento
%========================
\begin{document}
\frontmatter

\newcommand*{\frntspz}{%
  \begingroup\newlength{\drop}
  \drop=0.15\textheight
  \vspace{\drop}
  \centering
    \fontsize{16pt}{0in}%
    \selectfont\MakeUppercase\serie\\[0.5\drop]
    \fontsize{26pt}{0pt}%
    \selectfont\MakeUppercase\titolo\par
  \vspace{\drop}
    {\LARGE\descr}\par
  \vspace{2.5\drop}
    \large\editore
  \vskip2mm
    \large\Edizione\ - \anno\par
  \vspace{\drop}
  \endgroup}

\pdfbookmark{frontespizio}{frontespizio}
\pagenumbering{gobble}
% \begin{titlingpage}
 \frntspz
% \end{titlingpage}

% Copyright (c) 2015 Daniele Zambelli - daniele.zambelli@gmail.com

\pagenumbering{roman}
\thispagestyle{empty}
\pdfbookmark{colophon}{colophon}
{\setlength{\parindent}{0em}\small{
\begin{center}
{\large{\serie – \titolo}}

Copyright {\textcopyright} {\anno} \editore
\end{center}

\begin{wrapfloat}{figure}{I}{0pt}
\includegraphics[width=0.2\columnwidth]{img/by-sa.png}
\end{wrapfloat}

Questo libro, eccetto dove diversamente specificato, è rilasciato nei termini 
della licenza Creative Commons Attribuzione – Condividi allo stesso modo 3.0 
Italia (CC BY-SA 3.0) il cui testo integrale è disponibile al 
sito~\url{http://creativecommons.org/licenses/by-sa/3.0/it/legalcode}.

Tu sei libero:
di riprodurre, distribuire, comunicare al pubblico, esporre in pubblico, 
rappresentare, eseguire e recitare quest'opera, di modificare quest'opera, 
alle seguenti condizioni:

\emph{Attribuzione} --- Devi attribuire la paternità dell'opera nei modi 
indicati dall'autore o da chi ti ha dato l'opera in licenza e in modo tale 
da non suggerire che essi avallino te o il modo in cui tu usi l'opera.

\emph{Condividi allo stesso modo} --- Se alteri o trasformi quest'opera, 
o se la usi per crearne un'altra, puoi distribuire l'opera risultante solo 
con una licenza identica o equivalente a questa.

Per maggiori informazioni su questo particolare regime di diritto d'autore si 
legga il materiale informativo pubblicato su~\url{http://www.copyleft-italia.it}.

\mcpar{Coordinatori del Progetto} \coord .

\mcpar{Autori} \autori.

\mcpar{Hanno Collaborato} \colab.

\mcpar{Progettazione e Implementazione in \LaTeX} {Dimitrios Vrettos}.

\mcpar{Collaboratori} {\texcol}.

\mcpar{Collaborazione, commenti e suggerimenti} Se vuoi contribuire anche tu alla stesura e aggiornamento 
del manuale Matematica \(C^3\) - Algebra 1 o se  vuoi inviare i tuoi commenti e/o suggerimenti scrivi 
a~\mail{daniele.zambelli@istruzione.it}.

\vspace{2ex}
 Versione del documento: {\docvers} del {\oggi}.

 Stampa \edizione : \mese\ \anno.

 ISBN \mcisbn

\vspace{2ex}
 {\scshape{Dati tecnici per l'adozione del libro a scuola}}

 Titolo: \serie, \titolo\ -\edizione.

 Codice ISBN: \mcisbn 

 Editore: \href{http://www.matematicamente.it}{\editore}. 

 Anno di edizione: \anno.

 Prezzo pdf: \officialeuro\ 0,00.

 Formato: ebook (\scshape{pdf}).
}}
% \cleardoublepage

\intestazione{\matdir 00_intestazioni/les/}{indice}
\intestazione{\matdir 00_intestazioni/les/}{prefazione}

%..................................................
%% --------------------------------
%% Capitoli
%% --------------------------------
%....... Proposta di Luciana ......................
% Complementi di algebra (irrazionali, valori assoluti)
% Ellisse
% Iperbole
% Funzioni esponenziale e logaritmica
% Calcolo approssimato
% Calcolo combinatorio
% Probabilità
% Modelli
% Geometria euclidea nello spazio
%..................................................
\mainmatter
% \parte{\matdir 0_a_parti/}{part_01_d}
% \capitolo{\matdir 04_Equazioni/06_Grado_2+/compl201/}
%          {grasup2_irraz_valass}
% \capitolo{\matdir 05_Geometria_euclidea/03_Circonferenza/circ01/}
%          {circonferenza1}
% \capitolo{\matdir 05_Geometria_euclidea/04_Equiestensione/prop01/}
%          {proporzionalita1}
% \capitolo{\matdir 07_Geometria_analitica/04_Circonferenza/circ01/}
%         {circonferenza_pc}
% \capitolo{\matdir 07_Geometria_analitica/05_Ellisse_iperbole/ell01/}
%          {ellisse}
% \capitolo{\matdir 07_Geometria_analitica/05_Ellisse_iperbole/iper01/}
%          {iperbole}
%  \capitolo{\matdir 07_Geometria_analitica/06_Luoghi_geometrici/luo01/}
%           {luoghigeometrici}
 \capitolo{\matdir 10_Esp_log/01_Funz_eq_diseq/esplog01/}
         {esp_log}
% TODO \capitolo{\matdir 01_aritmeticaealgebra/11_calc_approssimato/}
%               {calc_approssimato }
% \capitolo{\matdir 15_Probabilita/01_Calcolo_combinatorio/comb01/}
%          {calc_combinatorio}
% \capitolo{\matdir 15_Probabilita/02_Calcolo_probabilita/prob02/}
%           {probabilita}
% \capitolo{\matdir 16_Economia/01_Matematica_finanziaria/finanz01/}
%          {finanziaria}
% \capitolo{\matdir 07_Geometria_analitica/08_Geometria_3d/g3d01/}
%          {geometria3d}
%..................................................
%
%% Azzeramento numerazione capitoli
%%
\renewcommand{\thechapter}{\Alph{chapter}}
\setcounter{chapter}{0}

\end{document}
