\documentclass[a4paper,10pt]{article}
\usepackage[utf8]{inputenc}
\usepackage[T1]{fontenc} 
\usepackage{textcomp} 	
\usepackage[italian]{babel} 
\usepackage{lmodern}

%opening
\title{Matematica Dolce\\ 
un libro di testo fuori dalla moda}
\author{Daniele Zambelli}

\begin{document}

\maketitle

\begin{abstract}
Da quasi una decina di anni ha incominciato a evolversi un libro di testo di 
matematica per le scuole superiori fuori dalla norma.

Il testo non è ``bello'', non è ``completo'', contiene errori;
ciò nonostante può essere uno strumento utile per quegli insegnanti che
\begin{itemize}
\item oltre a insegnare amano imparare;
\item cercano il confronto con altri;
\item sono attenti alle cause delle difficoltà di apprendimento dei propri 
alunni;
\item insegnano certi argomenti con metodi che non sono supportati dalle case 
editrici;
\item si prendono cura dei propri strumenti di lavoro;
\item sono critici nei confronti dei libri;
\item \dots
\end{itemize}

In questo intervento verrà brevemente illustrato un progetto:
\begin{itemize}
\item libero,
\item collaborativo,
\item evolutivo,
\item polimorfo,
\item accessibile.
\end{itemize}

\end{abstract}

\section{Libertà}

\section{Collaborazione}

\section{Evoluzione}

\section{Polimorfismo}

\section{Accessibilità}

\end{document}
