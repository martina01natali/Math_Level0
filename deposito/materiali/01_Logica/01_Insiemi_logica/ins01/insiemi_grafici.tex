% (c) 2014 Daniele Zambelli - daniele.zambelli@gmail.com
% 
% Tutti i grafici per il capitolo relativo agli iperreali
%


\newcommand{\trespolointervalli}[4]{% Grafo intervalli con enne assi
  % Esempio di chiamata:
% \disegno{
%   \trespolointervalli{9}{-6}
%                      {-1/\(A\), -2/\(B\), -3/\(C\), 
%                       -4/\(A \cup B\), -5/\(B \cap C\), -6/\(A - B\)}
%                      {1.5/-5, 3/-2, 4.5/3, 6/6, 7.5/9}
% }
  \def \dimx{#1}
  \def \dimy{#2}
  \def \eassi{#3}
  \def \epunti{#4}
   \foreach \posy/\ea in {#3}{
     \node at (0, \posy) [left] {\ea};
     \assev{0}{\posy}{\dimx}{\posy}
%      \assex{0}{5}{\posy}
   }
   \foreach \i/\e in {#4}
     \draw (\i, \dimy) -- (\i, -0.5) node [above] {\e};
}

\newcommand {\grafooperazioni}{% Illustarzione del grafo per operazioni
                              % tra intervalli
\disegno{
  \trespolointervalli{9}{-6}
                     {-1/\(A\), -2/\(B\), -3/\(C\), 
                      -4/\(A \cup B\), -5/\(B \cap C\), -6/\(A - B\)}
                     {1.5/\(-1\), 3/\(+1\), 4.5/\(+4\), 6/\(+6\), 
                      7.5/\(+9\)}
  \rayl{-1}{0}{6}{}{white}
  \inti{-2}{1.5}{4.5}{}{}{black}{white}
  \inti{-3}{3}{7.5}{}{}{white}{black}
  \rayl{-4}{0}{6}{}{white}
  \inti{-5}{3}{4.5}{}{}{white}{white}
  \rayl{-6}{0}{1.5}{}{white}  \inti{-6}{4.5}{6}{}{}{black}{white}
  \def \taba{14}
  \def \tabb{22}
  \node at (\taba, -1) {\(x < +6\)}; 
  \node at (\tabb, -1) {\(\intervaa{-\infty}{+6}\)};
  \node at (\taba, -2) {\(-1 \le x < +4\)}; 
  \node at (\tabb, -2) {\(\intervca{-1}{+4}\)};
  \node at (\taba, -3) {\(+1 < x \le +9\)}; 
  \node at (\tabb, -3) {\(\intervac{+1}{+9}\)};
  \node at (\taba, -4) {\(x < +6\)}; 
  \node at (\tabb, -4) {\(\intervaa{-\infty}{+6}\)};
  \node at (\taba, -5) {\(+1 < x < +4\)}; 
  \node at (\tabb, -5) {\(\intervaa{+1}{+4}\)};
  \node at (\taba, -6) {\(x < -1 \sor +4 \le x < +6\)}; 
  \node at (\tabb, -6) {\(\intervaa{-\infty}{-1} \cup \intervca{+4}{+6}\)};
}
}
