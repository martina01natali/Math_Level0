% (c) 2015 Daniele Zambelli daniele.zambelli@gmail.com

\input{\folder iperreali_grafici.tex}

\section{I numeri iperreali $\IR$}
% \section{I numeri iperreali}
\label{sec:insnum_iperreali}

In questa sezione verranno presentate le caratteristiche dei numeri 
iperreali e li useremo per modellizzare e risolvere nuove classi di 
problemi. 
La conoscenza di questi nuovi numeri non è molto diffusa neanche fra i 
matematici, perché sono abituati da più di un secolo e mezzo a procedimenti 
più impegnativi e sofisticati.
La ragione per la quale noi ne faremo uso è che questi nuovi numeri 
rendono certi concetti più semplici e immediati, senza per questo nuocere al 
rigore e alla precisione dei ragionamenti.

% Rispetto a quanto già sappiamo dell'insieme \(\R\), dovremo adattare alcune 
% regole di calcolo e riscontreremo proprietà nuove, mentre dovremo 
% abbandonarne, per ora, una conosciuta: il postulato di Eudosso-Archimede e con 
% lui la proprietà di completezza. 
% 
% {\noindent
% \begin{minipage}{.28\textwidth}
% La scelta di completare gli \emph{intervalli infinitamente piccoli} 
% riempiendoli con \emph{un solo numero} è molto comoda, ma non è l'unica 
% possibile. 
% Vedremo che in quello spazio infinitamente piccolo si 
% possono far stare infiniti numeri conservando quasi tutte le proprietà dei 
% numeri reali. 
% \end{minipage}
% \hfill
% \begin{minipage}{.70\textwidth}
% \begin{inaccessibleblock} 
% [Intervallo infinitesimo che contiene molti numeri.] 
% \begin{center}
% \molticonnome 
% {\footnotesize Nell'intervallo infinitesimo ci sono molti numeri.} 
% \end{center}
% \label{fig:mandelbrot} 
% \end{inaccessibleblock} 
% \end{minipage}
% }
% 
% Inventeremo così un nuovo insieme numerico e un nuova retta: 
% perderemo la proprietà di completezza, ma potremo usare dei numeri fantastici. 

\subsection{Il problema della velocità}
\label{subsec:insnum_velocita}

Alla fine del 1600 Newton e Leibniz studiavano problemi legati alla 
meccanica. 
Una delle grandezze alla base della meccanica è la \emph{velocità}. 
Ma cosa è la velocità? 
Se l'oggetto A percorre più strada dell'oggetto B possiamo dire 
che A è più veloce di B? No, non basta misurare lo spazio percorso da un 
oggetto per calcolare la sua velocità, bisogna anche misurare il tempo 
impiegato a percorrere quello spazio. Infatti sappiamo che:
\[\text{velocità} = 
  \frac{\text{spazio percorso}}{\text{tempo impiegato}}\]
La grandezza calcolata in questo modo è \emph{la velocità media} 
dell'oggetto, ma in ogni istante del percorso l'oggetto ha una propria 
velocità. 
Come faccio a calcolarla? Potrei misurare lo spazio percorso in un tempo 
molto piccolo, in questo modo avrò una velocità media tenuta in un percorso 
molto breve\dots ma resta sempre una velocità media. 

Per trovare la velocità istantanea dovrei dividere lo spazio percorso per 
un tempo (positivo) più piccolo di qualunque numero. L'unico numero reale
più piccolo, in valore assoluto, di qualunque numero è lo zero, ma 
non posso usarlo per il calcolo della velocità, perché la divisione per 
zero non è definita: i numeri reali non ci permettono di calcolare una 
grandezza così semplice e evidente come la velocità di un oggetto in un 
dato istante.

% Servirebbe un insieme numerico con numeri positivi più piccoli di un 
% qualsiasi altro numero positivo, ma diversi da zero! Ma è possibile 
% trovare tali numeri nell'insieme dei reali che, come abbiamo visto, 
% è un insieme (già) completo?

\subsection{Infinitesimi... e infiniti}
\label{subsec:insnum_nonarchimedei}

Abbiamo visto che se una sezione di razionali individua 
\emph{un solo nuovo numero}, 
otteniamo l'insieme dei numeri reali: \(\R\), se invece individua 
\emph{più numeri} otteniamo i numeri iperreali: \(\IR\).

Abbiamo anche visto che la distanza tra due di questi numeri è un numero 
inferiore a \(\frac{1}{n}\) per ogni \(n\) naturale diverso da zero.
Quindi la differenza tra due di questi numeri è un  numero, in valore 
assoluto, più piccolo di 
\[\frac{1}{n} \quad \forall~ n \in \Nz\]
I numeri di questo tipo sono dei numeri nuovi, nessuno dei numeri che 
abbiamo conosciuto finora ha questa caratteristica.

\begin{definizione}
Chiamiamo \emph{infinitesimo} un numero che, in valore assoluto, è minore di
\(\frac{1}{n}\) per ogni \(n\) naturale diverso da zero:
\[\epsilon \stext{è un infinitesimo se } 
\abs{\epsilon} < \frac{1}{n} \quad \forall~ n \in \Nz\]
\end{definizione}
% Se accettiamo che possa \textbf{non} valere il postulato di 
% Eudosso-Archimede, possiamo costruire un insieme numerico non archimedeo. 
% Per farlo, possiamo aggiungere all'insieme dei numeri reali un nuovo numero 
% (non reale) maggiore di zero ma più piccolo di qualunque numero reale 
% positivo:
% \[\epsilon > 0 \quad \text{ tale che } \quad 
% \epsilon < \frac{1}{n} \quad \text{ per qualunque } n \in \N\]
% tradotto in simboli:
% \[\exists \epsilon > 0 \quad | \quad \epsilon < \frac{1}{n} \quad \forall n 
% \in \N\]
% Un numero siffatto lo chiameremo un \emph{infinitesimo} e lo indicheremo 
% con una lettera minuscola dell'alfabeto greco, per esempio \(\epsilon\).
% Per quanto è già stato detto, un tale numero non può essere un numero
% reale.
\begin{osservazione}
 In un insieme che contenga numeri infinitesimi non vale il postulato di 
Eudosso-Archimede infatti se
\(\epsilon < \dfrac{1}{n} \quad \forall n \in \Nz\) 
moltiplicando entrambi i membri per \(n\) 
si ottiene: \(n \epsilon < 1\).

Quindi non si può ottenere un multiplo di un infinitesimo che sia maggiore di 
un numero grande quanto si vuole.
\end{osservazione}

\vspace{.5em}
La prima conseguenza dell'introduzione di un infinitesimo è che allora ce 
ne sono infiniti! 
Infatti anche la metà di un infinitesimo è un infinitesimo e 
sono infinitesimi anche il suo doppio o un suo sottomultiplo o un suo 
multiplo.

Altra conseguenza dell'aggiunta di numeri infinitesimi è che, se si possono 
fare le normali operazioni con questi nuovi numeri, allora esiste anche un 
numero maggiore di qualunque numero reale:
\[\text{se} \quad \epsilon < \frac{1}{n} \quad \forall n \in \Nz 
\quad \text{allora} \quad \frac{1}{\epsilon} > n \quad \forall n 
\in \Nz\]
Quindi se abbiamo dei numeri infinitesimi, e 
possiamo usarli nelle usuali 4 operazioni, allora avremo anche: 
\begin{itemize} [nosep]
\item un numero infinito di infinitesimi,
\item un numero infinito di infiniti.
\end{itemize}
Chiamiamo \emph{iperreali} i numeri che si ottengono combinando i reali con 
gli infinitesimi, e quindi anche con gli infiniti. 
Indichiamo l'insieme dei \emph{numeri iperreali} con il simbolo:~\(\IR\) 
(``erre star'').

\subsection{Tipi di iperreali}
\label{subsec:insnum_iperreali}

Abbiamo visto che l'introduzione di un elemento nuovo, così piccolo
da poterlo pensare trascurabile, ha reso piuttosto affollato il nuovo 
insieme numerico. 
Cerchiamo di fare un po' di ordine. 
L'insieme degli Iperreali contiene diversi tipi di numeri li vediamo qui di 
seguito. \\[.5em]

\affiancati{.69}{.29}{
\begin{description} [noitemsep]
 \item \textbf{Infinitesimi}:
numeri che, in valore assoluto, sono minori di qualunque numero reale 
positivo.
 \item \textbf{Infiniti}:
numeri che, in valore assoluto, sono maggiori di qualunque numero reale.
 \item \textbf{Zero}:
l'unico numero reale infinitesimo.
 \item \textbf{Infinitesimi non nulli}:
numeri infinitesimi, escluso lo zero.
 \item \textbf{Finiti}:
numeri che non sono infiniti.
 \item \textbf{Finiti non infinitesimi}:
numeri che non sono né infiniti né infinitesimi.
\end{description}
}{
\begin{center} \scalebox{0.7}{\iperrealiset} \end{center}
} \\[1em]
Per semplificare la scrittura (e complicare la lettura) adotteremo delle 
sigle e delle convenzioni per indicare questi diversi tipi di numeri:

\begin{center}
\begin{tabular}{ccc}\toprule
tipo & sigla & simboli \\\midrule

zero &  & 0 \\

infinitesimi & \emph{i} & 
\(\alpha, \beta, \gamma, \delta, \dots\) \\

infinitesimi non nulli & \emph{inn} & \\

finiti non infinitesimi & \emph{fni} & \(a, b, c, d, \dots\)\\

finiti & \emph{f} & \(a, b, c, d, \dots\) \\

infiniti & \emph{I} & \(A, B, C, \dots\)\\

qualsiasi &  & \(x, y, z, \dots\) \\\bottomrule
\end{tabular}
\label{tab:insnum_tipi}
\end{center}

\newpage %--------------------------------------------------

\begin{esempio}
 Individua il tipo delle seguenti espressioni:
%  (considerando, per semplicità \(\epsilon\) positivo):

\begin{multicols}{4}
\begin{enumerate} [nosep]
 \item \raisebox{+0mm}[4.8mm]{}\(\pi+\epsilon\)
 \item \raisebox{+0mm}[4.8mm]{}\(4\epsilon+\epsilon \cdot \delta\)
 \item \raisebox{+0mm}[4.8mm]{}\(M-7\)
 \item \raisebox{+0mm}[4.8mm]{}\(M+\dfrac{1}{\epsilon}\)
\end{enumerate}
\end{multicols}

Vediamo i vari casi:

\begin{enumerate}
 \item \(\pi+\epsilon\): 
è un numero finito perché \(\pi\) è un numero 
finito (\(3,141592653589793\dots\)) con infinite cifre decimali, ma 
\(\epsilon\) è più piccolo della più piccola cifra di \(\pi\) che possiamo 
pensare quindi aggiungere un infinitesimo ad un numero reale non cambia il 
numero reale che rimane un numero finito, in questo caso non infinitesimo.
 \item  \(4\epsilon+\epsilon \cdot \delta\): 
qui abbiamo la somma di due quantità, 
la prima è formata da 4 infinitesimi, ma per come abbiamo definito 
l'infinitesimo, anche 4 infinitesimi sono un numero infinitesimo; la 
seconda è formata dal prodotto di un infinitesimo per un altro infinitesimo 
che indica quindi un infinitesimo di un infinitesimo che è un infinitesimo 
ancora più infinitesimo di ciascuno dei due. La loro somma quindi è un 
infinitesimo con lo stesso segno di \(\epsilon\).
 \item \(M-7\):
possiamo distinguere due casi: 
 \begin{itemize} [noitemsep]
  \item se \(M\) è un infinito negativo, allora 
\(M-7\) sarà un numero in valore assoluto ancora più grande, 
 \item se \(M\) è un infinito positivo, \(M-7\) sarà numero più piccolo di 
\(M\) ma che non può essere un numero finito. 
Infatti, supponiamo che \(M-7\) sia un numero finito, chiamiamolo \(x\); ma 
se \(x\) è finito allora anche \(x+7\) è finito e questo avrebbe come 
conseguenza che anche \(M\) sia finito contraddicendo le nostre convenzioni.
 \end{itemize}
 \item \(M+\dfrac{1}{\epsilon}\):
 in questo caso dobbiamo fare una distinzione:
 \begin{itemize} [noitemsep]
  \item se \(M\) e \(\epsilon\) hanno lo stesso segno il calcolo precedente 
equivale a sommare due infiniti entrambi positivi (o negativi) e darà come 
risultato un infinito positivo (o negativo).
  \item se \(M\) e \(\epsilon\) hanno segni diversi bisogna avere più 
informazioni per poter stabilire il tipo del risultato.
 \end{itemize}

\end{enumerate}
\end{esempio}

\subsection{Numeri infinitamente vicini}
\label{subsec:insnum_infinitamentevicini}

% Bruno: inserire anche osservazioni sugli infiniti infinitamente vicini?

Nei numeri reali, due numeri o sono uguali o sono diversi (ovviamente).
Nel primo caso, la differenza tra i due numeri è zero, 
nel secondo, la differenza è un numero reale diverso da zero.

Negli iperreali, se due numeri sono diversi, la loro differenza può 
essere un numero infinito, un numero finito non infinitesimo o un numero 
infinitesimo.

\begin{esempio}
 Calcola la distanza tra \(a=7+\epsilon\) e \(b=10-5\epsilon\):
 
 \(\abs{b-a}=\abs{\tonda{10+\epsilon} - \tonda{7-5\epsilon}}=
   \abs{10+\epsilon - 7+5\epsilon}=
   \abs{10-7 + \epsilon+5\epsilon}=
   \abs{3 + 6\epsilon}
 \)
 
La distanza tra \(a\) e \(b\) è uguale a~3 più un infinitesimo 
(\emph{fni}).
\end{esempio}

\begin{esempio}
 Calcola la distanza tra \(a=5+\epsilon\) e \(b=5+\delta\):
 
 \(\abs{b-a}=\abs{\tonda{5+\delta} - \tonda{5+\epsilon}}=
   \abs{5+\delta - 5-\epsilon}=
   \abs{5-5 + \delta - \epsilon}=
   \abs{0 + \delta - \epsilon} = \abs{\gamma}
 \)
 
 In questo caso la distanza tra \(a\) e \(b\) è un infinitesimo 
(\emph{i}).
\end{esempio}

\noindent Negli iperreali la differenza tra due numeri può essere:

\begin{itemize} [noitemsep]
 \item un infinito;
 \item un finito non infinitesimo;
 \item un infinitesimo (zero se i due numeri sono uguali).
\end{itemize}

% \noindent Negli iperreali possiamo distinguere:
% 
% \begin{itemize} [noitemsep]
%  \item \(a\) e \(b\) sono uguali: \(b-a=0\);
%  \item \(a\) e \(b\) sono diversi, in questo caso possiamo distinguere
%  ulteriormente:
% \begin{itemize} [nosep]
%  \item \(a-b\) è un finito non infinitesimo;
%  \item \(a-b\) è un infinitesimo non nullo.
% \end{itemize}
% \end{itemize}

Quando la differenza di due numeri è un infinitesimo, diciamo 
che i due numeri sono \emph{infinitamente vicini}.

\begin{definizione}    % [Infinitamente vicini]
Due numeri si dicono \textbf{infinitamente vicini} (simbolo:~\(\approx\)) se 
la loro differenza è un infinitesimo:
\[x \approx y \Leftrightarrow x - y = \epsilon\]
\end{definizione}

Tutti gli infinitesimi sono infinitamente vicini tra di loro e sono 
infinitamente vicini a zero.

Due numeri infinitamente vicini, sono diversi tra di loro, ma la loro 
differenza è minore di qualunque numero reale positivo.

Dato un qualunque numero iperreale, possiamo considerare l'insieme di tutti 
i numeri infinitamente vicini a questo. 

\begin{definizione}
 Si chiama \emph{monade} del numero \(x\) l'insieme formato da tutti i 
numeri iperreali infinitamente vicini a \(x\).
\end{definizione}

Due monadi diverse non hanno elementi in comune.

Una monade può contenere al massimo un numero reale, due numeri reali 
diversi appartengono a monadi diverse.

\subsection{Iperreali finiti e parte standard}
\label{subsec:insnum_partestandard}

Tra i vari tipi di iperreali, hanno un ruolo particolare gli iperreali 
finiti perché sono quelli che assomigliano di più ai numeri che già 
conosciamo e possono essere facilmente tradotti in numeri reali e 
approssimati con numeri razionali. 

\begin{definizione}
 Un numero iperreale si dice \textbf{finito} se è compreso tra 
due numeri reali:

\[\text{se } x \in \IR \stext{e}  a < x < b \stext{con} 
  a, b \in \R \quad \text{ allora } x \text{ è finito.}\]
\end{definizione}

\begin{esempio}
 Individua quali dei seguenti numeri sono finiti (considerando, per 
semplicità \(\epsilon\) positivo):

\begin{multicols}{3}
\begin{enumerate}
 \item \(8+5\epsilon\)
 \item \(\tonda{8+5\epsilon}^2\)
 \item \(8+\dfrac{5}{\epsilon}\)
\end{enumerate}
\end{multicols}

Vediamo i tre casi:

\begin{enumerate}
 \item \(\tonda{8+5\epsilon}\) è un numero finito perché:
 \[\tonda{8-1} < \tonda{8+5\epsilon} < \tonda{8+1}\]
 \item Eseguiamo il quadrato:
 \(\tonda{8+5\epsilon}^2 = 64 +80 \epsilon +25 \epsilon^2\)
 ma: \(80 \epsilon\) è sicuramente un infinitesimo e anche \(25 \epsilon^2\)
 lo è e sarà un infinitesimo anche la loro somma, chiamiamo \(\delta\) 
questa somma quindi: 
\(\tonda{8+5\epsilon}^2 = 64 + \delta\)
e: 
\[\tonda{64 -1} < \tonda{64 + \delta} < \tonda{64 +1}\]
 \item Nell'ultimo caso possiamo osservare che, essendo \(\epsilon\) in 
valore assoluto minore di qualunque numero reale, 
 \(\dfrac{5}{\epsilon}\) è un numero maggiore di qualunque numero reale e 
la somma di~8 più un numero maggiore di qualunque altro, non può essere 
minore di un determinato numero reale, 
%  \[\nexists y \in \R \quad | \quad 8+\dfrac{5}{\epsilon} < y\]
perciò \(8+\dfrac{5}{\epsilon}\) è un numero infinito.
\end{enumerate}
\end{esempio}

\noindent Ogni numero \emph{finito} può essere visto come un numero 
\emph{reale} più un \emph{infinitesimo}.

Se \(x\) è finito allora \(x = a + \epsilon\) dove:
\begin{itemize} [noitemsep]
 \item \(x\) è un numero iperreale finito;
 \item \(a\) è un numero reale;
 \item \(\epsilon\) è un infinitesimo (anche zero).
\end{itemize}

Se \(x=a+\epsilon\) allora potremmo dire che \(x\) è infinitamente vicino 
ad \(a\) infatti la differenza tra i due dà un infinitesimo: 
\[x=a+\epsilon \sLRarrow
x-a = a+\epsilon-a \sLRarrow x-a = \epsilon \sLRarrow x \approx a\]
Un numero iperreale finito non può essere infinitamente vicino 
a due numeri reali diversi (perché?).
Quindi possiamo creare una funzione che ha come insieme di definizione 
i numeri iperreali finiti e come insieme immagine i numeri reali.
Chiameremo ``parte standard'' la funzione che associa ad ogni 
iperreale finito il numero reale a lui infinitamente vicino.

% \begin{definizione}
%  La parte standard di un numero iperreale finito è l'unico numero reale 
% infinitamente vicino:
% 
% \[\text{Se } x \in \IR \wedge x= a+\epsilon \text{ allora } 
% \st(x) = a \text{ (st(x) = parte standard).}\]
% \end{definizione}

\begin{definizione}
 Si dice che \(a\) è la \textbf{parte standard} di \(x\), e si scrive: 
 \(\st(x) = a\), se \(a\) è un numero reale e \(x\) è un numero iperreale
infinitamente vicino ad \(a\):
\[\st(x) = a \sLRarrow a \in \R \stext{e} x \approx a\mbox{, con }x\in \IR\]
\end{definizione}
\begin{osservazione}
\begin{itemize} [nosep]
 \item 
La parte standard di un reale è il reale stesso.
 \item 
La parte standard di un infinitesimo è zero infatti:
\(\epsilon = 0+\epsilon\).
 \item 
Un numero iperreale infinito non ha parte standard poiché non esiste nessun 
numero reale infinitamente vicino a un infinito.
\end{itemize}
\end{osservazione}
\vspace{1em}
La Parte Standard presenta alcune proprietà che possono essere facilmente 
dimostrate. Nel seguito \(x,\ y\) rappresentano iperreali finiti 
(poiché gli infiniti non hanno parte standard):

\begin{multicols}{2}
\begin{enumerate} [nosep]
 \item \raisebox{+0mm}[4.9mm]{}\(\pst{-x} = -\pst{x}\)
 \item \raisebox{+0mm}[5.5mm]{}\(\pst{x \pm y} = \pst{x} \pm \pst{y}\)
 \item \raisebox{+0mm}[5.5mm]{}\(\pst{xy} = \pst{x} \cdot \pst{y}\)
 \item \(se ~ \pst{y} \neq 0 ~ allora ~
 \pst{\frac{x}{y}} = \dfrac{\pst{x}}{\pst{y}}\)
 \item \(\pst{\sqrt[n]{x}} = \sqrt[n]{\pst{x}}\)
 \item \raisebox{+0mm}[5.5mm]{}\(se ~  x < y ~ allora ~ 
 \pst{x} \leqslant \pst{y}\).
\end{enumerate}
\end{multicols}

Ne dimostriamo alcune, chiamando \quad  \(a=\pst{x} \stext{ e } b=\pst{y}\).
\begin{enumerate}%[nosep]
 \item \(\pst{-x}=\pst{-a -\epsilon}=-a=-\tonda{a}=-\tonda{\pst{x}}\)\\
Se \(a\) è un numero infinitamente vicino a \(x\), allora \(-a\) sarà 
infinitamente vicino a \(-x\) e dato che \(a=\pst{x}\) si ottiene la tesi.
 \item \(\pst{x \pm y} = \pst{a+\epsilon \pm (b+\delta)}= 
         \pst{a\pm b+\epsilon \pm \delta}= a\pm b = \pst{x}\pm\pst{y}\)\\
Se \(a ~ \text{e} ~ b\) sono la parte standard di \(x ~ \text{e} ~ y\),
al posto di \(x ~ \text{e} ~ y\) posso scrivere 
\(a+\epsilon ~ \text{e} ~ b+\delta\), ma nella somma algebrica valgono le 
proprietà commutativa e associativa quindi otteniamo che \(x \pm y\) è 
uguale al numero reale \(a \pm b\) più un infinitesimo e la sua parte 
standard è proprio questo numero reale che può anche essere scritto:
\(\pst{x}\pm\pst{y}\).
 \item\(\pst{xy} = \pst{(a+\epsilon)(b+\delta)}=
        \pst{ab+a\delta+b\epsilon+\epsilon\delta}=ab=
        \pst{x} \cdot \pst{y}\) \\
Infatti \dots
 \item Perché valga la tesi, \(\frac{x}{y} \approx \frac{a}{b}\), la 
differenza tra le due frazioni deve essere un infinitesimo:\\
\(\dfrac{x}{y}- \dfrac{a}{b}=\dfrac{a+\epsilon}{b+\delta}-\dfrac{a}{b}=
  \dfrac{(a+\epsilon)b-a(b+\delta)}{(b+\delta)b}=
  \dfrac{ab+b\epsilon-ab-a\delta}{b^2+b\delta}=
  \dfrac{b\epsilon-a\delta}{b^2+b\delta}\)\\ quindi \dots
(continua tu, bastano poche considerazioni senza ulteriori calcoli).
\end{enumerate}

\subsection{Retta iperreale e strumenti ottici}
\label{subsec:insnum_retta}

Abbiamo accettato facilmente l'idea che 
ad ogni numero reale corrisponda un punto della retta e ad ogni 
punto della retta corrisponda un numero reale. 
Questa affermazione non è un teorema dimostrato, è un postulato. 
Fa parte del modello di numeri usato, questa idea è caratteristica dei 
numeri reali. 

Ma dato che ora stiamo cambiando modello, cambiamo anche questo postulato. 
Lo riformuliamo così:

\begin{postulato}
Ad ogni numero iperreale corrisponde un punto della retta (iperreale) e ad 
ogni punto della retta (iperreale) corrisponde un numero iperreale.
\end{postulato}

Oppure:

\begin{postulato}[Retta iperreale]
C'è una corrispondenza biunivoca tra i numeri iperreali e 
i punti della retta.
\end{postulato}

Abbiamo già una certa abitudine a rappresentare numeri reali sulla retta, 
per rappresentare i numeri iperreali dobbiamo procurarci degli strumenti 
particolari: \emph{microscopi}, \emph{telescopi}, \emph{grandangoli} 
\emph{non standard}.

Diamo una sbirciata al loro manuale di istruzioni.

\newpage %--------------------------------------------------

\subsubsection{Microscopi}
\label{subsec:insnum_microscopio}

Il microscopio permette di ingrandire una porzione di retta. 
Per esempio un microscopio permette di visualizzare i seguenti numeri:

\begin{esempio}
~

\begin{multicols}{4}
\begin{itemize}[nosep]
 \item \(+4,998\)
 \item \(-3,000002\)
 \item \(2-3\epsilon\)
 \item \(-4+2\delta\)
\end{itemize}
\end{multicols}

\begin{inaccessibleblock}
[Applicazione di diversi microscopi a numeri reali.]
\begin{minipage}{.48\linewidth}
 \begin{center}
\scalebox{0.7}{\microscopioa}
 \end{center}
% \caption{Microscopio per vedere \(5,004\).} \label{fig:microscopioa}
\end{minipage}
\hfill
\begin{minipage}{.48\linewidth}
 \begin{center}
\scalebox{0.7}{\microscopiob}
 \end{center}
% \caption{Microscopio per vedere \(-3,000002\).} \label{fig:microscopiob}
\end{minipage}
\end{inaccessibleblock}

\begin{inaccessibleblock}
[Applicazione di diversi microscopi a numeri iperreali.]
\begin{minipage}{.48\linewidth}
 \begin{center}
\scalebox{0.7}{\microscopioc}
 \end{center}
% \caption{Microscopio per \emph{non} vedere \(2-3\epsilon\).} 
\label{fig:microscopioc}
\end{minipage}
\hfill
\begin{minipage}{.48\linewidth}
 \begin{center}
\scalebox{0.7}{\microscopiod}
 \end{center}
% \caption{Microscopio per vedere \(2-3\epsilon\).} \label{fig:microscopiod}
\end{minipage}
\end{inaccessibleblock}
\end{esempio}
 
Si può osservare come ci siano microscopi ``standard'' che ingrandiscono un 
numero \emph{finito} di volte e microscopi ``non standard'' che 
ingrandiscono \emph{infinite} volte (ricordiamoci che \(\dfrac{1}{\epsilon}\) 
è un infinito.

\subsubsection{Telescopi}
\label{subsec:insnum_telescopi}

Il telescopio permette di avvicinare una porzione di retta senza cambiare 
la sua scala. Con un telescopio possiamo visualizzare i seguenti numeri:

% \newpage %--------------------------------------------------

\begin{esempio} ~

\begin{multicols}{4}
\begin{itemize}[nosep]
 \item \(+127034\)
 \item \(-3600\)
 \item \(A+3\)
 \item \(-B+2\)
\end{itemize}
\end{multicols}
\vspace{-5mm}

\begin{inaccessibleblock}[Applicazione di diversi telescopi.]
\begin{minipage}{.48\linewidth}
 \begin{center}
\scalebox{0.7}{\telescopioa}
 \end{center}
% \caption{Telescopio per vedere \(127034\).} \label{fig:telescopioa}
\end{minipage}
\hfill
\begin{minipage}{.48\linewidth}
 \begin{center}
\scalebox{0.7}{\telescopiob}
 \end{center}
% \caption{Telescopio per vedere \(A+3\).} \label{fig:telescopiob}
\end{minipage}
\end{inaccessibleblock}
\end{esempio}

Anche per i telescopi, i modelli più moderni offrono la possibilità di 
operare ingrandimenti ``standard'' o ``non standard'' a piacere.

\subsubsection{Grandangoli (Zoom Out)}
\label{subsec:insnum_zoom}

Il Grandangolo permette di cambiare la scala della visualizzazione della 
retta, in questo modo possiamo far rientrare nel campo visivo anche numeri 
molto lontani.
Possiamo usare un grandangolo per visualizzare i seguenti numeri:

\begin{esempio}~

\begin{multicols}{4}
\begin{itemize}[nosep]
 \item \(300\)
 \item \(-5000\)
 \item \(-2A\)
 \item \(3B\)
\end{itemize}
\end{multicols}
\vspace{-5mm}
\begin{inaccessibleblock}[Applicazione di diversi grandangoli.]
\begin{minipage}{.48\linewidth}
 \begin{center}
\scalebox{0.7}{\grandangoloa}
 \end{center}
% \caption{Grandangolo per vedere \(300\).} \label{fig:grandangoloa}
\end{minipage}
\hfill
\begin{minipage}{.48\linewidth}
 \begin{center}
\scalebox{0.7}{\grandangolob}
 \end{center}
% \caption{Grandangolo per vedere \(-2A\).} \label{fig:grandangolob}
\end{minipage}
\end{inaccessibleblock}
\end{esempio}

Anche per i grandangoli utilizzeremo versioni che permettono zoomate 
``standard'' e ``non standard''.

\begin{inaccessibleblock}[Combinazione di diversi strumenti.]
\begin{minipage}{.38\linewidth}
\subsubsection{Combinazione di strumenti}
\label{subsec:insnum_combinazione}

Questi strumenti sono ``modulari'', possono essere combinati a piacere. 
Per esempio per visualizzare il numero non standard: 
\(1741,998 +2\epsilon\) posso utilizzare in sequenza un telescopio per 
avvicinarmi al numero, un microscopio standard per poter vedere i 
millesimi e un microscopio non standard per vedere il numero infinitamente 
vicino al numero standard.
\end{minipage}
\hfill
\begin{minipage}{.58\linewidth}
 \begin{center}
\scalebox{0.7}{\combinazione}
 \end{center}
\end{minipage}
\end{inaccessibleblock}

\vspace{-5mm}

\subsection{Operazioni e tipo del risultato}
\label{subsec:insnum_operazioni}

Vediamo di seguito alcune regole relative alle operazioni 
che valgono nei numeri iperreali. 
A volte nell'eseguire un'operazione tra iperreali 
non siamo interessati al valore preciso, ma 
ci basta sapere il tipo del risultato. 
In questo paragrafo esploriamo il tipo del risultato delle quattro 
operazioni aritmetiche.

Teniamo presente che, quando il numero iperreale è un numero reale, 
valgono le stesse proprietà delle operazioni nei reali.

Perciò: 
\begin{description} %[noitemsep]
\item [zero] è l'elemento neutro dell'addizione e 
elemento assorbente per la moltiplicazione;
\item [uno] è elemento neutro della moltiplicazione;
\item [la divisione per zero] non è definita.
\end{description}


\bigskip % inutile!
\newpage %-----------------------------------------------

\subsubsection{Addizione}
\label{subsec:insnum_addizione}

\begin{multicols}{2}
Alcune osservazioni:

\begin{enumerate} [nosep]
 \item Le regole relative all'addizione valgono anche per la sottrazione, 
se uno degli addendi è negativo. 
%  \item Zero è l'elemento neutro dell'addizione nei Reali e continua ad 
% esserlo anche negli Iperreali: \(x+0=0+x=x\).
 \item Un infinitesimo più un altro infinitesimo dà per risultato un 
infinitesimo: \(\alpha+\beta=\gamma\).
 \item Un infinitesimo non nullo più un altro infinitesimo non nullo può 
dare per risultato anche zero: \dots
 \item Un finito più un infinitesimo dà come risultato un finito.
 \item Un finito più un finito dà come risultato un finito.
 \item Un finito più un finito può dare come risultato un infinitesimo.
%  \item Un infinito più un infinitesimo dà come risultato un infinito.
 \item Un infinito più un finito dà come risultato un infinito.
 \item Un infinito più un infinito può dare come risultato zero, un 
infinitesimo, un finito non infinitesimo, un infinito.
\end{enumerate}

% Nel precedente elenco abbiamo visto che alcune addizioni danno un 
% risultato che dipende solo dai tipi degli operandi, 
% altre operazioni danno dei risultati che dipendono dal valore degli 
% operandi. 
% Possiamo costruire una tabella che organizza le precedenti osservazioni.

\begin{center} \scalebox{0.8}{\tabopposto} \end{center}

\begin{center} \scalebox{0.8}{\tabadd} \end{center}

% \begin{center}
% \renewcommand{\arraystretch}{.0}
% \scalebox{0.7}{
% \begin{tabular}{p{.7cm}|p{.7cm}|p{.7cm}|p{.7cm}|p{.7cm}|p{.7cm}|}
% \centra{\(+\)} & \centra{0} & \centra{inn} & \centra{fni} & \centra{I} 
% \\\hline
% \centra{0} & \centra{0} & \centra{inn}& \centra{fni} & \centra{I} \\\hline
% \centra{inn} & \centra{inn} & \centra{i}& \centra{fni} & \centra{I} \\\hline
% \centra{fni} & \centra{fni} & \centra{fni}& \centra{f} & \centra{I} \\\hline
% \centra{I} & \centra{I} & \centra{I}& \centra{I} & \centra{?} \\\hline
% \end{tabular}}
% \end{center}
% % \vspace{2mm}
\end{multicols}

\subsubsection{Moltiplicazione}
\label{subsec:insnum_moltiplicazione}

\begin{multicols}{2}
Alcune osservazioni:
\begin{enumerate} [nosep]
%  \item Zero è l'elemento assorbente: il prodotto di un iperreale per zero
% dà come risultato zero: \(x \cdot 0=0 \cdot x=0\).
%  \item Uno è l'elemento neutro della moltiplicazione nei Reali e continua 
% ad esserlo anche negli Iperreali: \(x \cdot 1=1 \cdot x=x\).
 \item Un infinitesimo per un altro infinitesimo dà per risultato un 
infinitesimo: \(\alpha \cdot \beta=\gamma\).
 \item Un infinitesimo non nullo per un altro infinitesimo non nullo dà 
per risultato un infinitesimo non nullo.
 \item \dots
 \item \dots
%  \item Un finito per un infinitesimo dà come risultato un finito.
%  \item Un finito per un finito dà come risultato un finito.
%  \item Un finito per un finito può dare come risultato un infinitesimo.
% %  \item Un infinito per un infinitesimo dà come risultato un infinito.
%  \item Un infinito per un finito dà come risultato un infinito.
%  \item Un infinito per un infinito può dare come risultato zero, un 
% infinitesimo, un finito non infinitesimo, un finito, un infinito;
 \item Il prodotto fra un finito e un infinitesimo richiama le osservazioni 
fatte sul postulato di Eudosso-Archimede.
 \end{enumerate}

\begin{center} \scalebox{0.8}{\tabmul} \end{center}

% \begin{center}
% \renewcommand{\arraystretch}{.0}
% \scalebox{0.7}{
% \begin{tabular}{p{.7cm}|p{.7cm}|p{.7cm}|p{.7cm}|p{.7cm}|p{.7cm}|}
% \centra{\(\times\)} & \centra{0} & \centra{1} & 
% \centra{inn} & \centra{fni} & \centra{I} \\\hline
% \centra{0} & \centra{0} & \centra{0} & 
% \centra{0}& \centra{0} & \centra{0} \\\hline
% \centra{1} & \centra{0} & \centra{1} & 
% \centra{inn} & \centra{fni} & \centra{I} \\\hline
% \centra{inn} & \centra{0} & \centra{inn} & 
% \centra{inn}& \centra{inn} & \centra{?} \\\hline
% \centra{fni} & \centra{0} & \centra{fni} & 
% \centra{inn}& \centra{fni} & \centra{I} \\\hline
% \centra{I} & \centra{0} & \centra{I} & 
% \centra{?}& \centra{I} & \centra{I} \\\hline
% \end{tabular}}
% \end{center}
\end{multicols}

\subsubsection{Reciproco e divisione}
\label{subsec:insnum_reciproco}

\begin{multicols}{2}
Da quanto detto riguardo a infinitesimi e infiniti, 
possiamo ricavare la tabella dei reciproci.
% Alcune osservazioni:
% \begin{enumerate} [noitemsep]
%  \item Da quanto detto riguardo a infinitesimi e infiniti, 
%  possiamo ricavare la tabella dei reciproci.
% \end{enumerate}

\begin{center} \scalebox{0.8}{\tabrec} \end{center}

E combinandola con la tabella della moltiplicazione ottenere quella della 
divisione.
\begin{center} \scalebox{0.8}{\tabdiv} \end{center}

% E la tabella corrispondente:
% \begin{center}
% \renewcommand{\arraystretch}{.0}
% \scalebox{0.7}{
% \begin{tabular}{p{1.7cm}|p{.7cm}|p{.7cm}|p{.7cm}|p{.7cm}|p{.7cm}|}
% \centra{numero} & \centra{0} & \centra{1} & 
% \centra{inn} & \centra{fni} & \centra{I} \\\hline
% \centra{reciproco} &  & \centra{1} & 
% \centra{I} & \centra{fni} & \centra{inn} %\\\hline
% \end{tabular}}
% \end{center}
% % \vspace{11mm}
\end{multicols}

\newpage %--------------------------------------------------

% \subsubsection{Divisione}
% \label{subsec:insnum_divisione}
% 
% \begin{multicols}{2}
% Combinando la tabella della moltiplicazione con la tabella del 
% reciproco otteniamo la tabella della divisione. 
% 
% \begin{center} \scalebox{0.8}{\tabdiv} \end{center}
% 
% % Alcune osservazioni:
% % \begin{enumerate} [noitemsep]
% %  \item Anche negli Iperreali la divisione per zero non è definita.
% %  \item Uno può essere visto come un elemento neutro solo destro: \(x \div 
% % 1=x\).
% %  \item Per cercare i risultati possiamo rifarci alla definizione di 
% % quoziente.
% %  \item \dots
% % \end{enumerate}
% % E la tabella corrispondente:
% % \begin{center}
% % \renewcommand{\arraystretch}{.0}
% % \scalebox{0.7}{
% % \begin{tabular}{p{.7cm}|p{.7cm}|p{.7cm}|p{.7cm}|p{.7cm}|p{.7cm}|}
% % % \begin{tabular}{c|c|c|c|c|c|}
% % \centra{\(\div\)} & \centra{0} & \centra{1} & 
% % \centra{inn} & \centra{fni} & \centra{I} \\\hline
% % \centra{0} &  & \centra{0} & 
% % \centra{0}& \centra{0} & \centra{0} \\\hline
% % \centra{1} &  & \centra{1} & 
% % \centra{I} & \centra{fni} & \centra{inn} \\\hline
% % \centra{inn} &  & \centra{inn} & 
% % \centra{?}& \centra{inn} & \centra{inn} \\\hline
% % \centra{fni} &  & \centra{fni} & 
% % \centra{I}& \centra{fni} & \centra{inn} \\\hline
% % \centra{I} &  & \centra{I} & 
% % \centra{I}& \centra{I} & \centra{?} \\\hline
% % \end{tabular}}
% % \end{center}
% \end{multicols}

% \newpage %--------------------------------------------------

\begin{osservazione}
Non ci sono regole immediate per le seguenti operazioni:
\begin{multicols}{4}
\begin{itemize} [nosep]
 \item \(\dfrac{\epsilon}{\delta}\)
 \item \(\dfrac{A}{B}\)
 \item \(A \cdot \epsilon\)
 \item \(A + B\)
\end{itemize}
\end{multicols}
In questi casi il tipo di risultato dipende dall'effettivo valore degli 
operandi. Ad esempio, nel caso del quoziente tra due infinitesimi possiamo 
trovarci nelle seguenti situazioni:
\begin{multicols}{3}
\begin{itemize} [nosep]
 \item \(\dfrac{\epsilon^2}{\epsilon} = \epsilon\) \quad (i)
 \item \(\dfrac{2\epsilon}{\epsilon} = 2\) \quad (fni)
 \item \(\dfrac{\epsilon}{\epsilon^2} = \dfrac{1}{\epsilon}\) \quad (I)
\end{itemize}
\end{multicols}
\end{osservazione}

Possiamo ora esercitarci nel calcolo con questi nuovi numeri. 
Continuiamo ad utilizzare la convenzione di indicare gli 
infinitesimi con lettere greche minuscole
(\(\alpha,~\beta,~\gamma,~\delta,~\epsilon,~\dots\)), 
i finiti non infinitesimi con lettere latine minuscole 
(\(a,~b,~c~,~\dots,~m,~n,~\dots\)) 
e gli infiniti con lettere latine maiuscole 
(\(A,~B,~C~,~\dots,~M,~N,~\dots\)).

% \begin{exrig}
Semplifichiamo le seguenti espressioni scrivendo il tipo di risultato 
ottenuto.

 \begin{esempio}
  \(3\epsilon +5 +6M -2\epsilon +7 -2M = 4M +12 +\epsilon \quad (tipo=I)\)
 \end{esempio}

\begin{osservazione}
Quando il risultato è la somma di più elementi, li scriviamo ordinandoli 
dal più grande, in valore assoluto, al più piccolo.
%  Quando possibile scriviamo un risultato composto da più elementi 
% dal più grande, in valore assoluto, al più piccolo.
\end{osservazione}

 \begin{esempio}
\(7 +8M -5\epsilon  -4 +3\epsilon-2N = 8M -2N +3 -2\epsilon\)
\quad (tipo non definito)
 \end{esempio}
 
 \begin{esempio}
\(\tonda{3M +2\epsilon} \tonda{3M -2\epsilon} = 9M -4\epsilon\)
\quad (tipo=I)
 \end{esempio}
 
 \begin{esempio}
\(\tonda{M +3} \tonda{M -3} - \tonda{M+2}^2 +4\tonda{M +3}=\)

\(=M^2 -9 -M^2 -4M -4 +4M +12 = -1\)
\quad (tipo=fni)
 \end{esempio}
 
 \begin{esempio}
\(10a -\tonda{A +1}^2 -3a +2\tonda{a+2\alpha} +A^2 +6\tonda{b -3\alpha} 
+2A= 
\)

\(=10a -A^2 -2A -1 -3a +2a+4\alpha +A^2 +6b -18\alpha +2A = 9a +6b 
-14\alpha\) 
\quad (tipo=fni)
 \end{esempio}
% \end{exrig}

% \subsection{Espressioni}
% \label{subsec:insnum_espressioni}

\subsection{Confronto}
\label{subsec:insnum_confronto}

L'insieme dei numeri Reali ha un ordinamento completo, se \(a\) e \(b\) sono 
due numeri reali qualunque è sempre valida una e una sola delle seguenti 
affermazioni:
\[a<b \quad a=b \quad b<a\]
Per confrontare due numeri Reali possiamo utilizzare le seguenti regole:

\begin{enumerate} [noitemsep]
 \item qualunque numero negativo è minore di qualunque numero positivo;
 \item se due numeri sono negativi, è minore quello che ha il modulo 
maggiore;
 \item se \(a\) e \(b\) sono due numeri positivi, 
 \[a<b \sLRarrow a-b<0 \quad \text{ (o } \quad b-a>0 \text{ )}\]
oppure
 \[a<b \sLRarrow \frac{a}{b}<1 \quad \text{ (o } \quad \frac{b}{a}>1 
   \text{ )}\]
\end{enumerate}

\begin{osservazione}
Le prime due regole ci permettono di restringere le nostre riflessioni al 
solo caso del confronto tra numeri positivi.
Nei prossimi paragrafi assumeremo che le variabili si riferiscano solo a 
numeri positivi.
\end{osservazione}

\begin{osservazione}
Nella terza regola abbiamo presentato due criteri. Quello usato di solito
è il primo, ma useremo anche il secondo perché il rapporto tra due 
grandezze permette di ottenere informazioni interessanti.
\end{osservazione}

\vspace{1em}

% TODO forse è il caso di parlare di transfer prima di questo punto!

Per il principio di \emph{transfer} le stesse regole che valgono per i 
reali, valgono anche per i numeri iperreali.

% Anche negli Iperreali valgono le proprietà dei Reali richiamate sopra. 
% Vediamo allora come è possibile affrontare il problema del confronto tra 
% iperreali.

Restringendo l'osservazione ai numeri positivi possiamo affermare che gli 
infinitesimi sono più piccoli dei non infinitesimi e i finiti sono più 
piccoli degli infiniti:
\[i \quad < \quad fni \quad < \quad I\]
Passiamo ora al confronto all'interno dei diversi tipi di numeri iperreali.

\subsubsection{Confronto tra finiti non infinitesimi}
\label{subsubsec:insnum_confrontoreali}

Se due numeri iperreali hanno parte standard diversa allora è maggiore 
quello 
che ha la parte standard maggiore:
\[x < y \sLRarrow \st(x) < st(y)\]
Nel caso i due numeri abbiano la stessa parte standard si deve studiare 
l'ordinamento degli infinitesimi, cosa che faremo nel prossimo paragrafo.

% \newpage %-------------------------------------------------

\subsubsection{Confronto tra infinitesimi}
\label{subsubsec:insnum_confrontoreali}

Di seguito vediamo i diversi casi in cui ci possiamo 
imbattere quando vogliamo confrontare i numeri infinitesimi.

\paragraph{Zero}
Zero è minore di qualunque infinitesimo positivo:
\[\epsilon-0 = \epsilon>0\]
\paragraph{Somma}
La somma di infinitesimi positivi è maggiore di ognuno dei due:
\[\tonda{\epsilon+\delta}-\epsilon = \delta > 0\]
\paragraph{Multiplo}
Il multiplo di un infinitesimo positivo è maggiore dell'infinitesimo di 
partenza. Usando il primo metodo per il confronto:
\[\forall n > 1 \quad n\epsilon-\epsilon = \tonda{n-1}\epsilon > 0\]
e usando il secondo metodo: 
\[\forall n > 1 \quad \frac{n\epsilon}{\epsilon} = n > 1\]
\paragraph{Sottomultiplo}
Il sottomultiplo di un infinitesimo è minore dell'infinitesimo di partenza. 
Usando il primo metodo per il confronto:
\[\forall n > 1 \quad \frac{\epsilon}{n}-\epsilon = 
                      \frac{\epsilon-n\epsilon}{n} = 
                      \frac{\tonda{1-n}\epsilon}{n} < 0\]
e usando il secondo metodo: 
\[\forall n > 1 \quad \frac{\epsilon}{n}:\epsilon =
                      \frac{\epsilon}{n\epsilon} =
                      \frac{1}{n} < 1\]
\begin{definizione}
 Diremo che \(\gamma\) e \(\epsilon\) sono \textbf{infinitesimi dello 
stesso ordine} se il rapporto tra \(\gamma\) e \(\epsilon\) è un 
finito non infinitesimo.
\end{definizione}
\paragraph{Parte infinitesima di un infinitesimo}
La parte infinitesima di un infinitesimo positivo è minore 
dell'infinitesimo di partenza. Se \(\gamma\) è un infinitesimo di 
un infinitesimo \(\epsilon\) cioè \(\gamma= \delta \epsilon\) allora:
\[\frac{\gamma}{\epsilon} = \frac{\delta \epsilon}{\epsilon}=\delta < 1\]
% \begin{osservazione}
In questo caso il rapporto non solo è più piccolo di~1 ma è addirittura un 
\emph{infinitesimo}, 
cioè~\(\gamma\) è una parte infinitesima di~\(\epsilon\). 
% Quando il rapporto tra due infinitesimi è un infinitesimo 
% cioè se~\(\gamma\) è un infinitesimo di \(\epsilon\)
In questo caso 
si dice che \(\gamma\) è un infinitesimo di \emph{ordine superiore} a 
\(\epsilon\) e si scrive:
\[\gamma=o(\epsilon)\]
% \end{osservazione}
\begin{definizione}
Diremo che \(\gamma\) è un \textbf{infinitesimo di ordine superiore} 
all'infinitesimo \(\epsilon\) se 
il rapporto tra \(\gamma\) e \(\epsilon\) è un infinitesimo:
\[\gamma=o(\epsilon) \sLRarrow \frac{\gamma}{\epsilon}=\delta
\approx 0\]
Diremo anche che 
\(\epsilon\) è un \textbf{infinitesimo di ordine inferiore} a \(\gamma\).
\end{definizione}

\subsubsection{Confronto tra infiniti}
\label{subsubsec:insnum_confrontoreali}

Anche tra gli infiniti possiamo effettuare il confronto calcolando la 
differenza tra due numeri o il quoziente e anche tra gli infiniti l'uso del 
quoziente ci dà delle informazioni interessanti.
Per semplicità consideriamo \(M \stext{e} N\) infiniti positivi.

% \paragraph{Infinito più infinitesimo}
% Confrontiamo \(M+\epsilon\) con \(M\). 
% Usando il primo metodo:
% \[M+\epsilon-M = \epsilon > 0\]
% e usando il secondo metodo: 
% \[\frac{M+\epsilon}{M} =
%   \frac{M}{M} \frac{\epsilon}{M} = 
%   1 + \frac{\epsilon}{M} > 1\]

\paragraph{Infinito più finito}
Se \(a\) è un finito positivo (anche infinitesimo), confrontiamo \(M+a\) 
con \(M\). 
Usando il primo metodo:
\[M+a-M = a > 0\]
e usando il secondo metodo: 
\[\frac{M+a}{M} =
  \frac{M}{M} + \frac{a}{M} = 
  1 + \frac{a}{M} > 1\]
\paragraph{Somma di infiniti}
Se \(M\) e \(N\) sono due infiniti positivi, confrontiamo \(M+N\) 
con \(M\). 
Usando il primo metodo:
\[M+N-M = N > 0\]
e usando il secondo metodo: 
\[\frac{M+N}{M} =
  \frac{M}{M} + \frac{N}{M} = 
  1 + \frac{N}{M} > 1\]
\paragraph{Multiplo}
Se \(n>1\), confrontiamo \(nM\) con \(M\). 
Usando il primo metodo:
\[\forall n>1 \quad nM-M = \tonda{n-1}M > 0\]
e usando il secondo metodo: 
\[\frac{nM}{M} = n > 1\]
\begin{definizione}
 Diremo che \(M\) e \(N\) sono \textbf{infiniti dello stesso ordine}  
se il rapporto tra \(M\) e \(N\) è un finito non infinitesimo.
\end{definizione}
\paragraph{Infinito di infinito}
Confrontiamo \(MN\) con \(M\). 
Usando il primo metodo:
\[MN-M = \tonda{N-1}M > 0\]
e usando il secondo metodo: 
\[\frac{MN}{M} = N > 1\]
% \begin{osservazione}
In questo caso il rapporto non solo è maggiore di~1 ma è addirittura un 
\emph{infinito}. 
\begin{definizione}
 Diremo che \(M\) è un \textbf{infinito di ordine superiore} a 
\(N\) se il rapporto tra \(M\) e \(N\) è un infinito:
\[M=\omega(N) \sLRarrow \frac{M}{N}=I\]
Diremo anche che 
\(N\) è un \textbf{infinito di ordine inferiore} a \(M\).
\end{definizione}
% Quando il rapporto tra due infiniti è un infinito cioè 
% se~\(A\) vale infinite volte~\(B\) diremo che~\(A\) è un infinito di 
% \emph{ordine superiore} a~\(B\).
% \end{osservazione}
% 
% \vspace{1em}
% Ora confrontiamo \(M\) con \(MN\) usando il secondo metodo: 
% \[\frac{M}{MN} = \frac{1}{N} = \epsilon < 1\]
% 
% \begin{osservazione}
% In questo caso il rapporto non solo è minore di~1 ma è addirittura un 
% \emph{infinitesimo}. 
% Quando il rapporto tra due infiniti è un infinitesimo cioè 
% se~\(B\) vale infinite volte~\(A\) diremo che~\(A\) è un infinito di 
% \emph{ordine inferiore} a~\(B\).
% \end{osservazione}

\vspace{1em}
A volte il confronto tra due iperreali è meno immediato dei casi precedenti.
\begin{esempio}
 Confrontare \(M\) e \(2^M\). 
 Dobbiamo calcolare: \(\frac{M}{2^M}\). 
Possiamo usare un duplice trucco: 
\begin{itemize} [nosep]
 \item invece di confrontare \(M\) e \(2^M\) confrontiamo \(M^2\) e \(2^M\);
 \item invece che confrontare direttamente i due valori richiesti, vediamo 
come si comportano, con numeri naturali piccoli, le due funzioni 
\(y_1=x^2\) e \(y_2=2^x\)
\end{itemize}
% invece che confrontare direttamente i due 
% valori 
% richiesti, vediamo come si comportano, con numeri naturali piccoli le due 
% funzioni  
% % \(y_1=\angolare{x^2}\) e \(y_2=\angolare{2^x}\):
% \(y_1=x^2\) e \(y_2=2^x\):
\begin{center}
\begin{tabular}{cccccccc}
\(x^2\) & 0 & 1 & 4 & 9 & 16 & 25 & 36\\
\(2^x\) & 1 & 2 & 4 & 8 & 16 & 32 & 64
\end{tabular}
\end{center}
Possiamo vedere che dal quinto elemento in poi la prima successione è sempre
minore della seconda ed essendo l'infinito più grande di cinque 
otteniamo che \(2^M > M^2\) quindi possiamo scrivere:
\[\frac{M}{2^M} < \frac{M}{M^2} = \frac{1}{M} < 1\]
Ma \(\frac{1}{M}\) è un infinitesimo quindi \(M\) è un infinito di ordine 
inferiore a \(2^M\).
\end{esempio}

In conclusione, possiamo confrontare fra di loro i numeri iperreali 
utilizzando la differenza o il quoziente tra i numeri. L'uso del quoziente 
ci permette di ricavare un'informazione interessante: l'ordine di 
infinitesimo o di infinito.
\begin{itemize} [noitemsep]
 \item un infinitesimo di ordine superiore è un infinitesimo infinitamente 
 più piccolo;
 \item un infinito di ordine superiore è un infinito infinitamente più 
grande.
\end{itemize}

\subsection{Indistinguibili}
\label{subsec:insnum_indistinguibili}

Quando risolviamo un problema pratico, a noi serve, alla fine dei calcoli, 
ottenere un numero razionale, con un certo numero di cifre significative.
È chiaro che se il risultato di un calcolo è~\(4,37+5\epsilon\), sostituire 
questo risultato con il più semplice~\(4,37\) non ci fa perdere in precisione e 
in questo caso~\(5\epsilon\) può essere trascurato.
Ben diverso è se all'interno di un calcolo  
otteniamo:~\(\epsilon+5\epsilon\), in questo caso non posso 
trascurare~\(5\epsilon\) anche se è una quantità infinitesima. 

In certi casi posso avere due espressioni diverse che, in prima 
approssimazione, possono essere considerate equivalenti. Quando è così dirò 
che i due numeri iperreali sono \emph{indistinguibili}.
Due numeri sono indistinguibili quando la differenza tra i due è 
infinitesima rispetto a ciascuno dei due.

\begin{definizione}
Due numeri si dicono \textbf{indistinguibili} (simbolo:~\(\sim\)) se il 
rapporto tra la loro differenza e ciascuno di essi è un infinitesimo:
\[x \sim y \sLRarrow 
\tonda{\frac{y-x}{x} = \epsilon \sstext{e} \frac{y-x}{y} = \delta}
\]
\end{definizione}

\begin{osservazione}
 È importante osservare che per poter applicare la definizione entrambi i 
numeri che vogliamo confrontare devono essere diversi da \emph{zero}.
Cioè un numero diverso da zero non può mai essere considerato 
indistinguibile da zero.
\end{osservazione}

Di seguito esploriamo i tre casi possibili.

\subsubsection{Finiti non infinitesimi}
\label{subsubsec:insnum_finitini}

Se due numeri finiti non infinitesimi differiscono per un 
infinitesimo, sono indistinguibili.

\begin{teorema}
Due numeri \(x\) e \(y\), finiti non infinitesimi, 
sono indistinguibili se e solo se sono infinitamente vicini:
\[x \approx y \sLRarrow x \sim y\]
\end{teorema}

% \begin{proof}
Iniziamo dimostrando che se sono infinitamente vicini allora sono 
indistinguibili:
\begin{center}
Ipotesi: \(\tonda{x,~y:\ fni \stext{e} y = x+\epsilon} \qquad \sLRarrow 
\qquad\) 
Tesi: \(x \sim y\).
\end{center}
Dimostrazione
\[\frac{y-x}{x}=\frac{\tonda{x+\epsilon}-x}{x} = 
\frac{\epsilon}{x}= \gamma \sstext{e} 
\frac{y-x}{y}=\frac{\tonda{x+\epsilon}-x}{y} = 
\frac{\epsilon}{y}= \delta
\]
Il teorema inverso dirà:
\begin{center}
Ipotesi: \(\tonda{x,~y:\ fni \stext{e} x \sim y} \qquad \sLRarrow \qquad\) 
Tesi: \(x \approx y\).
\end{center}
Dimostrazione
\[\frac{y-x}{x} = \epsilon
\sRarrow y-x = \epsilon x \sRarrow y = x+\epsilon x = y+\beta
\]
(Il caso $\dfrac{y-x}{y}$ a questo punto è banale.)
% \end{proof}

\subsubsection{Infinitesimi}
\label{subsubsec:insnum_infinitesimi}

Per quanto riguarda gli infinitesimi, non basta che siano infinitamente 
vicini, infatti tutti gli infinitesimi sono infinitamente vicini tra di 
loro.
Per essere indistinguibili serve una condizione più restrittiva.

\begin{teorema}
Due numeri \(\alpha\) e \(\beta\), infinitesimi, 
sono indistinguibili se e solo se la loro differenza è un infinitesimo di 
ordine superiore.
\[\beta-\alpha = o(\alpha) \sLRarrow \alpha \sim \beta\] 
\end{teorema}

\begin{osservazione}
Se due infinitesimi differiscono per un infinitesimo di ordine superiore 
allora sono dello stesso ordine, quindi la differenza sarà di ordine 
superiore sia al primo sia al secondo infinitesimo.
\end{osservazione}

% \begin{proof}
Iniziamo dimostrando che se differiscono per un infinitesimo di ordine 
superiore allora sono indistinguibili:
\begin{center}
Ipotesi: \(\tonda{\alpha,~\beta:\ inn \stext{e} \beta = \alpha +o(\alpha)}
\qquad \sLRarrow\qquad\) 
Tesi: \(\alpha \sim \beta\)
\end{center}
Dimostrazione
\[\frac{\beta-\alpha}{\alpha}=
\frac{\tonda{\alpha +o(\alpha)}-\alpha}{\alpha} = 
\frac{o(\alpha)}{\alpha}= \gamma \sstext{e} 
\frac{\beta-\alpha}{\beta}=
\frac{\tonda{\alpha +o(\alpha)}-\alpha}{\beta} = 
\frac{o(\beta)}{\beta}= \delta
\]

% \pagebreak %-----------------------------------------------------

Il teorema inverso dirà:
\begin{center}
Ipotesi: \(\tonda{\alpha,~\beta:\ inn \stext{e} \alpha \sim \beta}
\qquad \sLRarrow \qquad\) 
Tesi: \(\beta-\alpha = o(\alpha)\)
\end{center}
Dimostrazione
\[\frac{\beta - \alpha}{\alpha} = \epsilon \sRarrow 
\beta - \alpha =\epsilon \alpha \sRarrow 
\beta - \alpha = o(\alpha)
\]
% \end{proof}

\subsubsection{Infiniti}
\label{subsubsec:insnum_infiniti}

La situazione si ribalta se i due numeri sono infiniti infatti, in 
questo caso, sono indistinguibili anche se differiscono di 
un valore finito o addirittura infinito. 

Si può dimostrare il seguente
\begin{teorema}
Due numeri \(M\) e \(N\), infiniti, 
sono indistinguibili se e solo se la loro differenza è un finito o 
un infinito di ordine inferiore.
\[M-N = a \sRarrow M \sim N\] 
e vale anche:
\[\tonda{M-N = P \text{ con P infinito di ordine inferiore }} 
\sRarrow M \sim N\] 
\end{teorema}
% \begin{proof}
Di seguito dimostriamo che se differiscono per un finito allora 
sono indistinguibili:
\begin{center}
Ipotesi: \(\tonda{M,~N:\ I \stext{e} N = M+a}
\qquad \sLRarrow\qquad\) 
Tesi: \(M \sim N\)
\end{center}
Dimostrazione
\[\frac{M-N}{M}=
\frac{M-\tonda{M +a}}{M} = 
-\frac{a}{M}= \epsilon \sstext{e} 
\frac{M-N}{N}=
\frac{M-\tonda{N +a}}{N} = 
-\frac{a}{N}= \delta
\]
In modo analogo si può procedere con la seconda parte del teorema.
% \end{proof}

\subsection{Principio di tranfer}
\label{subsec:insnum_nonarchimedei}

Abbiamo applicato agli iperreali le operazioni aritmetiche con grande 
naturalezza estendendo i metodi e i risultati che già conosciamo nei Reali. 
Ma è possibile fare ciò per qualunque funzione? 
Sì, è possibile assumere che per ogni funzione definita nei Reali esista 
una corrispondente funzione con dominio e codominio negli Iperreali che, 
ristretta ai Reali, coincida con la funzione reale.
In questo modo tutto quello che è possibile fare con i numeri Reali lo si 
può fare anche con gli Iperreali.

\begin{osservazione}
 Non vale il viceversa. Dato che gli Iperreali estendono i Reali, ci sono 
delle funzioni che, definite negli Iperreali, non hanno un valore 
corrispondente nei Reali. Ad esempio la funzione iperreale \emph{parte 
standard} non ha una funzione corrispondente nei numeri reali.

\begin{esempio}
 Consideriamo ad esempio la funzione: 
\(f: x \mapsto \frac{1}{x} \quad \forall x \in \R \stext{e} x \ne 0\)

È facile costruire la funzione \(\effestar\) (\emph{effe star}) con 
dominio e codominio negli Iperreali:

\(\effestar: x \mapsto \frac{1}{x} \quad 
\forall x \in \IR \stext{e} x \ne 0\).

Ogni volta che \(\effestar\) è applicata a numeri standard, si 
comporta come la funzione \(f\), applicata a \(x \in \R\); 
ma, in più, la funzione~\(\effestar\):
\begin{itemize} [noitemsep]
 \item 
è definita anche per valori infinitamente vicini a zero e 
in questo caso dà come risultato un valore infinito che non è un numero 
reale;
 \item 
è definita anche per valori infiniti e
in questo caso dà come risultato un valore infinitesimo che non è un numero 
reale. 
\end{itemize}
\end{esempio}
\end{osservazione}

\section{Applicazioni}
\label{sec:insnum_applicazioni}

Dopo aver dato un'occhiata a cosa sono e come funzionano i numeri iperreali 
vediamo qualche problema che si può convenientemente risolvere con 
questi numeri.

\subsection{Espressioni con gli iperreali}
\label{subsec:insnum_espressioni}

I problemi concreti si esprimono sempre con numeri razionali, poi, per 
semplificane la soluzione e poter utilizzare metodi più potenti, vengono 
trasportati in insiemi numerici più complessi come gli iperreali e, una volta 
risolti, le soluzioni vengono ritradotte in numeri razionali magari passando 
per i reali con l'uso della funzione \(\st()\).

% I numeri iperreali semplificano la ricerca della soluzione di molti 
% problemi.
% Il calcolo delle soluzioni ci porta a risultati espressi quasi sempre da 
% numeri standard, che corrispondono ai reali. Infatti, quasi sempre il 
% calcolo termina ricorrendo alla funzione \(\st()\).\\
% Questo metodo, cioè ricorrere ad un insieme più astratto di \(\R\), 
% svolgervi i calcoli secondo le nuove regole e alla fine esprimere i 
% risultati in \(\R\), 
% sembra inutilmente complicato, 
% ma in realtà semplifica la soluzione di molti problemi 
% (come vedremo più avanti).

Vediamo, con alcuni esempi, come si possono applicare 
le regole presentate in precedenza al calcolo di espressioni 
contenenti numeri iperreali. 

Ricordiamo che con lettere greche minuscole indichiamo infinitesimi e con 
lettere latine maiuscole indichiamo infiniti.

% Di seguito richiamiamo le convenzioni già presentate:
% \begin{itemize} [nosep]
%  \item con le lettere greche minuscole indichiamo gli infinitesimi non 
% nulli;
%  \item con \emph{x, y, z} indichiamo un numero iperreale qualsiasi;
%  \item con le altre lettere latine minuscole indichiamo i numeri finiti non 
% infinitesimi;
%  \item con le lettere latine maiuscole indichiamo gli infiniti.;
%  \item con \(\pst{x}\) indichiamo la parte standard di \(x\).
% \end{itemize}

\begin{esempio}
Calcola un valore reale dell'espressione iperreale: \quad
\(\dfrac{7 -2 \epsilon}{9 +3 \delta}\)
\[\pst{\dfrac{7 -2 \epsilon}{9 +3 \delta}} ~ \stackrel{1}{=} ~
  \dfrac{\pst{7 -2 \epsilon}}{\pst{9 +3 \delta}}~ \stackrel{2}{=} ~
  \dfrac{\pst{7 -\alpha}}{\pst{9 +\beta}} ~ \stackrel{3}{=} ~ 
\dfrac{7}{9}\]
Dove i passaggi hanno i seguenti motivi:
\begin{enumerate} [nosep] 
\item La parte standard di una frazione con dnominatore non infinitesimo
è la frazione delle parti standard;
\item se \(\epsilon \text{ e } \delta\) sono infinitesimi, 
anche \(2\epsilon\) e \(3 \delta\) sono infinitesimi;
\item la parte standard della soma di un reale e di un infinitesimo è il 
reale stesso.
\end{enumerate}
\end{esempio}

\begin{esempio}
Calcola un valore reale dell'espressione: \quad
\(\dfrac{4 \epsilon^4 -7 \epsilon^3}{5 \epsilon}\quad (\text{con }\epsilon \ne 0)\) \
\[\pst{\dfrac{4 \epsilon^4 -7 \epsilon^3}{5 \epsilon}} 
~ \stackrel{1}{=} ~
  \pst{\dfrac{\tonda{4 \epsilon^3 -7 \epsilon^2} \cancel{\epsilon}}
                    {5 \cancel{\epsilon}}} 
~ \stackrel{2}{=} ~ 
  \pst{\dfrac{4 \epsilon^3 -7 \epsilon^2}{5}}
~ \stackrel{3}{=} ~
  \pst{\dfrac{\alpha}{5}}
~ \stackrel{4}{=} ~
  \pst{\beta}
~ \stackrel{5}{=} ~
  0\]
Dove i passaggi hanno i seguenti motivi:
\begin{enumerate} [nosep]
 \item si può raccogliere \(\epsilon\) al numeratore; 
 \item dato che \(\epsilon\) è diverso da zero, si può semplificare la 
frazione; 
 \item finiti per infinitesimi sono infinitesimi, la loro
somma è un infinitesimo: \(\alpha\);
 \item il rapporto tra un infinitesimo e un non infinitesimo è un 
infinitesimo che chiamiamo \(\beta\);
 \item La parte standard di un infinitesimo è zero. 
\end{enumerate}
\end{esempio}

\begin{esempio}
Calcola:\quad \(\pst{\dfrac{-6\epsilon^2 +4 \epsilon^3 -8 \epsilon^5}
             {7\epsilon^3 + 2 \epsilon^4}}\quad (\text{con }\epsilon \ne 0)\)
\[\pst{\dfrac{-6\epsilon^2 +4 \epsilon^3 -8 \epsilon^5}
             {7\epsilon^3 + 2 \epsilon^4}} 
~ \stackrel{1}{=} ~
  \pst{\dfrac{-6\cancel{\epsilon^2}}
             {7\epsilon^{\cancel{3}}}} 
~ \stackrel{2}{=} ~
  \pst{-\dfrac{6}{7\epsilon}}  
~ \stackrel{3}{=} ~
  \pst{-M} 
~ \stackrel{4}{\longrightarrow} ~
  N.D.\]
Dove i passaggi hanno i seguenti motivi:
\begin{enumerate} [nosep]
 \item tengo solo la parte principale dei polinomi ottenendo un'espressione 
indistinguibile che, quindi ha la stessa parte standard;
 \item semplifico i fattori uguali;
 \item il quoziente tra un finito e un infinitesimo non nullo è un 
infinito (\(M\));
 \item non è definita la parte standard di un numero infinito.
\end{enumerate}
\begin{osservazione}
\end{osservazione}
\end{esempio}

\begin{esempio}
Riduci l'espressione \quad 
\(\dfrac{4 \epsilon^4 -7 \epsilon^3}{5 \epsilon} \quad (\text{con }\epsilon \ne 0)                    
\)
usando la relazione di indistinguibilità:
\[\dfrac{4 \epsilon^4 -7 \epsilon^3}{5 \epsilon} 
~ \stackrel{1}{\sim} ~
  \dfrac{7\epsilon^{\cancel{3}}}{5 \cancel{\epsilon}} 
~ \stackrel{2}{=} ~
  \dfrac{7\epsilon^2}{5}\]
Dove i passaggi hanno i seguenti motivi:
\begin{enumerate} [nosep]
 \item nei polinomi, considero solo gli infinitesimi di ordine inferiore; 
 \item riduco la frazione semplificando i fattori uguali.
\end{enumerate}
\end{esempio}

\begin{esempio}
Riduci l'espressione: \quad
\(\dfrac{5\epsilon -3 \epsilon^2 + 6 \epsilon^3}
         {2\epsilon + 4 \epsilon^2}\quad (\text{con }\epsilon \ne 0)\)
usando la relazione di indistinguibilità: 
\[\dfrac{5\epsilon -3 \epsilon^2 + 6 \epsilon^3}
         {2\epsilon + 4 \epsilon^2} 
~ \stackrel{1}{\sim} ~
  \dfrac{5 \cancel{\epsilon}}{2 \cancel{\epsilon}} 
~ \stackrel{2}{=} ~
  \dfrac{5}{2}\]
\begin{enumerate} [nosep]
 \item nei polinomi considero solo gli infinitesimi di ordine minore;
 \item riduco la frazione semplificando i fattori uguali.
\end{enumerate}
\end{esempio}

\begin{esempio}
Riduci l'espressione 
\(\pst{\dfrac{-3H^2 -4H}{2H^2 +4H -3}}\)
usando la relazione di indistinguibilità: 
\[\dfrac{-3H^2 -4H}{2H^2 +4H -3} 
~ \stackrel{1}{\sim} ~
  \dfrac{-3\cancel{H^2}}{2\cancel{H^2}} 
~ \stackrel{2}{=} ~
  \dfrac{-3}{2}\]
Dove i passaggi hanno i seguenti motivi:
\begin{enumerate} [nosep]
 \item nei polinomi considero solo gli infiniti ordine maggiore; 
 \item semplifico i fattori uguali al numeratore e al denominatore.
\end{enumerate}
\end{esempio}

\begin{esempio}
Riduci l'espressione \quad 
\(\pst{\tonda{7 -3\epsilon} - \tonda{7 +8\epsilon}}\):

\begin{osservazione}
 Si potrebbe pensare che essendo~\(7 -3\epsilon\) indistinguibile da~\(7\) 
e~\(7 +8\epsilon\) indistinguibile da~\(7\) la precedente espressione 
sia indistinguibile da \(7 - 7 = 0\).
Ma il concetto di indistinguibile non si può mai applicare tra un numero e 
lo zero quindi non possiamo dire che 
\(\tonda{7 -3\epsilon} - \tonda{7 +8\epsilon} \sim 0\) e 
tanto meno: \(\tonda{7 -3\epsilon} - \tonda{7 +8\epsilon} = 0\).
\end{osservazione}

In questo caso la soluzione è semplice...
\[\tonda{7 -3\epsilon} - \tonda{7 +8\epsilon}
~ \stackrel{1}{=} ~
   \cancel{+7} -3\epsilon ~ \cancel{-7} -8\epsilon 
~ \stackrel{2}{\sim} ~
   -11 \epsilon\]
Dove i passaggi hanno i seguenti motivi:
\begin{enumerate} [nosep]
 \item semplifichiamo l'espressione eliminando le parentesi; 
 \item \(+7\) e \(-7\) si annullano
 \item nel polinomio che otteniamo, la parte principale è costituita dagli 
infinitesimi di primo ordine.
\end{enumerate}
\end{esempio}

\begin{esempio}
Riduci l'espressione \quad 
\(\tonda{5 +\epsilon}^2 - \tonda{5 -\epsilon}\tonda{5 +\epsilon}\):
\[\tonda{5 +\epsilon}^2 - \tonda{5 -\epsilon}\tonda{5 +\epsilon}
~ \stackrel{1}{=} ~
\tonda{25 +2\epsilon+\epsilon^2} - \tonda{25 -\epsilon^2}
~ \stackrel{2}{=} ~
25 - 25 = 0
\]
\begin{osservazione}
Anche qui abbiamo usato l'indistinguibilità tra un numero diverso da zero 
e zero: \emph{non va bene}!
\end{osservazione}

In questo caso la soluzione corretta è:
\begin{align*}
\tonda{5 +\epsilon}^2 - \tonda{5 -\epsilon}\tonda{5 +\epsilon}
~ &\stackrel{1}{=} 
\tonda{25 +2\epsilon+\epsilon^2} - \tonda{25 -\epsilon^2}
~ \stackrel{2}{=} 
\cancel{25} +2\epsilon+\epsilon^2 ~ \cancel{-25} -\epsilon^2
~ = \\
&= 2\epsilon+2\epsilon^2 
~ \stackrel{3}{\sim} ~
2\epsilon
\end{align*}
Dove i passaggi hanno i seguenti motivi:
\begin{enumerate} [nosep]
 \item svolgiamo i prodotti (notevoli); 
 \item \(+25\) e \(-25\) si annullano; 
 \item nel polinomio, consideriamo gli infinitesimi di ordine minore.
\end{enumerate}
\end{esempio}

% \pagebreak %--------------------------------------------------

\begin{esempio}
Riduci l'espressione 
\(\sqrt{9H^2 -12H} - \sqrt{9H^2 +1}\)
usando la relazione di indistinguibilità: 

\begin{osservazione}
 Anche qui si potrebbe pensare che essendo~\(9H^2 -12H\) indistinguibile 
da~\(9H^2\) e~\(9H^2 +1\) indistinguibile da~\(9H^2\), la precedente 
espressione sia indistinguibile da \\
\(\sqrt{9H^2} - \sqrt{9H^2} = 0\)\\
Ma il concetto di indistinguibile, non si può \emph{mai} 
applicare tra un numero e lo zero quindi non possiamo dire che 
\(\sqrt{9H^2 -12H} - \sqrt{9H^2 +1} \sim 0\).
\end{osservazione}\\[1em]
Per risolvere questa situazione usiamo un trucco: la razionalizzazione del 
numeratore.

\(\sqrt{9H^2 -12H} - \sqrt{9H^2 +1}
~ \stackrel{1}{=} ~
   \tonda{\sqrt{9H^2 -12H} - \sqrt{9H^2 +1}} \cdot 1 
~ \stackrel{2}{=} ~\)

\(~ \stackrel{2}{=} ~\dfrac{\sqrt{9H^2 -12H}-\sqrt{9H^2 +1}}{1} \cdot 
   \dfrac{\sqrt{9H^2 -12H}+\sqrt{9H^2 +1}}{\sqrt{9H^2 -12H}+\sqrt{9H^2+1}}
~ \stackrel{3}{=} ~\)

\(~ \stackrel{3}{=} ~
   \dfrac{\cancel{9H^2} -12H - \cancel{9H^2} -1}
         {\sqrt{9H^2 -12H}+\sqrt{9H^2+1}}
~ \stackrel{4}{\sim} ~
   \dfrac{-12H}{\sqrt{9H^2} + \sqrt{9H^2}}
~ \stackrel{5}{=} ~
   \dfrac{-12H}{6\abs{H}} = -2\dfrac{H}{\abs{H}}\)\\

Dove i passaggi hanno i seguenti motivi:
\begin{enumerate} [nosep]
 \item la prima uguaglianza è banale essendo~1 l'elemento 
neutro della moltiplicazione; 
 \item al posto del numero~1 sostituisco un'opportuna frazione con 
il numeratore e il denominatore uguali; 
 \item eseguendo il prodotto magari tenendo conto di uno 
dei prodotti notevoli (\(a-b)(a+b)\)), ottengo questa frazione che
non sembra aver semplificato il problema iniziale;
 \item ma ora, a denominatore, ho una \emph{somma} tra le due radici 
e quindi posso ottenere un'espressione più semplice 
indistinguibile da quella originale;
 \item estraiamo le radici, sommiamo e semplifichiamo.
\end{enumerate}
\end{esempio}

\subsection{Problemi con gli iperreali}
\label{subsec:insnum_problemi}

% TODO Disegni

\subsubsection{Cornici e differenze}
\label{subsubsec:cornici_differenze}

\begin{esempio}
 % Quadrato di lato l+eps.
Calcola l'area iperreale di una cornice quadrata, di lato interno pari a 
\(l\) e spessore infinitesimo \(\epsilon\). 
Calcola il valore reale di questa area e 
il rapporto tra l'area e lo spessore infinitesimo della cornice.

\affiancati{.54}{.44}{
Chiamiamo \(dS\) l'area iperreale della cornice:
\[dS=\tonda{l+\epsilon}^2-l^2=
l^2+2l\epsilon+\epsilon^2-l^2=2l\epsilon+\epsilon^2\]
Chiamiamo \(\Delta S\) la corrispondente area reale:
\[\Delta S=\st\tonda{dS}=\st\tonda{2l\epsilon+\epsilon^2}=
\st(2l\epsilon)+\st(\epsilon^2)=0\]
Poiché la differenza di area \(dS\) è la somma di due infinitesimi, 
uno del primo e l'altro del secondo ordine, la parte standard di entrambi 
è nulla e la somma risulta nulla. 
}{
\begin{center} \diffarearettangolo \end{center}
}\\

Mentre l'area di questa cornice è infinitesima, il rapporto tra quest'area 
e il suo spessore ha un valore non infinitesimo:
\[\frac{dS}{\epsilon} = 
  \frac{\cancel{\epsilon} \tonda{2l+\epsilon}}{\cancel{\epsilon}} = 
  2l+\epsilon\]
e la sua parte standard è: \(\pst{2l+\epsilon} = 2l\).
\end{esempio}

\begin{esempio}
 % contrazione cerchio.
Calcola di quanto varia una circonferenza di raggio \(r\),
quando il raggio subisce una variazione infinitesima \(dr=\epsilon\) 
(\(\epsilon \ne 0\).\\

\affiancati{.54}{.44}{
Chiamiamo \(dC\) (differenza di C) la variazione della circonferenza:
\[dC=2\pi(r+\epsilon) -2\pi r=2\pi r -2\pi r -2\pi \epsilon =
     2\pi \epsilon\]
Possiamo osservare che la variazione della lunghezza della circonferenza 
ha lo stesso segno di \(\epsilon\), quindi se \(\epsilon > 0\) la 
circonferenza aumenta, se \(\epsilon < 0\) la circonferenza diminuisce.

In ogni caso la circonferenza varia di un infinitesimo e: 
\(\pst{2\pi \epsilon} = 0\).
}{
\begin{center} \diffcirconferenza \end{center}
}\\

Ma se calcoliamo il rapporto tra la variazione 
della circonferenza e la variazione del raggio, abbiamo:
\[\dfrac{dC}{dr}=\dfrac{2\pi \epsilon}{\epsilon}=2\pi\]
Ogni unità di variazione del raggio, comporta una variazione della 
circonferenza pari a \(2\pi\).
\end{esempio}

\begin{esempio}
 % guscio sferico.
Quanto volume acquisisce un guscio sferico di raggio \(r\) nel gonfiarsi
progressivamente? \\

\affiancati{.54}{.44}{
Volume iniziale: \\
\(V(r)=\dfrac{4}{3}\pi r^3\). 
Se il raggio aumenta e diventa \(r+\epsilon\) (con \(r\ne 0\)),
la variazione di volume sarà:
\begin{align*}
d V &= V(r+\epsilon)-V(r) = 
\dfrac{4}{3}\pi (r+\epsilon)^3-\dfrac{4}{3}\pi r^3= \\
    &= \dfrac{4}{3}\pi (r^3+3r^2\epsilon+3r\epsilon^2+\epsilon^3-r^3)= \\
    &= \dfrac{4}{3}\pi (3r^2\epsilon+3r\epsilon^2+\epsilon^3) =
       \dfrac{4}{3}\pi \epsilon(3r^2+3r\epsilon+\epsilon^2)
\end{align*}
}{
\begin{center} \incrementosferico \end{center}
}\\

Per sapere quanto varia il volume per ogni variazione infinitesima di 
raggio, si calcola:
\[\dfrac{dV}{dr}=
\dfrac{\dfrac{4}{3}\pi \epsilon(3r^2+3r\epsilon+\epsilon^2)}{\epsilon}=
\dfrac{4}{3}\pi (3r^2+3r\epsilon+\epsilon^2)\] 
che è un numero di finito.

La sua parte standard è \(\st\tonda{\dfrac{dV}{dr}}=4\pi r^2\). 
Nota che questa è l'espressione dell'area della superficie sferica. 
Come era prevedibile, infatti, un guscio sferico di spessore infinitesimo 
approssima la superficie sferica.
\end{esempio}

\subsubsection{Pendenza e tangenti}
\label{subsubsec:insnum_tangenti_parabola}

Riprendiamo il problema di calcolare la tangente ad una parabola in un suo 
punto e lo affrontiamo usando i numeri iperreali. Ricordiamo che la 
tangente ad una curva in un punto \(T\) è una retta che passa per \(T\) e 
ha la stessa pendenza della curva in quel punto. 
Possiamo riassumere così i passi da svolgere:

\begin{enumerate}[noitemsep]
 \item calcolo l'ordinata del punto di tangenza \(T\);
 \item calcolo la pendenza della curva in  \(T\);
 \item scrivo l'equazione del fascio di rette passanti per \(T\);
 \item sostituisco il valore trovato nell'equazione del fascio di rette.
\end{enumerate}
Il punto più complicato è il secondo, ma proprio qui il calcolo con gli 
infinitesimi ci può aiutare. 

Una retta ha, in ogni suo punto la stessa pendenza che calcoliamo come:
\[m=\dfrac{\Delta y}{\Delta x}=\dfrac{y_1-y_0}{x_1-x_0}=
\dfrac{f(x_0+\Delta x)-f(x_0)}{\tonda{x_0+\Delta x}-x_0}=
\dfrac{f(x_0+\Delta x)-f(x_0)}{\Delta x}\]
Questo valore è anche detto coefficiente angolare della retta.

Una parabola, o qualunque altra curva in ogni punto ha una pendenza 
diversa, nonostante questo, è possibile calcolare la pendenza di una 
parabola in un punto qualsiasi usando la stessa formula usata per la retta 
a patto di prendere un \(\Delta x\) infinitesimo e di calcolare il 
corrispondente \(\Delta y\).

Chiamando \(d x\) (de ics) un incremento infinitesimo di \(x\) e 
\(d y\) (de ipsilon) un incremento infinitesimo di \(y\),
la pendenza nel punto \(x_0\) sarà:
\[m=\pst{\dfrac{d y}{d x}}=\pst{\dfrac{y_1-y_0}{x_1-x_0}}=
    \pst{\dfrac{f(x_0+\epsilon)-f(x_0)}{\tonda{x_0+\epsilon}-x_0}}=
    \pst{\dfrac{f(x_0+\epsilon)-f(x_0)}{\epsilon}}\]
se esiste la parte standard e è sempre la stessa qualunque sia 
l'infinitesimo usato nel calcolo.

Vediamo qualche esempio.

% \newpage %--------------------------------------------------

\begin{esempio}
{~}

\begin{minipage}{.49\textwidth}
Considerata la parabola di equazione:
\[y=\dfrac{1}{2}x^2-4x+10\] 
calcola la sua pendenza nel punto di ascissa \(x_0=6\).

Chiamiamo \(\epsilon\) l'incremento infinitesimo di \(x\).
Il corrispondente incremento di \(y\) sarà: 
\begin{align*}
m&=\pst{\dfrac{d y}{d x}}=
\pst{\dfrac{f(6+\epsilon)-f(6)}{\epsilon}}=
\end{align*}
\end{minipage}
\hfill
\begin{minipage}{.49\textwidth}
\begin{center}\ipertangentea\end{center}
\end{minipage}
\begin{align*}
&=\pst{\dfrac{\dfrac{1}{2}(6+\epsilon)^2-4(6+\epsilon)+10
              -\dfrac{1}{2}(6)^2-4(6)+10}
             {\epsilon}}=\\
&=\pst{\dfrac{\dfrac{1}{2}\tonda{36+12\epsilon+\epsilon^2}-24-4\epsilon+10
              -4}
             {\epsilon}}=
   \pst{\dfrac{18+6\epsilon+\dfrac{1}{2}\epsilon^2-24-4\epsilon+10-4}
             {\epsilon}}=\\
 &=\pst{\dfrac{+2\epsilon+\dfrac{1}{2}\epsilon^2}{\epsilon}}=
   \pst{\dfrac{+2\cancel{\epsilon}}{\cancel{\epsilon}}}=2.
\end{align*}
La tangente in quel punto è:
\[y=m \tonda{x-x_0}+y_0 \sRarrow y=2 \tonda{x-6}+4 \sRarrow y=2x-8\]
\end{esempio}

\begin{esempio}
{~}

\begin{minipage}{.59\textwidth}
Calcola la tangente alla parabola:
\(y=\dfrac{1}{4}x^2-2x+6\)
nel punto di ascissa: \(x_0=2\).
\begin{enumerate}[noitemsep, wide, labelwidth=!,, labelindent=0pt]
\item calcoliamo l'ordinata del punto di tangenza \(T\):
\[f(2)=0,25 \cdot 2^2 -2 \cdot 2 +6=1-4+6=3\]
 \item scriviamo l'equazione del fascio di rette per \(T\):
\[y=m \tonda{x-x_0}+y_0 \sRarrow y=m \tonda{x-2}+3\]
\end{enumerate}
\end{minipage}
\hfill
\begin{minipage}{.39\textwidth}
\begin{center}\ipertangenteb\end{center}
\end{minipage}

\begin{enumerate}[noitemsep]
  \setcounter{enumi}{2}
 \item calcoliamo la pendenza della curva in \(T\):
\begin{align*}
m&=\pst{\dfrac{d y}{d x}}=
\pst{\dfrac{f(2+\epsilon)-f(2)}{\epsilon}}=
\pst{\dfrac{0,25(2+\epsilon)^2-2(2+\epsilon)+6-3}{\epsilon}}=\\
 &=\pst{\dfrac{0,25\tonda{4+4\epsilon+\epsilon^2}-4-2\epsilon+6-3}
              {\epsilon}}
 =\pst{\dfrac{1+\epsilon+0,25\epsilon^2-4-2\epsilon+6-3}
             {\epsilon}}=\\
&=\pst{\dfrac{-\epsilon+0,25\epsilon^2}{\epsilon}}=
\pst{\dfrac{\cancel{\epsilon}\tonda{-1+0,25\epsilon}}{\cancel{\epsilon}}}
=\pst{-1+0,25\epsilon} = - 1
\end{align*}
 \item sostituiamo il valore trovato nell'equazione del fascio di rette:
\[y = m \tonda{x-x_0}+y_0 \sRarrow y = - \tonda{x-2}+3 \sRarrow 
y = -x+2+3 \sRarrow y = -x+5\]
\end{enumerate}
\end{esempio}

\begin{esempio}
{~}

\begin{minipage}{.44\textwidth}
Calcola la tangente all'ellisse di equazione:
\(4x^2+3y^2=48\)
nel punto di coordinate \(T\punto{3}{2}\).

La funzione che descrive la parte di ellisse 
contenente \(T\) è:
\(y=+\sqrt{-\dfrac{4}{3}x^2+16}\)\\
L'equazione del fascio di rette per \(T\) è:\\
\(y=m\tonda{x-3}+2\)
\begin{align*}
m&=\pst{\dfrac{d y}{d x}}=
   \pst{\dfrac{f(3+\epsilon)-f(3)}{\epsilon}}=\\
 &=\pst{\dfrac{\sqrt{-\dfrac{4}{3}(3+\epsilon)^2+16}-2}{\epsilon}}=
\end{align*}
\end{minipage}
\hfill
\begin{minipage}{.54\textwidth}
\begin{center}\iperellisse\end{center}
\end{minipage}
\begin{align*}
m&=\pst{\dfrac{\sqrt{-\dfrac{4}{3}(3+\epsilon)^2+16}-2}{\epsilon} \cdot
        \dfrac{\sqrt{-\dfrac{4}{3}(3+\epsilon)^2+16}+2}
              {\sqrt{-\dfrac{4}{3}(3+\epsilon)^2+16}+2}}=\\
 &=\pst{\dfrac{-\dfrac{4}{3}\tonda{9+6\epsilon+\epsilon^2}+16-4}
        {\epsilon
         \tonda{\sqrt{-\dfrac{4}{3}\tonda{9+6\epsilon+\epsilon^2}+16}+2}}}=
   \pst{\dfrac{-8\epsilon-\frac{4}{3}\epsilon^2}{4 \epsilon}}=
   \pst{\dfrac{-8 \cancel{\epsilon}}{4 \cancel{\epsilon}}}=-2
\end{align*}
E la tangente è quindi:
\[y=m \tonda{x-x_0}+y_0 \sRarrow y=-2 \tonda{x-3}+2 \sRarrow y=-2x+8.\]

\end{esempio}

\begin{comment}
TODO
Iniziamo studiando più in particolare come eseguire i calcoli in 
\(\IR\). 
Riprendiamo il problema della velocità istantanea visto nel paragrafo 
\ref{subsec:insnum_velocita}.


\subsection{Vertice della parabola}
\label{subsec:insnum_vertice_parabola}


\subsection{Cerchio osculatore}
\label{subsec:insnum_cerchio_osculatore}


\subsection{Area di un segmento parabolico}
\label{subsec:insnum_segmento_parabolico}


\subsection{Problemi di massimo o minimo}
\label{subsec:insnum_massimo_minimo}

\end{comment}






































