% (c) 2015 Daniele Zambelli daniele.zambelli@gmail.com
% (c) 2017 Bruno Stecca

% % \vspace{-2ex}\input{\folder lbr/tab002}\vspace{-2ex}
% \begin{inaccessibleblock}
% [Immagine di una porzione dell'insieme di Mandelbrot.]
% \vspace{-2ex}
% \begin{center} \includegraphics[scale=0.25]{img/hiero3673.png} 
% \end{center}
% \vspace{-2ex}
% \end{inaccessibleblock}

\input{\folder dai_naturali_agli_iperreali_gra.tex}

\chapter{Dai Naturali agli Iperreali}

\section{Dai numeri naturali ai numeri irrazionali}
\label{sec:01_introduzione}

Riprendiamo i diversi insiemi numerici che abbiamo imparato a conoscerete
mettendo in evidenza il loro ruolo come modelli per risolvere alcune classi 
di problemi e le loro caratteristiche.

\subsection{I numeri naturali $\N$}
\label{subsec:insnum_naturali}

I primi numeri che abbiamo incontrato sono i numeri naturali. Sono quelli 
che 
permettono di contare oggetti. Se sul banco ho un quaderno, una penna e un 
libro posso dire che ci sono~3 oggetti. Si può capire come il numero Zero 
abbia avuto difficoltà a farsi accettare come numero: serve per contare un 
gruppo di oggetti dove non c'è niente da contare. 
Ma ora abbiamo capito che è molto comodo considerare lo zero come un numero.
Questi numeri sono chiamati numeri \emph{naturali} 
e l'insieme di questi numeri viene indicato con~$\N$.

Nei numeri naturali sono definite l'addizione, la moltiplicazione che sono 
sempre possibili. In queste due \emph{strutture} $\tonda{\N, +}$ e 
$\tonda{\N, \times}$ valgono le proprietà: associativa, commutativa e 
l'esistenza dell'elemento neutro.

Nei numeri naturali è definita anche la \emph{potenza} ma questa operazione 
non è definita quando sia la base sia l'esponente sono uguali a zero.

Oltre a queste, sono definite anche le loro operazioni inverse: la 
sottrazione, la divisione e la radice, queste non sono definite per ogni 
coppia di numeri.

D'altra parte se su un tavolo ho~5 oggetti posso toglierne~3 e ne restano~2:
\[5-3=2\]

Ma se sul tavolo ho~3 oggetti non ha senso cercare di toglierne~5!

\subsection{I numeri interi $\Z$}
\label{subsec:insnum_interi}

I numeri possono però essere utilizzati anche come modelli di altre 
situazioni. 
Supponiamo di avere la sequenza di oggetti e di voler riferirmi ad ognuno 
con un numero che equivale al suo indirizzo o indice. In certi casi potrei 
cercare il primo elemento della sequenza e chiamarlo zero, quello che viene 
dopo lo chiamo uno e così via. Ma se ci trovassimo a lavorare principalmente 
con gli elementi compresi tra il 273° elemento e il 310° elemento, questo 
modo di fare sarebbe piuttosto scomodo. 
Molto più semplice è mettersi d'accordo di chiamare zero il 273° elemento e 
partire da lì a contarli. Ora i numeri che dovremo usare saranno quelli 
compresi tra~0 e~37. 
Ci sono inoltre delle situazioni in cui è difficile, o impossibile, 
determinare un \emph{primo} elemento della sequenza e anche in questo caso 
ci si può mettere d'accordo di assegnare ad un preciso elemento della 
sequenza il valore zero.

E chiaro che lo \emph{zero} non sarà il \emph{primo} elemento della 
sequenza, ma un valore all'interno della sequenza. Quindi è possibile 
muoversi sia sopra lo zero, sia sotto lo zero.
Per non inventare dei nomi completamente nuovi per questi nuovi 
numeri, sono stati aggiunti semplicemente due segni:~``$+$'' per i numeri 
dopo lo zero e~``$-$'' per i numeri prima dello zero. 
Questi nuovi numeri sono chiamati numeri \emph{interi} 
e l'insieme di questi numeri viene indicato con~$\Z$.

In questa situazione l'addizione può essere vista come muoversi nel verso 
di crescita dei numeri e la sottrazione come muoversi nel verso della 
decrescita dei numeri. Dato che lo zero è un elemento convenzionale non c'è 
nessun problema a togliere~5 da~3 semplicemente si arriverà nella 
posizione~2 prima dello zero detta anche~$-2$.

In questo insieme di numeri è sempre definita anche la sottrazione, anzi la 
sottrazione diventa semplicemente un caso particolare di addizione.

I numeri interi permettono di risolvere sempre equazioni del tipo:
\[x+a=0\]

Il sottoinsieme di \(\Z\) formato dallo zero e da tutti i numeri positivi
si comporta esattamente come l'insieme dei numeri Naturali. Diremo che 
questo sottoinsieme è isomorfo all'insieme~\(\N\) 
e questo ci permette di di usare indifferentemente \(+7\) o \(7\) 
senza dover precisare che \(+7\) è un elemento di \(\Z\) 
mentre \(7\) è un elemento di \(\N\).

Anche questi numeri però non riescono a realizzare un modello in certe 
situazioni che invece nella pratica si possono risolvere facilmente con un 
po' di creatività. Ad esempio come possiamo dividere~3 uova, in parti 
uguali, tra~4 persone?

\subsection{I numeri razionali $\Q$}
\label{subsec:insnum_razionali}

Con le tre uova faccio una frittata che divido facilmente in~4 parti 
uguali. 
Possiamo costruire dei numeri che permettano di calcolare sempre il 
quoziente esatto di due numeri naturali anche quando la divisione tra i due 
dà un resto diverso da zero. 
Questi nuovi numeri sono chiamati numeri \emph{razionali} 
e l'insieme di questi numeri viene indicato con~$\Q$.

Mentre nei naturali e negli interi ad ogni numero corrisponde un 
\emph{nome} ben preciso, nei razionali lo stesso numero può essere indicato 
con molti nomi diversi. 
Ad esempio il numero che si ottiene dividendo~1 in due parti 
uguali può essere indicato in in uno di questi modi:
\[\frac{1}{2}=\frac{3}{6}=\dots=\frac{45}{90}=\frac{132}{264}=\dots=0,5\]

Ogni numero razionale può essere rappresentato con un numero con la virgola 
o con una qualunque delle infinite frazioni equivalenti.

Con i numeri razionali si può sempre calcolare il risultato della divisione 
tra due numeri (naturali, interi o razionali) tranne il caso particolare in 
cui il divisore sia uguale a zero. 
In questo caso la divisione non può essere eseguita.

I numeri razionali permettono di risolvere sempre equazioni del tipo:
\[ax+b=0 \quad \text{ con } \quad a \neq 0\]

I razionali hanno una caratteristica particolare che non avevano né i 
naturali né gli interi: formano un insieme \emph{denso} cioè tra due numeri 
razionali, per quanto vicini, se ne può trovare sempre almeno un altro.

Anche tra i razionali si può trovare un sottoinsieme isomorfo all'insieme 
degli interi, cioè che si comporta come l'insieme degli interi: è il 
sottoinsieme dei numeri che, scritti sotto forma di frazioni hanno come 
numeratore un multiplo del denominatore o che, ridotte ai minimi termini, 
hanno per denominatore uno. 
Questo fatto ci permette di poter scrivere:~\(-\dfrac{7}{1} = -7\) 
senza dover precisare che il primo numero appartiene a \(\Q\) e il secondo 
a \(Z\).

Ma ci sono ancora situazioni in cui i numeri razionali non permettono di 
risolvere problemi relativamente semplici da risolvere praticamente. 
Ad esempio è stato dimostrato (già qualche millennio fa) che se il lato di 
un quadrato è un numero razionale allora la sua diagonale non lo è. 

\subsection{I numeri reali $\R$}
\label{subsec:insnum_reali}

Se prendiamo un quadrato di lato~1, la sua diagonale, per il teorema di 
Pitagora, risulta lunga~$\sqrt{2}$. 
La radice di~2 è quel numero che elevato alla seconda dà come risultato~2.
Ebbene, è stato dimostrato che nessun numero razionale moltiplicato per se 
stesso dà come risultato~2. 
Quindi, o dà un numero più piccolo o un numero più grande.

Possiamo quindi costruire due sottoinsiemi dell'insieme $\Q$ in modo da 
mettere in uno tutti i numeri minori di un certo valore e nell'altro tutti i 
numeri maggiori o uguali a quel valore. Nel caso della radice di~2:

\begin{center}
 \begin{tabular}{ll}
\toprule
Valore per difetto di $\sqrt{2}$ &Valore per eccesso di $\sqrt{2}$ \\
\midrule
1& 2\\
1,4& 1,5 \\
1,41& 1,42\\
1,414& 1,415\\
1,4142& 1,4143\\
\ldots& \ldots\\
\bottomrule
\end{tabular}
\end{center}

Due sottoinsiemi costruiti in questo modo si chiamano classi contigue di 
numeri razionali, cioè due sottoinsiemi di \(Q\) tali che ogni elemento 
dellprimo è minore di qualunque elemento del secondo e che nel primo 
sottoinsieme ci sono numeri che si avvicinano quanto si vuole a certi 
numeri del secondo sottoinsieme.
Due classi contigue di numeri razionali definiscono un 
numero \emph{reale} $\R$. Dato un \emph{numero qualsiasi} possiamo sempre 
realizzare due classi contigue di numeri razionali. 
Se questo \emph{numero qualsiasi} appartiene al secondo sottoinsieme è, 
evidentemente un numero razionale, se non appartiene ai due sottoinsiemi è 
un numero irrazionale. 
Ognuna di queste partizioni, dette anche sezioni di Dedekind, può essere 
considerata come un numero, cioè è possibile costruire un ordine tra le 
sezioni, sommarle, moltiplicarle, \dots

I numeri reali formano un insieme \emph{ordinato}, \emph{denso} ma anche 
\emph{completo} cioè il numero individuato da una qualunque sezione di 
Dedekind è un numero reale.
Questo permette di far corrispondere ad ogni punto della \emph{retta reale} 
un numero \emph{reale} e viceversa ad ogni numero \emph{reale} un punto 
della \emph{retta reale}. 

Anche l'insieme dei Reali contiene un sottoinsieme isomorfo ai numeri 
razionali.

Bene l'insieme dei numeri reali permette di risolvere tutti i problemi che 
possiamo incontrare?
\vspace{-1em}
\begin{center} \emph{Per fortuna no!} \end{center}
\vspace{-.5em}
Ci sono tipi di problemi che non possono essere risolti con i numeri reali.
Ad esempio calcolare la radice quadrata di numeri negativi. 
Anche questo all'apparenza è un problema del tutto assurdo: calcolare la 
radice quadrata di un numero equivale a trovare la lunghezza del lato di un 
quadrato di cui si conosce l'area. 

Ora, trovare un quadrato con area piccola si può fare, magari anche con 
area nulla, impegnandosi un po', ma trovare un quadrato con area negativa 
è proprio impossibile. 
Ma come abbiamo visto per i naturali ci possono essere fenomeni nei quali 
hanno senso operazioni che in altri sistemi sono insensate.

Prima di procedere con i prossimi insiemi numerici, però, riflettiamo su 
una particolare proprietà degli insiemi visti fin'ora.

\subsubsection{Il postulato di Eudosso-Archimede}

Proviamo a fare un \emph{semplice} esperimento mentale. Prendo un foglio di 
carta e lo piego su se stesso un po' di volte. Che spessore raggiungo?
Per semplificarci i calcoli supponiamo che il foglio di carta abbia lo 
spessore di $0,1mm = 0,0001m = 10^{-4}m$. 
Che spessore otterrò piegando il foglio su se stesso~64 volte?

Il calcolo è abbastanza semplice:

\begin{center}
 \begin{tabular}{ccc}
\toprule
Numero piegature & spessore ottenuto & in metri\\
\midrule
0 & 1 & $10^{-4}$\\
1 & 2 & $2 \cdot 10^{-4}$\\
2 & 4 & $4 \cdot 10^{-4}$\\
3 & 8 & $8 \cdot 10^{-4}$\\
4 & 16 & $1,6 \cdot 10^{-3}$\\
5 & 32 & $3,2 \cdot 10^{-3}$\\
6 & 64 & $6,4 \cdot 10^{-3}$\\
7 & 128 & $1,28 \cdot 10^{-2}$\\
\ldots& \ldots\\
n & $2^n$ & \ldots\\
\bottomrule
\end{tabular}
\end{center}

Quindi piegando il foglio 64 volte ottengo uno spessore che è $2^{64}$ 
volte lo spessore di partenza quindi basta calcolare:
\[2^{64} = 18.446.744.073.709.551.616\]
che convertito in metri dà: $1.844.674.407.370.955m$ circa che è uno 
spessore considerevole, quasi duemila volte la distanza 
Terra-Sole:~$149.600.000.000m$.

Si fa risalire ai matematici Eudosso e Archimede l'osservazione che per 
quanto piccolo si prenda un numero (ad esempio lo spessore di un foglio di 
carta), basta moltiplicarlo per un numero sufficientemente grande~($2^{64}$) 
per farlo diventare maggiore di qualsiasi numero (ad esempio la distanza 
Terra-Sole).

\begin{postulato}[Eudosso-Archimede]
Dati due numeri positivi \(a, b\) si può sempre trovare un 
multiplo del più piccolo che sia maggiore del più grande:
\[\forall a, b \in \R \quad \exists n \in \N \quad | \quad na>b\]
\end{postulato}

Vale anche il contrario: per quanto grande sia un numero posso dividerlo 
per un numero abbastanza grande da farlo diventare più piccolo di un 
qualunque numero.

Ma questa osservazione di Eudosso-Archimede non è un teorema, non è 
un'osservazione dimostrata, è un postulato, un accordo fatto tra matematici 
che può essere utile in moltissimi casi e che vale per tutti gli insiemi 
numerici visti fin'ora. 
Ma cosa succede se ci accordiamo che \emph{non} valga il postulato di 
Eudosso-Archimede?

\subsection{I numeri complessi $\C$}
\label{subsec:insnum_complessi}

Riprendiamo il problema della radice di numeri negativi. Si può ampliare 
l'insieme dei numeri reali aggiungendo i numeri che sono le 
radici di tutti i numeri anche di quelli negativi. Per fare ciò si devono 
aggiungere molti altri numeri (infiniti) tutti questi nuovi numeri sono 
stati chiamati numeri \emph{immaginari} che combinati con i numeri reali 
formano l'insieme dei numeri \emph{complessi} insieme che viene indicato con 
$\C$. 
Anche per i numeri complessi tutti gli infiniti nuovi numeri si ottengono 
con la semplice aggiunta di un solo nuovo numero: \emph{l'unità 
immaginaria} indicato con il simbolo \(i\) o con il simbolo \(j\).
\begin{definizione}
 L'\textbf{unità immaginaria} è quel numero che elevato alla seconda dà 
come risultato \(-1\):
\[i^2 = -1\]
\end{definizione}

Questi numeri hanno molte applicazioni tecniche, ma risultano anche 
affascinanti da un punto di vista estetico. La ripetizione di un paio di 
calcoli aritmetici tra numeri complessi produce il sorprendente insieme di 
Mandelbrot.

\begin{wrapfloat}{figure}{r}{0pt}
\begin{inaccessibleblock}
[Immagine di una porzione dell'insieme di Mandelbrot.]
\includegraphics[scale=0.30]{img/fractal.jpg}
\end{inaccessibleblock}
\caption{Porzione dell'insieme di Mandelbrot.}
\label{fig:mandelbrot}
\end{wrapfloat}
Ma dato che l'insieme dei reali oltre che essere un campo ordinato è anche 
completo, non è possibile aggiungere elementi ai reali senza perdere 
qualche proprietà dell'insieme numerico. 
Nel caso dei complessi l'insieme ottenuto non è totalmente ordinato.

Inutile dire che possiamo prendere un sottoinsieme dei Complessi che sia 
isomorfo ai Reali.

% \vspace{24pt}

\section{I numeri iperreali $\IR$}
\label{sec:insnum_iperreali}

In questa sezione vedremo, e useremo, un nuovo insieme di numeri, utile a 
modellizzare e risolvere nuove classi di problemi. 
Rispetto a quanto già sappiamo dell'insieme \(\R\), dovremo adattare alcune 
regole di calcolo e riscontreremo proprietà nuove, mentre dovremo 
abbandonarne il postulato di Eudosso-Archimede. 

\subsection{Il problema della velocità}
\label{subsec:insnum_velocita}

Alla fine del 1600 Newton e Leibniz studiavano problemi legati alla 
meccanica. 
Una delle grandezze alla base della meccanica è la \emph{velocità}. 
Ma cosa è la velocità? 
Se l'oggetto A percorre più strada dell'oggetto B possiamo dire 
che A è più veloce di B? No, non basta misurare lo spazio percorso da un 
oggetto per calcolare la sua velocità, bisogna anche misurare il tempo 
impiegato a percorrere quello spazio. Infatti sappiamo che:
\[velocità = \frac{spazio percorso}{tempo impiegato}\]

La grandezza calcolata in questo modo è \emph{la velocità media} 
dell'oggetto, ma in ogni istante del percorso l'oggetto ha una propria 
velocità. 
Come faccio a calcolarla? Posso misurare lo spazio percorso in un tempo 
molto piccolo, in questo modo avrò una velocità media tenuta in un percorso 
molto breve. 
Più restringo l'intervallo di tempo, più la velocità media si avvicina alla 
velocità istantanea\dots ma resta sempre una velocità media. 

Per trovare la velocità istantanea dovrei dividere lo spazio percorso per 
un tempo (positivo) più piccolo di qualunque numero. L'unico numero reale
più piccolo, in valore assoluto, di qualunque numero è lo zero, ma 
non posso usarlo per il calcolo della velocità, perché la divisione per 
zero non è definita: i numeri reali non ci permettono di calcolare una 
grandezza così semplice e evidente come la velocità di un oggetto in un 
dato istante.

Servirebbe un insieme numerico con numeri positivi più piccoli di un 
qualsiasi altro numero positivo, ma diversi da zero! Ma è possibile 
trovare tali numeri nell'insieme dei reali che, come abbiamo visto, 
è un insieme (già) completo?

\subsection{Infinitesimi... e infiniti}
\label{subsec:insnum_nonarchimedei}

Se accettiamo che possa \textbf{non} valere il postulato di 
Eudosso-Archimede, possiamo costruire un insieme numerico non archimedeo. 
Per farlo, possiamo aggiungere all'insieme dei numeri reali un nuovo numero 
(non reale) maggiore di zero ma più piccolo di qualunque numero reale 
positivo:
\[\epsilon > 0 \quad \text{ tale che } \quad 
\epsilon < \frac{1}{n} \quad \text{ per qualunque } n \in \N\]
tradotto in simboli:
\[\exists \epsilon > 0 \quad | \quad \epsilon < \frac{1}{n} \quad \forall n 
\in \N\]
Un numero siffatto lo chiameremo un \emph{infinitesimo} e lo indicheremo 
con una lettera minuscola dell'alfabeto greco, per esempio $\epsilon$.
Per quanto è già stato detto, un tale numero non può essere un numero
reale. 
\begin{osservazione}
 In un insieme che contenga numeri infinitesimi non vale il postulato di 
Eudosso-Archimede infatti se
\(\epsilon < \dfrac{1}{n} \quad \forall n \in \N\) 
moltiplicando entrambi i membri per \(n\) 
si ottiene: \(n \epsilon < 1\).
\end{osservazione}

\vspace{1em}

La prima conseguenza dell'introduzione di un infinitesimo è che allora ce 
ne sono infiniti! 
Infatti anche la metà di un infinitesimo è un infinitesimo e 
sono infinitesimi anche il suo doppio o un suo sottomultiplo o un suo 
multiplo.

Altra conseguenza dell'aggiunta di un elemento infinitesimo all'insieme dei 
reali è che, se si possono fare le normali operazioni con questi nuovi 
numeri, allora esiste anche un numero maggiore di qualunque numero reale:

% Altra conseguenza dell'aggiunta di un elemento infinitesimo all'insieme 
% dei 
% reali è che, se esiste un numero maggiore di zero più piccolo di tutti i
% numeri reali, allora esiste anche un numero maggiore di qualunque altro 
% numero reale:

\[\text{se} \quad \epsilon < \frac{1}{n} \quad \forall n \in \N 
\quad \text{allora} \quad \frac{1}{\epsilon} > n \quad \forall n 
\in \N\]

Quindi se aggiungiamo all'insieme dei reali un numero infinitesimo e 
possiamo usarlo nelle usuali 4 operazioni, allora in quell'insieme avremo 
un numero infinito di infinitesimi e un numero infinito di infiniti.

Chiamiamo \emph{iperreali} questi numeri e indichiamo l'insieme degli 
iperreali con il simbolo:~$\IR$ (``erre star'').

\subsection{Tipi di Iperreali}
\label{subsec:insnum_iperreali}

Abbiamo visto che l'introduzione di un elemento nuovo, così piccolo
da poterlo pensare trascurabile, ha reso piuttosto affollato il nuovo 
insieme numerico. 
Cerchiamo di fare un po' di ordine. 
L'insieme degli Iperreali contiene diversi tipi di numeri li vediamo qui di 
seguito.

\begin{description} [noitemsep]
 \item \textbf{Infinitesimi}:
numeri che, in valore assoluto, sono minori di qualunque numero reale 
positivo.
 \item \textbf{Infiniti}:
numeri che, in valore assoluto, sono maggiori di qualunque numero reale.
 \item \textbf{Zero}:
l'unico numero reale infinitesimo.
 \item \textbf{Infinitesimi non nulli}:
numeri infinitesimi senza lo zero.
 \item \textbf{Finiti}:
numeri che non sono infiniti.
 \item \textbf{Finiti non infinitesimi}:
numeri che non sono né infiniti né infinitesimi.
\end{description}

Per semplificare la scrittura (e complicare la lettura) adotteremo delle 
sigle e delle convenzioni per indicare questi diversi tipi di numeri:

\begin{center}
\begin{tabular}{ccc}\toprule
tipo & sigla & simboli \\\midrule

zero &  & 0 \\

infinitesimi & \emph{i} & \\

infinitesimo non nullo & \emph{inn} & 
\(\alpha, \beta, \gamma, \delta, \dots\) \\

finito non infinitesimo & \emph{fni} & \(a, b, c, d, \dots\)\\

finito & \emph{f} & \(a, b, c, d, \dots\) \\

infinito & \emph{I} & \(A, B, C, \dots\)\\

qualsiasi &  & \(x, y, z, \dots\) \\\bottomrule
\end{tabular}
\label{tab:insnum_tipi}
\end{center}

\begin{esempio}
 Individua il tipo delle seguenti espressioni:
%  (considerando, per semplicità \(\epsilon\) positivo):

\begin{multicols}{4}
\begin{enumerate}
 \item \(\pi+\epsilon\)
 \item \(4\epsilon+\epsilon \cdot \delta\)
 \item \(M-7\)
 \item \(M+\dfrac{1}{\epsilon}\)
\end{enumerate}
\end{multicols}

Vediamo i vari casi:

\begin{enumerate}
 \item \(\pi+\epsilon\) 
è un numero finito perché \(\pi\) è un numero 
finito (\(3,141592653589793\dots\)) con infinite cifre decimali, ma 
\(\epsilon\) è più piccolo della più piccola cifra di \(\pi\) che possiamo 
pensare quindi aggiungere un infinitesimo ad un numero reale non cambia il 
numero reale che rimane un numero finito, in questo caso non infinitesimo.
 \item  \(4\epsilon+\epsilon \cdot \delta\) 
qui abbiamo la somma di due quantità, 
la prima è formata da 4 infinitesimi, ma per come abbiamo definito 
l'infinitesimo, anche 4 infinitesimi sono ancora un infinitesimo; la 
seconda è formata dal prodotto di un infinitesimo per un altro infinitesimo 
che indica quindi un infinitesimo di un infinitesimo che è un infinitesimo 
ancora più infinitesimo di ciascuno dei due. La loro somma quindi è un 
infinitesimo con lo stesso segno di \(\epsilon\).
 \item \(M-7\)
Possiamo distinguere due casi: 
 \begin{itemize} [noitemsep]
  \item se \(M\) è un infinito negativo, allora 
\(M-7\) sarà un numero in valore assoluto ancora più grande, 
 \item se \(M\) è un infinito positivo, \(M-7\) sarà numero più piccolo di 
\(M\) ma che non può essere un numero finito. 
Infatti, supponiamo che \(M-7\) sia un numero finito, chiamiamolo \(x\) ma 
se \(x\) è finito allora anche \(x+7\) è finito e questo avrebbe come 
conseguenza che anche \(M\) sia finito contraddicendo le nostre convenzioni.
 \end{itemize}
 \item \(M+\dfrac{1}{\epsilon}\)
 In questo caso dobbiamo fare una distinzione:
 \begin{itemize} [noitemsep]
  \item se \(M\) e \(\epsilon\) hanno lo stesso segno il calcolo precedente 
equivale a sommare due infiniti entrambi positivi (o negativi) e darà come 
risultato un infinito positivo (o negativo).
  \item se \(M\) e \(\epsilon\) hanno segni diversi bisogna avere più 
informazioni per poter stabilire il tipo del risultato.
 \end{itemize}

\end{enumerate}
\end{esempio}

\subsection{Numeri infinitamente vicini}
\label{subsec:insnum_infinitamentevicini}

Nei numeri reali, due numeri o sono uguali o sono diversi (ovviamente).
Nel primo caso, la differenza tra i due numeri è zero, 
nel secondo, la differenza è un numero reale diverso da zero.

Negli Iperreali, se due numeri sono diversi, la loro differenza può 
essere un numero finito non infinitesimo o un numero infinitesimo.

\begin{esempio}
 Calcola la distanza tra \(a=7+\epsilon\) e \(b=10-5\epsilon\):
 
 \(\abs{b-a}=\abs{\tonda{10+\epsilon} - \tonda{7-5\epsilon}}=
   \abs{10+\epsilon - 7+5\epsilon}=
   \abs{10-7 + \epsilon+5\epsilon}=
   \abs{3 + 6\epsilon}
 \)
 
 La distanza tra \(a\) e \(b\), è uguale a~3 più un infinitesimo.
\end{esempio}

\begin{esempio}
 Calcola la distanza tra \(a=5+\epsilon\) e \(b=5+\delta\):
 
 \(\abs{b-a}=\abs{\tonda{5+\delta} - \tonda{5+\epsilon}}=
   \abs{5+\delta - 5-\epsilon}=
   \abs{5-5 + \delta - \epsilon}=
   \abs{0 + \delta - \epsilon} = \abs{\gamma}
 \)
 
 In questo caso la distanza tra \(a\) e \(b\), è un infinitesimo.
\end{esempio}

\noindent Negli Iperreali possiamo distinguere:

\begin{itemize} [noitemsep]
 \item \(a\) e \(b\) sono uguali: \(b-a=0\);
 \item \(a\) e \(b\) sono diversi, in questo caso possiamo distinguere
 ulteriormente:
\begin{itemize} [nosep]
 \item \(a-b\) è un finito non infinitesimo;
 \item \(a-b\) è un infinitesimo non nullo.
\end{itemize}
\end{itemize}

Quando la differenza di due numeri è un infinitesimo, diciamo 
che i due numeri sono \emph{infinitamente vicini}.

\begin{definizione}    % [Infinitamente vicini]
Due numeri si dicono \textbf{infinitamente vicini} (simbolo:~$\approx$) se 
la loro differenza è un infinitesimo:
\[x \approx y \Leftrightarrow x - y = \epsilon\]
\end{definizione}

Tutti gli infinitesimi sono infinitamente vicini tra di loro e sono 
infinitamente vicini allo zero.

Due numeri infinitamente vicini, sono diversi tra di loro, ma la loro 
differenza è minore di qualunque numero reale positivo.

\subsection{Iperreali finiti e parte standard}
\label{subsec:insnum_partestandard}

Tra i vari tipi di Iperreali, hanno un ruolo particolare gli Iperreali 
finiti perché sono quelli che assomigliano di più ai numeri che già 
conosciamo e possono essere facilmente tradotti in numeri reali e 
approssimati con numeri razionali. 

\begin{definizione}
 Un numero iperreale si dice \textbf{finito} se è un numero compreso tra 
due numeri Reali:

\[\text{Se } x \in \IR \sand a, b \in \R \sand 
  a < x < b \quad \text{ allora } x \text{ è un Iperreale finito.}\]
\end{definizione}

\begin{esempio}
 Individua quali dei seguenti numeri sono finiti (considerando, per 
semplicità \(\epsilon\) positivo):

\begin{multicols}{3}
\begin{enumerate}
 \item \(8+5\epsilon\)
 \item \(\tonda{8+5\epsilon}^2\)
 \item \(8+\dfrac{5}{\epsilon}\)
\end{enumerate}
\end{multicols}

Vediamo i tre casi:

\begin{enumerate}
 \item \(\tonda{8+5\epsilon}\) è un numero finito perché:
 \[\tonda{8-\dfrac{1}{10^6}} < \tonda{8+5\epsilon} < 
   \tonda{8+\dfrac{1}{10^6}}\]
 \item Eseguiamo il quadrato:
 \(\tonda{8+5\epsilon}^2 = 64 +80 \epsilon +25 \epsilon^2\)
 ma: \(80 \epsilon\) è sicuramente un infinitesimo e anche \(25 \epsilon^2\)
 lo è e sarà un infinitesimo anche la loro somma, chiamiamo \(\delta\) 
questa somma quindi: 
\(\tonda{8+5\epsilon}^2 = 64 + \delta\)
e: 
\[63 < \tonda{64 + \delta} < 65\]
 \item Nell'ultimo caso possiamo osservare che, essendo \(\epsilon\) in 
valore assoluto minore di qualunque numero reale, 
 \(\dfrac{5}{\epsilon}\) è un numero maggiore di qualunque numero reale e 
la somma di~8 più un numero maggiore di qualunque altro, non può essere 
minore di un determinato numero reale:
 \[\nexists y \in \R \quad | \quad 8+\dfrac{5}{\epsilon} < y\]
perciò \(8+\dfrac{5}{\epsilon}\) non è un numero finito.
\end{enumerate}
\end{esempio}

\noindent Ogni numero finito può essere visto come un numero \emph{reale} 
più un \emph{infinitesimo}.

Se $x$ è finito allora $x = a + \epsilon$ dove:
\begin{itemize} [noitemsep]
 \item \(x\) è un numero iperreale finito;
 \item \(a\) è un numero reale;
 \item \(\epsilon\) è un infinitesimo (anche zero).
\end{itemize}

Se \(x=a+\epsilon\) allora potremmo dire che \(x\) è infinitamente vicino 
ad \(a\) infatti la differenza tra i due dà un infinitesimo: 
\[x=a+\epsilon \sLRarrow
x-a = a+\epsilon-a \sLRarrow x-a = \epsilon \sLRarrow x \approx a\] 

Un numero Iperreale finito non può essere infinitamente vicino 
a due numeri reali diversi (perché?) quindi esiste un solo numero Reale 
infinitamente vicino ad un dato numero Iperreale. 
Questo numero reale si chiama \emph{parte standard} del numero Iperreale.

% \begin{definizione}
%  La parte standard di un numero Iperreale finito è l'unico numero Reale 
% infinitamente vicino:
% 
% \[\text{Se } x \in \IR \wedge x= a+\epsilon \text{ allora } 
% \st(x) = a \text{ (st(x) = parte standard).}\]
% \end{definizione}

\begin{definizione}
 Si dice che \(a\) è la \textbf{parte standard} di \(x\), e si scrive: 
 \(\st(x) = a\), se \(a\) è un numero reale e \(x\) è 
infinitamente vicino ad \(a\):

\[\st(x) = a \sLRarrow a \in \R \sand x \approx a\]
\end{definizione}

\begin{osservazione}
\begin{itemize} [nosep]
 \item 
La parte standard di un infinitesimo è zero infatti:
\(\epsilon = 0+\epsilon\).
 \item 
Un numero iperreale infinito non ha parte standard poiché non esiste nessun 
numero reale infinitamente vicino a un infinito.
\end{itemize}
\end{osservazione}

Si può immaginare ogni Iperreale finito come una
nuvola contenente un numero standard $a$ e tutti gli infinitesimi che lo 
circondano, così vicini ad $a$ da non potersi confondere con gli altri 
infiniti iperreali di una nuvola vicina, appartenenti per esempio al numero 
iperreale $y=b+\delta$. 

% TODO
%  \begin{esempio}
%   $3\epsilon +5 +6M -2\epsilon +7 -2M = 4M +12 +\epsilon \quad (tipo=I)$
%  \end{esempio}


\subsection{Retta Iperreale e strumenti ottici}
\label{subsec:insnum_retta}

In un paragrafo precedente abbiamo visto che si può accettare l'idea che 
ad ogni numero reale corrisponda un punto della retta e ad ogni 
punto della retta corrisponda un numero reale. 
Questa affermazione non è un teorema dimostrato, è un postulato. 
Fa parte del modello di numeri usato, questa idea è caratteristica dei 
numeri reali. 
Ma dato che ora stiamo cambiando modello, cambiamo anche questo postulato. 
Lo riformuliamo così:

\begin{postulato}
Ad ogni numero Iperreale corrisponde un punto della retta (iperreale) e ad 
ogni punto della retta (iperreale) corrisponde un numero Iperreale.
\end{postulato}

Oppure:

\begin{postulato}[Retta iperreale]
C'è una corrispondenza biunivoca tra i numeri Iperreali e 
i punti della retta (iperreale).
\end{postulato}

Abbiamo già una certa abitudine a rappresentare numeri reali sulla retta, 
per rappresentare i numeri Iperreali dobbiamo procurarci degli strumenti 
particolari: \emph{microscopi}, \emph{telescopi}, \emph{grandangoli}.

Diamo una sbirciata al loro manuale di istruzioni.

\subsubsection{Microscopi}
\label{subsec:insnum_microscopio}

Il microscopio permette di ingrandire una porzione di retta. 
Per esempio un microscopio permette di visualizzare i seguenti numeri:

\begin{esempio}

\begin{multicols}{4}
\begin{itemize}[nosep]
 \item $+4,998$
 \item $-3,000002$
 \item $2-3\epsilon$
 \item $-4+2\delta$
\end{itemize}
\end{multicols}

\begin{inaccessibleblock}[Applicazione di diversi microscopi.]
\begin{minipage}{.48\linewidth}
 \begin{center}
\scalebox{0.8}{\microscopioa}
 \end{center}
% \caption{Microscopio per vedere \(5,004\).} \label{fig:microscopioa}
\end{minipage}
\hfill
\begin{minipage}{.48\linewidth}
 \begin{center}
\scalebox{0.8}{\microscopiob}
 \end{center}
% \caption{Microscopio per vedere \(-3,000002\).} \label{fig:microscopiob}
\end{minipage}

\begin{minipage}{.48\linewidth}
 \begin{center}
\scalebox{0.8}{\microscopioc}
 \end{center}
% \caption{Microscopio per \emph{non} vedere \(2-3\epsilon\).} 
\label{fig:microscopioc}
\end{minipage}
\hfill
\begin{minipage}{.48\linewidth}
 \begin{center}
\scalebox{0.8}{\microscopiod}
 \end{center}
% \caption{Microscopio per vedere \(2-3\epsilon\).} \label{fig:microscopiod}
\end{minipage}
\end{inaccessibleblock}
\end{esempio}
 
Si può osservare come ci siano microscopi ``standard'' che ingrandiscono un 
numero \emph{naturale} di volte e microscopi ``non standard'' che 
ingrandiscono infinite volte (ricordiamoci che \(\frac{1}{\epsilon}\) è un 
infinito.

\subsubsection{Telescopi}
\label{subsec:insnum_telescopi}

Il telescopio permette di avvicinare una porzione di retta senza cambiare 
la sua scala. Con un telescopio possiamo visualizzare i seguenti numeri:

\begin{esempio} ~

\begin{multicols}{4}
\begin{itemize}[nosep]
 \item $+127034$
 \item $-3600$
 \item $A+3$
 \item $-B+2$
\end{itemize}
\end{multicols}
\vspace{-5mm}

\begin{inaccessibleblock}[Applicazione di diversi telescopi.]
\begin{minipage}{.48\linewidth}
 \begin{center}
\scalebox{0.8}{\telescopioa}
 \end{center}
% \caption{Telescopio per vedere \(127034\).} \label{fig:telescopioa}
\end{minipage}
\hfill
\begin{minipage}{.48\linewidth}
 \begin{center}
\scalebox{0.8}{\telescopiob}
 \end{center}
% \caption{Telescopio per vedere \(A+3\).} \label{fig:telescopiob}
\end{minipage}
\end{inaccessibleblock}
\end{esempio}

Anche per i telescopi, i modelli più moderni offrono la possibilità di 
operare ingrandimenti ``standard'' o ``non standard'' a piacere.

\subsubsection{Grandangoli (Zoom)}
\label{subsec:insnum_zoom}

Il Grandangolo permette di cambiare la scala della visualizzazione della 
retta, in questo modo possiamo far rientrare nel campo visivo anche numeri 
molto lontani.
Possiamo usare uno zoom per visualizzare i seguenti numeri:

\begin{esempio}~

\begin{multicols}{4}
\begin{itemize}[nosep]
 \item $300$
 \item $-5000$
 \item $-2A$
 \item $3B$
\end{itemize}
\end{multicols}
\vspace{-5mm}
\begin{inaccessibleblock}[Applicazione di diversi grandangoli.]
\begin{minipage}{.48\linewidth}
 \begin{center}
\scalebox{0.8}{\grandangoloa}
 \end{center}
% \caption{Grandangolo per vedere \(300\).} \label{fig:grandangoloa}
\end{minipage}
\hfill
\begin{minipage}{.48\linewidth}
 \begin{center}
\scalebox{0.8}{\grandangolob}
 \end{center}
% \caption{Grandangolo per vedere \(-2A\).} \label{fig:grandangolob}
\end{minipage}
\end{inaccessibleblock}
\end{esempio}

Anche per i grandangoli utilizzeremo versioni che permettono zoomate 
``standard'' e ``non standard''.

\begin{inaccessibleblock}[Combinazione di diversi strumenti.]
\begin{minipage}{.38\linewidth}
\subsubsection{Combinazione di strumenti}
\label{subsec:insnum_combinazione}

Questi strumenti sono ``modulari'', possono essere combinati a piacere. 
Per esempio per visualizzare il numero non standard: 
\(1741,998 +2\epsilon\) posso utilizzare in sequenza un telescopio per 
avvicinarmi al numero, un microscopio standard per poter vedere i 
millesimi e un microscopio non standard per vedere il numero infinitamente 
vicino al numero standard.
\end{minipage}
\hfill
\begin{minipage}{.58\linewidth}
 \begin{center}
\scalebox{0.7}{\combinazione}
 \end{center}
\end{minipage}
\end{inaccessibleblock}

\vspace{-5mm}

\subsection{Operazioni}
\label{subsec:insnum_operazioni}

Vediamo di seguito alcune regole relative alle operazioni 
che valgono nei numeri Iperreali.

\subsubsection{Addizione}
\label{subsec:insnum_addizione}

\begin{multicols}{2}
Alcune osservazioni:

\begin{enumerate} [noitemsep]
 \item Le regole relative all'addizione valgono anche per la sottrazione, 
se 
uno degli addendi è negativo. 
 \item Zero è l'elemento neutro dell'addizione nei Reali e continua ad 
esserlo 
anche negli Iperreali: $x+0=0+x=x$.
 \item Un infinitesimo più un altro infinitesimo dà per risultato un 
infinitesimo: $\alpha+\beta=\gamma$.
 \item Un infinitesimo non nullo più un altro infinitesimo non nullo può 
dare 
per risultato anche zero: \dots
 \item Un finito più un infinitesimo dà come risultato un finito.
 \item Un finito più un finito dà come risultato un finito.
 \item Un finito più un finito può dare come risultato un infinitesimo.
%  \item Un infinito più un infinitesimo dà come risultato un infinito.
 \item Un infinito più un finito dà come risultato un infinito.
 \item Un infinito più un infinito può dare come risultato zero, un 
infinitesimo, un finito non infinitesimo, un infinito.
\end{enumerate}

Nel precedente elenco abbiamo visto che alcune addizioni danno un risultato 
che dipende solo dai tipi degli operandi, altre operazioni danno dei 
risultati 
che dipendono dal valore degli operandi. Possiamo costruire una tabella che 
organizza le precedenti osservazioni.

\begin{center}
\renewcommand{\arraystretch}{.0}
\scalebox{0.8}{
\begin{tabular}{p{.7cm}|p{.7cm}|p{.7cm}|p{.7cm}|p{.7cm}|p{.7cm}|}
\centra{$+$} & \centra{0} & \centra{inn} & \centra{fni} & \centra{I} 
\\\hline
\centra{0} & \centra{0} & \centra{inn}& \centra{fni} & \centra{I} \\\hline
\centra{inn} & \centra{inn} & \centra{i}& \centra{fni} & \centra{I} \\\hline
\centra{fni} & \centra{fni} & \centra{fni}& \centra{f} & \centra{I} \\\hline
\centra{I} & \centra{I} & \centra{I}& \centra{I} & \centra{?} \\\hline
\end{tabular}}
\end{center}
% \vspace{2mm}
\end{multicols}

\subsubsection{Moltiplicazione}
\label{subsec:insnum_moltiplicazione}

\begin{multicols}{2}
Alcune osservazioni:
\begin{enumerate} [noitemsep]
 \item Zero è l'elemento assorbente: il prodotto di un iperreale per zero
dà come risultato zero: $x \cdot 0=0 \cdot x=0$.
 \item Uno è l'elemento neutro della moltiplicazione nei Reali e continua 
ad 
esserlo anche negli Iperreali: $x \cdot 1=1 \cdot x=x$.
 \item Un infinitesimo per un altro infinitesimo dà per risultato un 
infinitesimo: $\alpha \cdot \beta=\gamma$.
 \item Un infinitesimo non nullo per un altro infinitesimo non nullo dà 
per risultato un infinitesimo non nullo.
 \item \dots
 \item \dots
%  \item Un finito per un infinitesimo dà come risultato un finito.
%  \item Un finito per un finito dà come risultato un finito.
%  \item Un finito per un finito può dare come risultato un infinitesimo.
% %  \item Un infinito per un infinitesimo dà come risultato un infinito.
%  \item Un infinito per un finito dà come risultato un infinito.
%  \item Un infinito per un infinito può dare come risultato zero, un 
% infinitesimo, un finito non infinitesimo, un finito, un infinito;
 \item Il prodotto fra un finito e un infinitesimo richiama le osservazioni 
fatte sul postulato di Eudosso-Archimede.
 \end{enumerate}
E la tabella corrispondente:
\begin{center}
\renewcommand{\arraystretch}{.0}
\scalebox{0.8}{
\begin{tabular}{p{.7cm}|p{.7cm}|p{.7cm}|p{.7cm}|p{.7cm}|p{.7cm}|}
\centra{$\times$} & \centra{0} & \centra{1} & 
\centra{inn} & \centra{fni} & \centra{I} \\\hline
\centra{0} & \centra{0} & \centra{0} & 
\centra{0}& \centra{0} & \centra{0} \\\hline
\centra{1} & \centra{0} & \centra{1} & 
\centra{inn} & \centra{fni} & \centra{I} \\\hline
\centra{inn} & \centra{0} & \centra{inn} & 
\centra{inn}& \centra{inn} & \centra{?} \\\hline
\centra{fni} & \centra{0} & \centra{fni} & 
\centra{inn}& \centra{fni} & \centra{I} \\\hline
\centra{I} & \centra{0} & \centra{I} & 
\centra{?}& \centra{I} & \centra{I} \\\hline
\end{tabular}}
\end{center}
\end{multicols}

\subsubsection{Divisione}
\label{subsec:insnum_divisione}

\begin{multicols}{2}
Alcune osservazioni:
\begin{enumerate} [noitemsep]
 \item Anche negli Iperreali la divisione per zero non è definita.
 \item Uno può essere visto come un elemento neutro solo destro: $x \div 
1=x$.
 \item Per cercare i risultati possiamo rifarci alla definizione di 
quoziente.
 \item \dots
\end{enumerate}
E la tabella corrispondente:
\begin{center}
\renewcommand{\arraystretch}{.0}
\scalebox{0.8}{
\begin{tabular}{p{.7cm}|p{.7cm}|p{.7cm}|p{.7cm}|p{.7cm}|p{.7cm}|}
% \begin{tabular}{c|c|c|c|c|c|}
\centra{$\div$} & \centra{0} & \centra{1} & 
\centra{inn} & \centra{fni} & \centra{I} \\\hline
\centra{0} &  & \centra{0} & 
\centra{0}& \centra{0} & \centra{0} \\\hline
\centra{1} &  & \centra{1} & 
\centra{I} & \centra{fni} & \centra{inn} \\\hline
\centra{inn} &  & \centra{inn} & 
\centra{?}& \centra{inn} & \centra{inn} \\\hline
\centra{fni} &  & \centra{fni} & 
\centra{I}& \centra{fni} & \centra{inn} \\\hline
\centra{I} &  & \centra{I} & 
\centra{I}& \centra{I} & \centra{?} \\\hline
\end{tabular}}
\end{center}
\end{multicols}

\subsubsection{Reciproco}
\label{subsec:insnum_reciproco}

\begin{multicols}{2}
Alcune osservazioni:
\begin{enumerate} [noitemsep]
 \item Dalla tabella precedente si può estrarre la riga corrispondente a~1
e si ottiene la tabella del reciproco.
 \item Una volta convinti della regola del reciproco, si può ricavare la 
tabella della divisione attraverso la 
regola:\\
$x : y = x \cdot \frac{1}{y}$.
\end{enumerate}
E la tabella corrispondente:
\begin{center}
\renewcommand{\arraystretch}{.0}
\scalebox{0.8}{
\begin{tabular}{p{1.7cm}|p{.7cm}|p{.7cm}|p{.7cm}|p{.7cm}|p{.7cm}|}
\centra{numero} & \centra{0} & \centra{1} & 
\centra{inn} & \centra{fni} & \centra{I} \\\hline
\centra{reciproco} &  & \centra{1} & 
\centra{I} & \centra{fni} & \centra{inn} %\\\hline
\end{tabular}}
\end{center}
% \vspace{11mm}
\end{multicols}

\begin{osservazione}
Non ci sono regole immediate per le seguenti operazioni:
\begin{multicols}{4}
\begin{itemize} [nosep]
 \item \(\dfrac{\epsilon}{\delta}\)
 \item \(\dfrac{A}{B}\)
 \item \(A \cdot \epsilon\)
 \item \(A + B\)
\end{itemize}
\end{multicols}
In questi casi il tipo di risultato dipende dall'effettivo valore degli 
operandi. Ad esempio, nel caso del quoziente tra due infinitesimi possiamo 
trovarci nelle seguenti situazioni:
\begin{multicols}{3}
\begin{itemize} [nosep]
 \item \(\dfrac{\epsilon^2}{\epsilon} = \epsilon\) \quad (i)
 \item \(\dfrac{2\epsilon}{\epsilon} = 2\) \quad (fni)
 \item \(\dfrac{\epsilon}{\epsilon^2} = \frac{1}{\epsilon}\) \quad (I)
\end{itemize}
\end{multicols}
\end{osservazione}

Possiamo ora esercitarci nel calcolo con questi nuovi numeri. 
Continuiamo ad utilizzare la convenzione di indicare gli 
infinitesimi con lettere greche minuscole
($\alpha,~\beta,~\gamma,~\delta,~\epsilon,~\dots$), 
i finiti non infinitesimi con lettere latine minuscole 
($a,~b,~c~,~\dots,~m,~n,~\dots$) 
e gli infiniti con lettere latine maiuscole 
($A,~B,~C~,~\dots,~M,~N,~\dots$).

\begin{exrig}
Semplifichiamo le seguenti espressioni scrivendo il tipo di risultato 
ottenuto.

 \begin{esempio}
  $3\epsilon +5 +6M -2\epsilon +7 -2M = 4M +12 +\epsilon \quad (tipo=I)$
 \end{esempio}

\begin{osservazione}
Quando il risultato è la somma di più elementi, li scriviamo, ordinandoli 
dal più grande, in valore assoluto, al più piccolo.
%  Quando possibile scriviamo un risultato composto da più elementi 
scrivendoli 
% dal più grande, in valore assoluto, al più piccolo.
\end{osservazione}

 \begin{esempio}
\(7 +8M -5\epsilon  -4 +3\epsilon-2N = 8M -2N +3 -2\epsilon\)
\quad (tipo non definito)
 \end{esempio}
 
 \begin{esempio}
\(\tonda{3M +2\epsilon} \tonda{3M -2\epsilon} = 9M -4\epsilon\)
\quad (tipo=I)
 \end{esempio}
 
 \begin{esempio}
\(\tonda{M +3} \tonda{M -3} - \tonda{M+2}^2 +4\tonda{M +3}=\)

\(=M^2 -9 -M^2 -4M -4 +4M +12 = -1\)
\quad (tipo=fni)
 \end{esempio}
 
 \begin{esempio}
\(10a -\tonda{A +1}^2 -3a +2\tonda{a+2\alpha} +A^2 +6\tonda{b -3\alpha} 
+2A= 
\)

\(=10a -A^2 -2A -1 -3a +2a+4\alpha +A^2 +6b -18\alpha +2A = 9a +6b 
-14\alpha\) 
\quad (tipo=fni)
 \end{esempio}
\end{exrig}

% \subsection{Espressioni}
% \label{subsec:insnum_espressioni}

\subsection{Confronto}
\label{subsec:insnum_confronto}

L'insieme dei numeri Reali ha un ordinamento completo, se $a$ e $b$ sono 
due numeri reali qualunque è sempre valida una e una sola delle seguenti 
affermazioni:

\[a<b \quad a=b \quad b<a\]

Per confrontare due numeri Reali possiamo utilizzare le seguenti regole:

\begin{enumerate} [noitemsep]
 \item qualunque numero negativo è minore di qualunque numero positivo;
 \item se due numeri sono negativi, è minore quello che ha il modulo 
maggiore;
 \item se $a$ e $b$ sono due numeri positivi, 
 \[a<b \sLRarrow a-b<0 \quad \text{ (o } \quad b-a>0 \text{ )}\]
oppure
 \[a<b \sLRarrow \frac{a}{b}<1 \quad \text{ (o } \quad \frac{b}{a}>1 
   \text{ )}\]
\end{enumerate}

\begin{osservazione}
Le prime due regole ci permettono di restringere le nostre riflessioni al 
solo caso del confronto tra numeri positivi.
Nei prossimi paragrafi assumeremo che le variabili si riferiscano solo a 
numeri positivi.
\end{osservazione}

\begin{osservazione}
Nella terza regola abbiamo presentato due criteri. Quello usato di solito
è il primo, ma useremo anche il secondo perché il rapporto tra due 
grandezze permette di ottenere informazioni interessanti.
\end{osservazione}

\vspace{1em}

Anche negli Iperreali valgono le proprietà dei Reali richiamate sopra. 
Ma l'insieme degli Iperreali non ha un ordinamento completo:
se di $\epsilon$ e $\delta$ sappiamo solo che sono due infinitesimi,
non è possibile dire se $\epsilon < \delta$ o $\epsilon > \delta$.
E questo si ripercuote anche su tutti gli altri numeri: senza ulteriori 
informazioni non possiamo dire se $a+\epsilon$ è maggiore minore o uguale 
a $a+\delta$. 
Problemi analoghi si incontrano nel confronto degli infiniti.
Vediamo allora come è possibile affrontare il problema del confronto tra 
Iperreali.

Restringendo l'osservazione ai numeri positivi possiamo affermare che gli 
infinitesimi sono più piccoli dei non infinitesimi e i finiti sono più 
piccoli degli infiniti:

\[i \quad < \quad fni \quad < \quad I\]

Passiamo ora al confronto all'interno dei diversi tipi di numeri Iperreali.

\subsubsection{Confronto tra finiti non infinitesimi}
\label{subsubsec:insnum_confrontoreali}

Se due numeri Iperreali hanno parte standard diversa allora è maggiore 
quello 
che ha la parte standard maggiore:
\[x < y \sLRarrow \st(x) < st(y)\]
Nel caso i due numeri abbiano la stessa parte standard si deve studiare 
l'ordinamento degli infinitesimi, cosa che faremo nel prossimo paragrafo.

\subsubsection{Confronto tra infinitesimi}
\label{subsubsec:insnum_confrontoreali}

Di seguito vediamo i diversi casi in cui ci possiamo 
imbattere quando vogliamo confrontare i numeri infinitesimi.

\paragraph{Zero}
Zero è minore di qualunque infinitesimo positivo:
\[\epsilon-0 = \epsilon>0\]
\paragraph{Somma}
La somma di infinitesimi positivi è maggiore di ognuno dei due:
\[\tonda{\epsilon+\delta}-\epsilon = \delta > 0\]
\paragraph{Multiplo}
Il multiplo di un infinitesimo positivo è maggiore dell'infinitesimo di 
partenza. Usando il primo metodo per il confronto:
\[\forall n > 1 \quad n\epsilon-\epsilon = \tonda{n-1}\epsilon > 0\]
e usando il secondo metodo: 
\[\forall n > 1 \quad \frac{n\epsilon}{\epsilon} = n > 1\]
\paragraph{Sottomultiplo}
Il sottomultiplo di un infinitesimo è minore dell'infinitesimo di partenza. 
Usando il primo metodo per il confronto:
\[\forall n > 1 \quad \frac{\epsilon}{n}-\epsilon = 
                      \frac{\epsilon-n\epsilon}{n} = 
                      \frac{\tonda{1-n}\epsilon}{n} < 0\]
e usando il secondo metodo: 
\[\forall n > 1 \quad \frac{\epsilon}{n}:\epsilon =
                      \frac{\epsilon}{n\epsilon} =
                      \frac{1}{n} < 1\]
\begin{definizione}
 Diremo che \(\gamma\) e \(\epsilon\) sono \textbf{infinitesimi dello 
stesso ordine} se il rapporto tra \(\gamma\) e \(\epsilon\) è un 
finito non infinitesimo.
\end{definizione}
\paragraph{Parte infinitesima di un infinitesimo}
La parte infinitesima di un infinitesimo positivo è minore 
dell'infinitesimo di partenza:
\[\frac{\epsilon \delta}{\epsilon} = \delta < 1\]
% \begin{osservazione}
In questo caso il rapporto non solo è più piccolo di~1 ma è addirittura un 
\emph{infinitesimo}, 
cioè~\(\gamma\) è una parte infinitesima di~\(\epsilon\). 
% Quando il rapporto tra due infinitesimi è un infinitesimo 
% cioè se~\(\gamma\) è un infinitesimo di \(\epsilon\)
In questo caso 
si dice che \(\gamma\) è un infinitesimo di \emph{ordine superiore} a 
\(\epsilon\) e si scrive:
\[\gamma=o(\epsilon)\]
% \end{osservazione}
\begin{definizione}
 Diremo che \(\gamma\) è un \textbf{infinitesimo di ordine superiore} a 
\(\epsilon\) se il rapporto tra \(\gamma\) e \(\epsilon\) è un infinitesimo:
\[\gamma=o(\epsilon) \sLRarrow \frac{\gamma}{\epsilon}=\delta\]
Diremo anche che 
\(\epsilon\) è un \textbf{infinitesimo di ordine inferiore} a \(\gamma\).
\end{definizione}

\subsubsection{Confronto tra infiniti}
\label{subsubsec:insnum_confrontoreali}

Anche tra gli infiniti possiamo effettuare il confronto calcolando la 
differenza tra due numeri o il quoziente e anche tra gli infiniti l'uso del 
quoziente ci dà delle informazioni interessanti.

% \paragraph{Infinito più infinitesimo}
% Confrontiamo \(M+\epsilon\) con \(M\). 
% Usando il primo metodo:
% \[M+\epsilon-M = \epsilon > 0\]
% e usando il secondo metodo: 
% \[\frac{M+\epsilon}{M} =
%   \frac{M}{M} \frac{\epsilon}{M} = 
%   1 + \frac{\epsilon}{M} > 1\]

\paragraph{Infinito più finito}
Se \(a\) è un finito positivo (anche infinitesimo), confrontiamo \(M+a\) 
con \(M\). 
Usando il primo metodo:
\[M+a-M = a > 0\]
e usando il secondo metodo: 
\[\frac{M+a}{M} =
  \frac{M}{M} + \frac{a}{M} = 
  1 + \frac{a}{M} > 1\]
\paragraph{Somma di infiniti}
Se \(M\) e \(N\) sono due infiniti positivi, confrontiamo \(M+N\) 
con \(M\). 
Usando il primo metodo:
\[M+N-M = N > 0\]
e usando il secondo metodo: 
\[\frac{M+N}{M} =
  \frac{M}{M} + \frac{N}{M} = 
  1 + \frac{N}{M} > 1\]
\paragraph{Multiplo}
Se \(n>1\), confrontiamo \(nM\) con \(M\). 
Usando il primo metodo:
\[\forall n>1 \quad nM-M = \tonda{n-1}M > 0\]
e usando il secondo metodo: 
\[\frac{nM}{M} = n > 1\]
\begin{definizione}
 Diremo che \(M\) e \(N\) sono \textbf{infiniti dello stesso ordine}  
se il rapporto tra \(M\) e \(N\) è un finito non infinitesimo.
\end{definizione}
\paragraph{Infinito di infinito}
Confrontiamo \(MN\) con \(M\). 
Usando il primo metodo:
\[MN-M = \tonda{N-1}M > 0\]
e usando il secondo metodo: 
\[\frac{MN}{M} = N > 1\]
% \begin{osservazione}
In questo caso il rapporto non solo è più maggiore di~1 ma è addirittura un 
\emph{infinito}. 
\begin{definizione}
 Diremo che \(M\) è un \textbf{infinito di ordine superiore} a 
\(N\) se il rapporto tra \(M\) e \(N\) è un infinito:
\[M=\omega(N) \sLRarrow \frac{M}{N}=I\]
Diremo anche che 
\(N\) è un \textbf{infinito di ordine inferiore} a \(M\).
\end{definizione}
% Quando il rapporto tra due infiniti è un infinito cioè 
% se~\(A\) vale infinite volte~\(B\) diremo che~\(A\) è un infinito di 
% \emph{ordine superiore} a~\(B\).
% \end{osservazione}
% 
% \vspace{1em}
% Ora confrontiamo \(M\) con \(MN\) usando il secondo metodo: 
% \[\frac{M}{MN} = \frac{1}{N} = \epsilon < 1\]
% 
% \begin{osservazione}
% In questo caso il rapporto non solo è minore di~1 ma è addirittura un 
% \emph{infinitesimo}. 
% Quando il rapporto tra due infiniti è un infinitesimo cioè 
% se~\(B\) vale infinite volte~\(A\) diremo che~\(A\) è un infinito di 
% \emph{ordine inferiore} a~\(B\).
% \end{osservazione}

\vspace{1em}
A volte il confronto tra due Iperreali è meno immediato dei casi precedenti:
\begin{esempio}
 Confrontare \(M\) e \(2^M\). 
 Dobbiamo calcolare: \(\frac{M}{2^M}\). 
Possiamo usare un duplice trucco: 
\begin{itemize} [nosep]
 \item invece di confrontare \(M\) e \(2^M\) confrontiamo \(M^2\) e \(2^M\);
 \item invece che confrontare direttamente i due valori richiesti, vediamo 
come si comportano, con numeri naturali piccoli, le due funzioni 
\(y_1=x^2\) e \(y_2=2^x\)
\end{itemize}
% invece che confrontare direttamente i due 
% valori 
% richiesti, vediamo come si comportano, con numeri naturali piccoli le due 
% funzioni  
% % \(y_1=\angolare{x^2}\) e \(y_2=\angolare{2^x}\):
% \(y_1=x^2\) e \(y_2=2^x\):
\begin{center}
\begin{tabular}{cccccccc}
$x^2$ & 0 & 1 & 4 & 9 & 16 & 25 & 36\\
$2^x$ & 1 & 2 & 4 & 8 & 16 & 32 & 64
\end{tabular}
\end{center}
Possiamo vedere che dal quinto elemento in poi la prima successione è sempre
maggiore della seconda ed essendo l'infinito più grande di cinque 
otteniamo che \(2^M > M^2\) quindi possiamo scrivere:
\[\frac{M}{2^M} < \frac{M}{M^2} = \frac{1}{M} < 1\]

Ma \(\frac{1}{M}\) è un infinitesimo quindi possiamo affermare che $M$ è un 
infinito di ordine inferiore a $2^M$.
\end{esempio}

In conclusione, possiamo confrontare fra di loro i numeri Iperreali 
utilizzando la differenza o il quoziente tra i numeri. L'uso del quoziente 
ci permette di ricavare un'informazione interessante l'ordine di 
infinitesimo o di infinito.
\begin{itemize} [noitemsep]
 \item un infinitesimo di ordine superiore è un infinitesimo infinitamente 
 più piccolo;
 \item un infinito di ordine superiore è un infinito infinitamente più 
grande.
\end{itemize}

\subsection{Indistinguibili}
\label{subsec:insnum_indistinguibili}

Quando risolviamo un problema pratico, a noi serve, alla fine dei calcoli, 
ottenere un numero razionale, con un certo numero di cifre significative.
È chiaro che se il risultato di un calcolo è~\(4,37+5\epsilon\) sostituire 
questo risultato con il più semplice~4,37 non ci fa perdere in precisione, 
in questo caso~\(5\epsilon\) può essere trascurato.
Ben diverso è se all'interno di un calcolo  
otteniamo:~\(\epsilon+5\epsilon\), in questo caso non posso 
trascurare~\(5\epsilon\) anche se è una quantità infinitesima. 

In certi casi posso avere due espressioni diverse che, in prima 
approssimazione, possono essere considerate equivalenti. Quando è così dirò 
che i due numeri iperreali sono \emph{indistinguibili}.
Due numeri sono indistinguibili quando la differenza tra i due è 
infinitesima rispetto a ciascuno dei due.

\begin{definizione}
Due numeri si dicono \textbf{indistinguibili} (simbolo:~$\sim$) se il 
rapporto tra la loro differenza e ciascuno di essi è un infinitesimo:
\[x \sim y \sLRarrow 
\tonda{\frac{y-x}{x} = \epsilon \quad \wedge \quad \frac{y-x}{y} = \delta}
\]
\end{definizione}

\begin{osservazione}
 È importante osservare che per poter applicare la definizione entrambi i 
numeri che vogliamo confrontare devono essere diversi da \emph{zero}.
Cioè nessun numero diverso da zero può essere considerato indistinguibile 
da zero:~\(\nexists~x \neq 0 ~|~x \sim 0\)
\end{osservazione}

Di seguito esploriamo i tre casi possibili.

\subsubsection{Finiti non infinitesimi}
\label{subsubsec:insnum_finitini}

Se due numeri finiti non infinitesimi differiscono per un 
infinitesimo, sono indistinguibili.

\begin{teorema}
Due numeri $x$ e $y$, finiti non infinitesimi, 
sono indistinguibili se e solo se sono infinitamente vicini:
\[x \approx y \sLRarrow x \sim y\]
\end{teorema}

\begin{proof}
Iniziamo dimostrando che se sono infinitamente vicini allora sono 
indistinguibili:
\begin{center}
Ipotesi: $\tonda{x,~y:\ fni \sand y = x+\epsilon} \qquad \sLRarrow \qquad$ 
Tesi: $x \sim y$.
\end{center}
Dimostrazione
\[\frac{y-x}{x}=\frac{x-\tonda{x+\epsilon}}{x} = 
\frac{\epsilon}{x}= \gamma \quad \wedge \quad 
\frac{y-x}{x}=\frac{x-\tonda{x+\epsilon}}{y} = 
\frac{\epsilon}{y}= \delta
\]
Il teorema inverso dirà:
\begin{center}
Ipotesi: $\tonda{x,~y:\ fni \sand x \sim y} \qquad \sLRarrow \qquad$ 
Tesi: $x \approx y$.
\end{center}
Dimostrazione
\[\frac{y-x}{x} = \epsilon
\sRarrow y-x=\epsilon x \sRarrow x=y+\epsilon x=y+\beta 
\]
\end{proof}

\subsubsection{Infinitesimi}
\label{subsubsec:insnum_infinitesimi}

Per quanto riguarda gli infinitesimi, non basta che siano infinitamente 
vicini infatti tutti gli infinitesimi sono infinitamente vicini tra di loro.
Per essere indistinguibili serve una condizione più ristretta.

\begin{teorema}
Due numeri \(\alpha\) e \(\beta\), infinitesimi, 
sono indistinguibili se e solo se la loro differenza è un infinitesimo di 
ordine superiore.
\[\beta-\alpha = o(\alpha) \sLRarrow \alpha \sim \beta\] 
\end{teorema}

\begin{osservazione}
 Se due infinitesimi differiscono per un infinitesimo di ordine superiore 
allora sono dello stesso ordine, quindi la differenza sarà di ordine 
superiore sia al primo sia al secondo infinitesimo.
\end{osservazione}

\begin{proof}
Iniziamo dimostrando che se differiscono per un infinitesimo di ordine 
superiore allora sono indistinguibili:
\begin{center}
Ipotesi: $\tonda{\alpha,~\beta:\ inn \sand \beta = \alpha +o(\alpha)}
\qquad \sLRarrow\qquad$ 
Tesi: $\alpha \sim \beta$
\end{center}
Dimostrazione
\[\frac{\beta-\alpha}{\alpha}=
\frac{\alpha-\tonda{\alpha +o(\alpha)}}{\alpha} = 
\frac{o(\alpha)}{\alpha}= \gamma \quad \wedge \quad 
\frac{\beta-\alpha}{\beta}=
\frac{\alpha-\tonda{\alpha +o(\beta)}}{\beta} = 
\frac{o(\beta)}{\beta}= \delta
\]
Il teorema inverso dirà:
\begin{center}
Ipotesi: $\tonda{\alpha,~\beta:\ inn \sand \alpha \sim \beta}
\qquad \sLRarrow \qquad$ 
Tesi: \(\beta-\alpha = o(\alpha)\)
\end{center}
Dimostrazione
\[\frac{\beta - \alpha}{\alpha} = \epsilon \sRarrow 
\beta - \alpha =\epsilon \alpha \sRarrow 
\beta - \alpha = o(\alpha)
\]
\end{proof}

\subsubsection{Infiniti}
\label{subsubsec:insnum_infiniti}

La situazione si ribalta se i due numeri sono infiniti infatti, in 
questo caso, sono indistinguibili anche se differiscono di 
un valore finito o addirittura infinito. 

Si può dimostrare il seguente
\begin{teorema}
Due numeri \(M\) e \(N\), infiniti, 
sono indistinguibili se e solo se la loro differenza è un finito o 
un infinito di ordine inferiore.
\[M-N = a \sRarrow M \sim N\] 
e vale anche:
\[\tonda{M-N = P \text{ con P infinito di ordine inferiore }} 
\sRarrow M \sim N\] 
\end{teorema}
\begin{proof}
Di seguito dimostriamo che se differiscono per un finito allora 
sono indistinguibili:
\begin{center}
Ipotesi: $\tonda{M,~N:\ I \sand N = M+a}
\qquad \sLRarrow\qquad$ 
Tesi: $M \sim N$
\end{center}
Dimostrazione
\[\frac{M-N}{M}=
\frac{M-\tonda{M +a}}{M} = 
-\frac{a}{M}= \epsilon \quad \wedge \quad 
\frac{M-N}{N}=
\frac{M-\tonda{N +a}}{N} = 
-\frac{a}{N}= \delta
\]
In modo analogo si può procedere con la second parte del teorema.
\end{proof}

\subsection{Principio di tranfer}
\label{subsec:insnum_nonarchimedei}

Abbiamo applicato agli Iperreali le operazioni aritmetiche con grande 
naturalezza estendendo i metodi e i risultati che già conosciamo nei Reali. 
Ma è possibile fare ciò per qualunque funzione? 
Sì, è possibile assumere che per ogni funzione definita nei Reali esista 
una 
corrispondente funzione con dominio e codominio negli Iperreali che, 
ristretta ai Reali, coincida con la funzione reale.
In questo modo tutto quello che è possibile fare con i numeri Reali lo si 
può 
fare anche con gli Iperreali.

\begin{osservazione}
 Non vale il viceversa. Dato che gli Iperreali estendono i Reali, ci sono 
delle funzioni che, definite negli Iperreali, non hanno un valore 
corrispondente nei Reali. Ad esempio la funzione iperreale \emph{parte 
standard} non ha una funzione corrispondente nei numeri reali.

\begin{esempio}
 Consideriamo ad esempio la funzione: 
$f: x \mapsto \frac{1}{x} \quad \forall x \in \R$, definita per $x\ne 0$

È facile costruire la funzione $\effestar$ (\emph{effe star}) con dominio e 
codominio negli Iperreali:

$\effestar: x \mapsto \frac{1}{x} \quad \forall x \in \IR$, definita per 
$x\ne 0$.

Ogni volta che $\effestar$ è applicata a numeri standard (fni), si comporta 
come
la funzione $f$, applicata a $x \in \R$; ma, in più, la 
funzione~$\effestar$:
\begin{itemize} [noitemsep]
 \item 
è definita anche per valori infinitamente vicini a zero e 
in questo caso dà come risultato un valore infinito che non è un numero 
reale;
 \item 
è definita anche per valori infiniti e
in questo caso dà come risultato un valore infinitesimo che non è un numero 
reale. 
\end{itemize}
\end{esempio}
\end{osservazione}

\section{Applicazioni}
\label{sec:insnum_applicazioni}

Dopo aver dato un'occhiata a cosa sono e come funzionano i numeri iperreali 
vediamo qualche problema che si può convenientemente risolvere con gli 
Iperreali.

\subsection{Problemi con gli Iperreali}
\label{subsec:insnum_problemi}

\begin{esempio}
 % Quadrato di lato l+eps.
Calcola l'area iperreale di una cornice quadrata, di lato interno pari a 
$l$ e
spessore infinitesimo $\epsilon$. Calcola infine l'area reale.\\
Chiamiamo $dS$ l'area iperreale della cornice:
\(dS=\tonda{l+\epsilon^2}
-l^2=l^2+2l\epsilon+\epsilon^2-l^2=2l\epsilon+\epsilon^2.\)\\
Chiamiamo $\Delta S$ la corrispondente area reale:
\(\Delta S=\st\tonda{dS}=\st\tonda{2l\epsilon+\epsilon^2}=
\st(2l\epsilon)+\st(\epsilon^2)=0+0=0\).\\
Poiché la differenza di area $dS$ è la somma di due infinitesimi, uno del 
primo e 
l'altro del secondo ordine, la parte standard di entrambi è nulla e la 
somma 
risulta nulla. In conclusione, se l'incremento del lato è infinitesimo,
cioè infinitamente vicino a zero, a maggior ragione sarà infinitamente 
vicino
a zero l'incremento dell'area.
\end{esempio}

\begin{esempio}
 % contrazione circonferenza.
Calcola di quanto diminuisce rispetto al raggio una circonferenza di raggio 
$r$,
quando il raggio subisce una contrazione infinitesima $dr=-\epsilon$.\\
Chiamiamo $dC$ (differenza di C) la contrazione della circonferenza:
\(dC=2\pi r-2\pi (r-\epsilon)=2\pi r-2\pi r+2\pi \epsilon= 2\pi \epsilon\).
Dunque la circonferenza si riduce di un infinitesimo, cioè $0$ in numeri 
standard.
Ma se misuriamo la riduzione della circonferenza in termini
di riduzione del raggio, si ha:
\(\frac{dC}{dr}=\frac{2\pi \epsilon}{-\epsilon}=-2\pi\): ogni unità di 
variazione 
del raggio, comporta una variazione della circonferenza pari a $2\pi$.
\end{esempio}

\begin{esempio}
 % guscio sferico.
Quanto volume acquisisce un guscio sferico di raggio $r$ nel gonfiarsi
progressivamente?\\
Volume iniziale: $V(r)=\frac{4}{3}\pi r^3$. Se il raggio aumenta e diventa 
$r+\epsilon$, 
la variazione di volume sarà:\\
\(V(r+\epsilon)-V(r)=\frac{4}{3}\pi (r+\epsilon)^3-\frac{4}{3}\pi r^3=
\frac{4}{3}\pi (r^3+3r^2\epsilon+3r\epsilon^2+\epsilon^3-r^3)=
\frac{4}{3}\pi (3r^2\epsilon+3r\epsilon^2+\epsilon^3)\).\\
Per sapere quanto varia il volume per ogni variazione infinitesima di 
raggio, si calcola:\\
\(\frac{dV}{dr}=\frac{\frac{4}{3}\pi 
(3r^2\epsilon+3r\epsilon^2+\epsilon^3)}{\epsilon}=
\frac{4}{3}\pi (3r^2+3r\epsilon+\epsilon^2)\), che è un numero di tipo inn.
La sua parte standard è $\st\tonda{\frac{dV}{dr}}=4\pi r^2$. Nota che questa
è l'espressione dell'area della superficie sferica. Come era prevedibile, 
infatti, 
un guscio sferico di spessore infinitesimo approssima la superficie sferica.
\end{esempio}

\subsection{Espressioni con gli Iperreali}
\label{subsec:insnum_espressioni}

I numeri iperreali semplificano la ricerca della soluzione di molti 
problemi.
Il calcolo delle soluzioni ci porta a risultati espressi quasi sempre da 
numeri standard, che corrispondono ai reali. Infatti ,quasi sempre, il 
calcolo termina ricorrendo alla funzione $\st()$.\\
Questo metodo, cioè ricorrere ad un insieme più astratto di \(\R\), 
svolgervi i calcoli secondo le nuove regole e alla fine esprimere i 
risultati in \(\R\), 
sembra inutilmente complicato, 
ma in realtà semplifica la soluzione di molti problemi 
(come vedremo più avanti).

Vediamo, con alcuni esempi, come si possono applicare 
le regole presentate in precedenza al calcolo di espressioni 
contenenti numeri Iperreali. 
Di seguito richiamiamo le convenzioni già presentate:
\begin{itemize} [nosep]
 \item con le lettere greche minuscole indichiamo gli infinitesimi non 
nulli;
 \item con \emph{x, y, z} indichiamo un numero iperreale qualsiasi;
 \item con le altre lettere latine minuscole indichiamo i numeri finiti non 
infinitesimi;
 \item con le lettere latine maiuscole indichiamo gli infiniti.;
 \item con \(\pst{x}\) indichiamo la parte standard di \(x\).
\end{itemize}

\begin{esempio}
% \(\pst{\dfrac{7 -3 \epsilon}{9 +2 \delta}}\)\\

\(\pst{\dfrac{7 -2 \epsilon}{9 +3 \delta}} 
~ \stackrel{1}{=} ~
  \dfrac{\pst{7 -2 \epsilon}}{\pst{9 +3 \delta}} 
~ \stackrel{2}{=} ~
  \dfrac{\pst{7 -\alpha}}{\pst{9 +\beta}} 
~ \stackrel{3}{\sim} ~ \pst{\dfrac{7}{9}}
~ \stackrel{4}{=} ~ \dfrac{7}{9}\)\\

Dove le uguaglianze hanno i seguenti motivi:
\begin{enumerate} [nosep]
 \item la parte standard di un quoziente, con il divisore finito non 
infinitesimo, è uguale al quoziente delle parti standard; 
 \item se \(\epsilon \text{ e } \delta\) sono infinitesimi, 
 anche \(3\epsilon\) e \(2 \delta\) sono infinitesimi;
 \item la somma algebrica di un numero non infinitesimo e un numero 
infinitesimo è indistinguibile dal numero non infinitesimo;
 \item la parte standard di un numero reale è quel numero reale.
\end{enumerate}
\end{esempio}

\begin{esempio}
% \(\pst{\dfrac{4 \epsilon^4 -7 \epsilon^3 + \epsilon^2}{5 \epsilon}}\)\\

\(\pst{\dfrac{4 \epsilon^4 -7 \epsilon^3 + \epsilon^2}{5 \epsilon}} 
~ \stackrel{1}{=} ~
  \pst{\dfrac{\tonda{4 \epsilon^3 -7 \epsilon^2 + \epsilon} \epsilon}
                    {5 \epsilon}} 
~ \stackrel{2}{=} ~\)

\(~ \stackrel{2}{=} ~ 
  \pst{\dfrac{4 \epsilon^3 -7 \epsilon^2 + \epsilon}{5}} 
~ \stackrel{3}{=} ~
  \pst{\dfrac{\alpha}{5}}
~ \stackrel{4}{=} ~
  \pst{\beta}
~ \stackrel{5}{=} ~
  0\)\\

Dove le uguaglianze hanno i seguenti motivi:
\begin{enumerate} [nosep]
 \item si può raccogliere \(\epsilon\) al numeratore; 
 \item dato che \(\epsilon\) è diverso da zero, si può semplificare la 
frazione; 
 \item i prodotti tra un finito e un infinitesimo sono infinitesimi e la 
somma di infinitesimi è un infinitesimo;
 \item il quoziente tra un infinitesimo e un non infinitesimo è un 
infinitesimo;
 \item la parte standard di un infinitesimo è zero.
\end{enumerate}
\end{esempio}

\begin{esempio}
Allo stesso risultato si perviene utilizzando la relazione: \emph{essere 
indistinguibili}:

\(\pst{\dfrac{4 \epsilon^4 -7 \epsilon^3 + \epsilon^2}{5 \epsilon}} 
~ \stackrel{1}{\sim} ~
  \pst{\dfrac{\epsilon^2}{5 \epsilon}} 
~ \stackrel{2}{=} ~
  \pst{\dfrac{\epsilon}{5}}
~ \stackrel{3}{=} ~
  0\)\\

Dove le uguaglianze hanno i seguenti motivi:
\begin{enumerate} [nosep]
 \item tengo solo la parte principale del numeratore e del denominatore, 
cioè ignoro gli infinitesimi di ordine superiore; 
 \item riduco la frazione semplificando i fattori uguali; 
 \item il quoziente tra un infinitesimo e un non infinitesimo è un 
infinitesimo e la parte standard di un infinitesimo è zero.
\end{enumerate}
\end{esempio}

\begin{esempio}
\(\pst{\dfrac{5\epsilon -3 \epsilon^2 + 6 \epsilon^3}
             {2\epsilon + 4 \epsilon^2}} 
~ \stackrel{1}{\sim} ~
  \pst{\dfrac{5 \epsilon}
             {2 \epsilon}} 
~ \stackrel{2}{=} ~
  \pst{\dfrac{5}
             {2}}~
~ \stackrel{3}{=} ~
  \dfrac{5}{2}\)\\

\begin{enumerate} [nosep]
 \item Tengo solo la parte principale dei polinomi ottenendo un'espressione 
indistinguibile;
 \item riduco la frazione semplificando i fattori uguali;
 \item la parte standard di un numero reale è il numero stesso.
\end{enumerate}
\end{esempio}

\begin{esempio}
\(\pst{\dfrac{-6\epsilon^2 +4 \epsilon^3 -8 \epsilon^5}
             {7\epsilon^3 + 2 \epsilon^4}} 
~ \stackrel{1}{=} ~
  \pst{\dfrac{-6\epsilon^2}
             {7\epsilon^3}} 
~ \stackrel{2}{=} ~
  \pst{-\dfrac{6}
             {7\epsilon}} 
~ \stackrel{3}{\longrightarrow} ~
  \infty\)\\

Dove le uguaglianze hanno i seguenti motivi:
\begin{enumerate} [nosep]
 \item tengo solo la parte principale dei polinomi ottenendo un'espressione 
indistinguibile;
 \item semplifico i fattori uguali;
 \item il quoziente tra un finito e un infinitesimo non nullo è un 
infinito. 
L'infinito non ha parte standard ma viene indicato in analisi con 
il simbolo: \(\infty\) (che, comunque, non è un numero reale).
\end{enumerate}
\begin{osservazione}
 In questo caso, se \(\epsilon\) è positivo, l'infinito sarà negativo, 
 se \(\epsilon\) è negativo, l'infinito sarà positivo.
\end{osservazione}
\end{esempio}

\begin{esempio}
\(\pst{\dfrac{-3H^2 -4H}
             {2H^2 +4H -3}} 
~ \stackrel{1}{\sim} ~
  \pst{\dfrac{-3H^2}
             {2H^2}} 
~ \stackrel{2}{=} ~
  \pst{\dfrac{-3}
             {2}} 
~ \stackrel{3}{=} ~
  -\dfrac{3}{2}\)\\

Dove le uguaglianze hanno i seguenti motivi:
\begin{enumerate} [nosep]
 \item teniamo la parte principale delle espressioni tenendo solo gli 
infiniti ordine maggiore; 
 \item semplifichiamo i fattori uguali al numeratore e al denominatore; 
 \item teniamo la parte standard.
\end{enumerate}
\end{esempio}

\begin{esempio}
\(\pst{\tonda{7 -3\epsilon} - \tonda{7 +8\epsilon}}\)

\begin{osservazione}
 Si potrebbe pensare che essendo~\(7 -3\epsilon\) indistinguibile da~\(7\) 
e~\(7 +8\epsilon\) indistinguibile da~\(7\) la precedente espressione 
sia indistinguibile da \(7 - 7 = 0\).
Ma il concetto di indistinguibile non si può mai applicare tra un numero e 
lo zero quindi non possiamo dire che 
\(\tonda{7 -3\epsilon} - \tonda{7 +8\epsilon} \sim 0\) e 
tanto meno: \(\tonda{7 -3\epsilon} - \tonda{7 +8\epsilon} = 0\).
\end{osservazione}\\

In questo caso la soluzione è semplice...

\(\pst{\tonda{7 -3\epsilon} - \tonda{7 +8\epsilon}}
~ \stackrel{1}{=} ~
   \pst{7 -3\epsilon -7 -8\epsilon} 
~ \stackrel{2}{=} ~
   \pst{-11 \epsilon}
~ \stackrel{3}{=} ~ 0\)\\

Dove le uguaglianze hanno i seguenti motivi:
\begin{enumerate} [nosep]
 \item semplifichiamo l'espressione eliminando le parentesi; 
 \item \(7\) e \(-7\) si annullano; 
 \item la parte standard di un infinitesimo è zero.
\end{enumerate}
\end{esempio}

\begin{esempio}
\(\sqrt{4H^2 -3H} - \sqrt{4H^2 +1}\)

\begin{osservazione}
 Anche qui si potrebbe pensare che essendo~\(4H^2 -3H\) indistinguibile 
da~\(4H^2\) e~\(4H^2 +1\) indistinguibile da~\(4H^2\) la precedente 
espressione sia indistinguibile da \\
\(\sqrt{4H^2} - \sqrt{4H^2} = 0\)\\
Ma il concetto di indistinguibile, per come è definito, non si può mai 
applicare tra un numero e lo zero quindi non possiamo dire che 
\(\sqrt{4H^2 -3H} - \sqrt{4H^2 +1} \approx 0\).
\end{osservazione}\\

In questo caso usiamo un trucco una specie di inverso della 
razionalizzazione:

\(\sqrt{4H^2 -3H} - \sqrt{4H^2 +1}
~ \stackrel{1}{=} ~
   \tonda{\sqrt{4H^2 -3H} - \sqrt{4H^2 +1}} \cdot 1 
~ \stackrel{2}{=} ~\)

\(~ \stackrel{2}{=} ~\tonda{\sqrt{4H^2 -3H}-\sqrt{4H^2 +1}} \cdot 
   \dfrac{\sqrt{4H^2 -3H}+\sqrt{4H^2 +1}}{\sqrt{4H^2 -3H}+\sqrt{4H^2+1}}
~ \stackrel{3}{=} ~\)

\(~ \stackrel{3}{=} ~
   \dfrac{\cancel{4H^2} -3H - \cancel{4H^2} -1}
         {\sqrt{4H^2 -3H}+\sqrt{4H^2+1}}
~ \stackrel{4}{\sim} ~
   \dfrac{-3H}{\sqrt{4H^2} + \sqrt{4H^2}}
~ \stackrel{5}{=} ~
   \dfrac{-3\cancel{H}}{4\cancel{H}}
~ \stackrel{}{=} ~
   -\dfrac{3}{4}\)\\

Dove le uguaglianze hanno i seguenti motivi:
\begin{enumerate} [nosep]
 \item la prima uguaglianza è banale essendo~1 l'elemento 
neutro della moltiplicazione; 
 \item al posto del numero~1 sostituiamo una frazione con 
il numeratore e il denominatore uguali; 
 \item eseguendo il prodotto magari tenendo conto di uno 
dei prodotti notevoli imparati in prima otteniamo questa 
frazione;
 \item tenendo conto che \(+H\) e \(-H\) si annullano 
otteniamo una nuova frazione che, a prima vista non sembra 
aver semplificato il problema iniziale, 
ma a denominatore non ho una differenza tra due radici 
ma una somma e quindi posso ottenere un'espressione più semplice 
indistinguibile da quella originale;
 \item estraiamo le radici, sommiamo, semplifichiamo.
\end{enumerate}
\end{esempio}



\begin{comment}
TODO
Iniziamo studiando più in particolare come eseguire i calcoli in 
\(\IR\). 
Riprendiamo il problema della velocità istantanea visto nel paragrafo 
\ref{subsec:insnum_velocita}.


\subsection{Vertice della parabola}
\label{subsec:insnum_vertice_parabola}


\subsection{Tangenti ad una parabola}
\label{subsec:insnum_tangenti_parabola}


\subsection{Cerchio osculatore}
\label{subsec:insnum_cerchio_osculatore}


\subsection{Area di un segmento parabolico}
\label{subsec:insnum_segmento_parabolico}


\subsection{Problemi di massimo o minimo}
\label{subsec:insnum_massimo_minimo}

\end{comment}






































