% (c) 2015 Daniele Zambelli daniele.zambelli@gmail.com
% (c) 2017 Bruno Stecca

% % \vspace{-2ex}\input{\folder lbr/tab002}\vspace{-2ex}
% \begin{inaccessibleblock}
% [Immagine di una porzione dell'insieme di Mandelbrot.]
% \vspace{-2ex}
% \begin{center} \includegraphics[scale=0.25]{img/hiero3673.png} \end{center}
% \vspace{-2ex}
% \end{inaccessibleblock}

\input{\folder iperreali_gra.tex}

\chapter{Iperreali}
\label{sec:01_introduzione}
Lo scopo di questo capitolo è riprendere familiarità con l'uso 
dei numeri iperreali, già descritti verso la fine del terzo anno. 
Le parti principali utili ai fini della nostra trattazione sono 
direttamente riportate dal terzo volume.\\
Il motivo di questi richiami è che in analisi matematica è normale 
avere a che fare con quantità infinitesime e quantità infinite e 
valutare il comportamento delle funzioni applicate a tali quantità. 
Quindi l'uso dei numeri iperreali diviene vantaggioso.\\
La conoscenza degli iperreali non è molto diffusa neanche fra i matematici, 
abituati da un secolo e mezzo a procedimenti più impegnativi e sofisticati.
La ragione per la quale noi invece ne facciamo uso è che ci rendono il 
calcolo 
più semplice e immediato, senza per questo nuocere al rigore e alla 
precisione dei ragionamenti.

\section{Alcune questioni sui numeri reali}
\label{sec:insnum_reali}

I numeri reali formano un insieme \emph{ordinato}, \emph{denso} e 
\emph{completo}: \(\R\). È un insieme ordinato perché fra due numeri reali
diversi sappiamo sempre indicare il maggiore e il minore. È denso
perché fra due numeri reali diversi, per quanto vicini, se ne può 
sempre trovare almeno un altro. E infine \(\R\) è un insieme completo 
perché il numero che troveremo fra i due vicini è ancora un numero reale.\\
Possiamo quindi far corrispondere ad ogni punto della \emph{retta reale} un 
numero \emph{reale} e, viceversa, ad ogni numero \emph{reale} un punto 
della \emph{retta reale}. In poche parole, siamo
autorizzati a pensare la retta reale come una retta ``priva di buchi'':
c'è almeno un punto in ogni posizione, anche osservando la retta al 
microscopio, con qualsiasi ingrandimento (ingrandimento reale, come 
vedremo).

Se usiamo la retta reale come immagine dell'insieme \(\R\) è perché
si tratta di una rappresentazione efficace. Ma ricordiamoci sempre che un 
insieme in matematica è un oggetto astratto, quindi la retta reale è solo
un modello che ci aiuta a capire le proprietà dell'insieme \(\R\).\\
Per esempio, a proposito dell'ordinamento in \(\R\), ci riesce facile
posizionare i numeri sulla retta, in corrispondenza di punti più vicini o 
più lontani dall'origine. 
Così possiamo verificare anche un'ulteriore proprietà, 
la \emph{proprietà archimedea}: per quanto piccolo sia un numero, si 
può sempre moltiplicare per un numero naturale abbastanza grande in modo 
che il prodotto superi un altro numero prefissato.

Sulla retta: dati due segmenti con un estremo nell'origine, potrai sempre 
trovare un  multiplo del più breve che superi il più lungo.

A causa della completezza di \(\R\), non è possibile inserire
nella retta reale dei punti che non corrispondano a numeri reali. Se 
inseriamo numeri di nuovo tipo, il modello cambia. Il nuovo insieme e la 
nuova retta sono diversi dall'insieme dei reali e dalla retta reale: si 
perde qualche proprietà e se ne acquisiscono di nuove.
Infatti...

% \vspace{24pt}

\newpage %--------------------------------------------------

\input{\folder iperreali.tex}

\begin{esempio}
{~}

\begin{minipage}{.44\textwidth}
Calcola la tangente all'ellisse di equazione:
\(4x^2+3y^2=48\)
nel punto di coordinate \(T\punto{3}{2}\).

La funzione che descrive la parte di ellisse 
contenente \(T\) è:
\(y=+\sqrt{-\dfrac{4}{3}x^2+16}\)\\
L'equazione del fascio di rette per \(T\) è:\\
\(y=m\tonda{x-3}+2\)
\begin{align*}
m&=\pst{\dfrac{d y}{d x}}=
   \pst{\dfrac{f(3+\epsilon)-f(3)}{\epsilon}}=\\
 &=\pst{\dfrac{\sqrt{-\dfrac{4}{3}(9+\epsilon)^2+16}-2}{\epsilon}}=
\end{align*}
\end{minipage}
\hfill
\begin{minipage}{.54\textwidth}
\begin{center}\iperellisse\end{center}
\end{minipage}
\begin{align*}
m&=\pst{\dfrac{\sqrt{-\dfrac{4}{3}(3+\epsilon)^2+16}-2}{\epsilon} \cdot
        \dfrac{\sqrt{-\dfrac{4}{3}(3+\epsilon)^2+16}+2}
              {\sqrt{-\dfrac{4}{3}(3+\epsilon)^2+16}+2}}=\\
 &=\pst{\dfrac{-\dfrac{4}{3}(3+\epsilon)^2+16-4}
              { \epsilon\tonda{\sqrt{-\dfrac{4}{3}(3+\epsilon)^2+16}+2}}}=
   \pst{\dfrac{-8\epsilon-\frac{4}{3}\epsilon^2}{4 \epsilon}}=
   \pst{\dfrac{-8 \cancel{\epsilon}}{4 \cancel{\epsilon}}}=-2
\end{align*}
E la tangente è quindi:
\[y=m \tonda{x-x_0}+y_0 \sRarrow y=-2 \tonda{x-3}+2 \sRarrow y=-2x+8\]

\end{esempio}

\begin{comment}

\begin{esempio}
 % limite notevole espon.
\(\pst{\tonda{1+\dfrac{k}{N}}^N}
~ \stackrel{1}{=} ~  
\pst{\tonda{1+\dfrac{1}{M}}^{kM}}
~ \stackrel{2}{=} ~
\pst{\quadra{\tonda{1+\dfrac{1}{M}}^M}}^k
~ \stackrel{3}{=} ~ e^k\).\\

Dove le uguaglianze hanno i seguenti motivi:
\begin{enumerate} [nosep]
 \item un altro sporco trucco: la sostituzione. Supponiamo
\(\frac{k}{N}=\dfrac{1}{M}\). Allora \(N=kM\);
 \item una potenza di potenza è una potenza che ha...
 \item per la definizione del numero \(e\) e per le proprietà della 
funzione 
\(\st()\).
\end{enumerate}
\end{esempio}

\begin{esempio}
 % limite notevole log.
\(\pst{\dfrac{a^\epsilon-1}{\epsilon}}
~ \stackrel{1}{=} ~  
\pst{\dfrac{\delta}{\log_a{(\delta+1)}}}
~ \stackrel{2}{=} ~
\pst{\frac{1}{\dfrac{\log_a{(\delta+1)}}{\delta}}}
~ \stackrel{3}{=} ~ 
\pst{\frac{1}{\dfrac{1}{\ln{a}}}}=\ln{a}\).\\

Dove le uguaglianze hanno i seguenti motivi:
\begin{enumerate} [nosep]
 \item ancora una sostituzione: poniamo
\(a^\epsilon-1=\delta\). Allora \(\epsilon=\log_a(\delta+1)\);
 \item una capriola algebrica: oplà! 
 \item per le forme di indecisione discusse a proposito del numero di Nepero
e per il cambiamento di base;
\end{enumerate}
\end{esempio}

\begin{esempio}
 % limite notevole seno e coseno
 \(\pst{\dfrac{1-\cos \delta}{\sin \delta}}\)=0.\\
~
Dove l'uguaglianza si giustifica per quanto detto a proposito dell'ordine 
degli infinitesimi, ma gli appassionati del calcolo possono provare a 
moltiplicare il numeratore e il denominatore per ...
\end{esempio}

\end{comment}
