% (c) 2015 Daniele Zambelli daniele.zambelli@gmail.com
% (c) 2017 Bruno Stecca

% % \vspace{-2ex}\input{\folder lbr/tab002}\vspace{-2ex}
% \begin{inaccessibleblock}
% [Immagine di una porzione dell'insieme di Mandelbrot.]
% \vspace{-2ex}
% \begin{center} \includegraphics[scale=0.25]{img/hiero3673.png} \end{center}
% \vspace{-2ex}
% \end{inaccessibleblock}

% \input{\folder iperreali_grafici.tex}

\chapter{Iperreali}
\label{sec:01_introduzione}
Lo scopo di questo capitolo è riprendere familiarità con l'uso 
dei numeri iperreali, già descritti verso la fine del terzo anno. 
Le parti principali utili ai fini della nostra trattazione sono 
direttamente riportate dal terzo volume.\\
Il motivo di questi richiami è che in analisi matematica è normale 
avere a che fare con quantità infinitesime e quantità infinite e 
valutare il comportamento delle funzioni applicate a tali quantità. 
Quindi l'uso dei numeri iperreali diviene vantaggioso.\\
La conoscenza degli iperreali non è molto diffusa neanche fra i matematici, 
abituati da un secolo e mezzo a procedimenti più impegnativi e sofisticati.
La ragione per la quale noi invece ne facciamo uso è che ci rendono il 
calcolo 
più semplice e immediato, senza per questo nuocere al rigore e alla 
precisione dei ragionamenti.

\section{Alcune questioni sui numeri reali}
\label{sec:insnum_reali}

I numeri reali formano un insieme \emph{ordinato}, \emph{denso} e 
\emph{completo}: \(\R\). 

È un insieme \emph{ordinato} perché dati due numeri reali diversi sappiamo 
sempre indicare il maggiore e il minore. 
È \emph{denso} perché fra due numeri reali diversi, per quanto vicini, se ne 
può sempre trovare almeno un altro. 

Per quanto riguarda la \emph{completezza} di \(\R\) la questione è piuttosto 
complicata. 
Come abbiamo visto %\Vedi \ref{})
I numeri razionali non sono sufficienti per indicare le lunghezze di tutti i 
segmenti (ad es. se la misura del lato di un quadrato è un numero razionale, la 
misura della sua diagonale non potrà essere razionale).
Sono stati inventati così i numeri reali che possono rappresentare la lunghezza 
di un qualunque segmento.
L'esempio classico riguarda la lunghezza della diagonale di un quadrato: 
\(l=1 \sRarrow d=\sqrt{2}\)
Possiamo dividere l'insieme dei razionali in due sottoinsiemi \(A\) che 
contiene tutti i numeri minori o uguali a \(\sqrt{2}\) e \(B\) che contiene 
tutti i numeri maggiori o uguali a \(\sqrt{2}\): per quanto cerchi di essere 
sempre più preciso, tra questi due insiemi resterà sempre uno spazio che 
contiene \emph{infiniti} altri numeri razionali. 
Ma se potessimo continuare a restringere questo spazio all'infinito otterremmo 
un \emph{intervallo infinitamente piccolo} che non può però contenere un numero 
razionale. I matematici hanno così deciso che in quello spazio ci sta \emph{uno e 
un solo numero}, di tipo nuovo: \emph{un numero reale}.


\begin{comment} Pezzo di Bruno
La \emph{completezza} di \(\R\) è una proprietà che i matematici richiedono per
usare la retta dei numeri come modello, cioè perché le proprietà dell'insieme
\(\R\) rispecchino le caratteristiche geometriche della retta, che per questo è 
detta retta reale. La retta reale non ha strappi o buchi, è continua: è 
immaginata come un insieme di punti uniti uno all'altro. La completezza di 
\(\R\) rappresenta tale continuità. Vediamo più in dettaglio.

Se gli unici numeri conosciuti fossero i naturali, gli interi e i razionali, 
non potremmo rappresentare determinate lunghezze sulla retta dei numeri.
% Per quanto riguarda la \emph{completezza} di \(\R\) la questione è piuttosto 
% complicata. 
% Come abbiamo visto %\Vedi \ref{})
Come i numeri interi non bastano a rappresentare i risultati delle divisioni, 
così i numeri razionali non sono sufficienti per tutte le necessità di misurare 
conosciute dai matematici: i numeri razionali non formano un insieme completo.
È stato così ideato l'insieme dei numeri reali, con i quali si rappresenta 
qualsiasi lunghezza. 

Prendiamo ad esempio una circonferenza  di diametro \(1\), 
tagliamola e stendiamola sulla retta dei numeri, a partire da \(0\). Il secondo
estremo cadrà allora in una posizione vicina a \(3,14\), appena oltre, ma non 
abbastanza da arrivare a \(3,15\). Se siamo in grado di rappresentare i millesimi,
vedremo che la misura supera \(3,141\), ma non arriva a \(3,142\). E con 
successive osservazioni sulla retta, sempre più raffinate, arrivando ai 
milionesimi, ai miliardesimi, ecc. vediamo che la misura esatta sta sempre fra due
numeri razionali, ma su nessuno esattamente.\\
Manca il nome del punto su cui cade quell'estremo, cioè manca il numero che esprime la misura.

Un altro esempio classico riguarda la lunghezza della diagonale di un quadrato:\\ 
\(l=1 \sRarrow d=\sqrt{2}\).
Possiamo dividere l'insieme dei razionali in due sottoinsiemi: \(A\) che 
contiene tutti i numeri minori o uguali a \(\sqrt{2}\) e \(B\) che contiene 
tutti i numeri maggiori o uguali a \(\sqrt{2}\). 

Facciamo una prova con uno dei numeri 
nell'insieme \(A\), per esempio quello che esprime il risultato della radice 
con la precisione del milionesimo: \(1,414213\). \\
Verifichiamo: \(1,414213^2=1,999998409\), cioè non si arriva 
esattamente a 2. Infatti, per quanto si cerchi di essere 
sempre più precisi proseguendo nel calcolo dei decimali e inserendo in \(A\) i risultati
per difetto e in \(B\) quelli per eccesso, tra \(A\) e \(B\) resterà sempre uno spazio
che contenere \emph{infiniti} altri numeri razionali. 
Ma se potessimo continuare a restringere questo spazio all'infinito otterremmo 
un \emph{intervallo infinitamente piccolo} che non può però contenere un numero 
razionale. I matematici hanno così deciso che in quello spazio ci sta \emph{uno e 
un solo numero}, di tipo nuovo: \emph{un numero reale}.
\end{comment}

In questo modo i numeri reali permettono di esprimere la misura di un qualunque 
segmento e di realizzare così una corrispondenza biunivoca tra i punti di una 
retta e i numeri reali. 
Possiamo cioè far corrispondere ad ogni punto della \emph{retta reale} un 
numero \emph{reale} e, viceversa, ad ogni numero \emph{reale} un punto 
della \emph{retta reale}. In poche parole, siamo
autorizzati a pensare la retta reale come una retta ``priva di buchi'':
c'è almeno un punto in ogni posizione, anche osservando la retta al 
microscopio, con qualsiasi ingrandimento (ingrandimento reale, come 
vedremo).

L'insieme dei numeri reali \(\R\) si dice quindi \emph{completo}. La sua 
completezza è data per assioma, è una caratteristica di questo insieme
per come è stato pensato dai matematici e non può essere dimostrata ricorrendo
alle altre proprietà di questo insieme.

Se usiamo la retta reale come immagine dell'insieme \(\R\) è perché
si tratta di una rappresentazione efficace. Ma ricordiamoci sempre che un 
insieme in matematica è un oggetto astratto, quindi la retta reale è solo
un modello che ci aiuta a capire le proprietà dell'insieme \(\R\).

La retta permette di visualizzare facilmente la densità di \(\R\): se due 
numeri diversi sono troppo vicini per poterli separare, basta prendere un 
microscopio adeguato e si potrà vedere che tra questi due punti ce ne sono 
infiniti altri.

Per quanto riguarda l'ordinamento in \(\R\), bisogna aggiungere un verso alla 
retta (ad esempio verso destra) in questo modo 
\(b > a\) equivale a \(b \quad \text{si trova più a destra di}\quad a\). 

La retta ci aiuta ad intuire anche un'ulteriore proprietà, 
la \emph{proprietà archimedea}: dati due numeri qualunque, si può sempre 
trovare un opportuno multiplo del più piccolo che sia maggiore del più grande.
Sulla retta: dati due segmenti con un estremo nell'origine, potrai sempre 
trovare un  multiplo del più breve che superi il più lungo.

Avendo deciso (postulato) che \(\R\) è completo, non è possibile inserire
nella retta reale dei punti che non corrispondano a numeri reali. 
Se vogliamo aggiungere dei nuovi numeri che non siano reali dovremo metterli 
fuori dalla retta o usare un nuovo tipo di retta, un nuovo modello.

La scelta di completare gli \emph{intervalli infinitamente piccoli} 
riempiendoli con \emph{un solo numero} è molto comoda, ma non è l'unica 
possibile. 
Nel prossimo paragrafo vedremo che in quello spazio infinitamente piccolo si 
possono far stare infiniti numeri conservando quasi tutte le proprietà dei 
numeri reali. 
Inventeremo così un nuovo insieme numerico e un nuova retta: 
perderemo qualche proprietà, ma ne acquisiremo di nuove.
Molto interessanti.

Infatti...

% \vspace{24pt}

\newpage %--------------------------------------------------

\input{\folder iperreali.tex}

\begin{esempio}
{~}

\begin{minipage}{.44\textwidth}
Calcola la tangente all'ellisse di equazione:
\(4x^2+3y^2=48\)
nel punto di coordinate \(T\punto{3}{2}\).

La funzione che descrive la parte di ellisse 
contenente \(T\) è:
\(y=+\sqrt{-\dfrac{4}{3}x^2+16}\)\\
L'equazione del fascio di rette per \(T\) è:\\
\(y=m\tonda{x-3}+2\)
\begin{align*}
m&=\pst{\dfrac{d y}{d x}}=
   \pst{\dfrac{f(3+\epsilon)-f(3)}{\epsilon}}=\\
 &=\pst{\dfrac{\sqrt{-\dfrac{4}{3}(9+\epsilon)^2+16}-2}{\epsilon}}=
\end{align*}
\end{minipage}
\hfill
\begin{minipage}{.54\textwidth}
\begin{center}\iperellisse\end{center}
\end{minipage}
\begin{align*}
m&=\pst{\dfrac{\sqrt{-\dfrac{4}{3}(3+\epsilon)^2+16}-2}{\epsilon} \cdot
        \dfrac{\sqrt{-\dfrac{4}{3}(3+\epsilon)^2+16}+2}
              {\sqrt{-\dfrac{4}{3}(3+\epsilon)^2+16}+2}}=\\
 &=\pst{\dfrac{-\dfrac{4}{3}(3+\epsilon)^2+16-4}
              { \epsilon\tonda{\sqrt{-\dfrac{4}{3}(3+\epsilon)^2+16}+2}}}=
   \pst{\dfrac{-8\epsilon-\frac{4}{3}\epsilon^2}{4 \epsilon}}=
   \pst{\dfrac{-8 \cancel{\epsilon}}{4 \cancel{\epsilon}}}=-2
\end{align*}
E la tangente è quindi:
\[y=m \tonda{x-x_0}+y_0 \sRarrow y=-2 \tonda{x-3}+2 \sRarrow y=-2x+8\]

\end{esempio}

\begin{comment}

\begin{esempio}
 % limite notevole espon.
\(\pst{\tonda{1+\dfrac{k}{N}}^N}
~ \stackrel{1}{=} ~  
\pst{\tonda{1+\dfrac{1}{M}}^{kM}}
~ \stackrel{2}{=} ~
\pst{\quadra{\tonda{1+\dfrac{1}{M}}^M}}^k
~ \stackrel{3}{=} ~ e^k\).\\

Dove le uguaglianze hanno i seguenti motivi:
\begin{enumerate} [nosep]
 \item un altro sporco trucco: la sostituzione. Supponiamo
\(\frac{k}{N}=\dfrac{1}{M}\). Allora \(N=kM\);
 \item una potenza di potenza è una potenza che ha...
 \item per la definizione del numero \(e\) e per le proprietà della 
funzione 
\(\st()\).
\end{enumerate}
\end{esempio}

\begin{esempio}
 % limite notevole log.
\(\pst{\dfrac{a^\epsilon-1}{\epsilon}}
~ \stackrel{1}{=} ~  
\pst{\dfrac{\delta}{\log_a{(\delta+1)}}}
~ \stackrel{2}{=} ~
\pst{\frac{1}{\dfrac{\log_a{(\delta+1)}}{\delta}}}
~ \stackrel{3}{=} ~ 
\pst{\frac{1}{\dfrac{1}{\ln{a}}}}=\ln{a}\).\\

Dove le uguaglianze hanno i seguenti motivi:
\begin{enumerate} [nosep]
 \item ancora una sostituzione: poniamo
\(a^\epsilon-1=\delta\). Allora \(\epsilon=\log_a(\delta+1)\);
 \item una capriola algebrica: oplà! 
 \item per le forme di indecisione discusse a proposito del numero di Nepero
e per il cambiamento di base;
\end{enumerate}
\end{esempio}

\begin{esempio}
 % limite notevole seno e coseno
 \(\pst{\dfrac{1-\cos \delta}{\sin \delta}}\)=0.\\
~
Dove l'uguaglianza si giustifica per quanto detto a proposito dell'ordine 
degli infinitesimi, ma gli appassionati del calcolo possono provare a 
moltiplicare il numeratore e il denominatore per ...
\end{esempio}

\end{comment}
