% (c) 2015 Daniele Zambelli daniele.zambelli@gmail.com
% (c) 2017 Bruno Stecca

% % \vspace{-2ex}\input{\folder lbr/tab002}\vspace{-2ex}
% \begin{inaccessibleblock}
% [Immagine di una porzione dell'insieme di Mandelbrot.]
% \vspace{-2ex}
% \begin{center} \includegraphics[scale=0.25]{img/hiero3673.png} 
% \end{center}
% \vspace{-2ex}
% \end{inaccessibleblock}

\input{\folder naturali_iperreali_grafici.tex}

\chapter{Dai Naturali agli Iperreali}

\section{Dai numeri naturali ai numeri complessi}
\label{sec:01_introduzione}

Riprendiamo i diversi insiemi numerici che abbiamo imparato a conoscerete
mettendo in evidenza il loro ruolo come modelli per risolvere alcune classi 
di problemi e le loro caratteristiche.

\subsection{I numeri naturali \(\N\)}
\label{subsec:insnum_naturali}

I primi numeri che abbiamo incontrato sono i numeri \emph{naturali}. 
Sono quelli che permettono di contare oggetti. 
Se sul banco ho un quaderno, una penna e un libro posso dire che ci sono~3 
oggetti. 
Si può capire come il numero Zero abbia avuto difficoltà a farsi accettare 
come numero: serve per contare un gruppo di oggetti dove non c'è niente da 
contare. 
Ma ora abbiamo capito che è molto comodo considerare lo zero come un numero.
L'insieme dei numeri \emph{naturali} viene indicato con~\(\N\).

Nei numeri naturali sono definite l'addizione, la moltiplicazione che sono 
sempre possibili. In queste due \emph{strutture} \(\tonda{\N, +}\) e 
\(\tonda{\N, \times}\) valgono le proprietà: associativa, commutativa e 
l'esistenza dell'elemento neutro.

Nei numeri naturali è definita anche la \emph{potenza} ma questa operazione 
non può essere eseguita quando sia la base sia l'esponente sono uguali a zero.

Oltre a queste, sono definite anche le loro operazioni inverse: la 
sottrazione, la divisione e la radice, ma queste non si possono eseguire
con ogni coppia di numeri.

D'altra parte se su un tavolo ho~5 oggetti posso toglierne~3 e ne restano~2:
\[5-3=2\]

Ma se sul tavolo ho~3 oggetti non ha senso cercare di toglierne~5!

\subsection{I numeri interi \(\Z\)}
\label{subsec:insnum_interi}

I numeri possono però essere utilizzati anche come modelli di altre 
situazioni. 
Supponiamo di avere la sequenza di oggetti e di voler riferirmi ad ognuno 
con un numero che equivale al suo indirizzo o indice. In certi casi potrei 
cercare il primo elemento della sequenza e chiamarlo zero, quello che viene 
dopo lo chiamo uno e così via. Ma se ci trovassimo a lavorare principalmente 
con gli elementi compresi tra il 273 elemento e il 310 elemento, questo 
modo di fare sarebbe piuttosto scomodo. 
Molto più semplice è mettersi d'accordo di chiamare zero il 273 elemento e 
partire da lì a contarli. In questo caso, i numeri che dovremo usare saranno 
quelli compresi tra~0 e~37. 
Ci sono inoltre delle situazioni in cui è difficile, o impossibile, 
determinare un \emph{primo} elemento della sequenza e anche in questo caso 
ci si può mettere d'accordo di assegnare ad un preciso elemento della 
sequenza il valore zero.

È chiaro che lo \emph{zero} non sarà il \emph{primo} elemento della 
sequenza, ma un valore all'interno della sequenza. 
È possibile muoversi sia sopra lo zero, sia sotto lo zero.
Per non inventare dei nomi completamente nuovi per questi nuovi 
numeri, sono stati aggiunti semplicemente due segni:~``\(+\)'' per i numeri 
dopo lo zero e~``\(-\)'' per i numeri prima dello zero. 
Questi nuovi numeri, ottenuti aggiungendo il segno ai numeri naturali, sono 
chiamati numeri \emph{interi} 
e l'insieme di questi numeri viene indicato con~\(\Z\).
L'insieme dei numeri interi non ha un elemento minimo.

Per le esigenze pratiche, i numeri interi non sono strettamente necessari, è 
sempre possibile dare un numero naturale e indicare se intendiamo ``prima o 
dopo'', ``sopra o sotto''. Lo facciamo spesso, ad esempio: 300 anni prima 
dell'era volgare, o: 1200m sotto il livello del mare, oppure: 4°C sotto zero. 
Ma usare questi nuovi numeri ci semplifica la vita in molte situazioni, 
soprattutto quando dobbiamo operare con calcoli.

Nei numeri interi, l'addizione può essere vista come muoversi nel verso 
di crescita dei numeri e la sottrazione come muoversi nel verso della 
decrescita dei numeri. Dato che lo zero è un elemento convenzionale non c'è 
nessun problema a togliere~5 da~3 semplicemente si arriverà nella 
posizione~2 prima dello zero detta~\(-2\).

In questo insieme di numeri è sempre definita anche la sottrazione, anzi la 
sottrazione diventa semplicemente un caso particolare di addizione.

I numeri interi permettono di risolvere sempre equazioni del tipo:
\[x+a=0\]

I numeri interi formano un'estensione dei naturali;
mantengono quasi tutte le proprietà dei naturali 
(non la proprietà di avere un minimo) ma in questo insieme, la sottrazione è 
sempre definita.
Il sottoinsieme di \(\Z\) formato dallo zero e da tutti i numeri positivi
si comporta esattamente come l'insieme dei numeri Naturali. Diremo che 
questo sottoinsieme è isomorfo all'insieme~\(\N\) 
e questo ci permette di usare indifferentemente \(+7\) o \(7\) 
senza dover precisare che \(+7\) è un elemento di \(\Z\) 
mentre \(7\) è un elemento di \(\N\).

\(\Z,~+\) verifica le proprietà associativa, l'esistenza dell'elemento 
neutro e l'esistenza degli elementi inversi diremo quindi che gli interi 
con l'operazione di addizione formano un \emph{gruppo} e valendo anche 
la proprietà commutativa il gruppo si dice commutativo o abeliano.

\(\Z,~\times\) verifica la proprietà associativa, diremo quindi che gli 
interi con l'operazione di moltiplicazione formano un \emph{semigruppo}.

Inoltre, dato che vale la proprietà distributiva, diremo che 
\(\Z,~+,~\times\) è un \emph{anello}. Dato che in \(\Z\) esiste l'elemento 
neutro anche rispetto alla moltiplicazione, diremo che l'anello è dotato di 
\emph{unità}.

Anche questi numeri però non riescono a realizzare un modello in certe 
situazioni che invece, nella pratica, si possono risolvere facilmente con un 
po' di creatività. Ad esempio come possiamo dividere~3 uova, in parti 
uguali, tra~4 persone?

\subsection{I numeri razionali \(\Q\)}
\label{subsec:insnum_razionali}

Con le tre uova faccio una frittata che divido facilmente in~4 parti 
uguali. 
Possiamo costruire dei numeri che permettano di calcolare il 
quoziente esatto di due numeri interi anche quando la divisione tra i due 
dà un resto diverso da zero. 
Questi nuovi numeri sono chiamati numeri \emph{razionali} 
e l'insieme di questi numeri viene indicato con~\(\Q\).

Mentre nei naturali e negli interi ad ogni numero corrisponde un 
\emph{nome} ben preciso, nei razionali lo stesso numero può essere indicato 
con molti nomi diversi. 
Ad esempio il numero che si ottiene dividendo~1 in due parti 
uguali può essere indicato in uno di questi modi:
\[\frac{1}{2}=\frac{3}{6}=\dots=\frac{45}{90}=\frac{132}{264}=\dots=0,5\]

Ogni numero razionale può essere rappresentato con un numero con la virgola 
limitato o periodico, o con una qualunque delle infinite frazioni equivalenti.

Con i numeri razionali si può sempre calcolare il risultato della divisione 
tra due numeri (naturali, interi o razionali) tranne il caso particolare in 
cui il divisore sia uguale a zero. 
In questo caso la divisione non può essere eseguita.

I numeri razionali ci permettono di risolvere tutti i problemi pratici che 
possiamo incontrare, dal calcolare il resto della spesa al far atterrare una 
sonda su Marte. 
Anzi i nostri computer, che sono in grado di fare cose meravigliose, usano un 
sottoinsieme estremamente limitato di numeri razionali.

Anche i numeri razionali non sono strettamente necessari per descrivere e 
modellizzare la realtà, basta usare in ogni occasione una appropriata unità 
di misura: invece di \(1,273\,km\) possiamo scrivere: \(1273\,m\); 
invece che \(\frac{5}{6}\) di torta possiamo dividere la torta in \(6\) fette
e considerarne \(5\).

L'uso dei numeri razionali comunque ci semplifica notevolmente la vita.
Nei numeri razionali è (quasi) sempre definita la divisione, anzi la 
divisione viene trasformata nella moltiplicazione tra il dividendo e il 
reciproco del divisore.

I numeri razionali permettono di risolvere sempre equazioni del tipo:
\[ax+b=0 \quad \text{ con } \quad a \neq 0.\]

I razionali formano un'estesione dei numeri interi, mantengono quasi tutte 
le proprietà degli interi, ma in questi numeri la divisione per un numero 
diverso da zero, è sempre definita.

Anche tra i razionali si può trovare un sottoinsieme isomorfo all'insieme 
degli interi, cioè che si comporta come l'insieme degli interi: è il 
sottoinsieme dei numeri che, scritti sotto forma di frazioni hanno come 
numeratore un multiplo del denominatore o che, ridotte ai minimi termini, 
hanno per denominatore uno. 
Questo fatto ci permette di poter scrivere:~\(-\dfrac{7}{1} = -7\) 
senza dover precisare che il primo numero appartiene a \(\Q\) e il secondo 
a \(\Z\).

I razionali hanno una caratteristica particolare che non avevano né i 
naturali né gli interi: formano un insieme \emph{denso} cioè tra due numeri 
razionali, per quanto vicini, se ne può trovare sempre almeno un altro.

Dato che in \(\Q,~+,~\times\) anche la seconda operazione ha l'inverso di 
ogni numero diverso da zero, i razionali, con le operazioni addizione e 
moltiplicazione e queste operazioni sono commutative, forma un \emph{campo}.

Ma ci sono ancora situazioni in cui i numeri razionali non permettono di 
risolvere, in modo esatto, problemi relativamente semplici da risolvere 
praticamente. 
Ad esempio è stato dimostrato (già qualche millennio fa) che se il lato di 
un quadrato è un numero razionale allora la sua diagonale non lo è:
nessun razionale elevato alla seconda dà \(2\).

\subsection{Il problema della misura}

Misurare significa ottenere il rapporto tra la grandezza da misurare e 
un'unità di misura.
Per misurare contiamo quante volte l'unità di misura è contenuta nella 
grandezza da misurare.
Se l'unità di misura non è contenuta un numero esatto di volte, si può 
cercare un opportuno sottomultiplo dell'unità di misura in modo che questo 
sia contenuto un numero esatto di volte. 

In pratica, quando si misura una grandezza, \(l\), si divide l'unità di 
misura in un certo numero naturale \(n\) di parti equivalenti, il massimo che 
la risoluzione dello strumento permette,
e poi si contano quante di queste parti (diciamo \(m\)) servono per ottenere 
una grandezza minore o uguale a quella da misurare mentre \(m+1\) di queste 
parti daranno una grandezza maggiore di quella che si deve misurare.

\affiancati{.49}{.49}{
Se si riesce a trovare un numero \(s\) tale che 
% \(\dfrac{s}{n}\) 
\emph{esse ennesimi} 
dell'unità di misura siano equivalenti alla grandezza, allora prendiamo 
\(\dfrac{s}{n}\) come misura della grandezza \(l\).
}{
\begin{center} \scalebox{.6}{\misura{6.7}} \end{center}
}

\affiancati{.49}{.49}{
\begin{center} \scalebox{.6}{\misura{7.5}} \end{center}
}{
Se non lo troviamo, consideriamo il più grande valore \(s\) che moltiplicato 
per il sottomultiplo dell'unità di misura dia una grandezza minore a \(l\) e 
prendiamo \(\frac{s}{n}\) come misura approssimata per difetto della 
grandezza \(l\).
}
In tal modo si otterrà un numero razionale (\(m/n\)) che sarà, in pratica, 
la migliore misura razionale approssimata per difetto della grandezza che si 
vuole misurare, e l'eventuale errore che si commette, minore di un 
\(n\)-esimo dell'unità di misura, sarà qualcosa di non apprezzabile dallo 
strumento usato.

\vspace{.4em}
\affiancati{.49}{.49}{
Ma, fissata un'unità di misura, molte grandezze non possono essere misurate 
\emph{esattamente} con i numeri razionali. 
Ad esempio in questi poligoni regolari, i segmenti \(h\) e \(d\) sono 
grandezze incommensurabili con il lato.
}{
\begin{center} \scalebox{.7}{\poligoniregolari} \end{center}
}

\paragraph{Approssimazioni di grandezze non commensurabili}

Le grandezze incommensurabili con l'unità di misura non si possono misurare 
esattamente contando i sottomultipli dell'unità di misura, ma di esse 
possiamo considerare tutte le approssimazioni razionali per difetto e 
per eccesso (e ognuna di queste si ottiene contando). 

Inoltre, comunque scelto un numero naturale  \(n \neq 0\), 
si potranno avere approssimazioni razionali una per eccesso e una per difetto 
che differiscono tra loro meno di un \(n\)-esimo dell'unità di misura.

Si potrebbe pensare che le due totalità delle approssimazioni razionali 
della misura quelle per difetto e quelle per eccesso costituiscano una 
indicazione della misura della grandezza data.

Ad esempio, possiamo approssimare la misura della diagonale \(d\) di un 
quadrato di lato 1, dividendo l’unità di misura in un sempre maggior 
numero di parti sempre più piccole.

\affiancati{.29}{.69}{
\begin{center}
\scalebox{.8}{\approssimazioni}
\end{center}
}{
% \begin{center}
 \begin{tabular}{ll}
\toprule
Valore per difetto di \(\sqrt{2}\) &Valore per eccesso di \(\sqrt{2}\) \\
\midrule
1& 2\\
1,4& 1,5 \\
1,41& 1,42\\
1,414& 1,415\\
1,4142& 1,4143\\
\ldots& \ldots\\
\bottomrule
\end{tabular}
% \end{center}
}

\begin{osservazione}
% Alcune caratteristiche delle classi di approssimazione di una grandezza 
% \(A\):
Per evitare di distinguere casi differenti, conveniamo che il razionale che 
è misura esatta di una grandezza \(A\) non sia un'approssimazione della 
stessa, ma che pure tale A sia approssimabile mediante razionali.
\begin{enumerate} [noitemsep]
\item 
Ogni grandezza ha insiemi non vuoti di approssimazioni razionali: \\
per difetto \(D_A\) e per eccesso \(E_A\). 
\item 
Le approssimazioni per difetto sono minori delle approssimazione per eccesso.
\item 
Un numero minore di un'approssimazione per difetto appartiene a \(D_A\) e un 
numero maggiore di un'approssimazione per eccesso appartiene a \(E_A\).
\item 
Le approssimazioni per difetto non hanno massimo e quelle per eccesso non 
hanno minimo.
\item 
Ogni razionale, con l'eventuale esclusione della misura esatta della 
grandezza \(A\), o è un'approssimazione per eccesso o è un'approssimazione 
per difetto.
\end{enumerate}
Da queste caratteristiche segue che per ogni numero naturale, diverso da 
zero, \(n\) ci sono un razionale \(q_1\) dell'insieme \(D_A\) e un razionale 
\(q_2\) dell'insieme \(E_A\) tali che \(q_2-q_1 < 1/n\).
\end{osservazione}

\subsection{Sezioni sui razionali}

Chiamiamo \emph{sezione dei razionali} una coppia di insiemi 
di razionali che abbia le caratteristiche delle coppie di insiemi costituiti 
dalle approssimazioni razionali per eccesso e da quelle per difetto di una 
grandezza.

Mentre le classi di approssimazioni di grandezze costituiscono un insieme 
numerabile, le sezioni dei razionali sono una quantità più che numerabile 
(tante quanti i sottoinsiemi dei numeri naturali). 

Per \emph{elementi separatori di una sezione dei razionali} 
si intendono gli enti maggiori di ogni numero razionale della classe 
inferiore e minori di ogni elemento della classe superiore.

Partendo dalle classi di approssimazione siamo arrivati al concetto simile, 
ma più generale di sezioni sui razionali. Ma cosa c'è tra le due classi della 
sezione? 
% Cosa è l'elemento separatore delle classi?

In certi casi è un numero razionale. Ad esempio se in una classe ci sono 
tutti i razionali minori di \(\frac{3}{4}\) e nell'altra tutti quelli 
maggiori di \(\frac{3}{4}\), questo numero razionale sarà un elemento 
separatore per tale sezione. Ma è l'unico?

Ma se l'elemento separatore non può essere un numero razionale come nel caso 
visto sopra dove abbiamo messo in una classe tutti i numeri minori di 
\(\sqrt{2}\) e nell'altra tutti i numeri maggiori di \(\sqrt{2}\) cosa viene 
individuato dalla sezione?

Dobbiamo inventarci qualcosa di nuovo e possiamo scegliere:
\begin{enumerate} [noitemsep]
\item un solo elemento separatore per sezione.
\item più elementi separatori per ogni sezione.
\end{enumerate}

\subsection{Numeri reali \(\R\)}

\affiancati{.49}{.49}{
\begin{center} \scalebox{.8}{\unoconnome} \end{center}
}{
Se decidiamo che ci sia un solo nuovo numero per ciascuna sezione, otteniamo 
una grandissima quantità di numeri che sono infinitamente di più rispetto a 
tutti i problemi che possiamo pensare di esprimere.
}

Questi nuovi numeri sono stati chiamati numeri \emph{reali} e l'insieme dei 
numeri reali viene indicato con il simbolo \(\R\).

Sono numeri molto interessanti che permettono di risolvere in modo semplice 
una grande quantità di problemi.

Ma non riescono a cogliere bene situazioni dinamiche riuscendo a 
rappresentare solo la posizione di un punto ma non la pendenza nel punto o la 
velocità istantanea nel passare per un punto.

I numeri reali formano un insieme \emph{ordinato}, \emph{denso} ma anche 
\emph{completo} cioè il numero individuato da una qualunque sezione sui 
razionali è un numero reale.
Questo permette di far corrispondere ad ogni lunghezza di un segmento 
un numero \emph{reale} e, viceversa, ad ogni numero \emph{reale} una 
lunghezza di un segmento. 
L'insieme dei Reali contiene un sottoinsieme isomorfo ai numeri 
razionali.

L'insieme dei numeri reali permette di risolvere tutti i problemi che 
possiamo incontrare?
\vspace{-1em}
\begin{center} \emph{Per fortuna no!} \end{center}
\vspace{-.5em}
Ci sono tipi di problemi che non possono essere risolti con i numeri reali.
Ad esempio calcolare la radice quadrata di numeri negativi. 
All'apparenza questo è un problema assurdo: calcolare la 
radice quadrata di un numero equivale a trovare la lunghezza del lato di un 
quadrato di cui si conosce l'area. 
Ora, trovare un quadrato con area piccola si può fare, magari anche con 
area nulla, impegnandosi un po', ma trovare un quadrato con area negativa 
è proprio impossibile. 
Ma come abbiamo visto per i naturali ci possono essere fenomeni nei quali 
hanno senso operazioni che in altri sistemi sono insensate.

Prima di procedere con i prossimi insiemi numerici, però, riflettiamo su 
una particolare proprietà degli insiemi numerici visti finora.

\subsubsection{Il postulato di Eudosso-Archimede}

Proviamo a fare un \emph{semplice} esperimento mentale. Prendo un foglio di 
carta e lo piego su se stesso un po' di volte. Che spessore raggiungo?
Per semplificarci i calcoli supponiamo che il foglio di carta abbia lo 
spessore di \(0,1mm = 0,0001m = 10^{-4}m\). 
Che spessore otterrò piegando il foglio su se stesso~64 volte?

Il calcolo è abbastanza semplice:

\begin{center}
 \begin{tabular}{ccc}
\toprule
Numero piegature & spessore ottenuto & in metri\\
\midrule
0 & 1 & \(10^{-4}\)\\
1 & 2 & \(2 \cdot 10^{-4}\)\\
2 & 4 & \(4 \cdot 10^{-4}\)\\
3 & 8 & \(8 \cdot 10^{-4}\)\\
4 & 16 & \(1,6 \cdot 10^{-3}\)\\
5 & 32 & \(3,2 \cdot 10^{-3}\)\\
6 & 64 & \(6,4 \cdot 10^{-3}\)\\
7 & 128 & \(1,28 \cdot 10^{-2}\)\\
\ldots& \ldots\\
n & \(2^n\) & \ldots\\
\bottomrule
\end{tabular}
\end{center}

Quindi piegando il foglio 64 volte ottengo uno spessore che è \(2^{64}\) 
volte lo spessore di partenza quindi basta calcolare:
\[2^{64} = 18.446.744.073.709.551.616\]
che convertito in metri dà: \(1.844.674.407.370.955m\) circa, che è uno 
spessore considerevole: dodicimila volte la distanza 
Terra-Sole:~\(149.600.000.000m\).

Si fa risalire ai matematici Eudosso e Archimede l'osservazione che per 
quanto piccolo si prenda un numero (ad esempio lo spessore di un foglio di 
carta), basta moltiplicarlo per un numero sufficientemente 
grande~(\(2^{64}\)) per farlo diventare maggiore di qualsiasi numero 
prefissato (ad esempio la distanza Terra-Sole).

\begin{postulato}[Eudosso-Archimede]
Dati due numeri positivi \(a, b\) si può sempre trovare un 
multiplo del più piccolo che sia maggiore del più grande:
\[\forall a, b \in \R \quad \exists n \in \N \quad | \quad na>b\]
\end{postulato}

Vale anche il contrario: 
dati due numeri positivi \(a, b\) si può sempre dividere il più grande 
in modo da ottenere un risultato minore del più piccolo.

Ma questa osservazione di Eudosso-Archimede non è un teorema, non è 
un'osservazione dimostrata, è un \emph{postulato}, un accordo fatto tra 
matematici, che può essere utile in moltissimi casi e che vale per tutti gli 
insiemi numerici visti finora. 
Ma cosa succede se ci accordiamo che \emph{non} valga il postulato di 
Eudosso-Archimede?

\subsection{Numeri iperreali \(\IR\)}

\affiancati{.49}{.49}{
Riprendiamo l'idea che ha portato alla creazione dei numeri reali.
Possiamo pensare che tra le classi di una sezione ci siano più elementi 
separatori, più numeri invece che uno solo.

La distanza tra questi dovrà essere minore di 
\(\frac{1}{n}\) per ogni numero naturale positivo \(n\). 
}{
\begin{center} \scalebox{.8}{\molticonnome} \end{center}
}

Anche in questo caso gli elementi separatori che si inventano devono indicare 
delle quantità come fanno i numeri, sicché dovranno essere integrati, 
dall'invenzione di operazioni come l'addizione e la moltiplicazione con le 
solite proprietà e di una relazione d'ordine. Sono utili se formano un campo 
numerico ordinato.

Se ci sono almeno due elementi separatori, allora ce ne sono infiniti, 
perché anche la media di questi è ancora un numero compreso tra le classi 
che formano la sezione di razionali.

Questi numeri sono alla base dell'Analisi Non Standard.

La differenza tra due numeri diversi che sono elementi separatori di una 
stessa sezione dei razionali deve essere un numero diverso 
da zero, ma, in valore assoluto, minore di \(\frac{1}{n}\) per ogni \(n\) 
dove \(n\) è un numero naturale diverso da zero: 
\[\dfrac{1}{n} \quad \forall n \in \Nz\]
dove con \(\Nz\) intendiamo \(\N \setminus \graffa{0}\).
Quindi la differenza \(\delta\) tra due di loro sarà tale che: 
\[\abs{\delta} < \dfrac{1}{n} \quad \forall n \in \Nz\]

Chiamiamo \emph{infinitesimo} un numero che in valore assoluto è minore di 
ogni numero razionale positivo.

Chiamiamo \emph{monade} ogni insieme di numeri che hanno tra di loro una 
distanza infinitesima.

\subsection{I numeri complessi \(\C\)}
\label{subsec:insnum_complessi}

Riprendiamo il problema della radice di numeri negativi. Si può ampliare 
l'insieme dei numeri reali aggiungendo i numeri che sono le 
radici di tutti i numeri anche di quelli negativi. Per fare ciò si devono 
aggiungere molti altri numeri (infiniti) tutti questi nuovi numeri sono 
stati chiamati numeri \emph{immaginari} che combinati con i numeri reali 
formano l'insieme dei numeri \emph{complessi} insieme che viene indicato 
con 
\(\C\). 
Anche per i numeri complessi tutti gli infiniti nuovi numeri si ottengono 
con la semplice aggiunta di un solo nuovo numero: \emph{l'unità 
immaginaria} indicato con il simbolo \(i\) o con il simbolo \(j\).
\begin{definizione}
 L'\textbf{unità immaginaria} è quel numero che elevato alla seconda dà 
come risultato \(-1\):
\[i^2 = -1\]
\end{definizione}

Questi numeri hanno molte applicazioni tecniche, ma risultano anche 
affascinanti da un punto di vista estetico. La ripetizione di un paio di 
calcoli aritmetici tra numeri complessi produce il sorprendente insieme di 
Mandelbrot.

\begin{figure}
\begin{center}
\begin{inaccessibleblock}
[Immagine di una porzione dell'insieme di Mandelbrot.]
\includegraphics[scale=0.40]{img/fractal.jpg}
\end{inaccessibleblock}
\caption{Porzione dell'insieme di Mandelbrot.}
\label{fig:mandelbrot}
\end{center}
\end{figure}

Ma dato che l'insieme dei reali oltre che essere un campo ordinato è anche 
completo, non è possibile aggiungere elementi ai reali senza perdere 
qualche proprietà dell'insieme numerico. 
Nel caso dei complessi l'insieme ottenuto non è ordinato.

È possibile prendere un sottoinsieme dei Complessi che sia isomorfo ai Reali.

Un numero complesso viene rappresentato dall'espressione:
\[a +ib\]
Dove \(a \stext{e} b\) sono numeri reali.
Se \(a \stext{e} b\) sono numeri iperreali otterremo i numeri 
ipercomplessi \(\IC\).

% \begin{wrapfloat}{figure}{r}{0pt}
% \begin{inaccessibleblock}
% [Immagine di una porzione dell'insieme di Mandelbrot.]
% \includegraphics[scale=0.30]{img/fractal.jpg}
% \end{inaccessibleblock}
% \caption{Porzione dell'insieme di Mandelbrot.}
% \label{fig:mandelbrot}
% \end{wrapfloat}
% Ma dato che l'insieme dei reali oltre che essere un campo ordinato è anche 
% completo, non è possibile aggiungere elementi ai reali senza perdere 
% qualche proprietà dell'insieme numerico. 
% Nel caso dei complessi l'insieme ottenuto non è totalmente ordinato.
% 
% Inutile dire che possiamo prendere un sottoinsieme dei Complessi che sia 
% isomorfo ai Reali.

% \vspace{24pt}

\newpage %--------------------------------------------------

\input{\folder iperreali.tex}

% \input{\folder naturali_iperreali_ese.tex}
