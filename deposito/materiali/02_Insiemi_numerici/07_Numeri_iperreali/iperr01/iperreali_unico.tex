% (c) 2015 Daniele Zambelli daniele.zambelli@gmail.com % (c) 2017 Bruno Stecca

% % \vspace{-2ex}\input{\folder lbr/tab002}\vspace{-2ex} 
% \begin{inaccessibleblock} 
% [Immagine di una porzione dell'insieme di Mandelbrot.] 
% \vspace{-2ex} 
% \begin{center} \includegraphics[scale=0.25]{img/hiero3673.png} \end{center} 
% \vspace{-2ex} 
% \end{inaccessibleblock}

\input{\folder iperreali_gra.tex}

\chapter{Iperreali}

\section{Dai numeri naturali ai numeri complessi} 
\label{sec:iperreali:01_introduzione}

Riprendiamo i diversi insiemi numerici che abbiamo imparato a conoscerete 
mettendo in evidenza il loro ruolo come modelli per risolvere alcune classi di 
problemi e le loro caratteristiche.

\subsection{I numeri naturali \(\N\)} 
\label{subsec:insnum_naturali}

I primi numeri che abbiamo incontrato sono i numeri naturali, ingenuamente 
chiamati ``i numeri'' per brevità. Sono quelli che permettono di contare 
oggetti. Se sul banco ho un quaderno, una penna e un libro 
posso dire che ci sono~3 oggetti. Però definire i numeri naturali come ``quei 
numeri che permettono di contare oggetti'' non è abbastanza preciso. Per 
esempio, si può capire come il numero Zero abbia avuto difficoltà a farsi 
accettare come numero: servirebbe a contare un gruppo di oggetti dove non c'è
niente da contare, quindi si direbbe inutile. Invece sappiamo che è molto comodo 
considerare lo zero come un numero. I numeri usati ``per contare'' sono chiamati
numeri \emph{naturali} e il loro insieme viene indicato con~\(\N\).

I numeri naturali formano un \emph{insieme ordinato}.
Nei numeri naturali sono definite l'addizione, la moltiplicazione che sono 
sempre possibili. In queste due \emph{strutture} \(\tonda{\N, +}\) e 
\(\tonda{\N, \times}\) valgono le proprietà: associativa, commutativa e 
l'esistenza dell'elemento neutro per entrambe e la proprietà distributiva.

Nei numeri naturali è definita anche l'operazione \emph{potenza} ma c'è un caso 
in cui la potenza non è determinata: quando sia la base sia l'esponente sono 
uguali a zero.

Oltre a queste operazioni, sono definite anche le loro inverse: la 
\emph{sottrazione}, la \emph{divisione} e la \emph{radice}, ma queste non sempre 
hanno una soluzione nei numeri naturali.

D'altra parte se su un tavolo ho~5 oggetti posso toglierne~3 e ne restano~2: 
\(5-3=2\)

Ma se sul tavolo ho~3 oggetti non ha senso cercare di toglierne~5!

\subsection{I numeri interi \(\Z\)} 
\label{subsec:insnum_interi}

I numeri possono però essere utilizzati anche come modelli di altre situazioni. 

Supponiamo che io abbia una sequenza di oggetti e che desideri riferirmi ad ognuno
con un numero che equivale al suo indirizzo o indice. In certi casi potrei cercare
il primo elemento della sequenza e chiamarlo zero, quello che viene dopo lo chiamo 
uno e così via. Ma se mi trovassi a lavorare principalmente con gli elementi 
compresi tra il 273° elemento e il 310° elemento, questo modo di fare sarebbe 
piuttosto scomodo. Molto più semplice è chiamare zero il 273° elemento e 
partire da lì a contarli: in questo modo i numeri da usare saranno quelli 
compresi tra~0 e~37. 
Ci sono inoltre delle situazioni in cui è difficile, 
o impossibile, determinare il \emph{primo} elemento della sequenza e anche in 
questo caso ci si può mettere d'accordo di assegnare l'indice zero ad un preciso 
elemento della sequenza.

In questo caso l'elemento con indice \emph{zero} non sarà il \emph{primo} 
elemento della sequenza, ma uno interno alla sequenza. Quindi è possibile
muoversi sia sopra lo zero, sia sotto lo zero. I nuovi numeri, che possono 
precedere o seguire lo zero, non hanno nomi nuovi e diversi dai naturali, ma 
hanno gli stessi nomi, preceduti semplicemente dai segni:~``\(+\)'', per 
i numeri dopo lo zero, e~``\(-\)'', per i numeri prima dello zero. Questi nuovi 
numeri sono chiamati numeri \emph{interi} e l'insieme di questi numeri viene 
indicato con~\(\Z\).

In questa situazione l'addizione può essere vista come muoversi nel verso della 
crescita dei numeri e la sottrazione come muoversi nel verso della decrescita 
dei numeri. Dato che lo zero è un elemento convenzionale non c'è nessun problema 
a togliere~5 da~3: semplicemente si arriverà nella posizione~2 prima dello zero, 
detta anche~\(-2\).

In questo insieme di numeri è sempre definita anche la sottrazione, anzi la 
sottrazione diventa semplicemente un caso particolare di addizione.
Nei numeri interi valgono tutte le proprietà dei naturali, inoltre esiste 
l'inverso additivo di ogni elemento.

I numeri interi permettono di risolvere sempre equazioni del tipo: 
\[x+a=0\]
Il sottoinsieme di \(\Z\) formato dallo zero e da tutti i numeri positivi si 
comporta esattamente come l'insieme dei numeri Naturali. Diremo che questo 
sottoinsieme è isomorfo all'insieme~\(\N\) e questo ci permette di usare 
indifferentemente \(+7\) o \(7\) senza dover precisare che \(+7\) è un elemento 
di \(\Z\) mentre \(7\) è un elemento di \(\N\).

Anche questi numeri però non riescono a realizzare un modello in certe 
situazioni che invece, nella pratica, si possono risolvere facilmente con un 
po' di creatività. Ad esempio come possiamo dividere~3 uova, in parti uguali, 
tra~4 persone?

\subsection{I numeri razionali \(\Q\)} 
\label{subsec:insnum_razionali}

Con le tre uova faccio una frittata che divido facilmente in~4 parti uguali. 
Possiamo costruire dei numeri che permettano di calcolare sempre il quoziente 
esatto di due numeri interi anche quando la divisione tra i due dà un resto 
diverso da zero. Questi nuovi numeri sono chiamati numeri \emph{razionali} e 
l'insieme di questi numeri viene indicato con~\(\Q\).

Mentre nei naturali e negli interi ad ogni numero corrisponde un \emph{nome} ben 
preciso, nei razionali lo stesso numero può essere indicato con infiniti nomi 
diversi. Ad esempio il numero che si ottiene dividendo~1 in due parti uguali può 
essere indicato in uno di questi modi: 
\[\frac{1}{2}=\frac{3}{6}=\dots=\frac{45}{90}=\frac{132}{264}=\dots=0,5\]
Ogni numero razionale può essere rappresentato con un numero con la virgola o 
con una qualunque delle infinite frazioni equivalenti.

Con i numeri razionali si può sempre calcolare il risultato della divisione tra 
due numeri (naturali, interi o razionali) tranne il caso particolare in cui il 
divisore sia uguale a zero. In questo caso la divisione non può essere eseguita.
Nei numeri razionali valgono tutte le proprietà degli interi, inoltre esiste 
l'inverso moltiplicativo di ogni elemento diverso da zero.

I numeri razionali permettono di risolvere sempre equazioni del tipo: 
\[ax+b=0 \quad \text{ con } \quad a \neq 0\]
I razionali hanno una caratteristica particolare che non avevano né i naturali 
né gli interi: formano un insieme \emph{denso} cioè tra due numeri razionali, 
per quanto vicini, se ne può trovare sempre almeno un altro.

Anche tra i razionali si può trovare un sottoinsieme isomorfo all'insieme degli 
interi, cioè che si comporta come l'insieme degli interi: è il sottoinsieme dei 
numeri che, scritti sotto forma di frazioni hanno come numeratore un multiplo 
del denominatore o che, ridotte ai minimi termini, hanno per denominatore uno. 
Questo fatto ci permette di poter scrivere:~\(-\dfrac{7}{1} = -7\) senza dover 
precisare che il primo numero appartiene a \(\Q\) e il secondo a \(Z\).

Ma ci sono ancora situazioni in cui i numeri razionali non permettono di 
risolvere problemi relativamente semplici da risolvere praticamente. Ad esempio 
è stato dimostrato (già qualche millennio fa) che se il lato di un quadrato è un 
numero razionale allora la sua diagonale non lo è. 

\subsection{I numeri reali \(\R\)} 
\label{subsec:insnum_reali}

Se prendiamo un quadrato di lato~1, per il teorema di Pitagora, la sua diagonale 
risulta lunga~\(\sqrt{2}\). La radice di~2 è quel numero che elevato alla 
seconda dà come risultato~2. Ebbene, abbiamo dimostrato che nessun numero 
razionale moltiplicato per se stesso dà come risultato~2 (vedi il secondo 
volume). 

Qualunque numero razionale, elevato al quadrato, o dà un numero più piccolo o un 
numero più grande di~2.

Possiamo costruire due sottoinsiemi dell'insieme \(\Q\) in modo da 
mettere nel primo tutti i numeri che elevati al quadrato sono minori di~2 e nel 
secondo tutti quelli che danno un risultato maggiore:

\begin{center} 
\begin{tabular}{ll} 
\toprule Valore per difetto di \(\sqrt{2}\) 
&Valore per eccesso di \(\sqrt{2}\) \\ 
\midrule 
1& 2\\ 
1,4& 1,5 \\ 
1,41& 1,42\\ 
1,414& 1,415\\ 
1,4142& 1,4143\\ 
\ldots& \ldots\\ 
\bottomrule \end{tabular} 
\end{center}

Due sottoinsiemi costruiti in questo modo si chiamano \emph{classi contigue} di 
numeri razionali, cioè due sottoinsiemi di \(\Q\) tali che ogni elemento del 
primo è minore di qualunque elemento del secondo e che nel primo sottoinsieme ci 
sono numeri che si avvicinano quanto si vuole a certi numeri del secondo 
sottoinsieme.

Il vero valore della  \(\sqrt{2}\) si troverà sempre tra due numeri razionali che, 
migliorando la precisione, possiamo trovare sempre più vicini, restringendo così
a piacere l'intervallo in cui si trova. 

{\noindent
\begin{minipage}{.28\textwidth}
Ma questo intervallo conterrà sempre infiniti altri numeri 
razionali, a meno che non riusciamo a restringerlo \emph{all'infinito}: in 
questo caso nell'intervallo infinitesimo non potrà trovarsi nessun numero 
razionale.
I matematici hanno così deciso che in quello spazio ci sta \emph{uno 
e un solo numero}, di tipo nuovo: \emph{un numero reale}.
\end{minipage}
\hfill
\begin{minipage}{.70\textwidth}
\begin{inaccessibleblock} 
[Intervallo infinitesimo che contiene un solo numero.] 
\begin{center}
\unoconnome 
{\footnotesize Nell'intervallo infinitesimo c'è un solo numero.} 
\end{center}
\label{fig:mandelbrot} 
\end{inaccessibleblock} 
\end{minipage}
}


% I matematici hanno pensato bene di inventare un numero da mettere in 
% questo intervallo e hanno chiamato l'insieme di tutti i numeri che si possono 
% inserire in questi intervalli infinitesimi: \emph{numeri reali}.

Due classi contigue di numeri razionali definiscono un \emph{numero reale}. 
L'insieme di tutti i numeri reali viene rappresentato dal simbolo: \(\R\). 

% Dato un \emph{numero qualsiasi} possiamo sempre 
% realizzare due classi contigue di numeri razionali. 
% Se questo \emph{numero qualsiasi} appartiene al secondo sottoinsieme è, 
% evidentemente un numero razionale, se non appartiene ai due sottoinsiemi è 
% un numero irrazionale. 
% Ognuna di queste partizioni, dette anche sezioni di Dedekind, può essere 
% considerata come un numero, cioè è possibile costruire un ordine tra le 
% sezioni, sommarle, moltiplicarle, \dots


\subsubsection{Alcune proprietà dei numeri reali} 
\label{subsubsec:insnum_reali}

% I numeri reali formano un insieme \emph{ordinato}, \emph{denso} e 
% \emph{completo}: \(\R\). 
% 
% È un insieme \emph{ordinato} perché dati due numeri reali diversi sappiamo 
% sempre indicare il maggiore e il minore. 
% 
% È \emph{denso} perché fra due numeri 
% reali diversi, per quanto vicini, se ne può sempre trovare almeno un altro. 


Nei numeri reali valgono tutte le proprietà dei razionali, in più ai reali è 
stata assegnata un'altra proprietà: la \emph{completezza}.

La questione è piuttosto complicata. 
% Come abbiamo visto %\Vedi \ref{}) 
% I numeri razionali non sono sufficienti per indicare le lunghezze di tutti i 
% segmenti (ad es. se la misura del lato di un quadrato è un numero razionale, 
% la misura della sua diagonale non potrà essere razionale). 
% Sono stati inventati così i numeri reali che possono rappresentare la 
% lunghezza di un qualunque segmento. % L'esempio classico riguarda la 
% lunghezza della diagonale di un quadrato: 
% \(l=1 \sRarrow d=\sqrt{2}\) 
% Possiamo dividere l'insieme dei razionali in due sottoinsiemi \(A\) che 
% contiene tutti i numeri minori o uguali a \(\sqrt{2}\) e \(B\) che contiene 
% tutti i numeri maggiori o uguali a \(\sqrt{2}\): per quanto cerchi di essere 
% sempre più preciso, tra questi due insiemi resterà sempre uno spazio che 
% contiene \emph{infiniti} altri numeri razionali. 
% Ma se potessimo continuare a restringere questo spazio all'infinito otterremmo 
% un \emph{intervallo infinitamente piccolo} che non può però contenere un 
% numero razionale. 
% I matematici hanno così deciso che in quello spazio ci sta 
% \emph{uno e un solo numero}, di tipo nuovo: \emph{un numero reale}.


La \emph{completezza} di \(\R\) è una proprietà che i matematici richiedono per 
usare la retta dei numeri come modello, cioè perché le proprietà dell'insieme 
\(\R\) rispecchino le caratteristiche geometriche della retta, che per questo è 
detta retta reale. La retta reale non ha strappi o buchi, è continua: è 
immaginata come un insieme di punti uniti uno all'altro. La completezza di 
\(\R\) rappresenta tale continuità. Vediamo più in dettaglio.

% Se gli unici numeri conosciuti fossero i naturali, gli interi e i razionali, non 
% potremmo rappresentare determinate lunghezze sulla retta dei numeri. 
% Per quanto riguarda la \emph{completezza} di \(\R\) la questione è piuttosto 
% complicata. 
% Come abbiamo visto 
%\Vedi \ref{}) 
% Come i numeri interi non bastano 
% a rappresentare i risultati delle divisioni, così i numeri razionali non sono 
% sufficienti per tutte le necessità di misurare conosciute dai matematici: i 
% numeri razionali non formano un insieme completo. È stato così ideato 
% l'insieme dei numeri reali, con i quali si rappresenta qualsiasi lunghezza. 
% 
% Prendiamo ad esempio una circonferenza  di diametro \(1\), tagliamola e 
% stendiamola sulla retta dei numeri, a partire da \(0\). 
% Il secondo estremo cadrà allora in una posizione vicina a \(3,14\), appena 
% oltre, ma non abbastanza da arrivare a \(3,15\). Se siamo in grado di 
% rappresentare i millesimi, vedremo che la misura supera \(3,141\), 
% ma non arriva a \(3,142\). 
% E con successive osservazioni sulla retta, sempre più raffinate, arrivando 
% ai milionesimi, ai miliardesimi, ecc. vediamo che la misura esatta sta 
% sempre fra due numeri razionali, ma su nessuno esattamente.\\ 
% Manca il nome del punto su cui cade quell'estremo, cioè manca il numero 
% che esprime la misura.

% Un altro esempio classico riguarda la lunghezza della diagonale di un 
% quadrato:\\ 
% \(l=1 \sRarrow d=\sqrt{2}\). 
% Possiamo dividere l'insieme dei razionali in due sottoinsiemi: \(A\) che 
% contiene tutti i numeri minori o uguali a \(\sqrt{2}\) e \(B\) che contiene 
% tutti i numeri maggiori o uguali a \(\sqrt{2}\). 
% 
% Facciamo una prova con uno dei numeri nell'insieme \(A\), per esempio quello 
% che % esprime il risultato della radice con la precisione del milionesimo: 
% \(1,414213\). \\ 
% Verifichiamo: \(1,414213^2=1,999998409\), cioè non si arriva 
% esattamente a 2. Infatti, per quanto si cerchi di essere sempre più precisi 
% proseguendo nel calcolo dei decimali e inserendo in \(A\) i risultati per 
% difetto e in \(B\) quelli per eccesso, tra \(A\) e \(B\) resterà sempre uno 
% spazio che contenere \emph{infiniti} altri numeri razionali. Ma se potessimo 
% continuare a restringere questo spazio all'infinito otterremmo un 
% \emph{intervallo infinitamente piccolo} che non può però contenere un numero 
% razionale. 

% Come abbiamo visto per la diagonale del quadrato di lato~1, i

I numeri reali 
permettono di esprimere la misura di un qualunque segmento e di realizzare così 
una corrispondenza biunivoca tra i punti di una retta e i numeri reali. 
Possiamo cioè far corrispondere ad ogni punto della \emph{retta reale} un 
numero \emph{reale} e, viceversa, ad ogni numero \emph{reale} un punto della 
\emph{retta reale}. 
Poiché immaginiamo la retta come una linea ``priva di buchi'', realizzata da un 
tratto continuo,
allora immaginiamo anche i numeri reali come un insieme che comprende tutti i 
numeri, anche quelli che riempiono \emph{tutti} i buchi lasciati
dai razionali, \emph{un solo numero per ogni buco}. 

Così numeri reali coprono tutta la retta, non c'è posto per altri numeri.
% : c'è almeno un 
% punto in ogni posizione, anche 
% osservando la retta al microscopio, con qualsiasi ingrandimento (ingrandimento 
% reale, come vedremo).
L'insieme dei numeri reali \(\R\) si dice quindi \emph{completo}.

Quando abbiamo parlato degli assi cartesiani abbiamo visto che la retta 
permette di rappresentare i numeri a patto che sia dotata di \emph{origine}, 
\emph{unità di misura} e \emph{verso}.
Se usiamo la retta reale come immagine dell'insieme \(\R\) è perché si tratta di 
una rappresentazione efficace. Ma ricordiamoci sempre che un insieme in 
matematica è un oggetto astratto, quindi la retta reale è solo un modello che ci 
aiuta a capire le proprietà dell'insieme \(\R\).

Per esempio, la retta permette di visualizzare facilmente la densità di \(\R\): 
se due numeri diversi sono troppo vicini per poterli separare, basta prendere un 
microscopio adeguato e si potrà vedere che tra questi due punti ce ne sono 
infiniti altri.

Per quanto riguarda l'ordinamento in \(\R\), se la retta cresce verso destra, 
allora 
\begin{center} 
\(b > a \quad \text{equivale a:} \quad 
b \quad \text{si trova più a destra di}\quad a\) 
\end{center}


\subsubsection{Il postulato di Eudosso-Archimede}

La completezza è strettamente legata a un'altra proprietà, la 
\emph{proprietà archimedea}: 
dati due numeri qualunque, si può sempre trovare un opportuno 
multiplo del più piccolo che sia maggiore del più grande. 
Sulla retta: dati due segmenti, potrai sempre trovare un multiplo del più breve 
che superi il più lungo.

Proviamo a fare un \emph{semplice} esperimento mentale partendo dalla 
suggestione della frase: ``sottile come un foglio di carta, distante come il 
Sole''. 
Prendo un foglio di carta e lo piego su se stesso un po' di volte. Che spessore 
raggiungo? 
Per semplificarci i calcoli supponiamo che il foglio di carta abbia 
lo spessore di \(0,1mm = 0,0001m = 10^{-4}m\). 
Che spessore otterrò piegando il foglio su se stesso~64 volte?

Il calcolo è abbastanza semplice:

\begin{center} 
\begin{tabular}{ccc} 
\toprule Numero piegature & spessore ottenuto & in metri\\ 
\midrule 0 & 1 & \(10^{-4}\)\\ 
1 & 2 & \(2 \cdot 10^{-4}\)\\ 
2 & 4 & \(4 \cdot 10^{-4}\)\\ 
3 & 8 & \(8 \cdot 10^{-4}\)\\ 
4 & 16 & \(1,6 \cdot 10^{-3}\)\\ 
5 & 32 & \(3,2 \cdot 10^{-3}\)\\ 
6 & 64 & \(6,4 \cdot 10^{-3}\)\\ 
7 & 128 & \(1,28 \cdot 10^{-2}\)\\ 
\ldots& \ldots\\ 
n & \(2^n\) & \ldots\\ 
\bottomrule 
\end{tabular} 
\end{center}

Quindi piegando il foglio 64 volte ottengo uno spessore che è \(2^{64}\) volte 
lo spessore di partenza quindi basta calcolare: 
\[2^{64} = 18.446.744.073.709.551.616\] 
che convertito in metri dà: 
\(1.844.674.407.370.955m\) circa che è uno spessore considerevole, quasi 
duemila 
volte la distanza Terra-Sole:~\(149.600.000.000m\).

Si fa risalire ai matematici Eudosso e Archimede l'osservazione che per quanto 
piccolo si prenda un numero (ad esempio lo spessore di un foglio di carta), 
basta moltiplicarlo per un numero sufficientemente grande~(\(2^{64}\)) per 
farlo diventare maggiore di un qualsiasi altro numero (ad esempio la distanza 
Terra-Sole).

\begin{postulato}[Eudosso-Archimede] Dati due numeri positivi \(a, b\) si può 
sempre trovare un multiplo del più piccolo che sia maggiore del più grande: 
\[\forall a, b \in \R \quad \exists n \in \N \quad | \quad na>b\] 
\end{postulato}

Vale anche il contrario: per quanto grande sia un numero posso dividerlo per un 
numero abbastanza grande da farlo diventare più piccolo di un qualunque numero.

Ma questa osservazione di Eudosso-Archimede non è un teorema, non è 
un'osservazione dimostrata, è un postulato, un accordo fatto tra matematici che 
può essere utile in moltissimi casi e che vale per tutti gli insiemi numerici 
visti finora. Ma cosa succede se ci accordiamo che \emph{non} valga il 
postulato di Eudosso-Archimede? 

Avendo deciso (postulato) che \(\R\) è completo, non è possibile inserire nella 
retta reale dei punti che non corrispondano a numeri reali. Se vogliamo 
aggiungere dei nuovi numeri che non siano reali dovremo metterli fuori dalla 
retta o usare un nuovo tipo di retta, un nuovo modello.
% 
% % TODO magari spostare questi discorsi all'inizio degli iperreali??? 
% La scelta di completare gli \emph{intervalli infinitamente piccoli} 
% riempiendoli 
% con \emph{un solo numero} è molto comoda, ma non è l'unica possibile. Nella 
% prossima sezione vedremo che in quello spazio infinitamente piccolo si possono 
% far stare infiniti numeri conservando quasi tutte le proprietà dei numeri 
% reali. Inventeremo così un nuovo insieme numerico e un nuova retta: perderemo 
% qualche proprietà, ma ne acquisiremo di nuove. Molto interessanti \dots
% 
% Ma di questo parleremo tra un po'.

Anche l'insieme dei Reali contiene un sottoinsieme isomorfo ai numeri razionali.

Bene: l'insieme dei numeri reali permette di risolvere tutti i problemi che 
possiamo incontrare? 

\vspace{-1em} \begin{center} \emph{Per fortuna no!} \end{center} \vspace{-.5em} 

Ci sono operazioni tra reali che non hanno soluzione reale, 
ad esempio: calcolare la radice quadrata di numeri negativi. 
All'apparenza questo è un calcolo del tutto assurdo: calcolare la radice 
quadrata di un numero equivale a trovare la lunghezza del lato di un quadrato di 
cui si conosce l'area.
Ora, trovare un quadrato con area piccola si può fare, magari anche con area 
nulla, impegnandosi un po', ma trovare un quadrato con area negativa è proprio 
impossibile. Ma come abbiamo visto per i naturali ci possono essere fenomeni 
nei quali hanno senso operazioni che in altri sistemi sono insensate.

\subsection{I numeri complessi \(\C\)} \label{subsec:insnum_complessi}

Riprendiamo il problema della radice di numeri negativi. Si può ampliare 
l'insieme dei numeri reali aggiungendo i numeri che sono le radici di tutti i 
numeri anche di quelli negativi. Per fare ciò si devono aggiungere molti altri 
numeri (infiniti) tutti questi nuovi numeri sono stati chiamati numeri 
\emph{immaginari} che combinati con i numeri reali formano l'insieme dei numeri 
\emph{complessi}, insieme che viene indicato con \(\C\). Anche per i numeri 
complessi tutti gli infiniti nuovi numeri si ottengono con la semplice aggiunta 
di un solo nuovo numero: \emph{l'unità immaginaria} indicato con il simbolo 
\(i\) (o con il simbolo \(j\)). \begin{definizione} L'\textbf{unità immaginaria} è 
quel numero che elevato alla seconda dà come risultato \(-1\): \[i^2 = -1.\] 
\end{definizione}

{\noindent
\begin{minipage}{.28\textwidth}
Questi numeri hanno molte applicazioni tecniche, ma risultano anche 
affascinanti da un punto di vista estetico. 
La ripetizione di un paio di calcoli aritmetici tra numeri complessi produce il 
sorprendente insieme di Mandelbrot\footnotemark.
Ma dato che l'insieme dei reali oltre che essere un campo ordinato è anche 
completo, non è possibile aggiungere elementi ai reali senza perdere qualche 
proprietà dell'insieme numerico. 
\end{minipage}
\hfill
\begin{minipage}{.70\textwidth}
\begin{inaccessibleblock} 
[Immagine di una porzione dell'insieme di Mandelbrot.] 
\begin{center}
\includegraphics[scale=0.35]{img/fractal.jpg} \\
{\footnotesize Porzione dell'insieme di Mandelbrot.} 
\end{center}
\label{fig:mandelbrot} 
\end{inaccessibleblock} 
\end{minipage}
}

\footnotetext{Partendo da un punto P nel piano complesso si applicano 
successivamente le seguenti iterazioni:
\[Z_0 = P \qquad Z_1 = Z_0^2 + P \qquad Z_2 = Z_1^2 + P \qquad Z_3 = \dots\]
Cercando la parola ``Mandelbrot'' con un qualsiasi motore di ricerca, si 
possono trovare molte immagini e generatori di immagini di questo insieme.}

\vspace{.5em}
L'insieme dei complessi mantiene tutte le proprietà dei reali tranne 
\emph{l'ordinamento}.

È possibile prendere un sottoinsieme dei Complessi che sia isomorfo ai Reali.


% \begin{wrapfigure}{R}{80mm}
% \begin{center} 
% \begin{inaccessibleblock} 
% [Immagine di una porzione dell'insieme di Mandelbrot.] 
% \includegraphics[scale=0.30]{img/fractal.jpg} 
% \end{inaccessibleblock} 
% \caption{Porzione dell'insieme di Mandelbrot.} 
% \label{fig:mandelbrot} 
% \end{center} 
% \end{wrapfigure}
% 
% Questi numeri hanno molte applicazioni tecniche, ma risultano anche 
% affascinanti 
% da un punto di vista estetico. 
% La ripetizione di un paio di calcoli aritmetici tra numeri complessi produce 
% il sorprendente insieme di Mandelbrot.
% 
% Ma dato che l'insieme dei reali oltre che essere un campo ordinato è anche 
% completo, non è possibile aggiungere elementi ai reali senza perdere qualche 
% proprietà dell'insieme numerico. L'insieme dei complessi mantiene tutte le 
% proprietà dei reali tranne \emph{l'ordinamento}.
% 
% È possibile prendere un sottoinsieme dei Complessi che sia isomorfo ai Reali.

\newpage %-------------------------------------------

\input{\folder iperreali.tex}

\begin{esempio} {~}

\begin{minipage}{.44\textwidth} Calcola la tangente all'ellisse di equazione: 
\(4x^2+3y^2=48\) nel punto di coordinate \(T\punto{3}{2}\).

La funzione che descrive la parte di ellisse contenente \(T\) è: 
\(y=+\sqrt{-\dfrac{4}{3}x^2+16}\)\\ L'equazione del fascio di rette per \(T\) 
è:\\ \(y=m\tonda{x-3}+2\) \begin{align*} m&=\pst{\dfrac{d y}{d x}}= 
\pst{\dfrac{f(3+\epsilon)-f(3)}{\epsilon}}=\\ 
&=\pst{\dfrac{\sqrt{-\dfrac{4}{3}(9+\epsilon)^2+16}-2}{\epsilon}}= \end{align*} 
\end{minipage} \hfill \begin{minipage}{.54\textwidth} 
\begin{center}\iperellisse\end{center} \end{minipage} \begin{align*} 
m&=\pst{\dfrac{\sqrt{-\dfrac{4}{3}(3+\epsilon)^2+16}-2}{\epsilon} \cdot 
\dfrac{\sqrt{-\dfrac{4}{3}(3+\epsilon)^2+16}+2} 
{\sqrt{-\dfrac{4}{3}(3+\epsilon)^2+16}+2}}=\\ 
&=\pst{\dfrac{-\dfrac{4}{3}(3+\epsilon)^2+16-4} { 
\epsilon\tonda{\sqrt{-\dfrac{4}{3}(3+\epsilon)^2+16}+2}}}= 
\pst{\dfrac{-8\epsilon-\frac{4}{3}\epsilon^2}{4 \epsilon}}= \pst{\dfrac{-8 
\cancel{\epsilon}}{4 \cancel{\epsilon}}}=-2 \end{align*} E la tangente è quindi: 
\[y=m \tonda{x-x_0}+y_0 \sRarrow y=-2 \tonda{x-3}+2 \sRarrow y=-2x+8\]

\end{esempio}

\begin{comment}

\begin{esempio} % limite notevole espon. \(\pst{\tonda{1+\dfrac{k}{N}}^N} ~ 
\stackrel{1}{=} ~  \pst{\tonda{1+\dfrac{1}{M}}^{kM}} ~ \stackrel{2}{=} ~ 
\pst{\quadra{\tonda{1+\dfrac{1}{M}}^M}}^k ~ \stackrel{3}{=} ~ e^k\).\\

Dove le uguaglianze hanno i seguenti motivi: \begin{enumerate} [nosep] \item un 
altro sporco trucco: la sostituzione. Supponiamo \(\frac{k}{N}=\dfrac{1}{M}\). 
Allora \(N=kM\); \item una potenza di potenza è una potenza che ha... \item per 
la definizione del numero \(e\) e per le proprietà della funzione \(\st()\). 
\end{enumerate} \end{esempio}

\begin{esempio} % limite notevole log. \(\pst{\dfrac{a^\epsilon-1}{\epsilon}} ~ 
\stackrel{1}{=} ~  \pst{\dfrac{\delta}{\log_a{(\delta+1)}}} ~ \stackrel{2}{=} ~ 
\pst{\frac{1}{\dfrac{\log_a{(\delta+1)}}{\delta}}} ~ \stackrel{3}{=} ~ 
\pst{\frac{1}{\dfrac{1}{\ln{a}}}}=\ln{a}\).\\

Dove le uguaglianze hanno i seguenti motivi: \begin{enumerate} [nosep] \item 
ancora una sostituzione: poniamo \(a^\epsilon-1=\delta\). Allora 
\(\epsilon=\log_a(\delta+1)\); \item una capriola algebrica: oplà! \item per le 
forme di indecisione discusse a proposito del numero di Nepero e per il 
cambiamento di base; \end{enumerate} \end{esempio}

\begin{esempio} % limite notevole seno e coseno \(\pst{\dfrac{1-\cos 
\delta}{\sin \delta}}\)=0.\\ ~ Dove l'uguaglianza si giustifica per quanto detto 
a proposito dell'ordine degli infinitesimi, ma gli appassionati del calcolo 
possono provare a moltiplicare il numeratore e il denominatore per ... 
\end{esempio}

\end{comment}
