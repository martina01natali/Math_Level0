% (c) 2014 Daniele Zambelli - daniele.zambelli@gmail.com
% 
% Tutti i grafici per il capitolo relativo agli iperreali
%


%--------------------------
% Reali - Iperreali


\newcommand{\iperrealiset}{% Insieme degli iperreali.
  \disegno{
    \draw [rounded corners=10, thick, blue!50!black] 
          (0,0) rectangle ++(8, 10)
          node [midway, above=4.7] {Iperreali}
          node [midway, above=3.6, rotate=10, red!50!black] {Infiniti}
          (0, 5.8) -- (8, 8.2)
          node [midway, below=.2, rotate=17, orange!50!black] {Finiti}
          (2.5, 4.2) ellipse [x radius=2.3, y radius=1.0, rotate=-17]
          (1.9, 4.5) node [blue!50!black] {Reali}
          (4.7, 2.5) ellipse [x radius=3.0, y radius=1.9, rotate=17]
          (4.6, 2.3) node [rotate=17, green!50!black] {Infinitesimi}
          (3.7, 3.5) node [blue!80!black] {zero};
  }
}

\newcommand{\microscopioa}{% Microscopio per vedenodere 5,004.
    \disegno{
    \assecontrattini{-1.}{+6.3}{0}{x}
    \draw (0, 0) [below] node{0} (1, 0) [below] node{1}
          (5, 0) [below] node{5};
    \microscopio{(5, 0)}{2.5}{90}{-50}{3.5}{(5.5, 8.5)}{\(\times 10^3\)}
    \segmentocontrattini{-.3}{+6}{5.5}{1}
    \draw (2.8, 5.5) [left] node [rotate=90] {5,000} 
          (3.8, 5.5) [left] node [rotate=90] {5,001}
          (0.8, 5.5) [left] node [rotate=90] {4,998};
    }
}

\newcommand{\microscopiob}{% Microscopio per vedere -3,000002.
    \disegno{
    \assecontrattini{-4.}{+3.3}{0}{x}
    \draw (0, 0) [below] node{0} (1, 0) [below] node{1}
          (-3, 0) [below] node{-3};
    \microscopio{(-3, 0)}{2.5}{70}{-115}{3.5}{(3.5, 9)}{}
    \segmentocontrattini{-3.9}{+2.5}{5.5}{1}
    }
}

\newcommand{\microscopioc}{% Microscopio per NON vedere 2 - 3 \epsilon.
    \disegno{
    \assecontrattini{-1.}{+6.3}{0}{x}
    \draw (0, 0) [below] node{0} (1, 0) [below] node{1}
          (2, 0) [below] node{2};
    \microscopio{(2, 0)}{2.5}{60}{-100}{3.5}{(7.5, 8.5)}{\(\times 10^{27}\)}
    \draw[very thin] (3.8, 6-.1) -- (3.8, 6+0.2);
    \draw [-] (0.4, 6) -- (7.3, 6);
    \draw [-{Stealth[length=2mm, round]}] 
      (3, 5) node [below] {\(2-\epsilon\)} -- (3.8, 6);
    \draw [-{Stealth[length=2mm, round]}] 
      (5, 5) node [below] {\(2\)} -- (3.8, 6);
    }
}

\newcommand{\microscopiod}{% Microscopio per vedere 2 - 3 \epsilon.
    \disegno{
    \assecontrattini{-1.}{+6.3}{0}{x}
    \draw (0, 0) [below] node{0} (1, 0) [below] node{1}
          (2, 0) [below] node{2};
    \microscopio{(2, 0)}{2.5}{60}{-100}{3.5}{(7.5, 8.5)}
                {\(\times \frac{1}{\epsilon}\)}
    \segmentocontrattini{.7}{7}{6}{1}
    \draw (.8, 6) [left] node [rotate=90] {\(2 - 3 \epsilon\)} 
          (2.8, 6) [left] node [rotate=90] {\(2 - \epsilon\)}
          (3.8, 6) [left] node [rotate=90] {2};
    }
}

\newcommand{\telescopioa}{% Telescopio per visualizzare 127034.
    \disegno{
    \assecontrattini{-1.3}{+6.3}{0}{x}
    \draw (0, 0) [below] node{0} (1, 0) [below] node{1};
    \draw (-1, 1) pic [rotate=0, scale=.5] {telescopio=127034};
    \microscopio{(-1, 1)}{2.5}{40}{-130}{3.5}{(7.5, 9.5)}{}
    \segmentocontrattini{0}{+6.2}{6}{1}
    \draw (1., 6) [left] node [rotate=90] {...} 
          (2., 6) [left] node [rotate=90] {127033} 
          (3., 6) [left] node [rotate=90] {127034}
          (4., 6) [left] node [rotate=90] {127035}
          (5., 6) [left] node [rotate=90] {\dots};
    }
}

\newcommand{\telescopiob}{% Telescopio per visualizzare A.
    \disegno{
    \assecontrattini{-1.3}{+6.3}{0}{x}
    \draw (0, 0) [below] node{0} (1, 0) [below] node{1};
    \draw (-1, 1) pic [rotate=0, scale=.5] {telescopio=\(A\)};
    \microscopio{(-1, 1)}{2.5}{40}{-130}{3.5}{(7.5, 9.5)}{}
    \segmentocontrattini{0}{+6.2}{6}{1}
    \draw (0, 6) [left] node [rotate=90] {\dots} 
          (1, 6) [left] node [rotate=90] {\(A - 1\)} 
          (2, 6) [left] node [rotate=90] {\(A ~~\quad\)}
          (3, 6) [left] node [rotate=90] {\(A + 1\)}
          (5, 6) [left] node [rotate=90] {\(A + 3\)}
          (6, 6) [left] node [rotate=90] {\dots};
    }
}

\newcommand{\grandangoloa}{% Grandangolo per vedenodere 300.
    \disegno{
    \assecontrattini{-1.}{+6.3}{0}{x}
    \draw (0, 0) [below] node{0} (1, 0) [below] node{1};
    \grandangolo{(0, 0)}{2.5}{90}{-126}{3.5}{(5.5, 8.0)}{\(\div 100\)}
    \segmentocontrattini{-1.1}{+5.2}{6}{1}
    \draw (1.9, 6) [left] node [rotate=90] {0} 
          (2.9, 6) [left] node [rotate=90] {100}
          (4.9, 6) [left] node [rotate=90] {300};
    }
}

\newcommand{\grandangolob}{% Grandangolo per vedenodere 300.
    \disegno{
    \assecontrattini{-1.}{+6.3}{0}{x}
    \draw (0, 0) [below] node{0} (1, 0) [below] node{1};
    \grandangolo{(0, 0)}{2.5}{90}{-126}{3.5}{(5.5, 8.0)}{\(\div A\)}
    \segmentocontrattini{-1.1}{+5.2}{6}{1}
    \draw (1.9, 6) [left] node [rotate=90] {0} 
          (2.9, 6) [left] node [rotate=90] {\(A\)}
          (-.1, 6) [left] node [rotate=90] {\(-2 A\)};
    }
}

\newcommand{\combinazione}{% Combinazione per: \(1741,998+2\epsilon\).
    \disegnod{4}{
    \assecontrattini{-1.}{+6.3}{0}{x}
    \draw (0, 0) [below] node{0} (1, 0) [below] node{1};
    \draw (-1, 1) pic [rotate=0, scale=.5] {telescopio=1742};
    \microscopio{(-1, 1)}{2.5}{40}{-130}{3.5}{(7.5, 9.5)}{}
    \segmentocontrattini{0}{+6.2}{6}{1}
    \draw (1., 6) [left] node [rotate=90] {\dots}
          (2., 6) [left] node [rotate=90] {1741}
          (3., 6) [left] node [rotate=90] {1742}
          (4., 6) [left] node [rotate=90] {1743}
          (5., 6) [left] node [rotate=90] {\dots};
    \microscopio{(3., 6)}{2.5}{130}{-40}{3.5}{(-5.2, 12.7)}{\(\times 1000\)}
    \segmentocontrattini{-4.5}{+1.9}{10.5}{1}
    \draw (-3.5, 10.5) [left] node [rotate=90] {\dots}
          (-2.5, 10.5) [left] node [rotate=90] {1741,998}
          (-1.5, 10.5) [left] node [rotate=90] {1741,999}
          (-0.5, 10.5) [left] node [rotate=90] {1742}
          (0.5, 10.5) [left] node [rotate=90] {\dots};
    \microscopio{(-2.5, 10.5)}{2.5}{30}{-120}{3.5}{(5, 17)}
                {\(\times \frac{1}{\epsilon}\)}
    \segmentocontrattini{-1.77}{+4.6}{15}{1}
    \draw (-0.7, 15) [right] node [rotate=90] {\dots}
          (0.3, 15) [right] node [rotate=90] {1742,998}
          (1.3, 15) [right] node [rotate=90] {\(1741,998+\epsilon\)}
          (2.3, 15) [right] node [rotate=90] {\(1741,998+2\epsilon\)}
          (3.3, 15) [right] node [rotate=90] {\dots};
    }
}

\newcommand{\tabopposto}{% Tabella degli opposti relativa ai tipi
  \disegno{
    \node [violet!50!black] at (0, 0) {\(x\)};
    \foreach \posx/\posy/\lab in {0/0/~, 2/0/i, 4/0/f, 6/+.04/I, 
                                  8/-.05/I~pos, 10/-.05/I~neg}{
      \node [green!50!black] at (\posx, \posy) {\(\lab\)};
      \draw (\posx+1, +1) -- (\posx+1, -3);
    }
    \draw (-1, -1) -- (+11, -1); % (-1, -3) -- (+11, -3);
    \node [violet!50!black] at (0, -2) {\(-x\)};
    \foreach \posx/\posy/\lab in {0/-2/~, 2/-2/i, 4/-2/f, 6/-1.96/I, 
                                  8/-2.05/I~neg, 10/-2.05/I~pos}{
      \node [blue!50!black] at (\posx, \posy) {\(\lab\)};}
%     \foreach \pos/\lab in {0/~, 2/i, 4/f, 6/I, 8/I~neg, 10/I~pos}{
%       \node [blue!50!black] at (\pos, -2) {\(\lab\)};}
  }
}

\newcommand{\tabadd}{% Tabella dell'addizione relativa ai tipi
  \disegno{
    \node [violet!50!black] at (0, 0) {\(+\)};
    \foreach \pos/\lab in {0/~, 2/i, 4/fni, 6/I}{
      \node [green!50!black] at (\pos, 0) {\(\lab\)};
      \node [green!50!black] at (0, -\pos) {\(\lab\)};
      \draw (\pos+1, +1) -- (\pos+1, -7);
      \draw (-1, -\pos-1) -- (+7, -\pos-1);
    }
    \foreach \ypos/\xl in {-2/{2/i, 4/fni, 6/I},
                           -4/{2/fni, 4/f, 6/I},
                           -6/{2/I, 4/I}}{
      \foreach \xpos/\lab in \xl{
        \node [blue!50!black] at (\xpos, \ypos) {\(\lab\)};
      }
    }
    \node [red!50!black] at (6, -6) {\(?\)};
  }
}

\newcommand{\tabmul}{% Tabella della moltiplicazione relativa ai tipi
  \disegno{
    \node [violet!50!black] at (0, 0) {\(\times\)};
    \foreach \pos/\lab in {0/~, 2/i, 4/fni, 6/I}{
      \node [green!50!black] at (\pos, 0) {\(\lab\)};
      \node [green!50!black] at (0, -\pos) {\(\lab\)};
      \draw (\pos+1, +1) -- (\pos+1, -7);
      \draw (-1, -\pos-1) -- (+7, -\pos-1);
    }
    \foreach \ypos/\xl in {2/{2/i, 4/i},
                           4/{2/i, 4/fni, 6/I},
                           6/{4/I, 6/I}}{
      \foreach \xpos/\lab in \xl{
        \node [blue!50!black] at (\xpos, -\ypos) {\(\lab\)};
      }
    }
    \node [red!50!black] at (6, -2) {\(?\)};
    \node [red!50!black] at (2, -6) {\(?\)};
  }
}
% 
% \newcommand{\tabmul}{% Tabella della moltiplicazione relativa ai tipi
%   \disegno{
%     \node [violet!50!black] at (0, 0) {\(\times\)};
%     \foreach \pos/\lab in {0/~, 2/0, 4/1, 6/inn, 8/fni, 10/I}{
%       \node [green!50!black] at (\pos, 0) {\(\lab\)};
%       \node [green!50!black] at (0, -\pos) {\(\lab\)};
%       \draw (\pos+1, +1) -- (\pos+1, -11);
%       \draw (-1, -\pos-1) -- (+11, -\pos-1);
%     }
%     \foreach \ypos/\xl in {2/{2/0, 4/0, 6/0, 8/0, 10/0}, 
%                            4/{2/0, 4/1, 6/inn, 8/fni, 10/I}, 
%                            6/{2/0, 4/inn, 6/inn, 8/inn},
%                            8/{2/0, 4/fni, 6/inn, 8/fni, 10/I},
%                            10/{2/0, 4/I, 8/I, 10/I}}{
%       \foreach \xpos/\lab in \xl{
%         \node [blue!50!black] at (\xpos, -\ypos) {\(\lab\)};
%       }
%     }
%     \node [red!50!black] at (10, -6) {\(?\)};
%     \node [red!50!black] at (6, -10) {\(?\)};
%   }
% }

\newcommand{\tabrec}{% Tabella dei reciproci relativa ai tipi
  \disegno{
    \node [violet!50!black] at (0, 0) {\(x\)};
    \foreach \pos/\lab in {0/~, 2/inn, 4/fni, 6/I}{
      \node [green!50!black] at (\pos, 0) {\(\lab\)};
      \draw (\pos+1, +1) -- (\pos+1, -3);
    }
    \draw (-1, -1) -- (+7, -1); % (-1, -3) -- (+11, -3);
    \node [violet!50!black] at (0, -2) {\(1/x\)};
    \foreach \pos/\lab in {2/I, 4/fni, 6/inn}{
      \node [blue!50!black] at (\pos, -2) {\(\lab\)};}
  }
}

% \newcommand{\tabdivvuota}{% Tabella della divisione relativa ai tipi
%   \disegno{
%     \node [violet!50!black] at (0, 0) {\(\div\)};
%     \foreach \pos/\lab in {0/~, 2/inn, 4/fni, 6/I}{
%       \node [green!50!black] at (\pos, 0) {\(\lab\)};
%       \node [green!50!black] at (0, -\pos) {\(\lab\)};
%       \draw (\pos+1, +1) -- (\pos+1, -7);
%       \draw (-1, -\pos-1) -- (+7, -\pos-1);
%     }
% %     \foreach \ypos/\xl in {2/{4/inn, 6/inn},
% %                            4/{2/I, 4/fni, 6/inn},
% %                            6/{2/I, 4/I}}{
% %       \foreach \xpos/\lab in \xl{
% %         \node [blue!50!black] at (\xpos, -\ypos) {\(\lab\)};
% %       }
% %     }
% %     \node [red!50!black] at (2, -2) {\(?\)};
% %     \node [red!50!black] at (6, -6) {\(?\)};
%   }
% }

\newcommand{\tabdiv}{% Tabella della divisione relativa ai tipi
  \disegno{
    \node [violet!50!black] at (0, 0) {\(\div\)};
    \foreach \pos/\lab in {0/~, 2/inn, 4/fni, 6/I}{
      \node [green!50!black] at (\pos, 0) {\(\lab\)};
      \node [green!50!black] at (0, -\pos) {\(\lab\)};
      \draw (\pos+1, +1) -- (\pos+1, -7);
      \draw (-1, -\pos-1) -- (+7, -\pos-1);
    }
    \foreach \ypos/\xl in {2/{4/inn, 6/inn},
                           4/{2/I, 4/fni, 6/inn},
                           6/{2/I, 4/I}}{
      \foreach \xpos/\lab in \xl{
        \node [blue!50!black] at (\xpos, -\ypos) {\(\lab\)};
      }
    }
    \node [red!50!black] at (2, -2) {\(?\)};
    \node [red!50!black] at (6, -6) {\(?\)};
  }
}

% \newcommand{\tabdiv}{% Tabella della divisione relativa ai tipi
%   \disegno{
%     \node [violet!50!black] at (0, 0) {\(\div\)};
%     \foreach \pos/\lab in {0/~, 2/0, 4/1, 6/inn, 8/fni, 10/I}{
%       \node [green!50!black] at (\pos, 0) {\(\lab\)};
%       \node [green!50!black] at (0, -\pos) {\(\lab\)};
%       \draw (\pos+1, +1) -- (\pos+1, -11);
%       \draw (-1, -\pos-1) -- (+11, -\pos-1);
%     }
%     \foreach \ypos/\xl in {2/{2/nan, 4/0, 6/0, 8/0, 10/0}, 
%                            4/{2/nan, 4/1, 6/I, 8/fni, 10/inn}, 
%                            6/{2/nan, 4/inn, 8/inn, 10/inn},
%                            8/{2/nan, 4/fni, 6/I, 8/fni, 10/inn},
%                            10/{2/nan, 4/I, 6/I, 8/I}}{
%       \foreach \xpos/\lab in \xl{
%         \node [blue!50!black] at (\xpos, -\ypos) {\(\lab\)};
%       }
%     }
%     \node [red!50!black] at (10, -10) {\(?\)};
%     \node [red!50!black] at (6, -6) {\(?\)};
%   }
% }

\newcommand{\diffarearettangolo}{% rettangolo con gnomone infinitesimo
  \def \xa{0}
  \def \ya{0}
  \def \xb{4}
  \def \yb{4}
  \def \d{2}
  \disegno[4]{
    \begin{scope}[font=\fontsize{8}{8}] %\scriptsize % \small
    \draw (\xa, \ya) node [below] {\(0\)} node [left] {\(0\)} -- 
    (\xb, \ya) node [below] {\(l \approx l + \epsilon\)} -- 
    (\xb, \yb) -- 
    (\xa, \yb) node [left] {\(l \approx l + \epsilon\)} -- cycle;
    \draw (\xb / 2, \yb / 2) node {\(l^2\)};
    \microscopio{(\xb, \yb)}{2}{120}{-50}{2}{(0.9*\xb, 9)}{\(\times n\)}
    \draw (-0.06*\xb, 1.7*\yb) -- (0.5*\xb, 1.7*\yb) -- (0.5*\xb, 1.32*\yb);
    \microscopio{(\xb, \yb)}{2}{30}{240}{2}{(2.3*\xb, 8)}{\(\times \infty\)}
    \fill [fill=cyan!50]
      (1.2*\xb, 1.8*\yb) -- (1.8*\xb, 1.8*\yb) -- (1.8*\xb, 1.19*\yb) --
      (1.6*\xb, 1.19*\yb) -- (1.6*\xb, 1.6*\yb) -- (1.19*\xb, 1.6*\yb) -- 
      cycle;
    \draw 
      (1.2*\xb, 1.8*\yb) -- (1.8*\xb, 1.8*\yb) -- (1.8*\xb, 1.19*\yb) 
      (1.6*\xb, 1.19*\yb) -- (1.6*\xb, 1.6*\yb) -- (1.19*\xb, 1.6*\yb);
    \node at (1.7*\xb, 1.19*\yb) [above] {\(\epsilon\)};
    \end{scope}
    }
}

\newcommand{\diffcirconferenza}{% circonferenza: contrazione infinitesima
  \def \raggio{2.5}
  \def \xp{0.5*\raggio}
  \def \yp{0.866*\raggio}
  \def \mfa{-.577*x +6.7}
  \def \mfb{-.577*x +7.5}
  \def \mfr{1/.577*x -7.5}
%   \def \xb{4}
%   \def \yb{4}
%   \def \d{2}
  \disegno[5]{
    \begin{scope}[font=\fontsize{8}{8}] %\scriptsize % \small
      \draw [thick, brown!50!black] circle (\raggio);
      \draw (0, 0) -- (\xp, \yp);
    \microscopio{(\xp, \yp)}{1.}{30}{200}{2}{(4.7*\xp, 2.4*\yp)}
                {\(\times \infty\)}
    \tkzInit[xmin=+1.3,xmax=+10.3,ymin=-+1.3,ymax=+6.3]
    \tkzFct[domain=+4.7:+8.1, thick, brown!50!black]{\mfa} 
    \tkzFct[domain=+5.03:+8.45, thick, Maroon!50!black]{\mfb} 
    \tkzFct[domain=+5.35:+6.5]{\mfr} 
    \node at (0.45, 0.9) [right, black] {\(r\)};
    \node at (2.4*\xp, 1.1*\yp) [right, black] {\(r\)};
    \node at (2.75*\xp, 1.55*\yp) [right, black] {\(\epsilon\)};
    \node at (-0.2, 1.1*\raggio) [black]
    {\(\mathcal{C}_r \approx \mathcal{C}_{r+\epsilon}\)};
    \node at (1.3*\xp, 1.9*\yp) [black] {\(\mathcal{C}_r\)};
    \node at (1.35*\xp, 2.3*\yp) [black] {\(\mathcal{C}_{r+\epsilon}\)};
    \end{scope}
    }
}

\newcommand{\incrementosferico}{% 
\disegno[10]{
    \shade[ball color=green!70!gray] (0,0) circle (2);
    \begin{scope}
        \clip (0:1.95) arc (0:90:1.95) to[out=225,in=100] (-0.7,-0.7) 
                                       to[out=-10,in=225] (0:1.95);
        \shade[ball color=green!30!gray!60!black,shading angle=180] 
              (0,0) circle (2);
    \end{scope}
}
}

\newcommand{\parabolaetangente}[4]{% 
  % Parabola con tangente
  \def \fcurva{#1}
  \def \ftangente{#2}
  \def \punto{#3}
  \def \poslabel{#4}
  \rcom{-1}{+10}{-1}{11}{gray!50, very thin, step=1}
  \tkzInit[xmin=-1.3,xmax=+10.3,ymin=-1.3,ymax=+11.3]
  \tkzFct[domain=-1.3:+10.3, ultra thick, Maroon!50!black]{\fcurva} 
  \tkzFct[thick, Cyan!50!black] {\ftangente}
  \draw \punto circle (2pt) node [\poslabel] {\(T\)};
}

\newcommand{\ipertangentea}{% 
  % Tangente ad una parabola nel punto (6; 4).
%   \def \punto{(6, 4)} % Perchè non riesco a usarlo?
  \disegno[4]{
  \parabolaetangente{0.5*x**2-4*x+10}{2*x-8}{(6, 4)}{below right}
  \microscopio{(6, 4)}{2}{30}{200}{3}{(12, 9.2)}{\(\times \infty\)}
   \draw [ultra thick, Maroon!50!black] (8.7, 3.7) -- (11.3, 8.9);
   \draw (9.35, 5.0) -- (10.85, 5.0) -- (10.85, 8.0);
   \node at (10.3, 4.6) {\(\scriptstyle dx=\epsilon\)};
   \node at (11.6, 6.3) {\(\scriptstyle dy\)};
    }
}

\newcommand{\ipertangenteb}{% 
  % Tangente ad una parabola nel punto (5; 4).
%   \def \punto{(2, 3)} % Perchè non riesco a usarlo?
  \disegno[4]{
  \parabolaetangente{0.25*x**2-2*x+6}{-x+5}{(2, 3)}{below left}
  \microscopio{(2, 3)}{2}{60}{250}{3}{(9, 8.7)}{\(\times \infty\)}
  \tkzFct[domain=1.6:+5.8, ultra thick, Maroon!50!black]{-x+11}
   \draw (3, 8) -- (5, 8) -- (5, 6);
   \node at (4, 8) [black, above] {\(\scriptstyle \epsilon\)};
   \node at (5, 7) [black, right] {\(\scriptstyle f(2+\epsilon)-f(2)\)};
    }
}

\newcommand{\iperellisse}{% 
  % Tangente ad una ellisse nel punto (3; 2).
  \disegno[4]{
  \rcom{-4}{+6}{-6}{+9}{gray!50, very thin, step=1}
  \tkzInit[xmin=-4.3,xmax=+6.3,ymin=-6.3,ymax=+9.3]
  \draw [ultra thick, Maroon!50!black] (3.4641, 0) 
        arc [start angle=0, end angle=180, x radius=3.4641, y radius=4];
  \draw [Maroon!50!black] (3.4641, 0) 
        arc [start angle=0, end angle=180, x radius=3.4641, y radius=-4];
  \tkzFct[domain=-1:+6.3, thick, Cyan!50!black] {-2*x+8} 
  \draw (3, 2) circle (2pt) node [below left] {\(T\)};
  \microscopio{(3, 2)}{2}{30}{200}{3}{(9, 7.5)}{\(\times \infty\)}
   \draw [ultra thick, Maroon!50!black] (8, 1.1) -- (5.5, 6.3);
   \draw (7.6, 2.0) -- (7.6, 5.0) -- (6.2, 5.0);
   \node at (6.8, 5.5) {\(\scriptstyle \epsilon\)};
   \node at (9.9, 3.5) {\(\scriptstyle f(3+\epsilon)-f(3)\)};
    }
}


\begin{comment}
\newcommand{\microscopioa}{% Microscopio per vedenodere 5,004.
    \disegno{
    \assecontrattini{-1.3}{+6.3}{0}{x}
    \draw (0, 0) [below] node{0} (1, 0) [below] node{1}
          (5, 0) [below] node{5};
    \microscopio{(5, 0)}{2.5}{90}{-50}{3.5}{(5.5, 8.5)}{\(\times 10^3\)}
    \segmentocontrattini{-.3}{+6}{5.5}{1}
    \draw (2.8, 5.5) [left] node [rotate=90] {5,000} 
          (3.8, 5.5) [left] node [rotate=90] {5,001}
          (0.8, 5.5) [left] node [rotate=90] {4,998};
    }
}

\newcommand{\microscopiob}{% Microscopio per vedere -3,000002.
    \disegno{
    \assecontrattini{-4.3}{+3.3}{0}{x}
    \draw (0, 0) [below] node{0} (1, 0) [below] node{1}
          (-3, 0) [below] node{-3};
    \microscopio{(-3, 0)}{2.5}{70}{-115}{3.5}{(2.5, 8.5)}{\(\times \dots\)}
    \segmentocontrattini{-3.9}{+2.5}{5.5}{1}
    }
}

\newcommand{\microscopioc}{% Microscopio per NON vedere 2 - 3 \epsilon.
    \disegno{
    \assecontrattini{-1.3}{+6.3}{0}{x}
    \draw (0, 0) [below] node{0} (1, 0) [below] node{1}
          (2, 0) [below] node{2};
    \microscopio{(2, 0)}{2.5}{60}{-100}{3.5}{(7.5, 8.5)}{\(\times 10^{27}\)}
    \draw[very thin] (3.8, 6-.1) -- (3.8, 6+0.2);
    \draw [-] (0.4, 6) -- (7.3, 6);
    \draw [-{Stealth[length=2mm, round]}] 
      (3, 5) node [below] {\(2-\epsilon\)} -- (3.8, 6);
    \draw [-{Stealth[length=2mm, round]}] 
      (5, 5) node [below] {\(2\)} -- (3.8, 6);
    }
}

\newcommand{\microscopiod}{% Microscopio per vedere 2 - 3 \epsilon.
    \disegno{
    \assecontrattini{-1.3}{+6.3}{0}{x}
    \draw (0, 0) [below] node{0} (1, 0) [below] node{1}
          (2, 0) [below] node{2};
    \microscopio{(2, 0)}{2.5}{60}{-100}{3.5}{(7.5, 8.5)}
                {\(\times \frac{1}{\epsilon}\)}
    \segmentocontrattini{.7}{7}{6}{1}
    \draw (.8, 6) [left] node [rotate=90] {\(2 - 3 \epsilon\)} 
          (2.8, 6) [left] node [rotate=90] {\(2 - \epsilon\)}
          (3.8, 6) [left] node [rotate=90] {2};
    }
}

\newcommand{\telescopioa}{% Telescopio per visualizzare 127034.
    \disegno{
    \assecontrattini{-1.3}{+6.3}{0}{x}
    \draw (0, 0) [below] node{0} (1, 0) [below] node{1};
    \draw (-1, 1) pic [rotate=0, scale=.5] {telescopio=127034};
    \microscopio{(-1, 1)}{2.5}{40}{-130}{3.5}{(7.5, 9.5)}{}
    \segmentocontrattini{0}{+6.2}{6}{1}
    \draw (1., 6) [left] node [rotate=90] {...} 
          (2., 6) [left] node [rotate=90] {127033} 
          (3., 6) [left] node [rotate=90] {127034}
          (4., 6) [left] node [rotate=90] {127035}
          (5., 6) [left] node [rotate=90] {\dots};
    }
}

\newcommand{\telescopiob}{% Telescopio per visualizzare 127034.
    \disegno{
    \assecontrattini{-1.3}{+6.3}{0}{x}
    \draw (0, 0) [below] node{0} (1, 0) [below] node{1};
    \draw (-1, 1) pic [rotate=0, scale=.5] {telescopio=\(A\)};
    \microscopio{(-1, 1)}{2.5}{40}{-130}{3.5}{(7.5, 9.5)}{}
    \segmentocontrattini{0}{+6.2}{6}{1}
    \draw (0, 6) [left] node [rotate=90] {\dots} 
          (1, 6) [left] node [rotate=90] {\(A - 1\)} 
          (2, 6) [left] node [rotate=90] {\(A ~~\quad\)}
          (3, 6) [left] node [rotate=90] {\(A + 1\)}
          (5, 6) [left] node [rotate=90] {\(A + 3\)}
          (6, 6) [left] node [rotate=90] {\dots};
    }
}

\newcommand{\grandangoloa}{% Grandangolo per vedenodere 300.
    \disegno{
    \assecontrattini{-1.3}{+6.3}{0}{x}
    \draw (0, 0) [below] node{0} (1, 0) [below] node{1};
    \grandangolo{(0, 0)}{2.5}{90}{-126}{3.5}{(5.5, 8.5)}{\(\div 100\)}
    \segmentocontrattini{-1.1}{+5.2}{6}{1}
    \draw (1.9, 6) [left] node [rotate=90] {0} 
          (2.9, 6) [left] node [rotate=90] {100}
          (4.9, 6) [left] node [rotate=90] {300};
    }
}

\newcommand{\grandangolob}{% Grandangolo per vedenodere 300.
    \disegno{
    \assecontrattini{-1.3}{+6.3}{0}{x}
    \draw (0, 0) [below] node{0} (1, 0) [below] node{1};
    \grandangolo{(0, 0)}{2.5}{90}{-126}{3.5}{(5.5, 8.5)}{\(\div A\)}
    \segmentocontrattini{-1.1}{+5.2}{6}{1}
    \draw (1.9, 6) [left] node [rotate=90] {0} 
          (2.9, 6) [left] node [rotate=90] {\(A\)}
          (-.1, 6) [left] node [rotate=90] {\(-2 A\)};
    }
}

\newcommand{\combinazione}{% Combinazione per: \(1741,998+2\epsilon\).
    \disegno{
    \assecontrattini{-1.3}{+6.3}{0}{x}
    \draw (0, 0) [below] node{0} (1, 0) [below] node{1};
    \draw (-1, 1) pic [rotate=0, scale=.5] {telescopio=1742};
    \microscopio{(-1, 1)}{2.5}{40}{-130}{3.5}{(7.5, 9.5)}{}
    \segmentocontrattini{0}{+6.2}{6}{1}
    \draw (1., 6) [left] node [rotate=90] {\dots}
          (2., 6) [left] node [rotate=90] {1741}
          (3., 6) [left] node [rotate=90] {1742}
          (4., 6) [left] node [rotate=90] {1743}
          (5., 6) [left] node [rotate=90] {\dots};
    \microscopio{(3., 6)}{2.5}{130}{-40}{3.5}{(-5.2, 12.7)}{\(\times 1000\)}
    \segmentocontrattini{-4.5}{+1.9}{10.5}{1}
    \draw (-3.5, 10.5) [left] node [rotate=90] {\dots}
          (-2.5, 10.5) [left] node [rotate=90] {1741,998}
          (-1.5, 10.5) [left] node [rotate=90] {1741,999}
          (-0.5, 10.5) [left] node [rotate=90] {1742}
          (0.5, 10.5) [left] node [rotate=90] {\dots};
    \microscopio{(-2.5, 10.5)}{2.5}{30}{-120}{3.5}{(5, 17)}
                {\(\times \frac{1}{\epsilon}\)}
    \segmentocontrattini{-1.77}{+4.6}{15}{1}
    \draw (-0.7, 15) [right] node [rotate=90] {\dots}
          (0.3, 15) [right] node [rotate=90] {1742,998}
          (1.3, 15) [right] node [rotate=90] {\(1741,998+\epsilon\)}
          (2.3, 15) [right] node [rotate=90] {\(1741,998+2\epsilon\)}
          (3.3, 15) [right] node [rotate=90] {\dots};
    }
}

\end{comment}
