% (c) 2012 Dimitrios Vrettos - d.vrettos@gmail.com
% (c) 2017 Daniele Zambelli - daniele.zambelli@gmail.com
% 
% Tutti i grafici per il capitolo relativo ai razionali
% 
% 

\newcommand{\divisionea}{% divisione 1523:7
\begin{tikzpicture}[node distance=-25ex]
  \begin{scope}[font=\ttfamily]
    \matrix (divisione) [matrix of nodes] % \matrix non viene riconosciuto 
                                          % all'interno di \newcommand
      {%
&   & |[green]|M & |[green]|C & |[green]|D & |[green]|U & ,
& |[green]|d & |[green]|c & |[green]|m &  &  &   
&  &  &  
&  &  &  &  &  &\\
&   & 1 & 5 & 2 & 3  & ,
& 0 & 0 & 0 & 0 & 0 & 0
& 7 & ~ & ~ 
&  &  &  &  & ~ &\\
& - & 1 & 4 &   &   & 
&  &  &  &  &  & ~ &
|[blue]|2 & |[blue]|1 & |[blue]|7 &
&  &  &  &  &  &\\
&   &   & 1 & 2 &   & 
&  &  &  &  &  &  &
|[green]|C & |[green]|D & |[green]|U & ,
& |[green]|d & |[green]|c & |[green]|m &  &  &\\
&   &   & - & 7 &   
&  &  &  &  &  &
&   &   &\\
&   &   &   & 5 & 3 
&  &  &  &  &  &
&   &   &\\
&   &   & - & 4 & 9 
&  &  &  &  &  &
&   &   &\\
&   &   &   &   & |[red]|4 
&  &  &  &  &  &
& & &\\
      };
  \end{scope}
  \draw(divisione-2-13.north east)--(divisione-3-13.south east);
  \draw(divisione-2-13.south east)--(divisione-2-21.south east);
%   \draw(divisione-3-2.south west)--(divisione-3-4.south east);
%   \draw(divisione-5-4.south west)--(divisione-5-5.south east);
%   \draw(divisione-7-4.south west)--(divisione-7-6.south east);
%   \draw[densely dotted,->] (divisione-2-5) -- (divisione-4-5);
%   \draw[densely dotted,->] (divisione-2-6) -- (divisione-6-6);
%   \node (c) [above=of divisione-1-6.north east] {3.};
\end{tikzpicture}
}

\newcommand{\frazione}{% parti della frazione
  \disegno{
    \node {\(\dfrac{42}{75}\)};
    \begin{scope}[blue!50!black]
    \draw [<-] (1, .5) to [out=0, in=180] (3, 2) node [right] 
      {~~~numeratore};
    \draw [->] (8.5, 2) to [out=0, in=180] (10.5, .5);
    \end{scope}
    \begin{scope}[green!50!black]
    \draw [<-] (1, 0) -- (3, 0) node [right] {linea di frazione};
    \draw [->] (8.5, 0) -- (10.5, 0);
    \end{scope}
    \begin{scope}[red!50!black]
    \draw [<-] (1, -.5) to [out=0, in=180] (3, -2) node [right] 
      {~~denominatore};
    \draw [->] (8.5, -2) to [out=0, in=180] (10.5, -.5);
    \end{scope}
    \node at (11.5, 0) {\(\dfrac{n}{d}\)};
  }
}
