% (c)~2019 Daniele Zambelli daniele.zambelli@gmail.com
% (c)~2019 Andrea Sellaroli

\section{Esercizi}

\subsection{Esercizi dei singoli paragrafi}


% \subsubsection*{\numnameref{subsec:divpol_divisione_euclide}}

\subsubsection{Capitalizzazione semplice}

\begin{esercizio}
Calcola il valore incognito in un regime di capitalizzazione semplice
 \begin{enumeratea}
 \item $C = 2000$ \euro ;\quad $i = 0.02$ ;\quad $t = 10$ anni;\quad $M=\;?$ \hfill [2400 \euro]
 \item $C = 5000$ \euro ;\quad $t = 5$ anni ;\quad $M = 5750$ \euro;\quad $i=\;?$ \hfill [$3 \%$]
 \item $M = 2530$ \euro ;\quad $i = 0.005$ ;\quad $t = 20$ anni;\quad $C=\;?$ \hfill [2300 \euro]

 \item $C = 60000$ \euro ;\quad $t = 6$ anni ;\quad $M = 63600$ \euro;\quad $i=\;?$ \hfill [$1 \%$]

 \item $C = 3000$ \euro ;\quad $i = 0,02$ ;\quad $t = 3$ mesi;\quad $M=\;?$ \hfill [3015 \euro]

 \end{enumeratea}
\end{esercizio}


\begin{esercizio}
Risolvi i seguenti problemi:
 \begin{enumeratea}
  \item Ho investito 15000 \euro\, in regime di capitalizzazione semplice per 3 anni a un tasso d'interesse del 5\% annuo. Quale interesse ho maturato?\hfill [2250 \euro]
  \item Cinque anni fa ho investito un certo capitale in regime di capitalizzazione composta al 3\% annuo. Oggi ho ritirato 12995 \euro. Quanti soldi avevo investito?\hfill [11300 \euro]
  \item Vengono investiti, in regime di capitalizzazione composta, 50000 \euro\; per 10 anni. Per ottenere un interesse di 1000 euro, a quale tasso d'interesse si deve investire tale somma?\hfill [0,02 \%]
 \end{enumeratea}
\end{esercizio}

\subsection{Capitalizzazione composta}
\begin{esercizio}
Calcola il valore incognito in un regime di capitalizzazione composta
 \begin{enumeratea}
 \item $C = 2000$ \euro ;\quad $i = 0.02$ ;\quad $t = 10$ anni;\quad $M=\;?$
 \item $C = 5000$ \euro ;\quad $t = 5$ anni ;\quad $M = 5750$ \euro;\quad $i=\;?$ 
 \item $M = 2530$ \euro ;\quad $i = 0.005$ ;\quad $t = 20$ anni;\quad $C=\;?$ \hfill

 \item $C = 60000$ \euro ;\quad $t = 6$ anni ;\quad $M = 63600$ \euro;\quad $i=\;?$ \hfill 

 \item $C = 3000$ \euro ;\quad $i = 0,02$ ;\quad $t = 3$ mesi;\quad $M=\;?$ \hfill

 \end{enumeratea}
\end{esercizio}


\begin{esercizio}
Risolvi i seguenti problemi:
 \begin{enumeratea}
  \item Ho investito 15000 \euro\, in regime di capitalizzazione composta per 3 anni a un tasso d'interesse del 5\% annuo. Quale interesse ho maturato?\hfill
  \item Cinque anni fa ho investito un certo capitale in regime di capitalizzazione semplice al 3\% annuo. Oggi ho ritirato 12995 \euro. Quanti soldi avevo investito?\hfill [11300 \euro]
  \item Vengono investiti, in regime di capitalizzazione semplice, 50000 \euro\; per 10 anni. Per ottenere un interesse di 1000 euro, a quale tasso d'interesse si deve investire tale somma?\hfill [0,02 \%]
 \end{enumeratea}
\end{esercizio}


\subsection{Trasporto di capitali nel tempo}
\begin{esercizio}
A seguito di alcuni investimenti dovrei riscuotere 7000 € tra 3 mesi, 10 000 € tra 6 mesi e 8500 € tra un anno. Quanto posso riscuotere tra 5 mesi, al tasso annuo del 2,5\%,
\hfill[25 386,75]
\end{esercizio}

\begin{esercizio}
Acquisto oggi un'automobile e posso pagarla così. Oggi verso 9500 €, tra 6 mesi verso 6000 € e tra 12 mesi verso 15700 €. Quanto vale l'automobile se viene applicato un tasso del 4\% annuo.
\end{esercizio}


\subsection{Rendite}
\begin{esercizio}
Una persona vuole costituire una somma che gli consenta, fra 4 anni, di poter cambiare l'auto; per questo versa E 1350 ogni quadrimestre a partire da oggi, ad un tasso annuo nominale convertibile quadrimestralmente del 6\%. Quale somma avrà a disposizione all'epoca stabilita?
\hfill[18468,45]
\end{esercizio}
\begin{esercizio}
Hai versato in banca E 8000 alla fine di ogni anno e per 6 anni, al tasso annuo del 2,5\%. Se decidi di
ritirare il capitale all'atto dell'ultimo versamento, di quale somma potrai disporre?
\hfill[51101,89]
\end{esercizio}
\begin{esercizio}
Fra 5 anni avremo bisogno di una somma 5200 € per restituire un prestito che ci è stato fatto. Decidiamo
allora di depositare ogni anno, alla fine dell'anno, una somma che sia in grado di costituire questo capitale. Qual eÁ il valore di questa somma al tasso annuo del 4\%?
\end{esercizio}
\begin{esercizio}
Una persona ha iniziato a versare 15 anni fa presso una banca 800 € all'anno ed ha proseguito i versamenti fino ad oggi; 4 anni fa, inoltre, ha depositato presso la stessa banca 9800 €. Per tutta la durata
dell'operazione, la banca ha mantenuto costante il tasso d'interesse al 2,5\% annuo. Se oggi questa persona preleva 23000 € qual è il saldo del suo conto?

\end{esercizio}

\begin{esercizio}
Riccardo sa che fra 6 anni avrà bisogno di 10000€ per festeggiare con un viaggio i suoi 25 anni di matrimonio. Calcola la rata annua anticipata al tasso dell'4,6\% annuo che Riccardo deve versare per poter costituire questo capitale.
\end{esercizio}
\begin{esercizio}
Un appartamento viene affittato per un anno ad un canone mensile di 2 000 €. Volendo pagare anticipatamente l'intero ammontare del canone, quanto si deve versare al proprietario se la valutazione viene fatta al 2\%
\end{esercizio}


\subsection{Ammortamenti}
\begin{esercizio}
Determina la rata di un mutuo per un capitale di $150\;000$, durata di $20$ anni e un tasso di interesse annuo del $4\%$
\end{esercizio}
\begin{esercizio}
Determina la rata di un prestito per un capitale di $4\;000$, durata di $5$ anni e un tasso di interesse annuo del $9\%$
\end{esercizio}
\begin{esercizio}
Determina la rata di un prestito per un capitale di $80\;000$, durata di $15$ anni e un tasso di interesse annuo del $1,5\%$
\end{esercizio}
\begin{esercizio}
Per un mutuo per $60\;000$ euro mi hanno proposto due opzioni: nella prima pago una rata di 530 € per 10 anni, nella seconda 380 € per 15 anni. Calcola i due tassi di interesse applicati.
\end{esercizio}


\begin{esercizio}
Stendi un piano di ammortamento per un prestito di $3\;500$, durata di $3$ anni e un tasso di interesse annuo del $2\%$
\end{esercizio}
\begin{esercizio}
Stendi un piano di ammortamento per un prestito di $42\;000$, durata di $6$ anni e un tasso di interesse annuo del $3\%$. Se decido di rimborsare tutto il capitale dopo 3 anni, quanto devo versare alla banca?
\end{esercizio}
