% (c) 2016 Andrea Sellaroli andrea.sellaroli@istruzione.it
% (c) 2015 Daniele Zambelli daniele.zambelli@gmail.com

\chapter{Matematica finanziaria}

La matematica finanziaria è quella parte della matematica applicata che si 
occupa degli scambi di somme di denaro disponibili in tempi diversi.

Se presto una certa somma di denaro (\textbf{capitale}) per un anno, io non 
potrò più usare quel denaro, e mi aspetto di essere remunerato per ciò. 
Specularmente, se uso del denaro che non possiedo, dovrò pagare una certa 
somma per il prestito ricevuto. 

L'\textbf{interesse} è il compenso che spetta a colui che concede in 
prestito un capitale, rinunciando per un certo periodo di tempo al suo 
utilizzo. 

Il \textbf{tasso di interesse} viene espresso come una percentuale per un 
dato periodo di tempo e indica quanta parte della somma prestata (detta 
capitale iniziale) debba essere corrisposta come interesse al termine del 
tempo considerato. Il debitore, infatti, ricevendo una 
somma di denaro, si impegna a pagare una somma superiore a quella ricevuta. 

\begin{exrig}
\begin{esempio}
Deposito in banca 3000 € \,e, un anno dopo, ne ritiro 3150 €. Gli interessi 
sono dati dalla differenza tra il capitale finale e il capitale iniziale e 
quindi sono 150 €. Il tasso di interesse è la percentuale degli interessi 
sul capitale iniziale e quindi
\[ i = \dfrac{150}{3000} = 0,05 \]
ovvero il 5\%. \`{E} importante notare che tale tasso di interesse è 
strettamente riferito al periodo di tempo (in questo caso un anno). 
Pertanto è più corretto affermare che il tasso di interesse è del 5\% 
annuo. 
\end{esempio}

\begin{esempio}
Ho acquistato un nuovo computer di 700 € sfruttando un offerta che mi 
permette di pagarlo tra un anno, pagando un interesse del 3\%.
L'interesse che dovrò pagare sarà di \(700 \cdot 0,03 = 21\) € ed alla fine 
il computer mi costerà 721 €. Per calcolare quantro dovro pagare il 
computer alla fine quindi devo fare \(700+700\cdot0,03\). Un modo più rapido 
per calcolarlo è quello di moltiplicare 700 per 1,03, infatti 
\(700(1+0,03)=700\cdot1,03=721\)
\end{esempio}

\end{exrig}

\vspace{.4cm}

I tassi d'interesse sono caratterizzati dal regime di capitalizzazione 
degli interessi, che può essere semplice o composto. Se la durata del 
prestito è superiore al periodo di tempo per cui l'interesse viene 
conteggiato (ad esempio un tasso di interesse annuo calcolato per 4 anni), 
si parla di tasso di interesse composto, perché vengono conteggiati nel 
calcolo dell'interesse finale anche gli interessi parziali già maturati per 
ogni periodo.

\section{Capitalizzazione semplice}
L'interesse viene detto semplice quando è proporzionale al capitale e al 
tempo. 
Ovvero gli interessi, maturati da un dato capitale nel periodo di 
tempo considerato, non vengono aggiunti al capitale che li ha prodotti 
(capitalizzazione) e, quindi, non maturano a loro volta interessi. 
Indichiamo:

\begin{itemize}
\item \textbf{C} il capitale iniziale;
\item \textbf{i} il tasso di interesse periodale (in genere tasso unitario 
annuo, ma può essere mensile, trimestrale...);
\item \textbf{t} durata temporale dell'operazione, espressa in numero di 
periodi (in genere anni);
\item \textbf{M} il capitale finale, detto anche montante, pari alla somma 
di capitale iniziale più gli interessi maturati. 
\end{itemize}

All'istante iniziale (\(t=0\)) possiamo dire che il montante coincide col 
capitale 
\[ M_{0}=C \]
Dopo un periodo di tempo (\(t=1\)), il montante sarà dato dal capitale più 
l'interesse. 
\[ M_{1}=M_{0} +iC = C + iC \]
Analogamente dopo 2, 3 e 4 periodi di tempo il capitale sarà dato da 
\[ M_{2}=M_{1} +iC = (C + iC) +iC = C +2iC\]
\[ M_{3}=M_{2} +iC = (C + 2iC) +iC = C +3iC\]
\[ M_{4}=M_{3} +iC = (C + 3iC) +iC = C +4iC\]
In generale possiamo calcolare il montante per il periodo successivo con la 
formula:
\[ M_{t}=M_{t-1}+iC=C(1+t\cdot i) \]
\begin{definizione}[Capitalizzazione semplice]
\[ M_{t}=C(1+t\cdot i) \]
\end{definizione}

\begin{exrig}
\begin{esempio}
Una persona deposita 5000 € in banca al tasso di interesse annuo del 5\%. 
Dopo 4 anni preleva la somma e la reinveste al tasso annuo del 6\%. Dopo 
altri 5 anni e 9 mesi di che cifra dispone?
Il capitale iniziale è di 5000 € ed applicando la formula di 
capitalizzazione semplice otteniamo che il montante dopo 4 anni è

\[M_4 = 5000(1+4\cdot0,05) = 6000\]
Quando questa cifra viene prelevata e reinvestita possiamo considerarla 
come il nuovo capitale iniziale. La durata temporale, in questo caso, è 
espressa in anni e in mesi. Convertendola in anni otteniamo 
\(t = 5 +\frac{9}{12} = 5,75\) e quindi

\[M_{5,75} = 6000(1+5,75\cdot0,06) = 8070\]
\end{esempio}
\end{exrig}

\section{Capitalizzazione composta}

L'interesse viene detto composto quando, invece di essere pagato o 
riscosso, è aggiunto al capitale iniziale che lo ha prodotto. Questo 
comporta che, alla maturazione degli interessi, il montante verrà 
riutilizzato come capitale iniziale per il periodo successivo. Quindi anche 
l'interesse stesso produce a sua volta interesse.

In questo caso quindi gli interessi si sommano al capitale iniziale che li 
ha prodotti al termine di ogni periodo.
Analogamente a prima il montante iniziale coincide col capitale
\[ M_{0}=C \]
Dopo un periodo il montante sarà dato dal capitale più l'interesse
\[M_{1}=C+iC=C(1+i)\]
Calcoliamo adesso l'interesse su tutto il montante \(M_1\) e troviamo
\[M_{2}=M_1+iM_1=M_1(1+i)=C(1+i)^2\]
\[M_{3}=M_2+iM_2=M_2(1+i)=C(1+i)^3\]
ed in generale risulta
\[M_{t}=M_{t-1}(1+i) = C(1+i)^{t-1}(1+i)=C(1+i)^{t}\]

\begin{definizione}[Capitalizzazione composta]
\[ M_{t}=C(1+i)^t \]
\end{definizione}

In generale il periodo considerato è l'anno. Spesso però vengono 
considerati anche gli interessi che maturano \(t\) volte durante l'anno, ma 
sempre in periodi definiti. 
In genere viene definito un tasso annuo nominale \(i\) al quale corrisponde 
un tasso convertibile \(i_c\) dato da

\(\ i_c = \frac{i}{t}\).

Per il calcolo del montante si applica la stessa formula impiegata per 
l'interesse composto
\(\ M_n = C (1+i_c)^{nt} = C \left(1+\frac{i}{t}\right)^{nt}\).

dove \(i_c\) è l'interesse convertibile e \(nt\) indica il numero di volte in 
cui l'interesse convertibile matura nell'intero periodo.

\begin{exrig}
\begin{esempio}
Un ``amico'', 20 anni fa, mi ha prestato 500 € al tasso di interesse del 
9\% in regime di capitalizzazione composta. Per capire quanto gli devo 
restituire oggi posso usare la formula per la capitalizzazione composta.
\[M_{20} = 500(1+0,09)^{20} = 2802,21\] 
Come si nota subito in 20 anni la cifra iniziale è più che quintuplicata. 
In effetti il tasso del 9 \% è un tasso da usura. 
\end{esempio}
\begin{esempio}
Un capitale di 10000 € è stato investito per un periodi di 30 mesi al tasso 
trimestrale convertibile dell'1\% . Considerando che ci sono 4 trimestri in 
un anno, il tasso annuo nominale risulta quindi
\[i=i_c\cdot t=1\cdot 4=4\%\]
Dopo 30 mesi, ovvero 10 trimestri, il montante risulta
\[M=10000(1+0,01)^{10}=11046,22\]
\end{esempio}
\end{exrig}

\subsection{Scindibilità finanziaria}
Si dice che un regime finanziario è \textbf{scindibile} se il montante di 
un'operazione finanziaria dipende solo dalla durata e non da eventuali 
operazioni di disinvestimento ed investimento intermedie (ossia da 
operazioni di capitalizzazione intermedie).
La capitalizzazione semplice non è scindibile nel tempo, infatti se investo 
3000 € per 5 anni con un tasso di interesse \(i=0,08\) ottengo
\[3000(1+5\cdot0,08)=4200\]
mentre se li disinvesto dopo 2 anni e li reinvesto immediatamente per altri 
3 ottengo
\[3000(1+2\cdot0,08)=3480\] 
e poi
\[3480(1+3\cdot0,08)=4315,2\] 
che sono cifre diverse.
Lo stesso problema, con la capitalizzazione composta, risulta
\[3000(1+cdot0,08)^5=4408\] 
se aspetto 5 anni oppure
\[3000(1+0,08)^2=3499\] 
e poi
\[3499(1+0,08)^3=4408\] 
se invece disinvesto dopo 2 anni e reinvesto per 3.
Come si vede in questo caso le due cifre sono uguali. Questo è un fatto più 
generale (dipende dalle proprietà delle potenze) e quindi la 
capitalizzazione composta è scindibile. Per questa importante proprietà, 
nel seguito, utilizzeremo sempre la capitalizzazione composta.

\begin{comment}
\section{Capitalizzazione composta continua}

In questo caso gli interessi si sommano al capitale che li ha prodotti ad 
ogni 
istante. Il tasso d'interesse composto a capitalizzazione continua ha 
applicazioni soprattutto teoriche, nella matematica finanziaria; sebbene 
sia 
rilevante nelle applicazioni relative alle più semplici operazioni 
finanziarie, 
è ad esempio ampiamente utilizzato nelle formule di valutazione di 
operazioni 
finanziarie complesse, come nella valutazione delle opzioni.

L'interesse in capitalizzazione continua può essere giustificato come 
segue. Si 
consideri un tasso annuale \(i\), e si supponga di suddividere l'anno in \(t\) 
periodi, al termine di ciascuno dei quali viene corrisposta una frazione 
dell'interesse relativo all'intero anno pari a \(\frac{i}{t}\), che viene 
immediatamente reinvestita. A partire da un capitale iniziale \(C\), il 
montante 
al termine di \(n\) anni sarà allora
\\[4pt]
\(\ M_n=C\left(1+\frac{i}{t}\right)^{nt}\)
\\[4pt]
Passando al limite per \(t\) che tende a infinito, si ha il caso in cui un 
flusso 
continuo di pagamenti viene reinvestito in maniera continua; il montante 
sarà 
dato da
\\[4pt]
\[\ M_n = \lim_{t\to\infty} C\left(1+\frac{i}{t}\right)^{nt} = Ce^{in}\],
\\[4pt]
ricorrendo al limite notevole che definisce il numero di Nepero \(e\).

\subsection{Tassi equivalenti}

Chiamiamo \emph{tassi equivalenti} due tassi d'interesse che, applicati 
allo stesso capitale,
producono lo stesso montante (ovviamente in tempi diversi).
Per determinare la relazione tra due tassi unitari ad interesse composto 
\(i_{c1}\) e \(i_{c2}\) è sufficiente uguagliare i montanti che sono prodotti 
da 
periodi di tempo \(t_1\) e \(t_2\) differenti
\[M = C(1+i_{c1})^{t_1} = C(1+i_{c2})^{t_2}\].
Da questa si ottengono le relazioni:
\[i_{c1} = (1+i_{c2})^\frac{t_2}{t_1}-1 \hspace{1cm} i_{c2} = 
(1+i_{c1})^\frac{t_1}{t_2}-1\]
\end{comment}

\section{Trasporto di capitali nel tempo}
Da quanto detto sulla capitalizzazione semplice e composta dovrebbe 
risultare abbastanza chiaro che non è possibile sommare, sottrarre o 
confrontare valori di denaro differiti nel tempo. Per prima cosa infatti è 
necessario riferirli allo stesso momento temporale. Abbiamo già visto che 
un capitale, trasportato in avanti nel tempo, diventa un montante (somma 
del capitale iniziale e degli interessi) e tale processo si chiama 
capitalizzazione. All'inverso, un capitale portato indietro nel tempo si 
chiama valore attuale (o valore scontato) e il processo si definisce 
attualizzazione (o sconto). 


\subsection{Capitalizzazione}
Per trasportare avanti nel tempo un capitale in un regime di 
capitalizzazione composta la formula è già stata vista e risulta 
\(M=C(1+i)^n\) dove \(n\) 
indica il numero di periodo (nella figura qui sotto, ad esempio, \(n=5\)).

\begin{figure}[htp]
\centering
\includegraphics[scale=.60]{img/capitalizzazione.png}
\end{figure}

\subsection{Attualizzazione}
Trasportare un capitale indietro nel tempo di \(n\) periodi, prevede di 
calcolare qual è la cifra di partenza, detta \(valore attuale\) (o sconto) 
per cui, dopo \(n\) intervalli, si è arrivati a quel capitale. Il valore 
attuale, quindi, dopo \(n\) invervalli, viene attualizzato e diventa il 
capitale di partenza, in formule \(C=V(1+i)^n\).
\begin{figure}[htp]
\centering
\includegraphics[scale=.60]{img/attualizzazione.png}
\end{figure}

Invertendo la formula risulta \(V=\dfrac{C}{(1+i)^n}\) e ricordando le 
proprietà delle potenze\footnote{ovvero che \(\dfrac{1}{x^n}=x^{-n}\)}
risulta

\begin{definizione}[Attualizzazione]
\[ V=C(1+i)^{-n}\]
\end{definizione}
\begin{exrig}
\begin{esempio}
Voglio sommare 2000 € che ho ricevuto 2 anni fa, 1000 € euro ricevuti oggi 
e 1600 € che riceverò tra 3 anni, utilizzando un tasso di interesse 
\(i=0,05\).
Trasporto tutti i capitali alla data attuale.
\begin{figure}[htp]
\centering
\includegraphics[scale=.60]{img/esempio.png}
\end{figure}
\begin{itemize}
\item I 2000 € vanno capitalizzati e quindi risulta \(M=2000(1+0,05)^2=2205\)
\item I 1000 € sono già alla data attuale
\item I 1600 € vanno attualizzati e risulta \(M=1600(1+0,05)^{-3}=1382\)
\end{itemize}
La somma totale alla data odierna è quindi \(2205+1000+1382=4587\)
\end{esempio}
\end{exrig}

È interessante notare che per frazioni dell'anno si possono usare 
esponenti frazionari, ad esempio se devo trasportare in avanti di 3 mesi 
(che sono \(\frac{3}{12}\) di anno) un capitale al tasso di interesse annuo 
del 5\% posso fare 
\(M=C(1+0,05)^{\frac{3}{12}}\)

\section{Intermezzo matematico: la serie geometrica}
Per il seguito è necessario imparare a sommare le prime \(n\) potenze di un 
numero fissato \(q\), detta \(ragione\). In alcuni casi posso farlo 
direttamente, ad esempio la somma \(2^0+2^2+2^2+2^3+2^4=31\) ma se devo fare
\(3^0+3^1+3^2+\dots+3^{42}\) devo sviluppare un altro metodo.
Indichiamo con \(S\) la somma delle prime \(n\) potenze di un numero generico 
\(q\)
\[ S= 1+q+q^2+q^3+\dots + q^{n-1}+q^n\]
moltiplichiamo da entrambi i lati per \(q-1\) 
\[ S(q-1)= (1+q+q^2+q^3+\dots + q^{n-1}+q^n)(q-1)\]
e quindi sviluppando il prodotto tra polinomio e binomio risulta
\[ S(q-1)= q+q^2+q^3+q^4+\dots + q^{n}+q^{n+1}-1-q-q^2-q^3-\dots - 
q^{n-1}-q^n\]
la maggior parte dei termini si semplifica compresi i termini sottintesi 
dai puntini e risulta
\[ S(q-1)= \cancel{q}+\cancel{q^2}+\cancel{q^3}+\cancel{q^4}+\dots + 
\cancel{q^n}+q^{n+1}-1-\cancel{q}-\cancel{q^2}-\cancel{q^3}-\dots - 
\cancel{q^{n-1}}-\cancel{q^n}\]
quindi
\[S(q-1)=q^{n+1}-1 \]
e semplificando risulta

\begin{definizione}[Serie geometrica]
\[ 1+q+q^2+q^3+\dots + q^{n-1}+q^n=\dfrac{q^{n+1}-1}{q-1} \]
\end{definizione}

\begin{exrig}
\begin{esempio} Si narra che l’inventore del gioco degli scacchi chiedesse 
di essere compensato con chicchi di grano: un chicco sulla prima casella, 
due sulla seconda, quattro sulla terza e così via, sempre raddoppiando il 
numero dei chicchi, fino alla sessantaquattresima casella. Assumendo che 
1000 chicchi pesino circa 38 g, calcola il peso in tonnellate della 
quantità di grano pretesa dall’inventore.

Come prima cosa sommiamo le prime \(64\) potenze di due; ricordandosi di 
contare anche \(2^0=1\).

\[ 1+2+2^2+2^3+\dots + 2^63=\dfrac{2^{63+1}-1}{2-1}=2^{64}-1=1,84 
\cdot10^{19}\]

Se 1000 chicchi di grano pesano 38 g, allora il peso totale dei chicchi 
sulla scacchiera vale  
\(7.00976\cdot10^{17} g\)
ossia 
\(7.00976\cdot10^{14} kg = 7.00976\cdot10^{11}\) tonnellate. 

\end{esempio}
\end{exrig}

\section{Rendite}
Una rendita finanziaria è una successione di importi, chiamate rate, da 
riscuotere (o da pagare) in epoche differenti, chiamate scadenze, ad 
intervalli di tempo determinati. Per semplicità considereremo nel seguito 
intervalli di tempo di durata costante e rate costanti (ovvero tutte 
uguali).

\begin{exrig}
\begin{esempio}
\label{nonna}
Una nonna versa alla nipote 500 € per ogni compleanno dalla prima candelina 
fino al diciottesimo. Alla maggiore età la nipote ritirerà un montante dato 
dalla somma delle 18 rate (9000 €) più gli interessi, che saranno dati 
dalla capitalizzazione di 17 periodi per i primi 500 euro, di 16 periodi per 
i 500 euro versati al secondo compleanno e così via fino ai 500 euro versati 
al diciottesimo su cui non saranno maturati interessi.
\end{esempio}
\end{exrig}

Esistono diversi tipi di rendite, che possono essere caratterizzate da 
diversi fattori

\begin{itemize}
\item La \textbf{rata}, ciò l'importo da pagare ad ogni periodo, che può 
essere costante o variabile nel tempo. Per semplicità, nel seguito, ci 
occuperemo solo di \textit{rate costanti}. Indicheremo la rata con \(R\)
\item Il \textbf{numero} di rate da pagare che considereremo sempre finito 
e indicheremo con \(n\). 
Esistono però anche rendite (dette perpetue o vitalizie) in cui \(n\) non è 
finito o non è determinato fin dall'inizio.
\item Il \textbf{periodo} ovvero l'intervallo di tempo che c'è tra una rata 
e l'altra. Generalmente tale intervallo sarà annuale, ma potrebbe essere 
anche mensile, trimestrale o altro. Il tasso di interesse \(i\) deve essere 
riferito al periodo.
\item La \textbf{decorrenza} che indica da quando può essere pagata (o 
riscossa) la prima rata. Noi ci occuperemo solo di rendite 
\textit{immediate}, ovvero quelle in cui si inizia a pagare nel primo 
periodo, ma esistono anche le rendite differite, in cui si paga o riscuote 
la prima rata dopo un certo numero di periodi (un esempio è la pensione).
\item La \textbf{scadenza} della rata ovvero se si paga all'inizio del 
periodo (\textit{rendita anticipata}) o alla fine del periodo 
(\textit{rendita posticipata}). Ad esempio una rendita annuale è anticipata 
se pago le rate al 1 gennaio e posticipata se le pago il 31 dicembre.
\end{itemize}


Per capire il meccanismo matematico che c'è dietro alle rendite 
analizzeremo soltanto alcune rendite particolari ovvero le rendite immediate 
con il numero delle rate totali finito e le scadenze separate da intervalli 
di tempo uguale. Studieremo queste rendite nel caso in cui siano anticipate 
o posticipate. 

\subsection{Rendite immediate posticipate a rate costanti}
Nelle rendite immediate posticipate a rate costanti abbiamo 
\begin{itemize}
\item La \textbf{rata} \(R\) che è costante e pagata alla fine di ogni periodo
\item Il \textbf{numero} di periodi \(n\) fissato (e la durata dei periodi 
costante)
\item Il \textbf{tasso di interesse} \(i\) riferito alla durata del periodo
\end{itemize}
Vediamo adesso come possiamo calcolare il \textbf{montante} \(M\). Si tratta 
di capitalizzare (ovvero trasportare avanti nel tempo) ogni singola rata.
\begin{figure}[htp]
\centering
\includegraphics[scale=1]{img/posticipata.png}

\end{figure}

Nella figura abbiamo \(n=5\) anche se la prima rata viene pagata alla fine 
del primo periodo; dovrà quindi essere trasportata in avanti di 4 periodi, 
in formule \(M=R(1+i)^4\). Per lo stesso motivo, la seconda rata dovrà essere 
portata in avanti di 3 periodi ovvero \(M=R(1+i)^3\). La penultima rata deve 
essere portata avanti solo di un periodo, quindi \(M=R(1+i)^1\) mentre 
l'ultima rata è già alla data in cui viene calcolato il montante, quindi 
\(R\) 
Il montante sarà quindi dato dalla somma di tutte le rate capitalizzate 
ovvero

\[M=R+R(1+i)+R(1+i)^2+R(1+i)^3+R(1+i)^4\]

nel caso di un numero di rate generico \(n\) tale formula diventa

\[M=R+R(1+i)+R(1+i)^2+\dots+R(1+i)^{n-1}\]
e, raccogliendo \(R\), risulta

\[M=R\big(1+(1+i)+(1+i)^2+\dots+(1+i)^{n-1}\big)\]
che è una serie geometria di ragione \((1+i)\) e quindi

\[M=R\dfrac{(1+i)^{(n-1+1)}-1}{1+i-1} \]
che semplificando risulta

\begin{definizione}[Montante di rendita posticipata]
\[ M=R\dfrac{(1+i)^n-1}{i} \]
\end{definizione}

Per trovare il valore attuale di questo montante basta attualizzarlo di \(n\) 
periodi ovvero
\[V=M(1+i)^{-n}\]
che sostituendo con il valore del montante per una rendita posticipata 
risulta

 \[V=R\dfrac{(1+i)^n-1}{i}(1+i)^{-n}\]
e applicando le proprietà delle potenze risulta


\begin{definizione}[Valore attuale di rendita posticipata]
\[ V=R\dfrac{1-(1+i)^{-n}}{i} \]
\end{definizione}

\begin{exrig}
\begin{esempio}
Rivediamo l'esempio \ref{nonna} (nonna e nipote). Abbiamo \(R=500\) ed \(n=18\).
Supponiamo un tasso d'interesse \(i=0,02\), possiamo calcolare il montante

\[ M=500\dfrac{(1+0,02)^{18}-1}{0,02}=10\;706,16 \]

ed il valore attuale

\[ V=500\dfrac{1-(1+0,02)^{-18}}{0,02}=7\;496,02 \]
\end{esempio}
\end{exrig}

\subsection{Rendite immediate anticipate a rate costanti}
Nel caso della rendita anticipata la rata viene pagata all'inizio del 
periodo. Senza dover ripetere il procedimento della serie geometrica posso 
immaginare una rendita anticipata come una rendita posticipata in cui ogni 
rata non vale \(R\) ma \(R(1+i)\). Infatti la differenza tra le due sta nel 
fatto che ogni rata viene trasportata avanti di un periodo, maturando 
un'interesse di \(i\).
\begin{figure}[htp]
\centering
\includegraphics[scale=1]{img/anticipata.png}

\end{figure}
Montante e valore attuale di una rendita anticipata risultano quindi:
\begin{definizione}[Montante di rendita anticipata]
\[ M=R(1+i)\dfrac{(1+i)^n-1}{i} \]
\end{definizione}


\begin{definizione}[Valore attuale di rendita anticipata]
\[ V=R(1+i)\dfrac{1-(1+i)^{-n}}{i} \]
\end{definizione}

\section{Ammortamenti}
In finanza, per ammortamento si intende il processo con il quale il 
debitore restituisce il capitale preso a prestito e gli interessi maturati 
sul debito.

Il capitale mutuato S viene diviso in quote di capitale a cui vengono 
aggiunti gli interessi, la somma dà la rata di ammortamento da 
corrispondere in epoche solitamente equi-intervallate.

Le rate di ammortamento comprendono una quota di capitale e una di 
interessi: 
\[R=Q+I\]

Esistono diversi tipi di ammortamenti (a capitale costante, a rate 
costanti, a due tassi); per il seguito ci occuperemo di quello a 
\textbf{rate costanti} detto ammortamento alla francese.

L'ammortamento francese prevede che le rate siano posticipate e che la 
somma ricevuta dal debitore all'inizio sia il valore attuale di una rendita 
a rate costanti. Ciascuna rata è comprensiva di parte del capitale (quota 
capitale) ed i relativi interessi (quota interessi) calcolati sul capitale 
residuo non ancora restituito (debito residuo). 

Per calcolare il valore della rata, noto capitale iniziale, numero di 
periodi e tasso d'interesse, basta applicare la formula inversa al valore 
attuale di una rendita posticipata 

\begin{definizione}[Rata]
\[ R=C\dfrac{i}{1-(1+i)^{-n}}\]
\end{definizione}

\begin{exrig}
\begin{esempio}
Ho chiesto un prestito di \(20\;000\) euro che voglio restituire in \(10\) 
anni. 
Se il tasso di interesse annuo è del \(3,5\%\), quanto dovrò pagare ad ogni 
rata?

Utilizzando la formula risulta:
\[ R=20\;000\dfrac{0,035}{1-(1,035)^{-10}}=2\;404,83\]
Quindi devo pagare 10 rate annue da circa 2405 €.
\end{esempio}
\end{exrig}

Come detto in precedenza ogni rata è composta da una quota di capitale e 
una quota di interessi. Visto che il debito residuo diminuisce ad ogni 
periodo la quota di interessi diminuirà sempre, e di conseguenza, essendo le 
rate costanti, aumenterà la quota di capitale.

\subsection{Piano di ammortamento}
Vediamo adesso come stendere un piano di ammortamento (alla francese), 
ovvero un programma di estinzione del debito. 
Per prima cosa si calcola il valore della rata, che sarà costante.
Poi si costruisce una tabella con una riga per ogni periodo (più una riga 
per il momento iniziale o periodo \(0\)) e 6 colonne che saranno
\begin{itemize}
\item il periodo di riferimento
\item \textbf{R}: il valore della rata (che nel nostro caso sarà costante)
\item \textbf{Q}: la quota di capitale della rata
\item \textbf{I}: la quota di interesse della rata
\item \textbf{D}: il debito residuo
\item \textbf{E}: il debito estinto
\end{itemize}
Per ogni periodo \(k\) la somma tra capitale e interesse deve corrispondere 
con il valore della rata, in formule:
\[R=Q_k+I_k\]
Inoltre la somma tra debito estinto e residuo in ogni periodo \(k\) dovrà 
equivalere al capitale iniziale.
\[C=D_k+E_k\]
Il principio con cui si costruisce il piano di ammortamento alla francese è 
che la quota di interesse per ogni periodo è calcolata sul debito residuo 
al periodo precedente. In formula
\[I_{k+1}=i\cdot D_k\]
Vediamo con un esempio come fare operativamente

\begin{exrig}
\begin{esempio}
Stendere un piano di ammortamento per un prestito di \(5\;000\) euro da 
restituire in \(4\) anni ad un tasso di interesse annuo del \(10\%\).

Per prima cosa calcoliamo il valore della rata
\[ R=5\;000\dfrac{0,1}{1-(1,1)^{-4}}=1\;577,35\]

Iniziamo a stendere la tabella con quello che conosciamo al momento 
iniziale.
I valori delle rate sono sempre costantemente uguali a \(1\;577,35\) (tranne 
che nel periodo 0 perché è posticipata), il debito residuo coincide col 
capitale iniziale ed il debito estinto è zero


\begin{tabular}{c|c|c|c|c|c}
\hline
    periodo & rata & quota capitale & quota interesse & debito residuo & 
debito estinto\\ \hline
	 & R & Q & I & D & E\\ \hline
	0 &  &  &  & \(5000\) & \(0\)\\
	1 & \(1\;577,35\) &  &  &  & \\
	2 & \(1\;577,35\) &  &  &  & \\
	3 & \(1\;577,35\) &  &  &  & \\
	4 & \(1\;577,35\) &  &  &  & \\
	\hline
\end{tabular}
La prima rata è formata da una quota di capitale ed una di interesse. La 
quota di interesse è calcolata sul debito residuo quindi \(I_1=i\cdot 
D_0=0,1\cdot 5000=500\). La quota capitale si calcola ricordando che la 
somma tra quota capitale e interesse è la rata: 
\(Q_1=R-I_1=1577,35-500=1077,35\). A questo punto ho rimborsato una parte del 
capitale, e quindi il debito residuo sarà dato dal debito al periodo 
precedente meno la quota di capitale pagata ovvero 
\(D_1=D_0-Q_1=5000-1077,35=3922,65\). Per calcolare la parte di debito estinto 
basta ricordare che \(C=D_k+E_k\) e quindi \(E_1=C-D_1=5000-3922,65=1577,35\).
Inserendo i dati in tabella risulta:

\begin{tabular}{c|c|c|c|c|c}
\hline
    periodo & rata & quota capitale & quota interesse & debito residuo & 
debito estinto\\ \hline
	 & R & Q & I & D & E\\ \hline
	0 &  &  &  & \(5000\) & \(0\)\\
	1 & \(1\;577,35\) & \(1\;077,35\) & \(500\) 
	  & \(3\;922,65\) & \(1\;577,35\)\\
	2 & \(1\;577,35\) &  &  &  & \\
	3 & \(1\;577,35\) &  &  &  & \\
	4 & \(1\;577,35\) &  &  &  & \\
	\hline
\end{tabular}
Adesso per calcolare la quota di interesse nella seconda rata si utilizzerà 
il debito residuo del primo periodo ovvero \(3\;922,65\) e si ripeteranno gli 
stessi calcoli. Alla fine il piano di ammortamento completo deve risultare:


\begin{tabular}{c|c|c|c|c|c}
\hline
    periodo & rata & quota capitale & quota interesse & debito residuo & 
debito estinto\\ \hline
	 & R & Q & I & D & E\\ \hline
	0 &  &  &  & \(5000\) & \(0\)\\
	1 & \(1\;577,35\) & \(1\;077,35\) & \(500\) & \(3\;922,65\) 
	  & \(1\;577,35\)\\
	2 & \(1\;577,35\) & \(1\;185,09\)& \(392,26\) & \(2\;737,56\) 
	  & \(2\;262,44\)\\
	3 & \(1\;577,35\) & \(1\;303,60\) & \(273,76\) & \(1\;433,96\) 
	  & \(3\;566,04\)\\
	4 & \(1\;577,35\) & \(1\;433,96\) & \(143,40\) & \(0\) 
	  & \(5\;000,00\)\\
	\hline
\end{tabular}
Come si può vedere la parte di interessi scende ad ogni rata ed aumenta la 
quota di capitale. Alla fine il debito residuo deve essere 0 e il debito 
estinto coincidere col capitale iniziale. 
\end{esempio}
\end{exrig}

\section{Esercizi}

\subsubsection{Capitalizzazione semplice}

\begin{esercizio}
Calcola il valore incognito in un regime di capitalizzazione semplice
 \begin{enumeratea}
 \item \(C = 2000\) \euro ;\quad \(i = 0.02\);\quad \(t = 10\) anni;
         \quad \(M=\;?\) \hfill [2400 \euro]
 \item \(C = 5000\) \euro ;\quad \(t = 5\) anni;\quad \(M = 5750\) \euro;
         \quad \(i=\;?\) \hfill [\(3 \%\)]
 \item \(M = 2530\) \euro ;\quad \(i = 0.005\);\quad \(t = 20\) anni;
         \quad \(C=\;?\) \hfill [2300 \euro]
 \item \(C = 60000\) \euro ;\quad \(t = 6\) anni;
         \quad \(M = 63600\) \euro;\quad 
\(i=\;?\) \hfill [\(1 \%\)]
 \item \(C = 3000\) \euro ;\quad \(i = 0,02\) ;\quad \(t = 3\) mesi;
         \quad \(M=\;?\) \hfill [3015 \euro]
 \end{enumeratea}
\end{esercizio}


\begin{esercizio}
Risolvi i seguenti problemi:
 \begin{enumeratea}
  \item Ho investito 15000 \euro\, in regime di capitalizzazione semplice 
per 3 anni a un tasso d'interesse del 5\% annuo. Quale interesse ho 
maturato?\hfill [2250 \euro]
  \item Cinque anni fa ho investito un certo capitale in regime di 
capitalizzazione composta al 3\% annuo. Oggi ho ritirato 12995 \euro. 
Quanti soldi avevo investito?\hfill [11300 \euro]
  \item Vengono investiti, in regime di capitalizzazione composta, 50000 
\euro\; per 10 anni. Per ottenere un interesse di 1000 euro, a quale tasso 
d'interesse si deve investire tale somma?\hfill [0,02 \%]
 \end{enumeratea}
\end{esercizio}

\subsection{Capitalizzazione composta}
\begin{esercizio}
Calcola il valore incognito in un regime di capitalizzazione composta
 \begin{enumeratea}
 \item \(C = 2000\) \euro ;\quad \(i = 0.02\) ;\quad \(t = 10\) anni;
       \quad \(M=\;?\)
 \item \(C = 5000\) \euro ;\quad \(t = 5\) anni ;\quad \(M = 5750\) \euro;
       \quad \(i=\;?\) 
 \item \(M = 2530\) \euro ;\quad \(i = 0.005\) ;\quad \(t = 20\) anni;
       \quad \(C=\;?\) \hfill
 \item \(C = 60000\) \euro ;\quad \(t = 6\) anni ;\quad \(M = 63600\) \euro;
       \quad \(i=\;?\) \hfill
 \item \(C = 3000\) \euro ;\quad \(i = 0,02\) ;\quad \(t = 3\) mesi;
       \quad \(M=\;?\) 
\hfill
 \end{enumeratea}
\end{esercizio}


\begin{esercizio}
Risolvi i seguenti problemi:
 \begin{enumeratea}
  \item Ho investito 15000 \euro\, in regime di capitalizzazione composta 
per 3 anni a un tasso d'interesse del 5\% annuo. Quale interesse ho 
maturato?\hfill
  \item Cinque anni fa ho investito un certo capitale in regime di 
capitalizzazione semplice al 3\% annuo. Oggi ho ritirato 12995 \euro. 
Quanti soldi avevo investito?\hfill [11300 \euro]
  \item Vengono investiti, in regime di capitalizzazione semplice, 50000 
\euro\; per 10 anni. Per ottenere un interesse di 1000 euro, a quale tasso 
d'interesse si deve investire tale somma?\hfill [0,02 \%]
 \end{enumeratea}
\end{esercizio}


\subsection{Trasporto di capitali nel tempo}
\begin{esercizio}
A seguito di alcuni investimenti dovrei riscuotere 7000 € tra 3 mesi, 10 
000 € tra 6 mesi e 8500 € tra un anno. Quanto posso riscuotere tra 5 mesi, 
al tasso annuo del 2,5\%,
\hfill[25 386,75]
\end{esercizio}

\begin{esercizio}
Acquisto oggi un'automobile e posso pagarla così. Oggi verso 9500 €, tra 6 
mesi verso 6000 € e tra 12 mesi verso 15700 €. Quanto vale l'automobile se 
viene applicato un tasso del 4\% annuo.
\end{esercizio}


\subsection{Rendite}
\begin{esercizio}
Una persona vuole costituire una somma che gli consenta, fra 4 anni, di 
poter cambiare l'auto; per questo versa E 1350 ogni quadrimestre a partire 
da oggi, ad un tasso annuo nominale convertibile quadrimestralmente del 
6\%. Quale somma avrà a disposizione all'epoca stabilita?
\hfill[18468,45]
\end{esercizio}
\begin{esercizio}
Hai versato in banca E 8000 alla fine di ogni anno e per 6 anni, al tasso 
annuo del 2,5\%. Se decidi di
ritirare il capitale all'atto dell'ultimo versamento, di quale somma potrai 
disporre?
\hfill[51101,89]
\end{esercizio}
\begin{esercizio}
Fra 5 anni avremo bisogno di una somma 5200 € per restituire un prestito 
che ci è stato fatto. Decidiamo
allora di depositare ogni anno, alla fine dell'anno, una somma che sia in 
grado di costituire questo capitale. Qual eÁ il valore di questa somma al 
tasso annuo del 4\%?
\end{esercizio}
\begin{esercizio}
Una persona ha iniziato a versare 15 anni fa presso una banca 800 € 
all'anno ed ha proseguito i versamenti fino ad oggi; 4 anni fa, inoltre, ha 
depositato presso la stessa banca 9800 €. Per tutta la durata
dell'operazione, la banca ha mantenuto costante il tasso d'interesse al 
2,5\% annuo. Se oggi questa persona preleva 23000 € qual è il saldo del suo 
conto?

\end{esercizio}

\begin{esercizio}
Riccardo sa che fra 6 anni avrà bisogno di 10000€ per festeggiare con un 
viaggio i suoi 25 anni di matrimonio. Calcola la rata annua anticipata al 
tasso dell'4,6\% annuo che Riccardo deve versare per poter costituire 
questo capitale.
\end{esercizio}
\begin{esercizio}
Un appartamento viene affittato per un anno ad un canone mensile di 2 000 
€. Volendo pagare anticipatamente l'intero ammontare del canone, quanto si 
deve versare al proprietario se la valutazione viene fatta al 2\%
\end{esercizio}


\subsection{Ammortamenti}
\begin{esercizio}
Determina la rata di un mutuo per un capitale di \(150\;000\), 
durata di \(20\) anni e un tasso di interesse annuo del \(4\%\)
\end{esercizio}
\begin{esercizio}
Determina la rata di un prestito per un capitale di \(4\;000\), 
durata di \(5\) anni e un tasso di interesse annuo del \(9\%\)
\end{esercizio}
\begin{esercizio}
Determina la rata di un prestito per un capitale di \(80\;000\), 
durata di \(15\) anni e un tasso di interesse annuo del \(1,5\%\)
\end{esercizio}
\begin{esercizio}
Per un mutuo per \(60\;000\) euro mi hanno proposto due opzioni: nella prima 
pago una rata di 530 € per 10 anni, nella seconda 380 € per 15 anni. 
Calcola i due tassi di interesse applicati.
\end{esercizio}


\begin{esercizio}
Stendi un piano di ammortamento per un prestito di \(3\;500\), durata di \(3\) 
anni e un tasso di interesse annuo del \(2\%\)
\end{esercizio}
\begin{esercizio}
Stendi un piano di ammortamento per un prestito di \(42\;000\), durata di \(6\) 
anni e un tasso di interesse annuo del \(3\%\). Se decido di rimborsare tutto 
il capitale dopo 3 anni, quanto devo versare alla banca?
\end{esercizio}

\end{document}
