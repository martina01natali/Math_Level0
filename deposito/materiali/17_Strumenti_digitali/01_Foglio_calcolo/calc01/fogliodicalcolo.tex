% (c) 2014 Daniele Zambelli - daniele.zambelli@gmail.com

\newcommand{\salvare}[1][.5]{
\emph{Questo è un buon momento per salvare il lavoro fatto.}
\vspace{#1em}
}

\newcommand{\sottotitolo}[1]{
\emph{#1}
\vspace{.5em}
}

\chapter{Foglio di calcolo}

\section{Avviamo ``Calc''}

\sottotitolo{Perché un foglio di calcolo}

In molti ambiti gli umani sono costretti ad effettuare molti calcoli,
pensiamo solo all'economia, alla ricerca scientifica o statistica,
alla progettazione, ...
I matematici spesso hanno realizzato strumenti per semplificare i calcoli,
inventando i computer hanno trovato il modo di far fare \emph{completamente} 
i calcoli a qualcun altro: al computer.

Se dobbiamo eseguire molte operazioni è più sicuro (e meno noioso),
farle fare ad un elaboratore elettronico. Ma come convincere un
calcolatore a fare i calcoli per noi? Il modo più semplice è quello di
avviare un apposito programma che si chiama genericamente
``foglio di calcolo'' o ``foglio elettronico'' o ``spreadsheet''.

Ne esistono molti in commercio, noi ci riferiremo a ``Calc'' che è il
foglio di calcolo del programma di ufficio: ``Libre Office''.
Se non avete ``Libre Office'' nel vostro computer, fatevi aiutare da qualcuno
esperto e installatelo, è facile.
È realizzato da una comunità di utenti e programmatori, si scarica 
gratuitamente da Internet ed è un \emph{software libero}\footnote{
Il softwre ``libero'' è caratterizzato da queste 4 libertà:
\begin{enumerate} [nosep]
  \setcounter{enumi}{-1}
\item Libertà di eseguire il programma per qualsiasi scopo. 
\item Libertà di studiare come funziona il programma e di modificarlo in base 
alle proprie necessità.
\item Libertà di ridistribuire copie del programma in modo da aiutare il 
prossimo.
\item Libertà di migliorare il programma e di distribuirne 
pubblicamente i miglioramenti, in modo tale che tutta la comunità ne tragga 
beneficio.
\end{enumerate}
}.

\noindent{
% \begin{figure}[htbp]
\begin{inaccessibleblock}[Figura: TODO]
 \includegraphics[scale=0.4]{img/calc_00.png}
\end{inaccessibleblock}
% \caption{Come si presenta una finestra di Calc.}
% \end{figure}
}

Una volta trovato (o installato) \emph{Libre Office} avviate il programma ``Calc''.
Vi troverete davanti un foglio di calcolo e tutta una cornice che contiene
gli strumenti per gestirlo, dall'alto in basso possiamo riconoscere:

\begin{itemize} [noitemsep]
\item il menu;
\item {}
la barra delle icone (individuate l'icona per salvare il lavoro,
per stampare il foglio, per le operazioni di taglia-copia-incolla, ...);
\item la barra di formattazione;
\item la barra di immissione;
\item i bordi del foglio;
\item il foglio vero e proprio;
\item la barra di stato.
\end{itemize}

Nei seguenti paragrafi vedremo cosa è e come si usa un \emph{foglio di calcolo}.

\section{Celle, colonne, righe... il foglio di calcolo}
\label{fogliodicalcolo:celle-colonne-righe-il-foglio-di-calcolo}

\sottotitolo{Cos'è, e come usare le funzioni di base di un foglio di
calcolo.}

Un Foglio di Calcolo è un'immensa tabella composta da alcune migliaia di
\emph{righe} e alcune centinaia di \emph{colonne} che generano una grande 
quantità di celle nei loro incroci.
L'elemento base di un Foglio di Calcolo, è dunque la cella.
Ogni cella ha: un \emph{indirizzo}, un \emph{contenuto} e un \emph{formato}:

\paragraph{Indirizzo}

Come nella battaglia navale l'indirizzo di ogni cella è composto da una
lettera seguita da un numero, ad es. B3 è la cella che si trova all'incrocio
della seconda colonna con la terza riga. Poiché le lettere sono solo 26 e
noi, a volte, abbiamo bisogno di più colonne, arrivati alla lettera ``Z''
proseguiamo con ``AA'', ``AB''... e così via.
Nella \emph{barra di immissione}, in alto a sinistra viene visualizzato
l'indirizzo della cella in cui ci troviamo.
Cliccando in diverse celle si può osservare l'indirizzo che cambia.

\paragraph{Contenuto}

Ogni cella può avere un contenuto che è uno di questi 3 oggetti:

\begin{itemize} [noitemsep]
\item 
\emph{Parole}, una stringa qualunque\footnote{Una stringa è una sequenza di 
caratteri (anche un numero può essere visto come stringa).}.
\item 
\emph{Numeri}, che possono rappresentare anche percentuali, ore o date.
\item 
\emph{Formule}, espressioni che iniziano con un uguale. Quando si termina di
inserire una formula, nella cella viene mostrato il risultato del calcolo,
mentre il testo della formula appare nella parte alta dello schermo nella
barra di immissione.
\end{itemize}

Gli operandi delle formule, possono essere: stringhe,  numeri o indirizzi di 
celle.
Quando viene modificato il contenuto di una cella, tutte
le formule che fanno riferimento al suo indirizzo vengono ricalcolate.

\paragraph{Formato}

Ogni cella ha diversi attributi che riguardano il suo formato o quello del
suo contenuto.
Ci sono decine di aspetti che possono essere modificati:

\begin{itemize} [noitemsep]
\item colore di sfondo;
\item bordo;
\item dimensioni;
\item font, colore, dimensione dei caratteri;
\item formato dei numeri;
\item allineamento del contenuto;
\item ...
\end{itemize}

\begin{esempio}
Possiamo applicare queste prime informazioni per realizzare un formulario di
geometria che calcoli perimetri e aree di vari poligoni.
Apriamo un nuovo foglio di calcolo. Prima ancora di incominciare a riempirlo
lo salviamo con nome:

\emph{Menu-File-Salva Come}

Conviene salvarle il documento
in una nostra cartella e darle per nome ``quadrilateri''.
Per salvare un file basta anche cliccare sull'icona con un dischetto,
di solito terza da sinistra o più rapidamente ancora premere il tasto:
\texttt{\textless{}Ctrl-s\textgreater{}}.

L'obiettivo è avere un foglio nel quale inserire alcuni dati relativi ai
quadrilateri notevoli e calcolare altre informazioni relative alla figura.
Possiamo distinguere con un colore di sfondo le celle nelle quali inserire
dati e con un altro colore quelle che conterranno i risultati.
Dovremo adattare la larghezza delle colonne a seconda dello spazio occupato
dal contenuto.
Potrebbe anche essere utile graficamente separare i vari problemi
riquadrando con un bordo le relative celle.
Suggerimenti per l'inizio del lavoro:

\noindent{
\begin{minipage}{.48\textwidth}
\begin{inaccessibleblock}[Formulario di geometria]
\begin{center}
 \includegraphics[scale=0.55]{img/quadrilateri_00.png}
\end{center}
\end{inaccessibleblock}
\end{minipage}
\hfill
\begin{minipage}{.48\textwidth}
\begin{itemize} [noitemsep]
\item \texttt{A1: Formulario di geometria: i quadrilateri}
(dimensione e colore a fantasia)
\item \texttt{A3: Problemi sul Quadrato} (grassetto, corsivo, dimensione=16)
\item \texttt{A5: lato:}
\item \texttt{B5: 3} (sfondo giallo)
\item \texttt{A6: perimetro:}
\item \texttt{B6: =B5 * 4}
\item \texttt{A7: area:}
\item \texttt{B7: =B5\textasciicircum{}2}
\item \texttt{A8: diagonale:}
\item \texttt{B8: =B5*radq(2)} (colore di sfondo: azzurro)
\item \texttt{A5:E8} (Menu-Formato-Cella-Bordo, dimensione=12)
\item \texttt{A5:A8} (Allineamento a destra, corsivo)
\end{itemize}
\end{minipage}
}

\salvare

Prima di procedere con il formulario conviene provare inserendo nella cella
\texttt{B5} diversi valori numerici prima semplici per controllare che il 
foglio esegua calcoli corretti, poi più strani, con la virgola, molto grandi 
o molto piccoli controllando i corrispondenti risultati.
Una volta risolti eventuali problemi riscontrati, possiamo salvare il lavoro
fatto e passare ai problemi inversi del quadrato che si trovano nelle colonne 
successive.
\end{esempio}

\begin{esempio}
Seguendo le indicazioni precedenti prosegui con il lavoro completando il 
formulario per il rettangolo.
\end{esempio}

\paragraph{Riassumendo}

\begin{itemize} [nosep]
\item 
Un foglio di calcolo è composto da un gran numero di ``celle'' organizzate
in ``righe'' e ``colonne''
\item 
Ogni cella è caratterizzata da:
\begin{itemize} [nosep]
\item un indirizzo, composto da una lettera o gruppo di lettere e un numero;
\item un contenuto, che può essere:
  \begin{itemize}
  \item un testo,
  \item un numero,
  \item una formula;
\end{itemize}
\item un formato.
\end{itemize}
\item È Importante salvare spesso il proprio lavoro.
\end{itemize}

\section{Formati}
\label{fogliodicalcolo:formati}

\sottotitolo{Come selezionare un blocco di celle, sommare dati, modificare la 
larghezza di una colonna.}
\vspace{1em}

Affrontiamo ora un problema un po' più complesso che ci darà l'opportunità di 
esplorare varie funzioni del foglio di calcolo.

Crea un nuovo foglio e salvalo con il nome ``continenti'' nella tua cartella 
di lavoro.
\vspace{.5em}

\noindent{
\begin{minipage}{.48\textwidth}
Procurati i dati aggiornati relativi alla superficie e alla popolazione dei 
continenti e realizza un foglio di calcolo simile a quello in figura. 

Per ora lavoreremo su pochi dati, ma cerchiamo di
ragionare pensando di avere a che fare con centinaia di righe di dati
invece che con solo queste sei.
\end{minipage}
\hfill
\begin{minipage}{.48\textwidth}
\begin{inaccessibleblock}[Tabella con i dati relativi ai continenti]
\begin{center}
\includegraphics[scale=0.6]{img/cont_00.png} 
\end{center}
\end{inaccessibleblock}
\end{minipage}
}

\vspace{.5em}
\salvare

I numeri con troppe cifre sono difficili da leggere e valutare,
per facilitare questo compito, di solito, si separano le cifre a gruppi di 3
con dei puntini, i separatori delle migliaia
(delle virgole per gli anglosassoni che
usano invece il punto per separare la parte intera da quella decimale).
Selezioniamo le celle da \texttt{B4} a \texttt{C10} e da 
\texttt{Menu-Formato-Celle-Numeri}
scegliamo il numero con il separatore delle migliaia e senza cifre decimali.

I caratteri con cui stiamo lavorando sono piuttosto piccoli, vogliamo
aumentare la dimensione della font dei caratteri per tutte le celle del
foglio. Per selezionarle tutte in un solo colpo possiamo cliccare
nell'angolo della cornice con le intestazioni delle righe e delle colonne,
il rettangolino che si trova sopra a ``1'' e a sinistra di ``A''. Una volta
selezionato tutto il foglio di lavoro, nella barra di formattazione cambiamo
la dimensione del font da 10 a 12.

A questo punto può succedere un effetto spiacevole: alcune celle
dove prima c'era un numerone ora appaiono tre \emph{cancelletti}: ``\#\#\#''.

\noindent{
\begin{minipage}{.48\textwidth}
\begin{inaccessibleblock}[]
\begin{center}
 \includegraphics[scale=0.55]{img/cont_01.png}
\end{center}
\end{inaccessibleblock}
\end{minipage}
\hfill
\begin{minipage}{.48\textwidth}
Cosa è successo? Se una cella non è abbastanza grande per contenere un
numero questo non viene tagliato.
Poiché non è accettabile che un numero venga visualizzato solo in parte,
quando non può essere contenuto in una cella, viene sostituito da un simbolo
convenzionale: ``\#\#\#''.
\end{minipage}
}
\vspace{.5em}

Per vedere di nuovo il nostro numero possiamo seguire una delle seguenti
strade:

\begin{enumerate} [noitemsep]
\item togliere i puntini delle migliaia;
\item diminuire le dimensioni del carattere;
\item allargare la cella.
\end{enumerate}

La soluzione più adatta nel nostro caso è la quarta:
\begin{itemize}
\item 
Togliamo tre zeri ad ogni numero relativo alla popolazione in modo da avere 
il numero di migliaia di invece che di unità. Dovremo, però, indicarlo nelle 
intestazioni.
% \item 
% Poi, dato che la parola ``popolazione'' non ci sta nella cella, allarghiamo 
% le tre colonne:
% Clicchiamo con il tasto destro del mouse sull'intestazione della colonna da
% allargare e dal menu a tendina che appare scegliamo la voce:
% ``Larghezza colonna''.
% Nel campo di inserimento al posto di \(2,62\) scriviamo \(2,8\).
\end{itemize}

\salvare

\noindent{
\begin{minipage}{.48\textwidth}
\begin{inaccessibleblock}[]
\begin{center}
 \includegraphics[scale=0.55]{img/cont_02.png}
\end{center}
\end{inaccessibleblock}
\end{minipage}
\hfill
\begin{minipage}{.48\textwidth}
I numeri sono a posto, ora dobbiamo sistemare le intestazioni.

Le unità di misura non vanno mai messe nelle celle dei dati ma vanno 
indicate nelle intestazioni.

Di seguito riporto le istruzioni per ottenere quanto si vede qui a sinistra:
\end{minipage}
}

\begin{itemize} [noitemsep]
\item
La cella \texttt{B3} contiene un carattere posto a indice. Per ottenerlo,
innanzitutto scrivono tutti i caratteri che vogliamo appaiano:
``Area (km2)'', poi con il mouse selezioniamo nella riga di immissione
il solo carattere ``2'' e da \texttt{Menu-Formato-Carattere-Posizione} scegliamo
``apice''. Confermando con invio otteniamo il risultato desiderato.
\item 
Le due celle \texttt{B3:C3} contengono una scritta troppo lunga che esce dai 
bordi, vorremmo che fosse spezzata su due righe. 
Selezioniamo le due celle\footnote{
Per selezionare un gruppo di celle contiguo e rettangolare basta cliccare
sulla cella in alto a sinistra e, tenendo premuto il tasto sinistro del
mouse, trascinare il cursore fino alla cella in basso a destra.
Quando si rilascia il tasto del mouse il colore delle celle selezionate
apparirà invertito.} 
e modifichiamo il formato della cella:
\texttt{Menu-Formato-Celle-Allineamento-Ritorno a capo automatico}.
\item 
Selezioniamo le celle \texttt{B3:C3} e attiviamo il grassetto, il corsivo e 
il centrato.
\item
A volte può aiutare lo sguardo avere le celle con un contorno. Selezionate le 
celle \texttt{A3:C9} e date il comando: \texttt{Menu-Formato-Celle-Bordi}.
\end{itemize}

\salvare[0]

\paragraph{Riassumendo}

\begin{itemize} [nosep]
\item È possibile selezionare un blocco di celle con il mouse o con la tastiera.
\item È possibile assegnare un formato a tutte le celle di un blocco.
\item È possibile calcolare la soma dei numeri contenuti in blocchi di celle.
\item Spesso ci sono molti modi diversi per eseguire la stessa operazione.
È importante saper usarne uno, poi gli altri si imparano con il tempo e
con l'uso.
\item È Importante salvare spesso il proprio lavoro.
\end{itemize}

\section{Ordinamento}
\label{fogliodicalcolo:ordinamenti}
\sottotitolo{Come riordinare i dati}

Se i continenti fossero decine o centinaia, per trovare i dati relativi ad
uno di questi sarebbe comodo averli scritti in ordine alfabetico.
Ma potrei essere interessato ai più grandi o ai più popolosi.

È molto comodo poter ordinare i dati rispetto ad un certo criterio.
Possiamo dire a \emph{Calc} di ordinare le righe in base al contenuto di una 
colonna (o le colonne in base al contenuto di una riga).
\begin{itemize} [nosep]
\item 
Se vogliamo i continenti ordinati dal più grande al più piccolo, 
dopo aver selezionato tutte le celle che contengono i dati da ordinare, 
\texttt{A4:C9} 
scegliamo dal \texttt{Menu-Dati-Ordina} come primo criterio la colonna 
\texttt{b} e come ordine quello discendente.
\item 
Se vogliamo i continenti ordinati dal più popolato al meno popolato, 
dopo aver selezionato le celle \texttt{A4:C9} ,
scegliamo dal \texttt{Menu-Dati-Ordina} come primo criterio la colonna 
\texttt{C} e come ordine quello discendente.
\item 
Se vogliamo ottenere i continenti in ordine alfabetico selezioniamo il blocco
di celle da e attraverso il \texttt{Menu-Dati-Ordina} 
scegliamo come primo criterio la colonna ``A''.
Confermando, otteniamo le righe ordinate in ordine alfabetico dall'Africa
all'Oceania.
\end{itemize}

\noindent{
\begin{minipage}{.48\textwidth}
\begin{inaccessibleblock}[Dati ordinati per superficie]
\begin{center}
 \includegraphics[scale=0.55]{img/cont_03.png}
\end{center}
\end{inaccessibleblock}
\end{minipage}
\hfill
\begin{minipage}{.48\textwidth}
\begin{inaccessibleblock}[Dati ordinati per popolazione]
\begin{center}
 \includegraphics[scale=0.55]{img/cont_04.png}
\end{center}
\end{inaccessibleblock}
\end{minipage}
}

\paragraph{Riassumendo}

\begin{itemize} [nosep]
\item Si può ordinare un blocco di celle in base a diversi criteri.
\item Si può effettuare l'ordinamento \emph{per righe}, o \emph{per colonne}.
\item Si può effettuare un ordinamento \emph{crescente}, o 
\emph{decrescente}.
\item È Importante salvare spesso il proprio lavoro.
\end{itemize}

\section{Copiare in modo intelligente}
\label{fogliodicalcolo:copiare-in-modo-intelligente}

\sottotitolo{Come ricopiare formule usando indirizzi relativi e assoluti.}


Vogliamo ricavare delle nuove informazioni a partire dai dati che già 
abbiamo.
Per prima cosa aggiungiamo i totali delle due colonne:

\noindent{
\begin{minipage}{.48\textwidth}
\begin{itemize} [noitemsep]
\item 
Aggiungiamo il testo della cella \texttt{A10} (corsivo, a destra),
\item
e le formule che calcolano la somma:
\texttt{B10: =somma(B4:B9)} (corsivo) \\
\texttt{C10: =somma(C4:C9)} (corsivo)
\end{itemize}
\end{minipage}
\hfill
\begin{minipage}{.48\textwidth}
\begin{inaccessibleblock}[]
\begin{center}
 \includegraphics[scale=0.55]{img/cont_05.png}
\end{center}
\end{inaccessibleblock}
\end{minipage}
}

\vspace{.5em}
\salvare

Se effettuiamo un doppio clic nella cella \texttt{B10} ci viene evidenziata 
la formula e la zona di celle su cui lavora.

Dato che la somma di un gruppo contiguo di celle è molto frequente, ci sono
molti modi per immettere queste formule. Proviamo a vederli, poi, a seconda
dei casi useremo quello più comodo. 

Per prima cosa cancelliamo il contenuto delle celle \texttt{B10:C10}.
Ci riportiamo nella cella \texttt{B10} e: iniziamo a scrivere la formula:

\texttt{=somma(}

selezioniamo con il mouse le celle \texttt{B4:B10},
chiudiamo la parentesi tonda e confermiamo con il tasto 
\texttt{\textless{}Invio\textgreater{}}

Per la cella \texttt{C11} proviamo ad usare un altro metodo.
Una volta portati nella cella \texttt{C11}, clicchiamo l'icona della 
\emph{sommatoria}
che si trova in alto a sinistra della casella di inserimento se le scelte
di Calc ci vanno bene, confermiamo la formula con il tasto 
\texttt{\textless{}Invio\textgreater{}}.

\salvare

Da questi dati possiamo ricavare altre informazioni, 
possiamo, ad esempio, far calcolare la densità di popolazione per mezzo
della formula: \emph{popolazione/superficie}.

\noindent{
\begin{minipage}{.48\textwidth}
\begin{inaccessibleblock}[]
\begin{center}
 \includegraphics[scale=0.55]{img/cont_06.png}
\end{center}
\end{inaccessibleblock}
\end{minipage}
\hfill
\begin{minipage}{.48\textwidth}
\begin{itemize} [noitemsep]
\item \texttt{D3: Densità ab/km2} 
(centrato, grassetto); \quad
Selezionare nella riga di input il solo~2 
(formato-carattere-posizione-apice); \quad
(formato cella-allineamento-acapo automatico)
\item \texttt{D4: =C4*1000/B4}
(formato-celle-numeri-zero decimali)
\item \texttt{D5: =C5*1000/B5}
(formato-celle-numeri-zero decimali)
\item \dots
\end{itemize}
\end{minipage}
}

Dato che i continenti sono solo \(6\) non è un grande problema scrivere le 
\(6\) formule diverse una sotto l'altra, ma in un foglio di calcolo spesso si 
devono scrivere decine o centinaia di formule simili a queste!
Chi ha progettato il foglio di calcolo ha previsto degli strumenti che
permettono di ricopiare velocemente le formule.

Scriviamo la prima formula nella cella \texttt{D4} e sistemiamo il formato.
Poi riportiamo il cursore su questa cella,
appare nell'angolo in basso a destra, della cella stessa, un quadratino nero;
con il mouse trasciniamo questo quadratino verso il basso fino a coprire
tutte le celle in cui vogliamo ricopiare la formula.

Non solo il programma ha ricopiato la formula ma ha anche aggiustato gli
indici, proprio come ci serviva.
Da notare che quando viene ricopiata una formula vengono anche ricopiati i
formati della celle in cui la formula è stata scritta.

\salvare

Un'altra informazione interessante che possiamo ricavare da questi dati è la 
percentuale rappresentata dalla superficie di un continente rispetto alla 
superficie totale delle terre emerse.

\noindent{
\begin{minipage}{.48\textwidth}
La percentuale non è altro che un rapporto, il quoziente tra la superficie
di un continente e il totale.
Procediamo con il lavoro:
\begin{itemize} [noitemsep]
\item \texttt{E3: superficie} (centrato, grassetto)
\item \texttt{E4: =B4/B10}
\end{itemize}
\end{minipage}
\hfill
\begin{minipage}{.48\textwidth}
\begin{inaccessibleblock}[]
\begin{center}
 \includegraphics[scale=0.55]{img/cont_07.png}
\end{center}
\end{inaccessibleblock}
\end{minipage}
}

Il risultato di questo calcolo è un numero compreso tra zero e uno,
non è certo la percentuale cercata,
se lavoriamo sulla carta, per trasformare questo numero nella percentuale
basta moltiplicarlo per \texttt{100}. Nei fogli di calcolo, invece, basta 
indicare nel formato della cella che quel numero deve essere inteso come una 
percentuale:

\begin{itemize} [noitemsep]
\item \texttt{E4: =B4/B10}
(formato-celle-numeri-percentuale)
\item \texttt{E5: =B5/B10}
(formato-celle-numeri-percentuale)
\item \dots
\end{itemize}

\subsection{Indirizzi relativi e assoluti}

Anche qui, invece di riscrivere tutte le formule possiamo sfruttare le
capacità del foglio di calcolo e farle ricopiare verso il basso.
Dopo esserci posizionati nella cella \texttt{E4}, prendiamo il quadratino che
appare in basso a destra e trasciniamolo verso il basso in modo da coprire
le celle di tutti i continenti.
Questa volta l'effetto non è quello desiderato:
otteniamo una serie di errori! Come mai?

Osserviamo una delle celle in cui è comparso l'errore, la cella
\texttt{E5} contiene la formula \texttt{=B5/B12}.
Per capire meglio la formula selezioniamo la cella con un doppio clic.
Vengono evidenziate in rosso e blu le celle che sono utilizzate nella formula
stessa.
Appare evidente che \texttt{B5} va bene, ma \texttt{B11} doveva essere 
\texttt{B10}!
Nella cella \texttt{B12} non c'è niente e il foglio di calcolo la interpreta
come se contenesse il valore \texttt{0}.
Giustamente produce un errore di divisione per \texttt{0}.

Noi vogliamo che, nel ricopiare le formule, l'indice numerico di \texttt{B4} 
venga
modificato ma quello di \texttt{B10} rimanga costante.
Nei termini dei fogli di calcolo si dice che \texttt{B4} deve essere un
\textbf{indirizzo relativo}, \texttt{B10} un \textbf{indirizzo assoluto}.
Per essere pignoli a noi non occorre che tutto \texttt{B10} sia assoluto,
siccome vogliamo ricopiare la formula verso il basso ci basta che sia
assoluta la parte numerica dell'indirizzo: il \texttt{10}.

Per comunicare questi desideri al foglio di calcolo basta mettere il 
carattere dollaro: ``\$'', davanti al riferimento che vogliamo rimanga 
invariato.
Questo fa si che il programma quando ricopia le formule non ne modifichi
il riferimento.
Aggiustiamo la prima formula:

\begin{itemize} [noitemsep]
\item \texttt{E4: =B4/B\$10}
(formato-celle-numeri-percentuale; )
\end{itemize}

Ora ricopiare la cella verso il basso produce l'effetto desiderato!
Nella cella \texttt{E5} ci sarà la formula \texttt{=B5/B\$10},
nella cella \texttt{E6} la formula \texttt{=B6/B\$10}, e così via.

L'elaborazione numerica dei nostri dati è completa,
disegniamo un bordo anche attorno alle nuove celle che abbiamo
riempito ottenendo così un foglio presentabile.

\emph{E salviamo il lavoro fatto.}
\vspace{.5em}

\paragraph{Riassumendo}

\begin{itemize} [nosep]
\item 
Si possono ``ricopiare'' formule trascinando il quadratino che appare in
basso a destra di una cella selezionata.
\item 
Quando ricopiamo una formula verticalmente gli indici relativi alla riga,
i numeri, vengono modificati (indirizzo relativo).
\item 
Quando ricopiamo una formula orizzontalmente gli indici relativi alla
colonna, le lettere, vengono modificati (indirizzo relativo).
\item 
Se vogliamo che, nel ricopiare una formula, un indice non venga modificato,
basta che lo facciamo precedere dal carattere: ``\$'' (indirizzo assoluto).
\item È Importante salvare spesso il proprio lavoro.
\end{itemize}

\section{Grafici}
\label{fogliodicalcolo:grafici}

\sottotitolo{Come rappresentare graficamente i dati.}

Spesso un grafico dà una più immediata comprensione di un fenomeno rispetto
ad una lista di numeri.
I fogli di calcolo permettono di disegnare grafici di diversa forma.

Riprendendo il foglio dei continenti vogliamo aggiungere due grafici per
rappresentare la superficie e la popolazione.

Selezioniamo le celle che
contengono i dati che vogliamo rappresentare.
Iniziamo costruendo un grafico a torta che riporti la superficie dei
diversi continenti.

\begin{enumerate} [noitemsep]
\item Selezioniamo le celle \texttt{A4:B9}.
\item Da menu scegliamo Inserisci-Grafico, viene così aperta una finestra
di dialogo che ci guida nella definizione del grafico.
\item Scegliamo il grafico a torta
poi clicchiamo su \texttt{Successivo>}.
\item Controlliamo che siano selezionate le caselle:
``Serie di dati in colonna'' e ``Prima colonna come didascalia''
poi clicchiamo su \texttt{Successivo>}.
\item Nella pagina ``Serie di dati'' non tocchiamo niente 
e clicchiamo su \texttt{Successivo>}.
\item Nell'ultima pagina scriviamo il titolo  ``Superficie dei continenti''
e confermiamo cliccando sul bottone ``Fine''.
\end{enumerate}

\noindent{
\begin{minipage}{.48\textwidth}
\begin{inaccessibleblock}[]
\begin{center}
 \includegraphics[scale=0.55]{img/cont_08.png}
\end{center}
\end{inaccessibleblock}
\end{minipage}
\hfill
\begin{minipage}{.48\textwidth}
A questo punto il grafico è selezionato ed è possibile intervenire in vari 
modi accedendo al menu di calc che è modificato rispetto a quello che appare 
quando è selezionata una cella e è possibile modificare tutte le 
caratteristiche del grafico.

Ad esempio attraverso \texttt{Menu-Formato-Area del grafico\dots} possiamo 
aggiungere un bordo.
\end{minipage}
}

\vspace{.5em}
Calc ci propone un grafico più grande di quello che ci serve, clicchiamo 
fuori dal grafico e poi di nuovo nel grafico per poter modificare le sue 
dimensioni e la posizione.
Agendo sulle maniglie di dimensionamento e trascinandolo lo mettiamo subito 
sotto ai dati, a sinistra.
\salvare

Se vogliamo modificare più profondamente il grafico possiamo
effettuare un doppio clic sul grafico stesso.
Il menu principale del foglio di calcolo cambia e cambiano anche i menu
contestuali (quelli legati al tasto destro) a seconda di cosa viene puntato
dal mouse.
% Dal menu ``Inserisci'' scegliamo ``Legenda'' e togliamo il segno di spunta su
% ``Visualizza''.
% 
% La Legenda scompare, ma adesso il grafico è di difficile interpretazione,
% operiamo dunque un'altra modifica:
% sempre dal menu Inserisci scegliamo ``Etichette'' e chiediamo che ci vengano
% mostrati i valori come percentuale e anche le etichette di testo.
% Se le etichette sono troppo lunghe e sbilanciano la rappresentazione conviene
% abbreviarle.
% Ora se il grafico risulta troppo piccolo e non riempie bene lo spazio
% a sua disposizione possiamo cliccare vicino alla \emph{torta} e allargarlo
% agendo sulle maniglie verdi che appaiono.

\salvare

Ora vogliamo un grafico che contenga i dati relativi al numero di
abitanti, dobbiamo selezionare i nomi dei continenti e i valori della
popolazione.
Purtroppo questi valori non sono contigui, per selezionarli
dobbiamo usare un trucco:

\begin{enumerate} [nosep]
\item selezioniamo con il mouse le celle \texttt{A4:A9} e
\item selezioniamo le celle \texttt{C4:C9} tenendo premuto contemporaneamente
il tasto \texttt{\textless{}Ctrl\textgreater{}}.
\end{enumerate}

\noindent{
\begin{minipage}{.48\textwidth}
\begin{inaccessibleblock}[]
\begin{center}
 \includegraphics[scale=0.55]{img/cont_09.png}
\end{center}
\end{inaccessibleblock}
\end{minipage}
\hfill
\begin{minipage}{.48\textwidth}
Il tasto \texttt{\textless{}Ctrl\textgreater{}} permette di effettuare 
selezioni multiple su blocchi rettangolari non contigui.
Dopo aver selezionato le aree contenenti i dati,
dal \texttt{menu-Inserisci} scegliamo la voce ``Grafico\dots''.
Questa volta creiamo un istogramma.
% Come prima assicuriamoci che sia selezionata la voce
% ``Prima colonna come didascalia''.
Nell'ultima pagina scriviamo il titolo del grafico: ``Popolazione'', 
deselezioniamo la voce ``Legenda'' e clicchiamo su ``Fine''.
\end{minipage}
}

A questo punto possiamo far disegnare un bordo attorno al grafico.

Poi cliccando fuori dal grafico e di nuovo dentro possiamo modificarne le 
dimensioni e la posizione.

Avendolo ristretto dobbiamo intervenire su diversi suoi elementi per renderlo 
di nuovo significativo. Doppio clic.
Le etichette dell'asse \(x\) devono essere scritte con caratteri 
più piccoli: doppio clic su una parola dell'etichetta e possiamo modificare 
tutte le caratteristiche dell'asse. Diamo ai caratteri la dimensione 8.
Spostiamo il titolo un po' più in alto, aumentiamo le dimensioni dell'area 
del grafico.
Ora i grafici sono come li volevamo.

\salvare[0]

\paragraph{Riassumendo}

\begin{itemize} [nosep]
\item Il modo più semplice per realizzare un grafico è quello di selezionare
i dati che vogliamo rappresentare e poi scegliere Menu-Inserisci-Diagramma.
\item Nel dialogo di costruzione di un grafico possiamo scegliere diverse
caratteristiche: etichette, tipo e sottotipo, assi, legenda, titoli, ...
\item Una volta costruito un grafico è possibile modificarlo usando il menu
che appare dopo aver effettuato un doppio clic sul grafico stesso.
\item È Importante salvare spesso il proprio lavoro.
\end{itemize}

\section{Impaginazione}
\label{fogliodicalcolo:impaginazione}

% Clicchiamo fuori dai grafici, in una cella qualunque,
% poi da \texttt{Menu-Visualizza} scegliamo ``Interruzioni di pagina''.
% Una linea blu delimiterà i contorni delle varie pagine, modifichiamo le
% dimensioni dei grafici o spostiamoli in modo da farli rientrare tutti
% in un'unica pagina, assieme ai dati.
% Se la scala della visualizzazione si è troppo ridotta possiamo cliccare
% con il destro sulla percentuale presente nella barra di stato (in basso)
% e scegliere il valore ``100\%''. 

\noindent{
\begin{minipage}{.36\textwidth}
Prima di considerare finito il lavoro dobbiamo controllare come apprirà nella 
stampa.

Attiviamo l'anteprima di stampa.
Modificando il formato della pagina, \texttt{Menu-Formato-Pagina\dots}, 
possiamo:
\begin{itemize}
\item agire sull'orientamento: verticale o orizzontale;
\item modificare i margini riducendoli per lasciare più posto ai contenuti;
\item modificare l'intestazione o il piè di pagina: togliamo l'intestazione 
e nel piè di pagina scriviamo a sinistra la data e a destra il nostro nome.
\end{itemize}
\end{minipage}
\hfill
\begin{minipage}{.62\textwidth}
\begin{inaccessibleblock}[]
\begin{center}
 \includegraphics[scale=0.5]{img/cont_10.png}
\end{center}
\end{inaccessibleblock}
\end{minipage}
}

\salvare

Un'occhiata al lavoro svolto con l'anteprima di stampa può rassicurarci che
è tutto disposto per bene nella pagina.
Se siamo soddisfatti possiamo considerare finito il lavoro, altrimenti
modifichiamo gli aspetti che non ci piacciono e salviamo.

\paragraph{Riassumendo}

\begin{itemize} [nosep]
\item L'anteprima di stampa permette di vedere come sarà impaginato il nostro 
lavoro.
\item Il menu-Formato-Pagina permette di intervenire sull'orientamento,
le dimensioni, i margini, le intestazioni, i piè di pagina, ...
\item È Importante salvare spesso il proprio lavoro.
\end{itemize}

\subsection{Trasmissione}

Se il foglio di calcolo lo abbiamo creato per nostro uso esclusivo, possiamo 
chiudere calc e riaprire il foglio quando ci serve o quando vogliamo creare 
qualcosa di analogo.

Ma se dobbiamo inviare il nostro lavoro a qualcun altro, possiamo trovarci in 
una delle seguenti situazioni:
\begin{enumerate} [noitemsep]
\item il destinatario è fuori dal mondo delle comunicazioni digitali;
\item il destinatario deve solo leggere il nostro lavoro;
\item il destinatario deve poter modificare il nostro lavoro.
\end{enumerate}

Vediamo i tre casi.

\paragraph{Fuori} 
Dobbiamo stampare il foglio di calcolo su un foglio di carta e recapitarglielo.

\paragraph{Sola lettura}
Dobbiamo trasformare il nostro documento in un formato ``a sola lettura''. 
Per fare questo basta cliccare sull'icona vicina a quella della stampa: 
``Esporta direttamente in PDF''.

Ooppure è possibile passare per il \texttt{menu-File-Esporta nel 
formato PDF\dots} per avere un maggior controllo sul risultato.

Si può anche esportare il documento in un formato grafico come il ``PNG'' con 
texttt{menu-File-Esporta\dots}

E poi si invia il file PDF (o PNG).

\paragraph{Modifica}
Si invia direttamente il file ``.ods'' che è in un formato standard 
modificabile da tutti i programmi che implementano lo standard 
``Open Document Format''.

\paragraph{Riassumendo}

\begin{itemize} [nosep]
\item Cliccando sull'apposita icona si può convertire direttamente in pdf.
\item Il \texttt{menu-File-Esporta nel formato PDF\dots} permette di produrre 
un pdf controllando varie caratteristiche.
\item Il \texttt{menu-File-Esporta\dots} permette di scegliere altri formati.
\item Se non è proprio necessario, evitiamo di produrre carte stampate.
\end{itemize}

\section{Esercizi}
\label{fogliodicalcolo:esercizi}

\begin{esercizio}
Riporta in un foglio di calcolo il numero di pagine dei diversi testi
scolastici. Calcola la media di pagine per libro e la somma delle pagine.
Trova quante pagine devi leggere ogni giorno di scuola per
``consumare'' tutti i libri.
\end{esercizio}

\begin{esercizio}
Realizza un formulario dinamico che permetta di calcolare volume,
superficie, diagonale di un parallelepipedo rettangolo
dati i suoi tre spigoli.
\end{esercizio}

\begin{esercizio}
Realizza un formulario dinamico che permetta di calcolare volume,
superficie laterale, superficie totale di un prisma retto a base
triangolare
dati lo spigolo di base e l'altezza.
\end{esercizio}

\begin{esercizio}
Ricerca la superficie e le popolazione delle regioni italiane e realizza
un foglio di calcolo simile a quello relativo ai continenti.
\end{esercizio}

\begin{esercizio}
Procurati l'altezza dei i tuoi compagni di classe. Realizza un
foglio di calcolo in cui venga calcolata la media la moda e la mediana
dei valori.
\end{esercizio}

\begin{esercizio}
Annota tutto quello che mangi in una giornata segnando anche le quantità
approssimative. Cerca il valore energetico dei diversi cibi da te
consumati. Costruisci una tabella che calcoli l'energia introdotta
durante la giornata.
\end{esercizio}

\begin{esercizio}
Annota l'ora di inizio e di fine di ogni volta che ti metti davanti ad
uno schermo: (cellulare, televisione, computer).
Crea un foglio di calcolo che calcoli il tempo dedicato agli schermi in
ogni singolo intervallo, li sommi, trovi la percentuale della giornata
relativa ad ogni singolo schermo e a tutti assieme.
\end{esercizio}

\begin{esercizio}
Ricerca i dati relativi al consumo di carburante in Italia negli ultimi
anni. Rappresenta questi dati con un grafico.
\end{esercizio}

\begin{esercizio}
Annota i mezzi di trasporto utilizzati dalla vostra classe per venire
a scuola. Organizza questi dati in un foglio di calcolo, ricavane
la distribuzione percentuale e rappresentali con un grafico.
\end{esercizio}

\begin{esercizio}
In classe scegliete un testo di almeno una pagina. Distribuendovi una
lettera dell'alfabeto a testa, ognuno conti le occorrenze della sua
lettera nel testo scelto. Riportate tutti i numeri in un foglio di calcolo
calcolate la percentuale di occorrenze di ogni singola lettera.
Ordinate le righe dalla lettera più frequente a quella meno frequente.
\end{esercizio}

\begin{esercizio}
Ripetete l'esercizio precedente con un altro testo di italiano e con un
testo scritto in un'altra lingua. Scrivi una congettura che puoi fare
già con questi pochi esperimenti.
\end{esercizio}
