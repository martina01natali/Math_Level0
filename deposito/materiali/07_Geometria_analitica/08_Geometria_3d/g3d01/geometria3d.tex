\chapter{Geometria cartesiana dello spazio}

\section{Punti e vettori}
\label{sec:Punti_e_vettori}

Abbiamo già imparato in precedenza quali tecniche e formule si possono utilizzare per manipolare oggetti nel piano cartesiano bidimensionale. Andiamo ora a studiare l'analogo caso tridimensionale. 

\vspace{7pt}

Il piano cartesiano è formato da 3 assi tra loro perpendicolari, che indicano le 3 direzioni possibili di movimento per un oggetto nello spazio. Un punto viene perciò individuato dalle 3 coordinate spaziali, e lo indicheremo perciò con la terna ordinata P\,$(x,y,z)$. La visualizzazione dei punti non è semplice, poiché deve rendere conto della profondità e, in certi casi, della prospettiva. Quindi ci affideremo per quanto possibile all'intuizione geometrica, senza legarci troppo ai disegni.

\vspace{7pt}

Dato un insieme di punti distinti, è possibile innanzitutto calcolarne le rispettive distanze. La formula per calcolare la \textbf{distanza tra 2 punti} non è altro che una generalizzazione del teorema di Pitagora, ovvero:
\[\overline{\text{AB}} = \sqrt{(x_A-x_B)^2+(y_A-y_B)^2+(z_A-z_B)^2}\]
Un'oggetto interessante, che ci permetterà poi di ricavare l'equazione di una retta e di un piano, è il vettore. Chiamiamo \textbf{vettore} una freccia che connette due punti. In genere, dati 2 punti A e B, indichiamo con $\vect{\text{AB}}$ il vettore che congiunge A con B, ovvero la freccia che parte da A e arriva a B, con la punta rivolta verso B. Chiamiamo \emph{coda} il punto A e \emph{punta} il punto B. Spesso si associa ad un vettore lo spostamento di un oggetto. Si può pensare per esempio che il punto B sia dato dallo spostamento di A lungo la freccia $\vect{\text{AB}}$. Ovvero:
\[\text{B} = \text{A}+\vect{\text{AB}} \quad \Rightarrow \quad \vect{\text{AB}} = \text{B}-\text{A}\]
Quindi per "costruire" un vettore di coda A e punta B è sufficiente calcolare la differenza tra le coordinate dei punti.
\[(esempio)\]
Notiamo che due vettori con stessa direzione e stesso verso (ovvero sovrapponibili mediante traslazione) sono indistinguibili:
\[(esempio del vettore A(0,0) B(2,4) e del vettore C(6,0) D(8,4)\]
Questa proprietà è detta \textbf{equipollenza}. Due vettori sono equipollenti se hanno stessa direzione e verso. Tale proprietà ci permette di pensare un vettore applicato in un qualunque punto (ovvero con la coda posizionata in tale punto) ed in particolare possiamo sempre pensarlo applicato nell'origine.

\section{Operazioni con i vettori}

Bla bla bla bla (componenti, somma, differenza, prodotto per uno scalare, opposto, vettore nullo, prodotto scalare e vettoriale)

Due vettori $ \vect{a} = (a_x,a_y,a_z) \;,\; \vect{b} = (b_x,b_y,b_z)$ sono paralleli se il rapporto tra i rispettivi coefficienti è sempre lo uguale (nel caso in cui una coordinata del primo vettore sia nulla, dev'esserlo anche nel secondo vettore). In formule:
\[\vect{a} \parallel \vect{b} \quad (a_j\,, b_j \neq 0) \quad \Longleftrightarrow \quad \frac{a_x}{b_x} = \frac{a_y}{b_y} = \frac{a_z}{b_z}\]
\[esempi\]
Due vettori sono perpendicolari se il corrispondente prodotto scalare è zero. In formule scriviamo:
\[ \vect{a} \perp \vect{b} \quad \Longleftrightarrow\quad \vect{a} \times \vect{b} = 0 \quad \text{ovvero:} \quad a_xb_x+a_yb_y+a_zb_z=0\]
\[esempi\]

\section{Retta (forma parametrica)}

Come si definisce una retta nel piano cartesiano? Uno dei modi più semplici per definirla si basa su un approccio pratico: per disegnare una retta qualunque si segna un punto, col righello si imposta l'inclinazione e quindi si traccia la linea. Quindi una retta è definita come l'insieme di tutti i quei punti che si ottengono spostandosi da un punto fissato P\,$(x_P,y_P,z_P)$ lungo la direzione $\vect{d}\,(d_x,d_y,d_z)$. In altre parole:
\[\vect{r} = P+t \, \vect{d} \quad \Rightarrow \quad \begin{cases}
x = x_P+t d_x \\
y = y_P+t d_y \\
z = z_P+t d_z
\end{cases}  \quad (t \in \mathbb{R})\]
Definiamo \textbf{direzione} della retta il vettore $\vect{d}$, o un qualunque vettore ad esso parallelo.
\[(esempio retta punto direzione)\]
\[(esempio retta 2 punti)\]

\'E possibile definire una retta anche come intersezione di due piani. Vedremo in seguito come fare.

\section{Piano (forma parametrica)}

Anche in questo caso non esiste un solo modo di definire un piano. Vediamo innanzitutto il modo più semplice: si costruisce un piano prendendo un punto P\,$(x_P,y_P,z_P)$ e spostandosi lungo due direzioni non parallele $\vect{a}\,(a_x,a_y,a_z)$ e $\vect{b}\,(b_x,b_y,b_z)$. Cioé:
\[\vect{r} = P+u \vect{a}+v \vect{b} \quad \Rightarrow \quad \begin{cases}
x = x_P+u a_x+v b_x \\
y = y_P+u a_y+v b_y \\
z = z_P+u a_z+v b_z
\end{cases} \quad (u \in \mathbb{R}\;,\, v \in \mathbb{R})\]
\[(esempio piano punto direzioni)\]
\[(esempio piano 3 punti)\]

\section{Piano (forma cartesiana)}
Il piano può anche essere definito in un altro modo, con una definizione più tecnica. Dato un punto $P_0\,(x_0,y_0,z_0)$ e un vettore $\vect{v}\,(a,b,c)$ ad esso applicato, è possibile definire un piano come l'insieme di tutti quei punti $P\,(x,y,z)$ tali per cui il vettore $\vect{P_0P}$ sia perpendicolare al vettore $\vect{v}$. Poiché due vettori sono perpendicolari se il corrispondente prodotto scalare è nullo, possiamo scrivere:
\[\vect{P_0P} \cdot \vect{v} =0 \quad \Rightarrow \quad (P-P_0)\cdot \vect{v}=0 \quad \Rightarrow \quad P \cdot \vect{v}-P_0\cdot \vect{v}=0\]
Svolgendo i calcoli otteniamo:
\[ a x+by+cz + \underbrace{(-a x_0-by_0-bz_0)}_{d}=0 \quad \Rightarrow \quad ax+by+cz+d=0\]
Quindi un piano è rappresentato da un'equazione di primo grado in 3 variabili. Notare che i coefficienti delle 3 variabili $(a,b,c)$ determinano anche il \emph{vettore normale} (perpendicolare) al piano stesso, che definisce la \textbf{direzione} stessa del piano. 
\[(esempio piano perpendicolare passante per)\]
\[(esempio piano di direzione e passante per)\]
\[(dato un piano, ricava direzione)\]

\section{Retta (forma cartesiana)}

Una volta visto come si definisce un piano in forma cartesiana, è possibile vedere un secondo modo di definire una retta, in forma cartesiana, come intersezione di 2 piani non paralleli.
\[\begin{cases}
a_1 x+b_1y+c_1 z+d_1 = 0 \\
a_2 x+b_2y+c_2 z+d_2 = 0
\end{cases} \quad \text{tali che } (a_1,b_1,c_1) \text{ non// } (a_2,b_2,c_2)\]
Da notare che il sistema non è determinato perché è costituito da 2 equazioni in 3 incognite. Il restante grado di libertà del sistema è proprio quello necessario a descrivere un insieme infinito di punti allineati, ovvero una retta.

\section{Da forma cartesiana a implicita (e viceversa)}
Come si passa da una forma all'altra? Vediamo come fare attraverso alcuni esempi:
\[(esempi vari)\]

\section{Posizioni reciproche tra rette e piani}

Ricordiamo innanzitutto che due rette nello spazio si possono trovare in 5 diverse configurazioni. Possono essere tra loro:
\begin{itemize}
\item \textbf{Incidenti}: se si intersecano in un punto;
\item \textbf{Coincidenti}: se sono identiche;
\item \textbf{Parallele}: se esiste un piano che le contiene entrambe, ma non sono incidenti;
\item \textbf{Perpendicolari}: se sono incidenti e formano tra esse 4 angoli retti;
\item \textbf{Sghembe}: Se non sono nè parallele nè incidenti;
\end{itemize}

\vspace{7pt}

Per parlare di parallelismo e perpendicolarità tra rette o piani ci si affida semplicemente a considerazioni sui vettori direttori. 

\vspace{7pt}

\textbf{\underline{Rette} :} \;La direzione di una retta, come definita in precedenza, è data semplicemente dal vettore $\vect{d}$, che può essere ricavato dalla forma parametrica della retta stessa. Per verificare dunque se 2 rette sono tra loro parallele, bisogna soltanto verificare che le due direzioni $\vect{d_1}$ e $\vect{d_2}$ siano a loro volta parallele. Per la perpendicolarità non basta invece. \'E necessario anche verificare che le rette assegnate siano incidenti.

\vspace{7pt}

\textbf{\underline{Piani} :} \; La direzione di un piano è data dal corrispondente vettore normale $\vect{n}$. Anche qui dunque non si deve far altro che basarsi sulla nozione stessa di parallelismo/perpendicolarità tra i vettori $\vect{n_1}$ e $\vect{n_2}$ dei due piani.