% (c) 2016 Daniele Zambelli daniele.zambelli@gmail.com
% (c) 2016 Elisabetta Campana

\section{Esercizi}

\subsection{Esercizi dei singoli paragrafi}

\subsubsection*{\numnameref{sec:circ_circtraslata}}

% \begin{esercizio}\label{ese:}
%  Dati il centro e il raggio, calcola l'equazione della circonferenza.
%  \begin{enumeratea}
%   \item  \(C \left (1; \quad 0 \right ); \quad r = 3\)
%    \hfill [\(x^2 + y^2 -2x -8 = 0\)]
%   \item  \(C \left (1; \quad -2 \right ); \quad r = -1\)
%    \hfill [\(x^2 + y^2 -2x +4y +4 = 0\)]
%   \item  \(C \left (-2; \quad -4 \right ); \quad r = -6\)
%    \hfill [\(x^2 + y^2 +4x +8y -16 = 0\)]
%   \item  \(C \left (-6; \quad -4 \right ); \quad r = -1\)
%    \hfill [\(x^2 + y^2 +12x +8y +51 = 0\)]
%   \item  \(C \left (1; \quad 0 \right ); \quad r = -1\)
%    \hfill [\(x^2 + y^2 -2x  = 0\)]
%   \item  \(C \left (-5; \quad -4 \right ); \quad r = -6\)
%    \hfill [\(x^2 + y^2 +10x +8y +5 = 0\)]
%   \item  \(C \left (4; \quad -4 \right ); \quad r = 3\)
%    \hfill [\(x^2 + y^2 -8x +8y +23 = 0\)]
%   \item  \(C \left (-1; \quad -6 \right ); \quad r = -2\)
%    \hfill [\(x^2 + y^2 +2x +12y +33 = 0\)]
% %   \item  \(C \left (-6; \quad 2 \right ); \quad r = -6\)
% %    \hfill [\(x^2 + y^2 +12x -4y +4 = 0\)]
% %   \item  \(C \left (-1; \quad 4 \right ); \quad r = -5\)
% %    \hfill [\(x^2 + y^2 +2x -8y -8 = 0\)]
%  \end{enumeratea}
% \end{esercizio}
\begin{esercizio}\label{ese:}
 Dati il centro e il raggio, calcola l'equazione della circonferenza.
 \begin{enumeratea}
  \item  \(C \left (-5;~-6 \right );~r = 1\)
   \hfill [\(x^2 + y^2 +10x +12y +60 = 0\)]
  \item  \(C \left (-3;~0 \right );~r = 2\)
   \hfill [\(x^2 + y^2 +6x +5 = 0\)]
  \item  \(C \left (2;~3 \right );~r = 1\)
   \hfill [\(x^2 + y^2 -4x -6y +12 = 0\)]
  \item  \(C \left (4;~0 \right );~r = 5\)
   \hfill [\(x^2 + y^2 -8x -9 = 0\)]
  \item  \(C \left (-5;~4 \right );~r = 2\)
   \hfill [\(x^2 + y^2 +10x -8y +37 = 0\)]
  \item  \(C \left (2;~-1 \right );~r = 5\)
   \hfill [\(x^2 + y^2 -4x +2y -20 = 0\)]
  \item  \(C \left (1;~1 \right );~r = 2\)
   \hfill [\(x^2 + y^2 -2x -2y -2 = 0\)]
  \item  \(C \left (3;~0 \right );~r = 5\)
   \hfill [\(x^2 + y^2 -6x -16 = 0\)]
  \item  \(C \left (3;~3 \right );~r = 4\)
   \hfill [\(x^2 + y^2 -6x -6y +2 = 0\)]
  \item  \(C \left (-3;~2 \right );~r = 5\)
   \hfill [\(x^2 + y^2 +6x -4y -12 = 0\)]
 \end{enumeratea}
\end{esercizio}


% \begin{esercizio}\label{ese:}
%  Calcola le coordinate del centro e il raggio della circonferenza data.
%  \begin{enumeratea}
%   \item  \(x^2 + y^2 +4x +10y +20 = 0\)
%    \hfill [\(C \left (-2; \quad -5 \right ); \quad r = 3\)]
%   \item  \(x^2 + y^2 +12x +10y +57 = 0\)
%    \hfill [\(C \left (-6; \quad -5 \right ); \quad r = 2\)]
%   \item  \(x^2 + y^2 +10x +8y +40 = 0\)
%    \hfill [\(C \left (-5; \quad -4 \right ); \quad r = -1\)]
%   \item  \(x^2 + y^2 +12y +35 = 0\)
%    \hfill [\(C \left (0; \quad -6 \right ); \quad r = -1\)]
%   \item  \(x^2 + y^2 +6x +10y +9 = 0\)
%    \hfill [\(C \left (-3; \quad -5 \right ); \quad r = -5\)]
%   \item  \(x^2 + y^2 -2x +10y +22 = 0\)
%    \hfill [\(C \left (1; \quad -5 \right ); \quad r = 2\)]
%   \item  \(x^2 + y^2 +8x -6y -11 = 0\)
%    \hfill [\(C \left (-4; \quad 3 \right ); \quad r = -6\)]
%   \item  \(x^2 + y^2 +10x -8y +32 = 0\)
%    \hfill [\(C \left (-5; \quad 4 \right ); \quad r = 3\)]
% %   \item  \(x^2 + y^2 +2x +2y -2 = 0\)
% %    \hfill [\(C \left (-1; \quad -1 \right ); \quad r = 2\)]
% %   \item  \(x^2 + y^2 -4x +10y -7 = 0\)
% %    \hfill [\(C \left (2; \quad -5 \right ); \quad r = -6\)]
%  \end{enumeratea}
% \end{esercizio}

\begin{esercizio}\label{ese:}
 Calcola le coordinate del centro e il raggio della circonferenza data.
 \begin{enumeratea}
  \item  \(x^2 + y^2 +4x +4y -8 = 0\)
   \hfill [\(C \left (-2;~-2 \right );~r = 4\)]
  \item  \(x^2 + y^2 -2x -4y +1 = 0\)
   \hfill [\(C \left (1;~2 \right );~r = 2\)]
  \item  \(x^2 + y^2 +12x +10y +57 = 0\)
   \hfill [\(C \left (-6;~-5 \right );~r = 2\)]
  \item  \(x^2 + y^2 -4x -4y +7 = 0\)
   \hfill [\(C \left (2;~2 \right );~r = 1\)]
  \item  \(x^2 + y^2 +10x +9 = 0\)
   \hfill [\(C \left (-5;~0 \right );~r = 4\)]
  \item  \(x^2 + y^2 -10x +9 = 0\)
   \hfill [\(C \left (5;~0 \right );~r = 4\)]
  \item  \(x^2 + y^2 +2x -4y -20 = 0\)
   \hfill [\(C \left (-1;~2 \right );~r = 5\)]
  \item  \(x^2 + y^2 -8x +12y +27 = 0\)
   \hfill [\(C \left (4;~-6 \right );~r = 5\)]
  \item  \(x^2 + y^2 -2x +12y +12 = 0\)
   \hfill [\(C \left (1;~-6 \right );~r = 5\)]
  \item  \(x^2 + y^2 -4y +3 = 0\)
   \hfill [\(C \left (0;~2 \right );~r = 1\)]
 \end{enumeratea}
\end{esercizio}


\subsubsection*{\numnameref{sec:circ_equazione}}

% \begin{esercizio}\label{ese:}
%  Dati il centro e un punto, calcola l'equazione della circonferenza.
%  \begin{enumeratea}
%   \item  \(C \left (-5; \quad -5 \right ); \quad P \left (-1; \quad 0 \right )\)
%    \hfill [\(x^2 + y^2 +10x +10y +9 = 0\)]
%   \item  \(C \left (4; \quad 5 \right ); \quad P \left (4; \quad -5 \right )\)
%    \hfill [\(x^2 + y^2 -8x -10y -59 = 0\)]
%   \item  \(C \left (0; \quad 5 \right ); \quad P \left (-3; \quad 2 \right )\)
%    \hfill [\(x^2 + y^2 -10y +7 = 0\)]
%   \item  \(C \left (4; \quad 2 \right ); \quad P \left (-5; \quad 5 \right )\)
%    \hfill [\(x^2 + y^2 -8x -4y -70 = 0\)]
%   \item  \(C \left (-3; \quad -2 \right ); \quad P \left (-4; \quad 0 \right )\)
%    \hfill [\(x^2 + y^2 +6x +4y +8 = 0\)]
%   \item  \(C \left (3; \quad 5 \right ); \quad P \left (-3; \quad -1 \right )\)
%    \hfill [\(x^2 + y^2 -6x -10y -38 = 0\)]
%   \item  \(C \left (0; \quad 4 \right ); \quad P \left (1; \quad -3 \right )\)
%    \hfill [\(x^2 + y^2 -8y -34 = 0\)]
%   \item  \(C \left (0; \quad -1 \right ); \quad P \left (-4; \quad 5 \right )\)
%    \hfill [\(x^2 + y^2 +2y -51 = 0\)]
% %   \item  \(C \left (-2; \quad 5 \right ); \quad P \left (4; \quad -5 \right )\)
% %    \hfill [\(x^2 + y^2 +4x -10y -107 = 0\)]
% %   \item  \(C \left (-3; \quad 4 \right ); \quad P \left (2; \quad 5 \right )\)
% %    \hfill [\(x^2 + y^2 +6x -8y - = 0\)]
%  \end{enumeratea}
% \end{esercizio}
% 
% 
% \begin{esercizio}\label{ese:}
%  Dati gli estremi di un diametro, calcola l'equazione della circonferenza.
%  \begin{enumeratea}
%   \item  \(A \left (-2; \quad 5 \right ); \quad B \left (4; \quad 3 \right )\)
%    \hfill [\(x^2 + y^2 -2x -8y +7 = 0\)]
%   \item  \(A \left (0; \quad -5 \right ); \quad B \left (0; \quad 4 \right )\)
%    \hfill [\(x^2 + y^2 +y -20 = 0\)]
%   \item  \(A \left (-1; \quad 1 \right ); \quad B \left (-6; \quad 2 \right )\)
%    \hfill [\(x^2 + y^2 +7x -3y +8 = 0\)]
%   \item  \(A \left (-6; \quad 3 \right ); \quad B \left (5; \quad 2 \right )\)
%    \hfill [\(x^2 + y^2 +x -5y -24 = 0\)]
%   \item  \(A \left (-1; \quad 1 \right ); \quad B \left (0; \quad 0 \right )\)
%    \hfill [\(x^2 + y^2 +x -y  = 0\)]
%   \item  \(A \left (-4; \quad -6 \right ); \quad B \left (-5; \quad -1 \right 
% )\)
%    \hfill [\(x^2 + y^2 +9x +7y +26 = 0\)]
%   \item  \(A \left (0; \quad -1 \right ); \quad B \left (0; \quad -4 \right )\)
%    \hfill [\(x^2 + y^2 +5y +4 = 0\)]
%   \item  \(A \left (-4; \quad -2 \right ); \quad B \left (-4; \quad 1 \right )\)
%    \hfill [\(x^2 + y^2 +8x +y +14 = 0\)]
% %   \item  \(A \left (3; \quad 0 \right ); \quad B \left (1; \quad 3 \right )\)
% %    \hfill [\(x^2 + y^2 -4x -3y +3 = 0\)]
% %   \item  \(A \left (3; \quad -2 \right ); \quad B \left (4; \quad -4 \right )\)
% %    \hfill [\(x^2 + y^2 -7x +6y +20 = 0\)]
%  \end{enumeratea}
% \end{esercizio}


\begin{esercizio}\label{ese:}
 Dati il centro e un punto, calcola l'equazione della circonferenza.
 \begin{enumeratea}
  \item  \(C \left (1;~0 \right ); \quad P \left (-3;~-2 \right )\)
   \hfill [\(x^2 + y^2 -2x -19 = 0\)]
  \item  \(C \left (2;~5 \right ); \quad P \left (-2;~3 \right )\)
   \hfill [\(x^2 + y^2 -4x -10y +9 = 0\)]
  \item  \(C \left (5;~-2 \right ); \quad P \left (5;~5 \right )\)
   \hfill [\(x^2 + y^2 -10x +4y -20 = 0\)]
  \item  \(C \left (-5;~0 \right ); \quad P \left (-6;~-6 \right )\)
   \hfill [\(x^2 + y^2 +10x -12 = 0\)]
  \item  \(C \left (-5;~2 \right ); \quad P \left (-4;~2 \right )\)
   \hfill [\(x^2 + y^2 +10x -4y +28 = 0\)]
  \item  \(C \left (1;~-2 \right ); \quad P \left (-5;~3 \right )\)
   \hfill [\(x^2 + y^2 -2x +4y -56 = 0\)]
  \item  \(C \left (-4;~-3 \right ); \quad P \left (5;~1 \right )\)
   \hfill [\(x^2 + y^2 +8x +6y -72 = 0\)]
  \item  \(C \left (0;~-1 \right ); \quad P \left (-6;~4 \right )\)
   \hfill [\(x^2 + y^2 +2y -60 = 0\)]
  \item  \(C \left (-1;~4 \right ); \quad P \left (2;~1 \right )\)
   \hfill [\(x^2 + y^2 +2x -8y - = 0\)]
  \item  \(C \left (-6;~-6 \right ); \quad P \left (0;~4 \right )\)
   \hfill [\(x^2 + y^2 +12x +12y -64 = 0\)]
 \end{enumeratea}
\end{esercizio}


\begin{esercizio}\label{ese:}
 Dati gli estremi di un diametro, calcola l'equazione della circonferenza.
 \begin{enumeratea}
  \item  \(A \left (2;~-2 \right );~B \left (1;~-6 \right )\)
   \hfill [\(x^2 + y^2 -3x +8y +14 = 0\)]
  \item  \(A \left (3;~2 \right );~B \left (2;~-5 \right )\)
   \hfill [\(x^2 + y^2 -5x +3y -4 = 0\)]
  \item  \(A \left (4;~-2 \right );~B \left (0;~-4 \right )\)
   \hfill [\(x^2 + y^2 -4x +6y +8 = 0\)]
  \item  \(A \left (-4;~-1 \right );~B \left (5;~3 \right )\)
   \hfill [\(x^2 + y^2 -x -2y -23 = 0\)]
  \item  \(A \left (5;~3 \right );~B \left (5;~2 \right )\)
   \hfill [\(x^2 + y^2 -10x -5y +31 = 0\)]
  \item  \(A \left (-2;~4 \right );~B \left (3;~5 \right )\)
   \hfill [\(x^2 + y^2 -x -9y +14 = 0\)]
  \item  \(A \left (-4;~3 \right );~B \left (-3;~-1 \right )\)
   \hfill [\(x^2 + y^2 +7x -2y +9 = 0\)]
  \item  \(A \left (-2;~-5 \right );~B \left (2;~-5 \right )\)
   \hfill [\(x^2 + y^2 +10y +21 = 0\)]
  \item  \(A \left (2;~0 \right );~B \left (4;~-2 \right )\)
   \hfill [\(x^2 + y^2 -6x +2y +8 = 0\)]
  \item  \(A \left (2;~5 \right );~B \left (-3;~-1 \right )\)
   \hfill [\(x^2 + y^2 +x -4y -11 = 0\)]
 \end{enumeratea}
\end{esercizio}


\begin{esercizio}\label{ese:}
 Calcola l'equazione della circonferenza passante per i tre punti, il suo 
centro e il suo raggio.
 \begin{enumeratea}
  \item  \(A = \left (1;~7 \right );~B = \left (7;~7 \right );~C = \left (0;~6 
\right )\)
   \hfill [\(x^2 + y^2 -8x -6y  = 0; \quad C \left (4;~3 \right );~r = 
\sqrt{25}\)]
  \item  \(A = \left (3;~1 \right );~B = \left (-7;~1 \right );~C = \left 
(-1;~-5 \right )\)
   \hfill [\(x^2 + y^2 +4x -22 = 0; \quad C \left (-2;~0 \right );~r = 
\sqrt{26}\)]
  \item  \(A = \left (3;~-3 \right );~B = \left (3;~-1 \right );~C = \left 
(7;~-1 \right )\)
   \hfill [\(x^2 + y^2 -10x +4y +24 = 0; \quad C \left (5;~-2 \right );~r = 
\sqrt{5}\)]
  \item  \(A = \left (-8;~-7 \right );~B = \left (2;~-7 \right );~C = \left 
(-8;~3 \right )\)\\
   \makebox[\linewidth][r]
   {[\(x^2 + y^2 +6x +4y -37 = 0; \quad C \left (-3;~-2 \right );~r = 
\sqrt{50}\)]}
  \item  \(A = \left (-3;~0 \right );~B = \left (-1;~-10 \right );~C = 
\left 
(-1;~0 \right )\)\\
   \makebox[\linewidth][r]
   {[\(x^2 + y^2 +4x +10y +3 = 0; \quad C \left (-2;~-5 \right );~r = 
\sqrt{26}\)]}
  \item  \(A = \left (1;~-4 \right );~B = \left (0;~-1 \right );~C = \left 
(-3;~-2 \right )\)\\
   \makebox[\linewidth][r]
   {[\(x^2 + y^2 +2x +6y +5 = 0; \quad C \left (-1;~-3 \right );~r = 
\sqrt{5}\)]}
  \item  \(A = \left (-2;~-3 \right );~B = \left (-1;~-6 \right );~C = 
\left 
(1;~-6 \right )\)
   \hfill [\(x^2 + y^2 +8y +11 = 0; \quad C \left (0;~-4 \right );~r = 
\sqrt{5}\)]
  \item  \(A = \left (1;~1 \right );~B = \left (3;~1 \right );~C = \left 
(3;~9 
\right )\)
   \hfill [\(x^2 + y^2 -4x -10y +12 = 0; \quad C \left (2;~5 \right );~r = 
\sqrt{17}\)]
  \item  \(A = \left (-10;~-5 \right );~B = \left (0;~5 \right );~C = \left 
(0;~-5 \right )\)
   \hfill [\(x^2 + y^2 +10x -25 = 0; \quad C \left (-5;~0 \right );~r = 
\sqrt{50}\)]
  \item  \(A = \left (-4;~-2 \right );~B = \left (-4;~-2 \right );~C = 
\left 
(-4;~2 \right )\)
   \hfill [\(x^2 + y^2 +12x +28 = 0; \quad C \left (-6;~0 \right );~r = 
\sqrt{8}\)]
 \end{enumeratea}
\end{esercizio}


% \begin{esercizio}\label{ese:}
%  Calcola l'equazione della circonferenza passante per i tre punti, il suo 
% centro e il suo raggio.
%  \begin{enumeratea}
% \item  \(A = \left (1;~-6 \right ); \quad B = \left (-1;~-5 \right ); 
% \quad 
% C = \left (-3;~-3 \right )\)
% 
% \hfill [\(x^2 + y^2 -7x -3y -48 = 0; \quad 
% C \left (\frac{7}{2};~\frac{3}{2} \right ); \quad 
% r = \sqrt{\frac{125}{2}}\)]
% 
% \item  \(A = \left (4;~5 \right ); \quad B = \left (-2;~1 \right ); \quad 
% C = \left (-3;~-1 \right )\)
% 
% \hfill [\(x^2 + y^2 -\frac{23}{2}x +\frac{33}{4}y -\frac{145}{4} = 0; 
% \quad 
% C \left (\frac{23}{4};~-\frac{33}{8} \right ); \quad 
% r = \sqrt{\frac{5525}{64}}\)]
% 
% \item  \(A = \left (4;~-3 \right ); \quad B = \left (-3;~-5 \right ); 
% \quad 
% C = \left (5;~1 \right )\)
% 
% \hfill [\(x^2 + y^2 +\frac{19}{13}x -\frac{8}{13}y -\frac{425}{13} = 0; 
% \quad
% C \left (-\frac{19}{26};~\frac{4}{13} \right ); \quad
% r = \sqrt{\frac{22525}{676}}\)]
% 
% \item  \(A = \left (-3;~-1 \right ); \quad B = \left (2;~3 \right ); 
% \quad 
% C = \left (-2;~-5 \right )\)
% 
% \hfill [\(x^2 + y^2 -\frac{11}{3}x +\frac{23}{6}y -\frac{103}{6} = 0; 
% \quad
% C \left (\frac{11}{6};~-\frac{23}{12} \right ); \quad
% r = \sqrt{\frac{3485}{144}}\)]
% 
% \item  \(A = \left (-1;~1 \right ); \quad B = \left (-6;~-1 \right ); 
% \quad 
% C = \left (-5;~2 \right )\)
% 
% \hfill [\(x^2 + y^2 +\frac{89}{13}x +\frac{5}{13}y +\frac{58}{13} = 0; 
% \quad
% C \left (-\frac{89}{26};~-\frac{5}{26} \right ); \quad
% r = \sqrt{\frac{2465}{338}}\)]
% 
% \item  \(A = \left (3;~-5 \right ); \quad B = \left (-4;~4 \right ); 
% \quad 
% C = \left (4;~-5 \right )\)
% 
% \hfill [\(x^2 + y^2 -7x -\frac{47}{9}y -\frac{352}{9} = 0; \quad
% C \left (\frac{7}{2};~\frac{47}{18} \right ); \quad
% r = \sqrt{\frac{9425}{162}}\)]
% 
% \item  \(A = \left (4;~4 \right ); \quad B = \left (-6;~0 \right ); \quad 
% C = \left (4;~3 \right )\)
% 
% \hfill [\(x^2 + y^2 +\frac{16}{5}x -7y -\frac{84}{5} = 0; \quad
% C \left (-\frac{8}{5};~\frac{7}{2} \right ); \quad
% r = \sqrt{\frac{3161}{100}}\)]
% 
% \item  \(A = \left (-2;~-5 \right ); \quad B = \left (3;~3 \right ); \quad
% C = \left (3;~0 \right )\)
% 
% \hfill [\(x^2 + y^2 +7x -3y -30 = 0; \quad
% C \left (-\frac{7}{2};~\frac{3}{2} \right ); \quad
% r = \sqrt{ \frac{89}{2}}\)]
% 
% \item  \(A = \left (4;~5 \right ); \quad B = \left (5;~-4 \right ); \quad 
% C = \left (-3;~-6 \right )\)
% 
% \hfill [\(x^2 + y^2 +\frac{18}{37}x +\frac{2}{37}y -\frac{1599}{37} = 0; 
% \quad
% C \left (-\frac{9}{37};~-\frac{1}{37} \right ); \quad
% r = \sqrt{\frac{59245}{1369}}\)]
% 
% \item  \(A = \left (-3;~5 \right ); \quad B = \left (-4;~-6 \right ); 
% \quad 
% C = \left (1;~-2 \right )\)
% 
% \hfill [\(x^2 + y^2 +\frac{445}{51}x +\frac{43}{51}y -\frac{614}{51} = 0; 
% \quad
% C \left (-\frac{445}{102};~-\frac{43}{102} \right ); \quad
% r = \sqrt{\frac{162565}{5202}}\)]
%  \end{enumeratea}
% \end{esercizio}
% 
% \begin{esercizio}\label{ese:}
%  Calcola l'equazione il centro e il raggio della circonferenza passante 
% per 
% i tre punti \(A,~B,~C\).
%  \begin{enumeratea}
%   \item  \(A \left (0; \quad 7 \right ); \quad B \left (-10; \quad 3 
% \right ); 
% \quad C \left (0; \quad 3 \right )\)
% 
%    \hfill [\(x^2 + y^2 +10x -10y +21 = 0; \quad C \left (-5; \quad 5 
% \right ); 
% \quad r = \sqrt{29}\)]
% 
%   \item  \(A \left (-1; \quad -1 \right ); \quad B \left (-5; \quad 1 
% \right ); 
% \quad C \left (-1; \quad 1 \right )\)
% 
%    \hfill [\(x^2 + y^2 +6x +4 = 0; \quad C \left (-3; \quad 0 \right ); 
% \quad r 
% = \sqrt{5}\)]
% 
%   \item  \(A \left (1; \quad 2 \right ); \quad B \left (-11; \quad -8 
% \right ); 
% \quad C \left (-11; \quad 2 \right )\)
% 
%    \hfill [\(x^2 + y^2 +10x +6y -27 = 0; \quad C \left (-5; \quad -3 
% \right ); 
% \quad r = \sqrt{61}\)]
% 
%   \item  \(A \left (-8; \quad -4 \right ); \quad B \left (0; \quad 4 
% \right ); 
% \quad C \left (-4; \quad -8 \right )\)
% 
%    \hfill [\(x^2 + y^2 +4x +4y -32 = 0; \quad C \left (-2; \quad -2 \right 
% ); 
% \quad r = \sqrt{40}\)]
% 
%   \item  \(A \left (3; \quad -2 \right ); \quad B \left (-1; \quad -8 
% \right ); 
% \quad C \left (-1; \quad -2 \right )\)
% 
%    \hfill [\(x^2 + y^2 -2x +10y +13 = 0; \quad C \left (1; \quad -5 \right 
% ); 
% \quad r = \sqrt{13}\)]
% 
%   \item  \(A \left (0; \quad -10 \right ); \quad B \left (-3; \quad -1 
% \right ); 
% \quad C \left (0; \quad 2 \right )\)
% 
%    \hfill [\(x^2 + y^2 -6x +8y -20 = 0; \quad C \left (3; \quad -4 \right 
% ); 
% \quad r = \sqrt{45}\)]
% 
%   \item  \(A \left (5; \quad -8 \right ); \quad B \left (9; \quad -2 
% \right ); 
% \quad C \left (3; \quad -8 \right )\)
% 
%    \hfill [\(x^2 + y^2 -8x +6y - = 0; \quad C \left (4; \quad -3 \right ); 
% \quad 
% r = \sqrt{26}\)]
% 
%   \item  \(A \left (-1; \quad 7 \right ); \quad B \left (3; \quad 7 \right 
% ); 
% \quad C \left (-3; \quad 5 \right )\)
% 
%    \hfill [\(x^2 + y^2 -2x -6y -10 = 0; \quad C \left (1; \quad 3 \right 
% ); 
% \quad r = \sqrt{20}\)]
% 
% %   \item  \(A \left (-8; \quad -6 \right ); \quad B \left (-8; \quad 0 
% \right ); 
% % \quad C \left (-2; \quad -6 \right )\)
% % 
% %    \hfill [\(x^2 + y^2 +10x +6y +16 = 0; \quad C \left (-5; \quad -3 
% \right ); 
% % \quad r = \sqrt{18}\)]
% % 
% %   \item  \(A \left (-5; \quad -4 \right ); \quad B \left (-5; \quad -2 
% \right ); 
% % \quad C \left (-5; \quad -2 \right )\)
% % 
% %    \hfill [\(x^2 + y^2 +12x +6y +43 = 0; \quad C \left (-6; \quad -3 
% \right ); 
% % \quad r = \sqrt{2}\)]
%  \end{enumeratea}
% \end{esercizio}


% \begin{esercizio}\label{ese:}
%  Calcola le intersezioni tra la circonferenza e la retta.
%  \begin{enumeratea}
%   \item  \(c:~x^2 + y^2 +8x +8y +23 = 0; \quad r:~y = -2 x -3\)
%    \hfill [\(\emptyset\)]
%   \item  \(c:~x^2 + y^2 +4x +12y +39 = 0; \quad r:~y = \frac{8}{5} x 
% -\frac{1}{5}\)
%    \hfill [\(\emptyset\)]
%   \item  \(c:~x^2 + y^2 -6x +8y +21 = 0; \quad r:~y = -4 x -15\)
%    \hfill [\(\emptyset\)]
%   \item  \(c:~x^2 + y^2 +8x -8y +31 = 0; \quad r:~y = \frac{9}{8} x 
% -\frac{1}{2}\)
%    \hfill [\(\emptyset\)]
%   \item  \(c:~x^2 + y^2 +4x -8y +11 = 0; \quad r:~y = \frac{4}{3} x 
% +\frac{8}{3}\)
%    \hfill [\(\left (-\frac{29}{225};~\frac{1684}{675} \right );~\left 
% (\frac{1}{9};~\frac{76}{27} \right )\)]
%   \item  \(c:~x^2 + y^2 -8y +7 = 0; \quad r:~y = \frac{3}{5} x \)
%    \hfill [\(\emptyset\)]
%   \item  \(c:~x^2 + y^2 +10x -6y +30 = 0; \quad r:~x = 1\)
%    \hfill [\(\emptyset\)]
%   \item  \(c:~x^2 + y^2 +6x +4y +9 = 0; \quad r:~y = -4\)
%    \hfill [\(\left (-3;~-4 \right )\)]
%  \end{enumeratea}
% \end{esercizio}


\subsubsection*{\numnameref{sec:circ_circrette}}

\begin{esercizio}\label{ese:}
 Calcola le intersezioni tra la circonferenza e la retta.
 \begin{enumeratea}
  \item  \(c:~x^2 + y^2 -10x +4y -32 = 0; \quad r:~y = 3 x +6\)
   \hfill [\(\left (-\frac{14}{5};~-\frac{12}{5} \right );~\left (-1;~3 
\right )\)]
  \item  \(c:~x^2 + y^2 -6x +2y +6 = 0; \quad r:~y = 2\)
   \hfill [\(\emptyset\)]
  \item  \(c:~x^2 + y^2 +2x +12y +12 = 0; \quad r:~y = -2 x -6\)
   \hfill [\(\left (-\frac{12}{5};~-\frac{6}{5} \right );~\left (2;~-10 
\right )\)]
  \item  \(c:~x^2 + y^2 -8x +2y -44 = 0; \quad r:~y = 3 x -4\)
   \hfill [\(\left (-1;~-7 \right );~\left (\frac{18}{5};~\frac{34}{5} 
\right 
)\)]
  \item  \(c:~x^2 + y^2 +10x +6y +16 = 0; \quad r:~y = -\frac{3}{8} x -9\)
   \hfill [\(\left (-8;~-6 \right );~\left 
(-\frac{344}{73};~-\frac{528}{73} 
\right )\)]
  \item  \(c:~x^2 + y^2 -6x +12y +20 = 0; \quad r:~x = -6\)
   \hfill [\(\emptyset\)]
  \item  \(c:~x^2 + y^2 +6x +6y -22 = 0; \quad r:~y = x +8\)
   \hfill [\(\left (-9;~-1 \right );~\left (-5;~3 \right )\)]
  \item  \(c:~x^2 + y^2 -4x -2y -21 = 0; \quad r:~y = x +5\)
   \hfill [\(\left (-3;~2 \right );~\left (1;~6 \right )\)]
  \item  \(c:~x^2 + y^2 +2x +2y -27 = 0; \quad r:~y = \frac{4}{3} x +8\)
   \hfill [\(\left (-\frac{159}{25};~-\frac{12}{25} \right );~\left (-3;~4 
\right )\)]
  \item  \(c:~x^2 + y^2 +6x +2y +6 = 0; \quad r:~x = -6\)
   \hfill [\(\emptyset\)]
  \item  \(c:~x^2 + y^2 +4x -8y -25 = 0; \quad r:~y = 19 x -9\)
   \hfill [\(\left (\frac{64}{181};~-\frac{413}{181} \right );~\left (1;~10 
\right )\)]
  \item  \(c:~x^2 + y^2 +6x -8y +9 = 0; \quad r:~y = -3\)
   \hfill [\(\emptyset\)]
  \item  \(c:~x^2 + y^2 +12x -6y +19 = 0; \quad r:~y = -\frac{1}{7} x -3\)
   \hfill [\(\left (-7;~-2 \right );~\left (-\frac{161}{25};~-\frac{52}{25} 
\right )\)]
  \item  \(c:~x^2 + y^2 -8x +12y +27 = 0; \quad r:~y = -\frac{1}{2} x +5\)
   \hfill [\(\emptyset\)]
  \item  \(c:~x^2 + y^2 -10x -8y +37 = 0; \quad r:~x = 1\)
   \hfill [\(\emptyset\)]
  \item  \(c:~x^2 + y^2 -8x +10y +33 = 0; \quad r:~y = -4 x +1\)
   \hfill [\(\left (\frac{22}{17};~-\frac{71}{17} \right );~\left (2;~-7 
\right )\)]
 \end{enumeratea}
\end{esercizio}

\newpage

\begin{esercizio}\label{ese:}
 Calcola l'equazione della circonferenza e le intersezioni con la retta.
 \begin{enumeratea}
  \item  \(Centro=\left (-3;~3 \right );~r = \sqrt{40}; \quad
           retta: y = x +10\)\\
   \makebox[\linewidth][r]
   {[\(x^2 + y^2 +6x -6y -22 = 0; \quad A = \left (-1;~9 \right );~B = 
\left (-9;~1 \right )\)]}
  \item  \(Centro=\left (-3;~4 \right );~r = \sqrt{13} \quad
           retta: x = -5\)\\
  \makebox[\linewidth][r]
   {[\(x^2 + y^2 +6x -8y +12 = 0; \quad A = \left (-5;~1 \right );~B = 
\left (-5;~7 \right )\)]}
  \item  \(Centro=\left (-2;~2 \right );~r = \sqrt{20} \quad
           retta: y = 
-\frac{1}{3} x +\frac{14}{3}\)\\
  \makebox[\linewidth][r]
   {[\(x^2 + y^2 +4x -4y -12 = 0; \quad A = \left (-4;~6 \right );~B = 
\left (2;~4 \right )\)]}
  \item  \(Centro=\left (3;~2 \right );~r = \sqrt{13} \quad
           retta: y = - x 
+10\)\\
  \makebox[\linewidth][r]
   {[\(x^2 + y^2 -6x -4y  = 0; \quad A = \left (5;~5 \right );~B = \left 
(6;~4 \right )\)]}
  \item  \(Centro=\left (-3;~3 \right );~r = \sqrt{20} \quad
           retta: y = 3 x 
+2\)\\
  \makebox[\linewidth][r]
   {[\(x^2 + y^2 +6x -6y -2 = 0; \quad A = \left (-1;~-1 \right );~B = 
\left (1;~5 \right )\)]}
  \item  \(Centro=\left (-2;~3 \right );~r = \sqrt{25} \quad
           retta: y = -1\)\\
  \makebox[\linewidth][r]
   {[\(x^2 + y^2 +4x -6y -12 = 0; \quad A = \left (1;~-1 \right );~B = 
\left (-5;~-1 \right )\)]}
  \item  \(Centro=\left (-1;~-3 \right );~r = \sqrt{26} \quad
           retta: y = x -6\)\\
  \makebox[\linewidth][r]
   {[\(x^2 + y^2 +2x +6y -16 = 0; \quad A = \left (-2;~-8 \right );~B = 
\left (4;~-2 \right )\)]}
  \item  \(Centro=\left (-6;~0 \right );~r = \sqrt{26} \quad
           retta: y = 
-\frac{1}{5} x -\frac{6}{5}\)\\
  \makebox[\linewidth][r]
   {[\(x^2 + y^2 +12x +10 = 0; \quad A = \left (-11;~1 \right );~B = 
\left (-1;~-1 \right )\)]}
  \item  \(Centro=\left (1;~5 \right );~r = \sqrt{29} \quad
           retta: y = - x 
+13\)\\
  \makebox[\linewidth][r]
   {[\(x^2 + y^2 -2x -10y -3 = 0; \quad A = \left (3;~10 \right );~B = 
\left (6;~7 \right )\)]}
  \item  \(Centro=\left (5;~4 \right );~r = \sqrt{45} \quad
           retta: y = - x +6\)\\
 \makebox[\linewidth][r]
   {[\(x^2 + y^2 -10x -8y -4 = 0; \quad A = \left (8;~-2 \right );~B = 
\left (-1;~7 \right )\)]}
 \end{enumeratea}
\end{esercizio}


% \begin{esercizio}\label{ese:}
%  Calcola le intersezioni tra la circonferenza e la retta.
%  \begin{enumeratea}
%   \item  \(c:~x^2 + y^2 -4x +12y -69 = 0; \quad r:~y = 4\)
%    \hfill [\(-1; \quad 5\)]
%   \item  \(c:~x^2 + y^2 -8x -2y -84 = 0; \quad r:~y = \frac{2}{3} x +6\)
%    \hfill [\(\frac{2 \mp \sqrt{2344}}{39}\)]
%   \item  \(c:~x^2 + y^2 +6x -6y +10 = 0; \quad r:~y = -4 x +1\)
%    \hfill [\(-1; \quad -\frac{5}{17}\)]
%   \item  \(c:~x^2 + y^2 +2x +8y -9 = 0; \quad r:~y = \frac{1}{2} x \)
%    \hfill [\(\frac{-3 \mp \sqrt{54}}{5}\)]
%   \item  \(c:~x^2 + y^2 -6x +4y -21 = 0; \quad r:~y = -2 x -3\)
%    \hfill [\(-2; \quad \frac{12}{5}\)]
%   \item  \(c:~x^2 + y^2 -2x -2y -23 = 0; \quad r:~y = - x +1\)
%    \hfill [\(-3; \quad 4\)]
%   \item  \(c:~x^2 + y^2 -8x +2y -12 = 0; \quad r:~y = -3 x -6\)
%    \hfill [\(-\frac{6}{5}; \quad -1\)]
%   \item  \(c:~x^2 + y^2 -10x +2y -75 = 0; \quad r:~y = -\frac{1}{5} x -1\)
%    \hfill [\(\frac{5 \mp \sqrt{2001}}{26}\)]
% %   \item  \(c:~x^2 + y^2 +10x -8y -68 = 0; \quad r:~y = \frac{5}{2} x 
% -1\)
% %    \hfill [\(\frac{15 \mp \sqrt{7069}}{58}\)]
% %   \item  \(c:~x^2 + y^2 +10x +12y -20 = 0; \quad r:~y = -\frac{15}{4} x 
% +9\)
% %    \hfill [\(\frac{205 \mp 283807i}{964}\)]
%  \end{enumeratea}
% \end{esercizio}
% 
% 
% \begin{esercizio}\label{ese:}
%  Calcola le intersezioni tra la circonferenza e la retta.
%  \begin{enumeratea}
%   \item  \(c:~x^2 + y^2 +2x +8y +17 = 0; \quad r:~y = -4 x -15\)
%    \hfill [\(\frac{-45 \mp 7i}{17}\)]
%   \item  \(c:~x^2 + y^2 +2x -6y +9 = 0; \quad r:~y = x +8\)
%    \hfill [\(\frac{-6 \mp 14i}{2}\)]
%   \item  \(c:~x^2 + y^2 -6x +2y +10 = 0; \quad r:~y = -\frac{5}{2} x 
% -\frac{27}{2}\)
%    \hfill [\(\frac{-113 \mp 25569i}{116}\)]
%   \item  \(c:~x^2 + y^2 +6x +10y +18 = 0; \quad r:~y = \frac{1}{2} x 
% -\frac{11}{2}\)
%    \hfill [\(\frac{-11 \mp \sqrt{391}}{20}\)]
%   \item  \(c:~x^2 + y^2 +2y -8 = 0; \quad r:~y = - x -2\)
%    \hfill [\(\frac{-1 \mp \sqrt{17}}{2}\)]
%   \item  \(c:~x^2 + y^2 +8x +2y -19 = 0; \quad r:~y = 4\)
%    \hfill [\(-4 \mp \sqrt{11}\)]
%   \item  \(c:~x^2 + y^2 +12x +12y +71 = 0; \quad r:~y = -\frac{7}{8} x 
% -\frac{13}{4}\)
%    \hfill [\(\frac{-115 \mp 294587i}{3616}\)]
%   \item  \(c:~x^2 + y^2 +10y  = 0; \quad r:~y = -\frac{3}{2} x 
% -\frac{9}{2}\)
%    \hfill [\(\frac{3 \mp \sqrt{2583}}{52}\)]
% %   \item  \(c:~x^2 + y^2 +2x +10y + = 0; \quad r:~y = x +5\)
% %    \hfill [\(\frac{-11 \mp 31i}{2}\)]
% %   \item  \(c:~x^2 + y^2 +10x -6y +18 = 0; \quad r:~y = 2 x +6\)
% %    \hfill [\(\frac{-11 \mp \sqrt{31}}{5}\)]
%  \end{enumeratea}
% \end{esercizio}
% 
% 
% \begin{esercizio}\label{ese:}
%  Calcola l'equazione della circonferenza e le intersezioni con la retta.
%  \begin{enumeratea}
%   \item  \(Centro=\left (-2; \quad 2 \right ); \quad r = \sqrt{26}; \quad 
% retta: y = - x +6\)
% 
%    \hfill [\(x^2 + y^2 +4x -4y -18 = 0; \quad A = \left (-1; \quad 7 
% \right ); 
% \quad B = \left (3; \quad 3 \right )\)]
% 
%   \item  \(Centro=\left (-1; \quad -2 \right ); \quad r = \sqrt{61}; 
% \quad 
% retta: y = - x +8\)
% 
%    \hfill [\(x^2 + y^2 +2x +4y -56 = 0; \quad A = \left (4; \quad 4 \right 
% ); 
% \quad B = \left (5; \quad 3 \right )\)]
% 
%   \item  \(Centro=\left (-5; \quad 4 \right ); \quad r = \sqrt{41}; \quad 
% retta: y = 0\)
% 
%    \hfill [\(x^2 + y^2 +10x -8y  = 0; \quad A = \left (0; \quad 0 \right 
% ); 
% \quad B = \left (-10; \quad 0 \right )\)]
% 
%   \item  \(Centro=\left (-1; \quad -2 \right ); \quad r = \sqrt{5}; \quad 
% retta: y = -2 x -4\)
% 
%    \hfill [\(x^2 + y^2 +2x +4y  = 0; \quad A = \left (-2; \quad 0 \right 
% ); 
% \quad B = \left (0; \quad -4 \right )\)]
% 
%   \item  \(Centro=\left (5; \quad -2 \right ); \quad r = \sqrt{29}; \quad 
% retta: y = \frac{3}{7} x -\frac{58}{7}\)
% 
%    \hfill [\(x^2 + y^2 -10x +4y  = 0; \quad A = \left (3; \quad -7 \right 
% ); 
% \quad B = \left (10; \quad -4 \right )\)]
% 
%   \item  \(Centro=\left (0; \quad -3 \right ); \quad r = \sqrt{17}; \quad 
% retta: y = -7\)
% 
%    \hfill [\(x^2 + y^2 +6y -8 = 0; \quad A = \left (1; \quad -7 \right ); 
% \quad 
% B = \left (-1; \quad -7 \right )\)]
% 
%   \item  \(Centro=\left (-3; \quad -3 \right ); \quad r = \sqrt{2}; \quad 
% retta: x = -4\)
% 
%    \hfill [\(x^2 + y^2 +6x +6y +16 = 0; \quad A = \left (-4; \quad -4 
% \right ); 
% \quad B = \left (-4; \quad -2 \right )\)]
% 
%   \item  \(Centro=\left (2; \quad -3 \right ); \quad r = \sqrt{13}; \quad 
% retta: y = \frac{2}{3} x -\frac{13}{3}\)
% 
%    \hfill [\(x^2 + y^2 -4x +6y  = 0; \quad A = \left (-1; \quad -5 \right 
% ); 
% \quad B = \left (5; \quad -1 \right )\)]
% 
% %   \item  \(Centro=\left (3; \quad 2 \right ); \quad r = \sqrt{10}; 
% \quad 
% % retta: y = \frac{1}{2} x -2\)
% % 
% %    \hfill [\(x^2 + y^2 -6x -4y +3 = 0; \quad A = \left (2; \quad -1 
% \right ); 
% % \quad B = \left (6; \quad 1 \right )\)]
% % 
% %   \item  \(Centro=\left (-2; \quad -6 \right ); \quad r = \sqrt{25}; 
% \quad 
% % retta: y = -\frac{3}{4} x -\frac{15}{2}\)
% % 
% %    \hfill [\(x^2 + y^2 +4x +12y +15 = 0; \quad A = \left (2; \quad -9 
% \right ); 
% % \quad B = \left (-6; \quad -3 \right )\)]
%  \end{enumeratea}
% \end{esercizio}
% 
% 
% \begin{esercizio}\label{ese:}
%  Calcola l'equazione della circonferenza e le intersezioni con la retta.
%  \begin{enumeratea}
%   \item  \(C=\left (\frac{81}{14}; \quad -\frac{31}{14} \right ); \quad 
% r = \sqrt{\frac{1885}{98}}; \quad retta:~y = -\frac{1}{8} x +\frac{1}{4}\)
% 
% \hfill [\(x^2 + y^2 -\frac{81}{7}x +\frac{31}{7}y +\frac{134}{7} = 0; 
% \quad 
% A = \left (10;~-1 \right ); \quad B = \left (2;~0 \right )\)]
% 
% \item  \(C=\left (-\frac{25}{4}; \quad -\frac{1}{8} \right ); \quad 
% r = \sqrt{\frac{9685}{64}}; \quad retta:~y = \frac{10}{7} x 
% +\frac{43}{7}\)
% 
% \hfill [\(x^2 + y^2 +\frac{25}{2}x +\frac{1}{4}y -\frac{449}{4} = 0; 
% \quad 
% A = \left (-12;~-11 \right ); \quad B = \left (2;~9 \right )\)]
% 
% \item  \(C=\left (-\frac{94}{57}; \quad -\frac{31}{114} \right ); \quad 
% r = \sqrt{\frac{761345}{12996}}; \quad retta:~y = \frac{7}{4} x 
% -\frac{21}{2}\)
% 
% \hfill [\(x^2 + y^2 +\frac{188}{57}x +\frac{31}{57}y -\frac{1060}{19} = 
% 0; 
% \quad A = \left (2;~-7 \right ); \quad B = \left (6;~0 \right )\)]
% 
% \item  \(C=\left (\frac{109}{2}; \quad 64 \right ); \quad 
% r = \sqrt{\frac{32513}{4}}; \quad retta:~y = -\frac{5}{6} x 
% -\frac{15}{2}\)
% 
% \hfill [\(x^2 + y^2 -109x -128y -1062 = 0; \quad 
% A = \left (3;~-10 \right ); \quad B = \left (-9;~0 \right )\)]
% 
% \item  \(C=\left (\frac{20}{27}; \quad -\frac{64}{27} \right ); \quad 
% r = \sqrt{\frac{108770}{729}}; \quad retta:~y = -\frac{7}{10} x 
% -\frac{13}{10}\)
% 
% \hfill [\(x^2 + y^2 -\frac{40}{27}x +\frac{128}{27}y -\frac{3862}{27} = 
% 0; 
% \quad A = \left (11;~-9 \right ); \quad B = \left (-9;~5 \right )\)]
% 
% \item  \(C=\left (\frac{41}{4}; \quad \frac{3}{4} \right ); \quad 
% r = \sqrt{\frac{697}{8}}; \quad retta:~y = x +1\)
% 
% \hfill [\(x^2 + y^2 -\frac{41}{2}x -\frac{3}{2}y +\frac{37}{2} = 0; \quad 
% A = \left (1;~2 \right ); \quad B = \left (9;~10 \right )\)]
% 
% \item  \(C=\left (\frac{79}{2}; \quad \frac{31}{2} \right ); \quad 
% r = \sqrt{\frac{2465}{2}}; \quad retta:~y = -3 x +24\)
% 
% \hfill [\(x^2 + y^2 -79x -31y +568 = 0; \quad 
% A = \left (5;~9 \right ); \quad B = \left (8;~0 \right )\)]
%  \end{enumeratea}
% \end{esercizio}
% 
% 
% \begin{esercizio}\label{ese:}
%  Data una circonferenza e un suo punto calcola l'equazione della tangente.
%  \begin{enumeratea}
%   \item  \(c:~x^2 + y^2 -8x +10y +24 = 0; \quad P \left (5; \quad -1 
% \right )\)
%    \hfill [\(y = -\frac{1}{4} x +\frac{1}{4}\)]
%   \item  \(c:~x^2 + y^2 -10x -10y -2 = 0; \quad P \left (1; \quad -1 
% \right )\)
%    \hfill [\(y = -\frac{2}{3} x -\frac{1}{3}\)]
%   \item  \(c:~x^2 + y^2 +10y -49 = 0; \quad P \left (5; \quad 2 \right )\)
%    \hfill [\(y = -\frac{5}{7} x +\frac{39}{7}\)]
%   \item  \(c:~x^2 + y^2 +8x +12y +15 = 0; \quad P \left (2; \quad -5 
% \right )\)
%    \hfill [\(y = -6 x +7\)]
%   \item  \(c:~x^2 + y^2 +2x +8y  = 0; \quad P \left (-5; \quad -5 \right 
% )\)
%    \hfill [\(y = -4 x -25\)]
%   \item  \(c:~x^2 + y^2 +4x +4y -33 = 0; \quad P \left (3; \quad -6 \right 
% )\)
%    \hfill [\(y = \frac{5}{4} x -\frac{39}{4}\)]
%   \item  \(c:~x^2 + y^2 +12x +4y -18 = 0; \quad P \left (-3; \quad 5 
% \right )\)
%    \hfill [\(y = -\frac{3}{7} x +\frac{26}{7}\)]
%   \item  \(c:~x^2 + y^2 -10x -10y +30 = 0; \quad P \left (1; \quad 3 
% \right )\)
%    \hfill [\(y = -2 x +5\)]
% %   \item  \(c:~x^2 + y^2 +4x -4y -33 = 0; \quad P \left (2; \quad -3 
% \right )\)
% %    \hfill [\(y = \frac{4}{5} x -\frac{23}{5}\)]
% %   \item  \(c:~x^2 + y^2 -10x +2y -80 = 0; \quad P \left (-4; \quad 4 
% \right )\)
% %    \hfill [\(y = \frac{9}{5} x +\frac{56}{5}\)]
%  \end{enumeratea}
% \end{esercizio}


\begin{esercizio}\label{ese:}
 Data una circonferenza e un suo punto calcola l'equazione della tangente.
 \begin{enumeratea}
  \item  \(c:~x^2 + y^2 +10x +8y +39 = 0; \quad P \left (-4;~-5 \right )\)
   \hfill [\(y = x -1\)]
  \item  \(c:~x^2 + y^2 +2x -4y -29 = 0; \quad P \left (2;~-3 \right )\)
   \hfill [\(y = \frac{3}{5} x -\frac{21}{5}\)]
  \item  \(c:~x^2 + y^2 -10x -4y -12 = 0; \quad P \left (1;~-3 \right )\)
   \hfill [\(y = -\frac{4}{5} x -\frac{11}{5}\)]
  \item  \(c:~x^2 + y^2 +10x +4y -24 = 0; \quad P \left (-3;~5 \right )\)
   \hfill [\(y = -\frac{2}{7} x +\frac{29}{7}\)]
  \item  \(c:~x^2 + y^2 +8x +10y +40 = 0; \quad P \left (-4;~-4 \right )\)
   \hfill [\(y = -4\)]
  \item  \(c:~x^2 + y^2 +2x -8y -3 = 0; \quad P \left (1;~0 \right )\)
   \hfill [\(y = \frac{1}{2} x -\frac{1}{2}\)]
  \item  \(c:~x^2 + y^2 -8x +6y -57 = 0; \quad P \left (-5;~-4 \right )\)
   \hfill [\(y = -9 x -49\)]
  \item  \(c:~x^2 + y^2 +6x +2y -75 = 0; \quad P \left (4;~5 \right )\)
   \hfill [\(y = -\frac{7}{6} x +\frac{29}{3}\)]
  \item  \(c:~x^2 + y^2 -4y -37 = 0; \quad P \left (4;~-3 \right )\)
   \hfill [\(y = \frac{4}{5} x -\frac{31}{5}\)]
  \item  \(c:~x^2 + y^2 +4x -4y -17 = 0; \quad P \left (2;~-1 \right )\)
   \hfill [\(y = \frac{4}{3} x -\frac{11}{3}\)]
 \end{enumeratea}
\end{esercizio}


\begin{esercizio}\label{ese:}
 Data una circonferenza e un punto P calcola le equazioni delle tangenti 
passanti per P.
 \begin{enumeratea}
  \item  \(c:~x^2 + y^2 +8x +2y +4 = 0; \quad P \left (\frac{1}{3};~-1 
\right )\)
   \hfill [\(y = -\frac{3}{2} x -\frac{1}{2}; y = \frac{3}{2} x 
-\frac{3}{2}\)]
  \item  \(c:~x^2 + y^2 +10x +6y  = 0; \quad P \left (-\frac{49}{3};~-3 
\right )\)
   \hfill [\(y = -\frac{3}{5} x -\frac{64}{5}; y = \frac{3}{5} x 
+\frac{34}{5}\)]
  \item  \(c:~x^2 + y^2 +4x -30 = 0; \quad P \left (-10;~2 \right )\)
   \hfill [\(y = -\frac{5}{3} x -\frac{44}{3}; y = \frac{3}{5} x +8\)]
  \item  \(c:~x^2 + y^2 -10x +6y +24 = 0; \quad P \left (5;~-13 \right )\)
   \hfill [\(y = 3 x -28; y = -3 x +2\)]
  \item  \(c:~x^2 + y^2 +6x +2y -15 = 0; \quad P \left 
(-\frac{46}{7};~\frac{18}{7} 
\right )\)
   \hfill [\(y = \frac{4}{3} x +\frac{34}{3}; y = \frac{3}{4} x 
+\frac{15}{2}\)]
  \item  \(c:~x^2 + y^2 +10x +12y +44 = 0; \quad P \left 
(-\frac{8}{5};~-\frac{47}{5} \right )\)
   \hfill [\(y = \frac{1}{4} x -9; y = 4 x -3\)]
  \item  \(c:~x^2 + y^2 -6x +10y +17 = 0; \quad P \left (8;~-2 \right )\)
   \hfill [\(y = 4 x -34; y = -\frac{1}{4} x \)]
  \item  \(c:~x^2 + y^2 -6x +4y -7 = 0; \quad P \left (9;~0 \right )\)
   \hfill [\(y = 2 x -18; y = -\frac{1}{2} x +\frac{9}{2}\)]
  \item  \(c:~x^2 + y^2 +8x +2y -8 = 0; \quad P \left (-\frac{41}{4};~-1 
\right )\)
   \hfill [\(y = \frac{4}{3} x +\frac{38}{3}; y = -\frac{4}{3} x 
-\frac{44}{3}\)]
  \item  \(c:~x^2 + y^2 -2x -8y -3 = 0; \quad P \left 
(-\frac{7}{3};~\frac{2}{3} 
\right )\)
   \hfill [\(y = -\frac{1}{2} x -\frac{1}{2}; y = -2 x -4\)]
 \end{enumeratea}
\end{esercizio}

\newpage
\begin{esercizio}\label{ese:}
 Calcola le intersezioni delle due circonferenze.
 \begin{enumeratea}
  \item  \(c0:~x^2 + y^2 +12x +8y +27 = 0; \quad c1:~x^2 + y^2 +12x +4y -1 = 
0\)\\
   \makebox[\linewidth][r]
   {[\(ar:~y = -7;~A \punto{-10}{-7};~B \punto{-2}{-7}\)]}
  \item  \(c0:~x^2 + y^2 +8x -4y -17 = 0; \quad c1:~x^2 + y^2 +4x -2y -3 = 
0\)\\
   \makebox[\linewidth][r]
   {[\(ar:~y = 2x -7;~\emptyset\)]}
  \item  \(c0:~x^2 + y^2 +12x -8y +18 = 0; \quad c1:~x^2 + y^2 +4x -46 = 
0\)\\
   \makebox[\linewidth][r]
   {[\(ar:~y = x +8;~A \punto{-1}{7};~B \punto{-9}{-1}\)]}
  \item  \(c0:~x^2 + y^2 -4x -10y -3 = 0; \quad c1:~x^2 + y^2 -4x -6y -39 = 
0\)\\
   \makebox[\linewidth][r]
   {[\(ar:~y = 9;~A \punto{6}{9};~B \punto{-2}{9}\)]}
  \item  \(c0:~x^2 + y^2 -6x -10y +21 = 0; \quad c1:~x^2 + y^2 -6x +12y +35 
= 0\)\\
   \makebox[\linewidth][r]
   {[\(ar:~y = -\frac{7}{11};~\emptyset\)]}
  \item  \(c0:~x^2 + y^2 +12x +4y +22 = 0; \quad c1:~x^2 + y^2 +6x +4y +4 = 
0\)\\
   \makebox[\linewidth][r]
   {[\(ar:~x = -3;~A \punto{-3}{1};~B \punto{-3}{1}\)]}
  \item  \(c0:~x^2 + y^2 +2x -4y -36 = 0; \quad c1:~x^2 + y^2 +2x -28 = 
0\)\\
   \makebox[\linewidth][r]
   {[\(ar:~y = -2;~A \punto{-6}{-2};~B \punto{4}{-2}\)]}
  \item  \(c0:~x^2 + y^2 +8x +10y +23 = 0; \quad c1:~x^2 + y^2 -6x +2y  = 
0\)\\
   \makebox[\linewidth][r]
   {[\(ar:y = -\frac{7}{4}x -\frac{23}{8};~\emptyset\)]}
  \item  \(c0:~x^2 + y^2 +6x -10y -7 = 0; \quad c1:~x^2 + y^2 +6x +10y -27 = 
0\)\\
   \makebox[\linewidth][r]
   {[\(ar:~y = 1;~A \punto{-8}{1};~B \punto{2}{1}\)]}
  \item  \(c0:~x^2 + y^2 -4x -8y +15 = 0; \quad c1:~x^2 + y^2 -8x -9 = 0\)\\
   \makebox[\linewidth][r]
   {[\(ar:~y = \frac{1}{2} x +3;~A \punto{0}{3};~B \punto{4}{5}\)]}
 \end{enumeratea}
\end{esercizio}


\begin{esercizio}\label{ese:}
 Calcola le intersezioni delle due circonferenze.
 \begin{enumeratea}
  \item  \(c0:~x^2 + y^2 +6x -43 = 0; \quad c1:~x^2 + y^2 -8x -14y -225 = 
0\)\\
   \makebox[\linewidth][r]
   {[\(ar:~y = - x -13;~A \punto{-9}{-4};~B \punto{-7}{-6}\)]}
  \item  \(c0:~x^2 + y^2 +4x -8y -14 = 0; \quad c1:~x^2 + y^2 +10x -14y +10 
= 0\)\\
   \makebox[\linewidth][r]
   {[\(ar:~y = x +4;~A \punto{3}{7};~B \punto{-5}{-1}\)]}
  \item  \(c0:~x^2 + y^2 +12x +12y +62 = 0; \quad c1:~x^2 + y^2 +12x +12y 
+62=0\)\\
   \makebox[\linewidth][r]
   {[\(ar:~y = x -2;~A \punto{-7}{-9};~B \punto{-3}{-5}\)]}
  \item  \(c0:~x^2 + y^2 -4y -46 = 0; \quad c1:~x^2 + y^2 +4x -54 = 0\)\\
   \makebox[\linewidth][r]
   {[\(ar:~y = - x +2;~A \punto{-5}{7};~B \punto{5}{-3}\)]}
  \item  \(c0:~x^2 + y^2 +12x -2y +32 = 0; \quad c1:~x^2 + y^2 -6x +16y -112 
= 0\)\\
   \makebox[\linewidth][r]
   {[\(ar:~y = x +8;~A \punto{-5}{3};~B \punto{-8}{0}\)]}
  \item  \(c0:~x^2 + y^2 +8x +6 = 0; \quad c1:~x^2 + y^2 +10x +y +9 = 0\)\\
   \makebox[\linewidth][r]
   {[\(ar:~y = -2 x -3;~A \punto{-1}{-1};~B \punto{-3}{3}\)]}
  \item  \(c0:~x^2 + y^2 +12x +23 = 0; \quad c1:~x^2 + y^2 -10x -22y -131 = 
0\)\\
   \makebox[\linewidth][r]
   {[\(ar:~y = - x -7;~A \punto{-4}{-3};~B \punto{-9}{2}\)]}
  \item  \(c0:~x^2 + y^2 +8x +2y -12 = 0; \quad c1:~x^2 + y^2 +8x +4y -6 = 
0\)\\
   \makebox[\linewidth][r]
   {[\(ar:~y = -3;~A \punto{1}{-3};~B \punto{-9}{-3}\)]}
  \item  \(c0:~x^2 + y^2 +4x +4y -37 = 0; \quad c1:~x^2 + y^2 +4x -8y -25 = 
0\)\\
   \makebox[\linewidth][r]
   {[\(ar:~y = 1;~A \punto{-8}{1};~B \punto{4}{1}\)]}
 \end{enumeratea}
\end{esercizio}

\begin{esercizio}\label{ese:}
 Calcola l'asse radicale e le intersezioni delle due circonferenze.
 \begin{enumeratea}
  \item  \(c0:~x^2 + y^2 -12x +6y -253 = 0; \quad c1:~x^2 + y^2 -18x -355 = 
0\)\\
  \makebox[\linewidth][r]
   {[\(ar:~y = - x -17;~A \left (-11;~-6 \right );~B \left (-20;~3 
\right )\)]}
  \item  \(c0:~x^2 + y^2 +2x +12y -109 = 0; \quad c1:~x^2 + y^2 +8x 
-\frac{6}{5}y 
-7 = 0\)\\
  \makebox[\linewidth][r]
   {[\(ar:~y = \frac{5}{11} x +\frac{85}{11};~A \left (-6;~5 \right )\)]}
  \item  \(c0:~x^2 + y^2 +10x -10y +24 = 0; \quad c1:~x^2 + y^2 -8x -10y +28 
= 
0\)\\
  \makebox[\linewidth][r]
   {[\(ar:~x = \frac{2}{9};~\emptyset\)]}
%   \item  \(c0:~x^2 + y^2 -18x +2y +41 = 0; \quad c1:~x^2 + y^2 -14x +12y 
-5 = 
% 0\)\\
% .
%    \hfill [\(ar:~y = -\frac{2}{5} x +\frac{23}{5};~A \left (4;~3 \right 
% );~B % \left (-\frac{45}{29};~\frac{446}{29} \right )\)]
  \item  \(c0:~x^2 + y^2 +18x +10y -54 = 0; \quad c1:~x^2 + y^2 +14x +12y 
-40 = 
0\)\\
  \makebox[\linewidth][r]
   {[\(ar:~y = 2 x -7;~A \left (3;~-1 \right );~B \left (-17;~-5 \right 
)\)]}
  \item  \(c0:~x^2 + y^2 -6x +10y +17 = 0; \quad c1:~x^2 + y^2 +4x +50y +17 
= 
0\)\\
  \makebox[\linewidth][r]
   {[\(ar:~y = -\frac{1}{4} x ;~A \left (4;~-1 \right )\)]}
  \item  \(c0:~x^2 + y^2 +6y +4 = 0; \quad c1:~x^2 + y^2 +2x -4y  = 0\)\\
  \makebox[\linewidth][r]
   {[\(ar:~y = \frac{1}{5} x -\frac{2}{5};~\emptyset\)]}
  \item  \(c0:~x^2 + y^2 -6x +8y -75 = 0; \quad c1:~x^2 + y^2 -8x +16y +39 = 
0\)\\
  \makebox[\linewidth][r]
   {[\(ar:~y = \frac{1}{4} x -\frac{57}{4};~A \left (9;~-12 \right );~B 
\left (-\frac{236}{17};~\frac{25}{17} \right )\)]}
  \item  \(c0:~x^2 + y^2 -10x -6y +24 = 0; \quad c1:~x^2 + y^2 +8x -10y +23 
= 
0\)\\
  \makebox[\linewidth][r]
   {[\(ar:~y = \frac{9}{2} x -\frac{1}{4};~\emptyset\)]}
  \item  \(c0:~x^2 + y^2 -10x -16y -144 = 0; \quad c1:~x^2 + y^2 +10x +24y 
+116 = 
0\)\\
  \makebox[\linewidth][r]
   {[\(ar:~y = -\frac{1}{2} x -\frac{13}{2};~A \left (-3;~-5 \right 
);~B \left (-\frac{31}{5};~-\frac{3}{5} \right )\)]}
  \item  \(c0:~x^2 + y^2 +6x +2y -6 = 0; \quad c1:~x^2 + y^2 +8x +2y -8 = 
0\)\\
  \makebox[\linewidth][r]
   {[\(ar:~x = 1;~A \left (1;~-1 \right )\)]}
%   \item  \(c0:~x^2 + y^2 +6x +14y -27 = 0; \quad c1:~x^2 + y^2 +4x -22y 
% -195 = 
% 0\)\\
% .
%    \hfill [\(ar:~y = -\frac{1}{18} x -\frac{14}{3};~A \left (6;~-5 \right 
% );~B % \left (-\frac{261}{65};~-\frac{762}{65} \right )\)]
%   \item  \(c0:~x^2 + y^2 -2x +8y -96 = 0; \quad c1:~x^2 + y^2 -12x +22y 
% -212 = 
% 0\)\\
% .
%    \hfill [\(ar:~y = \frac{5}{7} x +\frac{58}{7};~A \left (-6;~4 \right 
% );~B 
% \left (\frac{193}{37};~-\frac{159}{37} \right )\)]
%   \item  \(c0:~x^2 + y^2 -12x -12y -13 = 0; \quad c1:~x^2 + y^2 +14y -81 = 
% 0\)\\
% .
%    \hfill [\(ar:~y = -\frac{6}{13} x +\frac{34}{13};~A \left (-3;~4 
% \right 
% );~B \left (-\frac{88}{41};~\frac{423}{41} \right )\)]
%   \item  \(c0:~x^2 + y^2 +8x +12y +15 = 0; \quad c1:~x^2 + y^2 -4x +16y 
% +59 = 
% 0\)\\
% .
%    \hfill [\(ar:~y = 3 x -11;~A \left (2;~-5 \right );~B \left 
% (-\frac{52}{5};~\frac{1}{5} \right )\)]
  \item  \(c0:~x^2 + y^2 +10x +10y -75 = 0; \quad c1:~x^2 + y^2 +8x +9y -65 
= 
0\)\\
  \makebox[\linewidth][r]
   {[\(ar:~y = -2 x +10;~A \left (5;~0 \right )\)]}
%   \item  \(c0:~x^2 + y^2 +18x -14y -328 = 0; \quad c1:~x^2 + y^2 -16x +28 
% = 0\)\\
% .
%    \hfill [\(ar:~y = \frac{17}{7} x -\frac{178}{7};~A \left (8;~-6 \right 
% );~B % \left (\frac{720}{169};~\frac{2066}{169} \right )\)]
% %  \end{enumeratea}
% % \end{esercizio}
% % 
% % 
% % \begin{esercizio}\label{ese:}
% %  Calcola l'asse radicale e le intersezioni delle due circonferenze.
% %  \begin{enumeratea}
%   \item  \(c0:~x^2 + y^2 +10x +2y -27 = 0; \quad c1:~x^2 + y^2 +4x 
% +\frac{26}{7}y 
% -\frac{69}{7} = 0\)\\
% .
%    \hfill [\(ar:~y = \frac{7}{2} x -10;~A \left (2;~-3 \right )\)]
%   \item  \(c0:~x^2 + y^2 +2x -8y +4 = 0; \quad c1:~x^2 + y^2 +12x -23y +9 
% = 0\)\\
% .
%    \hfill [\(ar:~y = \frac{2}{3} x +\frac{1}{3};~A \left (1;~1 \right )\)]
%   \item  \(c0:~x^2 + y^2 -2x -8y +7 = 0; \quad c1:~x^2 + y^2 +12x 
% -\frac{10}{3}y 
% +21 
% = 0\)\\
% .
%    \hfill [\(ar:~y = -3 x -3;~A \left (-2;~3 \right )\)]
%   \item  \(c0:~x^2 + y^2 +2x +4y -5 = 0; \quad c1:~x^2 + y^2 -6x -20y +19 
% = 0\)\\
% .
%    \hfill [\(ar:~y = -\frac{1}{3} x +1;~A \left (0;~1 \right )\)]
%   \item  \(c0:~x^2 + y^2 -4x +10y -56 = 0; \quad c1:~x^2 + y^2 +2x +37y 
% -188 = 
% 0\)\\
% .
%    \hfill [\(ar:~y = -\frac{2}{9} x +\frac{44}{9};~A \left (4;~4 \right 
% )\)]
%   \item  \(c0:~x^2 + y^2 -4x +4y -37 = 0; \quad c1:~x^2 + y^2 +6y -11 = 
% 0\)\\
% .
%    \hfill [\(ar:~y = -2 x -13;~A \left (-4;~-5 \right )\)]
 \end{enumeratea}
\end{esercizio}


% \begin{esercizio}\label{ese:}
%  Calcola l'asse radicale e le intersezioni delle due circonferenze.
%  \begin{enumeratea}
%   \item  \(c0:~x^2 + y^2 -10x +8y +15 = 0; \quad c1:~x^2 + y^2 -4x -4y -12 
% = 0\)\\
% .
%    \hfill [\(ar:~y = \frac{1}{2} x -\frac{9}{4};~A (10, 33, 924)\)]
%   \item  \(c0:~x^2 + y^2 -10y - = 0; \quad c1:~x^2 + y^2 +2x +2y -11 = 
% 0\)\\
% .
%    \hfill [\(ar:~y = -\frac{1}{6} x +\frac{5}{6};~A (37, -25, 12132)\)]
%   \item  \(c0:~x^2 + y^2 -8x - = 0; \quad c1:~x^2 + y^2 -8x -8y +27 = 
% 0\)\\
% .
%    \hfill [\(ar:~y = \frac{7}{2};~A (2, 8, 19)\)]
%   \item  \(c0:~x^2 + y^2 -10x +10y +25 = 0; \quad c1:~x^2 + y^2 -8x -9 = 
% 0\)\\
% .
%    \hfill [\(ar:~y = \frac{1}{5} x -\frac{17}{5};~A (26, 117, 12025)\)]
%   \item  \(c0:~x^2 + y^2 -2y -9 = 0; \quad c1:~x^2 + y^2 -8x -18 = 0\)\\
% .
%    \hfill [\(ar:~y = 4 x +\frac{9}{2};~A (34, -28, 631)\)]
%   \item  \(c0:~x^2 + y^2 -32 = 0; \quad c1:~x^2 + y^2 -8x -4y -20 = 0\)\\
% .
%    \hfill [\(ar:~y = -2 x +3;~A (5, 6, 151)\)]
%   \item  \(c0:~x^2 + y^2 -2x -8y -44 = 0; \quad c1:~x^2 + y^2 -8x -2y -9 = 
% 0\)\\
% .
%    \hfill [\(ar:~y = x -\frac{35}{6};~A (12, 65, 1583)\)]
%  \end{enumeratea}
% \end{esercizio}


\subsection{Esercizi di riepilogo}

\begin{esercizio}\label{ese:}
Scrivere l'equazione della circonferenza tangente alle due rette 
\(y=0\) e \(y=5\) e con il centro sulla retta: \(3x-7y+7=0\).
\hfill[\(x^2+y^2-7x-5y+\frac{49}{4}=0\)]
\end{esercizio}

\begin{esercizio}\label{ese:}
Scrivi l’equazione della circonferenza tangente alle rette 
\(x=1\) e \(x=3\) e avente il centro sulla retta \(y=3x-1\).
\hfill[\(x^2+y^2-4x-10y+28=0\)]
\end{esercizio}

\begin{esercizio}\label{ese:}
Scrivi l’equazione della circonferenza tangente alle rette 
\(y=0\) e \(y=2\) e avente il centro sull’asse \(y\).
\hfill[\(x^2+y^2-2y=0\)]
\end{esercizio}

\begin{esercizio}\label{ese:}
Scrivi l’equazione della circonferenza tangente alle rette 
\(y=0\) e \(y=4\) e 
avente il centro sulla retta \(2x-y+2=0\).
\hfill[\(x^2+y^2-4y+2=0\)]
\end{esercizio}

\begin{esercizio}\label{ese:}
Determina i valori di \(k\) in modo che l’equazione \(x^2+y^2-2x+2y+k+3=0\) 
rappresenti:
\begin{enumerate} [label=\alph*), nosep]
\item una circonferenza di raggio~2
\item una circonferenza passante per l’origine
\end{enumerate}
\hfill[\(k=5; k=-3\)]
\end{esercizio}

\begin{esercizio}\label{ese:}
    Considera le circonferenze concentriche di equazione 
\(x^2+y^2+6y+k=0\),
\begin{enumerate} [label=\alph*), nosep]
\item calcolare le coordinate del centro 
\item disegnare quella con \(k=0\) 
\item determinare \(k\) affinché la circonferenza abbia raggio 5; 
\item determinare i punti \(A\) e \(B\) d’intersezione di questa ultima 
circonferenza con l’asse \(x\) 
\item determinare la parabola che ha vertice \(V\) nel centro della 
circonferenza precedente e che passa per \(A\).
\end{enumerate}
\hfill[\(y=3/16x^2-3\)]
\end{esercizio}

\begin{esercizio}\label{ese:}
Determina il valore di \(k\) in modo che le circonferenze seguenti 
siano tangenti:\\
\(x^2+y^2-6x+k=0\) e \(x^2+y^2-1=0\)
\hfill[\(k=5\)]

\end{esercizio}

\begin{esercizio}\label{ese:}
Dopo aver verificato che la retta s: \(2x-y-1=0\) è secante rispetto alla 
circonferenza \(x^2+y^2-2x=0\), determina la misura della corda individuata 
dalla retta s sulla circonferenza.

\hfill[\(\frac{3 \sqrt{5}}{5}\)]
\end{esercizio}

\begin{esercizio}\label{ese:}
La circonferenza di equazione \(x^2+y^2-4x-2y+3=0\) interseca l’asse \(x\) 
in due punti \(A\) e \(B\). Dopo averli determinati e disegnati, trova 
l’area del triangolo \(ABC\) dove \(C\) è il centro della circonferenza 
data.
\hfill[\(A=1\)]
\end{esercizio}

\begin{esercizio}\label{ese:}
Determinare il centro, il raggio e disegnare la circonferenza di 
equazione:\\
\(x^2+y^2+4x-2y-4=0\)\\
Determinare poi i punti di intersezione fra la circonferenza e la retta di 
equazione \(y=2x+2\).
\hfill [\(I_1\punto{-2}{-2}~I_2\punto{\frac{2}{5}}{\frac{14}{5}}\)]
\end{esercizio}

\begin{esercizio}\label{ese:}
Determinare il centro, il raggio e disegnare la circonferenza di equazione\\
\(x^2+y^2-2x+4y-4=0\)\\
Determinare poi i punti di intersezione fra la circonferenza e la retta di 
equazione \(y=2x-1\) 
\hfill [\(I_1\punto{1}{1}~I_2\punto{-\frac{7}{5}}{-\frac{19}{5}}\)]
\end{esercizio}

\begin{esercizio}\label{ese:}
Considerata la circonferenza di equazione \(x^2 +y^2 = 4\) dire se le seguenti 
rette sono secanti, non secanti oppure tangenti:
\begin{multicols}{3}
\begin{enumerate} [label=\alph*), nosep]
 \item \(y=x+4\)
 \item \(y=-x+1\)
 \item \(y=2x+2\sqrt{5}\)
\end{enumerate}
\end{multicols}
\end{esercizio}

\begin{esercizio}\label{ese:}
Considerata la circonferenza di equazione \(x^2 +y^2 = 4\) dire se le seguenti 
rette sono secanti, non secanti oppure tangenti:
\begin{multicols}{3}
\begin{enumerate} [label=\alph*), nosep]
 \item \(y=-x+5\)
 \item \(y=2x-1\)
 \item \(y=2x-2\sqrt{5}\)
\end{enumerate}
\end{multicols}
\end{esercizio}

\begin{esercizio}\label{ese:}
Scrivere l'equazione della circonferenza avente centro in \(\punto{3}{0}\) e 
passante per il punto \(\punto{6}{4}\).\hfill[\(x^2+y^2-6x-16=0\)]
\end{esercizio}

\begin{esercizio}\label{ese:}
Scrivere l'equazione della circonferenza avente per diametro il segmento di 
estremi:
\begin{enumerate} [label=\alph*), nosep]
  \item  \(A\punto{-3}{1},~B\punto{5}{-2}\)\hfill [\(x^2+y^2-2x+y-17=0\)]
  \item  \(A\punto{1}{0},~B\punto{3}{2}\)\hfill [\(x^2+y^2-4x-2y+3=0\)]
  \item  \(A\punto{0}{1},~B\punto{2}{3}\)\hfill [\(x^2+y^2-2x-4y+3=0\)]
 \end{enumerate}
\end{esercizio}

\begin{esercizio}\label{ese:}
Scrivere l'equazione della circonferenza passante per \(A\) e per \(B\) e 
avente il centro sulla retta \(r\):
\begin{enumerate} [label=\alph*), nosep]
  \item  \(A\punto{-2}{2},~B\punto{4}{-4},~r:~x+2y-8=0\)
  \hfill [\(x^2+y^2-8x-4y-16=0\)]
  \item  \(A\punto{2}{2}, B\punto{-4}{-4},~r:~x+2y=8\)
  \hfill [\(x^2+y^2+8x-4y-16=0\)]
  \item  \(A\punto{-2}{2}, B\punto{4}{0},~r:~3x-2y-1=0\)
  \hfill [\(x^2+y^2-2x-2y-8=0\)]
 \end{enumerate}
\end{esercizio}

\begin{esercizio}\label{ese:}
Scrivere l'equazione della circonferenza passante per \(A\), \(B\) e \(C\):
\begin{enumerate} [label=\alph*), nosep]
  \item  \(A\punto{1}{2},~B\punto{-1}{2},~C\punto{0}{0}\)
  \hfill [\(x^2+y^2-\frac{5}{2}y=0\)]
  \item  \(A\punto{1}{6}, B\punto{1}{0},~C\punto{-2}{6}\)
  \hfill [\(x^2+y^2+x-6y-2=0\)]
  \item  \(A\punto{-2}{2}, B\punto{4}{0},~C\punto{4}{-4}\)
  \hfill [\(x^2+y^2-2x-2y-8=0\)]
 \end{enumerate}
\end{esercizio}

\begin{esercizio}\label{ese:}        
Scrivere l'equazione della circonferenza passante per i punti 
\(A\punto{4}{1},~B\punto{2}{2}\) e avente il centro sulla retta \(r:~x-2y=0\).
Poi calcola la tangenti in A e in B alla circonferenza 
\hfill [\(x^2+y^2-6x-3y+10=0,~y=2x-7,~y=2x-2\)]
\end{esercizio}

\begin{esercizio}\label{ese:}
Scrivere l'equazione della circonferenza avente gli estremi del diametro 
nei punti di intersezione della retta \(x-3y-1=0\) con la retta \(x+2=0\)
e della retta \(x-2y=0\) con la retta \(x-2=0\)
\hfill[\(x^2+y^2 = 5\)]
\end{esercizio}

\begin{esercizio}\label{ese:}
Determinare il raggio e l'equazione della circonferenza avente centro in 
\(C\punto{2}{−6}\) e passante per \(P\punto{−7}{−1}\).
\hfill [\(r=\sqrt{106}; \quad \tonda{x-2}^2+\tonda{y+6}^2=106\)]
\end{esercizio}

\begin{esercizio}\label{ese:}
Determinare l'equazione della circonferenza passante per i punti 
\(A\punto{2}{0}, B\punto{−2}{4}\) 
ed avente il centro sulla retta  \(r:~x+y+2=0\)             
\hfill [\(\tonda{x+2}^2+y^2=16\)]
\end{esercizio}

\begin{esercizio}\label{ese:}
Determinare le equazioni delle circonferenze tangenti all'asse delle \(y\) nel 
punto di ordinata~4 ed aventi raggio pari a~5.
\hfill [\(\tonda{x-5}^2+\tonda{y-4}^2=25; \quad 
\tonda{x+5}^2+\tonda{y-4}^2=25\)]
\end{esercizio}

\begin{esercizio}\label{ese:}
Determinare le coordinate dei punti di intersezione della retta 
\(r:~x-y-2=0\) con la circonferenza \(x^2 +y^2 -2x-6y-6=0\)
\hfill [\(A\punto{1}{-1},~B\punto{5}{3}\)]  
\end{esercizio}

\begin{esercizio}\label{ese:} 
Determinare le equazioni delle rette tangenti condotte dal punto 
\(P\punto{7}{0}\) alla circonferenza \(x^2 +y^2 -2x-4y-15=0\)
\hfill [\(y=-2x+14; \quad y=\frac{1}{2}x-\frac{7}{2}\)]
\end{esercizio}

\begin{esercizio}\label{ese:}
Determinare le equazioni delle circonferenze passanti per 
\(A\punto{1}{0},~B\punto{0}{1}\) ed aventi raggio pari a \(\sqrt{5}\).          
\hfill [\(\tonda{x-2}^2+\tonda{y-2}^2=5; \quad \tonda{x+1}^2+\tonda{y+1}^2=5\)]
\end{esercizio}
