% (c) 2015 Daniele Zambelli daniele.zambelli@gmail.com

\chapter{Iperbole}

% \section{TODO}
% 
% \section{Ellisse con centro nell'origine e assi ...}
% \label{sec:01_formacanonica}

% \begin{wrapfloat}{figure}{r}{0pt}
% \includegraphics[scale=0.35]{img/fig000_.png}
% \caption{...}
% \label{fig:...}
% \end{wrapfloat}
% 
% \begin{center} \input{\folder lbr/fig000_.pgf} \end{center}

% \section{Altro paragrafo}
% \label{sec:ellisse_}

\section{L'iperbole}
\label{sec:iperbole_}

\noindent\begin{minipage}{.75\textwidth}
L'iperbole è la conica corrispondente all'intersezione fra un cono a doppia 
falda e un piano più inclinato della retta generatrice. In tal caso infatti il piano 
taglia entrambe le due falde del cono dando origine a una curva illimitata 
costituita da due parti, dette rami.
\end{minipage}
\hspace{.5cm}
\begin{minipage}{.2\textwidth}
  %    \begin{inaccessibleblock}[Cono a due falde tagliato da un piano
  \includegraphics[width=\textwidth]{img/iperbole2.jpg}
  %    \caption{Generazione di un cono a due falde}% 
  %\label{fig:ellissedalcono}
  %    \end{inaccessibleblock}
\end{minipage}  

\subsection{L'iperbole come luogo geometrico}
\label{subsec:iperbole_luogogeometrico}

L'iperbole può essere definita come luogo geometrico, nel seguente modo:
\begin{definizione}
  Dati nel piano \( \pi \) due punti \( F_{1} \) e \( F_{2} \), detti 
fuochi, si dice iperbole il luogo geometrico I dei punti P di \( \pi \) tali 
che sia costante la differenza delle distanze di P da \(F_{1}\) e \(F_{2}\) 

\begin{equation*}
% I=\{P \in\pi|\overline{PF_{1}}-\overline{PF_{2}}=2a,a\in R_{+}^{0}\}
  I=\{P \in\pi \sand \abs{PF_{1}-PF_{2}} = 2a,~2a \geq 0\}
\end{equation*}
\end{definizione}

% \begin{figure}[h]
  %\hspace{12pt}
  \begin{minipage}[c]{.70\textwidth}
Leggiamo la formulazione della definizione: la differenza delle distanze 
tra due punti fissati, chiamati \textbf{fuochi}, e un generico punto P 
dell'iperbole 
risulta fissata e pari a \(2a\), qualsiasi sia il punto dell'iperbole. 
Questa lunghezza \((2a)\) è associata ad un numero reale non negativo.
I diversi punti P appartenenti al luogo geometrico dovranno dunque 
mantenere costante la differenza tra le lunghezze dei segmenti 
\(\overline{PF_{1}}\) e \(\overline{PF_{2}}\) come indicato nella figura a 
fianco:
  \end{minipage}
  \hspace{.5cm}
  \begin{minipage}[c]{.25\textwidth}
    %    \begin{inaccessibleblock}[Cono a due falde tagliato da un piano
    %      che forma un'ellisse.]
    \includegraphics[width=1.1\textwidth]{img/iperbol1.jpg}
%     \caption{L'iperbole come luogo geometrico.}
    %\label{fig:ellissedalcono}
    %    \end{inaccessibleblock}
  \end{minipage}
% \end{figure}

% \vspace{7pt}

% \begin{figure}[h]
  %\hspace{12pt}
  \begin{minipage}[c]{.27\textwidth}
    %    \begin{inaccessibleblock}[Cono a due falde tagliato da un piano
    %      che forma un'ellisse.]
    \includegraphics[width=1.1\textwidth]{img/iperbol2.jpg}
%     \caption{i fuochi dell'iperbole.}
    %\label{fig:ellissedalcono}
    %    \end{inaccessibleblock}
  \end{minipage}
  \hspace{.8cm}
  \begin{minipage}[c]{.65\textwidth}
Cerchiamo ora di determinare l'equazione algebrica associata a 
questa curva.
Consideriamo i due fuochi \( F_{1}(-c;~0)\) e \( F_{2}(0;~c)\) sull'asse 
delle X e chiamiamo \textbf{centro} (dell'iperbole) il punto medio del segmento
\(\overline{F_{1}F_{2}}\), e \textbf{distanza focale} la misura di tale segmento 
\(\overline{F_{1}F_{2}}\), pari a \(2c\).

Applichiamo quindi la definizione 
\(\left|\overline{PF_{1}}-\overline{PF_{2}}\right|=2a\). 

Utilizzando la formula per la lunghezza di un segmento 
possiamo riscrivere la precedente relazione come:
    \(\left|\sqrt{(x-c)^{2}+y^{2}} -\sqrt{(x+c)^{2}+y^{2}}\right|=2a\).
  \end{minipage}
% \end{figure}

Sviluppando i calcoli come si è fatto per l'ellisse, con alcuni passaggi 
algebrici si ottiene l'espressione: \(\left( c^{2} -a^{2}\right) 
x^{2}+a^{2}y^{2}=a^{2}\left(c^{2}-a^{2}\right)\)

Applicando la sostituzione \( c^{2}-a^{2}=b^{2} \) 
otteniamo quindi l'equazione: \(b^{2}x^{2}-a^{2}y^{2}=a^{2}b^{2}\).  
Dividendo infine entrambi i membri per \(a^{2}b^{2}\), si ricava l'espressione:
\begin{equation*}
\dfrac{x^{2}}{a^{2}}-\dfrac{y^{2}}{b^{2}}=1
\end{equation*}
detta equazione canonica dell'iperbole avente i fuochi sull'asse X.

\subsection{Le caratteristiche dell'iperbole}
\label{subsec:iperbole_caratteristiche}

\begin{description}
\item [\textbf{Intersezioni con gli assi}]: il grafico dell'iperbole, 
come abbiamo visto, interseca solo l'asse delle X in due punti \( A_{1} 
(a;~0)\) e \( A_{2}(-a;~0)\) le cui coordinate possono facilmente essere 
trovate risolvendo il sistema:

\centerline{\(\begin{cases}  \dfrac{x^{2}}{a^{2}}-\dfrac{y^{2}}{b^{2}}=1   
\\ y =0  
  \end{cases} \Rightarrow \begin{cases}  x=\pm a  \\ y=0
  \end{cases}\)}

Tali punti vengono detti \textbf{vertici reali}; l'asse che li congiunge, che 
coincide con X, è detto \textbf{asse trasverso}, mentre il secondo asse, dove 
non vi sono 
intersezioni con l'iperbole, viene detto \textbf{asse non trasverso}.\\
\item [\textbf{Simmetrie dell'iperbole}]: dato che nell'equazione canonica le 
variabili
compaiono solo con grado 2, se \(P (x, y)\) è 
un generico punto dell'iperbole anche i punti \( P_{1}(-x;~y)\),     \( 
P_{2}(-x;~-y)\) e \( P_{3}(x;~-y)\) appartengono all'iperbole. Possiamo 
affermare dunque che l'iperbole è una curva simmetrica rispetto all'asse X, 
rispetto all'asse Y e rispetto all'origine.\\
\item [\textbf{Vertici non reali, asintoti e disegno}]:
 abbiamo mostrato che il parametro \(a\) è legato alle ascisse dei punti di 
intersezione dell'iperbole con l'asse delle X. Possiamo affermare qualcosa 
di simile per il parametro \(b\)? Sicuramente non nella stessa forma, in quanto 
l'iperbole non ha intersezioni con l'asse Y. 

% \begin{figure}[h]
  %\hspace{12pt}
  \begin{minipage}[c]{.65\textwidth}
  Tuttavia risulta comodo definire, in corrispondenza ai due vertici 
reali, altri due vertici, detti \textbf{vertici non reali}, sull'asse delle Y: i 
punti \( 
B_{1} (0; b)\) e \( B_{2} (0; -b)\). Sono detti \emph{non reali} in quanto non 
identificano una 
reale intersezione.\\ Costruiamo ora un rettangolo di lati \(2a\) e \(2b\) con 
i 
punti \( A_{1} \), \( A_{2} \), \( B_{1} \) e \( B_{2} \) (punti medi di tali 
lati): 
il segmento che congiunge l'origine ad uno dei vertici del rettangolo risulta 
lungo c. Infatti, da 
quanto visto nella determinazione dell'equazione dell'iperbole, vale la 
relazione \( 
c^{2} = a^{2} + b^{2} \).
  \end{minipage}
  \hspace{.5cm}
  \begin{minipage}[c]{.3\textwidth}
    %    \begin{inaccessibleblock}[Cono a due falde tagliato da un piano
    %      che forma un'ellisse.]
    \includegraphics[width=\textwidth]{img/rettangolo.jpg}
%     \caption{Relazione tra i parametri dell' iperbole.}
    %\label{fig:ellissedalcono}
    %    \end{inaccessibleblock}
  \end{minipage}
% \end{figure}

% \vspace{0.2cm}

Cerchiamo di capire la relazione tra il rettangolo appena determinato e 
l'iperbole.

% \begin{figure}[h]
  %\hspace{12pt}
  \begin{minipage}[c]{.65\textwidth}
    L'iperbole tocca il rettangolo \(A_{1} A_{2} B_{1} B_{2}\) 
soltanto nei vertici reali \(A_{1}\) e \(A_{2}\) e si sviluppa 
illimitatamente all'interno delle due parti di piano delimitate da due 
rette, chiamate \textbf{asintoti}. Gli asintoti non sono altro che la 
prosecuzione 
delle diagonali del rettangolo ed hanno come coefficiente angolare \( m=\pm 
b/a\). Gli stessi asintoti forniscono una sorta di limite invalicabile e 
irraggiungibile
da parte dell'iperbole. Le equazioni di queste rette, 
passanti per l'origine sono \(y = \pm \dfrac{b}{a} x\).
  \end{minipage}
  \hspace{.2cm}
  \begin{minipage}[c]{.3\textwidth}
    %    \begin{inaccessibleblock}[Cono a due falde tagliato da un piano
    %      che forma un'ellisse.]
    \includegraphics[width=\textwidth]{img/asintoti.jpg}
%     \caption{il rettangolo caratteristico dell'iperbole.}
    %\label{fig:ellissedalcono}
    %    \end{inaccessibleblock}
  \end{minipage}
% \end{figure}

Notiamo infine che possiamo esprimere le coordinate dei fuochi in funzione 
di a e b come: \( F_{1}(\sqrt{a^{2}+b^{2}}, 0) \), \( 
F_{2}(-\sqrt{a^{2}+b^{2}}, 0) \)
\item [\textbf{Eccentricità}]: Analogamente a 
quanto visto per l'ellisse definiamo l'eccentricità di un iperbole come
rapporto tra distanza focale e lunghezza dell'asse trasverso:
\begin{equation*}
e=\dfrac{\text{distanza focale}}{\text{asse trasverso}}=
\dfrac{2c}{2a}=\dfrac{c}{a}=\dfrac{\sqrt{a^{2}+b^{2}}}{a}
\end{equation*}
poiché dalla precedente formula \(c>a\), osserviamo che \(e>1\).
Per comprendere il significato geometrico dell'eccentricità e
la relazione che tale valore ha col grafico dell'iperbole, studiamo come varia 
l'eccentricità al variare di \(b\) (tenendo fisso il parametro \(a\)), con i 
seguenti esempi:
\begin{figure}[htbp]
  \centering
  %    \begin{inaccessibleblock}[Cono a due falde tagliato da un piano
  %      che forma un'ellisse.]
  \includegraphics[width=\textwidth]{img/4iperboli.jpg}
    \caption{Eccentricità dell'iperbole al variare del 
parametro b.}%
  %\label{fig:ellissedalcono}
  %    \end{inaccessibleblock}
\end{figure}
\end{description}
In particolare, si può notare che:
\begin{itemize}
 \item \(e \rightarrow 0\) (caso limite): l'iperbole tende ad appiattirsi fino a 
coincidere con l'asse \(x\)
\item \(e>1\): i fuochi si allargano, determinando un'apertura dei rami 
dell'iperbole
\item \(e \rightarrow +\infty\) (caso limite):
l'iperbole si allarga sempre più, fino a coincidere con le rette \(x=\pm a\)
\end{itemize}

\subsection{L'iperbole con i fuochi sull'asse Y}

Se i fuochi, al contrario di quanto visto finora, giacciono sull'asse Y e 
hanno coordinate \( F_{1} (0;~-c)\) e \( F_{2} (0;~c)\) prende forma una nuova 
tipologia di iperbole che invece di svilupparsi a sinistra e a destra 
dell'origine si sviluppa sopra e sotto di tale punto.
Con ragionamenti molto simili ai precedenti, partendo stavolta dalla relazione 
\(\left|\overline{PF_{1}}-\overline{PF_{2}}\right|=2b\) riusciamo a 
determinare l'equazione canonica di questo tipo di iperbole, ovvero: 
\begin{equation*}
-\dfrac{x^{2}}{a^{2}}+\dfrac{y^{2}}{b^{2}}=1
\end{equation*}
Si può facilmente verificare che:
\begin{itemize} [noitemsep]
  \item l'iperbole è simmetrica rispetto all'origine e agli assi 
cartesiani ;
  \item l'asse Y è l'asse trasverso e su di esso giacciono i vertici 
reali \( B_{1} (0;~b)\) e \( B_{2} (0;~-b)\);
  \item l'asse X è l'asse non trasverso dove giacciono i vertici non 
reali \( A_{1} (a;~0)\) e \( A_{2} (-a;~0)\);
  \item le rette di equazione \(y= \pm \dfrac{b}{a}  x\) sono gli asintoti 
dell'iperbole;
  \item l'eccentricità è definita dalla formula 
\(e=\dfrac{\text{distanza focale}}{\text{asse trasverso}}=
\dfrac{c}{b}=\dfrac{\sqrt{b^{2}+a^{2}}}{b} \)
\end{itemize}

\subsection{Condizioni per determinare l'equazione dell'iperbole}

Similmente a quanto visto per l'ellisse, poiché anche l'iperbole 
è determinata da due parametri (\(a\), \(b\)), serviranno solo due condizioni 
per 
determinarne l'equazione canonica.
Le coppie di informazioni che insieme consentono di determinare un'iperbole 
sono:
\begin{itemize}[noitemsep]
\item Due punti appartenenti all'iperbole (non simmetrici rispetto 
agli assi o rispetto all'origine);
\item Punto dell'iperbole e fuoco (o vertice);
\item Fuoco e vertice;
\item Eccentricità e un fuoco (o vertice, o punto dell'iperbole);
\item Asintoto e fuoco (o vertice, o punto dell'iperbole).
\end{itemize}

\subsection{L'iperbole equilatera e la funzione omografica}
\label{subsec:iperbole_omografica}

Se le lunghezze del semiasse trasverso e non trasverso sono uguali, ovvero 
\(a=b\), 
l'equazione dell'iperbole (con i fuochi sull'asse \(x\)) diventa: 
\begin{equation*}
\dfrac{x^{2}}{a^{2}}-\dfrac{y^{2}}{a^{2}}=1 \qquad \longrightarrow \qquad
x^{2}-y^{2} =  a^{2} 
\end{equation*}
Tale forma di iperbole, con un solo parametro, è detta 
\emph{iperbole equilatera riferita ai propri assi}.

% \vspace{12pt}
% \begin{figure}[h]
  %\hspace{12pt}
  \noindent\begin{minipage}[c]{.65\textwidth} 
In questo caso gli altri elementi che caratterizzano l'iperbole diventano:
  \begin{itemize} [nosep]
    \item asintoti: \(y\pm=x\) (bisettrici dei quadranti);
    \item semidistanza focale: \(c=a \sqrt{2} \);
    \item eccentricità: \(e = \dfrac{\sqrt{a^{2}+a^{2}}}{a}=\sqrt{2} \)
    \item fuochi: \( F_{1} \left(a \sqrt{2};~0\right)\) e 
                        \( F_{2}\left(-a \sqrt{2};~0\right)\)
    \item vertici reali: \( A_{1}(a;~0)\) e \(A_{2}(-a;~0)\)
  \end{itemize}    
  \end{minipage}
  \hfill
  \begin{minipage}[c]{.3\textwidth}
    %    \begin{inaccessibleblock}[Cono a due falde tagliato da un piano
    %      che forma un'ellisse.]
    \includegraphics[height=4cm, width=4cm,]{img/equilatera.jpg}
%     \caption{L'iperbole equilatera.}
    %\label{fig:ellissedalcono}
    %    \end{inaccessibleblock}
  \end{minipage}
% \end{figure}

% \vspace{12pt}
Da notare che il rettangolo, costruito con i vertici, in questo caso diventa un 
quadrato di lato pari a \(2a\).
Nel caso simmetrico, con i fuochi sull'asse Y, l'equazione diventa:
\( x^{2} - y^{2} =- a^{2} \), con i fuochi 
\( F_{1} \left(0; a \sqrt{2}\right)\), \( F_{2} \left(0;~-a \sqrt{2}\right)\) 
e vertici reali \( B_{1} (0; a)\),\( B_{2} (0; -a)\).

% \vspace{7pt}

Ruotando di \(45\grado\) l'iperbole equilatera riferita ai propri 
assi otteniamo una nuova iperbole che ha come asintoti gli assi cartesiani 
e come assi di simmetria le bisettrici dei quadranti.
Chiamiamo questo tipo 
di iperbole \emph{iperbole equilatera riferita ai propri asintoti}.

\affiancati{.39}{.59}{
Si può dimostrare che l'equazione di tale iperbole si può scrivere nella 
forma: 
\[x y=k \hspace{1cm} con \hspace{0.2cm}k \neq 0\]
Notiamo che l'equazione \(xy=k\) non è altro che l'espressione della 
proporzionalità inversa tra due grandezze che mantengono costante il loro 
prodotto, dove \(k\) è la costante di proporzionalità. 
}{
% In figura \ref{fig:iperboleequilatera} è possibile vedere il grafico di 
% tale iperbole, nei casi \(k>0\) e \(k<0\).
% \begin{figure}[!h]
%   \centering
  %    \begin{inaccessibleblock}[Cono a due falde tagliato da un piano
  %      che forma un'ellisse.]
  \includegraphics[scale=.5]{img/equilatera2.jpg}
%   \caption{Iperbole equilatera con k>0 e k<0.}
%   \label{fig:iperboleequilatera}
  %    \end{inaccessibleblock}
% \end{figure}

% \vspace{7pt}
}

I vertici di quest'iperbole, utili per disegnarla, sono dati 
dall'intersezione dell'iperbole con la bisettrice del primo e terzo 
quadrante (oppure secondo e quarto, a seconda del segno di \(k\)). 
Intersecando l'iperbole \(xy=k\) (con \(k>0\))
e la bisettrice \(y=x\) otteniamo: \(A_{1} \punto{\sqrt{k}}{\sqrt{k}}\) e 
\(A_{2} \punto{- \sqrt{k}}{-\sqrt{k}}\). Nel caso di \(k<0\) cambiano 
solamente i segni (e si mette \(|k|\) all'interno delle radici), che in ogni 
caso si possono dedurre in modo semplice dal grafico, ragionando sui quadranti. 
I fuochi hanno invece coordinate
\(F_1 \punto{\sqrt{2k}}{\sqrt{2k}}\) e \(F_2 \punto{-\sqrt{2k}}{-\sqrt{2k}}
\), 
con analogo ragionamento sui segni nel caso \(k<0\)

% \vspace{12pt}

\noindent\begin{minipage}{.6\textwidth}
L'ultima tipologia di iperbole, ampiamente utilizzata in matematica,
è la cosiddetta \emph{funzione omografica}. Si tratta semplicemente di
una traslazione di un'iperbole equilatera riferita ai propri asintoti (ovvero 
\(xy=k\)).
L'equazione che ne deriva ha la seguente forma:
\begin{equation*}
y= \dfrac{ax+b}{cx+d}
\end{equation*}

con \(a,b,c,d\in \R\), \(c \neq 0\) e \( ad-bc \neq 0\)

% \vspace{6pt}
Il centro e gli asintoti di quest'iperbole sono:
\[C \punto{-\dfrac{d}{c}}{\dfrac{a}{c}} \qquad y= \dfrac{a}{c} \quad; \quad  x=-\dfrac{d}{c}\]
\end{minipage}
\hfill
\begin{minipage}{.35\textwidth}
% \begin{figure}[!htbp]
  \centering
  %    \begin{inaccessibleblock}[Cono a due falde tagliato da un piano
  %      che forma un'ellisse.]
  \includegraphics[height=5cm, width=5cm]{img/omografica.jpg}
%   \caption{Funzione omografica.}%
  %\label{fig:ellissedalcono}
  %    \end{inaccessibleblock}
% \end{figure}
\end{minipage}

% \vspace{15pt}
Le due condizioni precedentemente poste nell'equazione della funzione 
omografica sono molto importanti, infatti:

\begin{itemize} [noitemsep]
  \item se \(c= 0\) otteniamo \(y= \dfrac{ax+b}{d} \) cioè \(y= 
\dfrac{ax}{d} + \dfrac{b}{d} \), che rappresenta una semplice retta; \\[3pt]
  \item se \(ad-bc=0\), ovvero \( \dfrac{d}{c} = \dfrac{b}{a} \), supponendo \(x\neq-d/c\), si ha: 
\[y=\dfrac{ax+b}{cx+d}=  \dfrac{a\left(x+ \frac{b}{a} \right)}{c\left(x+ 
\frac{d}{c} \right)} = \dfrac{a \cancel{\tonda{x+ \frac{b}{a}}}}{c\cancel{\tonda{x+ 
\frac{b}{a} }}} = 
\dfrac{a}{c} \] e quindi otteniamo una retta parallela all'asse \(x\), di equazione \(y=a/c\), 
definita per tutti valori di \(x\), tranne \(x=-d/c\).  
\end{itemize}

Come si disegna una funzione omografica? Inizialmente si determinano il centro e gli asintoti,
usando le formule descritte in precendenza. Quindi, per ricavare i vertici dell'iperbole è
sufficiente trovare l'intersezione di quest'ultima con la parallela alla bisettrice del
primo/terzo quadrante passante per il centro dell'iperbole, ovvero \(y-y_C = \pm (x-x_C)\).

\begin{esempio}
~

% \begin{figure}[h]
  %\hspace{12pt}
  \begin{minipage}[c]{.6\textwidth}
    \emph{Data l'iperbole equilatera \( x^{2} - y^{2} =6\), determina 
vertici, fuochi e corrispondente grafico.}\\[7pt]
    Poiché \(a= \sqrt{6} \), i vertici reali sono dati dai punti \( A_{1} 
\punto{\sqrt{6}}{0}\) e \( A_{2} 
\punto{-\sqrt{6}}{0}\). Il parametro \(c\) si può calcolare con:
\[c=\sqrt{6}  \sqrt{2} = \sqrt{12} =2 \sqrt{3}\] Grazie a \(c\) 
calcoliamo i fuochi \( F = \punto{\pm 2\sqrt{3}}{0}\). Il grafico
corrispondente è disegnato a lato.
  \end{minipage}
  \hspace{.2cm}
  \begin{minipage}[c]{.35\textwidth}
    %    \begin{inaccessibleblock}[Cono a due falde tagliato da un piano
    %      che forma un'ellisse.]
    \includegraphics[width=\textwidth]{img/equilatera1.jpg}
    %\caption{Generazione di un'ellisse da un cono a due falde}
    %\label{fig:ellissedalcono}
    %    \end{inaccessibleblock}
  \end{minipage}
% \end{figure}
\end{esempio}

% \newpage %--------------------------------------------------

\begin{esempio}
~

% \begin{figure}[h]
  %\hspace{12pt}
  \begin{minipage}[c]{.6\textwidth}
    \emph{Data l'iperbole equilatera \(xy=8\), determina vertici, fuochi 
e corrispondente grafico.}

% \vspace{7pt}

Si tratta di un'iperbole equilatera riferita ai propri 
asintoti e, poiché \(k> 0\), si trova nel primo e terzo quadrante. I vertici 
sono: \( A_{1} \punto{2\sqrt{2}}{2\sqrt{2}}\) e \( A_{2} 
\punto{-2\sqrt{2}}{-2\sqrt{2}}\). I fuochi avranno quindi coordinate 
\( F_{1} \punto{4}{4}\) e \( F_{2} \punto{-4}{-4}\)
  \end{minipage}
  \hspace{.2cm}
  \begin{minipage}[c]{.35\textwidth}
    %    \begin{inaccessibleblock}[Cono a due falde tagliato da un piano
    %      che forma un'ellisse.]
    \includegraphics[width=\textwidth]{img/equilatera2a.jpg}
    %\caption{Generazione di un'ellisse da un cono a due falde}
    %\label{fig:ellissedalcono}
    %    \end{inaccessibleblock}
  \end{minipage}
% \end{figure}
\end{esempio}

\begin{esempio} \emph{Data la funzione omografica \(y= \dfrac{x-3}{x+2}\), 
dopo aver verificato che è un'iperbole , determinane gli 
asintoti e il centro di simmetria. Completa l'esercizio disegnando il grafico della funzione.}\\[7pt]
Per verificare se si tratta di un iperbole, è sufficiente vedere che 
\(c=1\neq 0\) e che \(ad-bc=5\neq 0\). 
Possiamo ora determinare gli asintoti: quello verticale è 
\(x=- \dfrac{d}{c} =-2\) e quello orizzontale è \(y= \dfrac{a}{c} =1\); 
il centro di simmetria è \(C(-2; ~1)\).

\noindent \begin{minipage}[c]{.65\textwidth}
Per disegnare la funzione, una volta inseriti gli asintoti, è possibile andare
a calcolare le intersezioni dell'iperbole con l'asse \(x\) e l'asse \(y\), in modo
da avere qualche punto di riferimento per il grafico (e i corrispondenti punti simmetrici
rispetto al centro dell'iperbole). Calcolando, si trova: \( P_{1}  \left(0;~ - \dfrac{3}{2} \right)\) e \( P_{2} =(3;~0)\).
  \end{minipage}
  \hspace{0.5cm}
  \begin{minipage}[c]{.3\textwidth}
    %    \begin{inaccessibleblock}[Cono a due falde tagliato da un piano
    %      che forma un'ellisse.]
    \includegraphics[width=\textwidth]{img/equilatera2b.jpg}
    %\caption{Generazione di un'ellisse da un cono a due falde}
    %\label{fig:ellissedalcono}
    %    \end{inaccessibleblock}
  \end{minipage}
% \end{figure}
\end{esempio}
