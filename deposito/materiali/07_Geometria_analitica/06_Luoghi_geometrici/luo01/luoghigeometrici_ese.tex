% (c) 2015 Daniele Zambelli daniele.zambelli@gmail.com

\section{Esercizi}

\subsection{Esercizi dei singoli paragrafi}

\subsubsection*{\numnameref{sec:coniche_e_retta}}

\begin{esercizio}
  \label{ese:div.003}
  Considera la conica data e stabilisci se le rette al suo fianco 
sono secanti, tangenti od esterne.
  \begin{enumeratea}
\item \(y=3 x^{2} +6x-4;~r:~y=2x+3,~s:~y=\dfrac{1}{4}x-8,~t:~y=-3x-1\)\\ 
\hfill  \(\left[r~secante,~s ~esterna,~ t~secante\right]\)
\item \(y=-x^{2}+2x+4;~r:~y=4x+5,~s:~y=3x+1,~t:~y=-2x+8\)\\
\hfill \(\left[r~tangente,~s~esterna,~t~tangente\right]\)
\item \(\dfrac{x^{2}}{18}+\dfrac{y^{2}}{36}=1;~
r:~y=3x-2,~s:~y=-6,~t:~y=-2x+8\)\\
\hfill \(\left[r~secante,~s~tangente,~t~secante\right]\)
\item \(\dfrac{x^{2}}{25}+\dfrac{y^{2}}{4}=1;~r:~y=-2x+1,~s:~y=3x,~t:~y= 
\dfrac{x}{2} +6\)\\
\hfill \(\left[r~secante,~s~secante,~t~esterna\right]\)
\item \( x^{2}+y^{2}=4;~r:~y=-x-1,~s:~y=x+3,~t:~x+\sqrt{3}y=4\)\\
\hfill \(\left[r~secante,~s~esterna,~t~tangente\right]\)
\item \(4 x^{2}-5 y^{2}=20;~r:~y=~-x-1,~s:~y=3x,~t:~y=3x+7\)\\
\hfill \(\left[r~tangente,~s~esterna,~t~secante\right]\)
\item \(\dfrac{x^{2}}{9}-\dfrac{y^{2}}{4}=1;~r:~5x-6y-9=0,~s:~x=4,~t:~ 
x-3y-3=0\)\\
\hfill \(\left[r~tangente,~s~secante,~t~secante\right]\)
\item \(x^{2}+y^{2}-4x+2y=0;~r:~x+2y-5=0,~s:~x+4y-5=0,~t:~-x+2y-8=0\)\\
\hfill \(\left[r~tangente,~s~secante,~t~esterna\right]\)
\end{enumeratea}
\end{esercizio}

\begin{esercizio}
  \label{ese:div.003}
  Determina i punti di intersezione tra le coniche e le rette 
sottostanti.
  \begin{enumeratea}
  \item \( x^{2}+4y^{2}=1 \), \(y=x+1\)
  \hfill\(\left[P_{1}\left( -\dfrac{3}{5};~\dfrac{2}{5} \right);~ 
P_{2}\left(-1;~0\right)\right]\)
  \item  \(4x^{2}+y^{2}=4 \), \(y=x+2\)
  \hfill\(\left[P_{1}\left( -\dfrac{4}{5};~\dfrac{6}{5} \right);~ 
P_{2}\left(0;~2\right)\right]\)
  \item \(5x^{2}-y^{2}=11 \), \(y=2\)
  \hfill\(\left[P_{1}\left( \sqrt{3};~2 \right);~ 
P_{2}\left(-\sqrt{3};~2\right)\right]\)
  \item \(9x^{2}-25y^{2}=225 \), \(y=\dfrac{2}{5}x+2\)
  \hfill \(\left[P_{1}\left(13;~\dfrac{36}{5} \right);~ 
P_{2}\left(-5;~0\right)\right]\)
  \end{enumeratea}
\end{esercizio}

\subsubsection*{\numnameref{sec:coniche_tangenti}}

\begin{esercizio}
  \label{ese:div.003}
  Determina le rette tangenti alla conica indicata passanti per il 
punto A, ad essa esterno.
  \begin{enumeratea}

\item \(9x^{2}+4y^{2}=36,~A(2;~5)\)  
\hfill \(\left[y=\dfrac{4}{5}x+\dfrac{17}{5};~x=2\right]\)
\item \(x^{2}+2y^{2}=2,~A(2;~1)\)
\hfill \(\left[y=1;~y=2x+3\right]\)
\item \(\dfrac{x^{2}}{4}+\dfrac{y^{2}}{5}=1,~A(3;~0)\)
\hfill \(\left[y=-x+3;~y=x-3\right]\)
\item \(\dfrac{x^{2}}{16}+\dfrac{y^{2}}{9}=1,~A(6;~-1)\)
\hfill \(\left[y=-x+5;~y=\dfrac{2}{5}x-\dfrac{17}{5}\right]\)
\item \(\dfrac{x^{2}}{9}-\dfrac{y^{2}}{4}=1,~A(0;~3)\)
\hfill \(\left[y=-\dfrac{6}{5}x+3;~y=\dfrac{6}{5}x+3\right]\)
\item \(\dfrac{x^{2}}{25}-\dfrac{y^{2}}{16}=1,~A(-5;~-2)\) 
\hfill \(\left[x=-5;~y=x+3\right]\)
\item \(x^{2}-2y^{2}=2,~A(1;~2)\)
\hfill \(\left[y=-5x+7;~y=x+1\right]\)
\item \(4x^{2}-9y^{2}=144,~A(0;~2)\)
\hfill \(\left[y=\dfrac{3}{4}x+2;~y=-\dfrac{3}{4}x+2\right]\)
\end{enumeratea}
\end{esercizio}

\begin{esercizio}
  \label{ese:div.003}
  Applicando la formula dello sdoppiamento determina la tangente alla 
conica data passante per il suo punto A.
  \begin{enumeratea}
\item \(\dfrac{x^{2}}{4}+\dfrac{y^{2}}{9}=1,~A\left(-\dfrac{8}{5};~- 
\dfrac{9}{5}\right)\)  
\hfill \(\left[y=-2x-5\right]\)
\item \(\dfrac{x^{2}}{25}+\dfrac{y^{2}}{16}=1,~A\left(-3;~\dfrac{16}{5} 
\right)\)
\hfill \(\left[y=\dfrac{3}{5}x+5\right]\)
\item \(x^{2}+3y^{2}=3,~A\left(\dfrac{3}{2};~\dfrac{1}{2} \right)\)
\hfill \(\left[y=-x+2\right]\)
\item \(x^{2}+9y^{2}=9,~A(0;~-1)\)
\hfill \(\left[y=-1\right]\)
\item \(\dfrac{x^{2}}{9}-\dfrac{y^{2}}{4}=1,~A\left(3\sqrt{2};~2\right)\)
\hfill \(\left[y=\dfrac{2\sqrt{2}}{3}x-2\right]\)
\item \(2x^{2}-y^{2}=2,~A(3;~4)\)
\hfill \(\left[y=\dfrac{3}{2}x-\dfrac{1}{2}\right]\)
\item \(4x^{2}-3y^{2}=4,~A(2;~2)\)
\hfill \(\left[y=\dfrac{4}{3}x-\dfrac{2}{3}\right]\)
\item \(\dfrac{5x^{2}}{16}-\dfrac{y^{2}}{4}=1,~A(2;~1)\)
\hfill\(\left[y=\dfrac{5}{2}x-4\right]\)
\end{enumeratea}
\end{esercizio}


\subsubsection*{\numnameref{sec:coniche_curve_deducibili}}

\begin{esercizio}
  \label{ese:div.003}
  Date le seguenti funzioni irrazionali, identificane la conica che 
ne consente di determinare il grafico e, dopo aver impostato il sistema che 
le definisce, disegnale. 
  \begin{enumeratea}
    \item \(y=\sqrt{9-x}\)
    \item \(y=\sqrt{4-x^{2}}\)
    \item \(y=\sqrt{9-4x^{2}}\)
    \item \(y=\sqrt{4x^{2}-25}\)
    \item \(y=\sqrt{4x-x^{2}}\)
    \item \(y=\sqrt{4-\dfrac{x^{2}}{4}}\)
  \end{enumeratea}
\end{esercizio}


\subsection{Esercizi riepilogativi}

% \begin{esercizio}
% \label{ese:D.19}
% testo esercizio
% \end{esercizio}

\begin{esercizio}\label{ese:03.1}
Per ognuna delle seguenti equazioni:
\begin{enumerate} [nosep]
 \item individua la conica rappresentata;
 \item calcola gli elementi caratteristici (per la parabola: vertice, asse 
di simmetria, intersezioni con gli assi; per la circonferenza: centro e 
raggio; per l'ellisse: vertici, fuochi e eccentricità; per l'iperbole: 
vertici, fuochi, eccentricità e asintoti);
 \item disegna il grafico.
\end{enumerate}

\begin{multicols}{2}
 \begin{enumeratea}
  \item  \(y=-x^2-4x+3\)
  \item  \(x^2+y^2-4x+6y-3=0\)
  \item  \(x^2-9y^2=36\)
  \item  \(4x^2+9y^2=36\)
  \item  \(y=\dfrac{1}{2}x^2+2x-5\)
  \item  \(x^2+y^2+2x-4y-20=0\)
  \item  \(4x^2+9y^2=360\)
  \item  \(y=x^2-2x-5\)
  \item  \(3x^2-4y^2=12\)
  \item  \(\dfrac{x^2}{25}+\dfrac{y^2}{100}=1\)
 \end{enumeratea}
\end{multicols}
\end{esercizio}

\begin{esercizio}\label{ese:03.1}
Scrivi l'equazione delle coniche descritte di seguito:
 \begin{enumerate} [nosep]
  \item  Ellisse con centro nell'origine degli assi, 
 \(semiassex=5\) e passante per il punto \(P\punto{-4}{+6}\).
  \item  Iperbole con centro nell'origine degli assi, 
  \(semiassex=6\) e \(semiassey=2\).
  \item  Ellisse con centro nell'origine degli assi, 
  \(semiassex=3\) e \(semiassey=2\).
  \item  Parabola con vertice in \(V\punto{-2}{7}\) 
  passante per \(P\punto{0}{3}\)
  \item  Circonferenza con centro in \(\punto{2}{-3}\) e raggio~4.
  \item  Parabola passante per \(A\punto{-6}{+1}\), \(B\punto{-4}{-5}\) e 
  \(C\punto{+2}{+1}\).
  \item  Iperbole con i fuochi sull'asse x, di semiasse reale~2 e
  passante per il punto \(P\punto{-4}{+3}\)
  \item  Circonferenza di diametro \(AB\) con
  \(A\punto{-5}{+5}\) e \(B\punto{+3}{-1}\) .
  \item  Parabola passante per \(A\punto{-1}{-2}\), \(B\punto{+2}{-5}\) e 
  \(C\punto{+4}{+3}\).
  \item  Ellisse con centro nell'origine passante per i punti:
  \(A\punto{+3}{+6}\), \(B\punto{+9}{+2}\)
 \end{enumerate}
\end{esercizio}

\begin{esercizio}\label{ese:03.1}
Collega le coniche dei due esercizi precedenti.
\end{esercizio}

\begin{esercizio}\label{ese:03.1}
Per ognuna delle seguenti coppie di coniche:
\begin{enumerate} [nosep]
 \item calcola le intersezioni;
 \item disegna il grafico.
\end{enumerate}

 \begin{enumeratea}
  \item  \(C_0:~x^2+y^2=40; \quad C_1:~y=\dfrac{1}{4}x^2 -2x+5\)
  \hfill [\(\punto{-0,6}{6,3};~\punto{6}{2}\)]
  \item  \(C_0:~-\dfrac{x^2}{20} + \dfrac{y^2}{4}=1; \quad 
           R_1:~x+5y-10=0\)
  \hfill [\(\punto{-5}{+3};~\punto{0}{2}\)]
  \item  \(C_0:~\tonda{x-2}^2+\tonda{y-1}^2=41; \quad C_1:~x^2-4x+8y+20=0\)
  \hfill [\(\punto{-2}{-4};~\punto{6}{-4}\)]
  \item  \(C_0:~x^2+y^2=25; \quad C_1:~-\dfrac{x^2}{3} + \dfrac{y^2}{4}=1\)
  \hfill [\(\punto{\mp 3}{\mp 4}\)]
  \item  \(C_0:~\dfrac{x^2}{16} - \dfrac{y^2}{16}=1; \quad 
           C_1:~-\dfrac{x^2}{20} + \dfrac{y^2}{4}=1\)
  \hfill [\(\punto{\mp 5}{\mp 3}\)]
  \item  \(C_0:~\dfrac{x^2}{12} + \dfrac{y^2}{36}=1; \quad 
           C_1:~\dfrac{5x^2}{48} + \dfrac{y^2}{144}=1\)
  \hfill [\(\punto{\mp 3}{\mp 3}\)]
  \item  \(C_0:~x^2+y^2=25; \quad C_1:~20x^2+9y^2=324\)
  \hfill [\(\punto{\mp 3}{\mp 4}\)]
  \item  \(C_0:~x^2+y^2=25; \quad R_1:~y=-x-1\)
  \hfill [\(\punto{-4}{+3};~\punto{+3}{-4}\)]
  \item  \(C_0:~\dfrac{x^2}{12} + \dfrac{y^2}{36}=1; \quad 
           C_1:~-\dfrac{5x^2}{36} + \dfrac{y^2}{4}=1\)
  \hfill [\(\punto{\mp 3}{\mp 3}\)]
  \item  \(C_0:~\dfrac{2}{35}x^2 + \dfrac{3}{140}y^2=1; \quad 
           R_1:~4x+3y+10=0\)
  \hfill [\(\punto{-4}{2};~\punto{2}{-6}\)]
 \end{enumeratea}
\end{esercizio}

\begin{esercizio}\label{ese:03.1}
Collega le coniche dei due esercizi precedenti.
\end{esercizio}
% \goodbreak
\begin{esercizio}\label{ese:03.1}
Considera la circonferenza avente equazione \quad \(x^2+y^2=4\) \quad e 
rappresenta il suo grafico.
\begin{enumeratea}
\item Scrivi l’equazione della parabola che ha vertice V nel punto in 
cui la circonferenza interseca il semiasse positivo delle ordinate e che 
passa per il punto \(P \punto{1}{0}\). 
\item Determina la misura della corda, appartenente al terzo e quarto 
quadrante, che la circonferenza e la parabola hanno in comune.
\hfill [\(y=2-x^2;~2\sqrt{3}\)]
\end{enumeratea}
\end{esercizio}

\begin{esercizio}\label{ese:03.1}
Scrivere l’equazione della tangente all’ellisse di equazione 
\quad \(\dfrac{x^2}{9}+\dfrac{y^2}{4}=1\) \quad
passante per il punto \(P \punto{0}{2}\).
\hfill [\(y-2=0\)]
\end{esercizio}

\begin{esercizio}\label{ese:03.1}
Trova l’equazione dell’iperbole equilatera riferita agli asintoti, 
sapendo che è tangente alla circonferenza con centro nell’origine e 
raggio~2.
\hfill [\(xy=2\)]
\end{esercizio}

\begin{esercizio}\label{ese:03.1}
Determina l’equazione dell’iperbole i cui fuochi sono due vertici 
dell’ellisse avente equazione \quad \(9x^2+4y^2=36\) \quad e i cui vertici 
sono i fuochi di tale ellisse. 
\hfill [\(-5x^2+4y^2=20\)]
\end{esercizio}

\begin{esercizio}\label{ese:03.1}
Verifica che la retta di equazione \quad \(x-2y=-6\) \quad è tangente 
all’iperbole equilatera di equazione \quad \(-x^2+y^2=12\) \quad e trova le 
coordinate del punto di contatto. 
\hfill [\(T\punto{-2}{-4}\)]
\end{esercizio}

\begin{esercizio}\label{ese:03.1}
Scrivi l’equazione della retta tangente all’ellisse di equazione
\quad \(\dfrac{x^2}{9}+\dfrac{y^2}{4}=1\) \quad
passante per il punto \(P\punto{3}{0}\). 
\hfill [\(x-3=0\)]
\end{esercizio}

\begin{esercizio}\label{ese:03.1}
Stabilire per quali valori di \(k\) l’equazione:
\quad \(\tonda{k+2}x^2-ky^2=1\) \quad
rappresenta:
\begin{enumeratea}
\item un’ellisse;
\item una circonferenza;
\item calcola l’area del quadrato circoscritto alla circonferenza.
\hfill [\(-2<k<0\);~\(k=-1\);~\(A=4\)]
\end{enumeratea}
\end{esercizio}

\begin{esercizio}\label{ese:03.1}
Riconosci le curve di equazioni 
\quad \(\dfrac{x^2}{10}+y^2=1\) \quad e \quad \(\dfrac{x^2}{8}-y^2=1\). 
Verifica che hanno gli stessi fuochi. 
I loro punti di intersezione appartengono ad una stessa circonferenza con 
centro nell’origine, di cui devi determinare l’equazione. 
\hfill [\(x^2+y^2=9\)]
\end{esercizio}

\begin{esercizio}\label{ese:03.1}
Considera la parabola di equazione 
\quad \(y=-x^2+4x\) 
\begin{enumeratea}
\item determina il suo vertice V
\item determina l’equazione della retta s passante per V e parallela alla 
bisettrice del primo e terzo quadrante. 
\item trova la lunghezza della corda staccata sulla parabola dalla retta s.
\hfill [\(\sqrt{2}\)]
\end{enumeratea}
\end{esercizio}

\begin{esercizio}\label{ese:03.1}
Trova per quali valori di \(a\) l’equazione:
\quad \(\tonda{a-1}x^2+\sqrt{25-a^2}y^2-6x+6y-3=0\)
\begin{enumeratea} %[label=\alph*, nosep]
 \item rappresenta una circonferenza e determina la sua equazione
 \hfill [\(a=4\)];
 \item rappresenta una parabola con asse parallelo all’asse delle \(y\)
 \hfill [\(a=\mp 5\)];
 \item rappresenta una parabola il cui il vertice ha ascissa negativa 
 \hfill [\(a=-5\)].
\end{enumeratea}
Poi calcola l'equazione dell’iperbole con i fuochi sull’asse delle 
\(x\), passante per il centro della circonferenza e per il vertice trovato
\hfill [\(5x^2-4y^2=1\)]
\end{esercizio}

