% (c) 2014 Daniele Zambelli - daniele.zambelli@gmail.com

\begin{comment}
% per disegnare il simbolo >>> in 05_02_tartaruga.
%\newcommand{\tggg}[0]{\textgreater\textgreater\textgreater}
%

Con riga e compasso.

\begin{procedura}[<Nome procedura>]\label{proc:geoint2_}
  <Descrizione problema>:
  \begin{enumerate} [nosep]
    \item 
    Traccia i punti A e B.
    \item 
    Traccia la circonferenza di centro A e passante per B.
    \item 
    Traccia la circonferenza di centro B e passante per A.
    \item 
    Individua un punto C di intersezione delle due circonferenze.
    \item 
    Il poligono ABC è il triangolo richiesto.    
  \end{enumerate}
\end{procedura}

Con la geometria interattiva.

\lstinputlisting[firstline=2]{\folder src/.py}


\subsection{\lstinline{Bisector}}
\label{sub:geoint_bisector}

\begin{description}
 \item [Scopo] 
 \item [Sintassi] 
 \item [Osservazioni]

 \item [Esempio] 

\end{description}


\end{comment}

% Forse è meglio mettere questa istruzione nel file delle intestazioni dopo la 
% lettura della libreia??????????????
\lstset{numbers=left, numberstyle=\tiny, frame=trbl, frameround=ftff,
        language=Python}

\chapter{Approfondimenti di Geometria interattiva}

\section{Altri strumenti della la geometria interattiva}
\label{sec:introduzione}

\emph{Come fare le stesse cose in modo più semplice.}

Nella prima parte della geometria interattiva abbiamo scritto alcuni programmi 
per realizzare degli oggetti semplici: triangoli equilateri, punti medi, 
bisettrici, rette parallele, rette perpendicolari, ...
Alcuni di questi oggetti sono utili per costruire degli oggetti più complessi 
ad esempio per costruire un quadrato posso:

\begin{procedura}[Quadrato]
 Dato il segmento AB costruire il quadrato che ha AB come lato.
  \begin{enumerate} [nosep]
    \item 
    Traccia i punti A e B e il segmento AB.
    \item 
    Traccia la circonferenza di centro A e passante per B.
    \item 
    Traccia la perpendicolare a AB passante per A.
    \item 
    Individua un punto D di intersezione della circonferenza e 
    della perpendicolare.
    \item 
    Traccia la perpendicolare a AB passante per B.
    \item 
    Traccia la perpendicolare a AD passante per D.
    \item 
    Individua un punto C di intersezione della circonferenza e 
    della perpendicolare.
    \item 
    Il poligono ABCD è il quadrato richiesto.    
  \end{enumerate}
\end{procedura}

Ora abbiamo imparato a tracciare la perpendicolare ad una retta passante per un 
punto: è un'operazione che richiede 4 o 5 operazioni semplici, ma nella 
procedura precedente dobbiamo eseguirla 3 volte. Un programma fatto così 
diventa lungo noioso e soggetto facilmente ad errori. Chi ha progettato la 
libreria \lstinline{pyig} ha pensato di mettere a disposizione anche altre 
classi oltre a quelle già presentate. In articolare esiste la classe che 
permette di costruire una retta perpendicolare: 
\lstinline{pyig.Orthogonal(<line>, <point>, ...)} 
Usando questo oggetto al posto della sequenza di operazioni vista in precedenza 
la costruzione del quadrato diventa:

\lstinputlisting[firstline=2]{\folder src/11quadrato.py}

Vediamo di seguito alcuni nuovi strumenti che la libreria ci fornisce come 
``primitivi'' ma che noi abbiamo già o possiamo comunque costruire con 
``riga e compasso''.

\subsection{\lstinline{Orthogonal}}
\label{sub:geoint_orthogonal}

\begin{description}
 \item [Scopo] Crea la retta perpendicolare ad una retta data passante per un 
punto.
 \item [Sintassi] \lstinline{Orthogonal(line, point, ...)}
%  \item [Osservazioni] 
% 
% \lstinline{line} è la retta alla quale si costruisce la perpendicolare 
% passante per \lstinline{point}.
% 
% Vedi anche i metodi delle classi \emph{linea} presentati nella classe 
% \lstinline{Segment}.

 \item [Esempio] Disegna la perpendicolare ad una retta data passante per un 
punto.

\begin{lstlisting}
# lettura delle librerie 
import pyig
# programma principale
ip = pyig.InteractivePlane('Orthogonal')
retta = pyig.Line(pyig.Point(-4, -1, width=5, name="A"), 
                  pyig.Point(6, 2, width=5, name="B"), 
                  width=3, color='DarkOrange1', name='r')
punto = pyig.Point(-3, 5, width=5, name='P')
pyig.Orthogonal(retta, punto)
## attivazione della finestra grafica
ip.mainloop()
\end{lstlisting}

\end{description}

\begin{exrig}
 
\textbf{Nota}: 
in tutti gli esempi seguenti, per motivi di spazio, saranno omesse 
le seguenti righe, prima dell'esempio:

\begin{lstlisting}
# lettura delle librerie 
import pyig
# programma principale
ip = pyig.InteractivePlane('Orthogonal')
\end{lstlisting}
e dopo l'esempio:
\begin{lstlisting}
## attivazione della finestra grafica
ip.mainloop()
\end{lstlisting}

Per eseguire l'esempio bisogna ricordarsi di aggiungerle.

\end{exrig}

\subsection{\lstinline{Parallel}}
\label{sub:geoint_parallel}

\begin{description}
 \item [Scopo] Crea la retta parallela ad una retta data passante per un punto.
 \item [Sintassi] \lstinline{Parallel(line, point, ...)}
%  \item [Osservazioni] 
% 
% \lstinline{line} è la retta alla quale si costruisce la parallela passante per
% \lstinline{point}.
% 
% Vedi anche i metodi delle classi \emph{linea} presentati nella classe 
% \lstinline{Segment}.

 \item [Esempio] Disegna la parallela ad una retta data passante per un punto.
\begin{lstlisting}
retta = pyig.Line(pyig.Point(-4, -1, width=5), 
                  pyig.Point(6, 2, width=5), 
                  width=3, color='DarkOrange1', name='r')
punto = pyig.Point(-3, 5, width=5, name='P')
pyig.Parallel(retta, punto)
\end{lstlisting}

\end{description}

\subsection{\lstinline{MidPoints}}
\label{sub:geoint_midpoints}

\begin{description}
 \item [Scopo] Crea il punto medio tra due punti.
 \item [Sintassi] \lstinline{MidPoints(point0, point1, ...)}
%  \item [Osservazioni] 
% 
% \lstinline{point0} e \lstinline{point1} sono due punti.

 \item [Esempio]Punto medio tra due punti.

\begin{lstlisting}
# creo due punti
p_0 = pyig.Point(-2, -5)
p_1 = pyig.Point(4, 7)
# cambio i loro attributi
p_0.color = "#00a600"
p_0.width = 5
p_1.color = "#006a00"
p_1.width = 5
# creao il punto medio tra p0 e p1
m = pyig.MidPoints(p0, p1, name='M')
# cambio gli attributi di m
m.color = "#f0f000"
m.width = 10
\end{lstlisting}

\end{description}

\subsection{\lstinline{MidPoint}}
\label{sub:geoint_midpoint}

\begin{description}
 \item [Scopo] Crea il punto medio di un segmento
 \item [Sintassi] \lstinline{MidPoint(segment)}
%  \item [Osservazioni] 
% 
% \lstinline{segment} è un oggetto che ha un \lstinline{point0} e un 
% \lstinline{point1}.

 \item [Esempio] Punto medio di un segmento.

\begin{lstlisting}
# creo un segmento
s = pyig.Segment(pyig.Point(-2, -1, color="#a60000", width=5),
                pyig.Point(5, 7, color="#6a0000", width=5), 
                color="#a0a0a0")
# creo il suo punto medio
pyig.MidPoint(s, color="#6f6f00", width=10, name=\'M\')
\end{lstlisting}

\end{description}

\subsection{\lstinline{Bisector}}
\label{sub:geoint_bisector}

\begin{description}
 \item [Scopo] Retta bisettrice di un angolo.
 \item [Sintassi] \lstinline{Bisector(angolo, ...)}
%  \item [Osservazioni]

 \item [Esempio] Disegna l'incentro di un triangolo.

\begin{lstlisting}
p_a = pyig.Point(-7, -3, color="#40c040", width=5, name="A")
p_b = pyig.Point(5, -5, color="#40c040", width=5, name="B")
p_c = pyig.Point(-3, 8, color="#40c040", width=5, name="C")
pyig.Polygon((p_a, p_b, p_c))      # Il triangolo
cba = pyig.Angle(p_c, p_b, p_a)  # Due angoli del triangolo
bac = pyig.Angle(p_b, p_a, p_c)
b1 = pyig.Bisector(cba, color="#a0c040") # Le bisettrici
b2 = pyig.Bisector(bac, color="#a0c040")
pyig.Intersection(b1, b2, color="#c040c0", width=5, name="Incentro")
\end{lstlisting}

\end{description}

\subsection{\lstinline{PointOn}}
\label{sub:geoint_pointon}

\begin{description}
 \item [Scopo] Punto disegnato su un oggetto in una posizione fissa.
 \item [Sintassi] \lstinline{PointOn(obj, parameter)}
 \item [Osservazioni]
L'oggetto deve essere una linea o una retta o una circonferenza,
\lstinline{parameter} è un numero che individua una precisa posizione 
sull'oggetto.
Sia le rette sia le circonferenze hanno una loro metrica che è legata ai punti
base dell'oggetto. Su una retta una semiretta o un segmento \lstinline{point0}
corrisponde al parametro 0 mentre \lstinline{point1} corrisponde al parametro 1.
Nelle circonferenze il punto di base della circonferenza stessa corrisponde
al parametro 0 l'intera circonferenza vale 2.
Il punto creato con \lstinline{PointOn} non può essere trascinato con il mouse.

 \item [Esempio] Disegna il simmetrico di un punto rispetto ad una retta.

\begin{lstlisting}
asse = pyig.Line(pyig.Point(-4, -11, width=5), 
                 pyig.Point(-2, 12, width=5), 
                 width=3, color='DarkOrange1', name='r')
punto = pyig.Point(-7, 3, width=5, name='P')
perpendicolare = pyig.Orthogonal(asse, punto, width=1)
pyig.PointOn(perpendicolare, -1, width=5, color="DebianRed", name="P'")
\end{lstlisting}

\end{description}

\subsection{\lstinline{ConstrainedPoint}}
\label{sub:geoint_constrainedpoint}

\begin{description}
 \item [Scopo] Punto legato ad un oggetto.
 \item [Sintassi] \lstinline{ConstrainedPoint(obj, parameter)}
 \item [Osservazioni] Per quanto riguarda \lstinline{parameter}, valgono le 
osservazioni fatte per \lstinline{PoinOn}.
Questo punto però può essere trascinato con il mouse pur restando sempre
sull'oggetto. Dato che può essere trascinato con il mouse
ha un colore di default diverso da quello degli altri oggetti.

 \item [Esempio] Circonferenza e proiezioni sugli assi.

\begin{lstlisting}
circ = pyig.Circle(pyig.Point(0, 0, visible=False),
                   pyig.Point(10, 0, visible=False), color="gray10")
cursor = pyig.ConstrainedPoint(circ, .25, name='cursor', width=5)
pyig.Point(cursor.xcoord(), 0, width=5, color='dark turquoise')
pyig.Point(0, cursor.ycoord(), width=5, color='pink')
\end{lstlisting}

\end{description}

\subsection{\lstinline{parameter}}
\label{sub:geoint_parameter}

\begin{description}
 \item [Scopo] I punti legati agli oggetti hanno un metodo che permette di 
ottenere il parametro relativo alla loro posizione.

 \item [Sintassi] \lstinline{<constrained point>.parameter()}
 \item [Osservazioni]
In \lstinline{PointOn} il parametro è fissato nel momento della costruzione
dell'oggetto e non è modificabile. In \lstinline{ConstrainedPoint} il parametro 
può essere variato trascinando il punto con il mouse.

 \item [Esempio] Scrivi i dati relativi a un punto collegato a un oggetto.

\begin{lstlisting}
c0 = pyig.Circle(pyig.Point(-6, 6, width=6), pyig.Point(-1, 5, width=6))
c1 = pyig.Circle(pyig.Point(6, 6, width=6), pyig.Point(1, 5, width=6))
a = pyig.PointOn(c0, 0.5, width=6, name='A')
b = pyig.ConstrainedPoint(c1, 0.5, width=6, name='B')
pyig.VarText(-5, -1, 'ascissa di A: {0}', a.xcoord())
pyig.VarText(-5, -2, 'ordinata di A: {0}', a.ycoord())
pyig.VarText(-5, -3, 'posizione di A: {0}', a.coords())
pyig.VarText(-5, -4, 'parametro di A: {0}', a.parameter())
pyig.VarText(5, -1, 'ascissa di B: {0}', b.xcoord())
pyig.VarText(5, -2, 'ordinata di B: {0}', b.ycoord())
pyig.VarText(5, -3, 'posizione di B: {0}', b.coords())
pyig.VarText(5, -4, 'parametro di B: {0}', b.parameter())
\end{lstlisting}

\end{description}


\subsection{\lstinline{Text}}
\label{sub:geoint_text}

\begin{description}
 \item [Scopo] Crea un testo posizionato in un punto del piano.
 \item [Sintassi] \lstinline{Text(x, y, text, ...)}
%  \item [Osservazioni] 
% \begin{itemize} [noitemsep]
% \item \lstinline{x} e \lstinline{y} sono due numeri interi o razionali 
% relativi; 
% \lstinline{x} è l'ascissa e \lstinline{y} l'ordinata del punto.
% \item \lstinline{text} è la stringa che verrà visualizzata.
% \item Se sono presenti più piani interattivi, si può specificare l'argomento
% \lstinline{iplane} per indicare in quale di questi la scritta deve essere
% visualizzata.
% \end{itemize}

 \item [Esempio] Scrive due frasi.

\begin{lstlisting}
pyig.Text(-2, 2, "Prove di testo blue violet", 
          color='blue violet', width=20)
pyig.Text(-2, -2, "Prove di testo pale green", 
          color='pale green', width=20)
\end{lstlisting}

\end{description}


\subsection{\lstinline{VarText}}
\label{sub:geoint_vartext}

\begin{description}
 \item [Scopo] Crea un testo variabile. 
 Il testo contiene dei ``segnaposto'' che verranno sostituiti con i valori 
prodotti dai dati presenti nel parametro \lstinline{variables}.
 \item [Sintassi] \lstinline{VarText(x, y, text, variables, ...)}
 \item [Osservazioni] ~\\
\begin{itemize} [nosep]
\item \lstinline{text} è la stringa che contiene la parte costante e i 
segnaposto.
\item In genere i \emph{segnaposto} saranno nella forma: ``\{0\}'' che indica a 
Python di convertire in stringa il risultato prodotto dal dato.
\item \lstinline{variables} è un dato o una tupla di dati.
visualizzata.
\end{itemize}

 \item [Esempio] Un testo che riporta la posizione dei un punto.

\begin{lstlisting}
p0 = pyig.Point(7, 3, color='green', width=10, name='A')
pyig.VarText(-4, -3, "Posizione del punto A: ({0}; {1})",
             (p0.xcoord(), p0.ycoord()),
             color='green', width=10)
\end{lstlisting}

\end{description}

% pyig.Text(-7, 13, """PointD""", color='#408040', width=12)
% circ = pyig.Circle(pyig.Point(0, 0, visible=False),
%                    pyig.Point(10, 0, visible=False), color="gray10")
% cursor = pyig.ConstrainedPoint(circ, .25, name='cursor', width=5)
% pyig.Point(cursor.xcoord(), 0, width=5, color='dark turquoise')
% pyig.Point(0, cursor.ycoord(), width=5, color='pink')
% pyig.VarText(-5, -4, "x = {0}", cursor.xcoord(), color='olive drab')
% pyig.VarText(-5, -5, "y = {0}", cursor.ycoord(), color='olive drab')

\subsection{\lstinline{Calc}}
\label{sub:geoint_calc}

\begin{description}
 \item [Scopo] È un oggetto \lstinline{Dato} che contiene il risultato di un 
calcolo.
 \item [Sintassi] \lstinline{Calc(function, variables)}
 \item [Osservazioni]  ~\\
\begin{itemize} [nosep]
\item \lstinline{function} è una funzione Python, al momento del calcolo, alla
funzione vengono passati come argomenti il contenuto di \lstinline{variables}.
\item \lstinline{variables} è un oggetto \lstinline{Data} o una \emph{tupla} 
che contiene oggetti \lstinline{Data}. 
Il risultato è memorizzato all'interno dell'oggetto \lstinline{Calc} e può
essere visualizzato con \lstinline{VarText} o utilizzato per definire la 
posizione di un punto.
\end{itemize}

 \item [Esempio] Calcola il quadrato di un lato e la somma dei quadrati degli 
altri due lati di
un triangolo.

\begin{lstlisting}
circ = pyig.Circle(pyig.Point(2, 4), pyig.Point(-3, 4), width=1)
ip.defwidth = 5
p_a = pyig.PointOn(circ, 0, name="A")
p_b = pyig.PointOn(circ, 1, name="B")
p_c = pyig.Point(-1, 8, name="C")
ab = pyig.Segment(p_a, p_b, color="#40c040")
bc = pyig.Segment(p_b, p_c, color="#c04040")
ca = pyig.Segment(p_c, p_a, color="#c04040")
q1 = pyig.Calc(lambda a: a*a, ab.length())
q2 = pyig.Calc(lambda a, b: a*a+b*b, (bc.length(), ca.length()))
pyig.VarText(-5, -5, "ab^2 = {0}", q1, color="#40c040")
pyig.VarText(-5, -6, "bc^2 + ca^2 = {0}", q2, color="#c04040")
\end{lstlisting}

\end{description}

% pyig.Text(-7, 13, """PointD""", color='#408040', width=12)
% a0 = pyig.Point(0, -2, visible=False, name="0")
% b0 = pyig.Point(1, -2, visible=False, name="1")
% line0 = pyig.Line(a0, b0)
% cursor = pyig.ConstrainedPoint(line0, 1, name='cursor', width=5)
% par = pyig.Calc(lambda x: 0.5*x*x, cursor.parameter())
% pyig.Point(cursor.parameter(), par, width=5, color='dark turquoise')
% pyig.VarText(-5, -4, "x = {0}", cursor.parameter(),
%              color='olive drab')
% pyig.VarText(-5, -5, "y = 0,5*x**2 = {0}", par, color='olive drab')

La Libreria \lstinline{pyig} mette a disposizione molte altre classi e altri 
metodi con cui si possono creare vari oggetti utili nelle costruzioni 
geometriche. 

Per avere informazioni sugli oggetti messi a disposizione dalla libreria 
possiamo dare i seguenti comandi:
 \begin{lstlisting}[numbers=none]
>>> import pyig
>>> help(pyig)
 \end{lstlisting}

Allargando il numero di strumenti che sappiamo usare possiamo risolvere 
problemi più complessi.

\subsection{Riassumendo}
\begin{itemize} [noitemsep]
\item Per affrontare problemi più complessi con la geometria interattiva è 
utile conoscere anche altri strumenti messi a disposizione dalla libreria 
\lstinline{pyig}.
\begin{itemize}
 \item \lstinline{Orthogonal(linea, punto, ...)} 
 per tracciare rette perpendicolari.
 \item \lstinline{Parallel(linea, punto, ...)} 
 per tracciare rette parallele.
 \item \lstinline{MidPoints(punto, punto, ...)} 
 per tracciare il punto medio tra due punti.
 \item \lstinline{MidPoint(segmento, ...)} 
 per tracciare il punto medio di un segmento.
 \item \lstinline{Bisector(angolo, ...)} 
 per tracciare la bisettrice di un angolo.
 \item \lstinline{PointOn(linea, parametro, ...)} 
 per tracciare un punto in una posizione fissata di una linea.
 \item \lstinline{ContrainedPoint(linea, parametro, ...)} 
 per tracciare un punto vincolato ad una linea.
 \item \lstinline{<punto_vincolato>.parameter()} 
 per ottenere il parametro di un punto vincolato ad una linea.
 \item \lstinline{Text(x, y, testo)} 
 per scrivere un testo in una posizione del piano.
 \item \lstinline{VarText(x, y, testo, variabili)} 
 per scrivere un testo con delle parti variabili in una posizione del piano.
 \item \lstinline{Calc(funzione, variabili)} 
 per eseguire dei calcoli.
\end{itemize}
\item Per avere informazioni su tutti gli strumenti messi a disposizione della 
libreria si può consultare il manale distribuito assieme alla libreria stessa 
nella cartella \lstinline{docs} o visualizzare l'help in una finestra della 
shell con le due istruzioni:
 \begin{lstlisting}[numbers=none]
>>> import pyig
>>> help(pyig)
 \end{lstlisting}
\end{itemize}

\section{Insegnare a pyig}
\label{sec:geo_int_insegnare}

\emph{Come insegnare nuove costruzioni a \lstinline{pyig}.}

\subsection{Funzioni}
\label{subsec:geo_int_funzioni}

Partiamo da un problema:

\subsubsection{problema}

Costruire tre triangoli equilateri sui lati di un triangolo.

Sappiamo già costruire triangoli, sappiamo costruire triangoli equilateri, 
quindi la soluzione è abbastanza semplice.

Con riga e compasso.

\begin{procedura}[Triangolo con triangoli equilateri]\label{proc:geoint2_}
  Costruire tre triangoli equilateri sui lati di un triangolo:
  \begin{enumerate} [nosep]
    \item 
    Traccia i punti A, B, C.
    \item 
    Traccia il triangolo ABC.
    \item 
    Traccia il triangolo equilatero di lato AB.
    \item 
    Traccia il triangolo equilatero di lato BC.
    \item 
    Traccia il triangolo equilatero di lato CA.  
  \end{enumerate}
\end{procedura}

Il problema è che per tracciare i vari triangoli equilateri devo ripetere una 
sequenza di istruzioni (sempre le stesse) per tre volte. In questo caso posso 
armarmi di pazienza e utilizzando il ``copia-incolla'' me la posso cavare 
abbastanza presto. Ma se dovessi ripetere la stessa costruzione decine o 
centinaia di volte? I computer sono fatti apposta per ripetere istruzioni 
banali e gli informatici hanno inventato metodi proprio per evitare di doverlo 
fare loro.

Dobbiamo generalizzare la soluzione del sottoproblema di disegnare un triangolo 
equilatero partendo da un suo lato. Per farlo dobbiamo:
\begin{itemize} [nosep]
 \item trovare un nome per questa funzione, in questo caso,  dato che il 
risultato è un triangolo equilatero la chiameremo: \lstinline{triequi};
 \item individuare i dati da cui parte la costruzione, in questo caso sono due 
punti estremi del primo lato che potremmo chiamare:
\lstinline{p_0} e \lstinline{p_1}.
 \item decidere quale deve essere il risultato, in questo caso un poligono.
\end{itemize}

Di seguito riporto un programma che risolve il problema seguito dal suo 
commento.

\subsubsection{copiare}

\lstinputlisting{\folder src/12triequi1.py}

\subsubsection{capire}

\begin{description}
 \item [linea 1] 
 L'intestazione del programma che contiene le solite tre informazioni.
 \item [linee 2-5]
 Descrizione del problema risolto da questo programma.
 \item [linea 7]
 Lettura della libreria.
 \item [linee 9-15]
 Definizione delle funzioni utilizzate dal programma principale.
\begin{description}
 \item [linea 10]
 Prima riga della definizione di una funzione. È formata da queste parti:
 \begin{itemize} [nosep]
  \item la parola riservata ``\lstinline{def}'';
  \item il nome della funzione in questo caso: ``\lstinline{triequi}'';
  \item una sequenza di parametri:
  \begin{itemize}
   \item \lstinline{p_0} conterrà il primo estremo del lato,
   \item \lstinline{p_1} conterrà il secondo estremo del lato,
   \item \lstinline{**kargs} conterrà tutti gli altri argomenti passati alla 
funzione.
  \end{itemize}
  \item il carattere duepunti: ``\lstinline{:}''.
  \end{itemize}
 \item [linea 11]
 La \lstinline{docstring}, un testo che descrive cosa fa la funzione.
 \item [linea 12-14]
 Vengono create le solite due circonferenze e la loro intersezione.
 \item [linea 15]
 Viene creato il poligono che ha per vertici i due punti dati e l'intersezione 
e questo poligono viene dato come risultato della funzione.
\end{description}
 \item [linea 18-26]
 Programma principale.
\begin{description}
 \item [linea 18]
 Creazione del piano interattivo.
 \item [linea 19-21]
 Creazione dei tre vertici del triangolo.
 \item [linea 22-24]
 Creazione dei tre triangoli equilateri.
 \item [linea 26]
 Attivazione della finestra grafica.
\end{description}
\end{description}

\subsubsection{osservazioni}

\begin{itemize} [noitemsep]
 \item La definizione di una funzione è composta da un'intestazione seguita da 
un blocco di istruzioni. Il \emph{blocco}di istruzioni è indicato 
dall'\emph{indentazione} cioè dal rientro: tutte le istruzioni \emph{indentate} 
fanno parte del blocco.
 \item Quando la funzione \lstinline{triequi} viene chiamata le vengono passati 
due argomenti che verranno messi nei due parametri \lstinline{p_0} e
\lstinline{p_1} più eventuali altri argomenti che vengono riuniti tutti dentro 
il contenitore \lstinline{kargs}.
 \item La funzione \lstinline{triequi}  utilizza i due parametri 
\lstinline{p_0} 
e \lstinline{p_1} mentre non si interessa di cosa sia contenuto in 
\lstinline{kargs} e passa il contenuto direttamente alla funzione che crea il 
poligono.
 \item Il risparmio di codice che si ottiene scrivendo una funzione, nel nostro 
caso, è abbastanza ridotto, ma se dovessi creare centinaia di triangoli 
equilateri, la differenza sarebbe sensibile. Non solo, ma da un punto di vista 
informatico, avere del codice duplicato non è una buona idea. Supponiamo di 
aver trovato un errore nella procedura che costruisce il triangolo, o di voler 
modificarla, se ho scritto una funzione so esattamente dove modificare e i 
cambiamenti valgono per tutto il programma, se ho ripetuto il codice in diversi 
punti del programma devo trovarli tutti e eseguire in ognuno gli stessi 
cambiamenti altrimenti il programma diventa incoerente.
\end{itemize}

\subsubsection{modificare}

\begin{enumerate} [noitemsep]
 \item Modifica la funzione \lstinline{triequi} in modo che vengano 
visualizzati, con spessore 1 gli elementi di costruzione.
 \item Modifica il programma in modo che la finestra grafica abbia un titolo e 
che non visualizzi gli assi e la griglia.
 \item Fa in maniera che sia colorato anche l'interno del triangolo di partenza.
 \item Cambia lo spessore e il colore dei punti base e dei triangoli equilateri.
\end{enumerate}

\subsection{Riassumendo}
\begin{itemize} [noitemsep]
\item Quando un problema diventa complesso è buona norma suddividerlo in parti.
Una funzione è un pezzo di programma che risolve una parte di problema.
\item La sintassi per la definizione di una funzione è:
\begin{lstlisting}[numbers=none]
def <nome_funzione>(<parametri>):
    <istruzioni>
\end{lstlisting}
\item Una funzione può dare come risultato un oggetto se c'è l'istruzione:
\begin{lstlisting}[numbers=none]
    return <oggetto>
\end{lstlisting}
\item Un programma Python generalmente è composto da queste sezioni:
\begin{lstlisting}[numbers=none]
# Intestazioni
# Lettura delle librerie
# Definizione delle funzioni
# Programma principale
\end{lstlisting}
\end{itemize}

\section{Iterazione}
\label{subsec:geo_int_}

\emph{Come sfruttare qualche ``trucco'' di Python.}

Osservando il programma precedente possiamo vedere che ci sono alcune righe 
praticamente uguali e questo non va bene. Python mette a disposizione 
un'istruzione che permette di \emph{iterare}, cioè di ripetere, delle 
istruzioni per tutti gli elementi di una sequenza. Vediamo come è possibile 
eliminare le operazioni ripetute nel nostro programma. 

\subsubsection{copiare}

Riporto solo il programma principale, il resto rimane invariato.

\lstinputlisting[firstline=17, firstnumber=17]{\folder src/12triequi2.py}

\subsubsection{capire}

\begin{description}
 \item [linee -21] 
 Fin qui il programma è invariato
 \item [linea 22] 
 È l'intestazione di un ciclo. Le due variabili \lstinline{p_0} e 
\lstinline{p_1} assumono i valori delle coppie contenute nella sequenza 
delimitata dalla parentesi tonda. Quindi nella prima iterazione \\
\lstinline{p_0} sarà uguale a \lstinline{p_a} \quad e \quad
\lstinline{p_1} sarà uguale a \lstinline{p_b}, \\
nella seconda iterazione \\
\lstinline{p_0} sarà uguale a \lstinline{p_b} \quad e \quad
\lstinline{p_1} sarà uguale a \lstinline{p_c}, \\
nella terza \\
\lstinline{p_0} sarà uguale a \lstinline{p_c} \quad e \quad
\lstinline{p_1} sarà uguale a \lstinline{p_a}. \\
Poi non ci sono più elementi 
nella sequenza e il ciclo termina.
 \item [linea 23] 
 Per ogni interazione vengono eseguite tutte le istruzioni del blocco indentato 
che segue il comando \lstinline{for}, in questo caso una sola che crea un 
triangolo equilatero. Ma quest'unica istruzione viene ripetuta tre volte.
\end{description}

\subsubsection{osservazioni}

\begin{itemize} [noitemsep]
 \item Tre righe di programma sono diventate due, ma ho tolto delle parti 
ripetitive e anche dovessi disegnare centinaia di triangoli equilateri lo 
potrei fare sempre con due righe di codice.
\end{itemize}

\subsubsection{modificare}

\begin{enumerate} [noitemsep]
 \item Cambia ora le caratteristiche dei triangoli, su quante linee di codce 
devi intervenire.
 \item È possibile, con questo meccanismo, modificare anche le caratteristiche 
dei singoli triangoli equilateri. L'intestazione del ciclo potrebbe essere:
\begin{lstlisting}[firstnumber=22]
for p_0, p_1, intcol in ((p_a, p_b, "orchid1"), (p_b, p_c, "orchid2"),
                         (p_c, p_a, "orchid3")): 
...
\end{lstlisting}
In quale variabile va a finire il nome del colore? Come dovrà cambiare 
l'istruzione che si trova nel corpo del ciclo?
 \item In questo caso la lunghezza del programma non è diminuita, ma abbiamo 
operato un altro miglioramento importante, abbiamo separato i dati 
dall'algoritmo. Nell'intestazione c'è una sequenza che contiene i dati, il 
blocco di istruzioni della sequenza è un algoritmo che lavora su dei dati 
ricevuti dall'eterno.
 \item Modifica il programma in modo che i tre triangoli abbiano diversi anche 
i colori del bordo.
\end{enumerate}

È possibile sintetizzare anche le tre righe~19-21 in un ciclo, ma qui dobbiamo 
usare un metodo un po' più complesso, infatti, non basta creare i tre punti, 
dobbiamo anche mantenere un riferimento a ciascuno di loro per poterli 
utilizzare in seguito.

\subsubsection{copiare}

Anche qui riporto solo il programma principale.

\lstinputlisting[firstline=17, firstnumber=17]{\folder src/12triequi3.py}

\subsubsection{capire}

\begin{description}
 \item [linea 19] 
 All'identificatore \lstinline{vertici} viene associata una sequenza di coppie 
di numeri, le coordinate dei punti.
 \item [linea 22] 
 Questa linea è complessa, analizziamo le sue parti:
 \begin{description} [nosep]
  \item [p\_a, p\_b, p\_c]
  sono tre identificatori separati da una virgola a questi saranno associati i 
tre oggetti prodotti dalla parte di istruzione che si trova a destra 
dell'uguale.
  \item [(pyig.Point(x, y, width=6) for x, y in vertici)]
  produce una sequenza di punti le cui coordinate \lstinline{x} e 
\lstinline{y} sono ricavate scorrendo la sequenza associata a 
\lstinline{vertici}.
 \end{description}
 Questa linea realizza un ciclo \lstinline{for}, ma raccoglie tutti i risultati 
in una sequenza. 
\end{description}

\subsubsection{osservazioni}

\begin{itemize} [noitemsep]
 \item Anche in questo caso, nel nostro programma, abbiamo risparmiato una sola 
riga su tre, ma se \lstinline{vertici} fosse una sequenza di~200 coppie di 
numeri, potremmo ottenere una sequenza di~200 punti con un'unica istruzione 
(non male).
 \item Se un programma richiede che siano costruiti diversi punti base da cui 
partire per il resto della costruzione, questo può essere un buon modo per 
crearli.
\end{itemize}

\subsubsection{modificare}

\begin{enumerate} [noitemsep]
 \item Cambia il colore dei punti base.
 \item Cambia il programma in modo che i triangoli siano disegnati sui lati di 
un quadrilatero.
 \item Cambia il programma in modo che i triangoli siano disegnati sui lati di 
un pentagono.
\end{enumerate}


\subsection{Poligoni regolari}
\label{subsec:geo_int_}

\emph{Come combinare una caratteristica di pyig con una di Python.}

Problema vogliamo disegnare un ettagono regolare.

\subsubsection{copiare}

\lstinputlisting{\folder src/13polreg.py}

\subsubsection{capire}

\begin{description}
 \item [linee 8] 
 Meglio mettere il numero di lati desiderati in una variabile. In questo caso, 
dato che non viene cambiata durante tutto il programma, la considereremo una 
costante e per convenzione la scriviamo tutta in maiuscolo.
 \item [linee 12-13] 
 Viene creata una circonferenza
 \item [linea 14] 
 Ad ogni linea di \lstinline{pyig} è associata una metrica. Per convenzione una 
circonferenza è lunga~2. In realtà sarebbe \(2\pi\), ma chi ha creato 
\lstinline{pyig} ha pensato fosse più semplice usare la costante 2 come 
lunghezza della circonferenza. Dividendo la lunghezza della circonferenza per 
il numero di lati del poligono, trovo la lunghezza di ogni singolo arco.
 \item [linee 15-16]
 Queste due linee costruiscono una lista di punti posizionati sulla 
circonferenza. Analizziamola partendo dal fondo.
\begin{description}
 \item [range(NUMLATI)] 
 restituisce la sequenza di numeri da zero a NUMLATI 
meno uno, nel nostro caso: (0, 1, 2, 3, 4, 5, 6).
 \item [for cont]
 Esegue un ciclo in cui \lstinline{cont} assume uno alla volta i valori da~0 
 a~6.
 \item [pyig.PointOn(c\_cp, arco*cont)]
 Crea i punti fissati nella posizione \lstinline{arco*cont} sulla circonferenza 
\lstinline{c_cp}.
 \item [vertici = ]
 Associa all'identificatore \lstinline{vertici} la lista di punti appena 
creata.
\end{description}
 \item [linea 17]
 Costruisce il poligono.
\end{description}

\subsubsection{osservazioni}

\begin{itemize} [noitemsep]
 \item Quando è possibile conviene usare ``costanti'' al posto di numeri.
 \item La costruzione di liste (``listcomprehension'') è uno strumento di 
Python molto potente, anche se non di immediata comprensione.
 \item In \lstinline{pyig} le circonferenze hanno sempre ``lunghezza''~2 e le 
rette hanno una lunghezza infinita dove però il primo punto di costruzione ha 
coordinata~0 e il secondo coordinata~1.
\end{itemize}

\subsubsection{modificare}

\begin{enumerate} [noitemsep]
 \item Modifica colori e spessori degli elementi dell'ettagono.
 \item Rendi invisibile la circonferenza di costruzione.
 \item Disegna poligoni regolari con un diverso numero di lati.
 \item Puoi disegnare un bilatero regolare? e un monolatero? e un zerolatero?
\end{enumerate}

\subsection{Riassumendo}
\begin{itemize} [noitemsep]
\item L'istruzione \lstinline{range(<num>)} restituisce una sequenza di numeri 
che vanno da~\(0\) a \(\text{<num>}-1\).
\item Un ciclo \lstinline{for} itera un'operazione per ogni elemento della 
sequenza a cui è applicato.
\item È possibile costruire delle sequenze per mezzo della 
``listcomprehension''.
\end{itemize}

\section{Altri problemi}

\begin{enumerate} [noitemsep]
\item Scrivi la funzione che disegni un parallelogramma dati tre vertici.
\item Scrivi la funzione che disegni un quadrato dati due vertici consecutivi.
\item Scrivi la funzione che disegni un quadrato dati due vertici opposti.
\item Scrivi la funzione che disegni un rettangolo dati una semiretta e due 
segmenti.
\item Scrivi la funzione che disegni un rombo dati una semiretta e due 
segmenti.
\item Scrivi la funzione che disegni un esagono dati due vertici consecutivi.
\item Scrivi un programma che disegni nel primo quadrante un pentagono, nel 
secondo un esagono, nel terzo un ettagono e nel quarto un ottagono tutti con lo 
stesso raggio.
\item Scrivi una funzione che, dati tre vertici, restituisca il baricentro del 
triangolo.
\item Scrivi una funzione che, dati tre vertici, restituisca il circocentro del 
triangolo.
\item Scrivi una funzione che, dati tre vertici, restituisca l'incentro del 
triangolo.
\item Scrivi una funzione che, dati tre vertici, restituisca l'ortocentro del 
triangolo.
\item Scrivi la funzione che, dati una circonferenza e un punto, disegni le 
tangenti alla circonferenza per il punto.
\item Scrivi la funzione che, dati un asse e un punto, restituisca il 
simmetrico del punto rispetto all'asse.
\item Scrivi la funzione che, dati un asse e un centro, restituisca il 
simmetrico del punto rispetto al centro.
\item Scrivi la funzione che, dati un punto e un vettore, e restituisca 
il traslato del punto.
\item Scrivi la funzione che, dati un punto, un centro e un angolo,  
e restituisca punto ruotato.
\item Scrivi la funzione che, dati un punto, un centro e un numero,  
e restituisca punto omotetico.
\item Scrivi un programma che illustri il teorema di Pitagora.
\item Scrivi la funzione che dati gli estremi di un segmento ne individui la 
sua sezione aurea.
\end{enumerate}
