% (c) 2017 Bruno Stecca
% (c) 2017 Daniele Zambelli - daniele.zambelli@gmail.com
% 
% Tutti i grafici per il capitolo relativo agli integrali
% 
% 

\newcommand{\areasottesasegmento}{% Area sottesa a un segmento
  \disegno{
    \coordinate (a) at (2.5, 4);
    \coordinate (b) at (4.3, 5.2);
    \coordinate (o) at (0, 0);
    \fill [top color=green!30!black!20,bottom color=green!50!black!10] 
    (a) -- (b) |- (o) -| (a) -- cycle;
    \rcom{-1}{+6}{-1}{+6}{gray!50, very thin, step=1}
    \filldraw [thick, color=blue!50!black] 
        (a) circle (.5pt) node [above] {\(A\)} -- 
        (b) circle (.5pt) node [above] {\(B\)};
    \begin{scope}[thin]
    \draw (a -| o) node [left] {\(y_A\)} -- (a) --
          (a |- o) node [below] {\(x_A\)}
          (b -| o) node [left] {\(y_B\)} -- (b) --
          (b |- o) node [below] {\(x_B\)};
    \end{scope}
    }
}

\newcommand{\areasottesasegmenti}{% Area sottesa a un segmento
  \disegno{
    \rcom{-1}{+6}{-1}{+8}{gray!50, very thin, step=1}
    \foreach \xa/\y in {2/5, 2.5/4, 3/6, 3.5/3, 4/7, 4.5/5}{
      \fill [top color=green!30!black!20,bottom color=green!50!black!10] 
      (\xa, \y) -- (\xa+.5, \y) -- (\xa+.5, 0.015) -- (\xa, 0.015) -- cycle;
      \draw [thick, cap=butt] (\xa, \y) -- (\xa+.5, \y);
      \draw [very thin](\xa,\y) -- (\xa,0.018) (\xa+.5,\y) -- (\xa+.5,0.018);
    }
  }
}

\def \definizionilinea{
    \def \xi{-0.3};
    \def \yi{6.7};
    \def \xf{10.3};
    \def \yf{0.5};
    \def \xa{0.4};
    \def \ya{4.9};
    \def \xb{8.8};
    \def \yb{5.55};
    \coordinate (i) at (\xi, \yi);
    \coordinate (f) at (\xf, \yf);
    \coordinate (a) at (\xa, \ya);
    \coordinate (b) at (\xb, \yb);
    \coordinate (ctrli) at (3, -4);
    \coordinate (ctrlf) at (7, 17);
    \def \linea{(i) .. controls (ctrli) and (ctrlf) .. (f)}
    \node at (9.8, 3.5) {\(f\)};
}

\newcommand{\areasottesacurva}{% Area sottesa ad una curva.
  \disegno{
    \definizionilinea
    \begin{scope}
      \clip (\xa, 0) rectangle (\xb, 8);
      \fill [top color=green!30!black!20, bottom color=green!50!black!10] 
      \linea -- (\xf, 0) -- (\xi, 0) -- cycle;
    \end{scope}

    \rcom{0}{+10}{0}{+8}{gray!50, very thin, step=1}
    
    \begin{scope}
      \clip (\xi, 0) rectangle (\xf, 8);
      \draw [thick] \linea;
    \end{scope}

    \begin{scope}
    \draw [-] (a) -- (\xa, 0) node [below] {\(a\)};
    \draw [-] (b) -- (\xb, 0) node [below] {\(b\)};
    \node at (4.5, 2.5) {\(\mathcal{A}_f(a;~b)\)};
    \end{scope}
    }
}

\newcommand{\propradditiva}{% Propr. additiva per area sottesa ad una curva.
  \disegno{
    \definizionilinea
    \def \xc{5.5};
    \def \yc{6.25};
    \coordinate (c) at (\xc, \yc);

    \begin{scope}
      \clip (\xa, 0) rectangle (\xb, 8);
      \fill [top color=green!30!black!20, bottom color=green!50!black!10] 
      \linea -- (\xf, 0) -- (\xi, 0) -- cycle;
    \end{scope}

    \rcom{0}{+10}{0}{+8}{gray!50, very thin, step=1}
    
    \begin{scope}
      \clip (\xi, 0) rectangle (\xf, 8);
      \draw [thick] \linea;
    \end{scope}

    \begin{scope}
    \draw [-] (a) -- (\xa, 0) node [below] {\(a\)};
    \draw [-] (c) -- (\xc, 0) node [below] {\(t\)};
    \draw [-] (b) -- (\xb, 0) node [below] {\(b\)};
    \node at (3, 2.5) {\(\mathcal{A}_f(a;~t)\)};
    \node at (7.2, 2.5) {\(\mathcal{A}_f(t;~b)\)};
    \end{scope}
    }
}

\newcommand{\arearettangolare}[4]{% Rettangolo.
  \disegno{
    \definizionilinea
    \def \xm{#1};
    \def \ym{#2};
    \def \lm{#3};
    \def \h{#4};

    \begin{scope}
      \clip (\xa, 0) rectangle (\xb, 8);
      \fill [top color=green!30!black!20, bottom color=green!50!black!10] 
      (\xa, 0) rectangle (\xb, \ym);
    \end{scope}

    \rcom{0}{+10}{0}{+8}{gray!50, very thin, step=1}
    
    \begin{scope}
      \clip (\xi, 0) rectangle (\xf, 8);
      \draw [thick] \linea;
    \end{scope}

    \begin{scope}[thin]
    \draw  (\xa, 0) node [below] {\(a\)} -- (\xa, \ym) -- 
           (\xb, \ym)  -- (\xb, 0) node [below] {\(b\)};
    \draw [densely dotted] (\xa, \ym) -- (0, \ym) node [left] {\(\h\)};
    \node at (4.5, 2.5) {\(\lm \cdot (b -a))\)};
%     \node at (4.5, 2.5) {\(\mathcal{A}_{\lm}(a;~b)\)};
    \end{scope}
    }
}

\newcommand{\areaminore}{% Rettangolo minore.
  \arearettangolare{2}{3.49}{m}{m}
}

\newcommand{\areamaggiore}{% Rettangolo maggiore.
  \arearettangolare{7}{7.12}{M}{M}
}

% \newcommand{\velocita}{% Velocità media
%   \def \fun{6*sqrt(x)-x-3}
%   \def \funa{6*sqrt(\x)-\x-3}
%   \def \mix{0}
%   \def \max{7}
%   \def \miy{0}
%   \def \may{7}
%   \def \xa{1.5}
%   \def \xb{4}
%   \def \x{\xa} \pgfmathparse{\funa} \let\ya\pgfmathresult
%   \def \x{\xb} \pgfmathparse{\funa} \let\yb\pgfmathresult
%   \pgfmathparse{(\yb - \ya) / (\xb - \xa)} \let\m\pgfmathresult
%   \disegno{
%     \rcomvar{\mix}{\max}{\miy}{\may}{gray!50, very thin, step=1}{t}{s}
%     \tkzInit[xmin=\mix-0.3,xmax=\max+0.3,ymin=\miy-0.3,ymax=\may+0.3]
%     \tkzFct[ultra thick, color=red!50!black, domain=\mix:\max+0.3]{\fun}
%     \fill [fill=red!50!black] 
%           (\xa, \ya) circle(2pt) node [above left]{\(A\)} 
%           (\xb, \yb) circle(2pt) node [above left]{\(B\)};
%     \tkzFct[thick, color=blue!50!black, domain=\mix:\max+0.3]
%            {\m*(x - \xa) + \ya}
%     \draw [thick, dashed, blue!50!black]
%           (0, \ya) node [left]{\(s(t_1)\)} -- (\xa, \ya) -- 
%           (\xa, 0) node [below]{\(t_1\)}
%           (0, \yb) node [left]{\(s(t_2)\)} -- (\xb, \yb) -- 
%           (\xb, 0) node [below]{\(t_2\)}
%           (\xa, \ya) -- (\xb, \ya) node [midway, below] {\(\Delta t\)}
%           (\xb, \yb/2 + \ya/2) node [right] {\(\Delta s\)};
%   }
% }

% \newcommand{\sommariemannesea}[1]{% Somma di Riemann per gli esercizi.
%   \disegno{
%   \def \fun{.5*x +3}
%   \def \mix{-2}
%   \def \max{7}
%   \def \miy{0}
%   \def \may{6}
%   \def \Dt{#1};
%   \def \xa{-2};
%   \def \xb{+5};
%   \def \lab{y=\frac{1}{2}x +3};
% %     \def \xi{-0.3};
% %     \def \xf{8.3};
% %     \begin{scope}
% %       \clip (0, 0) rectangle (\xf, 10.3);
% %       \fill [top color=green!30!black!20, bottom color=green!50!black!10] 
% %       (\xi, \m*\xi) -- (\xf,  \m*\xf) -- (\xf, 0) -- (\xi, 0) -- cycle;
% %     \end{scope}
%     \rcom{\mix}{\max}{\miy}{\may}{gray!50, very thin, step=1}
%     \tkzInit[xmin=\mix-0.3,xmax=\max+0.3,ymin=\miy-0.3,ymax=\may+0.3]
%     \tkzFct[ultra thick, color=black, domain=\mix-0.3:\max+0.3]{\fun}
% %     \begin{scope}
% %       \clip (-0.3, -0.3) rectangle (\xf, 10.3);
% %       \draw [thick] (\xi, \m*\xi) -- (\xf,  \m*\xf);
%     \node at (2.8, 5.5) {\(\lab\)};
% %     \end{scope}
% % 
% %     \draw [-] (\xa, \m*\xa) -- (\xa, 0) node [below, yshift=-4pt] {\(a\)};
% %     \foreach \x in {\xxi,...,\xxf}
% %       \draw [-] (\x, \m*\x) -- (\x, 0) node [below, yshift=-2pt] {\(\x\)};
%   }
% }

% Da ristrutturare completamente passando la funzione invece che 
% la lista delle altezze
\newcommand{\fillrettangoli}[6]{% Rettangoli retinati
  \def \dxx{#1}
  \def \xxa{#2};
  \def \altezze{#3}
  \def \xxh{#4};
  \def \yyh{#5};
  \def \xxb{#6};
  \begin{scope}
  \foreach \yy [count = \i] in \altezze{
    \fill [top color=green!30!black!20,bottom color=green!50!black!10]
          (\xxa+\i*\dxx-\dxx, \yy) rectangle (\xxa+\i*\dxx, 0);}
  \fill [top color=green!30!black!20,bottom color=green!50!black!10]
        (\xxh, \yyh) rectangle (\xxb, 0);
  \end{scope}
}

\newcommand{\rettangoli}[6]{% Contorno dei rettangoli
  \def \dxx{#1}
  \def \xxa{#2};
  \def \altezze{#3}
  \def \xxh{#4};
  \def \yyh{#5};
  \def \xxb{#6};
  \begin{scope}
  \foreach \yy [count = \i] in \altezze{
    \draw (\xxa+\i*\dxx-\dxx, \yy) rectangle (\xxa+\i*\dxx, 0);}
  \draw (\xxh, \yyh) rectangle (\xxb, 0);
  \end{scope}
}

\newcommand{\sommariemann}{% Somma di Riemann.
  \def \Dt{1}
  \disegno{
    \definizionilinea
    \fillrettangoli{\Dt}{.4}{4.9, 3.65, 3.57, 4.17, 5.15, 6.2, 6.9, 7.06}
                   {8.4}{6.25}{8.8}
    \rcom{0}{+10}{0}{+8}{gray!50, very thin, step=1}
    \rettangoli{1}{.4}{4.9, 3.65, 3.57, 4.17, 5.15, 6.2, 6.9, 7.06}
                   {8.4}{6.25}{8.8}
    \begin{scope}
      \clip (\xi, 0) rectangle (\xf, 8);
      \draw [thick] \linea;
    \end{scope}
    \node at (\xa, 0) [below, yshift=-1.2em] {\(a\)};
    \node at (\xb, 0) [below, yshift=-1em] {\(b\)};
    \foreach \n in {0,1,2}
      \node at (\xa+\n*\Dt, 0) [below] {\(t_\n\)};
    \foreach \n in {3.5,6.5}
      \node at (\xa+\n*\Dt, 0) [below, yshift=-6pt] {\(\dots\)};
    \node at (\xa+5*\Dt, 0) [below] {\(t_i\)};
    \node at (\xa+8*\Dt, 0) [below] {\(t_h\)};

    }
}

\newcommand{\sommariemannesempa}{% Somma di Riemann per gli esempi.
  \def \fun{.5*x +3}
  \def \mix{-2.5}
  \def \max{6}
  \def \miy{0}
  \def \may{6}
  \def \Dt{1}
  \def \xa{-2}
  \def \xb{+5}
  \def \lab{y=\frac{1}{2}x +3}
  \disegno{
    \fillrettangoli{\Dt}{\xa}{2, 2.5, 3, 3.5, 4, 4.5}{4}{5}{5}
    \rcom{\mix}{\max}{\miy}{\may}{gray!50, very thin, step=1}
    \tkzInit[xmin=\mix-0.3,xmax=\max+0.3,ymin=\miy-0.3,ymax=\may+0.3]
    \tkzFct[ultra thick, color=black, domain=\mix-0.3:\max+0.3]{\fun}
    \rettangoli{\Dt}{\xa}{2, 2.5, 3, 3.5, 4, 4.5}{4}{5}{5}
    \node at (\xa, 0) [below, yshift=-1.2em] {\(a\)};
    \node at (\xb, 0) [below, yshift=-1em] {\(b\)};
    \foreach \n in {0,...,7}{
      \node at (\xa+\n*\Dt, 0) [below] {\(t_\n\)};}
    \draw (5, 5.5) -- (5, 0);
  }
}

\newcommand{\sommariemannesempb}{% Somma di Riemann per gli esempi.
  \def \fun{.5*x +3}
  \def \mix{-2.5}
  \def \max{6}
  \def \miy{0}
  \def \may{6}
  \def \Dt{2}
  \def \xa{-2}
  \def \xb{+5}
  \def \lab{y=\frac{1}{2}x +3}
  \disegno{
    \fillrettangoli{\Dt}{\xa}{2, 3, 4}{4}{5}{5}
    \rcom{\mix}{\max}{\miy}{\may}{gray!50, very thin, step=1}
    \tkzInit[xmin=\mix-0.3,xmax=\max+0.3,ymin=\miy-0.3,ymax=\may+0.3]
    \tkzFct[ultra thick, color=black, domain=\mix-0.3:\max+0.3]{\fun}
    \rettangoli{\Dt}{\xa}{2, 3, 4}{4}{5}{5}
    \node at (\xa, 0) [below, yshift=-1.2em] {\(a\)};
    \node at (\xb, 0) [below, yshift=-1em] {\(b\)};
    \foreach \n in {0,...,3}{
      \node at (\xa+\n*\Dt, 0) [below] {\(t_\n\)};}
    \draw (5, 5.5) -- (5, 0);
  }
}

\newcommand{\sommariemannesempc}{% Somma di Riemann per gli esempi.
  \def \fun{.25*x*x -x +2}
  \def \mix{0}
  \def \max{9}
  \def \miy{0}
  \def \may{10}
  \def \Dt{2}
  \def \xa{1}
  \def \ya{1}
  \def \xb{+8}
  \def \yb{+10}
%   \def \alt{1.25, 1.25, 3.25}
  \disegno{
    \fillrettangoli{\Dt}{\xa}{1.25, 1.25, 3.25}{7}{7.25}{8}
    \rcom{\mix}{\max}{\miy}{\may}{gray!50, very thin, step=1}
    \tkzInit[xmin=\mix-0.3,xmax=\max+0.3,ymin=\miy-0.3,ymax=\may+0.3]
    \tkzFct[ultra thick, color=black, domain=\mix-0.3:\max+0.3]{\fun}
    \rettangoli{\Dt}{\xa}{1.25, 1.25, 3.25}{7}{7.25}{8}
    \node at (\xa, 0) [below, yshift=-1.2em] {\(a\)};
    \node at (\xb, 0) [below, yshift=-1em] {\(b\)};
    \foreach \n in {0,...,3}{
      \node at (\xa+\n*\Dt, 0) [below] {\(t_\n\)};}
    \draw (\xa, \ya) -- (\xa, 0) (\xb, \yb) -- (\xb, 0);
  }
}

\newcommand{\sommariemannesempd}{% Somma di Riemann per gli esempi.
  \def \fun{.25*x*x -x +2}
  \def \mix{0}
  \def \max{9}
  \def \miy{0}
  \def \may{10}
  \def \Dt{2}
  \def \xa{1}
  \def \ya{1}
  \def \xh{+7}
  \def \xb{+8}
  \def \yb{+10}
%   \def \alt{1.25, 1.25, 3.25}
  \disegno{
    \fillrettangoli{\Dt}{\xa}{1.25, 3.25, 7.25}{\xh}{\yb}{\xb}
    \rcom{\mix}{\max}{\miy}{\may}{gray!50, very thin, step=1}
    \tkzInit[xmin=\mix-0.3,xmax=\max+0.3,ymin=\miy-0.3,ymax=\may+0.3]
    \tkzFct[ultra thick, color=black, domain=\mix-0.3:\max+0.3]{\fun}
    \rettangoli{\Dt}{\xa}{1.25, 3.25, 7.25}{\xh}{\yb}{\xb}
    \node at (\xa, 0) [below, yshift=-1.2em] {\(a\)};
    \node at (\xb, 0) [below, yshift=-1em] {\(b\)};
    \foreach \n in {0,...,3}{
      \node at (\xa+\n*\Dt, 0) [below] {\(t_\n\)};}
    \draw (\xa, \ya) -- (\xa, 0) (\xb, \yb) -- (\xb, 0);
  }
}

\newcommand{\sommariemannesempe}{% Somma di Riemann per gli esempi.
  \def \fun{.25*x*x -x +2}
  \def \mix{0}
  \def \max{9}
  \def \miy{0}
  \def \may{10}
  \def \Dt{2}
  \def \xa{1}
  \def \ya{1}
  \def \xh{+7}
  \def \xb{+8}
  \def \yb{+8.5}
%   \def \alt{1.25, 1.25, 3.25}
  \disegno{
    \fillrettangoli{\Dt}{\xa}{1, 2, 5}{\xh}{\yb}{\xb}
    \rcom{\mix}{\max}{\miy}{\may}{gray!50, very thin, step=1}
    \tkzInit[xmin=\mix-0.3,xmax=\max+0.3,ymin=\miy-0.3,ymax=\may+0.3]
    \tkzFct[ultra thick, color=black, domain=\mix-0.3:\max+0.3]{\fun}
    \rettangoli{\Dt}{\xa}{1, 2, 5}{\xh}{\yb}{\xb}
    \node at (\xa, 0) [below, yshift=-1.2em] {\(a\)};
    \node at (\xb, 0) [below, yshift=-1em] {\(b\)};
    \foreach \n in {0,...,3}{
      \node at (\xa+\n*\Dt, 0) [below] {\(t_\n\)};}
    \draw (\xa, \ya) -- (\xa, 0) (\xb, \yb) -- (\xb, 0);
  }
}

\newcommand{\riemanninferiore}{% Somma di Riemann inferiore.
  \disegno{
    \definizionilinea
    \def \dx{1};
%     \def \ym{2};
%     \def \lm{3};

    \foreach \xa/\y in {0.4/3.66, 1.4/3.45, 2.4/3.57, 3.4/4.17, 4.4/5.15, 
                        5.4/6.15, 6.4/6.9, 7.4/6.25}{
      \fill [top color=green!30!black!20,bottom color=green!50!black!10]
      (\xa, \y) -- (\xa+\dx, \y) -- (\xa+\dx, 0.015) -- (\xa, 0.015) -- cycle;
      \draw [thick, cap=butt] (\xa, \y) -- (\xa+\dx, \y);
      \draw [very thin](\xa,\y) -- (\xa,0.015) (\xa+\dx,\y) -- 
                       (\xa+\dx,0.015);
    }
    \fill [top color=green!30!black!20,bottom color=green!50!black!10]
      (8.43, 5.55) -- (\xb, 5.55) -- (\xb, 0.015) -- (8.43, 0.015) -- cycle;

    \draw [thick, cap=butt] (8.4, 5.55) -- (\xb, 5.55);
    \draw [very thin] (\xb, 5.55) -- (\xb, 0.015);

    \rcom{0}{+10}{0}{+8}{gray!50, very thin, step=1}

    \begin{scope}
      \clip (\xi, 0) rectangle (\xf, 8);
      \draw [thick] \linea;
    \end{scope}
    
    \node at (\xa, 0) [below, yshift=-1.2em] {\(a\)};
    \node at (\xb, 0) [below, yshift=-1em] {\(b\)};
    \foreach \n in {0,1,2}
      \node at (\xa+\n*\dx, 0) [below] {\(t_\n\)};
    \foreach \n in {3.5,6.5}
      \node at (\xa+\n*\dx, 0) [below, yshift=-6pt] {\(\dots\)};
    \node at (\xa+5*\dx, 0) [below] {\(t_i\)};
    \node at (\xa+8*\dx, 0) [below] {\(t_h\)};

    }
}

\newcommand{\riemannsuperiore}{% Somma di Riemann superiore.
  \disegno{
    \definizionilinea
    \def \dx{1};
%     \def \ym{2};
%     \def \lm{3};

    \foreach \xa/\y in {0.4/4.9, 1.4/3.67, 2.4/4.17, 3.4/5.15, 4.4/6.2, 
                        5.4/6.9, 6.4/7.15, 7.4/7.06}{
      \fill [top color=green!30!black!20,bottom color=green!50!black!10]
      (\xa, \y) -- (\xa+\dx, \y) -- (\xa+\dx, 0.015) -- (\xa, 0.015) -- cycle;
      \draw [thick, cap=butt] (\xa, \y) -- (\xa+\dx, \y);
      \draw [very thin](\xa,\y) -- (\xa,0.015) (\xa+\dx,\y) -- 
                       (\xa+\dx,0.015);
    }
    \def \y{6.25}
    \fill [top color=green!30!black!20,bottom color=green!50!black!10]
      (8.43, \y) -- (\xb, \y) -- (\xb, 0.015) -- (8.43, 0.015) -- cycle;

    \draw [thick, cap=butt] (8.4, \y) -- (\xb, \y);
    \draw [very thin] (\xb, \y) -- (\xb, 0.015);

    \rcom{0}{+10}{0}{+8}{gray!50, very thin, step=1}

    \begin{scope}
      \clip (\xi, 0) rectangle (\xf, 8);
      \draw [thick] \linea;
    \end{scope}
    
    \node at (\xa, 0) [below, yshift=-1.2em] {\(a\)};
    \node at (\xb, 0) [below, yshift=-1em] {\(b\)};
    \foreach \n in {0,1,2}
      \node at (\xa+\n*\dx, 0) [below] {\(t_\n\)};
    \foreach \n in {3.5,6.5}
      \node at (\xa+\n*\dx, 0) [below, yshift=-6pt] {\(\dots\)};
    \node at (\xa+5*\dx, 0) [below] {\(t_i\)};
    \node at (\xa+8*\dx, 0) [below] {\(t_h\)};

    }
}

\newcommand{\funzint}[5]{% Area sottesa ad una costante.
  \disegno{
    \def \k{#1};
    \def \xa{#2};
    \def \xxi{#3};
    \def \xxf{#4};
    \def \label{#5};
    \def \xi{-1.3};
    \def \xf{7.3};
    \begin{scope}
      \clip (\xa, 0) rectangle (\xf, 5);
      \fill [top color=green!30!black!20, bottom color=green!50!black!10] 
      (\xi, \k) -- (\xf, \k) -- (\xf, 0) -- (\xi, 0) -- cycle;
    \end{scope}

    \rcom{-1}{+7}{0}{+6}{gray!50, very thin, step=1}
    
    \draw [thick] (\xi, \k) -- (\xf, \k);
    \node at (0, \k) [above left] {\(\label\)};

    \draw [-] (\xa, \k) -- (\xa, 0) node [below, yshift=-4pt] {\(a\)};
    \foreach \x in {\xxi,...,\xxf}
      \draw [-] (\x, \k) -- (\x, 0) node [below, yshift=-2pt] {\(\x\)};
    \draw [-] (6.7, \k) -- (6.7, 0) node [below, yshift=-4pt] {\(x\)};
    }
}

\newcommand{\funzintk}[4]{% Area sottesa ad una costante.
  \disegno{
    \def \k{#1};
    \def \xa{#2};
    \def \x{#3};
    \def \label{#4};
    \def \xi{-2.3};
    \def \xf{7.3};
    \begin{scope}
      \clip (\xa, 0) rectangle (\xf, 5);
      \fill [top color=green!30!black!20, bottom color=green!50!black!10] 
      (\xi, \k) -- (\xf, \k) -- (\xf, 0) -- (\xi, 0) -- cycle;
    \end{scope}

    \rcom{-2}{+7}{0}{+5}{gray!50, very thin, step=1}
    
    \draw [thick] (\xi, \k) -- (\xf, \k);
    \node at (0, \k) [above left] {\(\label\)};

    \draw [-] (\xa, \k) -- (\xa, 0) node [below, yshift=-4pt] {\(a\)};
    \draw [-] (\x, \k) -- (\x, 0) node [below, yshift=-2pt] {\(x\)};
    }
}

\newcommand{\funzintx}[5]{% Area sottesa ad una retta.
  \disegno{
    \def \m{#1};
    \def \xa{#2};
    \def \xxi{#3};
    \def \xxf{#4};
    \def \label{#5};
    \def \xi{-0.3};
    \def \xf{8.3};
    \begin{scope}
      \clip (0, 0) rectangle (\xf, 10.3);
      \fill [top color=green!30!black!20, bottom color=green!50!black!10] 
      (\xi, \m*\xi) -- (\xf,  \m*\xf) -- (\xf, 0) -- (\xi, 0) -- cycle;
    \end{scope}

    \rcom{-1}{+8}{0}{+9}{gray!50, very thin, step=1}
    
    \begin{scope}
      \clip (-0.3, -0.3) rectangle (\xf, 10.3);
      \draw [thick] (\xi, \m*\xi) -- (\xf,  \m*\xf);
    \node at (7, 8.5) {\(\label\)};
    \end{scope}

    \draw [-] (\xa, \m*\xa) -- (\xa, 0) node [below, yshift=-4pt] {\(a\)};
    \foreach \x in {\xxi,...,\xxf}
      \draw [-] (\x, \m*\x) -- (\x, 0) node [below, yshift=-2pt] {\(\x\)};
    }
}

\newcommand{\intfinox}{% Area sottesa fino a x.
  \disegno{
    \def \k{2};
    \def \xi{-4.3};
    \def \yi{2};
    \def \xf{7.3};
    \def \yf{4};
    \def \xa{2.6};
    \def \ya{5.10};
    \def \x{5.5};
    \def \y{5.55};
    \def \label{p};
    \def \linea{(\xi, \yi) .. controls (-2, -2) and (3, 10) .. (\xf, \yf)};
    \begin{scope}
      \clip (\xi, 0) rectangle (\xa, 6);
      \fill [left color=red!10!black!5, right color=red!60!black!20] 
        \linea -- (\xf, 0) -- (\xi, 0) -- cycle;
    \end{scope}
    \begin{scope}
      \clip (\xa, 0) rectangle (\xf, 6);
      \fill [top color=green!40!black!30, bottom color=green!40!black!5] 
        \linea -- (\xf, 0) -- (\xi, 0) -- cycle;
    \end{scope}

    \rcom{-4}{+7}{0}{+7}{gray!50, very thin, step=1}
    
    \draw [thick] \linea;
    \node at (6.5, 5.5) {\(f\)};

    \draw [-] (\xa, \ya) -- (\xa, 0) node [below, yshift=-4pt] {\(a\)};
    \draw [-] (\x, \y) -- (\x, 0) node [below, yshift=-2pt] {\(x\)};
    \node at (0.5, 1.5) {\(\mathcal{S}(\dots;~a)\)};
    \node at (4.2, 1.5) {\(\mathcal{S}(a;~x)\)};
    }
}

\newcommand{\teoremafonda}{% Teorema fondamentale incremento non infinitesimo.
  \def \k{2}
  \def \xi{-4.3}
  \def \yi{2}
  \def \xf{7.3}
  \def \yf{4}
  \def \xa{2.6}
  \def \ya{5.10}
  \def \xm{3}
  \def \ym{2.5}
  \def \x{3.4}
  \def \y{5.55}
  \def \label{p}
  \def \linea{(\xi, \yi) .. controls (-2, -2) and (3, 10) .. (\xf, \yf)}
  \disegno{
    \begin{scope}
      \clip (\xi, 0) rectangle (\xf, 6);
      \fill [left color=red!10!black!5, right color=red!60!black!20] 
        \linea -- (\xf, 0) -- (\xi, 0) -- cycle;
    \end{scope}
    \begin{scope}
      \clip (\xa, 0) rectangle (\x, 6);
      \fill [top color=green!40!black!30, bottom color=green!40!black!5] 
        (\xa, 0) -- (\xa, \ya) -- (\x, \ya) -- (\x, 0) -- cycle;
    \end{scope}
    \rcom{-4}{+7}{0}{+7}{gray!50, very thin, step=1}
    \draw [thick] \linea;
    \node at (6.5, 5.5) {\(f\)};
    \draw [-] (\xa, \ya) -- (\xa, 0) node [below, yshift=-8pt] {\(x\)}
              (\xa, \ya) -- (\x, \ya)
              (\x, \ya) -- (\x, 0); 
    \node at (\x, 0) [below, xshift=-5pt] {\footnotesize \(\Delta t\)};
    \draw [thin] (\xa, \ya) -- (0, \ya) node [left] {\(f(x)\)};
    \freccia{(3.9, \ym)}{(\xm-.1, \ym)}{right}{\Delta S}
  }
}

\newcommand{\intdefesempa}{% Area sottesa ad una funzione da a a b.
  \disegno{
    \def \xa{-1};
    \def \ya{4};
    \def \xb{2};
    \def \yb{7};
    \def \linea{(-1.68, 8.3) parabola bend (.25, 0.875) (2.18, 8.3)};

    \begin{scope}
      \clip (\xa, 0) rectangle (\xb, \yb);
      \fill [top color=green!30!black!20,bottom color=green!50!black!10] 
        \linea -- (\xb, 0) -- (\xa, 0) -- cycle;
    \end{scope}
    
    \rcom{-3}{+5}{0}{+8}{gray!50, very thin, step=1}
    
    \node at (-1.9, 7.5) {\(f\)};
    \draw [thick] \linea;
    \draw [-] (\xa, \ya) -- (\xa, 0) node [below] {\(a\)};
    \draw [-] (\xb, \yb) -- (\xb, 0) node [below] {\(b\)};
    }
}

\newcommand{\intdefesempb}{% Area sottesa ad una funzione da a a b.
  \def \fun{x +1/x +1/(x*x)}
  \def \mix{0}
  \def \max{9}
  \def \miy{0}
  \def \may{10}
  \def \Dt{2}
  \def \xa{1}
  \def \ya{3}
  \def \xb{+7}
  \def \yb{+7.1633}
  \disegno{
    \fill [top color=green!30!black!20,bottom color=green!50!black!10]
          (\xa, 0) -- (\xa, \ya) --
          (\xa, \ya) .. controls (+1.7, +1.7) and (+3, +3.5) ..  
          (\xb, \yb) -- (\xb, 0);
    \rcom{\mix}{\max}{\miy}{\may}{gray!50, very thin, step=1}
    \tkzInit[xmin=\mix-0.3,xmax=\max+0.3,ymin=\miy-0.3,ymax=\may+0.3]
    \tkzFct[ultra thick, color=black, domain=\mix+0.3:\max+0.3]{\fun}
    \draw (\xa, \ya) -- (\xa, 0) node [below] {\(a\)} 
          (\xb, \yb) -- (\xb, 0) node [below] {\(b\)}; 
    \node at (7.7, 8.4) {\(f\)};
%     \def \xa{-1};
%     \def \ya{4};
%     \def \xb{2};
%     \def \yb{7};
%     \def \linea{(-1.68, 8.3) parabola bend (.25, 0.875) (2.18, 8.3)};
% 
%     \begin{scope}
%       \clip (\xa, 0) rectangle (\xb, \yb);
%       \fill [top color=green!40!black!30, bottom color=green!60!black!20] 
%         \linea -- (\xb, 0) -- (\xa, 0) -- cycle;
%     \end{scope}
%     
%     \rcom{-3}{+5}{0}{+8}{gray!50, very thin, step=1}
%     
%     \node at (-1.9, 7.5) {\(f\)};
%     \draw [thick] \linea;
%     \draw [-] (\xa, \ya) -- (\xa, 0) node [below] {\(a\)};
%     \draw [-] (\xb, \yb) -- (\xb, 0) node [below] {\(b\)};
    }
}
