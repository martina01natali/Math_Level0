% (c) 2015 Daniele Zambelli daniele.zambelli@gmail.com
%  Bruno Stecca

\input{\folder differenziazione_grafici.tex}

\chapter{Derivate}

% \affiancati{.50}{.50}{
% \disegno{\polregincirc{5}{6}}
% }{
% \disegno{\polregincirc{5}{12}}
% }
% 
% \agnesia

\section{Premessa}
\footnote{Per scrivere questo capitolo mi sono ispirato 
ai lavori di Giorgio Goldoni ``Il calcolo delle differenze e il calcolo 
differenziale''. 
Chi volesse approfondire l'argomento può acquistare il testo 
all'indirizzo: 
\href{https://www.unilibro.it/libri/f/autore/goldoni\_giorgio}
     {www.unilibro.it/libri/f/autore/goldoni\_giorgio}.}
Le nuove tecniche di calcolo che presentiamo si basano sulla conoscenza 
dell'insieme degli Iperreali e sull'applicazione diffusa del \emph{principio di estensione naturale} agli Iperreali delle proprietà utili, che sono vere nell'insieme dei Reali.\\
Per esempio, il concetto di funzione che abbiamo già definito nei Reali, sarà 
applicato silenziosamente anche agli Iperreali e non scriveremo \(f^*\) al 
posto 
di \(f\), per indicare \(f\) estesa, oppure \(f(x^*)\) o simili. 
Se ci sarà bisogno, chiariremo i dubbi che possono insorgere in questa 
estensione o i casi particolari più interessanti.\\
L'argomento del capitolo è centrato sull'uso degli infinitesimi, quindi dei 
numeri iperreali. Come si vedrà, il percorso tipico di soluzione di un problema 
sarà: 
\begin{enumerate}[noitemsep, nosep]
\item definire il problema nell'insieme dei reali;
\item trovare la soluzione mediante un calcolo con numeri iperreali;
\item applicare la funzione \(\pst{}\) alla soluzione, per convertirla in 
numero 
reale.
\end{enumerate}

\section{Introduzione}
\label{sec:differenziazione_introduzione}
Il problema di determinare la velocità istantanea ci ha portati a conoscere 
i numeri infinitesimi e, attraverso questi, l'insieme dei numeri iperreali.
Ora siamo in grado di cercare la risposta alla domanda rimasta in sospeso: 
come si determina la velocità istantanea?\\
La risposta, che conosciamo nelle forme moderne da più di 400 anni,
propone al nostro studio un nuovo potentissimo strumento di calcolo, 
adatto a risolvere problemi in ogni ambito scientifico: la derivata.

\subsection{Velocità di caduta}
\label{subsec:differenziazione_velcaduta}
Nel Settecento fiorirono alcune leggende su Galileo Galilei. Una di queste 
racconta che per dimostrare che i gravi cadono con la stessa velocità, 
gettò dalla Torre di Pisa due sfere di peso diverso, ma di uguali 
dimensioni: i due oggetti raggiunsero il suolo 
contemporaneamente.\\
La Torre di Pisa è alta circa \(56m\) e immaginiamo, per semplificare, che la 
distanza percorsa dai due oggetti sia di \(56m\) (ti lascio calcolare il 
percorso effettivo: tieni presente che al giorno d'oggi l'inclinazione della 
Torre è di \(4,8^\circ\).\\
Oggi sappiamo che un oggetto in caduta libera partendo da fermo ha la seguente
legge del moto:
\(s=\frac{1}{2}gt^2\). Come al solito, \(s\) è lo spazio in metri, \(t\) è il 
tempo in secondi, \(g=9,81 m/s^2\) è l'accelerazione di gravità, costante vicino
alla superficie terrestre.\\
Se cerchiamo la velocità media, basta dividere lo spazio percorso per il 
tempo impiegato:

\begin{align*}
 s_{tot} &= 56m\\
 s=\frac{1}{2}gt^2 \sRarrow t_{tot} &= \sqrt{\frac{2s_{tot}}{g}}=
 \sqrt{\frac{2\times 56}{9,81}}=3,36 s\\
 v_m &= \frac{s_{tot}}{t_{tot}}=\frac{56 m}{3,36 s}=16,67 m/s
\end{align*}
che corrispondono a circa \(60 km/h\).

\subsubsection{La velocità istantanea}

Ma gli oggetti partono fermi e arrivano velocissimi: 
è possibile sapere quale è loro velocità in ogni istante?

Non è facile definire con precisione un istante di tempo: possiamo immaginare 
che sia un tempo assolutamente breve, un tempo di durata infinitesima.\\ 
Considerando valori di tempo istantanei siamo portati naturalmente a passare da 
valori \(t_1 \ t_2, \ \dots\), nell'insieme dei reali, a valori \(t_1, \ 
t_1+dt, \ 
t_1+2dt,\ \dots\ t_2, \ \dots\) nell'insieme degli iperreali. È quindi il 
momento 
di usare le quantità infinitesime.\\ Chiamiamo \(dt\) (in questo caso sarà \(dt 
> 
0\)) un intervallo di tempo infinitesimo, fra due istanti successivi \(t\) e 
\(t+dt\). Lo spazio percorso nella caduta, in quell'intervallo di tempo, 
applicando la legge del moto, sarà: 
\[ 
ds=\frac{1}{2}g\tonda{t+dt}^2-\frac{1}{2}gt^2= \frac{1}{2}g\tonda{t^2+2t\cdot 
dt+(dt)^2}-\frac{1}{2}gt^2= gt\cdot dt+\frac{1}{2}(dt)^2\] 
Dividendo il tutto 
per \(dt\) si ottiene la velocità istantanea, che in questo caso cambia istante 
dopo istante: 
\[v(t)=\frac{ds}{dt}=\frac{gtdt+\frac{1}{2}dt^2}{dt}=gt+\frac{1}{2}dt\] 
L'espressione \(gt+\frac{1}{2}dt=9,81t+\frac{1}{2}dt\) diventa un numero ben 
preciso per ogni valore di \(t\) e per ogni valore \(dt\), un iperreale finito 
che 
è 
la somma di un numero standard e di un numero infinitesimo.\\ Se fissiamo 
l'attenzione su un certo istante, supponiamo \(\overline{t}= 3s\), la velocità 
\(v(3)=9,81\cdot 3+\frac{1}{2}dt\) non ha comunque un valore completamente 
fissato 
perché dipende ancora dalla durata infinitesima \(dt\). I \(dt\) possibili sono 
infiniti, ma questo non cambia granché nel risultato, che si attesta 
sostanzialmente attorno al valore \(9,81\cdot 3=29,43m/s\).\\
Conosciamo già la funzione che ci consente di ottenere la parte sostanziale 
di un risultato iperreale, cioè di trascurare la sua variabilità 
residua dovuta agli infinitesimi: si tratta di applicare la funzione  
la parte standard.
\[ \pst{9,81t+\frac{1}{2}dt}=9,81t.\] 
Questa è la velocità istantanea che cerchiamo: dipende unicamente dal tempo 
\(t\), dai secondi che passano a partire dall'istante del lancio.

 \begin{minipage}{0.3\textwidth}
 \begin{center}
\begin{tabular}{cc}\toprule
\(t\)        & \(v=9,81\times t\)  \\
(in \(s\))   &   (in \(m/s\)) \\\midrule
\(0\)        & \(0\)  \\
\(1\)        & \(9,81\) \\
\(2\)        & \(19,62\) \\
\dots        & \dots \\
\(3,6\)      & \dots  \\\bottomrule
\end{tabular}
\label{tab:diff_velocita}
\end{center}
 \end{minipage}
  \hfill
 \begin{minipage}{.68 \textwidth}
\affiancati{.48}{.48}{
\begin{inaccessibleblock}[Grafico tempo-spazio della caduta libera]
\begin{center} \scalebox{1}{\tempospaziocaduta} \end{center}
\end{inaccessibleblock}
\label{graf:tempospaziocaduta}
}{
\begin{inaccessibleblock}[Grafico tempo-velocità della caduta libera]
\begin{center} \scalebox{1}{\tempovelocitacaduta} \end{center}
\end{inaccessibleblock}
\label{graf:tempovelocitacaduta}
}
 \end{minipage}

Il grafico visualizza la legge oraria del 
moto di caduta libera sotto due aspetti diversi: il ramo di parabola 
(spazio/tempo) mostra che intervalli di tempo uguali a partire da istanti 
diversi corrispondono a spazi progressivamente sempre maggiori. 
Infatti la retta (velocità/tempo) mostra che la progressione della 
velocità è lineare: man mano che il tempo 
scorre la velocità cresce proporzionalmente, come già si vede dalla tabella.
La formula \(v=9,81\times t\) ci permette il calcolo della velocità per 
ogni valore reale di \(t\) e questa è la risposta alla domanda iniziale. 

\subsection{La pendenza}
\label{subsec:differenziazione_pendenza}
Per studiare i moti in laboratorio, Galileo usava piani inclinati. Una sferetta 
rotola lungo un piano inclinato con accelerazione diversa a seconda 
dell'inclinazione del piano. Minore è la pendenza del piano, minore è 
l'accelerazione costante, minore è la velocità che si sviluppa. \\
La pendenza del 
piano e la velocità che si sviluppa, da zero al massimo, sono intimamente 
collegate. Se il piano è orizzontale non c'è movimento e la velocità è 
\(v=0\), come la pendenza. Se il piano è verticale, la caduta è libera e  
la velocità progredisce come in tabella, teoricamente fino all'infinito, dato 
che la pendenza è infinita.

Useremo il termine \emph{pendenza} nel suo significato matematico.
\begin{definizione}
La \emph{pendenza} è la rapidità con cui varia una certa variabile
numerica, in un particolare punto della sua variazione. 
\end{definizione}

Per variazione intendiamo in cambiamento di valore, da un valore iniziale a un 
valore finale. Molto spesso invece di variazione si usa il termine 
\emph{incremento}, o differenza.

\subsubsection{Il Rapporto Incrementale e la pendenza}
\label{subsubsec:RI}
\begin{esempio}
Misurando le temperature in una recente notte invernale, abbiamo registrato:

 \begin{minipage}{0.48\textwidth}
 \begin{center}
\begin{tabular}{cccc}\toprule
h (ore) & \(T (^\circ C)\) & \(\Delta T\) 
             & \(frac{\Delta T}{\Delta h}\)\\\midrule
\(21\) & \(3\) & \(-3\) &\(-3\) \\
\(22\) & \(0\) & 0 & 0 \\
\(23\) & \(0\) & \(-1\) & \(-1\) \\
\(24\) & \(-1\) &\dots & \dots\\
\(1\)  & \(-3\) & \(-1\) &\(-0,5\)\\
\(3\)  & \(-4\) &\dots & \dots\\
\(4\)  & \(-5\) & \(-5\)  &\(0\)\\
\(5\)  & \(-5\) & \(0\) & \(2,5\)\\
\(7\)  & \(0\)  &
\\ \bottomrule
\end{tabular}
\label{tab:temperaturea}
\end{center}
\end{minipage}
 \hfill
\begin{minipage}{.48 \textwidth}
\begin{inaccessibleblock}[Grafico delle temperature per punti distanziati]
\begin{center} \scalebox{1}{\temperaturea} \end{center}
\end{inaccessibleblock}
\label{graf:temperaturea}
\end{minipage}

È facile calcolare le variazioni \(\Delta T\), ora per ora. 
Fra le 22 e le 23 non c'è variazione: l'incremento è zero. \\
Alle 23 la temperatura inizia a calare di \(1^\circ C\) 
rispetto alle 24, quindi l'incremento \( \Delta T=-1\). \\
Fra le \(5\) e le \(7\) l' incremento \(\Delta T= 5 ^\circ C\). 
Ma stavolta la differenza è su un intervallo di \(2\) ore, 
quindi una variazione media di \(2,5 ^\circ C\) all'ora.

Mettendo in grafico i dati di questa funzione empirica, ne risulta una 
spezzata. L'inclinazione dei vari segmenti si può calcolare direttamente sul 
grafico con la formula del coefficiente angolare vista in geometria analitica 
e corrisponde ai risultati dell'ultima colonna, che in effetti forniscono le 
pendenze dei vari segmenti.  Vi sono pendenze positive, negative e nulle, in 
corrispondenza di segmenti crescenti, decrescenti o orizzontali. 
\end{esempio}
 
\(\dfrac{\Delta T}{\Delta h}\), il rapporto che calcola le pendenze, è il 
rapporto fra gli incrementi della temperatura e gli incrementi orari. Un 
rapporto di questo tipo si chiama Rapporto Incrementale.
\begin{definizione}
Il Rapporto Incrementale (RI) di una funzione \(f(x)\) è il rapporto che 
si calcola fra l'incremento di \(f\) e l'incremento di \(x\), quando \(x\) 
cambia valore da \(x_0\) a \(x_0+\Delta x\).
\[
 RI= \frac{f(x_0+\Delta x)-f(x_0)}{(x_0+\Delta x)- x_0}=
 \frac{\Delta f(x_0)}{\Delta x}.
\]
\end{definizione}
L'incremento \(\Delta\) di una funzione non è sempre lo stesso, perché 
cambia sia per la funzione, sia per il punto in cui lo si vuole calcolare, 
sia rispetto a \(\Delta x\). 
È quindi a sua volta una funzione, una funzione di tre 
variabili: \(\Delta=\Delta(f, x, \Delta x)\). \\
Di conseguenza avviene lo stesso per il Rapporto Incrementale: \(RI=RI(f, x, 
\Delta x)\).

\subsubsection{Il Rapporto Differenziale e la pendenza}
\label{subsubsec:RD}

\begin{minipage}{0.48\textwidth}
% \begin{center}
Immaginiamo ora di avere esigenze scientifiche molto raffinate e di 
predisporre un apparato che misura le temperature ogni secondo. 
Allora il grafico appena visto avrebbe spigoli meno visibili e i vari 
incrementi \(\Delta T\) sarebbero numeri vicini allo zero. 
Potremmo conoscere il fenomeno con grande precisione e individuare molti 
più dettagli, se potessimo visualizzare le diverse pendenze.
%\end{center}
\end{minipage}
 \hfill
\begin{minipage}{.48 \textwidth}
 \begin{center}
 \temperatureb
\label{graf:temperatureb}
 \end{center}
\label{graf:temperature}
\end{minipage}


\affiancati{.50}{.50}{
%grafico temperature smussato
\begin{inaccessibleblock}[Grafico delle temperature con punti molto vicini]
\begin{center} \scalebox{1}{\temperatureb} \end{center}
\end{inaccessibleblock}
\label{graf:temperatureb}
}{
%grafico temperature smussato
\begin{inaccessibleblock}[Grafico continuo delle temperature]
\begin{center} \scalebox{1}{\temperaturec} \end{center}
\end{inaccessibleblock}
\label{graf:temperaturec}
}

In una situazione ideale, non legata a strumenti di misura necessariamente 
imperfetti, consideriamo una funzione matematica e il suo grafico e andiamo 
alla ricerca della pendenza della funzione in un determinato punto. Questa 
volta, operando nell'insieme \(\IR\), potremo considerare incrementi 
infinitamente piccoli. Per ragioni storiche questi incrementi infinitesimi si
chiamano \emph{differenziali}.
\begin{definizione}
 Il differenziale \(df\) di una funzione \(f\) è l'incremento infinitesimo, se 
c'è, che la funzione subisce a causa di una variazione infinitesima della sua 
variabile indipendente \(x\), a partire da un valore fissato \(x_0\):
\[df(x_0) = f \tonda{x_0 + \epsilon} - f(x_0).\]
\end{definizione}

Come già visto per l'incremento finito \(\Delta f\), anche il differenziale 
\(df\) 
dipende dall'espressione della funzione, dal valore fissato \(x_0\) dove inizia 
la variazione della \(x\) e dal suo incremento infinitesimo~\(\epsilon\).\\
Quindi il differenziale di una funzione \(df\) è esso stesso  una funzione:
\(df=df(f, x_0, \epsilon)\). 

\begin{osservazione}
Per segnalare che il simbolo \(df\) (oppure \(dx\), ecc.) non è il prodotto fra 
due variabili (cioè \(df\ne d \cdot f,\ dx\ne d\cdot x,\ \dots)\) ma indica 
una differenza infinitesima, nella lettura si pronuncia:
\textit{de effe (de ics)}.
\end{osservazione}

\begin{minipage}{.48 \textwidth}
\begin{esempio}
Vogliamo calcolare l'incremento della 
funzione:~\(f(x) = \dfrac{1}{4} x^2 -x -3\)
quando \(x\) parte da~\(7\) e ha un incremento di~\(\epsilon\).
\begin{align*}
  df(7) &= f(7+\epsilon) - f(7) = \\
        &= \dfrac{\tonda{7 +\epsilon}^2}{4}  -\tonda{7 +\epsilon} -3 - 
           \dfrac{7^2}{4}  +7 +3 =\\
        &= \dfrac{49 +14 \epsilon +\epsilon^2}{4} -10 -\epsilon - 
           \dfrac{49}{4} +10 =\\
        &= \dfrac{14 \epsilon +\epsilon^2}{4} -\epsilon 
        = \dfrac{10 \epsilon +\epsilon^2}{4} =\\
        &= 2,5 \epsilon + \dfrac{\epsilon^2}{4}, \quad \forall \epsilon. 
%\sim 2,5 \epsilon
\end{align*}
\end{esempio}
\end{minipage}
 \hfill
\begin{minipage}{.48 \textwidth}
 \begin{center}
\differenziale
 \end{center}
\end{minipage}
L'incremento della funzione è un differenziale: un infinitesimo che è la somma 
di due infinitesimi di ordine diverso. Il risultato non cambia 
sostanzialmente, se al posto di \(\epsilon\) svolgiamo lo stesso calcolo 
utilizzando un diverso infinitesimo \(\delta\). Per significare che il 
risultato non dipende dalla scelta dell'infinitesimo, si indica: \(\forall 
\epsilon\).

Nell'espressione della funzione riconosciamo che si tratta di una parabola, 
una funzione il cui grafico ha pendenze sempre diverse, che aumentano (o 
diminuiscono) gradualmente, allontanandosi dal vertice. Seguendo lo schema 
usato per il grafico delle temperature, cerchiamo la pendenza per \(x=7\) 
calcolando il rapporto fra gli incrementi, cioè in questo caso, fra i 
differenziali.
\begin{definizione}
Il Rapporto Differenziale (RD) di una funzione \(f(x)\) è il 
rapporto che si calcola fra l'incremento infinitesimo \(df\) della funzione 
e l'incremento infinitesimo di \(x\), quando \(x\) cambia valore da \(x_0\) a 
\(x_0+\epsilon\).
\[
 RI= \frac{f(x_0+\epsilon)-f(x_0)}{(x_0+\epsilon)- 
x_0}=\frac{df(x_0)}{\epsilon}.
\]
\end{definizione}
Ricordando che in ogni rapporto il denominatore deve essere diverso 
da zero, vediamo i calcoli.

\[
 RD=\frac{df(7)}{\epsilon}=\frac{f(7+\epsilon) - f(7)}{\epsilon}=\dots=
 \frac{2,5 \epsilon + \dfrac{\epsilon^2}{4}}{\epsilon}=
 \frac{\cancel{\epsilon}\tonda{2,5 + \dfrac{\epsilon}{4}}}{\cancel{\epsilon}}=
 2,5 + \frac{\epsilon}{4}, \quad \forall \epsilon\ne 0. 
\]

Le pendenze del grafico di temperature sono numeri reali e  con questi è facile 
stabilire se una temperatura è in crescita di più o di meno rispetto ad 
un'altra. Ora invece abbiamo un risultato parzialmente incerto, perché la 
pendenza risulta un numero finito iperreale, la somma di uno standard e di 
un infinitesimo. Si tratta di un numero non completamente esprimibile in 
cifre, perché la sua parte infinitesima ha un valore sconosciuto.
Per fortuna, però, questa parte infinitesima è trascurabile e si può 
ragionevolmente concludere che, anche considerando gli infiniti possibili 
valori dell'infinitesimo \(\dfrac{\epsilon}{4}\), tuttavia il risultato è 
sostanzialmente uguale a \(2,5\).

Tradotto in termini matematici, questo ragionamento corrisponde al calcolo 
della parte standard.
\[
 \text{pendenza}=\pst{RD}=\pst{2,5 + \frac{\epsilon}{4}}=2,5.
\]
\begin{definizione}
 La pendenza di una funzione \(f\) in un punto \(x_0\) del suo dominio, è la 
parte standard del Rapporto Differenziale di \(f\) calcolato in \(x_0\), se 
questa esiste ed è sempre uguale, qualsiasi sia l'incremento infinitesimo 
diverso da zero che si sta usando.
\end{definizione}
\begin{esempio}
 Calcoliamo per la stessa funzione la pendenza nel punto del vertice. 
Trattandosi dell'unico punto a tangente orizzontale, non potrà che risultare 
una pendenza uguale a zero.\\
\begin{align*}
 &f(x) = \dfrac{1}{4} x^2 -x -3 \quad x_0=2;\\
 &RD=\frac{df(2)}{\epsilon}=\frac{f(2+\epsilon) - f(2)}{\epsilon}=
  \frac{\tonda{2 +\epsilon}^2}{4}  -\tonda{2 +\epsilon} -3 - 
           \frac{2^2}{4}  +2 +3 =\\
  &= \dfrac{4 +4 \epsilon +\epsilon^2}{4} -5 -\epsilon - 1 +5 =
 +1+ \epsilon +\frac{\epsilon^2}{4} -5 -\epsilon - 1 +5 =\frac{\epsilon^2}{4}.
  \end{align*}
Considerando che il rapporto differenziale esiste per qualsiasi infinitesimo 
non nullo \(\epsilon\), calcoliamo la sua parte standard.
\[
 \pst{RD}=\pst{\frac{\epsilon^2}{4}}=0.
\]
Il risultato è sempre lo stesso \(\forall \epsilon \ne 0\), abbiamo perciò la 
conferma: la pendenza è uguale a zero per \(x=2\).
\end{esempio}

\subsubsection{Controesempi}
\label{subsubsec:controesempi}
Cerchiamo la pendenza \(f'(0)\) per la funzione \(f(x)=\sqrt[3]{x^2}\). Da 
notare che \(f(0)\) c’è e vale \(0\).\\
\begin{align*}
&f(x)=\sqrt[3]{x^2}\ \quad x_0=0.
&RD=\frac{\sqrt[3]{(0+\epsilon)^2}-\sqrt[3]{0^2}}{\epsilon}=
 \frac{\sqrt[3]{(\epsilon)^2}}{\epsilon}=\frac{1}{\sqrt[3]{\epsilon}}.
\end{align*}

\(f'(0)\) non c'è. Infatti qualsiasi sia l'infinitesimo non nullo \(\epsilon\), 
\(frac{1}{\sqrt[3]{\epsilon}}\) è un iperreale infinito e non esiste la parte
standard  di un numero infinito.\\
\vspace{3mm}
Ciò è sufficiente per concludere che \(f\) non ha pendenza in \(0\).

Anche nel prossimo esempio la pendenza non esiste, anche se la ragione è 
diversa.

\begin{esempio}
 %valore assoluto
\label{esempio:diff01_derimodulo}
 Calcola la pendenza di \(f(x)=|x|\) per \(x=0\).\\
 La funzione contiene un valore assoluto: riscriviamola per casi.\\
 
% \begin{figure}[h!]
 \begin{minipage}[]{.39 \textwidth}
 \begin{center}
\begin{inaccessibleblock}
  [Derivata |x|]
  \derivamodulouno
\end{inaccessibleblock}
\end{center}
 \end{minipage} 
 \hfill
 \begin{minipage}[]{.59 \textwidth}
  Funzione: \(f(x)=|x|=\begin{cases}
   x\quad &\mbox{per }x\ge 0\\
  -x\quad &\mbox{ per }x<0
  \end{cases}\)\\
  Differenziale per \(x=0\):\\ 
\(df(0)=f(0+dx)-f(0)=0+dx-0=dx\);\\
  Rapporto Differenziale:\\ 
\(\dfrac{df(0)}{\epsilon}=\dfrac{dx}{\epsilon}=
  \begin{cases}
  1\quad &\mbox{per }x\ge 0\\
 -1\quad &\mbox{ per }x<0
 \end{cases}\)\\
\end{minipage}
\label{}
% \end{figure} 

Si tratta di due semirette che si uniscono nell'origine. Hanno pendenza \(m=1\) 
per \(x\ge 0\) e \(m=-1\) per \(x<0\). Quale è la pendenza giusta per
\(x=0\), nel punto cioè dove il grafico cambia pendenza all'improvviso?\\
Nella monade di \(x=0\) gli infinitesimi \(\epsilon\) positivi causano una 
variazione infinitesima positiva sull'asse \(y\) e lo stesso avviene per 
gli \(\epsilon\) negativi. 
Perciò per \(x=0\) la parte standard del Rapporto Differenziale non è la stessa 
per qualsiasi \(\epsilon\ne 0\) e la pendenza non si può calcolare.
\end{esempio}

I due esempi chiariscono perché nella definizione di pendenza si debba 
specificare: 
``...\textit{purché essa esista e sia la stessa} \(\forall \epsilon\ne 0\)''.


\subsubsection{In conclusione:} 
\label{subsubsec:ricetta_pendenza}
\begin{itemize}
 \item La pendenza è un numero reale che consente di capire con quale velocità 
variano i dati di un fenomeno in un certo momento oppure quanto sia inclinato 
il grafico di una funzione in un certo punto.

\begin{enumerate}
\item La pendenza si calcola nell'insieme \(\IR\);
 \item si considerano infinitesimi diversi da zero;
 \item si scrive e si calcola il Rapporto Differenziale;
 \item se ne calcola la parte standard, se esiste;
 \item si controlla se la parte standard ha sempre lo stesso risultato per 
qualsiasi variazione infinitesima, ma diversa da zero, della variabile \(x\);
 \item se si supera il punto precedente, si ottiene il risultato che è un 
numero reale.
\end{enumerate}
\end{itemize}
\begin{osservazione}
 Ai punti 4 e 5 dell'elenco si indicano alcuni controlli importanti. Vediamone 
i motivi.
 \begin{itemize}
  \item [4:] Può non esistere la parte standard del Rapporto 
Differenziale? Sì, perché si tratta di un rapporto fra due infinitesimi e non 
è detto che risulti un numero finito. La parte standard si può calcolare solo 
se l'argomento è un numero finito.
\item [5:] La parte standard può dare risultati diversi a seconda 
dell'infinitesimo che si usa? Sì, per esempio può cambiare di segno, come 
accade quando si cerca la pendenza di \(f(x)=|x|\) in \(x_0=0\). L'esmpio sarà 
discusso più avanti.
 \end{itemize}

\end{osservazione}

\subsubsection{Brevi note di storia}
Il calcolo appena descritto fu inventato nel 1600 (Calcolo infinitesimale, o 
differenziale, o semplicemente Calcolo) con contenuti più ingenui ma 
sostanzialmente uguali ai nostri. 

Il Calcolo fiorì con grande successo per 150 anni a partire 
dall'epoca di Newton e Leibniz. 
Ma suscitava vivaci polemiche fra gli specialisti, perché non si era 
in grado di spiegare la regola che permetteva di far sparire gli 
infinitesimi al termine dei calcoli, che per noi è il punto 4 dell'elenco.

A quel tempo non si conosceva la teoria degli insiemi numerici, gli iperreali e 
la funzione parte standard.
Oggi i matematici conoscono meglio la materia e quelle critiche  
sono superate. 
Siamo quindi in grado di procedere nello studio di questa nuova branca 
della matematica, che si chiama \emph{Analisi infinitesimale}.


\subsection{Pendenze in grafico}
\label{subsec:pendenze_grafico}
Nel grafico di \(f\), le diverse pendenze della curva di solito si colgono a 
vista. Ma si possono anche disegnare? Se fosse possibile esprimerle con un 
disegno sarebbe molto più facile confrontarle e interpretarle.\\
Per provare a rispondere, ricordiamo il grafico \ref{graf:temperature}: lì gli 
incrementi delle temperature, ora per ora, si riflettono nella diversa 
inclinazione dei segmenti, che siamo abituati a calcolare con la nota formula 
del coefficiente angolare.  
È facile vedere che le stesse pendenze si calcolano anche con le 
regole dell'elenco in \ref{subsubsec:ricetta_pendenza}. 
Dimostriamolo. 

Consideriamo una retta generica, non verticale: \(y=mx+q\) e un suo generico 
punto di ascissa \(x_0\).
\begin{teorema}
\label{teo:pendenza_retta}
  Per le funzioni lineari la pendenza è il coefficiente angolare.
\end{teorema}
\noindent Ipotesi: \(f(x)=mx+q\) \tab Tesi: \(\text{pendenza }=m\)
\begin{proof}
Per qualsiasi \(\epsilon\ne 0\):
\begin{align*}
RD &=\dfrac{df(x_0)}{\epsilon} =\dfrac{f(x_0+\epsilon)-f(x_0)}{\epsilon}=\\
               &=\dfrac{m(x_0+\epsilon)+q-\tonda{mx_0+q}}{\epsilon}=
                 \dfrac{mx_0-m\epsilon+q-mx_0-q}{\epsilon}=
                 \dfrac{m\epsilon}{\epsilon}=m.\\
\pst{RD}&= \pst{m}=m.
\end{align*}
\end{proof}
L'ultimo passaggio è corretto perché il Rapporto Differenziale è un numero 
finito e perché la parte standard non cambia al variare di \(\epsilon\), 
infinitesimo non nullo.

Poiché nel risultato non compare \(x_0\), anche in questo caso \(df(x)\) non 
dipende dal punto \(x_0\), cioè vale \(\forall x\): la pendenza di una retta è 
un valore costante.

Si tratta di un risultato molto utile per individuare le pendenze nei 
nostri grafici. 

Per visualizzare la pendenza di una curva in un suo punto 
possiamo disegnare la retta che passa per quel punto e che ha per coefficiente 
angolare proprio quella pendenza, come nell'esempio che segue.

\begin{esempio}
Calcola la pendenza della parabola:
\(f(x) = -\dfrac{1}{4}x^2+2x +1\) \quad nel suo punto di ascissa:~\(x_P=8\).
Prendiamo sulla curva un punto \(P'\), diverso da \(P\) a 
distanza infinitesima da \(P\) diversa da zero.

L'ascissa di \(P'\) sarà \(x_{P'}=8+\delta\). 
Con \(\delta\) infinitesimo e diverso da zero.

I nostri due punti sono:
\(P\punto{8}{f(8)}\) e \(P'\punto{8+\delta}{f(8+\delta})\).

Per poter distinguere i due punti dovrò usare un microscopio con un 
opportuno ingrandimento infinito.

Calcoliamo i corrispondenti valori della funzione:

\affiancati{.40}{.58}{
\parabolaetangentea
}{
\begin{align*}
f(8) &= -\frac{1}{4}8^2 + 2 \cdot 8 + 1 = -16+16+1 = +1\\
f(8+\delta) &= -\frac{1}{4}\tonda{8+\delta}^2 + 2 \tonda{8+\delta} + 1 = \\
&= -\frac{1}{4}\tonda{64+16\delta+\delta^2} + 16 + 2\delta + 1 =\\
&= -16-4\delta-\frac{\delta^2}{4} + 16 + 2\delta + 1 =
+1 - 2\delta -\frac{\delta^2}{4}.
\end{align*}
}

Quindi i due punti sono:
\(P\punto{8}{1}\) e 
\(P'\punto{8+\delta}{+1 - 2\delta -\dfrac{\delta^2}{4}}\).

\affiancati{.58}{.40}{
Ora possiamo calcolare la pendenza per questi due punti:
\begin{align*}
% m &= \frac{f(x + \delta) - f(x)}{x + \delta - x}=
RD &= \frac{y_{P'} - y_{P}}{x_{P'} - x_{P}}=
\frac{\cancel{+1} - 2\delta -\dfrac{\delta^2}{4}~ \cancel{-1}}
     {8 + \delta - 8}= \\
&=\frac{- 2\delta -\dfrac{\delta^2}{4}}{\delta}= 
\frac{\delta \tonda{- 2-\dfrac{\delta}{4}}}{\delta}= 
- 2-\dfrac{\delta}{4}.
\end{align*}
}{
\parabolaetangenteb
}
\(-2-\dfrac{\delta}{4}\) è il Rapporto Differenziale della curva, calcolato nel 
tratto infinitesimo fra i punti \(P\) e \(P'\). Trattandosi di due punti 
sostanzialmente coincidenti, il tratto di curva \(PP'\) è sostanzialmente un 
segmento, che ha il coefficiente angolare espresso dal Rapporto Differenziale. 
Un metodo efficace per visualizzare la pendenza è proprio quello di disegnare 
la retta per \(PP'\). 

Quante sono le rette che passano per \(PP'\)?\\
Sono infinite, perché il loro coefficiente angolare è \(-2-\dfrac{\delta}{4}\), 
che rappresenta infiniti numeri indistinguibili, sostanzialmente 
uguali a \(-2\). 

Isoliamo una di queste rette, in modo che il suo disegno le rappresenti tutte. 
Per fare ciò, calcoliamo \(\pst{-2-\dfrac{\delta}{4}}\).
Possiamo farlo, dato che \(-2-\dfrac{\delta}{4}\) è un iperreale finito:
\[
\pst{RD}=\pst{-2-\dfrac{\delta}{4}} = \pst{-2}-\pst{\dfrac{\delta}{4}} =
  -2 - 0 = -2.
\]
Il risultato non dipende dal valore dell'infinitesimo \(\delta\), così 
possiamo concludere che la pendenza che cerchiamo è sostanzialmente \(-2\), sia 
per il tratto di curva, sia per la retta che passa per gli stessi 
punti.

Nel piano cartesiano reale, cioè con numeri reali, \(P\) e \(P'\) coincidono, 
perciò \(-2\) è la pendenza della retta tangente.
\end{esempio}

In conclusione, ora abbiamo un metodo per calcolare la pendenza di una curva in 
un suo punto ed anche un modo di mostrarla nel piano cartesiano: con il 
disegno della tangente alla curva in quel punto. 



\subsection{Derivata di una funzione}
\label{subsec:differenziazione_derivatafunzione}
Abbiamo visto che \(RD=RD(f, x_0, \epsilon)\). 

Negli esempi precedenti la funzione \(f\) era stata scelta. Se \(f\) è fissata, 
allora \(RD=RD(x_0,\epsilon)\). Se poi se la pendenza esiste, quindi dopo avere 
applicato la parte standard, gli infinitesimi spariscono e quindi la 
pendenza dipende solo da \(x_0\).

Quindi, per ogni funzione \(f\), si può considerare una ulteriore funzione:  
``pendenza di \(f\)''. \\
``pendenza di \(f\)'' ha come argomento un numero reale, \(x\), 
e ha come risultato un altro numero reale, cioè la pendenza della curva che 
rappresenta \(f\) nel piano cartesiano.

Questa funzione viene chiamata derivata della funzione \(f\) 
e viene indicata con \(f'\).

\begin{definizione}
Chiamiamo \emph{derivata} della funzione \(f\) la funzione \(f'\) che
ha come argomento \(x\) e calcola la pendenza della funzione \(f(x)\) per ogni 
valore di \(x\).
\end{definizione}

\(f'\) quindi ci dà la formula per calcolare le pendenze 
di \(f\). Questa formula, ovviamente diversa da funzione a funzione, si applica
all'espressione di \(f\) secondo il procedimento in 
\ref{subsubsec:ricetta_pendenza}, come negli esempi.

\begin{esempio}
\label{esempio:esempio_5}
Se \(f(x)=\dfrac{1}{3}x^2-2x+5\), quale è l'espressione di \(f'(x)\)?
 
Non essendo indicato un particolare valore di \(x_0\), proviamo a inventarne 
qualcuno. Per esempio, per \(x_0=4\) scriviamo:
\begin{align*}
 &\forall \delta\ne 0:\quad x_0= 4\quad\quad 
f(x_0)=f(4)=\frac{1}{3}(4)^2-2\cdot 4+5 = \dfrac{7}{3}\\
&\pst{\dfrac{f(x_0+\delta)+f(x_0)}{\delta}}=\dfrac{\dfrac{1}{3}(4+\delta)^2-2(4+
\delta) + 5 - (\dfrac{1}{3}4^2-2\cdot 4+5)}{\delta}=\dots
\end{align*}
Al termine di alcuni tentativi, il quadro della situazione è:

 \begin{minipage}{0.48\textwidth}
 \begin{center}
\begin{tabular}{ccc}\toprule
\(x_0\) & \(f(x_0)=\dfrac{1}{3}x_0^2-2x_0+5\) & \(f'(x_0)=?\) \\\midrule
\(-2\) & \(31/3\) &  \(-10/3\)\\ 
\(-1\) & \(22/3\) &  \(-8/3\) \\
\(0\)  & \(5\)    & \(-2\)  \\
\(1\)  & \(10/3\) & \(-4/3\) \\
\(2\)  & \(7/3\)  & \(-2/3\)\\
\(3\)  & \(2\)    & \(0\)\\
\(4\)  & \(7/3\)  &\dots \\
\(5\)  & \(10/3\) \\
\(6\)  & \dots   \\
\(\dots\)  \\
 \\
\\ \bottomrule
\end{tabular}
\label{tab:temperaturea}
\end{center}
\end{minipage}
 \hfill
\begin{minipage}{.48 \textwidth}
\begin{inaccessibleblock}[Grafico di una funzione e della sua derivata.]
\begin{center} \scalebox{1}{\funzioneederivata} \end{center}
\end{inaccessibleblock}
\label{graf:funzioneederivata}
\end{minipage}
\end{esempio}
Per ogni particolare \(x_0\) risulta uno e un solo valore di pendenza, come 
deve succedere quando due insiemi (quello degli \(x_0\) e quello di 
\(f'(x_0)\)) si corrispondono secondo una funzione. In 
questo caso i valori della pendenza si allineano lungo la retta 
\(y=\dfrac{2}{3}x-2\).


\subsubsection{Nomi per la derivata}
Il nome \emph{derivata} per indicare il calcolo che abbiamo descritto ha 
origini storiche. Si è diffuso  ovunque (derivative, derivada, dérivée, 
...) anche se non rende pienamente il significato di ciò che rappresenta. 
Se ne potrà intuire la ragione in un capitolo successivo, quando parleremo 
anche di funzioni primitive.\\
Sempre per ragioni storiche, si sono diffusi vari simboli che rappresentano 
l'operazione di derivazione:
\begin{enumerate}[noitemsep]
 \item \(f'(x_0)\) è il simbolo per il risultato della derivazione di \(f\) 
 per \(x= x_0\): è la pendenza della curva in \(x_0\);
 \item \(\mathit{D}\quadra{f(x)}\) indica la formula della derivazione di 
\(f\),  per es. \(\mathit{D}\quadra{5x\sqrt[3]{x^2}}=5\sqrt[3]{x^2}+\dfrac{10x}
 {3\sqrt[3]{x}}\);
 \item \(\dot{f}\) equivale a \(f'\) e si usa in alcuni corsi universitari;
 \item \(\dfrac{d}{dx}f(x)\) è come \(f'(x)\), cioè corrisponde alla parte 
standard del Rapporto Differenziale;
 \item \(\dfrac{df(x)}{dx}\) si trova spesso nei libri come se il Rapporto
 Differenziale e la derivata fossero la stessa cosa. Se la derivata 
esiste, quest'uguaglianza si può accettare, trattandosi di quantità 
indistinguibili. 
\end{enumerate}


\section{Derivata di alcune funzioni semplici}
\label{sec:differenziazione_derivatafsemplici}
A questo punto andiamo alla ricerca di regole semplici che ci permettano di 
calcolare \(f'\) senza ripetere procedimenti laboriosi, come 
nell'esempio precedente che in realtà è un caso fra i meno complicati.

Deriveremo alcune funzioni molto comuni, cioè non cercheremo la pendenza di 
queste funzioni per ogni particolare \(x_0\), ma troveremo una formula, 
unica per ogni \(f\), che si applica per qualsiasi\(x\) del suo dominio 
reale.

\subsection{Derivata della funzione identica}
\label{subsec:differenziazione_derivatafidentica}
\begin{esempio}
 Deriviamo la funzione più semplice: \(f(x)=x\).\\
 Innanzitutto dovremmo scrivere il differenziale della funzione \(df\), che 
essendo \(f(x)=x\), risulta: 
\(df(x)=f(x+\epsilon)-f(x)=x+\epsilon-x=\epsilon\). 
\\
In pratica, nella funzione identica il differenziale \(dx=\epsilon\).\\
Perciò: RD \(=\dfrac{df(x)}{dx}=\dfrac{dx}{dx}=1, \quad  \forall dx \ne 0\);\\
Poiché il risultato è un numero finito, la parte standard esiste. Inoltre il 
suo valore non dipende dall'infinitesimo considerato:\\
\(f'(x)=\pst{\dfrac{df(x)}{dx}}=\pst{1}=1\).\\
\end{esempio}
Abbiamo così dimostrato un semplice teorema:
\begin{teorema}
 La derivata della funzione identica \(f(x)=x\) è 
\(\mathit{D}\quadra{x}=1\).
\end{teorema}
\begin{osservazione}
 Non avendo specificato un particolare \(x_0\), il risultato vale per qualsiasi 
\(x\).
\end{osservazione}

Dall'esempio precedente impariamo anche un dettaglio: al 
posto di \(\epsilon\) si può scrivere il differenziale \(dx\). Questo ci dà 
modo di ridefinire la derivata.
\begin{definizione}
 La derivata di una funzione \(f\) è la parte standard del Rapporto 
Differenziale, se questa esiste e se non cambia valore al variare 
dell'infinitesimo \(dx\), preso diverso da zero.
\[
 f'(x)=\pst{\frac{df(x)}{dx}}.
\]
\end{definizione}

\subsection{Derivata della funzione costante}
\label{subsec:differenziazione_derivatafcostante}
Una funzione \(f(x)\) è costante se qualunque sia il valore di \(x\) il 
risultato è sempre lo stesso. Possiamo indicare questa funzione in diversi modi:
\[f: x \mapsto k \quad \text{o} \quad f(x)=k \quad \text{o} \quad y = k.\]

Il suo differenziale sarà:
\[df(x_0)=f(x_0+\epsilon)-f(x_0)=k-k=0,\quad \forall \epsilon.\]
Quindi, se la funzione è costante, il suo differenziale è nullo.
Infatti, avendo sempre lo stesso valore per qualsiasi \(x\), la differenza 
tra due suoi valori è zero. 
\begin{teorema}
  La derivata di una funzione costante è \(\mathit{D}\quadra{k}=0\):\quad 
\(\mathit{D}\quadra{k}=0\).
\end{teorema}
\noindent Ipotesi: \(f(x)=k\).\tab Tesi: \(f'(x)=0\).
\begin{proof}
 Infatti \(df(x)=0\).
\end{proof}


\subsection{Derivata della funzione lineare}
\label{subsec:differenziazione_derivataflineare}
Una funzione lineare è una funzione espressa da un polinomio di primo grado:
\[f: x \mapsto mx +q \quad \text{o} \quad   f(x)=mx +q \quad \text{o} \quad 
  y = mx +q.\]
\begin{teorema}
  La derivata di una funzione lineare è il coefficiente angolare \(m\) della 
retta che disegna il suo grafico.
  %: \(\mathit{D}\quadra{k}=0\).
\end{teorema}
\noindent Ipotesi: \(f(x)=mx+q\).\tab Tesi: \(f'(x)=m\).
\begin{proof}
 Per la dimostrazione, basta adattare il Teorema \ref{teo:pendenza_retta}, che 
è pensato per un particolare \(x_0\), mentre qui consideriamo che \(x\) sia 
qualsiasi.
\end{proof}


\subsection{Derivata della funzione quadratica}
\label{subsec:differenziazione_derivatafquadratica}
In alcuni esempi abbiamo già calcolato la pendenza di una funzione espressa da 
un polinomio di secondo grado. Concentriamoci sul caso generale e semplice: 
\(f(x)=x^2\).
\begin{teorema}
  La derivata della funzione quadratica è: \(\mathit{D}\quadra{x^2}=2x\).
\end{teorema}
\noindent Ipotesi: \(f(x)=x^2\) .\tab Tesi: \(f'(x)=2x\).
\begin{proof}
\begin{align*}
 \forall dx&\ne 0:\\
 RD&=\dfrac{df(x)}{dx}=\dfrac{(x+dx)^2-x^2}{dx}=\dfrac{x^2+2xdx+(dx)^2-x^2}{dx}=
  \dfrac{2xdx+(dx)^2}{dx}=\dfrac{\cancel{dx}\tonda{2x+dx}}{\cancel{dx}}=\\
  &=2x+dx.
\end{align*}
Il Rapporto Differenziale è un numero finito, quindi esiste la parte standard.
\[
 \pst{RD}=\pst{2x+dx}=2x.
\]
Poiché la parte standard, che esiste, non dipende dal valore dell'infinitesimo 
\(dx\), allora esiste \(f'(x)\) ed è uguale a \(2x\).
\end{proof}

% \begin{figure}[h!]
 \begin{minipage}[]{.48\textwidth}
\begin{inaccessibleblock}
  [m tangenti a parabola]
 \begin{center}
 \parabola
 \end{center}
\end{inaccessibleblock}
 \end{minipage} 
 \hfill
 \begin{minipage}[]{.48\textwidth}
 \begin{center}
  \tangentiparabola
 \end{center}
 \end{minipage}
% \caption{\(y=x^2\) e la pendenza \(y=m(x)=2x\) delle sue tangenti.} 
% \label{}
% \end{figure}

%\vspace{1em}
Grafico di \(f(x)=x^2\) con alcune sue tangenti, per mostrare le 
diverse pendenze delle curva. \\
Accanto: grafico dell'insieme delle pendenze, cioè della funzione derivata 
\(f'(x)=2x\).\\
Iniziamo dal ramo sinistro del primo grafico: al crescere di \(x\), la curva e 
le sue tangenti passano da un'inclinazione fortemente verso il basso (\(m<0\)) 
alla direzione orizzontale, nel vertice della parabola. Per \(x>0\), poi, 
l'inclinazione aumenta progressivamente. Il progresso della pendenza delle 
tangenti è costante: per questo motivo il grafico della derivata è una
una retta per l'origine.

\begin{osservazione}
Nota che la funzione derivata di una funzione quadratica è una funzione 
lineare: la pendenza delle tangenti a una parabola varia come varia la \(y\)
rispetto alla \(x\) in una retta.
\end{osservazione}

\section{Teoremi sulle derivate}
\label{sec:differenziazione_teoremi}
\begin{definizione}
Una funzione per la quale la derivata è calcolabile si dice derivabile.
Se non si specifica diversamente, si intende che la pendenza della funzione 
è calcolabile per ogni \(x\) del suo dominio.
\end{definizione}

Per evitare calcoli laboriosi come quelli dell'Esempio \ref{esempio:esempio_5}, 
che in realtà sono fra i più agevoli che ci potrebbero capitare, 
cerchiamo regole di derivazione più immediate. 

In quell'esempio, la funzione è \(f(x)=\dfrac{1}{3}x^2-2x+5\),  una somma 
algebrica di tre addendi: \(\dfrac{1}{3}x^2\),\quad \(-2x\)\quad e\quad  \(+5 
\): non potremmo derivarli separatamente e poi unire i risultati 
sommandoli? È quello che stiamo per mostrare con i prossimi teoremi.

\subsection{Derivata della somma (algebrica) di funzioni}
\label{subsec:differenziazione_derivatasomma}
\begin{teorema}
 Se una funzione è la somma di più funzioni, anche la sua 
 derivata, se c'è, è la somma delle derivate degli addendi.
\end{teorema}
\noindent Ipotesi: \(f=g + h\) e esistono \(f'(x), \).\tab Tesi: \(f'=g' + 
h'\).\\
(Ovvero: La derivata di una somma è la somma delle derivate)

\begin{proof}
Consideriamo il Rapporto Differenziale, calcolato in \(x\), dove \(g\), 
\(h\) e quindi anche \(f\), hanno valore. Allora, per ogni infinitesimo non 
nullo \(dx\):
\begin{align*}
 RD&=\dfrac{df(x)}{dx}=\dfrac{d[g(x)+ h(x)]}{dx}=\dfrac{[g(x+dx)+
     h(x+dx)]-[g(x) + h(x)]}{dx}=\\
 &= \dfrac{[g(x+dx)-g(x)] + [h(x+dx)-h(x)]}{dx}= 
 \dfrac{g(x+dx)-g(x)}{dx}+\dfrac{h(x+dx)-h(x)}{dx}=\\
  &=\dfrac{dg(x)}{dx}+\dfrac{dh(x)}{dx}.
\end{align*}
Si è detto che \(f'(x)\) esiste, quindi il Rapporto Differenziale 
\(\dfrac{df(x)}{dx}\) è un numero finito e la sua parte standard non dipende da 
\(dx\). Di conseguenza anche i Rapporti Differenziali per \(g(x)\) e per
\(h(x)\) sono numeri finiti e le loro parti standard esistono e non possono 
cambiare valore cambiando \(dx\).\\
Per le proprietà della funzione parte standard vale quindi:
\[
 f'(x)=\pst{\dfrac{df}{dx}}=\pst{\dfrac{dg(x)}{dx}+\dfrac{dh(x)}{dx}}=
 \pst{\dfrac{dg(x)}{dx}}+\pst{\dfrac{dh(x)}{dx}}=g'(x)+h'(x).
 \]
\end{proof}

\begin{osservazione}
 Il teorema vale anche per la differenza, in modo analogo e si può applicare 
anche alla somma algebrica di più funzioni, per esempio ai polinomi.
\end{osservazione}

\begin{esempio}
 Deriviamo la funzione \(f(x)= x^2-x+12\).\\
 In base a quanto già sappiamo, la derivata di \(x^2\) è \(2x\), la derivata 
di \(x\) è \(1\) e la derivata di una costante è \(0\), quindi 
\(f'(x)=2x-1\).\\
Come si vede, non è stato necessario scrivere il Rapporto Differenziale.
\end{esempio}


\subsection{Derivata del prodotto di funzioni}
\label{subsec:differenziazione_derivataprodotto}

In questo caso non presentiamo una vera dimostrazione, ma ricorriamo a un
a un disegno che assumiamo riesca a descrivere efficacemente la 
regola.

Immaginiamo che le due funzioni, calcolate in un generico punto \(x\),
esprimano la base e l'altezza di un rettangolo:
\(b(x)=b\) sarà la base  e \(a(x)=a\) sarà l'altezza. 
L'area \(\mathit{S}\) ovviamente si ottiene da \(b(x)\cdot 
a(x)=\mathit{S}(x)\). 
Differenziare il prodotto \(d\quadra{\mathit{S}(x)}\) vuol dire calcolare di
quanto aumenta l'area del rettangolo, se i lati subiscono un incremento 
infinitesimo. 

\begin{osservazione}
Gli incrementi della base e dell'altezza possono essere 
diversi, perché \(b(x)\) e \(a(x)\) sono funzioni diverse, le quali possono
reagire in modo diverso all'incremento \(dx\).
\end{osservazione}

\begin{teorema}
 Se una funzione è il prodotto di due funzioni (le cui derivate esistono), la 
sua  derivata, se esiste, è la somma fra due prodotti: la derivata della
 prima funzione per la seconda (non derivata) più la prima funzione (non 
derivata) per il la derivata della seconda.
\end{teorema}
\noindent Ipotesi: \(\mathit{S}(x)=b(x)\cdot a(x)\) \text{ e esistono }
\(\mathit{S'}(x),\ b'(x),\ a'(x)\).\\ 
Tesi: \(\mathit{D}\quadra{\mathit S(x)}=b'(x)\cdot a(x)+b(x)\cdot a'(x)\).

\begin{figure}[h]
\begin{inaccessibleblock}
  [Rettangolo con uno gnomone finito e rettangolo con gnomone infinitesimo.]
 \begin{center}
 \begin{minipage}[]{.38 \textwidth}
  \vspace{23mm} 
  \incrementaleprodotto
 \end{minipage} 
 \hfill
 \begin{minipage}[]{.58 \textwidth}
  \differenzialeprodotto
 \end{minipage}
 \end{center}
\end{inaccessibleblock}
\caption{Incrementi finito e infinitesimo dell'area di un rettangolo} 
\label{fig:Incre_prodotto}
\end{figure}
%\vspace{-.5em}
Non daremo una vera dimostrazione della regola, ma ci lasceremo guidare da un 
disegno che spesso si usa per rappresentare il prodotto fra due binomi. 
Segnaliamo, per correttezza, che si potrebbe discutere sul fatto che un'area 
rappresenti un prodotto tra funzioni. A noi serve per avere una guida algebrica 
nel giustificare la regola.

Prima di impostare la derivata, ragioniamo sull'espressione 
dell'incremento infinitesimo di area che il disegno ci suggerisce. 

Concentriamoci sulla zona colorata del disegno, la figura a forma 
di L rovesciata, detta \emph{gnomone}. È formata da tre parti:
\begin{itemize} [nosep]
 \item un rettangolo sottile, verticale e sulla destra, di base
 infinitesima \(db(x)\) e altezza \(a(x)\);
 \item un rettangolo orizzontale, in alto, di base \(b(x)\) e
 altezza infinitesima \(da(x)\);
 \item un rettangolino in alto a destra, di area \(db(x) \cdot da(x)\).
\end{itemize}

Dato che l'ultimo termine è un infinitesimo di ordine
superiore, il risultato può essere approssimato alla sua parte principale, 
perché è indistinguibile:\\
\[d\mathit{S}(x)\sim db(x)\cdot a(x)+b(x)\cdot da(x).\]
Possiamo allora scrivere il Rapporto Differenziale:
\begin{align*}
\forall dx &\ne 0:\\
RD& =   \dfrac{d\mathit {S}(x)}{dx}=\dfrac{d[b((x)\cdot 
a(x)]}{dx}\sim \dfrac{db(x)\cdot a(x)+b(x)\cdot da(x)}{dx}=\\
&=\dfrac{db(x)\cdot a(x)}{dx}+\dfrac{b(x)\cdot da(x)}{dx}=
\dfrac{db(x)}{dx} a(x)+b(x)\dfrac{da(x)}{dx}.
\end{align*}
Per ipotesi esistono le derivate delle tre funzioni, perciò 
i Rapporti Differenziali sono numeri finiti e si può ricavare la regola:
\[
 S'(x)= b'(x)\cdot a(x)+ b(x)\cdot a'(x).
\]

\begin{esempio}
 Deriva la funzione \(f(x)=12x\). Si può svolgere questo semplice esercizio 
scrivendo il rapporto differenziale, ma ora vogliamo applicare la regola del 
prodotto.
\begin{align*}
 &f(x) = 12x\quad \text{ funzione 1: } a(x)=12\quad \text{ funzione 2: } 
b(x)=x.\\
 &\text{Derivate dei fattori: }a'(x)=0;\quad b'(x)=1.\\
 &\text{Regola: } f'(x)=a'(x)\cdot b(x)+a(x)\cdot b'(x),\quad \text{ quindi:}\\
 &f'(x)=0\cdot x + 12\cdot 1= 12.
 \end{align*}
 \end{esempio}

 \subsubsection{Derivata del prodotto per una costante}
\label{subsubsec:derivata_f_per_k}
 \begin{teorema}
La derivata di una funzione \(f(x)=k\cdot g(x)\), prodotto fra una 
funzione derivabile \(g(x)\) e una costante \(k\), è il prodotto fra la 
costante \(k\) e la derivata \(g'(x)\): 
\(\mathit{D}\quadra{k\cdot g(x)}=k \cdot g'(x)\).
\end{teorema}
\noindent Ipotesi: \(f(x)=k\cdot g(x) \text{ e esiste }g'(x)\).
\tab Tesi: \(f'(x)=k \cdot g'(x)\). 
\begin{proof}
Per la dimostrazione si può seguire la traccia dell'esempio precedente oppure 
impostare il Rapporto Differenziale e arrivare senza difficoltà alla 
conclusione.
\end{proof}


\begin{osservazione}
Questi due ultimi teoremi consentono una facile dimostrazione alla regola di 
derivazione per le funzioni lineari, alternativa a quella data nel Teorema 
\ref{subsec:differenziazione_derivataflineare}.
\end{osservazione}


\begin{esempio}
 %valore assoluto
 È derivabile la funzione \(f(x)=\dfrac{1}{2}|x-2|+2\)?\\
 La funzione contiene un valore assoluto: seguendo l’esempio 
\ref{esempio:diff01_derimodulo} riscriviamola come se fosse divisa in due rami:

% \begin{figure}[h!]
 \begin{minipage}[]{.33 \textwidth}
% \begin{center}
\begin{inaccessibleblock}
  [Derivate radice]
  \derivamodulodue
\end{inaccessibleblock}
%\end{center}
%   \caption{}
 \end{minipage} 
 \hfill
 \begin{minipage}[]{.65 \textwidth}
\(f(x)= \dfrac{1}{2}|x-2|+2
=\begin{cases}
  \dfrac{x-2}{2}+2 &\text{ per } x-2< 0  \\
 \dfrac{-(x-2)}{2}+2 &\text{ per } x-2 \ge 0
\end{cases} 
=\\ =
\begin{cases}
 \dfrac{x}{2}+1 &\text{ per } x<2 \\
 -\dfrac{x}{2}+3 &\text{ per } x\ge 2
\end{cases}.\)
\end{minipage}
% \end{figure} 
\vspace{1em}
Si tratta di due semirette che si uniscono in \(\punto{2}{2}\). L'equazione
di ciascuna di loro è una funzione lineare e già sappiamo che per ognuno dei 
rami la pendenza esiste ed è il coefficiente angolare. Quindi, anche senza 
calcolarla, abbiamo:
\(f'(x)= \begin{cases}
 \dfrac{1}{2} &\text{ per } x<2 \\
 -\dfrac{1}{2} &\text{ per } x > 2
\end{cases}\)\\
Quale è la pendenza giusta della tangente per \(x=2\), nel punto cioè dove il 
grafico cambia pendenza?

Per ragioni analoghe a quelle illustrate in \ref{esempio:diff01_derimodulo}
concludiamo che \(f'(2)\)  non esiste e quindi \(f\) è derivabile, ma non in 
tutto l'insieme di definizione.
\end{esempio}


\subsubsection{Derivata di una funzione potenza con esponente naturale}
\label{subsubsec:derivata_f_potenza_N}
La regola della derivata di un prodotto ci consente di dimostrare un ampio 
insieme di altre regole. Iniziamo da una derivata che abbiamo già 
dimostrato con il calcolo del Rapporto Differenziale e calcolato più volte.
\begin{esempio}
 Deriva la funzione \(f(x)=x^2\) con la regola del prodotto. 
\begin{align*}
 &f(x) = x^2=x\cdot x;\quad \text{ funzione 1: } a(x)=x\quad \text{ funzione 2: 
} 
b(x)=x.\\
 &\text{Derivate dei fattori: }a'(x)=1;\quad b'(x)=1.\\
 &\text{Regola: } f'(x)=a'(x)\cdot b(x)+a(x)\cdot b'(x),\quad \text{ quindi:}\\
 &f'(x)=1\cdot x + x\cdot 1= 2x.
 \end{align*}
 \end{esempio}
 Ora sviluppiamo alcuni casi successivi nella speranza di individuare una 
regola generale.
 \begin{esempio}
 Deriva la funzione \(f(x)=x^3\) con la regola del prodotto. 
\begin{align*}
 &f(x) = x^3=x^2\cdot x; \quad \text{ funzione 1: } a(x)=x^2\quad \text{ 
funzione 2: } 
b(x)=x.\\
 &\text{Derivate dei fattori: }a'(x)=2x;\quad b'(x)=1.\\
 &\text{Regola: } f'(x)=a'(x)\cdot b(x)+a(x)\cdot b'(x),\quad \text{ quindi:}\\
 &f'(x)=2x\cdot x + x^2\cdot 1= 3x^2.
 \end{align*}
 \end{esempio}
 
 \begin{esempio}
 Deriva la funzione \(f(x)=x^4\) con la regola del prodotto. 
\begin{align*}
 &f(x) = x^4=x^3 \cdot x;\quad \text{ funzione 1: } a(x)=x^3\quad \text{ 
funzione 2: } b(x)=x.\\
 &\text{Derivate dei fattori: }a'(x)=3x^2;\quad b'(x)=1.\\
 &\text{Regola: } f'(x)=a'(x)\cdot b(x)+a(x)\cdot b'(x),\quad \text{ quindi:}\\
 &f'(x)=3x^2\cdot x + x^3\cdot 1= 4x^3.
 \end{align*}
 \end{esempio}
 
 In effetti c'è una regolarità nei risultati e possiamo enunciare il teorema.
 
 \begin{teorema}
  \label{diff01_teoderpotenza}
  La derivata della funzione potenza \(f(x)= x^n\), con esponente 
naturale, è: \(\mathit{D}\quadra{x^n}=~nx^{n-1}\).
  \end{teorema}
\noindent Ipotesi: \(f(x)=x^n\), con \(n\in \N\) .\tab Tesi: 
\(f'(x)=nx^{n-1}\). 
\begin{proof}
%  Una funzione potenza è una funzione che dà come risultato la potenza 
% della variabile indipendente:
% \[f: x \mapsto x^n \quad \text{o} \quad 
%   f(x)=x^n \quad \text{o} \quad 
%   y = x^n\]
Ricaviamo per gradi il differenziale della funzione potenza \(f(x)=x^n\), 
con un procedimento per induzione.

Iniziamo dai casi già noti \(f(x)=x\) e \(f(x)=x^2\) e esaminiamo i successivi
aumentando progressivamente l'esponente. Ancora una volta non inseriamo 
l'indicazione relativa a \(x_0\), perché ci siamo accorti che è inutile. 
Infatti i risultati cambiano al cambiare di \(x_0\), il che equivale a dire che 
dipendono da \(x\).
\begin{align*}
  d(x) &=x+dx-x =dx\\
  d(x^2) &=(x+dx)^2-x^2 = x^2 +2xdx +(dx)^2 -x^2 = 2xdx +(dx)^2\\
  d(x^3) &=(x+dx)^3-x^3 =[x^3+3x^2dx+3x(dx)^2+(dx)^3]-x^3=
                      3x^2dx+3x(dx)^2+(dx)^3\\
  d(x^4) &=(x+dx)^4-x^4 = [x^4+4x^3dx+6x^2(dx)^2+4x(dx)^3+(dx)^4]-x^4=\\
                      &=4x^3dx+6x^2(dx)^2+4x(dx)^3+(dx)^4  \\
 d(x^5) &= (x+dx)^5-x^5 = [x^5+5x^4dx+\dots +(dx)^5 ]-x^5=\\
                      &=5x^4dx+\dots+(dx)^5\\
                      &\dots
\end{align*}
Possiamo osservare che il termine non infinitesimo si annulla sempre. Quindi,
qualunque sia il valore di \(x\), l'incremento della funzione è infinitesimo. 
Queste funzioni sono quindi continue in tutto \(\IR\) perché a variazioni 
infinitesime della variabile \(x\) corrispondono sempre variazioni
infinitesime della funzione.

I vari differenziali hanno un'espressione sempre più lunga, ma se consideriamo 
la parte principale di ogni espressione, cioè se trascuriamo 
gli infinitesimi di ordine superiore, il risultato è molto semplice.

% \newpage %----------------------------------------
Quindi se invece del valore esatto ci accontentiamo della parte principale, 
dato che è indistinguibile, concludiamo:
\nopagebreak
\begin{align*}
  d(x) &=x+dx-x =dx\\
  d(x^2) &=2xdx +(dx)^2 \sim 2xdx\\
  d(x^3) &=3x^2dx+3x(dx)^2+(dx)^3 \sim 3x^2dx\\
  d(x^4) &=4x^3dx+6x^2(dx)^2+4x(dx)^3+(dx)^4 \sim 4x^3dx\\
  d(x^5) &=5x^4dx+10x^3(dx)^2+10x^2(dx)^3+5x(dx)^4+(dx)^5 \sim 5x^4dx\\
  d(x^6) &=6x^5dx+\dots+(dx)^6 \sim 6x^5dx\\
  d(x^7) &=7x^6(dx)+\dots+(dx)^7 \sim 7x^6dx\\
  \dots &= \dots\\
  d(x^{10}) &=10x^9(dx)+\dots+(dx)^{10} \sim 10x^9dx\\
  \dots &= \dots\\
  d(x^n) &=nx^{n-1}(dx)+\dots+(dx)^{n} \sim nx^{n-1}dx.\\    
\end{align*}
Ora che il meccanismo è chiaro, abbiamo la regola per il differenziale di 
questa 
funzione: \(d(x^n) \sim nx^{n-1}dx\) 
e la sfruttiamo per ricavare il Rapporto Differenziale.
\[
 \forall dx\ne 0:\quad
 RD =\dfrac{d(x^n)}{dx}\sim \dfrac{nx^{n-1}dx}{dx}=nx^{n-1}.
\]
Il Rapporto Differenziale è indistinguibile da una quantità finita indipendente 
da \(dx\), possiamo calcolarne la parte standard. Questa operazione, fra 
l'altro, ci permette di eliminare gli infinitesimi di ordine superiore e quindi 
di scrivere il segno di \(=\) al posto del segno \(\sim\).
\[
 f'(x)=\pst{\dfrac{d(x^n)}{dx}}=\pst{nx^{n-1}}=nx^{n-1}.
\]
\end{proof}

Come esempio di derivata della funzione potenza, consideriamo \(f(x)=x^3\)
e il suo grafico nel piano cartesiano. 

\begin{minipage}[]{.48\textwidth}
\begin{inaccessibleblock}
  [m tangenti a cubica]
 \begin{center}
 \cubica
 \end{center}
\end{inaccessibleblock}
 \end{minipage} 
 \hfill
 \begin{minipage}[]{.48\textwidth}
 \begin{center}
  \tangenticubica
 \end{center}
 \end{minipage}
% \caption{\(y=x^3\) e la pendenza \(y=m(x)=3x^2\) delle sue tangenti.} 
% \label{}
% \end{figure}
%\vspace{1em}

I due rami del grafico sono simmetrici rispetto all'origine e quindi lo sono 
anche le pendenze delle tangenti. 

Considerando le \(x\) crescenti, quindi da sinistra verso destra, le pendenze 
delle tangenti sono sempre positive, all'inizio molto accentuate, poi 
diminuiscono fino a \(m=0\). Poi riprendono a crescere, in maniera 
sempre più evidente. Il grafico della funzione \(f'(x)=3 x^2\) ha 
infatti la forma di una parabola simmetrica rispetto all'asse \(y\).

\begin {esempio}
Ora disponiamo di un'arma molto più potente di prima, che, insieme alle 
precedenti, ci permette di calcolare derivate come questa.\\
Derivare la funzione polinomiale 
\(f(x)= -\dfrac{4}{15}x^5+ \dfrac{2}{3}x^3-13kx\).
 \begin{align*}
 &f(x) = -\dfrac{4}{15}x^5+ \dfrac{2}{3}x^3-13kx\\
 &\text{funzione 1:} a(x)=-\dfrac{4}{15}x^5\quad \text{ funzione 2: } 
b(x) =+ \dfrac{2}{3}x^3\quad \text{ funzione 3: } c(x) =-13kx.\\
 &\text{Derivate degli addendi: }a'(x)=\dfrac{4}{3}x^4;\quad b'(x)=+2 x^2;
 \quad c'(x)=-13k.\\
 &\text{Quindi: } f'(x)=\dfrac{4}{3}x^4+2x^2-13k.
 \end{align*}
\end {esempio}

 \subsubsection{Derivata di una funzione potenza con esponente intero}
\label{subsubsec:derivata_f_potenza_Z} 
 
 \begin{teorema}
  La derivata della funzione reciproca \(f(x)=\dfrac{1}{x}\) è: \quad
\(\mathit{D}\quadra{\dfrac{1}{x}}=-\dfrac{1}{x^2}\).
\end{teorema}
\noindent Ipotesi: \(f(x)=\dfrac{1}{x}\).\tab Tesi: \(f'(x)=-\dfrac{1}{x^2}\).
\begin{proof}
\begin{align*}
 &f(x)= \dfrac{1}{x}\quad\text{ con } x\ne 0.\quad\text{ Allora, } \forall 
dx \ne  0:\\
 &RD=\dfrac{df(x)}{dx}=\dfrac{\dfrac{1}{x+dx}-\dfrac{1}{x}}{dx}= 
\dfrac{\dfrac{x-x-dx}{x(x+dx)}}{dx}=
\dfrac{\dfrac{-\cancel{dx}}{x(x+dx)}}{\cancel{dx}}=\dfrac{-1}{x(x+dx)}.
\end{align*}
Date le ipotesi, il Rapporto Differenziale è un numero finito e 
possiamo applicare la definizione di derivata. Per le proprietà della funzione 
parte standard:
\[
 \pst{\dfrac{-1}{x(x+dx)}}=
\dfrac{-1}{\pst{x(x+dx)}}=\dfrac{-1}{\pst{x}\pst{x+dx}}=
      -\dfrac{1}{x\cdot x}=-\dfrac{1}{x^2}.
\]
Poiché il risultato non cambia \(\forall dx \ne 0\), la tesi è dimostrata.
\end{proof}

% \begin{figure}[h!]
\begin{inaccessibleblock}
  [Grafico della funzione 1/x e della sua derivata: 1/(x^2)]
 \begin{minipage}[]{.48\textwidth}
 \begin{center}
\scalebox{.9}{ \recip}
 \end{center}
 \end{minipage} 
 \hfill
 \begin{minipage}[]{.48\textwidth}
 \begin{center}
\scalebox{.9}{  \tangentirecip}
 \end{center}
 \end{minipage}
\end{inaccessibleblock}
%\caption{\(y=\dfrac{1}{x}\) e la pendenza \(y=m(x)=\dfrac{-1}{x^2}\) delle sue 
%   tangenti.} 
% \label{}
% \end{figure}
\begin{center}Grafico di \(f(x)=\frac{1}{x}\) e di \(f'(x)\) \end{center}

Se proseguiamo con gli esempi possiamo ricavare le regole di 
derivazione per le potenze successive \(f(x)=\dfrac{1}{x^2}\), 
\(f(x)=\dfrac{1}{x^3}\), e così via, con \(x\ne 0\). Ma ci arriviamo 
più comodamente come segue.

\begin{teorema}
 La regola della derivata di una funzione potenza vale anche se 
l'esponente è un intero negativo, quindi vale anche con esponente \(z\in\Z\).
\end{teorema}

\begin{proof}
 Riscriviamo: \(f(x)=\frac{1}{x}=x^{-1}\),\quad  \(f(x)=\dfrac{1}{x^2}= 
x^{-2}\), \dots , \(f(x)=\dfrac{1}{x^n}=x^{-n}\), con \(n\in \N\). 
Poiché \(n\geqslant 0\), fissiamo \(z=-n\), certamente positivo.
Possiamo a questo punto riusare la regola delle funzioni potenza:
\begin{align*}
f(x)&=x^{-n}=x^z, \quad \text{con }x\ne 0.\quad  \text{Allora esiste la 
derivata:}\\
f'(x)&= zx^{z-1}=-nx^{-n-1}.
\end{align*}
\end{proof}

\begin{esempio}
 Derivare \(f(x)=\dfrac{1}{x^2}\).\\
 \begin{align*}
  f(x)= \dfrac{1}{x^2}= x^{-2},\quad \text{con }x\ne 0.\\
  f'(x)=-2x^{-2-1}=-2x^{-3}=\dfrac{-2}{x^3}.
 \end{align*}
\end{esempio}


\subsection{Derivata del quoziente di funzioni}
\label{subsec:differenziazione_derivataquoziente}
\begin{teorema}
 Se una funzione derivabile è data dal rapporto fra due funzioni derivabili, 
con il denominatore non nullo, la sua derivata si ottiene calcolando
 la differenza fra due prodotti (la derivata del numeratore per il 
 denominatore meno il numeratore per la derivata del 
denominatore) e dividendo il risultato per il quadrato del denominatore.
\end{teorema}
\noindent Ipotesi: \(a(x)=\dfrac{\mathit{S}(x)}{b(x)}\), 
con \(b(x)\neq 0\) \tab 
Tesi: 
\(a'(x) = \dfrac{d\mathit{A}(x) \cdot b(x)-\mathit{A}(x) \cdot db(x)}
             {\tonda{b(x)}^2}\)

\begin{proof}
Ricorriamo alla geometria anche in questo caso: nel disegno \(\mathit{S}(x)\) è 
la funzione che genera i valori per l'area, \(b(x)\) e \(a(x)\) generano le 
possibili basi e altezze. Per prima cosa calcoliamo \(d\mathit{S}(x)\).

\begin{minipage}[]{.48 \textwidth}
Essendo \(\mathit{S}(x)=b(x) \cdot a(x)\) allora:\\ 
\(a(x)=\frac{S(x)}{b(x)}\), con \(b(x)\neq 0\).\\
Guardando il disegno, \(da(x)\) è l'incremento infinitesimo dell'altezza, si 
tratta dell'altezza della fascia superiore colorata, che si si calcola
dividendo il rettangolo superiore dello gnomone per la base del rettangolo.
Il rettangolo superiore dello gnomone è uguale a tutto lo gnomone 
infinitesimo, \(d\mathit{S}\), meno il rettangolo destro infinitesimo, di area
\(a\cdot db\) e meno il rettangolino, in alto a destra,
\end{minipage} 
 \hfill
 \begin{minipage}[]{.48 \textwidth}
 \begin{center}
 \begin{inaccessibleblock}
  [Altezza rettangolo con gnomone infinitesimo.]
  \differenzialerapporto
 \end{inaccessibleblock}
 \end{center}
\end{minipage}
anch'esso infinitesimo. Dunque:
\begin{align*}
 d\quadra{\frac{\mathit{S}(x)}{b(x)}}&=da(x)=
 \frac{\quadra{d\mathit{S}(x) - a \cdot db(x) - db(x) \cdot da(x)}}
        {b(x)}=\\
 &=\frac{\quadra{d\mathit{S}(x) - \dfrac{\mathit{S(x)}}{b(x)} \cdot db(x) - 
          db(x) \cdot da(x)}}{b(x)}=
 \frac{\dfrac{d\mathit{S}(x) \cdot b(x) - \mathit{S(x)} \cdot db(x) -
               b(x) \cdot db(x) \cdot da(x)}{b(x)}}{b(x)}=\\
 &=\frac{d\mathit{S}(x) \cdot b(x) - \mathit{S(x)} \cdot db(x) -
              b(x) \cdot db(x) \cdot da(x)}{\tonda{b(x)}^2} \sim 
 \frac{d\mathit{S}(x) \cdot b(x) - \mathit{S(x)} \cdot db(x)}{\tonda{b(x)}^2}.
\end{align*}
Come si vede, l'ultimo termine al numeratore nel risultato manca in quanto è 
infinitesimo di ordine superiore ed è quindi trascurabile rispetto agli altri 
termini. Vediamo ora il Rapporto differenziale e la derivata.
\begin{align*}
 a(x)&=\frac{\mathit{S}(x)}{b(x)}.\quad \forall dx\ne 0:\\
 RD  &=\dfrac{da(x)}{dx}\sim  \dfrac{\dfrac{d\mathit{S}(x) \cdot b(x) - 
\mathit{S(x)} \cdot db(x)}{\tonda{b(x)}^2}}{dx}=
\dfrac{d\mathit{S}(x) \cdot b(x) - \mathit{S(x)} \cdot 
 db(x)}{dx}\cdot\dfrac{1}{\tonda{b(x)}^2}=\\
&=\tonda{\dfrac{d\mathit{S}(x) \cdot b(x)}{dx}-
  \frac{\mathit{S(x)} \cdot  db(x)}{dx} }\cdot\dfrac{1}{\tonda{b(x)}^2}= 
\dfrac{\dfrac{\mathit{S(x)}}{dx}b(x)-
\mathit{S(x)}\dfrac{db(x)}{dx}}{\tonda{b(x)}^2}.
\end{align*}
Poiché \(\mathit{S(x)}\) e \(b(x)\) sono per ipotesi derivabili, esistono le 
parti standard dei Rapporti Differenziali, indipendenti dai \(dx\) non nulli 
usati, e si ottiene così \(
a'(x)=\dfrac{\mathit{S'(x)}\cdot b(x)-
      \mathit{S(x)} \cdot b'(x)}{(b(x))^2}\).
 
Tornando alla notazione abituale, la regola della derivata del rapporto di 
funzioni diventa:
\begin{align*}
 f(x) &=\dfrac{g(x)}{h(x)},\quad\text{ con } h(x)\ne 0.\\
 f'(x) &=\dfrac{g'(x)\cdot h(x)-g(x)\cdot h'(x)}{(h(x))^2}.
\end{align*}
\end{proof}
\begin{esempio}
 Anche se abbiamo già il risultato, applichiamo la regola alla funzione 
\(f(x)=\frac{1}{x}\), in cui poniamo \(g(x)=1\) e \(h(x)=x\), per cui 
\(g'(x)=0\) e \(h'(x)=1\).
\[
f'(x)= \dfrac{g'(x)\cdot h(x)-g(x)\cdot h'(x)}{(h(x))^2}=\dfrac{0\cdot 
x-1\cdot 1}{x^2}=-\dfrac{1}{x^2}.
\]
\end{esempio}

\begin{esempio}
Proviamo ora con \(f(x)= \dfrac{1}{x^2}\). 
\[
f'(x)= \dfrac{g'(x)\cdot h(x)-g(x)\cdot h'(x)}{(h(x))^2}=
\dfrac{0\cdot x^2-1\cdot 2x}{x^4}=-\dfrac{2}{x^3}.
\]
\end{esempio}

Grazie quest'ultima regola siamo in grado di derivare funzioni 
come la seguente:
\begin{esempio}
 Calcola la pendenza della funzione \(f(x)=\dfrac{x^2}{x-2}\) nel punto 
di ascissa \(x=1\).\\
%\begin{figure}[h]
\begin{inaccessibleblock}
  %\begin{center}
 \begin{minipage}[]{.40 \textwidth}
   \vspace{-.5em}
  \derivaomografica
 % \caption{}
 \end{minipage} 
 \hfill
 \begin{minipage}[]{.58 \textwidth}
Applichiamo direttamente la regola 
\ref{subsec:differenziazione_derivataquoziente}.
Dato che il numeratore e il 
denominatore sono entrambe funzioni derivabili e che \(x=1\) non annulla il 
denominatore, si ha: 

\begin{align*}
f'(x)&=\dfrac{2x\cdot(x-2)-x^2\cdot 1}{(x-2)^2}=\dfrac{x^2-4x}{(x-2)^2}.\\
f'(1)&=\dfrac{1^2-4\cdot 1}{(1-2)^2}=\dfrac{-3}{1}=-3.
\end{align*}
\(f'(x)\) è la derivata. \(f'(1)\) è la derivata calcolata per 
\(x=1\), cioè la pendenza della curva nel punto \((1,\ f(1)).\)
\end{minipage}
%\end{center}
\end{inaccessibleblock}
%\label{}
%\end{figure} 
Per \(x=1\), la curva ha pendenza \(m=-3\).\\
\end{esempio}

\subsubsection{Sintesi provvisoria}
\label{subsubsec:diff01_sintesiprovvisoria}
Abbiamo dimostrato le regole di derivazione per alcune funzioni elementari: 
\begin{enumerate} [noitemsep]
 \item \(\mathit{D}\quadra{k} = 0\) \tab  derivata di una 
costante;
 \item \(\mathit{D}\quadra{x} = 1\) \tab derivata della funzione identica;
 \item \(\mathit{D}\quadra{x^\alpha} = \alpha x^{\alpha-1}\) 
\tab  derivata della funzione potenza (\(\alpha \in \R\)).
\end{enumerate}
\begin{osservazione}
\label{oss:regola3}
 La regola 3 anticipa un risultato giustificato per ora solo con \(\alpha \in 
\Z\), ma che in realtà ha valore con qualsiasi esponente reale 
(vedi pag.~\pageref{teo:funzione_potenza_generica}).
\end{osservazione}

Conosciamo anche le regole di derivazione per alcune funzioni che sono 
somma, prodotto o quoziente di altre:
\begin{enumerate} [noitemsep]
 \item \(\mathit{D}\quadra{k\cdot f}=kf'\) \tab 
 derivata del prodotto per una costante;
 \item \(\mathit{D}\quadra{f\pm g}=f'\pm g'\) \tab 
 derivata di una somma o differenza;
 \item \(\mathit{D}\quadra{f\cdot g}= f'\cdot g+f\cdot g'\) \tab 
 derivata del prodotto;
 \item \(\mathit{D}\quadra{\dfrac{f}{g}}=\dfrac{f'\cdot 
g-f\cdot g'}{g^2}\)\tab 
derivata  del rapporto (\(g \ne 0\)).
\end{enumerate}

\subsubsection{Derivata della funzione radice quadrata}
\label{subsubsec:f_radice}
\begin{teorema}
  La derivata della funzione \(f(x)=\sqrt{x}\), con \(x \geq 0\), è: 
\(\mathit{D}\quadra{\dfrac{1}{2\sqrt{x}}}\), con \(x > 0\).
\end{teorema}
\noindent Ipotesi: \(f(x)=\sqrt{x}\), con \(x\geq 0\) .\tab Tesi: 
    \(f'(x)=\dfrac{1}{2\sqrt{x}}\).
\begin{proof}
 Calcoliamo separatamente \(df(x)\), che richiede una specie di 
razionalizzazione al contrario. Perciò, \(\forall dx\ne 0\) e se \(x\ne 0\):
  \begin{align*}
   df(x) &= f(x+dx)-f(x)=\sqrt{x+dx}-\sqrt{x}=
          \tonda{\sqrt{x+dx}-\sqrt{x}}\cdot
          \frac{\sqrt{x+dx}+ \sqrt{x}}{\sqrt{x+dx}+\sqrt{x}}=\\
       &=\frac{x+dx-x}{\sqrt{x+dx}+\sqrt{x}}=
         \frac{dx}{\sqrt{x+dx}+\sqrt{x}}\\
   RD&=\dfrac{df(x)}{dx}=\dfrac{dx}{dx\tonda{\sqrt{x+dx}+\sqrt{x}}}=
       \dfrac{1}{\sqrt{x+dx}+\sqrt{x}}.
  \end{align*}
Poiché il Rapporto Differenziale  con \(x\ne 0\) è un numero finito, 
esiste la parte standard. Allora:
\begin{align*}
 \pst{\dfrac{df(x)}{dx}}&= \pst{\dfrac{1}{\sqrt{x+dx}+\sqrt{x}}}=
      \dfrac{1}{\pst{\sqrt{x+dx}+\sqrt{x}}}=
      \dfrac{1}{\pst{\sqrt{x+dx}}+\pst{\sqrt{x}}}=\\
      &=\dfrac{1}{\sqrt{x}+\sqrt{x}}=\dfrac{1}{2\sqrt{x}}.
\end{align*}
La parte standard non cambia al cambiare del \(dx\) usato, quindi si conclude 
che la derivata esiste e 
\[
 f'(x)=\dfrac{1}{2\sqrt{x}}\quad (x>0).
\]
\end{proof}

% \begin{figure}[h!]
\begin{inaccessibleblock}
  [m tangenti a radquad]
 \begin{minipage}[]{.48\textwidth}
 \begin{center}
\scalebox{.9}{ \radquad}
 \end{center}
 \end{minipage} 
 \hfill
 \begin{minipage}[]{.48\textwidth}
 \begin{center}
\scalebox{.9}{  \tangentiradquad}
 \end{center}
 \end{minipage}
\end{inaccessibleblock}

Le rette tangenti ai punti vicini all'origine hanno una pendenza elevata, 
che si attenua gradualmente man mano che \(x\) aumenta, fino ad assestarsi quasi
orizzontalmente.

\begin{osservazione}
 Per confermare l'osservazione di pag.~\pageref{oss:regola3}  sulle 
funzioni potenza, adottiamo quella regola anche per il caso appena visto, che 
deriviamo di nuovo dopo averlo espresso come potenza.
\begin{align*}
 f(x)&=\sqrt{x}=x^{\frac{1}{2}},\quad \text{ con  } x\geq 0.\\
 f'(x)&=\dfrac{1}{2}x^{\frac{1}{2}-1}=\dfrac{1}{2}x^{-\frac{1}{2}}=
 \dfrac{1}{2x^{\frac{1}{2}}}=\dfrac{1}{2\sqrt{x}},\quad \text{ con  } x > 0.
\end{align*}
\end{osservazione}

\subsection{Differenziale e incremento}
\label{subsec:diff01_diff_inc}
Abbiamo visto più volte che il Rapporto Differenziale di una funzione e la 
sua derivata sono cose diverse: infatti, in una funzione derivabile,
\(f'(x_0)=\pst{RD}\). Applicando la parte standard si eliminano dal Rapporto 
Differenziale gli infinitesimi che il rapporto genera, così rimane solo il 
numero reale che si accetta come valore della pendenza.

Fra \(f'(x_0)\) e \(RD\) c'è solo una differenza di infinitesimi:
\(RD-f'(x_0)=\epsilon\). Concentrando i nostri ragionamenti su un particolare 
\(x_0\) e sviluppando le formule, si ha:\\
\(\dfrac{df(x_0)}{dx}-f'(x_0)=\epsilon\quad \rightarrow\quad
df(x_0)=f'(x_0)dx+\epsilon\cdot dx\quad \rightarrow\)\\
\(\rightarrow\quad f(x_0+dx)-f(x_0)=f'(x_0)dx+\epsilon\cdot dx
\rightarrow\quad f(x_0+dx)=f'(x_0)dx+f(x_0)+\epsilon\cdot dx\).\\
Se mancasse l'ultimo termine, l'equazione \(f(x_0+dx)=f'(x_0)dx+f(x_0)\) 
sarebbe quella di una retta: la si ottiene scrivendo \(x-x_0\) invece di \(dx\) 
e \(m\) invece di \(f'(x_0)\), cioè \(f(x)=m(x-x_0) +f(x_0)\). 
La retta, che passa per il punto della curva con ascissa \(x_0\) e ha la stessa 
pendenza della curva, è la sua tangente per quel punto. 

La conclusione è che \emph{a distanza \(dx\) dal punto che si vuole esaminare, 
fra la curva e la tangente c'è una differenza pari a \(\epsilon\cdot dx\), un 
infinitesimo di ordine superiore a \(dx\)}. Questa differenza è
\emph{l'incremento infinitesimo della curva rispetto alla tangente}.

\begin{inaccessibleblock}
  [differenziale della tangente]
  \begin{minipage}[]{.47\textwidth}
    \begin{center} \derivata \end{center}
 \end{minipage} 
  \hfill
 \begin{minipage}[]{.47\textwidth} \vspace{2.5em}
Nel punto \(\punto{x_0}{f(x_0)}\) il grafico della funzione e la tangente
sono indistinguibili. 
Il campo visivo del primo microscopio mostra \({x_0}\) e \(dx\), 
perché l'ingrandimento è infinito. A questo livello 
microscopico la curvatura del grafico non esiste,  per cui la curva e 
la tangente sono sovrapposte. Nemmeno un secondo microscopio infinito, con 
cui l'ingrandimento complessivo diventa \(\infty^2\), puntato su \(x_0\) può 
distinguere i due grafici. Ma un terzo microscopio, centrato a 
distanza infinitesima dal punto nel primo microscopio, mostra la tangente e 
la curva come rette parallele, che distano fra loro 
\(\epsilon\cdot dx\), cioè un infinitesimo di un infinitesimo.
 \end{minipage}
\end{inaccessibleblock}
\label{}

\begin{teorema}
\label{teo:inc}
 \emph{Teorema dell'incremento.} A distanza infinitesima dal punto di ascissa 
\(x_0\), la funzione derivabile \(f(x)\) ha un valore che è la somma fra 
il valore della sua tangente \(t(x)\) passante per \(\punto{x_0}{f(x_0)}\) e un 
infinitesimo dell'infinitesimo \(dx\):
\[f(x_0+dx)=\underbrace{f'(x_0)dx+f(x_0)}_{t(x)}+\epsilon\cdot dx.\]
\end{teorema}
L'importante conseguenza di questo teorema è che se ci troviamo in 
difficoltà a calcolare l'incremento infinitesimo di 
\(f\), possiamo considerare al suo posto l'incremento lungo la 
tangente, perché l'errore che si commette è una quantità trascurabile anche 
rispetto a \(dx\).

\subsection{Derivata di funzioni composte}
\label{subsec:differenziazione_derivatacomposta}

\begin{esempio}
  Deriva la funzione \(f(g)=\dfrac{g^2}{8}\). 
  Soluzione: \(f'(g)=\dfrac{1}{8}2g=\dfrac{g}{4}\). \\
Deriva la funzione \(g(x)=3x-2\). Soluzione: \(g'(x)=3\).\\
Combiniamo i due esempi: \(f=f(g)\) e \(g=g(x)\), cioè \(f\) è funzione di 
\(g\), nel senso che facciamo dipendere i suoi valori dai quadrati, divisi per 
\(8\), dei numeri \(g\).
Invece \(g\) è funzione di \(x\), nel senso che i suoi valori sono i valori 
\(x\) triplicati e poi ridotti di \(2\). abbiamo quindi la funzione 
\(f=f(g(x))\). Scritto in forma matematica: 
\(f(g(x))=\dfrac{[g(x)]^2}{8}=\dfrac{(3x-2)^2}{8}\).
\end{esempio}
\begin{inaccessibleblock}
  [box funzione composta]
 \begin{center}
 \begin{minipage}[]{.48\textwidth}
  \boxfcomposta
 \end{minipage} 
 \hfill
 \begin{minipage}[]{.48\textwidth}
Si tratta di una macchina che incatena due calcoli successivi.\\
Immettiamo ad esempio il valore \(t=6\). La macchina sviluppa al suo interno 
\(u(6)=3\cdot 6-2=16\) grazie a \(u\) e infine produce 
\(v(16)=\dfrac{16^2}{8}=32\).
Una catena del genere si chiama \emph{funzione di funzione} o
\emph{funzione composta}: in questo caso \(f(g(x))\).\\
 \end{minipage}
 \end{center}
\end{inaccessibleblock}
\label{}

Considerando che i valori \(x\) per arrivare alla trasformazione prevista 
da \(f\) devono prima essere trasformati da \(g\), come deriviamo  \(f\) 
rispetto a \(x\)? La formula sintetica è: 
\(f'(g(x))=\pst{\dfrac{df(g(x))}{dx}}\) e il punto da studiare è il calcolo 
del differenziale al numeratore.

\begin{esempio}
  Calcolare \(f'(x)\), con \(f(x)=\sqrt{3-x^2}\).\\
  \(f(x)\) è composta da due funzioni entrambe derivabili. Si può pensare
  formata così: \(g(x)=3-x^2\) e \(f(g(x))=\sqrt{g(x)}=\sqrt{3-x^2}\).\\
  Il differenziale \(df\) si potrebbe calcolare dalla definizione: 
 \(df(g)=f(g+dg)-f(g)\) e questo obbliga poi a sviluppare \(dg\) e inglobarlo 
nel calcolo. \\
 C'è una via più diretta e semplice: fare riferirmento al Teorema~\ref{teo:inc}.
La conseguenza del teorema è che si può calcolare il differenziale della 
tangente invece di quello della funzione, perché si tralasciano quantità 
che comunque sono trascurabili rispetto a \(dx\). Quindi:
\begin{align*}
 &df(g)\sim f'(g)dg\quad \rightarrow\quad 
d\tonda{\sqrt{g(x)}}\sim\frac{1}{2\sqrt{g(x)}}\cdot dg(x)\quad \rightarrow\quad 
RD=\dfrac{df(g(x))}{dg(x)}\sim\dfrac{1}{2\sqrt{g(x)}}.\\
&dg(x)=d(3-x^2)\sim-2x\cdot dx\quad \rightarrow\quad RD= \dfrac{dg(x)}{dx}\sim
-2x.\\
&RD=\dfrac{df(g(x))}{dx}\sim\dfrac{1}{2\sqrt{g(x)}}\cdot \dfrac{dg(x)}{dx}\sim
  \frac{1}{2\sqrt{3-x^2}}\cdot (-2x).
\end{align*}
Poiché le funzioni sono derivabili, possiamo scrivere direttamente:
 \[f'(x)=\pst{\dfrac{-2x}{2\sqrt{3-x^2}}}=
  \dfrac{-x}{\sqrt{3-x^2}}.\]
\end{esempio}

\begin{teorema}
  \label{teo:diff01_dericomp}
 Se esistono le derivate \(g'(x)\) e \(f'(g(x))\) per il medesimo valore \(x\),
 la funzione composta \(f(g(x))\) è derivabile e la sua derivata si calcola
 così: \(f'(x)=f'(g(x))=f'(g)\cdot g'(x)\), cioè la derivata di una funzione 
composta è il prodotto delle derivate delle funzioni componenti, ciascuna 
rispetto alla propria variabile.
\end{teorema}
\noindent Ipotesi: \(f(x)=f(g(x))\), \(f\), \(g\) derivabili .\tab 
Tesi: \(f'(x)=f'(g(x))=f'(g(x))\cdot g'(x)\).

\begin{proof}

Anzitutto, poiché esiste \(g'(x)\), \(dg(x)\) è infinitesimo per ogni 
infinitesimo \(dx \).\\
Poi, poiché esiste \(f'(g)\), dal teorema dell'incremento si deduce che per 
ogni 
infinitesimo \(dg\) c'è un infinitesimo \(\epsilon\) tale che 
\(df(g(x))=f'(g)\cdot dg + \epsilon\cdot dg.\)


Così, dividendo per qualsiasi infinitesimo non nullo \(dx\)
%{\footnotesize
 \begin{align*}
 &\pst{\frac{df(x)}{dx}}
 =\pst{\frac{df(g)}{dx}}=\\
 &=\pst{\frac{f'(g)\cdot dg + \epsilon\cdot dg}{d x}}=
 \pst{\frac{f'(g)\cdot dg}{d x}}+ \pst{\frac{\epsilon\cdot dg}{dx}}=\\
 &= f'(g)\cdot \pst{\frac{dg}{dx}}+ \pst{\epsilon}\cdot\pst{\frac{dg}{dx}}=
 f'(x)\cdot g'(x).
\end{align*}
\end{proof}

Vogliamo rappresentare nella stessa immagine le funzioni 
\(g(x)\), \(f(g)\) e \(f(x) = f(g(x))\).
Andrebbero rappresentate tutte nello stesso piano cartesiano, ma 
sovrapporle renderebbe difficile la comprensione. 
Proviamo a vedere se riproducendo più versioni dello stesso piano, 
specchiato e ruotato, si riesce a seguire il meccanismo della funzione 
composta.

\noindent\begin{minipage}{.32\textwidth}
\begin{center}
Partiamo dalla funzione \(g\):

\disegno[5]{
  \begin{scope}[red!50!black] \funzioneg \end{scope}
}
\end{center}
\end{minipage}
\begin{minipage}{.32\textwidth}
\begin{center}
\(g(x)\) specchiata:

\disegno[5]{
  \begin{scope}[yscale=-1, red!50!black] \funzioneg \end{scope}
}
\end{center}
\end{minipage}
\begin{minipage}{.32\textwidth}
\begin{center}
\(g(x)\) ruotata di \(-90\text{°}\):

\disegno[5]{
  \begin{scope}[rotate=-90, yscale=-1, red!50!black]
    \funzioneg
  \end{scope}
}
\end{center}
\end{minipage}

\noindent\begin{minipage}{.32\textwidth}
\begin{center}
Aggiungiamo la funzione \(f(t)\) specchiata rispetto a \(y\).

\disegno[5]{
  \begin{scope}[rotate=-90, yscale=-1, red!50!black]
    \funzioneg
  \end{scope}
  \begin{scope}[xscale=-1, blue!50!black]
    \funzioneh
  \end{scope}
}
\end{center}
\end{minipage}
\begin{minipage}{.65\textwidth}
\begin{center}
Mettiamo tutto insieme e anche, \(f(x) = f(g(x))\): \\[.5em]

\derivatacomposta
\end{center}
\end{minipage}


\begin{esempio}
  Derivare \(f(x)=\tonda{-\dfrac{3}{2}x^3+2x^2-6}^5\).\\
  Poniamo \(g(x)=-\dfrac{2}{3}x^3+2x^2-6\)\quad e\quad \(f(g)=g^5\). \\
  Allora: 
  \(f'(g)= 5g^4\)\quad e\quad \(g'(x)=-2x^2+4x\),\quad quindi: \\
  \(f'(x)=f'(g)\cdot g'(x)=5g^4(-2x^2+4x)=
  5\tonda{-\dfrac{2}{3}x^3+2x^2-6}^4(-2x^2+4x)\).
\end{esempio}
\begin{osservazione}
 La regola della funzione composta si estende ai casi in cui le funzioni 
 in gioco sono tre, o più:
 \(\mathit{D}\quadra{f(g(h(x)))}=f'(g)\cdot g'(h)\cdot h'(x)\).
\end{osservazione}



\subsection{Derivata di funzioni inverse}
\label{subsec:differenziazione_derivatainverse}
 Cosa si intende per funzioni inverse? \(y=x\cdot k\) e \(x=\frac{y}{k}\), per 
esempio, sono formule inverse l'una dell'altra, ma non sono funzioni inverse 
rispetto alla stessa variabile \(x\). Le due diverse espressioni esprimono la 
stessa iperbole equilatera e lo stesso grafico, quindi hanno le stesse tangenti 
al grafico e la stessa derivata rispetto a \(x\).\\
\begin{definizione}
 Un funzione \(x=g(y)\) si dice inversa di una funzione \(y=f(x)\) se la 
composizione delle due funzioni \(f(g(y))=x\).
\end{definizione}

\begin{osservazione}
Data una qualsiasi funzione \(f(x)\), non è scontato che la sua inversa esista 
\(\forall x\) nel dominio di \(f\). Per questo, quando si cerca l'inversa di 
una funzione, succede di dover restringere il dominio di questa. Per esempio, 
\(x=g(y)=\sqrt{y}\) è inversa di \(y=f(x)=x^2\),  perché 
\(f(g(y))=(g(y))^2=(\sqrt{x})^2=x\), ma la composizione è possibile 
solo se \(x \ge 0\), mentre \(f\) vale \(\forall x\). In questo caso, quindi, 
dobbiamo considerare per \(f\) il dominio più ristretto, perchè al di fuori di 
questo l'inversa non esiste.

Una buona regola pratica per capire se \(g=f^{-1}\) esiste, è tagliare 
il grafico di \(f\) con una retta orizzontale: se la
retta incrocia il grafico di \(f\) in più punti, \(f^{-1}\) non esiste.\\
\end{osservazione}

Considera il caso semplice che segue.
\begin{esempio}
  Derivare \(f(x)=\sqrt{x^2}\), con \(x \ge 0\).\\
  Si dirà: non c'è problema, \(f(x)\) corrisponde algebricamente a 
  \(f(x)=\sqrt{x^2}=x\), perciò  \(f'(x)=1\).\\
  Vero. Ma poniamo \(x=g(y)=y^2\) e \(f(x)=f(g(y))=\sqrt{g(y)}\).\\
  Con la regola delle funzioni composte si ha:\\ 
  \(f'(x)=f'(g)\cdot g'(x)= \dfrac{1}{2\sqrt{g}}\cdot 2x=
  \dfrac{1}{2\sqrt{x^2}}\cdot 2x=\dfrac{1}{2x}\cdot 2x=1\).\\
  Conclusione: 
La derivata \(\mathit{D}\quadra{x^2}=\dfrac{1}{\mathit{D}\quadra{\sqrt{x}}}\). 
In questo calcolo, inoltre, bisogna segnalare che \(g'(x)\) esiste se \(x\ne0\).
\end{esempio}
Si intuisce che: siccome \(f'(g)\cdot g'(x)=1\), allora 
\(g'(x)=\dfrac{1}{f'(g)}\).
L'intuizione è corretta ed effettivamente questa regola vale. 
Occorre però precisare che la regola vale
 \begin{enumerate} [noitemsep]
  \item se esiste l'inversa della funzione da derivare;
  \item se entrambe le funzioni sono derivabili;
  \item se  \(f'(g)\ne 0\).
 \end{enumerate}


\begin{inaccessibleblock}
  [differenziale funzione inversa]
 \begin{center}
 \begin{minipage}[]{.55\textwidth}
  \diffinversa
 \end{minipage} 
  \hfill
 \begin{minipage}[]{.42\textwidth}
Se valgono tutte le condizioni favorevoli, allora esistono la funzione
\(f\) e la sua inversa \(g=f^{-1}\). La funzione e la sua inversa, se esiste,
hanno grafici simmetrici rispetto alla bisettrice \(y=x\).

Ogni punto \(\punto{x}{f^{-1}(x)}\) sulla curva della funzione inversa ha un
corrispondente \(\punto{y}{f(y)}\) sulla curva \(y=f(x)\), nella simmetria 
rispetto alla bisettrice. Guardiamo come si corrispondono i differenziali:
\(dx\) e \(dy\) di una curva sono invertiti rispetto ai differenziali 
dell'altra.
Quindi le derivate corrispondenti sono reciproche l'una con l'altra.
 \end{minipage}
 \end{center}
\end{inaccessibleblock}
\label{}

\begin{teorema}
\label{teo:derinversa}
% In realtà l'ipotesi sarebbe diversa, vedi Ruggero appunti per corso Firenze
Le derivate di due funzioni \(f\), \(g\), inverse l'una dell'altra, se esistono
 e sono diverse da zero, sono reciproche l'una rispetto all'altra.
\end{teorema}
\noindent Ipotesi: \(y=f(x)\), \(x=g(y)\) \(f\), \(g\) derivabili, con 
\(f'\ne0\), 
\(g'\ne 0\).
\hspace{2cm} Tesi: \(f'(x)=\dfrac{1}{g'(y)}\).
\begin{proof}
  Grazie alle proprietà della funzione \(\pst{}\), abbiamo:\\
  \(f'(x)\cdot g'(y)=\pst{\dfrac{dy}{dx}}\cdot\pst{\dfrac{dx}{dy}}=
  \pst{\dfrac{dy}{dx}\cdot\dfrac{dx}{dy}}=\pst{1}=1\)\\
  per cui: \(f'(x)=\dfrac{1}{g'(y)}\).
\end{proof}
\begin{osservazione}
 Dire che il Rapporto Differenziale \(frac{dy}{dx}\) è reciproco di 
\(frac{dx}{dy}\) non è banale come dire che la frazione \(frac{3}{4}\) è 
reciproca di \(frac{4}{3}\).
 Una frazione è un rapporto fra numeri e genera un numero, il rapporto 
differenziale è un rapporto fra funzioni e genera una funzione. In più,  
in una frazione come la frazione \(frac{3}{4}\) i numeri \(3\) e \(4\) sono 
indipendenti, invece il differenziale \(dy\) dipende da \(dx\) nel rapporto
\(frac{dy}{dx}\),  e \(dx\) dipende da \(dy\) nel rapporto inverso. \\
\end{osservazione}

A proposito della derivata delle funzioni potenza, abbiamo anticipato che la 
regola \(f'(x)= \alpha x^{\alpha-1}\) vale anche con esponenti razionali. Ora 
siamo in grado di giustificarlo:
\begin{proof}
 Derivare \(y=f(x)=\sqrt[m]{x}\), con \(m\in \N\).\\
 Poiché l'inversa \(g(y)=y^m\) ha derivata \(g'(y)=my^{m-1}\), ricordando il 
Teorema~\ref{teo:derinversa} calcoliamo:\\
\(f'(x)=\dfrac{1}{g'(y)}=\dfrac{1}{my^{m-1}}=\dfrac{1}{m(\sqrt[m]{x})^{m-1}}=
\dfrac {1} {m} x^{-\frac{m-1}{m}}=\dfrac{1}{m} x^{\frac{1}{m}-1}\), con 
\(x\ne 0\),\\
ed è lo stesso risultato che si ottiene scrivendo \(f(x)=x^\frac{1}{m}\) e 
derivando in base alla regola delle funzioni potenza.

Il caso in cui \(y=f(x)=\sqrt[m]{x^n}=x^\frac{m}{n}\) si  dimostra in modo 
analogo, anche avvalendosi del Teorema \ref{teo:diff01_dericomp}.
Invece la dimostrazione valida per qualsiasi esponente reale è a 
pag.~\pageref{teo:funzione_potenza_generica}.
\end{proof}

\begin{esempio}
 Derivare \(y=f(x)=\sqrt[3]{x}\).\\
 Poiché l'inversa \(g(y)=y^3\) ha derivata \(g'(y)=3y^2\), per il 
Teorema~\ref{teo:derinversa}, calcoliamo:\\
\(f'(x)=\dfrac{1}{g'(y)}=\dfrac{1}{3y^2}=\dfrac{1}{3(\sqrt[3]{x})^2}=\dfrac{1}{3
} x^{-\frac{2}{3}}\), con \(x\ne 0\).\\
Se esprimiamo \(f(x)\) come potenza, abbiamo:\\
\(f(x)=\sqrt[3]{x}=x^\frac{1}{3}\). Applicando la regola \(f'(x)= \alpha 
x^{\alpha-1}\) 
risulta: \(f'(x)=\dfrac{1}{3}x^{\frac{1}{3}-1}=\dfrac{1}{3}x^{-\frac{2}{3}}\)
e le due derivazioni danno lo stesso risultato.
\end{esempio}

Nel prossimo esercizio sfruttiamo sia la regola per la derivata di una funzione 
composta che la regola per la derivata della funzione inversa.
\begin{esempio}
  Trova la derivata di \(f(x)=\dfrac{1}{\sqrt{5-x^2}}\).
  \begin{enumerate}[noitemsep]
   \item Usando il teorema \ref{teo:diff01_dericomp} e le regole 
precedenti:\\
   \(f'(x)=\mathit{D}\quadra{\dfrac{1}{\sqrt{5-x}}}=
   \mathit{D}\quadra{(5-x)^\frac{-1}{2}}=-\dfrac{1}{2}(5-x)
   ^\frac{-3}{2}(-1) = \dfrac{1}{2(\sqrt{5-x})^3}\).
   \item Usando la regola del Teorema~\ref{teo:derinversa}:\\
   Costruiamo la formula inversa con pochi passaggi algebrici: riavremo la 
   stessa funzione, in cui \(y\) figura come variabile indipendente: 
\(x=f(y)\).\\
   Quindi deriviamo: \(\mathit{D}\quadra{x}=x'=f'(y)=\pst{\dfrac{dx}{dy}}\).\\
   \(f(x)=y=\dfrac{1}{\sqrt{5-x}}\srarrow y^2=\dfrac{1}{5-x}\srarrow 
   y^{-2}=5-x\srarrow x=5-y^{-2}\)  (formula inversa)\\
   \(x'=\pst{\dfrac{dx}{dy}}=-2y^{-3}\) (derivata della funzione inversa)\\
   \(\srarrow y'=\pst{\dfrac{dy}{dx}}=\dfrac{y^3}{2}=
   \dfrac{1}{2(\sqrt{5-x})^3}\).
  \end{enumerate}
\end{esempio}

% \section{Derivata di funzioni trascendenti}
% \label{sec:differenziazione_trascendenti}
% 
% \subsection{Derivata della funzione esponenziale}
% \label{subsec:differenziazione_derivatafesponenziale}
% 
% \subsection{Derivata della funzione logaritmo}
% \label{subsec:differenziazione_derivataflogaritmo}
% 
% \subsection{Derivata della funzione seno}
% \label{subsec:differenziazione_derivatafseno}
% 
% \subsection{Derivata della funzione coseno}
% \label{subsec:differenziazione_derivatafcoseno}
% 
% \subsection{Derivata della funzione tangente}
% \label{subsec:differenziazione_derivataftangente}
% 
% \section{Riassunto}
% \label{sec:differenziazione_sunto}
% 
% \subsection{Differenziale}
% \label{subsec:differenziazione_differenziale}
% 
% \subsection{Schema riassuntivo}
% \label{subsec:differenziazione_schemaderivate}
% 
% \section{Applicazioni delle derivate}
% \label{sec:differenziazione_applicazioni}
% 
% \subsection{Derivata e tangente}
% \label{subsec:differenziazione_derivataetangente}
% 
% \subsection{Derivata e normale}
% \label{subsec:differenziazione_derivataenormale}
% 
% \subsection{Derivata della derivata}
% \label{subsec:differenziazione_derivataseconda}
% 
% \subsection{Altre applicazioni}
% \label{subsec:differenziazione_altreapplicazioni}


\section{Derivare funzioni trascendenti}
\label{sec:diff01_deritrasc}
Finora abbiamo imparato a derivare le funzioni algebriche. In 
questa sezione ci occupiamo della derivata delle funzioni trascendenti.

\subsection{Derivata di \(f(x)=a^x\)}
\label{subsubsec:deri_a_alla_x}
Il grafico di una generica funzione esponenziale \(y=a^x\), confrontato con 
il grafico dell'andamento delle sue pendenze è una sorpresa rispetto ai 
confronti che abbiamo fatto per altre funzioni. 
Prendiamo ad esempio la funzione \(f(x) = 2^x\).
Possiamo osservare che:
\begin{enumerate} [nosep]
\item dato che la funzione è sempre crescente: 
la sua \emph{pendenza è sempre positiva};
\item per valori negativi (molto piccoli) dell'argomento, 
la \emph{pendenza è molto vicina a zero};
\item per valori positivi dell'argomento la \emph{pendenza cresce molto 
rapidamente}.
\end{enumerate}
Anche la funzione ha un andamento simile:
\begin{enumerate} [nosep]
\item la \emph{funzione è sempre positiva};
\item per valori negativi (molto piccoli) dell'argomento, 
la \emph{funzione è molto vicina a zero};
\item per valori positivi dell'argomento la \emph{funzione cresce molto 
rapidamente}.
\end{enumerate}

La funzione e la sua derivata si assomigliano, sono funzioni della stessa 
``famiglia''.
% I due grafici praticamente si accompagnano: 
% rivelano uguali pendenze in coppie di punti di uguale ordinata.

\begin{inaccessibleblock}
  [esponenziale e pendenze]
\hspace{-20mm}\affiancati{.55}{.43}{
\begin{center} \scalebox{.8}{   \esp} \end{center}
}{
\begin{center} \scalebox{.8}{\derivataesp} \end{center}
}
\end{inaccessibleblock}
\label{}
\begin{center} Il grafico di \(y=a^x\), con alcune sue tangenti, e 
la sua derivata.\end{center}

% \begin{inaccessibleblock}
%   [esponenziale e pendenze]
%   \begin{minipage}[]{.49\textwidth}
% \begin{center} \scalebox{.8}{   \esp} \end{center}
%  \end{minipage} 
%   \hfill
%  \begin{minipage}[]{.49\textwidth}
% \begin{center} \scalebox{.8}{\derivataesp} \end{center}
%  \end{minipage}
% \end{inaccessibleblock}
% \label{}
% \begin{center} Il grafico di \(y=a^x\), con alcune sue tangenti, e 
% la sua derivata.\end{center}

Anche se i due grafici non sono identici, le pendenze delle tangenti sembrano
avere un andamento anch'esso esponenziale: la derivata della 
funzione ha un grafico che somiglia molto al grafico della funzione.
Sviluppiamo matematicamente questa intuizione, ricordando le proprietà
delle potenze.\\
Differenziale di \(f(x)=a^x\): \quad
\(df=a^{x+dx}-a^x=a^xa^{dx}-a^x=a^x\tonda{a^{dx}-1}\).\\
\(RD=\dfrac{df(x)}{dx}=\dfrac{\tonda{a^{dx}-1}}{dx}a^x\).\\
Per poter derivare occorre che il rapporto sia finito.

L'espressione del Rapporto Differenziale contiene sia \(a^x\), che è 
una funzione esponenziale e per ogni fissato \(x\) genera un numero finito, sia 
il fattore \(\dfrac{\tonda{a^{dx}-1}}{dx}\) che è da interpretare.\\
\(\dfrac{\tonda{a^{dx}-1}}{dx}= \dfrac{a^{(0+dx)}-a^0}{dx}=
\dfrac{d(a^x)}{dx}\bigg |_{x=0}= \dfrac{df(0)}{dx}\).\\
Il Rapporto Differenziale risulta:\\
\(RD=\dfrac{df(x)}{dx}=\dfrac{\tonda{a^{dx}-1}}{dx}a^x=
\dfrac{df(0)}{dx}\cdot a^x\).\\
Perciò il Rapporto Differenziale è un prodotto fra l'esponenziale stesso 
\(a^x\) e il Rapporto Differenziale di \(a^x)\) calcolato per \(x=0\).
Dobbiamo presumere che quest'ultimo sia un numero non infinito per proseguire 
nel ragionamento e in effetti lo potremo verificare fra poco.

Applicando la parte standard e si ottiene:\\
\(f'(x)= \pst{\dfrac{df(0)}{dx}\cdot a^x}=f'(0)\cdot f(x)\).\\
\emph{La derivata di una funzione esponenziale è proporzionale alla
funzione stessa, attraverso un fattore che corrisponde alla derivata 
calcolata in} \(x=0\).\\

Volevamo calcolare la derivata di \(f(x)=a^x\) e ci ritroviamo con
un risultato che contiene la derivata stessa \(f'(0)\), insomma non si direbbe 
che ci siano stati grandi progressi. Ma fingiamo per un attimo che \(f'(0)\) 
non incida sul risultato, cioè che \(f'(0)=1\). In questo modo la funzione e la
sua derivata sarebbero proprio identiche e i due grafici sarebbero 
sovrapponibili.\\
\(f'(0)=1\srarrow \pst{\dfrac{a^{dx}-1}{dx}}=1\srarrow a^{dx}\sim dx +1
\srarrow a\sim(dx+1)^\frac{1}{dx}\).\\
Abbiamo già incontrato un'espressione analoga in passato:
l'espressione individua il Numero di Nepero \(e=\pst{(dx+1)^\frac{1}{dx}}\).\\
Conclusione: perché una funzione esponenziale generica \(a^x\) coincida con la
sua derivata occorre che la base sia \(a=e\). \(f(x)=e^x\) è la funzione 
esponenziale pura.


\begin{teorema}
 La derivata della funzione esponenziale \(f(x)=e^x\) coincide con la funzione 
stessa: \(\mathit{D}\quadra{e^x}=e^x\)
\end{teorema}
\noindent Ipotesi: \(f(x)=e^x\). \tab \(f'(x)=e^x\).
\begin{proof}
 Non abbiamo dato una vera dimostrazione, abbiamo sviluppato un ragionamento 
a partire da un'intuizione sull'analogia dei due grafici. Per completarlo, 
occorrerebbe dimostrare l'unicità della tesi, ma non è essenziale per i nostri 
scopi. Resta comunque stabilito che  \emph{la funzione
 esponenziale pura \(f(x)=e^x\) coincide con la propria derivata}.
\end{proof}
Attraverso l'uso del numero \(e\) siamo finalmente in grado di derivare la 
funzione esponenziale generica \(f(x)=a^x\) e così risolvere anche i dubbi 
sulla finitezza del Rapporto Differenziale.

\begin{teorema}
  La derivata della funzione esponenziale generica \(f(x)=a^x\), con 
\(a>0\), è:\\ 
\(f'(x)= a^x\ln{a}\).
\end{teorema}
\noindent Ipotesi: \(f(x)=a^x\); \tab Tesi: \(f'(x)=a^x\ln{a}\).
\begin{proof}
 Usiamo una trasformazione appresa con lo studio dei logaritmi e applichiamo 
 il teorema a pag.~\pageref{teo:diff01_dericomp}:
 \(f(x)~=~a^x~=~e^{\ln a^x}\). Se poniamo \(g(x)=\ln a^x=x\ln a\), si ottiene:\\
 \(f(g(x))=e^{g(x)}\srarrow f'(g(x))=f'(g)g'(x)= e^{x\ln a}\ln a=
 e^{\ln a^x}\ln a=a^x\ln a\).
\end{proof}
\begin{esempio}
  Calcola la derivata di \(f(x)=3e^{x-1}\).\\
  Poniamo \(g(x)=x-1\). \(f(x)=3e^{g(x)}\srarrow f'(x)=3e^{g(x)}\cdot g'(x)=
  3e^{x-1}\cdot 1=3e^{x-1}\).
\end{esempio}
\begin{esempio}
  Calcola la derivata di \(f(x)=e^{x^2}\).\\
  Poniamo \(g(x)=x^2\srarrow f(x)=e^{g(x)}\srarrow f'(x)=e^{g(x)}\cdot g'(x)=
  e^{x^2}\cdot 2x=2xe^{x^2}\).
\end{esempio}

\begin{esempio}
  Calcola la derivata di \(f(x)=2^{x^2}\).\\
  Poniamo \(g(x)=x^2\srarrow f(x)=2^{g(x)}\).\\
  \(f'(x)=\tonda{2^{g(x)}\ln 2} g'(x)=   2^{x^2}\ln 2\cdot 2x=x2^{x^2+1}\ln 2\).
\end{esempio}

\subsection{Derivata di \(f(x)=\log_a x\)}
\label{}
\begin{esempio}
  Calcola la derivata di \(f(x)=e^{\ln x}\).\\
  Poniamo \(g(x)=\ln x\srarrow f(x)=e^{g(x)}\srarrow f'(x)=e^{g(x)}\cdot g'(x)=
  e^{\ln x}\) \dots ???.\\
  Ragioniamo: dalle proprietà
  dei logaritmi si ha: \(e^{\ln x} =x\), che è la funzione identica. Quindi
  \begin{enumerate}[noitemsep]
    \item \(e^{\ln x} =x\) e anche \(\ln e^x= x\ln e=x\), così come 
    \(f(f^{-1}(x))=f^{-1}(f(x))=x\): le due funzioni sono una inversa 
dell'altra,
    il logaritmo naturale \(g(x)=\ln x\) è la funzione inversa della
    funzione esponenziale \(f(x)=e^x\);
    \item 
\(\mathit{D}\quadra{f^{-1}(x)}=\dfrac{1}{\mathit{D}\quadra{(f^{-1}(x)}}\);
    \item \(\mathit{D}\quadra{\ln x}=\dfrac{1}{\mathit{D}\quadra{e^{g(x)}}}
    =\dfrac{1}{e^{\ln x}}=\dfrac{1}{x}\).
  \end{enumerate}
\end{esempio}

\begin{inaccessibleblock}
  [esponenziale e logaritmo]
  \begin{minipage}[]{.55\textwidth}
\begin{center} \scalebox{.8}{\esplog} \end{center}
 \end{minipage} 
  \hfill
 \begin{minipage}[]{.42\textwidth}
 \begin{teorema}
  La derivata della funzione \\
  logaritmo naturale è: \(\mathit{D}\quadra{\ln x}=
  \dfrac{1}{x}, \stext{con} x>0\).
\end{teorema}
\noindent Ipotesi: \(f(x)=\ln x, \mbox{ con } x>0\).\\
\noindent Tesi: \(f'(x)=\dfrac{1}{x}\).
\begin{proof}
La dimostrazione è nei ragionamenti dell'esempio precedente, ai quali 
bisogna aggiungere le precauzioni perché le due funzioni siano invertibili e 
derivabili: poiché \(\ln x\) esiste per \(x>0\), i ragionamenti valgono solo 
per 
\(x>0\)
\end{proof} 
 \end{minipage}
\end{inaccessibleblock}
\label{}
\\

Vediamo ora il caso generale, quando la base del logaritmo è genericamente 
\(a>0\).
\begin{teorema}
  La derivata della funzione logaritmo in base \(a\) è: 
  \(\mathit{D}\quadra{\log_a x}= \dfrac{1}{x\ln a}\).
\end{teorema}
\noindent Ipotesi: \(f(x)=\log_a x\). \tab \(f'(x)=\dfrac{1}{x\ln a}\)
\begin{proof}
Si ottiene direttamente dalla formula del cambiamento di base:\\
\(\log_a x=\dfrac{1}{\ln a}\ln x\).
\end{proof}

\begin{esempio}
    Derivare la funzione \(f(x)=Log(x^2+1)^2\).\\
    \(g(x)=(x^2+1)^2\srarrow f(x)=Log (g(x))\)\\
    \( f'(x)=\dfrac{1}{\ln 10}
    \dfrac{1}{g(x)}g'(x)=
    \dfrac{1}{(\ln 10)(x^2+1)^2}2(x^2+1)2x=\dfrac{4x}{(\ln 10)(x^2+1)}\).\\
    Nota che \(g(x)=(x^2+1)^2\) è a sua volta una funzione composta del tipo 
    \(g(x)=\quadra{h(x)}^2\) e quindi è stata applicata la regola della derivata
    di più funzioni composte.
\end{esempio}

Abbiamo ora tutti gli strumenti per convalidare l'osservazione al teorema 
\ref{diff01_teoderpotenza}, a proposito delle funzioni potenza.

\begin{teorema}
\label{teo:funzione_potenza_generica}
  La derivata della funzione potenza \(f(x)=x^\alpha\) è: \hspace{5mm}
  \(\mathit{D}\quadra{x^\alpha}=(\alpha-1)x^\alpha\), \(\forall\alpha\).
\end{teorema}
\noindent Ipotesi: \(f(x)=x^\alpha\). \tab \(f'(x)=(\alpha-1)x^\alpha\), 
\(\forall \alpha\).
\begin{proof}
Combinando alcune delle regole precedenti, si ha:\\
\(f(x)=x^\alpha=e^{\ln x^\alpha}=e^{\alpha\ln x}\)\\
\(f'(x)=e^{\alpha\ln x}\alpha\dfrac{1}{x}=x^\alpha\frac{\alpha}{x}=
\alpha x^{\alpha-1}\).\\
Poiché non è stata fatta nessuna particolare ipotesi sull'esponente (intero 
o razionale, positivo o negativo, o irrazionale), allora il teorema vale 
per qualsiasi esponente.
\end{proof}

\begin{esempio}
  Derivare \(f(x)=x^{\sqrt{2}}\).\\
  \(f'(x)=\sqrt{2}x^{\sqrt{2}-1}\).
\end{esempio}


\subsection{Derivata di funzioni circolari}
\label{}
Anche per queste funzioni dobbiamo dapprima definire il differenziale. 
Per una migliore comprensione, ci affidiamo soprattutto al piano cartesiano.
\subsubsection{Derivata di \(f(x)=\sen x\)}
Abbiamo già visto (pag.~\pageref{limiti:par_f_seno}) che per angoli 
infinitesimi il seno e l'angolo sono indistinguibili: \(\st\tonda{\frac{\sen 
\epsilon}{\epsilon}}=1\).
Dall'analisi del disegno ricaviamo l'espressione del differenziale
\(df(x)=d(\sen x)= \sen (x+dx) -\sen x\).

\begin{inaccessibleblock}
  [differenziale del seno]
  \begin{minipage}[]{.40\textwidth}
   \dseno 
 \end{minipage} 
  \hfill
 \begin{minipage}[]{.56\textwidth}
Nell'ingrandimento al microscopio non standard, l'incremento infinitesimo di
arco \(\overset{\frown}{AB}\) (che corrisponde all'incremento di angolo da \(x\)
a \(x+dx\)) è racchiuso fra due raggi indistinguibili da segmenti paralleli 
nei punti \(A\equiv \punto{x}{\sen x}\) e \(B\equiv \punto{x+dx}{\sen(x+dx)}\). 
L'arco, a sua volta, risulta indistinguibile dal segmento rettilineo \(AB\).
I segmenti che uniscono \(A\) e \(B\) con le loro proiezioni sull'asse \(X\) 
sono
verticali e paralleli, perciò \(ABC\) è un triangolo rettangolo infinitesimo, 
simile al triangolo \(BOC\). La sua altezza \(BC\) corrisponde a \(d\sen x\). 
 \end{minipage}
\end{inaccessibleblock}
\label{fig_diff01dseno}\\

Risolviamo il triangolo rettangolo \(ABC\) rispetto al lato \(BC\):\\
\(BC=AB\cdot \cos x \srarrow d(\sen x)= dx\cdot cos x\)  

\begin{teorema}
  La derivata della funzione \(f(x)=\sen x\) è \(\mathit{D}\quadra{\sen 
x}=\cos x\).
\end{teorema}
\noindent Ipotesi: \(f(x)=\sen x\). \tab \(f'(x)=\cos x\).
\begin{proof}
 Il commento al disegno giustifica la tesi. 
\end{proof}


\begin{osservazione}
  Si potrebbe criticare il metodo per la dimostrazione: chi assicura che 
negli
  altri quadranti le relazioni fra le variabili non cambino? Saremo troppo
  legati al disegno?\\
  Ci sono altri modi per dimostrare la tesi, più vincolati al calcolo
  e meno al disegno.  Per esempio, dalle formule di addizione abbiamo:
  \(RD=\sen(x+dx)=\sen x \cos dx + \sen dx \cos x\). Allora:\\
  \(\dfrac{\sen(x+dx)-\sen x}{dx}=\dfrac{\sen x \cos dx + \sen dx \cos x -
   \sen x}{dx}=\\
  =\sen x \dfrac{\cos dx-1}{dx}+ \cos x\dfrac{\sen dx}{dx}\sim
  \sen x\cdot 0 + \cos x\cdot 1=\cos x\),\\
  in cui si fa uso delle forme indeterminate discusse a pag. 
  \pageref{subsubsec:insnum_fseno}. Quando poi, per ottenere la derivata, 
si applica la parte standard, gli infinitesimi che vengono sottointesi dal 
segno \(\sim\) si eliminano e si ha l'uguaglianza in tutti i passaggi.
\end{osservazione}

\begin{osservazione}
Anche il grafico dell'andamento delle tangenti conferma la tesi
in modo assai espressivo. \\
\begin{inaccessibleblock}
  [differenziale del seno]
  \begin{minipage}[]{.47\textwidth}
    \begin{center} \seno \end{center}
 \end{minipage} 
  \hfill
 \begin{minipage}[]{.47\textwidth}
 \begin{center} \tangentiseno \end{center}
 \end{minipage}
\end{inaccessibleblock}
\label{}
\end{osservazione}

\begin{esempio}
Quale pendenza ha il grafico di \(y=\sen x\) nell'origine?\\
\(f(x)= \sen x \srarrow f'(x)=\cos x\srarrow f'(0)=\cos 0=1\).\\
La tangente al grafico nell'origine è la retta \(y=x\).
\end{esempio}

\begin{esempio}
Derivare \(f(x)=\sen^2 x\) e \(g(x)= \sen x^2\).\\
\(f'(x)=2\sen x \cos x\) e \(g'(x)=cos x^2\cdot 2x= 2x \cos x^2\).
\end{esempio}

\begin{esempio}
Derivare \(f(x)=\sen^2 x\) e \(g(x)= \sen 2x\).\\
\(f'(x)=2\sen x \cos x\) e \(g'(x)=cos 2x\cdot 2= 2 \cos 2x\).
\end{esempio}

\begin{esempio}
  Derivare \(f(x)=x^{\sen x}\).\\
  Si tratta di una funzione di tipo nuovo, un misto fra una funzione potenza
  e una funzione esponenziale. Si risolve con una trasformazione che abbiamo
  già visto e con l'uso delle regole della funzione composta e del prodotto.\\
  \(x^{\sen x}=e^{{(\ln x)}^{\sen x}}=e^{\sen x\ln x}\).\\
  \(f'(x)=e^{\sen x\ln x}(\cos x \ln x +\dfrac{\sen x}{x})=
  x^{\sen x}(\cos x \ln x +\dfrac{\sen x}{x})\).
\end{esempio}

 \subsubsection{Derivata di \(f(x)=\cos x\)}
\begin{teorema}
  La derivata della funzione \(f(x)=\cos x\) è \(\mathit{D}\quadra{\cos x}=
  -\sen x\).
\end{teorema}
\noindent Ipotesi: \(f(x)=\cos x\). \tab \(f'(x)=-\sen x\).
\begin{proof}
  Il disegno con cui dimostrare la tesi è uguale a quello di pag.
  \pageref{fig_diff01dseno}. Lo puoi riprodurre, tenendo però l'attenzione
  concentrata sul segmento \(AC\).\\
L'unica osservazione importante è che nel passare da \(x\) a \( x+dx\), cioè 
mentre l'angolo cresce, il valore del coseno decresce. Infatti, al contrario di 
quanto avviene per il seno, nel primo quadrante si ha: \( \cos(x+dx)<\cos x\). 
Questa è la ragione del segno meno nel risultato.
\end{proof}

\begin{inaccessibleblock}
  [differenziale del coseno]
  \begin{minipage}[]{.47\textwidth}
    \begin{center} \coseno \end{center}
 \end{minipage} 
  \hfill
 \begin{minipage}[]{.47\textwidth}
 \begin{center} \tangenticoseno \end{center}
 \end{minipage}
\end{inaccessibleblock}
\label{}

\begin{esempio}
Quale pendenza ha il grafico di \(y=\cos x\) per \(x=0\)?\\
\(f(x)= \cos x \srarrow f'(x)=-\sen x\srarrow f'(0)=-\sen 0=0\).\\
In \(x=0\) la tangente al grafico è orizzontale.
\end{esempio} 

\begin{esempio}
Derivare \(f(x)=\cos^2 x\) e \(g(x)= \cos x^2\).\\
\(f'(x)=2\cos x(-\sen x)=-2\sen x\cos x\) e \(g'(x)=-\sen x^2\cdot 2x=-2x\sen 
x^2\).
\end{esempio}

\begin{esempio}
Derivare \(f(x)=\cos^2 x +\sen^2 x\).\\
\(f'(x)=-2\sen x\cos x + 2\sen x \cos x = 0\).
\end{esempio}


\subsubsection{Derivata di \(f(x)=\tg x\)}

\affiancati{.39}{.59}{
La funzione \(f(x)=\tg x\) è discontinua per \(x= \pm\frac{\pi}{2}\). La 
derivata quindi non può esistere nei punti corrispondenti, come mostra il 
grafico.
}{
\begin{inaccessibleblock}
  [differenziale della tangente]
  \begin{minipage}[]{.49\textwidth}
    \begin{center} \tangente \end{center}
 \end{minipage} 
  \hfill
 \begin{minipage}[]{.49\textwidth}
 \begin{center} \tangentitangente \end{center}
 \end{minipage}
\end{inaccessibleblock}
\label{}
}
\begin{teorema}
   La derivata della funzione \(f(x)=\tg x\) è \(\mathit{D}\quadra{\tg x}=
   \dfrac{1}{\cos^2 x}=1+tg^2 x\) per \(x\ne \pm\frac{\pi}{2}\).
\end{teorema}
\noindent Ipotesi: \(f(x)=\tg x\). \tab \(f'(x)=\dfrac{1}{\cos^2 x}=1+tg^2 x\),
per \(x\ne\pm\frac{\pi}{2}\).
\begin{proof}
Per calcolare la derivata nei punti in cui la funzione è continua, 
ricorriamo alla seconda relazione fondamentale: \(\tg x=\frac{\sin x}{\cos x}\) 
e
sfruttiamo la regola della derivata di un quoziente (pag. 
\pageref{sec:diff01_regolederivate}).\\
\(\mathit{D}\quadra{\tg x}=\mathit{D}\quadra{\dfrac{\sin x}{\cos x}}=
\dfrac{\mathit{D}\quadra{\sen x}\cdot \cos x-\sen x\cdot 
  \mathit{D}\quadra{\cos x}} {\cos^2 x}=
\dfrac{\sen^2 x +\cos^2 x}{\cos^2 x}=\dfrac{1}{\cos^2 x}=\tg^2+1\)
\end{proof}

\begin {esempio}
Quale è la pendenza del grafico di \(y=\tg x\), per \(x=\dfrac{\pi}{4}\)? E per
\(x=\dfrac{\pi}{2}\)?\\
\(f'(x)=1+tg^2 x\srarrow f'(\dfrac{\pi}{4})=1+\tg^2\dfrac{\pi}{4}=2\)\\
\(f'(x)=1+tg^2 x\srarrow f'(\dfrac{\pi}{2})=1+\tg^2\dfrac{\pi}{2}=\) ???\\
Per \(x\approx\dfrac{\pi}{2}\) il grafico della funzione cresce verticalmente,
la sua pendenza è un numero infinito e la parte standard di un infinito non
esiste. D'altra parte, se \(x\) è esattamente uguale a \(\dfrac{\pi}{2}\) ,
la tangente ha un punto di discontinuità.
\end {esempio}


\section{Applicazioni}
\label{sec:diff01_applicazioni}
Si è tanto parlato delle tangenti ai grafici di funzione e delle loro 
pendenze,
senza mai arrivare a definire l'effettiva equazione delle tangenti che
interessano. Ora cercheremo di colmare questa lacuna.

\subsection{Derivata e tangente}
 Hai già incontrato negli anni scorsi dei problemi in cui si chiedeva di 
 calcolare la tangente ad una parabola in un suo punto. Il metodo di 
calcolo 
 algebrico che usavi è efficace ma macchinoso e, sfortunatamente, vale solo 
 per le coniche. Il metodo delle derivate, invece, si rivela
 molto più potente e rapido.\\
 Poiché la tangente è una retta, la sua equazione è del tipo 
\(y-y_0=m(x-x_0)\), 
 dove  \(\punto{x_0}{y_0}\) è il punto di tangenza 
 e \(m\) è la pendenza della retta, sulla quale sappiamo ormai tutto.
 Si ha \(y=m(x-x_0)+y_0\) e poiché \(m=f'(x_0)\), relativo alla funzione 
\(f(x)\)
 di cui si sta studiando il grafico, l'equazione risolvente è:\\
 \(y=f'(x_0)(x-x_0)+y_0\).
 
\begin{esempio}
  Trova le equazioni delle tangenti alla parabola \(f(x)=x^2\) nei suoi punti
  \(V~\equiv~\punto{0}{f(0)}\) e \(B\equiv\punto{-6}{f(-6)}\).\\
  Soluzione. Nel punto \(V\): \(f'(0)~=~2\cdot 0~=~0=m\). La tangente è 
  orizzontale e coincide con l'asse \(X\): \(y=~m(x~-~x_0)~+~y_0~=~0\).\\
  Nel punto \(B\): \(f'(-6)=2 (-6)=-12\). \(m=-12\), la tangente è inclinata
  verso il basso:
  \(y=m(x-x_0)+y_0= -12(x+6)+36\srarrow y=-12x-36\).
 \end{esempio}

 \pagebreak %--------------------------------------------------

\begin{esempio}
  Trova i punti di intersezione degli assi con la tangente in 
\(\punto{2}{f(2)}\)
  alla curva \(f(x)=2x^3-x\).\\ Soluzione.
  Ricerca della tangente per \(x=2\): \(f'(x)=6x^2-1\) e \(f'(2)=6\cdot 
4-1=23\).\\
  \(y_0=f(2)=2\cdot 2^3-2=14\). La tangente: \(y=23(x-2)+14=23x-32\).
  Le intersezioni: \\
  Con l'asse \(X\): \(y=0\srarrow x=\dfrac{32}{23}\srarrow
  \punto{\dfrac{32}{23}}{0}\).\\
  Con l'asse \(Y\): \(x=0 \srarrow y=-32\srarrow \punto{0}{-32}\).
\end{esempio}

\begin{esempio}
  In quale punto del suo grafico la parabola \(y=4x^2-3x+6\) è inclinata di 
  \(45^\circ\)?\\
  Soluzione. Nel punto che cerchiamo, la parabola avrà un'inclinazione 
  indistinguibile da quella della tangente.
  Le rette inclinate di \(45^\circ\) hanno pendenza \(m=1\), come la bisettrice 
  del primo-terzo quadrante.  Dobbiamo quindi imporre alla derivata il valore 
  \(1\).\\
  \(f(x)=4x^2-3x+6\srarrow f'(x)=8x-3\).\\
  \(8x-3=1\srarrow x=\dfrac{1}{2}\). Il punto è 
\(\punto{\dfrac{1}{2}}{\dfrac{11}{2}}\).
\end{esempio}

\begin{esempio}
  È vero che l'iperbole equilatera di equazione \(xy=16\) ha per vertici i 
punti 
  medi del segmento che gli assi staccano sulle tangenti ai vertici? \\
  Risposta. Consideriamo per comodità solo il ramo destro del grafico.
  Il vertice sarà un punto \(V\) di coordinate uguali, essendo l'iperbole
  equilatera. Quindi \(V=V\punto{4}{4}\).\\
  Poiché la funzione è \(y=\dfrac{16}{x}\), la sua derivata in \(V\) è 
  \(y'|_{x=4}= -\dfrac{16}{x^2}\bigg|_{x=4}=-1\)
  e l'equazione della tangente in \(V\) è \(y=-1(x-4)+4=-x+8\).\\
  La retta \(y=-x+8\) interseca gli assi in \(\punto{8}{0}\) e \(\punto{0}{8}\) 
ed 
  è facile verificare che il punto \(V\) è medio fra i due.
  Per ragioni di simmetria accade lo stesso
  con il vertice opposto \(\punto{-4}{-4}\).
  \begin{osservazione}
   In realtà si tratta di una proprietà generale dell'iperbole equilatera.
   Qualsiasi retta tangente al grafico stacca sugli assi coordinati dei 
   segmenti che hanno il punto medio coincidente con il punto di tangenza. 
   Non è difficile dimostrarlo usando l'equazione generica \(yx=k^2\) e per 
   punto di tangenza le coordinate \(\punto{a}{\dfrac{k^2}{a}}\).
  \end{osservazione}
 \end{esempio}

\begin{esempio}
  È vero che è inclinato di \(30^\circ\) il raggio  
  della circonferenza \(x^2+y^2=20\) che unisce l'origine al suo punto di 
ascissa
  \(4\)?\\
  Risposta. No, non è vero. \\
  Il modo più elementare per verificarlo è calcolare l'ordinata
  del punto e cercare l'angolo di inclinazione dell'ipotenusa coincidente 
con 
  il raggio.\\
  L'alternativa è calcolare la derivata:
  \(x^2+y^2=20\srarrow y=\sqrt{20-x^2}\) (data lo posizione del punto,
  consideriamo solo la semicirconferenza per \(y>0\)).\\
  \(f'(4)= \dfrac{-2x}{2\sqrt{20-x^2}}\bigg|_{x=4}=\dfrac{-4}{\sqrt{20-16}}=
  \dfrac{-4}{2}=-2\).\\
  Dunque la tangente ha una pendenza pari a \(-2\). Poiché il raggio e la 
  tangente sono perpendicolari, la retta che contiene questo raggio avrà 
  pendenza \(-\dfrac{1}{-2}=\dfrac{1}{2}\).\\
  Possiamo controllare la risposta con la calcolatrice.
  \end{esempio}
 
\subsection{Derivata e normale}
\label{}
Come si vede dall'ultimo esempio, una volta che si sappia come calcolare 
la tangente ad una curva, il calcolo della normale risulta molto facile.
Poiché la tangente e la normale, se passano per lo stesso punto, sono rette
perpendicolari e quindi hanno i coefficienti angolari antireciproci,
l'equazione di una normale ad una curva \(y=f(x)\) in un punto 
\(\punto{x_0}{y_0}\)
sarà:\\
\(y=\dfrac{-1}{f'(x_0)}(x-x_0)+y_0\),\\
dove la pendenza della normale \(m_n=\frac{-1}{m_t}\) è appunto 
l'antireciproco della pendenza della tangente.

\begin{esempio}
Scrivi l'equazione della tangente e della normale alla curva di equazione
\(y=\dfrac{x^2-1}{\ln x -1}\) nel suo punto di ascissa \(1\).\\
Soluzione. \(y'|_{x=1}=\dfrac{2x(\ln x-1)-(x^2-1)\dfrac{1}{x}}
{(\ln x -1)^2}\bigg|_{x=1}=\dfrac{2\cdot 1(0-1)-(1-1)\cdot 
1}{(0-1)^2}=-2\).\\
La pendenza della tangente è \(m=-2\). Per \(x=1\) la funzione vale:
\(\dfrac{1^2-1}{\ln 1 -1}=0=y_0\). L'equazione della tangente è quindi:
\(y=-2(x-1)=-2x+2\). Di conseguenza la normale ha equazione 
\(y=\dfrac{1}{2}(x-1)=\dfrac{1}{2}x-\dfrac{1}{2}\).
\end{esempio}


\begin{esempio}
Scrivi l'equazione della tangente e della normale alla curva di equazione
\(y=\dfrac{x^2+1}{\ln x +1}\) nel suo punto di ascissa \(1\).\\
Soluzione. \(y'|_{x=1}=\dfrac{2x(\ln x+1)-(x^2+1)\dfrac{1}{x}}
{(\ln x +1)^2}\bigg|_{x=1}=\dfrac{2\cdot 1(0+1)-(1+1)\cdot 1}{(0+1)^2}=0\).\\
La tangente è quindi una retta orizzontale. Di conseguenza la normale è 
verticale, come si vede subito se si prova a calcolare l'antireciproco di 
\(0\).
\end{esempio}

\subsection{Derivata della derivata}
\label{}
Abbiamo già notato che la derivata  di una funzione dipende dal punto in 
cui si 
calcola e che, una volta stabilito questo punto, ha un unico risultato,
se esiste.
Quindi la derivata di una funzione è a sua volta una funzione e,
se ci sono le condizioni, può essere derivata a sua volta.
\begin{definizione}
 Se una funzione \(f(x)\) è derivabile, la sua derivata è la funzione \(f'(x)\).
 Se anche \(f'(x)\) è derivabile, allora esiste la funzione \(f''(x)\) ed è
 chiamata \emph{derivata seconda di \(f(x)\)}.
\end{definizione}
Le regole di calcolo della derivata seconda sono le stesse regole che 
abbiamo già visto, quindi la seconda derivazione, se è possibile, non comporta 
problemi diversi da quelli conosciuti.\\
Riferendoci a un generico grafico di funzione \(y=f(x)\), la derivata prima 
\(f'(x)\) ci consente di trovare le pendenze delle tangenti al grafico. La
derivata seconda \(f''(x)\) descrive con quanta rapidità (o lentezza) variano 
queste pendenze, perciò ci indica quanto siano aperte o chiuse le concavità 
che \(y=f(x)\) disegna nel piano cartesiano.\\
Se le condizioni sono favorevoli, esistono e sono calcolabili anche le 
derivate terze, quarte, ecc. di una funzione, anche se non sono essenziali 
per i nostri scopi. Il loro calcolo segue i metodi già visti.

\begin{esempio}
Calcola \(f''(1)\) di \(f(x)=2x^5-3x^4+x^3+5x^2-6x+9\).\\
Derivata prima: \(f'(x)=10 x^4-12 x^3+3x^2+10 x-6\).\\
Derivata seconda per \(x=1\): \((40x^3-36x^2+6x+10)|_{x=1}=40-36+6+10=20\).
\end{esempio}

\begin{esempio}
Calcola \(f''(x)\) di \(f(x)=\ln x\).\\
\(f'(x)=\dfrac{1}{x}\) e \(f''(x)=-\dfrac{1}{x^2}\).\\
\begin{osservazione}
 La funzione \(\ln x\) esiste per \(x>0\). Le derivate prima e seconda 
esistono per \(x\ne 0\).
 In generale, l'esistenza di una derivata (prima, seconda, terza \dots) è
 indipendente dall'esistenza della funzione da derivare.
\end{osservazione}
\end{esempio}

\begin{esempio}
 Calcola le derivate successive di \(f(x)=\sen x\).\\
 \(f'(x)=\cos x\) \hspace{1cm}  \(f''(x)=-\sen x\) \hspace{1cm}
 \(f'''(x)=-\cos x\) \hspace{1cm} \(f^{IV}(x)=\sen x\) \dots
\end{esempio}

\begin {comment} 

\ifcoding
Con Python.
\lstinputlisting[firstline=2]{\folder src/01incremento.py} %, lastline=5]
\fi

Quindi la funzione ~ \texttt{incremento} ~ ha~\(3\) 
parametri, nel programma precedente l'ho invocata tre volte con argomenti 
diversi e ottenendo: nel primo caso~\(6\), nel secondo~\(2\) e nel 
terzo~\(0\). 
Ovviamente, utilizzando una funzione diversa, gli incrementi calcolati con 
gli stessi parametri saranno, in generale diversi. (Prova ad esempio 
con la funzione \(f(x)= 2^x\))

\begin{comment}
\subsubsection{Differenziale di una funzione}
\label{subsubsec:diff01_parteprincipale}

\subsection{Differenziale della variabile \(x\)}
=======
In matematica, e nelle sue applicazioni, sono particolarmente importanti 
gli incrementi che una funzione subisce quando la variabile 
indipendente (\(x\)) varia di una quantità infinitesima. Incrementi di 
questo tipo, se esistono e se sono infinitesimi, si chiamano ''differenziali``.

\begin{definizione}
 Il differenziale di una funzione è l'incremento infinitesimo, se c'è, 
 che la funzione subisce a causa di una variazione infinitesima della sua 
variabile indipendente \(x\), a partire da un valore fissato \(x\):
\[df(x) = f \tonda{x + \epsilon} - f(x).\]
\end{definizione}

Come già visto per l'incremento finito \(\Delta f\), anche il differenziale 
\(df\) 
dipende dall'espressione della funzione, dal valore fissato dove inizia la 
variazione della \(x\) e dal suo incremento infinitesimo~\(\epsilon\).\\
Quindi il differenziale di una funzione \(df\) è esso stesso  una funzione:
\(df=F(f, x, \epsilon)\).
\begin{osservazione}
Per segnalare che il simbolo \(df\) (oppure \(dx\), ecc.) non è il prodotto fra 
due variabili ma indica una differenza infinitesima, nella lettura si pronuncia:
\textit{de effe (de ics)}.
\end{osservazione}
\begin{osservazione}
I differenziali, essendo infinitesimi, sono osservabili solo con 
microscopi non standard.
\end{osservazione}

\begin{minipage}{.48 \textwidth}
\begin{esempio}
Vogliamo calcolare l'incremento della 
funzione:~\(f(x) = \dfrac{1}{4} x^2 -x -3\)
quando \(x\) parte da~\(7\) e aumenta di~\(\epsilon\).
\begin{align*}
  df(7) &= f(7+\epsilon) - f(7) = \\
        &= \dfrac{\tonda{7 +\epsilon}^2}{4}  -\tonda{7 +\epsilon} -3 - 
           \dfrac{7^2}{4}  +7 +3 =\\
        &= \dfrac{49 +14 \epsilon +\epsilon^2}{4} -10 -\epsilon - 
           \dfrac{49}{4} +10 =\\
        &= \dfrac{14 \epsilon +\epsilon^2}{4} -\epsilon 
        = \dfrac{10 \epsilon +\epsilon^2}{4} =\\
        &= 2,5 \epsilon + \dfrac{\epsilon^2}{4}, \quad \forall \epsilon. 
%\sim 2,5 \epsilon
\end{align*}
\end{esempio}
\end{minipage}
 \hfill
\begin{minipage}{.48 \textwidth}
 \begin{center}
\differenziale
 \end{center}
\end{minipage}
L'incremento della funzione è un differenziale: un infinitesimo che è la somma 
di due infinitesimi di ordine diverso. Il risultato non cambia 
sostanzialmente, se al posto di \(\epsilon\) svolgiamo lo stesso calcolo 
utilizzando un diverso infinitesimo \(\delta\). Per significare che il 
risultato non dipende dalla scelta dell'infinitesimo, si indica: \(\forall 
\epsilon\).

 
\subsection{Differenziale della variabile \(x\)}
>>>>>>> 32ec66d75f147bd133f36affa54bb3284453feee
\label{subsec:diff01_diffx}

Chiamiamo \(dx\) la differenza fra i due valori 
infinitamente vicini della funzione \(f(x)=x\) (\(dx\) si legge 
''\emph{de x}``).

Vogliamo calcolare il differenziale di \(f(x)=(x\) partendo dal 
punto \(x=x_0\) quando l'incremento è \(\epsilon\). In simboli: 
\(dx|_{x_0}\). 
Questa espressione si legge: ''\emph{de x, a partire da x zero}``.

Svolgendo i calcoli si ottiene:
\[dx|_{x_0}=(x_0+\epsilon)-x_0=\epsilon\]
Si può osservare che il risultato non dipende dal valore in cui viene 
calcolato il differenziale, ma solo dal valore dell'incremento \(\epsilon\),
infatti, nel risultato, \(x_0\) scompare.

\begin{osservazione}
 L'infinitesimo \(\epsilon\) potrebbe anche essere negativo, in questo caso 
sarebbe un ''decremento``. 
Il segno di \(\epsilon\) non cambia comunque il calcolo.

\(x_0 + \epsilon\) indica valori che possono trovarsi a destra di \(x_0\) 
(più grandi) o alla sua sinistra (più piccoli).
\end{osservazione}

\begin{comment}
\begin{esempio}
 Calcola il differenziale della variabile \(x\) nel punto \(x=-7\). Ripeti poi 
 il  calcolo in altri punti.\\
  \(dx|_{x=-7} =(-7+\epsilon)-(-7)=\epsilon\)\\
  \(dx|_{x=7} =(7+\epsilon)-7)=\epsilon\)\\
  \(dx|_{x=3} =(3+\epsilon)-3=\epsilon\)\\
  \(dx|_{x=\frac{1}{4}} = 
          \tonda{\frac{1}{4}+\epsilon}-\frac{1}{4}=\epsilon\)\\
  \(...\quad = \qquad ...\)\\
  \(dx|_{x=a}  =(a+\epsilon)-a=\epsilon\)\\
  \(dx|_{x=x_0}  =\ ...\ =\epsilon\)\\
  Se il risultato del differenziale è indifferente da \(x_0\), allora si 
 evita di indicare \(|_{x=x_0}\): \(dx=\epsilon\), \(\forall x\). 
\end{esempio}

\begin{esempio}
 Calcola il differenziale della variabile \(frac{10}{13}x\) nei punti \(x=9\) 
e 
 \(x=-\dfrac{1}{5}\).\\
 
\(d\tonda{\dfrac{10}{13}x}\bigg|_{x=9}=\quadra{\dfrac{10}{13}
\cdot(9+\epsilon)}-
  \dfrac{10}{13}\cdot 9 = \dfrac{10}{13}\epsilon.\)\\
 
\(d\tonda{\dfrac{10}{13}x}\bigg|_{x=-\dfrac{1}{5}}=\quadra{\dfrac{10}{13}
\cdot
  \tonda{-\dfrac{1}{5}+\epsilon}}-
 \dfrac{10}{13}\cdot \tonda{-\dfrac{1}{5}}= \dfrac{10}{13}\epsilon.\)\\
 \(d\tonda{\dfrac{10}{13}x}=\dfrac{10}{13}\epsilon\), \(\forall x\).
 \end{esempio}

Anche se due risultati uguali non bastano per fare una prova, e nemmeno i 
sei del primo esempio, si può essere sicuri che mille altri tentativi non
sortirebbero un esito diverso. La prova si ottiene utilizzando \(x_0\) 
(oppure una costante analoga) al posto di un valore numerico.

\begin{osservazione}
 L'infinitesimo \(\epsilon\) potrebbe anche essere negativo. Questo non 
cambierebbe il calcolo.\\
\end{osservazione}

L'uso di un valore numerico al posto di \(x_0\) è essenziale per precisare il
punto a partire dal quale si vuole svolgere il calcolo. Negli esempi 
precedenti tale indicazione è risultata indifferente, ma nella maggior 
parte dei casi, invece, ha un diretto influsso sul risultato.\\

\begin{esempio}
 Calcola \(df(x)|_{x=5}\), con \(f(x)=x^2\). Calcola poi \(df(x)|_{x=-5}\) 
 e infine \(df(x)|_{x=2}\).
 \begin{align*} 
  d(x^2)|_{x=5} & 
=(5+\epsilon)^2-5^2=25+10\epsilon+\epsilon^2-25=10\epsilon+\epsilon^2\\
  d(x^2)|_{x=-5}& 
=(-5+\epsilon)^2-(-5)^2=25-10\epsilon+\epsilon^2-25=-10\epsilon+\epsilon^2\\
  d(x^2)|_{x=2} & 
=(2+\epsilon)^2-2^2=4+4\epsilon+\epsilon^2-4=4\epsilon+\epsilon^2\\
 \end{align*}
\end{esempio}


\begin{osservazione}
 Non abbiamo fatto alcuna ipotesi su \(\epsilon\). 
 Potrebbe essere un infinitesimo positivo o negativo, 
 potrebbe essere il triplo o il quadrato di un altro infinitesimo. 
 Il risultato non cambia e ha valore per qualsiasi \(\epsilon\).
\end{osservazione}

\subsection{Differenziale di alcune funzioni}
\label{subsec:diff01_difffun}

% Nello svolgere i calcoli che seguono potremo indicare il differenziale in due
% modi diversi, ma equivalenti. Per esempio: \(df(7)\), come nell'esempio 
% precedente, oppure \(df(x)|_{x=7}\).\\
% Altro esempio: se \(f(x)= 5x-9\), il suo differenziale calcolato in \(x_0\) 
si 
% indica con:\\
% \(df(x_0)=f(x_0+\epsilon)-f(x_0)=5(x_0+\epsilon)-9- (5x_0-9)=\dots\)\\ oppure 
% con: \(df(x)|_{x=x_0}=f(x_0+\epsilon)-f(x_0)=\dots\). 

\begin{osservazione}
 L'infinitesimo \(\epsilon\) potrebbe anche essere negativo, in questo caso 
rappresenterebbe un ''decremento``. 
\(x_0 + \epsilon\) indica numeri iperreali che sulla retta iperreale possono
trovarsi a destra di \(x_0\) (più grandi) o alla sua sinistra (più piccoli).\\
\noindent Quando il segno di \(\epsilon\) diventerà importante ai fini del 
calcolo, ne tratteremo esplicitamente.\\
\end{osservazione}

Iniziamo a differenziare le funzioni più semplici, in un generico punto 
\(x_0\). Ma prima di tutto, una precisazione essenziale

\begin{osservazione}
 Il differenziale di una funzione è calcolabile solo negli intervalli in cui
 la funzione è continua, perché solo in questo caso a incrementi 
\emph{infinitesimi} di \(x\) corrispondono incrementi \emph{infinitesimi} di 
\(f(x)\).
\end{osservazione}

\subsubsection{Funzione costante}
\label{subsubsec:diff01_diffcostante}

Una funzione \(f(x)\) è costante se qualunque sia il valore di \(x\) il 
risultato 
è sempre lo stesso. Possiamo indicare questa funzione in diversi modi:
\[f: x \mapsto k \quad \text{o} \quad f(x)=k \quad \text{o} \quad y = k.\]

Il suo differenziale sarà:
\[df(x_0)=f(x_0+\epsilon)-f(x_0)=k-k=0,\quad \forall \epsilon.\]
Quindi, se la funzione è costante, il suo differenziale è nullo.
Infatti, avendo sempre lo stesso valore per qualsiasi \(x\), la differenza 
tra due suoi valori è zero. 

%\begin{figure}[h]
\begin{inaccessibleblock}
  [Differenziale di f costante]
 \begin{center}
 \begin{minipage}[]{.38 \textwidth}
  \diffcostante
%  \caption{\(y=k\srarrow dy=0\)}
 \end{minipage} 
 \hfill
 \begin{minipage}[]{.58 \textwidth}
Resta così dimostrato il seguente   
\begin{teorema}
Il differenziale di una costante è nullo.
\end{teorema}
Nel piano cartesiano, la funzione  \(y=k\) è una retta orizzontale e, come 
tutte le rette, è una funzione continua. Quindi il risultato non dipende da 
\(x_0\) e vale su tutto l'asse iperreale.
 \end{minipage}
 \end{center}
\end{inaccessibleblock}

\label{fig:diff01_diffcostante}
%\end{figure}

\subsubsection{Funzione identica}
\label{subsubsec:diff01_diffidentica}

La funzione identica (o identità) è una funzione che riceve un valore e dà 
come risultato lo stesso valore ricevuto. Possiamo indicare questa funzione 
in diversi modi:
\[f: x \mapsto x \quad \text{o} \quad f(x)=x \quad \text{o} \quad y = x.\]

Se \(f(x)=x\), allora, banalmente: \(df(x)=dx=\epsilon\). Il risultato è
generale, cioè non dipende da \(x_0\). Infatti:\\
\[df(x_0)=f(x_0+\epsilon)-f(x_0)=(x_0+\epsilon)-x_0=\epsilon,\]
ma anche
\[
 df(x)=f(x+\epsilon)-f(x)=(x+\epsilon)-x=\epsilon, \quad \forall \epsilon.
\]

%\begin{figure}[h!]
\begin{inaccessibleblock}
  [Differenziale di f costante]
 \begin{center}
 \begin{minipage}[]{.38 \textwidth}
  \rettabisettrice
%  \caption{\(y=x\srarrow dy=dx\)}
 \end{minipage} 
 \hfill
 \begin{minipage}[]{.55 \textwidth}
È dimostrato così il seguente
\begin{teorema}
Il differenziale della funzione identica è \(dx=\epsilon\).
\end{teorema}
D'ora in poi, per indicare una variazione infinitesima della variabile 
\(x\) potremo usare indifferentemente \(dx\) oppure \(\epsilon\), 
dato che sono equivalenti.\\
Il grafico di \(f(x)=x\) nel piano cartesiano è dato dalla retta \(y=x\). 
Che significato dobbiamo attribuire a \(dy=dx\)?
L'uguaglianza dei due differenziali indica che due punti infinitamente
vicini sulla retta individuano sugli assi due differenze infinitesime 
uguali.
 \end{minipage}
 \end{center}
\end{inaccessibleblock}
\label{fig:diff01_diffcostante}
%\end{figure}

Anche con altre rette, più o meno inclinate e passanti o non passanti per 
l'origine, succede che uno spostamento infinitesimo sull'asse \(X\) causi 
un identico spostamento sull'asse \(Y\)?

\subsubsection{Funzione lineare}
\label{subsubsec:diff01_flineare}

Una funzione lineare è una funzione espressa da un polinomio di primo grado:
\[f: x \mapsto mx +q \quad \text{o} \quad 
  f(x)=mx +q \quad \text{o} \quad 
  y = mx +q.\]

\begin{esempio}
 Iniziamo con un esempio numerico, supponiamo \(m=\dfrac{2}{3}\) e \(q=4\).
 Proviamo quindi a differenziare in \(x_0\) la 
 funzione \(f(x)=\dfrac{2}{3}x +4\).
\begin{align*}
df(x_0) &=f(x_0+dx)-f(x_0)=\\
             &=\frac{2}{3}(x_0+dx)+4-\tonda{\frac{2}{3}x_0+4}=
                 \frac{2}{3}x_0+\frac{2}{3}dx+4-\frac{2}{3}x_0-4=
                 \frac{2}{3}dx,\quad \forall dx.
\end{align*}
Se disegni il grafico della funzione \(y=\frac{2}{3}x\) puoi verificare che
l'incremento infinitesimo dei valori \(y\) corrisponde a \(frac{2}{3}\) 
dell'incremento infinitesimo dei valori \(x\). 
Il risultato è generale, cioè vale \(\forall x \in \IR\): il 
differenziale non dipende dal particolare \(x_0\) in cui lo si calcola.
\end{esempio}

\begin{esempio}
Proviamo con un'altra funzione lineare: \(f(x) = -5x-2\). 
Ci aspettiamo che anche in questo caso il differenziale sia indipendente 
da \(x_0\):
\begin{align*}
df(x)|_{x=x_0} &=f(x_0+dx)-f(x_0)=\\
               &=-5(x_0+dx)-2-\tonda{-5x_0-2}=
                 -5x_0-5dx-2+5x_0+2=
                 -5dx,\quad \forall dx.
\end{align*}
Quindi \(df(x_0)=-5dx, \forall x_0 \in \IR\). Sottolineare che il risultato è 
indifferente dal valore di \(x_0\) è come dire che vale per qualsiasi \(x\): 
\(df(x_0)=-5dx, \forall x \in \IR\).
\end{esempio}

\begin{teorema}
 Il differenziale di una funzione lineare \(f(x)=mx+q\), (con \(m\ne 0\)) è 
\(mdx\), 
 \(\forall x, \forall dx \in \IR\).
\end{teorema}

\noindent Ipotesi: \(f(x)=mx+q\) \tab Tesi: \(df(x)=mdx\)

\begin{proof}
\begin{align*}
df(x_0) &=f(x_0+dx)-f(x_0)=\\
               &=m(x_0+dx)+q-\tonda{mx_0+q}=
                 mx_0-mdx+q-mx_0-q=
                 mdx,\quad \forall dx.
\end{align*}
Poiché nel risultato non compare \(x_0\), anche in questo caso \(df(x)\) non 
dipende dal punto \(x_0\), cioè vale \(\forall x\).
\end{proof}

\subsection{Pendenza di una retta e infinitesimi }
Dalla geometria analitica abbiamo imparato che nell'equazione di una retta 
\(y=mx +q\) \(m\) rappresenta il coefficiente angolare, cioè l'inclinazione 
rispetto all'asse delle X del grafico della retta nel piano cartesiano.\\
Conoscendo le coordinate di due punti qualsiasi della retta, \(A\) e 
\(B\), il calcolo di \(m\) è facile:
\(m=\frac{y_B-y_A}{x_B-x_A}\).\\
\(m\) è quindi il rapporto fra gli incrementi lungo l'asse \(Y\) e 
lungo l'asse \(X\) che si ottengono muovendosi da \(A\) a \(B\) lungo la 
retta.

Dal momento che ci occupiamo di incrementi infinitesimi, riscriviamo con i 
differenziali la formula che corrisponde a \(m\) per una generica funzione 
lineare \(f(x)=mx +q\) a partire da un qualsiasi punto \(x_0\), come se 
volessimo muoverci lungo la retta nel passare da \(x_A=x_0\) a \(x_B=x_0+dx\). 
In pratica \(x_B \approx x_A\) perché \(x_B-x_A=dx\).
\[\frac{f(x_0+dx)-f(x_0)}{(x_0+dx)-x_0}=\frac{df(x_0)}{dx}.\]
Ne risulta è un rapporto fra infinitesimi, che esiste se \(dx\ne 0\). Si 
tratta di una formula importante.
\begin{definizione}
 Si chiama Rapporto Differenziale (RD) il rapporto  fra l'incremento 
infinitesimo di una funzione e l'incremento infinitesimo (non nullo) della sua 
variabile.
\end{definizione}
Quanto vale il RD per una funzione lineare, ponendo \(dx\ne 0\)?
\begin{align*}
 \text{funzione: }\quad f(x)&= mx+q\\
 \text{RD: }\quad \frac{df(x_0)}{dx}& =\frac{f(x_0+dx)-f(x_0)}{(x_0+dx)-x_0}=
 \frac{[m(x_0+dx)+q]-[mx_0+q]}{(x_0+dx)-x_0}=\\
 &=\frac{mx_0+mdx+q-mx_0-q}{dx}=\frac{m\cancel{dx}}{\cancel{dx}}=m.
\end{align*}
\begin{osservazione}
 Ragioniamo su quanto è stato fatto. Calcolando il coefficiente angolare della 
retta per punti infinitamente vicini siamo entrati nell'insieme \(\IR\),
abbiamo quindi operato sulla funzione estesa \(^\star f\) al posto di quella 
reale \(f\).\\
Il numero finito \(m\in \R\)  ha come corrispondente iperreale il numero finito 
\(^*m\) che non è esattamente identico, perché nell'insieme \(\IR\) i 
numeri finiti hanno la forma \(a+\epsilon\). Quindi anche \(m\) iperreale è in 
realtà \(m+\epsilon\): non è un solo numero, ma una monade di numeri 
tutti infinitamente vicini al reale \(m\). Per prendere la parte sostanziale di 
questo insieme infinito e tornare a operare nell'insieme dei reali, dove non 
esistono infiniti e infinitesimi, dobbiamo applicare la funzione \(\pst{}\).
\end{osservazione}

Conclusione: anche con gli iperreali si può calcolare l'inclinazione di una 
retta, basta calcolare il RD. Per ottenere il numero reale corrispondente al  
coefficiente angolare \(m\), basta calcolare \(\pst{RD}\).

\subsection{Derivata}
Il procedimento appena visto è di importanza fondamentale e si applica a tutte 
le funzioni, non solo quelle lineari. Per questo richiede una definizione 
precisa.
\begin{definizione}
 Si dice derivata \(f'(x_0)\) di una funzione \(f\) in un punto di ascissa 
\(x_0\) del suo dominio iperreale, la parte standard del RD di \(f\), calcolato 
in \(x_0\): 
\[
 f'(x_0)=\pst{\frac{df(x_0)}{dx}} 
\]
se tale parte standard esiste ed è sempre la stessa al variare 
dell'infinitesimo \(dx\), che deve essere diverso da 0.
\end{definizione}



\subsubsection{Funzione quadratica}
\label{subsubsec:diff01_diffquad}

La funzione quadratica più elementare dà come risultato il quadrato 
della variabile indipendente:
\[f: x \mapsto x^2 \quad \text{o} \quad 
  f(x)=x^2 \quad \text{o} \quad 
  y = x^2\]
\begin{teorema}
 Il differenziale della funzione quadratica \(f(x)=x^2\) ~ è ~ \(2xdx+(dx)^2\), 
~\(\forall x\in \IR\).
\end{teorema}

\noindent Ipotesi: \(f(x)=x^2\).\tab Tesi: \(df(x)=2xdx+(dx)^2\).

\begin{proof}
\[ df(x_0)= f(x_0+dx)-f(x_0)=(x_0+dx)^2-x_0^2=
            x_0^2+2x_0dx+(dx)^2-x_0^2=2x_0dx+(dx)^2\]
Questa volta nel risultato compare \(x_0\). Quindi il valore del 
differenziale della funzione cambia al cambiare del punto \(x_0\) che viene 
incrementato.
Anche in questo caso il differenziale è un infinitesimo, ma questa volta è 
dato dalla somma di due infinitesimi di diverso ordine.
\end{proof}

Prova a differenziare altre funzioni quadratiche, espresse da polinomi,
e verifica che i risultati sono analoghi a quello dimostrato: 
\begin{align*}
&f(x) = x^2 - 12  &df(x_0)=\\
&f(x) = 3x^2 +\dfrac{2}{3}  &df(x_0)=\\
&f(x) = \dfrac{1}{2}x^2 - x  &df(x_0)=\\
&f(x) = x^2 - \dfrac{2}{3}x +5 &df(x)|_{x=x_0}=\\
\end{align*}

\subsubsection{Funzioni potenza}
\label{subsubsec:diff01_diffpot}

Una funzione potenza è una funzione che dà come risultato la potenza 
della variabile indipendente:
\[f: x \mapsto x^n \quad \text{o} \quad 
  f(x)=x^n \quad \text{o} \quad 
  y = x^n\]
Ricaviamo per gradi il differenziale della funzione potenza è \(f(x)=x^n\), 
con un procedimento per induzione.\\
Iniziamo dai casi già noti \(f(x)=x\) e \(f(x)=x^2\) e esaminiamo i successivi
aumentando progressivamente l'esponente. Questa volta non inseriamo 
l'indicazione relativa a \(x_0\) perché ci siamo accorti che i risultati 
cambiano 
al cambiare di \(x_0\), il che equivale a dire che dipendono da \(x\).
\begin{align*}
  d(x) &=x+dx-x =dx\\
  d(x^2) &=(x+dx)^2-x^2 = x^2 +2xdx +(dx)^2 -x^2 = 2xdx +(dx)^2\\
  d(x^3) &=(x+dx)^3-x^3 =[x^3+3x^2dx+3x(dx)^2+(dx)^3]-x^3=
                      3x^2dx+3x(dx)^2+(dx)^3\\
  d(x^4) &=(x+dx)^4-x^4 = [x^4+4x^3dx+6x^2(dx)^2+4x(dx)^3+(dx)^4]-x^4=\\
                      &=4x^3dx+6x^2(dx)^2+4x(dx)^3+(dx)^4      
\end{align*}
Possiamo osservare che il termine non infinitesimo si annulla sempre. Quindi,
qualunque sia il valore di \(x\), l'incremento della funzione è infinitesimo. 
Queste funzioni sono quindi continue in tutto \(\IR\) perché a variazioni 
infinitesime della variabile \(x\) corrispondono sempre variazioni
infinitesime della funzione.

Possiamo anche osservare che il differenziale è sempre più complesso, ma 
se consideriamo la parte principale dell'infinitesimo, cioè se trascuriamo 
gli infinitesimi di ordine superiore, il risultato si semplifica e diventa 
facilmente memorizzabile.

% \newpage %----------------------------------------
Quindi se invece del valore esatto ci accontentiamo della parte principale, 
abbiamo:
\nopagebreak
\begin{align*}
  d(x) &=x+dx-x =dx\\
  d(x^2) &=2xdx +(dx)^2 \sim 2xdx\\
  d(x^3) &=3x^2dx+3x(dx)^2+(dx)^3 \sim 3x^2dx\\
  d(x^4) &=4x^3dx+6x^2(dx)^2+4x(dx)^3+(dx)^4 \sim 4x^3dx\\
  d(x^5) &=5x^4dx+10x^3(dx)^2+10x^2(dx)^3+5x(dx)^4+(dx)^5 \sim 5x^4dx\\
  d(x^6) &=6x^5dx+\dots+(dx)^6 \sim 6x^5dx\\
  d(x^7) &=7x^6(dx)+\dots+(dx)^7 \sim 7x^6dx\\
  \dots &= \dots\\
  d(x^{10}) &=10x^9(dx)+\dots+(dx)^{10} \sim 10x^9dx\\
  \dots &= \dots\\
  d(x^n) &=nx^{n-1}(dx)+\dots+(dx)^{n} \sim nx^{n-1}dx.\\    
\end{align*}
Ora che il meccanismo è chiaro, possiamo ritenere dimostrato il teorema 
seguente.

\begin{teorema}
 Il differenziale della funzione potenza, con esponente naturale, è 
 \[d(x^n) \sim nx^{n-1}dx.\]
\end{teorema}

\begin{osservazione}
Il teorema precedente si riferisce alle sole potenze con esponente
naturale. Per altre vie si dimostrerà un teorema analogo, relativo
a \emph{qualsiasi esponente reale}.
Ne hai un esempio nei casi che seguono.
\end{osservazione}

\subsubsection{Funzione radice quadrata}
\label{subsubsec:diff01_diffradq}

\begin{teorema}
 Il differenziale della funzione radice quadrata \(f(x)=\sqrt{x}\) è
 \(\sim\frac{dx}{2\sqrt{x}}\), \(\forall x\in \IR\), \(x\neq 0\).
\end{teorema}

\noindent Ipotesi: \(f(x)=\sqrt{x}\).\tab Tesi: 
\(df(x)\sim\dfrac{dx}{2\sqrt{x}}\).

\begin{proof}
\begin{align*}
 df(x) &= f(x+dx)-f(x)=\sqrt{x+dx}-\sqrt{x}=
          \tonda{\sqrt{x+dx}-\sqrt{x}}\times
          \frac{\sqrt{x+dx}+ \sqrt{x}}{\sqrt{x+dx}+\sqrt{x}}=\\
       &=\frac{x+dx-x}{\sqrt{x+dx}+\sqrt{x}}=
         \frac{dx}{\sqrt{x+dx}+\sqrt{x}}\sim\frac{dx}{2\sqrt{x}}.
\end{align*}
 Poiché nel risultato compare \(x\), anche questa volta il risultato
 dipende dal valore a partire dal quale si vuole differenziare. 

Prendendo la parte principale dell'infinitesimo possiamo confrontare il 
risultato con quello ottenuto applicando la regola della funzione potenza:
\[d(\sqrt{x}) = d\tonda{x^{\frac{1}{2}}}
  \sim\frac{1}{2}x^{\tonda{\frac{1}{2}-1}}dx 
  =\frac{1}{2}x^{-\frac{1}{2}}dx
  =\frac{dx}{2x^{\frac{1}{2}}}
  =\frac{dx}{2\sqrt{x}}.\]

\end{proof}

\subsubsection{Funzione reciproca}
\label{subsubsec:diff01_diffrecip}
\begin{teorema}
 Il differenziale della funzione reciproca \(f(x)=\frac{1}{x}\) (con \(x \ne 
0\))
 è  \[d\tonda{\frac{1}{x}}\sim\frac{dx}{x^2}.\]
\end{teorema}

\noindent Ipotesi: \(f(x)=\dfrac{1}{x}\)\tab 
Tesi: \(df(x)\sim-\dfrac{dx}{x^2}.\)

\begin{proof}
\[
 df(x)= f(x+dx)-f(x)=\frac{1}{(x+dx)}-\frac{1}{x}=
 \frac{x-x-dx}{x(x+dx)}=\frac{-dx}{x^2+xdx}\sim-\frac{dx}{x^2}.
\]
Anche questa volta il valore del differenziale dipende dal punto in cui
lo si calcola. 

Riscriviamo la funzione reciproca come una funzione potenza:
\[d\tonda{\frac{1}{x}} = d\tonda{x^{-1}}
  \sim -x^{\tonda{-1-1}}dx 
  =-x^{-2}dx
  =-\frac{dx}{x^{2}}.\]

\end{proof}

Da questi esempi traiamo il prossimo enunciato, che ci riserviamo di 
dimostrare in seguito:

\begin{teorema}
 Il differenziale della funzione potenza, con esponenete reale qualsiasi, è 
 \[d(x^a) \sim ax^{a-1}dx \quad \forall a \in \R.\]
\end{teorema}

\subsubsection{Differenziali problematici}
\label{subsubsec:diff01_diffproblemi}

L'ultimo esempio ci porta un dubbio: dato che
\(x\) risulta al denominatore, abbiamo un problema. Che succede se \(x=0\)?
\begin{esempio}
 Calcola \(df(x)|_{x=0}\), con \(f(x)=\frac{1}{x}\).\\
 \(d\tonda{\frac{1}{x}}|_{x=x_0}=\frac{1}{0+dx}-\frac{1}{0}=\) ?\\
 La funzione non è differenziabile per \(x=0\) perché la frazione \(1/0\) non 
esiste.
\end{esempio}

\begin{esempio}
Differenzia la funzione \(f(x)=\frac{1}{x^2-1}\) per \(x=1\) e \(x=-1\).\\
\(d\tonda{\frac{1}{x^2-1}}\big|_{x=1}=\frac{1}{(1+dx)^2-1}-\frac{1}{1^2-1}=
\frac{1}{2dx+(dx)^2}-\frac{1}{0}=\)?\\
\(d\tonda{\frac{1}{x^2-1}}\big|_{x=-1}=\frac{1}{(-1+dx)^2-1}-\frac{1}{(-1)^2-1}=
\frac{1}{-2dx+(dx)^2}-\frac{1}{0}=\)?\\
Questa volta i punti critici sono due. Poiché la funzione non è calcolabile
per \(x=1\) e \(x=-1\), non è calcolabile nemmeno il suo differenziale.
\end{esempio}

Nei due esempi precedenti si tenta di calcolare il differenziale in un 
punto in cui la funzione non è definita. Ci vuole poca immaginazione per 
capire che se non è definita la funzione non può essere definito neppure il 
differenziale.

\begin{esempio}
Ma che dire del differenziale della radice quadrata in zero? Lì la funzione 
è definita: \(f(0)=\sqrt{0}=0\). 
Mentre non è definita l'espressione che calcola il suo differenziale: 
\(df(0) \sim \dfrac{dx}{2 \sqrt{0}}.\)

Però se applichiamo la definizione, il differenziale della radice in zero non 
sembra presentare difficoltà:
\[df(0) = f(0+dx)-f(0)=\sqrt{0+dx}-\sqrt{0}=\sqrt{dx}.\]
(Per poter effettuare i calcoli \(dx\) deve essere positivo: assumiamo perciò:
\(dx>0\).)

Il risultato, \(\sqrt{dx}\), è un infinitesimo ''molto`` più 
grande di \(dx\). 
Se \(\sqrt{dx} = \delta\) allora \(dx = \delta \cdot \delta\) cioè \(dx\) è 
un infinitesimo ''più potente`` rispetto a \(\sqrt{dx}\). Se con uno strumento 
ottico non standard  visualizzo \(dx\), non posso vedere contemporaneamente
anche \(\sqrt{dx}\), perché questo è fuori dal campo visivo, è all'infinito.
\end{esempio}


Nel piano cartesiano tracciamo il grafico delle funzioni degli ultimi 
tre esempi: \(y=\sqrt{x}\), \(y=\frac{1}{x}\) e \(y=\frac{1}{x^2-1}\).
\begin{figure}[h]
\begin{inaccessibleblock}[Grafici di funzioni diverse.]
 \begin{center}
 \begin{minipage}[]{.23 \textwidth}
%   \vspace*{4mm} 
  \radice
  \vspace*{-5mm} 
  \caption{\(y=\sqrt{x}\)}
%   \vspace*{1mm} 
 \end{minipage} 
 \begin{minipage}[]{.37 \textwidth}
  \iperbole
  \caption{\(y=\frac{1}{x}\)}
 \end{minipage} 
 \begin{minipage}[]{.37 \textwidth}
  \iperbolequad
  \caption{\(y=\frac{1}{x^2-1}\)}
 \end{minipage}
 \end{center}
\end{inaccessibleblock}
\label{fig:diff01_grafici}
\end{figure}

Guardando i grafici si ha evidenza di quanto appena discusso.
Nel secondo e nel terzo grafico: se si volesse 
differenziare la funzione ''a cavallo``di un punto in cui essa non è 
definita, le differenze sull'asse \(Y\) risulterebbero infinite.\\
Relativamente al primo grafico: si può tentare di differenziare la funzione 
solo se \(x \ge 0\). Nell'infinitamente vicino all'origine il grafico della 
funzione è pressoché verticale e qualsiasi incremento infinitesimo lungo 
l'asse \(X\) provoca un incremento ''non  così infinitesimo`` lungo l'asse 
\(Y\).

% \begin{figure}[h]
 \begin{center}
 \begin{minipage}[]{.38 \textwidth}
\begin{inaccessibleblock}
  [Discontinuità a salto]
  \salto
\end{inaccessibleblock}
%   \caption{Discontinuità a salto.}
 \end{minipage} 
 \hfill
 \begin{minipage}[]{.58 \textwidth}
Consideriamo un tipo diverso di problema.\\
\[f(x)=\begin{cases} 
x-1, & \mbox{se }x<2 \\ 
x+1, & \mbox{se }x\ge 2
\end{cases}
\]
\(f(x)\) ha due rami e il grafico compie un salto per \(x=2\), dove 
\(f(x)=3\).
Se \(dx>0\), \(df(x)|_2\) è calcolabile e risulta \(=dx\), mentre se 
\(dx<0\)
le differenze non possono essere infinitesime perché la funzione ha un 
salto nel punto considerato. 
In casi come questo \(f(x)\) è differenziabile solo a destra 
del punto critico \(x=2\).
 \end{minipage}
 \end{center}
\label{fig:diff01_salto}
% \end{figure}

\subsection{Combinare differenziali}
\label{subsec:diff01_combdiff}
Nella sezione \ref{subsubsec:diff01_flineare} e in altre ci siamo avvalsi 
di proprietà così naturali che non è stato necessario sottolinearle. 
Ma è meglio non lasciarcele sfuggire.

\subsubsection{Differenziale del prodotto per una costante}
\label{}
\begin{teorema}
 Se una funzione è moltiplicata per una costante, anche il suo 
 differenziale risulta moltiplicato per la stessa costante.
\end{teorema}
\noindent Ipotesi: \(f(x)=a\cdot g(x)\).\tab Tesi: \(df(x)=a\cdot dg(x)\).

\begin{proof}
\[df(x)=d\quadra{a\cdot g(x)}= a\cdot g(x+dx)-a\cdot 
g(x)=a\cdot\quadra{g(x+dx)-g(x)}
 =a\cdot dg(x)\]
\end{proof}

\subsubsection{Differenziale di una somma di funzioni}
\label{}

\begin{teorema}
 Se una funzione è la somma (la differenza) di due funzioni, anche il suo 
 differenziale sarà la somma (la differenza) dei due differenziali.
\end{teorema}
\noindent Ipotesi: \(f(x)=f_1(x)\pm f_2(x)\).\tab Tesi: \(df(x)=df_1(x)\pm 
df_2(x)\).

\begin{proof}
\begin{align*}
 df(x)&=d[f_1(x)\pm f_2(x)]=[f_1(x+dx)\pm f_2(x+dx)]-[f_1(x)\pm f_2(x)]=\\
      &= [f_1(x+dx)-f_1(x)]\pm [f_2(x+dx)-f_2(x)]= df_1(x)\pm df_2(x)
\end{align*}
\end{proof}


\begin{esempio}
 Un generico polinomio di secondo grado \(f(x)=ax^2+bx+c\) è composto da 
tre termini. 
Quindi abbiamo che: \(f(x)=f_1+f_2+f_3\) \quad e \quad
 \(df(x)~=~df_1+df_2+df_3\).
\[d(ax^2)\sim 2axdx \qquad d(bx) =bdx \qquad d(c) =0 \quad \sRarrow
  df(x) \sim 2axdx+bdx\]
Il grafico della funzione è una parabola generica
e il differenziale ci dice che l'incremento infinitesimo 
\(dx\) provoca un incremento (eventualmente di segno opposto) infinitesimo
variabile sull'asse \(Y\), che dipende dal valore \(x\) a partire dal quale si 
calcola \(dx\).
\end{esempio}

Completiamo il quadro delle regole di calcolo con l'esame dei differenziali
del prodotto e del rapporto di funzioni. Verrebbe da pensare: ''siccome il 
differenziale di una somma è la somma dei differenziali e lo stesso avviene
per la differenza, succederà una cosa simile anche per il prodotto e per il 
rapporto``. 
Per (s)fortuna le cose a volte sono un po' meno banali.

\subsubsection{Differenziale del prodotto di due funzioni}
\label{}
Questa volta, al posto della immarcescibile dimostrazione algebrica, 
ricorriamo alla geometria. 
Immaginiamo che le due funzioni, calcolate in un generico punto \(x\),
esprimano la base e l'altezza di un rettangolo:
\(b(x)=b\) sarà la base  e \(h(x)=h\) sarà l'altezza. 
L'area ovviamente si ottiene da \(b(x)\cdot h(x)=\mathit{A}(x)\). 
Differenziare il prodotto \(d\quadra{\mathit{A}(x)}\) vuol dire calcolare di
quanto aumenta l'area del rettangolo, se i lati subiscono un incremento 
infinitesimo. 

\begin{osservazione}
Gli incrementi della base e dell'altezza possono essere 
diversi, perché \(b(x)\) e \(h(x)\) sono funzioni diverse, le quali possono
reagire in modo diverso all'incremento \(dx\).
\end{osservazione}

\begin{teorema}
 Se una funzione è il prodotto di due funzioni, il suo  differenziale 
 sarà dato da una somma fra tre prodotti: il differenziale della
 prima funzione per la seconda più la prima funzione per il differenziale 
 della seconda più il prodotto dei due differenziali.
\end{teorema}
\noindent Ipotesi: \(\mathit{A}(x)=b(x)\cdot h(x)\).\qquad 
Tesi: \(d\mathit{A}(x)=db(x)\cdot h(x)+b(x)\cdot dh(x)+ db(x)\cdot dh(x)\).

\begin{figure}[h]
\begin{inaccessibleblock}
  [Rettangolo con uno gnomone finito e rettangolo con gnomone infinitesimo.]
 \begin{center}
 \begin{minipage}[]{.38 \textwidth}
  \vspace{23mm} 
  \incrementaleprodotto
 \end{minipage} 
 \hfill
 \begin{minipage}[]{.58 \textwidth}
  \differenzialeprodotto
 \end{minipage}
 \end{center}
\end{inaccessibleblock}
\caption{Incrementi finito e infinitesimo dell'area di un rettangolo} 
\label{fig:Incre_prodotto}
\end{figure}
\vspace{-.5em}
\begin{osservazione}
 Si chiama ''gnomone`` la figura, a forma di L rovesciata, che rappresenta 
la crescita dell'area di un rettangolo.
\end{osservazione}

\begin{proof}
L'incremento infinitesimo di area è la zona colorata del disegno, lo 
\emph{gnomone}. È formato da tre parti:
\begin{itemize} [nosep]
 \item un rettangolo sottile, verticale e sulla destra, di base
 infinitesima \(db(x)\) e altezza \(h(x)\);
 \item un rettangolo orizzontale, in alto, di base \(b(x)\) e
 altezza infinitesima \(dh(x)\);
 \item un rettangolino in alto a destra, di area \(db(x) \times dh(x)\).
\end{itemize}
La descrizione geometrica rappresenta bene la tesi e per i nostri scopi è
una prova sufficiente. 
\end{proof}
Dato che l'ultimo termine è un infinitesimo di ordine
superiore, il risultato può essere approssimato alla sua parte principale, 
senza gravi danni: \(d\mathit{A}(x)\sim db(x)\cdot h(x)+b(x)\cdot dh(x)\).

\subsubsection{Differenziale del rapporto fra due funzioni}
\label{}
\begin{teorema}
 Se una funzione è data dal rapporto fra due funzioni, con il denominatore
 non nullo, il suo  differenziale si ottiene calcolando
 la differenza fra due prodotti (il differenziale del numeratore per il 
 denominatore meno il numeratore per il differenziale del denominatore)
 e dividendo il risultato per il quadrato del denominatore.
\end{teorema}
\noindent Ipotesi: \(h(x)=\dfrac{\mathit{A}(x)}{b(x)}\), 
con \(b(x)\neq 0\) \tab 
Tesi: 
\(dh(x) \sim \dfrac{d\mathit{A}(x) \cdot b(x)-\mathit{A}(x) \cdot db(x)}
             {\tonda{b(x)}^2}\)

\begin{proof}
Ricorriamo alla geometria anche in questo caso.

% Usare {figure} è una disgrazia!

 \begin{minipage}[]{.38 \textwidth}
Essendo \(\mathit{A}(x)=b(x) \times h(x)\) allora: 
\(h(x)=\frac{A(x)}{b(x)}\). 
Ovviamente è necessario che \(b(x)\neq 0\).\\
Guardando il disegno, possiamo osservare che \(dh(x)\) è l'incremento 
infinitesimo dell'altezza, si tratta dell'altezza della fascia superiore 
colorata.
Questa altezza si può calcolare dividendo il rettangolo 
superiore dello gnomone per la base del rettangolo.
Il rettangolo superiore dello gnomone è uguale a tutto lo gnomone 
infinitesimo, \(d\mathit{A}\),
 \end{minipage} 
 \hfill
 \begin{minipage}[]{.58 \textwidth}

 \begin{center}
 \begin{inaccessibleblock}
  [Altezza rettangolo con gnomone infinitesimo.]
  \differenzialerapporto
 \end{inaccessibleblock}
 \end{center}
\begin{comment}
 \end{minipage}
meno il rettangolo destro infinitesimo, di 
area \(h\cdot db\) e meno il rettangolino, sempre infinitesimo, che si trova 
in alto a destra.
Dunque:
\begin{align*}
 d\quadra{\frac{\mathit{A}(x)}{b(x)}}&=dh(x)=\\
 &=\frac{\quadra{d\mathit{A}(x) - h \cdot db(x) - db(x) \cdot dh(x)}}
        {b(x)}=\\
 &=\frac{\quadra{d\mathit{A}(x) - \dfrac{\mathit{A(x)}}{b(x)} \cdot db(x) - 
          db(x) \cdot dh(x)}}{b(x)}=\\
 &=\frac{\dfrac{d\mathit{A}(x) \cdot b(x) - \mathit{A(x)} \cdot db(x) -
               b(x) \cdot db(x) \cdot dh(x)}
              {b(x)}}
        {b(x)}=\\
 &=\frac{d\mathit{A}(x) \cdot b(x) - \mathit{A(x)} \cdot db(x) -
              b(x) \cdot db(x) \cdot dh(x)}
        {\tonda{b(x)}^2} \sim \\
 & \sim \frac{d\mathit{A}(x) \cdot b(x) - \mathit{A(x)} \cdot db(x)}
             {\tonda{b(x)}^2}
\end{align*}
\end{proof}


\subsubsection{Sintesi della sezione}
\label{subsubsec:diff01_diffsint}
Abbiamo calcolato alcuni differenziali elementari: 
\begin{enumerate} [noitemsep]
 \item \(d(k) = 0\) \tab 
 differenziale di una costante;
 \item \(d(x) = dx\) \tab 
 differenziale della funzione identica;
 \item \(d(x^\alpha) \sim \alpha x^{\alpha-1}dx\) \tab 
 differenziale di una potenza.
\end{enumerate}
 
% Avremo modo di vedere i differenziali delle funzioni trascendenti nella 
% prossima sezione.

% I risultati che abbiamo visto valgono sotto le ovvie ipotesi
% che si parli di funzioni continue e che i differenziali siano calcolabili
% per tutti i possibili \(x\) del dominio di tali funzioni. Unificando i 
% simboli e restando all'essenziale, abbiamo:

E il differenziale di alcune operazioni tra funzioni:
\begin{enumerate} [noitemsep]
 \item \(d(a\cdot f)=adf\) \tab 
 differenziale del prodotto per una costante;
 \item \(d\tonda{f\pm g}=df\pm dg\) \tab 
 differenziale di una somma o differenza;
 \item \(d(f\cdot g)\sim df\cdot g+f\cdot dg\) \tab 
 differenziale del prodotto;
 \item \(d\tonda{\frac{f}{g}}\sim\frac{df\cdot g-f\cdot dg}{g^2}\)\tab 
differenziale  del rapporto (\(g \ne 0\)).
\end{enumerate}

Alcune delle uguaglianze precedenti sono ridondanti: il prodotto tra una 
costante e una funzione è un caso particolare del prodotto tra funzioni, il 
differenziale di una potenza si può ottenere applicando, più volte, il 
differenziale del prodotto di funzioni.
% dove  \(k\), \(a\), \(\alpha\) rappresentano delle costanti, mentre f e g 
sono 
% funzioni continue. 

Esaminiamo ora alcuni problemi facilmente risolvibili con l'aiuto dei 
differenziali.

\subsection{Problemi con i differenziali}
\label{subsec:diff01__problemi}

\begin{esempio}
 % Triangolo equilatero: 2p=f(h) 
Un triangolo equilatero ha l'altezza di \(8\) cm. 
Di quanto aumenta il suo perimetro, man mano che aumenta l'altezza? 
L'aumento è legato alla misura iniziale di \(h\)?\\
Il perimetro è \(2p=3l\) e con il Teorema di Pitagora si ha: 
\(h=\sqrt{l^2-\tonda{\frac{l}{2}}^2}=\frac{\sqrt{3}}{2}l\). 
Quindi \(l=\frac{2}{\sqrt{3}}h\) e \(2p=2\sqrt{3}h\). 
Incrementiamo l'altezza a partire da \(h_0=8\) e ricaviamo il perimetro 
corrispondente.\\
\(d(2p)|_{h_0=8}=
d\tonda{2\sqrt{3}h}|_{h_0=8}=
2 \sqrt{3} \cdot (8+dh)-2\sqrt{3} \cdot 8 = 2 \sqrt{3} \cdot dh\).\\
Per ogni incremento infinitesimo dell'altezza, il perimetro aumenta di 
\(2 \sqrt{3}\) volte quell'incremento.
Si tratta di rapporto costante, perché non dipende dalla misura iniziale
dell'altezza. Infatti, se si ripete il calcolo scrivendo il simbolo \(h_0\) al 
posto della sua misura \(8\), \(h_0\) non compare nel risultato.\\
La soluzione può essere ricavata in modo più diretto, applicando le regole
4 e 2 della sintesi a pag.\pageref{subsubsec:diff01_diffsint}.
\end{esempio}

\begin{esempio}
 % Triangolo equilatero: l=f(A)
Di quanto aumenta il lato di un triangolo equilatero,
man mano che aumenta la sua area? L'aumento è legato al valore iniziale del 
lato?\\
Dalla formula dell'area \(\mathit{A}=\frac{bh}{2}\) e dall'esempio precedente
(\(h=\frac{\sqrt{3}}{2}l\)), ricaviamo: \(\mathit{A}=\frac{\sqrt{3}}{4}l^2\).\\
Differenziando, con l'aiuto delle regole 4 e 3 della sintesi a pag.
\pageref{subsubsec:diff01_diffsint}, abbiamo:\\
\(d\tonda{\mathit{A}}=d\tonda{\frac{\sqrt{3}}{4}l^2}=
 \frac{\sqrt{3}}{4}d\tonda{l^2}=
 \frac{\sqrt{3}}{4}\tonda{2l \cdot dl+(dl)^2}=
 \frac{\sqrt{3}}{4}\tonda{2l+dl}dl\).\\
Questa volta la relazione con l'incremento del lato non è elementare: per 
ogni incremento infinitesimo del lato si ha un incremento di area pari a 
\(frac{\sqrt{3}}{4}\tonda{2l+dl}\), che dipende dalla misura iniziale del 
lato e dallo stesso incremento. Per gestire il risultato, occorre approssimare
questo numero all'indistinguibile più vicino:
\(d\tonda{\mathit{A}}=\frac{\sqrt{3}}{4}\tonda{2l+dl}dl\sim
\frac{\sqrt{3}}{2}ldl\).

Da qui, applicando la formula inversa, si ottengono le risposte:
\(dl\sim\frac{2}{\sqrt{3}}\frac{d\tonda{\mathit{A}}}{l}\).
\begin{osservazione}
 Una via più diretta per giungere alla soluzione potrebbe essere: \\
 \(\mathit{A}=\frac{\sqrt{3}}{4}l^2\srarrow 
l=\sqrt{\frac{4}{\sqrt{3}}\mathit{A}}=
 \frac{2}{\sqrt[4]{3}}\sqrt{\mathit{A}}\srarrow dl=
 d\tonda{\frac{2}{\sqrt[4]{3}}\sqrt{\mathit{A}}}=
 \frac{2}{\sqrt[4]{3}}d\tonda{\sqrt{\mathit{A}}}\).\\
 A questo punto dobbiamo fermare il calcolo, perché 
 sappiamo calcolare \(d\tonda{\sqrt{x}}\), ma 
 non sappiamo ancora come calcolare \(d\tonda{\sqrt{f(x)}}\). Per farlo, 
 occorre approfondire le conoscenze precedenti.
\end{osservazione}
\end{esempio}

\section{Introduzione alla derivata}
\label{sec:diff01_derivata}
La derivata è un ente matematico conosciuto dalla metà del 1700, che 
da allora si applica utilmente allo studio di fenomeni naturali di ogni 
tipo.\\
Studieremo l'argomento puntando lo sguardo sulle funzioni e sui loro
grafici nel piano cartesiano. Iniziamo dai grafici più semplici.

\subsection{Pendenza di una retta}
\label{subsec:diff01_pendretta}

\begin{figure}[h]
\begin{inaccessibleblock}[pendenza di una retta.]
 \begin{center}
 \begin{minipage}[]{.31 \textwidth}
%  \vspace*{4mm} 
  \rettadueterzi
  \caption{\(y=\frac{3}{2}x-1\)}
 \end{minipage} 
 \begin{minipage}[]{.31 \textwidth}
  \rettamenounquarto
  \caption{\(y=-\frac{1}{4}x+\frac{1}{2}\)}
 \end{minipage} 
 \begin{minipage}[]{.31 \textwidth}
  \retteorvert
  \caption{\(y=-\frac{3}{2}\) e \(x=-2,8\)}
 \end{minipage}
 \end{center}
\end{inaccessibleblock}
\label{fig:diff01_ret}
\end{figure}
Sappiamo già calcolare la pendenza di una retta dalla semplice osservazione 
del suo grafico: si fissano sulla retta due punti \(A(x_A; y_A)\) e \(B(x_B; 
y_B)\)
e si calcola il rapporto \(m=\frac{y_B-y_A}{x_B-x_A}\).\\
È come se si volesse misurare la distanza verticale
fra i due punti usando la loro distanza orizzontale come unità di misura. 
Nel caso della retta \(r\), \(m=\frac{3}{2}\) e si potrebbe dire: ''un punto 
che si muove sulla retta, se si sposta di due quadretti in orizzontale
ne guadagna (o perde) tre in verticale``.\\
Un punto che scorre sulla retta orizzontale, non subisce
alcuna variazione lungo l'asse \(y\) e per questo \(m=0\); al contrario per la
retta verticale le variazioni sono solo verticali e la pendenza è
infinita.

Sintetizziamo la formula come rapporto fra differenze:
\(m=\frac{y_B-y_A}{x_B-x_A}=\frac{\Delta y}{\Delta x}\). Il simbolo \(m\) 
ci riporta all'equazione di una retta generica in forma esplicita
\(y=mx+q\), dove \(m\) rappresenta appunto il coefficiente angolare, cioè 
l'inclinazione o pendenza.
\nopagebreak
\begin{osservazione}
Secondo l'uso della sezione, le indicazioni con la lettera
maiuscola \(\Delta\) (\(\Delta x\), \(\Delta y\)) si riferiscono a 
\emph{numeri standard}.
\end{osservazione}

\subsubsection{Rapporto incrementale}
\label{subsubsec:diff01_rappincr}
C'è un fatto importante: per calcolare la pendenza di una retta, 
la scelta dei due punti è indifferente. Possono essere molto vicini o molto 
lontani, scambiati l'uno con l'altro o presso l'origine, oppure no:
\(m=\frac{\Delta y}{\Delta x}\) è sempre lo stesso, come è giusto che sia per 
una retta.
Da \(x_B-x_A=\Delta x\) ricaviamo banalmente \(x_B=x_A+\Delta x\), cioè nel 
piano cartesiano
\(B\) si colloca a destra (se \(\Delta x\ge 0\)) di \(A\) di una quantità 
finita,
grande o piccola che sia.\\
\(\Delta x\), \(\Delta y\) sono anche chiamati \emph{incrementi} e quindi...
\begin{definizione}
  Si dice \emph{Rapporto Incrementale} (R.I.) il rapporto degli
  incrementi, cioè la quantità R.I.~=~\(frac{\Delta y}{\Delta x}\).
\end{definizione}
Si tratta di una quantità finita, calcolabile se \(\Delta x \ne 0\).\\
Il Rapporto Incrementale, calcolato su una retta fornisce la sua pendenza 
ed è un valore costante, come abbiamo visto.\\
Ma il calcolo si può applicare a qualsiasi funzione, anche a quelle che nel 
piano cartesiano sono rappresentate da curve. Allora però le cose cambiano.

\begin{esempio}
I prossimi grafici appartengono alla stessa funzione.

\begin{figure}[h]
\begin{inaccessibleblock}
  [Secanti a una curva]
% \begin{center}
 \begin{minipage}[]{.45\textwidth}
 \curvacubica
 \end{minipage} 
 \hfill
 \begin{minipage}[]{.55\textwidth}
  \secanticubica
 \end{minipage}
% \end{center}
\end{inaccessibleblock}
\caption{Rapporti incrementali in una curva e secanti.} 
\label{}
\end{figure}

Scegliamo alcuni punti sulla curva e mettiamo in evidenza gli intervalli che
consentono il calcolo del rapporto incrementale, in un caso, e la pendenza 
delle secanti nell'altro.\\
Rapporti Incrementali:
\begin{align*}
  \frac{\Delta y}{\Delta x}\bigg\lvert_{AB}= &\frac{y_B-y_A}{x_B-x_A}=
    \frac{2-5}{-2-(-3.5)}=\frac{-3}{1.5}=-2 &
  \frac{\Delta y}{\Delta x}\bigg\rvert_{BC}=\frac{y_C-y_B}{x_C-x_B}=
  \frac{3-2}{0-(-2)}=\frac{1}{2}\\
  \frac{\Delta y}{\Delta x}\bigg\lvert_{CD}= &\frac{y_D-y_C}{x_D-x_C}=
  \frac{3.8-3)}{2.4-0}=\frac{0.8}{2.4}=\frac{1}{3} &
  \frac{\Delta y}{\Delta x}\bigg\rvert_{DE}=\frac{y_E-y_D}{x_E-x_D}=
  \frac{1-3.8}{3.5-2.3)}=\frac{-2.8}{1.2}=-\frac{7}{3}
\end{align*}
\nopagebreak
Pendenze: \qquad
\(m_{AB}=-2 \qquad m_{BC}=\frac{1}{2}\qquad m_{CD}=\frac{1}{3}\qquad  
m_{DE}=-\frac{7}{3}\)
\end{esempio}

I calcoli confermano che se il grafico non è una retta, il Rapporto 
Incrementale, calcolato fra varie coppie di punti, ha valori diversi. 
Il R.I. cambia a seconda della coppia di punti fissati sulla curva.\\
Se si traccia la retta che unisce la coppia di punti, ne risulta una secante
alla curva.\\
In conclusione, si hanno le seguenti proprietà:

\begin{enumerate}[nosep]
\item Il R.I. è un numero finito e esiste solo se \(\Delta x\ne 0\).
\item Il R.I fra le coppie di valori di una funzione è 
 a sua volta una funzione, che dipende dalla coppia scelta. 
 \item La funzione è costante se applicata al grafico di una retta. In 
questo 
 caso il R.I calcola  la sua pendenza.
\item  In generale, R.I. calcola la pendenza della retta secante che unisce 
due punti del grafico.
\end{enumerate}


\subsubsection{Rapporto differenziale}
\label{subsubsec:diff01_rappdiff}

\begin{esempio}
Fissiamo su una curva due punti: uno fisso (\(A\)) e l'altro mobile \(P\), cioè 
in grado di spostarsi lungo la curva dalla posizione più lontana \(P_1\), 
alla più prossima ad \(A\), cioè oltre \(P_7\), fin quasi a sovrapporsi con 
\(A\).

% \begin{figure}[h]
\begin{inaccessibleblock}[Verso la tangente a una curva.]
 \begin{center}
\secanticurva
 \end{center}
\end{inaccessibleblock}
% \caption{Dalle secanti alla tangente.} \label{fig:diff01_sectang}
% \end{figure}
\end{esempio}

Tracciamo le secanti che uniscono \(A\) con le varie posizioni di \(P_n\). 
Man mano che \(P\) si avvicina ad \(A\), la secante che li unisce tende 
ad allinearsi alla tangente ideale.\\
Quando \(P\) è così vicino ad \(A\) che la loro distanza è 
\(\overline{AP}<\frac{1}{n}\),
\(\forall n\), siamo nel campo degli infinitesimi: cambia la natura del 
Rapporto Incrementale che avevamo imparato a calcolare. Il R.I. si 
trasforma da un rapporto fra quantità finite a un rapporto fra infinitesimi,
quindi non possiamo essere certi su quale sia il tipo del risultato che 
fornisce.\\
Se escludiamo il caso \(dx=0\) (\(P_n\) coinciderebbe con \(A\)) e se il 
rapporto
dà un risultato finito, otterremo la pendenza della secante fra i due 
punti infinitamente vicini \(A\punto{x_A}{y_A}\) e 
\(P_n\punto{x_A+dx}{f(x_A+dx)}\),
quindi di una retta infinitamente vicina alla tangente, cioè distinguibile da 
essa solo grazie ad un microscopio a ingrandimento infinito.

\begin{definizione}
 Si dice Rapporto Differenziale della funzione \(f(x)\), relativo a \(x_0\),
 il rapporto \(frac{df(x)}{dx}\big|_{x=x_0}\) fra il differenziale della 
funzione  e quello della variabile, calcolati nel punto \(x_0\). \\
 \(frac{df(x)}{dx}\big|_{x=x_0}=\frac{f(x_0+dx)-f(x_0)}{dx}\), con \(dx\ne 0\).
\end{definizione}

\begin{figure}[h]
\begin{inaccessibleblock}[Secante per P approx A.]
 \begin{center}
\secRD
 \end{center}
\end{inaccessibleblock}
\caption{Secante per due punti infinitamente vicini.} 
\label{fig:diff01_tangente}
\end{figure}

\begin{esempio}
  \label{esempio:diff01_mdiff}
  La curva della Fig.12 rappresenta la parabola di equazione 
  \(y=\dfrac{x^2}{5}-\dfrac{3}{5}x+2\). Calcoliamo la pendenza della secante che
  passa per \(A\punto{5}{4}\) e per un altro punto infinitamente vicino.\\
  La funzione è continua nel punto \(A\) e il differenziale per 
  \(x=x_A\), si può calcolare applicando la definizione (oppure applicando le 
  regole della sintesi \ref{subsubsec:diff01_diffsint}) alla funzione
  \(f(x)=\dfrac{x^2}{5}-\dfrac{3}{5}x+2\):\\ 
  \(d(f(5))=f(5+dx)-f(5)=\tonda{\dfrac{(5+dx)^2}{5}-\dfrac{3}{5}(5+dx)+2}
  -\tonda{\dfrac{5^2}{5}-\dfrac{3}{5}5+2}=\\
  =\dfrac{\tonda{25+10dx+\tonda{dx}^2}}{5}-3-\dfrac{3}{5}dx+2-5+3-2= \\
  =5+2dx+\dfrac{1}{5}\tonda{dx}^2-3-\dfrac{3}{5}dx+2-5+3-2=\\
  =2dx+\dfrac{1}{5}(dx)^2-\dfrac{3}{5}dx=
  \dfrac{7}{5}dx+\dfrac{1}{5}(dx)^2=\tonda{\dfrac{7}{5}+\dfrac{1}{5}dx}dx\).\\
 
 Il rapporto differenziale è: 
  \(\dfrac{df(5)}{dx}=\dfrac{\tonda{\dfrac{7}{5}+\dfrac{1}{5}dx}dx}
  {dx}=\dfrac{7}{5}+\dfrac{1}{5}dx\).\\
  Come si vede, la pendenza di questa secante è un numero finito del tipo 
  \(a+\epsilon\), che dipende sia dal valore \(x_A=5\), sia dall'infinitesimo 
  \(dx\) che compare nel risultato. Si tratta dunque di una pendenza 
  infinitamente vicina al valore \(m=\dfrac{7}{5}\) e questo vale per qualsiasi
  valore \(dx\ne 0\) possiamo immaginare.
\end{esempio}

\begin{esempio}
 \label{esempio:diff01_m0diff}
 Ripetiamo il calcolo precedente, con riferimento all'ascissa del vertice 
 \(x_V=\dfrac{3}{2}\). Questa volta applichiamo direttamente le regole della 
sintesi 
 \ref{subsubsec:diff01_diffsint} alla funzione\\ 
\(f(x)=\dfrac{x^2}{5}-\dfrac{3}{5}x+2\).\\
 \(d(f(x)\big|_{x=3/2}=
  \dfrac{1}{5}\tonda{2\dfrac{3}{2} dx+(dx)^2}-\dfrac{3}{5}dx=
  \dfrac{3}{5}dx+\dfrac{1}{5}(dx)^2-\dfrac{3}{5}dx=\dfrac{1}{5}(dx)^2\).\\
 Quindi il rapporto differenziale è: 
 \(\dfrac{d(f(x)}{dx}\bigg|_{x=3/2}=\dfrac{\dfrac{1}{5}(dx)^2}{dx}
 =\dfrac{1}{5}dx\)
 La secante per punti infinitamente vicini al vertice della parabola
 differisce dalla retta orizzontale per un infinitesimo, cioè è
 infinitamente vicina alla retta orizzontale che passa per il vertice.
 Anche in questo caso, il significato del rapporto differenziale non cambierebbe
 se al posto di \(dx\) si scrivesse un infinitesimo diverso, come \(\epsilon\) 
 oppure \(\delta\) o altro: la retta secante che abbiamo trovato è 
indistinguibile da una retta orizzontale.
 \end{esempio}

\begin{osservazione}
La pendenza calcolata nell'esempio \ref{esempio:diff01_mdiff} è 
\(m\approx\dfrac{7}{5}\), mentre in quest'ultimo esempio 
\ref{esempio:diff01_m0diff} è
\(m\approx 0\). Questo conferma che \(m\) cambia a seconda del punto della 
curva: \(m=m(x)\).
\end{osservazione}

\begin{esempio}
 In quale punto del piano cartesiano la parabola precedente è inclinata 
 di \(45^\circ\) rispetto all'orizzontale?\\
 Risposta: poiché solo le rette \(y=x+k\) hanno l'inclinazione richiesta dal 
problema, occorre cercare in quale punto  la parabola risulta inclinata come
una di queste rette, cioè ha lo stesso coefficiente  angolare. 
È chiaro che non si può calcolare il coefficiente angolare di una parabola, 
ma si può immaginare che nel punto desiderato esista una retta tangente che
risponde alle nostre esigenze.\\
Cerchiamo quindi per quale punto della parabola passa una retta che ha un
coefficiente angolare infinitamente vicino a \(m=1\).\\
 La funzione: \(f(x)=\dfrac{x^2}{5}-\dfrac{3}{5}x+2\).\\
 Il differenziale, calcolato per un generico valore \(x\) è:
\begin{align*}
d(f(x))&=f(x+dx)-f(x)=\tonda{\dfrac{(x+dx)^2}{5}-\dfrac{3}{5}(x+dx)+2}
  -\tonda{\dfrac{x^2}{5}-\dfrac{3}{5}x+2}=\\
 &=\tonda{\dfrac{x^2+2xdx+(dx)^2}{5}-\dfrac{3}{5}x-\dfrac{3}{5}dx+2}
  -\dfrac{x^2}{5}+\dfrac{3}{5}x-2=\\
 &=\dfrac{1}{5}x^2+\dfrac{2}{5}xdx+\dfrac{1}{5}(dx)^2-\dfrac{3}{5}x-
  \dfrac{3}{5}dx+2-\dfrac{x^2}{5}+\dfrac{3}{5}x-2=
  \dfrac{2}{5}xdx+\dfrac{1}{5}(dx)^2-\dfrac{3}{5}dx
\end{align*}
E quindi il rapporto differenziale è: 
\(\dfrac{df(x)}{dx}=\dfrac{2}{5}x+\dfrac{1}{5}dx-\dfrac{3}{5}\) \\
Il problema chiede per quale valore \(x\) il rapporto differenziale diventa 
uguale a 1, quindi:
\[\dfrac{df(x)}{dx}=1 \srarrow 
 \dfrac{2}{5}x+\dfrac{1}{5}dx-\dfrac{3}{5}=1
 \srarrow \dfrac{2}{5}x=1+\dfrac{3}{5}-\dfrac{1}{5}dx
 \srarrow x=\dfrac{8-dx}{5}\dfrac{5}{2}=\dfrac{8-dx}{2}\]
 Il punto in questione ha ascissa \(x= 4-\dfrac{1}{2}dx \approx 4\). Se si 
 trascura la parte infinitesima \(-\dfrac{1}{2}dx\), per qualsiasi \(dx\), si 
ottiene il valore reale infinitamente vicino al risultato esatto iperreale.
\end{esempio}

Verifichiamo che per \(x=4-\dfrac{1}{2}dx\) la parabola abbia davvero 
l'inclinazione cercata, cioè che la secante per due punti infinitamente 
vicini a \(x=4-\dfrac{1}{2}dx\) abbia coefficiente angolare \(m \approx 1\).
\begin{align*}
 &\dfrac{df(x)}{dx}=\dfrac{2}{5}x+\dfrac{1}{5}dx-\dfrac{3}{5} 
 \text{, (con } dx\ne 0)
 \qquad  \text{e} \qquad  x=4-\dfrac{1}{2}dx;\\
 \srarrow &\dfrac{df(x)}{dx}=\dfrac{2}{5}\tonda{4-\dfrac{1}{2}dx}
+\dfrac{1}{5}dx-\dfrac{3}{5}=\dfrac{8}{5}-\dfrac{1}{5}dx+
\dfrac{1}{5}dx-\dfrac{3}{5}=\dfrac{5}{5}=1.\\
\end{align*}

Abbiamo un risultato importante: se \(x\) è esattamente uguale a 
\(4-\dfrac{1}{2}dx\), allora la pendenza della retta \emph{tangente} alla
parabola è esattamente \(=1\). Non conoscendo il valore numerico di \(dx\)
(sappiamo solo che \(0<|dx|<1/n, \forall n \in \N\)), il risultato vale
per qualsiasi infinitesimo: \(\forall dx \ne 0\).\\
Poiché il numero iperreale \(x=4-\dfrac{1}{2}dx\) appartiene alla monade di 
\(4\),
la \emph{tangente} cercata passa per uno dei punti con ascissa \(x \in 
\mon{4}\).
Sono punti talmente vicini fra loro che la loro diversità diventa irrilevante
rispetto alla parte principale del risultato, cioè \(x=4\). La cosa si può 
esprimere
in due modi ugualmente corretti:
\begin{itemize}[nosep]
 \item se prendiamo due ascisse nella monade di \(x=4\), per i punti della 
parabola con queste ascisse passa una secante infinitamente vicina alla 
tangente desiderata;
\item la tangente desiderata passa per il punto della parabola con ascissa 
\(x=4\).
\end{itemize}
È chiaro che, se si può, sceglieremo sempre il secondo modo. Prima di tutto 
perché il ragionamento è corretto, poi perché è un modo molto più comodo, come 
si vede anche dal prossimo esempio.
\begin{esempio}
 Proseguendo l'esercizio dell'esempio \ref{esempio:diff01_mdiff}, calcoliamo 
 in due modi, negli iperreali e nei reali, la coordinata \(y\) del punto di 
 tangenza:\\
 Negli iperreali: \(x=4-\dfrac{1}{2}dx\srarrow y=\dfrac{x^2}{5}-\dfrac{3}{5}x+2=
 \dfrac{(4-\dfrac{1}{2}dx)^2}{5}-\dfrac{3}{5}(4-\dfrac{1}{2}dx)+2=\\
 =\dfrac{1}{5}\tonda{16-dx+\dfrac{1}{4}(dx)^2}-\dfrac{12}{5}+\dfrac{3}{10}dx+2=
 \cdots= \dfrac{14}{5}+\dfrac{1}{2}dx+\dfrac{1}{20}(dx)^2.\)\\
 Nei reali: \(x\approx 4\srarrow 
 y\approx\dfrac{4^2}{5}-\dfrac{3}{5}4+2=
 \dfrac{16}{5}-\dfrac{12}{5}+2=\dfrac{14}{5}\).\\
\end{esempio}

Dato che gli infinitesimi che vogliamo trascurare hanno origine nel 
rapporto differenziale, chiediamoci: 
\emph{possiamo trascurare gli infinitesimi del rapporto differenziale,
cioè fingere che valgano zero, anche se abbiamo specificato che deve essere 
\(dx \ne 0\)?}\\
Sappiamo già che esiste la risposta corretta: \emph{la funzione parte standard 
\(\pst{}\) di un iperreale finito isola il numero reale infinitamente vicino
e consente di ignorare i suoi infinitesimi}.

\section{Derivata: definizione}
\label{sec:diff01_deriv}
\begin{definizione}
  La \emph{derivata} della funzione \(f(x)\) nel punto \(\punto{x_0}{f(x_0)}\)
  è la parte standard del rapporto differenziale della funzione,
  calcolato in \(x_0\), purché essa esista e sia la stessa \(\forall dx\ne 
0\).\\
  La derivata si indica con \(f'(x_0)\) e 
  \(f'(x_0)=\pst{\dfrac{d(f(x)}{dx}\bigg|_{x=x_0}}=
            \pst{\dfrac{d(f(x_0)}{dx}}\)
\end{definizione}

La derivata, cioè l'applicazione della funzione \(\pst{}\) al rapporto
differenziale, soddisfa le nostre esigenze: fornisce il \emph{numero reale} 
più vicino al numero iperreale che risulta dal rapporto differenziale. 
La differenza fra questo e la derivata è un infinitesimo.\\
\(\dfrac{df(x)}{dx}\bigg|_{x=x_0}=f'(x_0)+\epsilon(x_0)\srarrow f'(x_0)
\approx \dfrac{df(x)}{dx}\bigg|_{x=x_0}\).

\begin{esempio}
 Deriviamo la funzione più semplice: \(f(x)=x\) per \(x=0\).\\
 Funzione: \(f(x)=x\);\\
 Differenziale per \(x=0\): \(df(0)=f(0+dx)-f(0)=0+dx-0=dx\);\\
 Rapporto differenziale \(\dfrac{df(0)}{dx}=\dfrac{dx}{dx}=1, \quad 
 \forall dx \ne 0\);\\
 Derivata: \(f'(0)=\pst{\dfrac{df(0)}{dx}}=\pst{1}=1\).\\
 In questo semplice caso il Rapporto differenziale e la derivata coincidono. 
\end{esempio}

Anche nel prossimo esempio il calcolo è molto semplice, tuttavia la derivata 
non esiste.

\begin{esempio}
 %valore assoluto
\label{esempio:diff01_derimodulo}
 Deriviamo \(f(x)=|x|\) per \(x=0\).\\
 La funzione contiene un valore assoluto: riscriviamola come se fosse 
 divisa in due rami:\\
 
% \begin{figure}[h!]
 \begin{minipage}[]{.39 \textwidth}
 \begin{center}
\begin{inaccessibleblock}
  [Derivata |x|]
  \derivamodulouno
\end{inaccessibleblock}
\end{center}
 \end{minipage} 
 \hfill
 \begin{minipage}[]{.59 \textwidth}
  Funzione: \(f(x)=|x|=\begin{cases}
   x\quad &\mbox{per }x\ge 0\\
  -x\quad &\mbox{ per }x<0
  \end{cases}\)\\
  Differenziale per \(x=0\):\\ 
\(df(0)=f(0+dx)-f(0)=0+dx-0=dx\);\\
  Rapporto differenziale:\\ 
\(\dfrac{df(0)}{dx}=\dfrac{dx}{dx}=
  \begin{cases}
  1\quad &\mbox{per }x\ge 0\\
 -1\quad &\mbox{ per }x<0
 \end{cases}\)\\
\end{minipage}
\label{}
% \end{figure} 

Si tratta di due semirette che si uniscono nell'origine. Hanno pendenza \(m=1\) 
per \(x\ge 0\) e \(m=-1\) per \(x<0\). Quale è la pendenza giusta della 
tangente per
\(x=0\), nel punto cioè dove il grafico cambia pendenza all'improvviso?\\
Nella monade di \(x=0\) gli infinitesimi positivi hanno sul grafico punti 
corrispondenti \(dy\) positivi e lo stesso avviene per i valori \(dx\) 
negativi. 
Perciò per \(x=0\) la parte standard del rapporto 
differenziale non è la stessa per qualsiasi \(dx\ne 0\) e la derivata non si può
calcolare, anche se la funzione è continua per \(x=0\).
\end{esempio}

L'esempio chiarisce perché nella definizione di derivata si debba specificare: 
``...purché essa esista e sia la stessa \(\forall dx\ne 0\)''.

\subsubsection{Significato della derivata}
L'operazione di derivazione consente di calcolare il tasso di variazione di una
funzione in un dato punto. Per tasso di variazione non si intende semplicemente
la differenza fra due valori prossimi della funzione \(df(x)\), ma la misura di 
tale differenza, ottenuta usando come unità di misura \(dx\), cioè 
confrontandola
con la variazione della variabile.\\
Dal punto di vista geometrico, se si considera il grafico della funzione 
nel piano cartesiano, la derivata in un punto misura la pendenza della tangente
al grafico in quel punto.

\begin{osservazione}
 L'operazione di derivazione è conosciuta dai tempi di Leibniz e di Newton, 
 più o meno nei termini che qui sono stati esposti. Il problema attorno al 
 quale i matematici di quell'epoca concentravano i loro sforzi era
 relativo alle variazioni: le variabili erano chiamate 
 \emph{quantità fluenti}  e le variazioni di queste erano dette 
 \emph{flussioni}.
 Calcolare una velocità, per esempio, era calcolare il rapporto
 fra la flussione dello spazio rispetto alla flussione del tempo.
\end{osservazione}

\subsubsection{Nomi per la derivata}
Il nome \emph{derivata} per indicare il calcolo che abbiamo descritto ha 
origini storiche. Si è diffuso  ovunque (derivative, derivada, dérivée, 
...) anche se non rende pienamente il significato di ciò che rappresenta. 
Se ne potrà intuire la ragione in un capitolo successivo, quando parleremo 
anche di funzioni primitive.\\
Sempre per ragioni storiche, si sono diffusi vari simboli che rappresentano 
l'operazione di derivazione:
\begin{enumerate}[noitemsep]
 \item \(f'(x_0)\) è il simbolo per il risultato della derivazione di \(f\) 
 per \(x= x_0\): semplice e sintetico;
 \item \(\mathit{D}\quadra{f(x)}\) indica la formula della derivazione di 
\(f\),  per es. \(\mathit{D}\quadra{5x\sqrt[3]{x^2}}=5\sqrt[3]{x^2}+\dfrac{10x}
 {3\sqrt[3]{x}}\);
 \item \(\dot{f}\) equivale a \(f'\); si usa in alcuni corsi universitari;
 \item \(\dfrac{d}{dx}f(x)\) è come \(f'(x)\): si pone in evidenza che si 
tratta di un rapporto con \(dx\);
 \item \(\dfrac{df(x)}{dx}\) si trova spesso nei libri come se il rapporto
 differenziale fosse identico alla derivata. In realtà sono sono due cose
 diverse.   Nella maggior parte dei casi quest'uguaglianza si può accettare, 
trattandosi di quantità indistinguibili. Per praticità, potremo
 anche noi seguire quest'uso, specificando la distinzione solo quando sarà 
 necessario.
\end{enumerate}


\subsubsection{Derivate facili e meno facili}
Vediamo alcuni esempi di derivazione.

\begin{esempio}
  Calcola \(f'(4)\) per la funzione \(f(x)=1-2\sqrt{x}\).
%\begin{figure}[h]
\begin{inaccessibleblock}
  [Derivate radice]
 \begin{center}
 \begin{minipage}[]{.40 \textwidth}
    \vspace{-5mm} 
  \derivaradice
%  \caption{\(f'(4)=-\frac{1}{2}\)}
 \end{minipage} 
 \hfill
 \begin{minipage}[]{.58 \textwidth}
   \vspace{5mm}
  Si richiede la derivata di \(f(x)=1-2\sqrt{x}\) nel punto 
  \(\punto{4}{f(4)}\), che corrisponde, nel grafico, alla pendenza della 
  retta tangente alla curva, per \(x=4\). Cioè dobbiamo calcolare:
  \begin{enumerate} [noitemsep]
   \item il differenziale della funzione;
   \item il rapporto fra questo e \(dx\) per \(x=4\);
   \item la parte standard del risultato precedente.
  \end{enumerate}
\end{minipage}
\end{center}
\end{inaccessibleblock}
\label{}
%\end{figure}

Lo svolgimento dei calcoli:
\begin{enumerate} [noitemsep]
 \item calcolare il differenziale della funzione: dalle regole apprese 
  sui differenziali (pag.\pageref{subsubsec:diff01_diffradq}) sappiamo che
  \begin{enumerate} [noitemsep]
   \item il differenziale di una differenza è la differenza dei 
   differenziali:\\
    \(d(1-\sqrt{x})=d(1)-d(2\sqrt{x})\);
   \item il differenziale di una costante è nullo: \(d(1)=0\);
   \item il differenziale del prodotto fra una costante e una funzione è
   \(d(k(f(x))=kdf(x)\), quindi: \(d(2\sqrt{x})=
   2\dfrac{dx}{(\sqrt{x+dx}+\sqrt{x})}\sim \dfrac{dx}{2\sqrt{x}}\). 
  \end{enumerate} 
  Per cui: \(d(1-2\sqrt{x})\sim\tonda{0-\frac{dx}{\sqrt{x}}}=
  -\frac{dx}{\sqrt{x}}\).
 \item calcolare il rapporto fra questo e \(dx\) nel punto richiesto:\\
  \(\tonda{\frac{d(f(x)}{dx}}\bigg|_{x=4}\sim
  \tonda{\frac{-\frac{dx}{\sqrt{x}}}{dx}}\bigg|_{x=4}=-\frac{1}{\sqrt{4}}=
  -\frac{1}{2}\);
 \item calcolare la parte standard del risultato: 
  la parte standard di \(-\frac{1}{2}\) è
  semplicemente: \(\pst{-\frac{1}{2}}=-\frac{1}{2}\).
\end{enumerate}
La retta tangente in \(\punto{4}{f(4)}\) ha pendenza pari a \(-\frac{1}{2}\).
\end{esempio}

Con le regole già date sui differenziali il calcolo è privo di difficoltà, 
non sembra che la derivata per questa funzione possa creare problemi.

\begin{esempio}
  \label{esempio:diff01_deriradice}
Calcola \(f'(0)\) per la funzione \(f(x)=1-2\sqrt{x}\).\\
Riutilizziamo i calcoli precedenti.
\begin{enumerate} [noitemsep]
 \item \(d(1-2\sqrt{x})=-\frac{dx}{\sqrt{x}}\);
 \item \(\tonda{\frac{d(f(x)}{dx}}\bigg|_{x=0}\sim
  \tonda{\frac{-\frac{dx}{\sqrt{x}}}{dx}}\bigg|_{x=0}=-\frac{1}{\sqrt{0}}=
  \dots\)?
 \item ?
\end{enumerate}
Una frazione nulla al denominatore non ha senso, il rapporto differenziale 
non è calcolabile e la derivata non esiste. 
\end{esempio}

Cerchiamo allora di capire cosa succede se il radicando è un infinitesimo 
non 
nullo \(\epsilon>0\), quindi infinitamente vicino a 0.

\begin{esempio}
Calcolare \(f'(\epsilon)\), sempre per  \(f(x)=1-2\sqrt{x}\).\\
\begin{enumerate} [noitemsep]
 \item \(d(1-2\sqrt{x})=-\frac{dx}{\sqrt{x}}\);
 \item \(\tonda{\frac{d(f(x)}{dx}}\bigg|_{x=\epsilon}=
  \tonda{\frac{-\frac{dx}{\sqrt{x}}}{dx}}\bigg|_{x=\epsilon}\sim
  -\frac{1}{\sqrt{\epsilon}}=-M\) (con \(\epsilon, M >0\));
 \item \(\pst{\frac{d(f(x)}{dx}\bigg|_{x=\epsilon}}=\pst{-M}=\) ?
\end{enumerate} 
\begin{osservazione}
 \(-M\) è un infinito negativo perché \(\epsilon\) si suppone positivo. Non 
avrebbe  senso, comunque, fare un tentativo con \(\epsilon\) negativo, perché 
la radice quadrata di numeri negativi (reali e iperreali) non è definita.\\
\end{osservazione} 

La parte standard di un numero infinito non esiste. La derivata non esiste, 
quindi la pendenza della tangente per \(x=0\) non può essere calcolata.\\
Esiste però la pendenza della retta secante
fra i due punti infinitamente vicini \(\punto{0}{f(0)=1}\) e 
\(\punto{\epsilon}{f(\epsilon)}\). Infatti il rapporto 
differenziale appena calcolato approssima questa pendenza.\\
Vediamo nel dettaglio l'equazione di questa retta, con la formula della
retta passante per i due punti: \(A\punto{0}{1}\) e 
\(B\punto{\epsilon}{f(\epsilon)=1-2\sqrt{\epsilon}}\).\\
\(\dfrac{x-x_A}{x_B-x_A}=\dfrac{y-y_A}{y_B-y_A}\srarrow
\dfrac{x-0}{\epsilon-0}=\dfrac{y-1}{1-2\sqrt{\epsilon}-1}\srarrow
\dfrac{x}{\epsilon}=\dfrac{y-1}{-2\sqrt{\epsilon}}\srarrow
y=\dfrac{-2\sqrt{\epsilon}}{\epsilon}x+\dfrac{1}{2\sqrt{\epsilon}}\).\\
La pendenza di questa secante è \(m=\dfrac{-2\sqrt{\epsilon}}{\epsilon}=
-\dfrac{2}{\sqrt{\epsilon}}\).\\
La frazione ha senso per qualsiasi \(\epsilon>0\), che è un numero 
infinitamente vicino a \(0\).  Se consideriamo vari \(\epsilon\), sempre 
più piccoli, \(m\) diventa sempre più negativo. Possiamo quindi che per \(x=0\)
la tangente ipotetica sarebbe una retta verticale.\\
\end{esempio}

\begin{esempio}
 %discontinua
 Per la funzione \(f(x)=\dfrac{1}{x-2}\) calcola la derivata \(f'(1)\).\\
%\begin{figure}[h]
\begin{inaccessibleblock}
  [Derivate radice]
 \begin{center}
 \begin{minipage}[]{.40 \textwidth}
   \vspace{-.5em}
  \derivaomografica
 % \caption{}
 \end{minipage} 
 \hfill
 \begin{minipage}[]{.58 \textwidth}
   Per le regole che presto approfondiremo, 
\(d\quadra{(x-2)^{-1}}~=~d(x^{-1})\)
   perciò possiamo fare riferimento al teorema 
pag.\pageref{subsubsec:diff01_diffrecip}.
\begin{enumerate} [noitemsep]
 \item \(d(f(x))=d\tonda {\frac{1}{x-2}}=-\dfrac{dx}{(x-2)^2+(x-2)dx}\);
 \item \(\tonda{\dfrac{d(f(x)}{dx}}\bigg|_{x=1}=
  \tonda{\dfrac{-\frac{dx}{(x-2)^2+(x-2)dx}}{dx}}\bigg|_{x=1}=\\ 
  =-\dfrac{1}{1-dx}\);
  \item \(\pst{-\dfrac{1}{1-dx}}=-1\).
\end{enumerate}
\end{minipage}
\end{center}
\end{inaccessibleblock}
\label{}
%\end{figure} 
Per \(x=1\), la tangente ha pendenza \(m=-1\).\\
\end{esempio}

\begin{esempio}
Per la funzione \(f(x)=\dfrac{1}{x-2}\) calcola la derivata \(f'(2)\).\\
\(f(x)\) è la stessa dell'esempio precedente, quindi:
\begin{enumerate} [noitemsep]
 \item \(d(f(x))=d\tonda {\frac{1}{x-2}}=-\dfrac{dx}{(x-2)^2+(x-2)dx}\);
 \item \(\tonda{\dfrac{d(f(x)}{dx}}\bigg|_{x=2}=
  \tonda{\dfrac{-\frac{dx}{(x-2)^2+(x-2)dx}}{dx}}\bigg|_{x=2} 
  =-\dfrac{1}{0-0dx}=\)?;
  \item è inutile calcolare la parte standard di un numero privo di senso.
\end{enumerate}
Cosa è successo in questo secondo caso? Che la funzione non è definita per 
\(x=2\).
Lo rende evidente il grafico, ma sarebbe stato meglio, prima ancora di
disegnarlo, studiare l'insieme di definizione e evitare calcoli inutili.
Infatti il differenziale è calcolabile solo nei punti in cui \(f(x)\) è
definita, e lo stesso vale anche per la derivata.
\end{esempio}


\begin{esempio}
 %valore assoluto
 Per la funzione \(f(x)=\dfrac{1}{2}|x-2|+2\) calcola le derivate \(f'(0)\), 
\(f'(4)\) e \(f'(2)\).\\
 La funzione contiene un valore assoluto: seguendo l’esempio 
\ref{esempio:diff01_derimodulo} riscriviamola come se fosse 
 divisa in due rami:

% \begin{figure}[h!]
 \begin{minipage}[]{.38 \textwidth}
 \begin{center}
\begin{inaccessibleblock}
  [Derivate radice]
  \derivamodulodue
\end{inaccessibleblock}
\end{center}
%   \caption{}
 \end{minipage} 
 \hfill
 \begin{minipage}[]{.58 \textwidth}
\(f(x)= \dfrac{1}{2}|x-2|+2=\\
=\begin{cases}
  \dfrac{x-2}{2}+2 &\text{ per } x-2< 0  \\
 \dfrac{-(x-2)}{2}+2 &\text{ per } x-2 \ge 0
\end{cases} 
\srarrow\\
\srarrow f(x)= \begin{cases}
 \dfrac{x}{2}+1 &\text{ per } x<2 \\
 -\dfrac{x}{2}+3 &\text{ per } x\ge 2
\end{cases}\)
\end{minipage}
% \end{figure} 
\vspace{1em}
Si tratta di due semirette che si uniscono in \(\punto{2}{2}\). L'equazione
di ciascuna di loro è una funzione lineare. Abbiamo già visto che la derivata 
dell'equazione di una retta coincide con il suo coefficiente angolare, 
quindi, anche senza derivare, avremo:
\(f'(0)=\dfrac{1}{2}\) e \(f'(4)=-\dfrac{1}{2}\).

Il calcolo di \(f'(2)\) invece è meno banale:\\
Abbiamo
\(f'(x)= \begin{cases}
 \dfrac{1}{2} &\text{ per } x<2 \\
 -\dfrac{1}{2} &\text{ per } x > 2
\end{cases}\)
Quale è la pendenza giusta della tangente per
\(x=2\), nel punto cioè dove il grafico cambia pendenza all'improvviso?\\
Per ragioni analoghe a quelle illustrate in \ref{esempio:diff01_derimodulo}
concludiamo che \(f'(2)\)  non esiste.
\end{esempio}

Da tutti questi esempi impariamo che per poter calcolare la derivata:
\begin{enumerate}[noitemsep]
 \item \(f(x)\) deve essere continua nel punto desiderato ed è una condizione
 necessaria per poter derivare (ma non sufficiente);
 \item il rapporto differenziale deve essere un numero finito;
 \item il risultato deve essere indipendente dalla scelta di \(dx\), cioè
 deve valere \(\forall dx\);
\end{enumerate}
\begin{osservazione}
 Inoltre abbiamo visto un altro fatto importante: la derivata ha un 
risultato
 in genere diverso a seconda del valore \(x_0\) per il quale viene calcolata, 
 cioè varia al variare di \(x_0\). Poiché se si fissa \(x_0\) il risultato, se
 esiste, è unico allora la derivata di una funzione è a sua volta una 
funzione.
 \end{osservazione}

 \begin{definizione}
  Una funzione per la quale la derivata è calcolabile \(\forall x_0\) del suo
 dominio si dice funzione derivabile.
\end{definizione}

\begin{osservazione}
 Una funzione derivabile è sicuramente continua, mentre il contrario non 
vale.
\end{osservazione}


\section{Derivare funzioni algebriche}
\label{sec:diff01_derialg}
Sistemate le questioni preliminari, passiamo al calcolo: impariamo a 
derivare.
Nei casi semplici ci avvarremo di quanto visto a proposito dei 
differenziali, ma, per le funzioni non trattate allora, dovremo calcolare 
anche questi.
Al termine, raccoglieremo i risultati utili in un prospetto riassuntivo.\\
Immaginiamo che le funzioni da derivare siano derivabili \(\forall x\) 
dell'insieme di definizione, per cui la derivata di \(f\) nel generico 
punto \(\punto{x}{f(x)}\) sarà \(f'(x)\).\\
Grazie al capitolo \ref{subsubsec:diff01_diffsint}, sappiamo già come
differenziare alcune funzioni algebriche: da quelle regole e dalla 
definizione di derivata ...deriva direttamente quanto segue.
\begin{teorema}
  La derivata di una funzione costante è \(0\): \(\mathit{D}\quadra{k}=0\).
\end{teorema}
\noindent Ipotesi: \(f(x)=k\).\tab Tesi: \(f'(x)=0\).
\begin{proof}
 Infatti \(df(x)=0\).
\end{proof}

\begin{teorema}
  La derivata della funzione identica è 1: \(\mathit{D}\quadra{x}=1\).
\end{teorema}
\noindent Ipotesi: \(f(x)=x\).\tab Tesi: \(f'(x)=1\).
\begin{proof}
 Infatti \(df(x)=\epsilon=dx\), quindi il rapporto differenziale è \(1\) e così
 anche la sua parte standard.
\end{proof}
\begin{osservazione}
 \(m=1\) è quindi anche la pendenza della bisettrice \(y=x\), cosa ormai 
risaputa.
\end{osservazione}

\begin{teorema}
  La derivata della funzione quadratica è: \(\mathit{D}\quadra{x^2}=2x\).
\end{teorema}
\noindent Ipotesi: \(f(x)=x^2\) .\tab Tesi: \(f'(x)=2x\).
\begin{proof}
  Infatti \(df(x)=2xdx+(dx)^2\) e il rapporto differenziale è \(2x+dx\) da cui,
  applicando la definizione di derivata, ...
\end{proof}

% \begin{figure}[h!]
 \begin{minipage}[]{.48\textwidth}
\begin{inaccessibleblock}
  [m tangenti a parabola]
 \begin{center}
 \parabola
 \end{center}
\end{inaccessibleblock}
 \end{minipage} 
 \hfill
 \begin{minipage}[]{.48\textwidth}
 \begin{center}
  \tangentiparabola
 \end{center}
 \end{minipage}
% \caption{\(y=x^2\) e la pendenza \(y=m(x)=2x\) delle sue tangenti.} 
% \label{}
% \end{figure}
\vspace{1em}
Iniziamo dal ramo sinistro del grafico: al crescere di \(x\), la curva e le 
sue tangenti, indistinguibili da essa nei punti di tangenza, passano
da un'inclinazione fortemente verso il basso (\(m<0\)) alla direzione 
orizzontale,
nel vertice. Per \(x>0\), poi, l'inclinazione aumenta progressivamente. Il 
progresso della pendenza delle tangenti è costante: per questo motivo il 
grafico di \(y=m(x)\) è una retta per l'origine.

\begin{osservazione}
Nota che la funzione derivata di una funzione quadratica è una funzione 
lineare: la pendenza delle tangenti a una parabola varia come varia la \(y\)
rispetto alla \(x\) in una retta.
\end{osservazione}

\begin{teorema}
  \label{diff01_teoderpotenza}
  La derivata della generica  è: \(\mathit{D}\quadra{x^n}=
  nx^{n-1}\).
\end{teorema}
\noindent Ipotesi: \(f(x)=x^n\) .\tab Tesi: \(f'(x)=nx^{n-1}\).
\begin{proof}
  Infatti il differenziale è \(df(x)=nx^{n-1}dx+ \delta(x)\) e,
  applicando la definizione di derivata, ...
\end{proof}

\begin{osservazione}
 Ripetendo l'osservazione a pag.\pageref{subsubsec:diff01_diffpot} relativa 
 a queste funzioni, il teorema \ref{diff01_teoderpotenza} è del tutto generale: 
si  applica con qualsiasi esponente reale.
Vale quindi anche per le funzioni radicali di qualsiasi indice e per le funzioni
razionali fratte, come esemplifichiamo nei prossimi due casi, molto comuni.
\end{osservazione}

% \begin{figure}[h!]
 \begin{minipage}[]{.48\textwidth}
\begin{inaccessibleblock}
  [m tangenti a cubica]
 \begin{center}
 \cubica
 \end{center}
\end{inaccessibleblock}
 \end{minipage} 
 \hfill
 \begin{minipage}[]{.48\textwidth}
 \begin{center}
  \tangenticubica
 \end{center}
 \end{minipage}
% \caption{\(y=x^3\) e la pendenza \(y=m(x)=3x^2\) delle sue tangenti.} 
% \label{}
% \end{figure}
\vspace{1em}
Come esempio di derivata della funzione potenza, consideriamo \(f(x)=x^3\)
e il suo grafico nel piano cartesiano. 
I due rami del grafico sono simmetrici rispetto all'origine e quindi lo sono 
anche le pendenze delle tangenti. 
Considerando le \(x\) crescenti, quindi da sinistra verso destra, le pendenze 
delle tangenti sono sempre positive, all'inizio molto accentuate, poi 
diminuiscono fino a \(m=0\). Oltre l'origine, riprendono a crescere, in maniera 
sempre più evidente. Il grafico di \(y=m(x)=3 x^2\) ha infatti la forma di una 
parabola simmetrica rispetto all'asse \(Y\).


\begin{corollario}
  La derivata della funzione radice quadrata è: 
\(\mathit{D}\quadra{\sqrt{x}}=
    \dfrac{1}{2\sqrt{x}}\), con la restrizione \(x\ne 0\).
\end{corollario}
\noindent Ipotesi: \(f(x)=\sqrt{x}\), con \(x\ne 0\) .\tab Tesi: 
    \(f'(x)=\dfrac{1}{2\sqrt{x}}\).

\pagebreak %-------------------------------------------

% \emph{Dimostrazione}
\begin{proof}
  Infatti il differenziale è \(df(x)=\dfrac{dx}{\sqrt{x+dx}+\sqrt{x}}\) e,
  applicando la definizione di derivata, si ha:\\
  \(\pst{\dfrac{dx}{dx\tonda{\sqrt{x+dx}+\sqrt{x}}}}=
  \pst{\dfrac{1}{\sqrt{x+dx}+\sqrt{x}}}=
      \dfrac{1}{\pst{\sqrt{x+dx}+\sqrt{x}}}=
      \dfrac{1}{\pst{\sqrt{x+dx}}+\pst{\sqrt{x}}}=\\
      =\dfrac{1}{\sqrt{x}+\sqrt{x}}=\dfrac{1}{2\sqrt{x}}\).
\end{proof}

% \begin{figure}[h!]
\begin{inaccessibleblock}
  [m tangenti a radquad]
 \begin{minipage}[]{.48\textwidth}
 \begin{center}
\scalebox{.9}{ \radquad}
 \end{center}
 \end{minipage} 
 \hfill
 \begin{minipage}[]{.48\textwidth}
 \begin{center}
\scalebox{.9}{  \tangentiradquad}
 \end{center}
 \end{minipage}
\end{inaccessibleblock}
% \caption{\(y=\sqrt{x}\) e la pendenza \(y=m(x)=\dfrac{1}{2\sqrt{x}}\) delle 
sue 
%   tangenti.} 
% \label{}
% \end{figure}

Le rette tangenti ai punti vicini all'origine hanno una pendenza elevata, 
che 
si attenua gradualmente man mano che \(x\) aumenta, fino ad assestarsi quasi
orizzontalmente.

\begin{corollario}
  La derivata della funzione reciproca è: \(\mathit{D}\quadra{\dfrac{1}{x}}=
    -\dfrac{1}{x^2}\).
\end{corollario}
\noindent Ipotesi: \(f(x)=\dfrac{1}{x}\).\tab Tesi: \(f'(x)=-\dfrac{1}{x^2}\).
\begin{proof}
  Infatti il differenziale è \(df(x)=\dfrac{-dx}{x(x+dx)}\) e,
  applicando la definizione di derivata, si ha:\\
  \(\pst{\dfrac{-dx}{dx(x(x+dx))}}=
\dfrac{-1}{\pst{x(x+dx)}}=\dfrac{-1}{\pst{x}\pst{x+dx}}=
      -\dfrac{1}{x\cdot x}=-\dfrac{1}{x^2}\).
\end{proof}

% \begin{figure}[h!]
\begin{inaccessibleblock}
  [Grafico della funzione 1/x e della sua derivata: 1/(x^2)]
 \begin{minipage}[]{.48\textwidth}
 \begin{center}
\scalebox{.9}{ \recip}
 \end{center}
 \end{minipage} 
 \hfill
 \begin{minipage}[]{.48\textwidth}
 \begin{center}
\scalebox{.9}{  \tangentirecip}
 \end{center}
 \end{minipage}
\end{inaccessibleblock}
% \caption{\(y=\dfrac{1}{x}\) e la pendenza \(y=m(x)=\dfrac{-1}{x^2}\) delle 
sue 
%   tangenti.} 
% \label{}
% \end{figure}

\begin{osservazione}
 Ovviamente, applicando alla lettera il teorema sulla derivata delle funzioni
 potenza si ottengono gli stessi risultati esposti in questi due ultimi 
 corollari.
\end{osservazione}

\section{Regole di derivazione}
\label{sec:diff01_regolederivate}
Possiamo applicare i teoremi precedenti a casi meno elementari, cioè
a funzioni algebriche che contengono somme, prodotti e quozienti di
funzioni elementari. 
\begin{esempio}
  Derivare la funzione \(f(x)= 3x-\dfrac{3}{x}\) in \(x_0=3\).\\
  Si tratta di una funzione nuova, ma è facile riconoscere che è formata 
  dalla somma (algebrica) di due funzioni e ciascuna di queste è data dal 
  prodotto fra la costante \(3\) e una funzione appena trattata. Perciò:\\
  \(\mathit{D}\quadra{3x}=3\cdot\mathit{D}\quadra{x}=3\cdot 1=3\); \tab
  \(\mathit{D}\quadra{\dfrac{3}{x}}=3\cdot\mathit{D}\quadra{\dfrac{1}{x}}
  =3\cdot \dfrac{-1}{x^2}=\dfrac{-3}{x^2}\);\\
  \(f'(x)=\mathit{D}\quadra{3x-\dfrac{3}{x}}=\mathit{D}\quadra{3x}-
  \mathit{D}\quadra{\dfrac{3}{x}}= 3-\dfrac{-3}{x^2}= 3+\dfrac{3}{x^2}\);\\
  \(f'(3)=3+\dfrac{3}{9}= 10\).
\end{esempio}
Senza troppi problemi, abbiamo dato per scontato che 
\begin{enumerate}[noitemsep]
 \item La derivata del prodotto tra una costante e  una funzione è il 
prodotto  fra la costante e la derivata della funzione: \quad 
 \(\mathit{D}\quadra{kf(x)}=k\mathit{D}\quadra{f(x)}\).
 \item La derivata della somma algebrica fra due funzioni è la somma 
algebrica  delle due derivate: \quad
\(\mathit{D}\quadra{f(x)+g(x)}=\mathit{D}\quadra{f(x)}+\mathit{D}\quadra{g(x)
}\).
\end{enumerate}
Da dove derivano queste certezze? Basta tornare alle regole di composizione
dei differenziali (pag.\pageref{subsec:diff01_combdiff}) per averne la 
conferma.

\begin{esempio}
Deriva la funzione che nel piano cartesiano è rappresentata dalla retta 
\(y=x+9\).\\
\(\mathit{D}\quadra{x+9}=1\).
\end{esempio}

Sempre in riferimento a quanto appreso sui differenziali 
e vista la definizione di derivata e le proprietà della parte standard, 
anche le regole 3 e 4 risultano giustificate:
\begin{enumerate}[noitemsep]
\setcounter{enumi}{2}
\item La derivata del prodotto fra due funzioni è la somma fra due prodotti:
la derivata della prima funzione per la seconda (non derivata) più la prima 
funzione (non derivata) per la derivata della seconda: \quad 
\(\mathit{D}\quadra{f(x)\cdot g(x)}=f'(x)\cdot g(x)+f(x)\cdot g'(x)\).
\item La derivata del quoziente fra due funzioni è la frazione che ha per 
denominatore il quadrato del divisore e per numeratore la differenza fra due
prodotti: la derivata della prima funzione per la seconda (non derivata) 
meno
la prima funzione (non derivata) per la derivata della seconda: \quad 
\(\mathit{D}\quadra{\dfrac{f(x)}{g(x)}}=\dfrac{f'(x)\cdot g(x)-f(x)\cdot 
g'(x)}
  {\quadra{g(x)}^2}\).
\end{enumerate} 

\begin{esempio}
  Calcola la derivata del prodotto \(f(x)=x\sqrt{x}\).\\
  \(f'(x)=1\cdot \sqrt{x}+x\cdot\frac{1}{2\sqrt{x}}=
  \sqrt{x}+\frac{x}{2\sqrt{x}}\) (se \(x\ne 0\)).\\
  Fin qui l'applicazione della regola 3. Ma il risultato si può scrivere 
  in forma più compatta, perché \(\sqrt{x}+\frac{x}{2\sqrt{x}}=
  \sqrt{x}+\frac{\sqrt{x}}{2}=\frac{3}{2}\sqrt{x}\).
  \end{esempio}
\begin{osservazione}
  In realtà per fare questo calcolo non siamo obbligati ad applicare la regola 
  3, poiché \(f(x)=x\sqrt{x}=x^{1+\frac{1}{2}}=x^{\frac{3}{2}}\). Verifica che 
  applicando il teorema \ref{diff01_teoderpotenza} a \(f(x)\) espressa in 
questa 
  forma la derivata risulta la stessa.
 \end{osservazione}


\begin{esempio}
  Derivare \(f(x)=\dfrac{x}{\sqrt{x}}\).\\
  Seguendo la regola n.4: \(f'(x)= 
\dfrac{1\cdot\sqrt{x}-x\dfrac{1}{2\sqrt{x}}}
  {\tonda{\sqrt{x}}^2}=\dfrac{\sqrt{x}-\dfrac{\sqrt{x}}{2}}{x}=
  \dfrac{\dfrac{\sqrt{x}}{2}}{x}=\dfrac{1}{2\sqrt{x}}\).\\
  Ma guarda che combinazione: abbiamo ottenuto la derivata della radice! 
Allora la funzione di partenza è equivalente a \(f(x)=\sqrt{x}\)? (Ad essere 
precisi, 
non   esattamente. Infatti \dots)
\end{esempio}
\begin{esempio}
  Sappiamo già che \(\mathit{D}\tonda{\dfrac{1}{x}}=\dfrac{-1}{x^2}\). 
  Mettiamo alla prova ancora una volta la regola n.4:
  \(f'(x)=\dfrac{0\cdot x-1\cdot 1}{x^2}= \dots\)
\end{esempio}

\begin{esempio}
 Ora finalmente un calcolo che si può svolgere solo con la regola n.4.
 Derivare \(f(x)= \dfrac{x+2}{x^3-x+4}\).\\
 \(f'(x)=\dfrac{1\cdot(x^3-x+4)-(x+2)(3x^2-1)}{(x^3-x+4)^2}\). Fin qui 
 l'applicazione della regola. \\
 Con ulteriori calcoli:
 \(\dots =\dfrac{x^3-x+4-(3x^3-x+6x^2-2)}{(x^3-x+4)^2}=
 \dfrac{-2x^3-6x^2+6}{(x^3-x+4)^2}\).\\
\end{esempio}


\section{Derivare funzioni composte e funzioni inverse}
\label{}
\subsection{Funzioni composte}
\label{subsec:diff01_dericomp}

\begin{esempio}
 \(s(t)~=~s_0~+~v_0t~+~\frac{1}{2}at^2\) è la legge oraria del moto rettilineo
 uniformemente accelerato. Anche se nella formula mancano le usuali 
 sigle \(f(x)\), \(y\), \(x\), si tratta di una comune funzione polinomiale
 di 2° grado e le si possono applicare le regole che stiamo studiando.\\
 Derivando si ottiene \(s'(t)=0+v_0+\frac{1}{2}\cdot 2at=v_0+at\) 
che, essendo la derivata dello spazio rispetto al tempo, esprime la velocita 
\(v(t)\) in questo tipo di moto.
\end{esempio}
L'esempio serve a ricordare che le funzioni e le variabili si esprimono con 
sigle qualsiasi, ma questo non cambia le regole. La libertà di uso dei simboli
può facilitare i calcoli, come si vede nel prossimo caso.
\begin{esempio}
  Deriva la funzione \(v(u)=\dfrac{u^2}{8}\). 
  Soluzione: \(v'(u)=\dfrac{1}{8}2u=\dfrac{u}{4}\). Infatti il differenziale è 
\(dv(u)=\dfrac{1}{8}[2udu+(du)^2]\) perché \(df(x)=2xdx+(dx)^2\) 
  e \(d(kf(x)=kdf(x)\). La parte standard del rapporto differenziale fornisce
  il risultato \(\dfrac{u}{4}\).
\end{esempio}

\begin{esempio}
Deriva la funzione \(u(t)=3t-2\). Soluzione: \(u'(t)=3\). Infatti il 
 differenziale è \(du(t)=3dt-0\). Allora la parte 
 standard del rapporto differenziale fornisce il risultato \(3\).
\end{esempio}

Combiniamo i due esempi: \(v=f(u)\) e \(u=f(t)\), cioè \(v\) è funzione di 
\(u\), 
perché i suoi valori dipendono dai quadrati, divisi per \(8\), dei numeri \(u\).
Invece \(u\) è funzione di \(t\), nel senso che i suoi valori sono i valori 
\(t\) 
triplicati e poi ridotti di 2. In ``matematichese'':\\
\(v(u(t))=\dfrac{[u(t)]^2}{8}=\dfrac{(3t-2)^2}{8}\).


\begin{inaccessibleblock}
  [box funzione composta]
 \begin{center}
 \begin{minipage}[]{.48\textwidth}
  \boxfcomposta
 \end{minipage} 
 \hfill
 \begin{minipage}[]{.48\textwidth}
Si tratta di una macchina che incatena due calcoli successivi.\\
Immettiamo ad esempio il valore \(t=6\). La macchina sviluppa al suo interno 
\(u(6)=3\cdot 6-2=16\) grazie a \(u\) e infine produce 
\(v(16)=\dfrac{16^2}{8}=32\).
Una catena del genere si chiama \emph{funzione di funzione} o
\emph{funzione composta}: \(v(u(t))\).\\
 \end{minipage}
 \end{center}
\end{inaccessibleblock}
\label{}

Come deriviamo  \(v\) rispetto a \(t\)? Dalla definizione di derivata: 
\(\mathit{D}\quadra{v(u(t))}=\pst{\dfrac{d(v(u(t)))}{dt}}\), 
e il punto è il calcolo dei differenziali.
\begin{inaccessibleblock}
  [differenziale funzione composta]
 \begin{center}
 \begin{minipage}[]{.48\textwidth}
  \diffcomposta
 \end{minipage} 
  \hfill
 \begin{minipage}[]{.48\textwidth}
Dal primo esempio sappiamo che \(dv=\dfrac{u}{4}du\ +\) infinitesimi di 
ordine 
superiore.
Poiché \(u=3t-2\), \(du=3dt\), avremo:\\
\(dv\approx \dfrac{u}{4}du\). \hspace{1cm} \(du=3dt\). \(\srarrow 
dv=\dfrac{3t-2}{4}3dt \srarrow\\ 
\srarrow 
\pst{\dfrac{d(v(u(t)))}{dt}}=\pst{\dfrac{(3t-2)3dt}{4dt}}=\)
\(=\dfrac{9t-6}{4}\).
 \end{minipage}
 \end{center}
\end{inaccessibleblock}
\label{}

C'è un modo più semplice? In questo caso, sì: basta sviluppare il quadrato
\((3t-2)^2\), dividere ogni termine per \(8\) e poi derivare il polinomio. 
Ma a volte il modo più semplice non c'è.
\begin{esempio}
  Calcolare \(f'(x)\), con \(f(x)=\sqrt{3-x^2}\).\\
  \(f(x)\) è composta: si può pensare
  formata così: \(g(x)=3-x^2\) e \(f(g(x))=\sqrt{g(x)}=\sqrt{3-x^2}\).\\
  In questo modo si vedono meglio i differenziali.
  \(df(x)=d\tonda{\sqrt{g(x)}}\approx\frac{1}{2\sqrt{g(x)}}\cdot dg(x)\) e 
  \(dg(x)=d(3-x^2)\approx-2xdx\). Per brevità, raccogliamo sotto un'unica
  sigla \(\epsilon\) tutti gli infinitesimi 
  di ordine superiore, che poi la parte standard si incaricherà di far 
  scomparire nel momento di calcolare la derivata.\\
  Il differenziale: \(df(x)=df(g(x))=\frac{1}{2\sqrt{g(x)}}\cdot 
dg(x)+\delta=
  \frac{1}{2\sqrt{3-x^2}}\cdot (-2x)dx+ \epsilon\).\\
  Da qui la derivata: 
  \(f'(x)=\pst{\dfrac{\frac{-2xdx}{2\sqrt{3-x^2}} +\epsilon}{dx}}=
  \dfrac{-x}{\sqrt{3-x^2}}\).
\end{esempio}
Esaminiamo in modo astratto come abbiamo costruito il rapporto 
differenziale della funzione composta nell'esempio precedente: 
\(\dfrac{df}{dx}=\dfrac{df}{dg}\dfrac{dg}{dx}\). \\
Sembra un'uguaglianza banale, perché semplificando si otterrebbe l'identità.
In realtà i due differenziali \(dg\) sono scritti allo stesso modo ma hanno un
significato diverso e non è detto che si possano semplificare.\\
\(dg\) al denominatore differenzia la variabile indipendente della funzione 
\(f\), \(dg\) al numeratore è la conseguenza del differenziale \(dx\). Perciò
se varia \(x\) si ha un effetto su \(g(x)\) e se varia \(g\) si ha un effetto 
su  \(f(g)\). In generale le due variazioni \(dg\) saranno diverse per 
infinitesimi
di ordine superiore, che possono essere eliminati considerando le parti 
standard.\\
Non avremo occasione di affrontare queste difficoltà, nei nostri esercizi.
Ma sono questioni importanti per una regola che vogliamo sempre valida. 
Perciò il prossimo enunciato deve contenere tutti i dettagli.

\begin{teorema}
  \label{teo:diff01_dericomp}
 Se esistono le derivate \(g'(x)\) e \(f'(g(x))\) per il medesimo valore \(x\),
 la funzione composta \(f(g(x))\) è derivabile e la sua derivata si calcola
 così: \(f'(x)=f'(g(x))=f'(g)\cdot g'(x)\), cioè la derivata di una funzione 
composta è il prodotto delle derivate delle funzioni componenti, 
ciascuna rispetto alla propria variabile.
\end{teorema}
\noindent Ipotesi: \(f(x)=f(g(x))\), \(f\), \(g\) derivabili .\tab 
Tesi: \(f'(x)=f'(g(x))=f'(g(x))\cdot g'(x)\).
% \begin{proof}
%   Tralasciando di specificare la scomparsa degli infinitesimi di ordine 
%   superiore e grazie alle proprietà della funzione \(\pst{}\), abbiamo:\\
%   \(f'(g(x))=\pst{\dfrac{df(x)}{dx}}=
%   \pst{\dfrac{df(g)}{dg}\dfrac{dg(x)}{dx}}=
%   \pst{\dfrac{df(g))}{dg}}\pst{\dfrac{dg(x)}{dx}}=
%   f'(g(x))g'(x)\).
% \end{proof}
\begin{esempio}
  Derivare \(f(x)=\tonda{-\dfrac{3}{2}x^3+2x^2-6}^5\).\\
  Poniamo \(g(x)=-\dfrac{2}{3}x^3+2x^2-6\) e \(f(g)=g^5\). \\
  Allora: 
  \(f'(g)= 5g^4\) e \(g'(x)=-2x^2+4x\), \\
  quindi: 
  \(f'(x)=f'(g)\cdot g'(x)=5g^4(-2x^2+4x)=
  5\tonda{-\dfrac{2}{3}x^3+2x^2-6}^4(-2x^2+4x)\).
\end{esempio}
\begin{osservazione}
 La regola della funzione composta si estende ai casi in cui le funzioni 
 in gioco sono tre, o più:
 \(\mathit{D}\quadra{f(g(h(x)))}=f'(g)\cdot g'(h)\cdot h'(x)\).
\end{osservazione}
% \begin{osservazione}
%  Lo studente smart si era già accorto che la derivata di un prodotto non 
%  è il prodotto delle derivate. Ora arriva la conferma:
%  il prodotto delle derivate non è la derivata di un prodotto.
% \end{osservazione}

\subsection{Funzioni inverse}
\label{subsec:diff01_derifuninverse}

 Cosa si intende per funzioni inverse? \(y=kx\) e \(x=\frac{y}{k}\), per 
esempio, 
sono formule inverse l'una dell'altra, ma non sono funzioni inverse rispetto 
alla stessa variabile \(x\). Esse esprimono con due differenti espressioni la 
stessa iperbole equilatera e hanno lo stesso grafico, quindi le stesse tangenti 
al grafico e le stesse derivate rispetto a \(x\).\\
La funzione inversa rispetto a \(x\) di \(y=kx\) è \(y=\frac{x}{k}\), cioè è
la formula inversa, ma applicata a \(x\). 

Nella tabella che segue consideriamo 
come funzione inversa l'inversione della formula \(y=\dots\), come ad esempio 
\(x=\frac{y}{k}\). Per questo la derivata viene calcolata rispetto a \(y\): 
\(x'=\pst{\dfrac{dx}{dy}}\).\\

\affiancati{.39}{.59}{
Ecco alcuni esempi di semplici funzioni algebriche, con le loro 
formule dirette e inverse e con le rispettive derivate.
}{
\begin{center}
\begin{tabular}{cccc}
\(y=f(x)\) & \(x=g(y)\) & \(f'(x)\) & \(g'(y)\)\\\hline
\(y=x+c\) & \(x=y-c\) & 1 & 1\\
\(y=kx\) & \(x=y/k\) & \(k\) & \(frac{1}{k}\)\\
\(y=x^2\), \(x > 0\) & \(x=\sqrt{y}\) & \(2x\) & 
\(frac{1}{2\sqrt{y}}=\frac{1}{2x}\)\\
\(y=\frac{1}{x}\) & \(x=\frac{1}{y}\) & \(-\frac{1}{x^2}\) & 
\(-\frac{1}{y^2}=-x^2\)\\
% \hline
\end{tabular}
\end{center}
}

\begin{osservazione}
La funzione  \(y=x^2\), nella terza riga della tabella, è definita \(\forall 
x\).
Tuttavia qui il dominio viene ristretto, in modo da considerare un solo ramo 
della parabola. In questo modo si può definire come funzione la formula inversa
\(x=g(y)\) (\(x\ge 0\)) e la sua derivata (\(x\ne 0\)).\\
Occorre sempre porre attenzione al dominio di \(f\), quando si vuole 
definire la sua inversa. Una buona regola pratica per capire se \(g=f^{-1}\) è 
definibile, è di tagliare il grafico di \(f\) con una retta orizzontale: se la
retta incrocia il grafico di \(f\) in più punti, \(f^{-1}\) non esiste.
\end{osservazione}

Considera il caso semplice che segue.
\begin{esempio}
  Derivare \(f(x)=\sqrt{x^2}\) restringendo il dominio a \(x > 0\).\\
  Si dirà: non c'è problema, \(f(x)\) corrisponde algebricamente a 
  \(f(x)=\sqrt{x^2}=x\), perciò  \(f'(x)=1\).\\
  Vero. Ma poniamo \(g(x)=x^2\) e \(f(x)=f(g(x))=\sqrt{g(x)}\).\\
  Con la regola delle funzioni composte si ha:\\ 
  \(f'(x)=f'(g)\cdot g'(x)= \dfrac{1}{2\sqrt{g}}\cdot 2x=
  \dfrac{1}{2\sqrt{x^2}}\cdot 2x=\dfrac{1}{2x}\cdot 2x=1\).\\
  Conclusione: 
\(\mathit{D}\quadra{x^2}=\dfrac{1}{\mathit{D}\quadra{\sqrt{x}}}\).
\end{esempio}
Si intuisce che: siccome \(f'(g)\cdot g'(x)=1\), allora 
\(g'(x)=\dfrac{1}{f'(g)}\).
L'intuizione è corretta ed effettivamente una regola simile esiste. Occorre 
però precisare che la regola vale
 \begin{enumerate} [noitemsep]
  \item se esiste l'inversa della funzione da derivare;
  \item se entrambe le funzioni sono derivabili;
  \item se  \(f'(g)\ne 0\).
 \end{enumerate}
Infatti, nell'esempio tutto funziona , ma solo se \(x\ne 0\)
(ricorda anche l'esempio \ref{esempio:diff01_deriradice}). 


\begin{inaccessibleblock}
  [differenziale funzione invera]
 \begin{center}
 \begin{minipage}[]{.55\textwidth}
  \diffinversa
 \end{minipage} 
  \hfill
 \begin{minipage}[]{.42\textwidth}
Se valgono tutte le condizioni favorevoli, allora esistono la funzione
\(f\) e la sua inversa \(g=f^{-1}\). La funzione e la sua inversa, se esiste,
hanno grafici simmetrici rispetto alla bisettrice \(y=x\).

Ogni punto \(\punto{x}{f^{-1}(x)}\) sulla curva della funzione inversa ha un
corrispondente \(\punto{y}{f(y)}\) sulla curva \(y=f(x)\), nella simmetria 
rispetto alla bisettrice. Guardiamo come si corrispondono i differenziali:
\(dx\) e \(dy\) di una curva sono invertiti rispetto ai differenziali 
dell'altra.
Quindi le derivate corrispondenti sono reciproche l'una con l'altra.
 \end{minipage}
 \end{center}
\end{inaccessibleblock}
\label{}

\begin{teorema}
 Le derivate di due funzioni \(f\), \(g\), inverse l'una dell'altra, se esistono
 e sono diverse da zero, sono reciproche l'una rispetto all'altra.
\end{teorema}
\noindent Ipotesi: \(y=f(x)\), \(x=g(y)\) \(f\), \(g\) derivabili, con 
\(f'\ne0\), 
\(g'\ne 0\).
\hspace{2cm} Tesi: \(f'(x)=\dfrac{1}{g'(y)}\).
% \begin{proof}
%   Grazie alle proprietà della funzione \(\pst{}\), abbiamo:\\
%   \(f'(x)\cdot g'(y)=\pst{\dfrac{dy}{dx}}\cdot\pst{\dfrac{dx}{dy}}=
%   \pst{\dfrac{dy}{dx}\cdot\dfrac{dx}{dy}}=\pst{1}=1\)\\
%   per cui: \(f'(x)=\dfrac{1}{g'(y)}\).
% \end{proof}
\begin{osservazione}
 Dire che il rapporto differenziale \(frac{dy}{dx}\) è reciproco di 
\(frac{dx}{dy}\) 
 non è banale come dire che la frazione \(frac{3}{4}\) è reciproca di 
\(frac{4}{3}\).
 Una frazione è un rapporto fra numeri e genera un numero, il rapporto 
differenziale è un rapporto fra funzioni e genera una funzione. In più,  
in una frazione come la frazione \(frac{3}{4}\) i numeri \(3\) e \(4\) sono 
indipendenti, invece il differenziale \(dy\) dipende da \(dx\) nel rapporto
\(frac{dy}{dx}\),  e \(dx\) dipende da \(dy\) nel rapporto inverso. 
Questo comporta questioni delicate, simili a quelle descritte a proposito del 
teorema \ref{teo:diff01_dericomp}. 
Per questo bisogna essere precisi nell'enunciato.
\end{osservazione}
\begin{esempio}
  Trova la derivata di \(f(x)=\dfrac{1}{\sqrt{5-x^2}}\).
  \begin{enumerate}[noitemsep]
   \item Usando il teorema \ref{teo:diff01_dericomp} e le regole 
precedenti:\\
   \(f'(x)=\mathit{D}\quadra{\dfrac{1}{\sqrt{5-x}}}=
   \mathit{D}\quadra{(5-x)^\frac{-1}{2}}=-\dfrac{1}{2}(5-x)
   ^\frac{-3}{2}(-1) = \dfrac{1}{2(\sqrt{5-x})^3}\).
   \item Usando la regola appena appresa:\\
   Costruiamo la formula inversa con pochi passaggi algebrici: riavremo la 
   stessa funzione, in cui \(y\) figura come variabile indipendente: 
\(x=f(y)\).\\
   Quindi deriviamo: \(\mathit{D}\quadra{x}=x'=f'(y)=\pst{\dfrac{dx}{dy}}\).\\
   \(f(x)=y=\dfrac{1}{\sqrt{5-x}}\srarrow y^2=\dfrac{1}{5-x}\srarrow 
   y^{-2}=5-x\srarrow x=5-y^{-2}\)  (formula inversa)\\
   \(x'=\pst{\dfrac{dx}{dy}}=-2y^{-3}\) (derivata della funzione inversa)\\
   \(\srarrow y'=\pst{\dfrac{dy}{dx}}=\dfrac{y^3}{2}=
   \dfrac{1}{2(\sqrt{5-x})^3}\).
  \end{enumerate}
% In genere la funzione inversa si costruisce in pochi passaggi semplici, poi 
% la derivazione risulta elementare.
\end{esempio}


\subsection{Derivata, differenza, differenziale, incremento}
\label{subsec:diff01_deridiff}

\begin{inaccessibleblock}
  [differenziale della tangente]
  \begin{minipage}[]{.47\textwidth}
    \begin{center} \derivata \end{center}
 \end{minipage} 
  \hfill
 \begin{minipage}[]{.47\textwidth} \vspace{2.5em}
Nel punto \(\punto{x_0}{f(x_0)}\) il grafico della funzione e la tangente
sono indistinguibili. 
Il campo visivo del primo microscopio mostra \({x_0}\) e \(dx\), 
uno fra gli infiniti infinitesimi nella monade di \(x_0\). A livello 
microscopico la curvatura del grafico non esiste,  per cui il grafico e la 
tangente sono sovrapposti. Per cogliere la distinzione fra i due occorre 
un secondo microscopio non standard, centrato a distanza infinitesima dal 
punto. Nel suo campo visivo la tangente e il grafico della funzione 
appaiono come rette parallele. Nella rappresentazione doppiamente ingrandita
il punto di coordinate reali più vicino a quello raffigurato  si trova a 
distanza infinita (\(\infty^2\)).
 \end{minipage}
\end{inaccessibleblock}
\label{}
La figura mostra che la tangente e la secante per due punti infinitamente 
vicini sono distinguibili solo al dettaglio degli infinitesimi. Lo 
stesso avviene per la derivata e il rapporto differenziale.\\
Dalla definizione di derivata \(f'(x)=\pst{\dfrac{df(x)}{dx}}\) ricaviamo che
\(f'(x)\sim \dfrac{df(x)}{dx}\): la derivata e il rapporto differenziale 
sono quantità quasi, ma non esattamente, uguali. Possiamo esprimere 
meglio questo concetto:\\
\(\dfrac{df(x)}{dx}=f'(x)+\epsilon(x)\) e quindi 
\(df(x)=f'(x)dx+\epsilon(x)dx\).\\
\(\epsilon(x)\) è l'infinitesimo, o l'insieme di infinitesimi, che fa la 
differenza fra la derivata e il rapporto differenziale.  
\(\epsilon(x)dx\), un prodotto fra infinitesimi, forma un infinitesimo 
di ordine superiore rispetto a \(f'(x)dx\).
Nella maggior parte dei casi pratici si tratta di una differenza 
trascurabile 
e si può accettare l'espressione \(f'(x)dx\) al posto dell'espressione
\(df(x)\), che può essere meno comoda da calcolare.\\
Nella storia del calcolo infinitesimale l'uso di una formula al posto 
dell'altra
è diventato normale e molti testi definiscono differenziale della 
funzione il prodotto \(f'(x)dx\), invece della differenza infinitesimale 
\(df(x)\).

Il problema diventa più critico nelle applicazioni pratiche, quando si 
devono 
usare le differenze finite al posto dei differenziali. Si usa allora, per
analogia:\\
\(\Delta f(x)=f'(x_0)\Delta x + \delta(x)\Delta x\).\\
Dato che l'ultimo termine è il meno rilevante, si ha:\\
\(\Delta f(x)\cong f'(x_0)\Delta x \srarrow
f(x)-f(x_0)\cong f'(x_0)(x-x_0)\srarrow f(x)\cong f'(x_0)(x-x_0)+ f(x_0)\).\\
Si tratta dell'usuale equazione della tangente per \(x=x_0\).\\
La formula è esatta solo per le funzioni rappresentate da rette.
Per le altre funzioni la differenza \(\Delta f(x)\) fra due valori della 
funzione 
può essere anche molto diversa da \(f'(x_0)\Delta x\), che è in realtà la 
differenza fra due valori \(y\), calcolati lungo la tangente.

\begin{inaccessibleblock}
  [differenziale della tangente]
  \begin{minipage}[]{.4\textwidth}
    \begin{center} \scalebox{1}{\falsodifferenziale} \end{center}
 \end{minipage} 
  \hfill
 \begin{minipage}[]{.55\textwidth}
 Allontanandosi da \(x_0\) di una quantità finita \(\Delta x\), le differenze 
della funzione \(\Delta f(x)\), calcolate a partire da \(x_0\), possono 
 essere anche molto diverse dalle differenze \(f'(x_0)\Delta x\), calcolate 
lungo la tangente.
 \end{minipage}
\end{inaccessibleblock}
\label{}

Nei testi in cui si scrive che \(\Delta f(x)=f'(x_0)\Delta x+\delta(x)\Delta 
x\), \(f'(x_0)\Delta x\) viene chiamato differenziale, anche se si tratta di 
una 
differenza, una quantità finita, non infinitesima. In tali testi la differenza
\(\Delta f(x)\) è detta incremento e l'equazione\\
\(\Delta f(x)\cong f'(x_0)\Delta x\) \hspace{.5cm}(\emph{Equazione alle 
differenze})\\ 
esprime il cosiddetto \emph{Teorema dell'incremento}.\\
Ai fini pratici l'Equazione alle differenze è un'equazione utile, 
soprattutto 
quando si studiano i fenomeni naturali, perché le variazioni che si 
misurano in
questi ambiti sono differenze finite. 
Ovviamente i risultati che si ottengono utilizzando il teorema 
dell'incremento
saranno tanto più precisi quanto più piccola è la variazione \(\Delta x\), in 
rapporto ai valori \(x\).

\begin {esempio}
Fare una stima ragionevole della quantità \(\sqrt{25,162}\).\\
Si sta usando la funzione \(f(x)=\sqrt{x}\), la cui derivata è: 
\(f'(x)=\dfrac{1}{2\sqrt{x}}\).\\
Utilizziamo il teorema dell'incremento, fissando \(x_0=25\) e \(\Delta 
x=0,162\).\\
\(\Delta f(x)=f'(x_0)\Delta x+\delta(x)\Delta x\cong f'(x_0)\Delta x
\srarrow f(x)\cong f'(x_0)\Delta x+ f(x_0)\)\\
\(f(25,162)\cong f'(25)\cdot 0,162+f(25)\srarrow 
\sqrt{25,162}\cong \dfrac{1}{2\sqrt{25}}\cdot 0,162+ \sqrt{25}=
\dfrac{0,162}{10}+5\cong 5,0162\).\\
Confronta il risultato con quanto propone la calcolatrice.\\
Poiché l'approssimazione è tanto migliore quanto più piccolo è \(\Delta x\),
ripeti l'esercizio con \(x_0=25,1001\) (la cui radice è \(5,01\)) e quindi
\(\Delta x=0,06199\).
\end {esempio}

\pagebreak %----------------------------------------------------

\subsection{Sintesi}
\label{subsec:diff01_derisintesi}
Nel seguente schema riassumiamo le derivate immediate e le regole di 
derivazione:

\begin{minipage}{.48\textwidth}
\begin{center}
\emph{Derivate immediate}
\vspace{.5em}
\begin{tabular}{c|c} 
\(f(x)\) & \(f'(x)\)\\ \hline
\(k\)  & 0 \\   
\(x\) & 1 \\  
\(x^\alpha\) & \(\alpha x^{\alpha-1}\) \\ 
\(e^x\) & \(e^x\)\\    
\(a^x\) & \(a^x\ln a\)\\
\(\ln x\) & \(\dfrac{1}{x}\)\\
\(\log_a x\) & \(\dfrac{1}{x\ln a}\)\\
\(\sen x\) & \(\cos x\) \\ 
\(\cos x\) & \(-\sen x\) \\ 
\(\tg x\) & \(\dfrac{1}{cos^2 x}=\tg^2 x +1\)
\end{tabular}
\end{center}
\end{minipage}
\hfill
\begin{minipage}{.48\textwidth}
\begin{center}
\emph{Regole di derivazione}
\vspace{.5em}
\begin{tabular}{l}
\vspace{.5em}  Regole di derivazione\\\vspace{.5em}
\(\mathit{D}\quadra{f(x)+g(x)}=f'(x)+g'(x)\)\\\vspace{.5em}
\(\mathit{D}\quadra{kf(x)}=kf'(x)\)\\\vspace{.5em}
\(\mathit{D}\quadra{f(x)\cdot g(x)}=f'(x)\cdot g(x)+f(x)\cdot 
g'(x)\)\\\vspace{.5em}
\(\mathit{D}\quadra{\dfrac{1}{f(x)}}=-\dfrac{f'(x)}{f^2(x)}\)\\\vspace{.5em}
\(\mathit{D}\quadra{\dfrac{f(x)}{g(x)}}=\dfrac{f'(x)\cdot g(x)-f(x)
    \cdot g'(x)}{g^2(x)}\)\\\vspace{.5em}
\(\mathit{D}\quadra{f(g(x))=f'(g(x))\cdot g'(x)}\)
\end{tabular}
\end{center}
\end{minipage}

\subsection{Applicazioni non solo matematiche}
Il calcolo della derivata è entrato da protagonista nella descrizione 
matematica dei fenomeni naturali da almeno \(300\) anni e più recentemente
anche nello studio delle scienze umane e sociali. 
Gli esempi che seguono si servono di questo calcolo in due modi:
\begin{enumerate}[noitemsep]
 \item per trovare il tasso di variazione:
data una funzione, si deve cercare quanto rapidamente essa varia rispetto
alla sua variabile;
 \item attraverso l'equazione alle differenze, 
nella forma diretta \(\Delta f(x)\cong f'(x_0)\Delta x\), o nella forma
inversa \(\Delta x \cong \dfrac{\Delta f(x)}{f'(x_0)}\).
\end{enumerate}
L' utilità dell'equazione alle differenze viene dal
fatto che si tratta di un'equazione di primo grado in \(\Delta x\), perché i 
termini infinitesimi di grado superiore sono trascurati. Le soluzioni che 
così si ottengono sono approssimate, ma in genere il grado di imprecisione è 
sopportabile.

\begin{esempio}
 Se lanci verso l'alto una palla alla velocità iniziale \(v=20\) m/s, questa 
viene  frenata dalla forza di gravità e la sua legge del moto risulta 
all'incirca  \(h(t)=20t-5t^2\). Trova a quale altezza \(h\) la palla si ferma.\\
 Risposta. Se la palla si ferma, la sua velocità è nulla, quindi:\\
 \(v(t)=h'(t)= 20-10t=0\srarrow t=2 s \srarrow h(2)=20\cdot 
2-5\cdot  2^2=20 m\).\\
\end{esempio}

\begin{esempio}
L'aereo A parte da Milano a mezzogiorno e vola in direzione Ovest 
mediamente a 
\(800\) km/h, mentre l'aereo B parte due ore dopo e si dirige a Sud a \(800\) 
km/h. Se volano alla stessa quota, con quale velocità si allontanano l'uno 
dall'altro dopo 4 ore?\\
Soluzione. Le due equazioni del moto sono \(s_A=800t\) e \(s_B=800(t-2)\). 
Calcoliamo prima la distanza fra i due, poi la loro velocità relativa.
Si tratta di direzioni perpendicolari e possiamo applicare il teorema di 
Pitagora.\\
\(s_{AB}=\sqrt{s_A^2+s_B^2}=\sqrt{(800t)^2+[(800(t-2)]^2}=
800\sqrt{t^2+t^2-2t+4}=800\sqrt{2t^2-2t+4}\).\\
\(v_{AB}|_{t=4}=\pst{\dfrac{ds_{AB}}{dt}}\bigg|_{t=4}=\dfrac{800(4t-2)}{2\sqrt{
2t
^2-2t+4}}\bigg|_{t=4}\cong 907\) km/h.
\end{esempio}

\begin{esempio}
 Un circuito è percorso da corrente variabile. Infatti la carica che 
attraversa il conduttore ad un certo istante \(t\) è data da \(q(t)=t^3-24t\).
È possibile che in qualche istante le cariche siano ferme?\\
Risposta. Se le cariche sono ferme, la corrente è nulla.\\ 
\(i(t)=q'(t)=3t^2-24=0\srarrow t_1=\sqrt{8}\cong 2,8
\mbox{  e } t_2=-\sqrt{8} \cong -2,8\) s.\\
La corrente è nulla \(2,8\) s prima e dopo l'istante \(t\).
\end{esempio}

\begin{esempio}
 Il biologo Jacques Monod mostrò che lo sviluppo di una colonia di batteri 
 di Escherichia Coli segue una crescita esponenziale, se sufficientemente 
 nutrita. Ogni microrganismo si scinde in due dopo circa \(20\) minuti, per cui 
 la popolazione al tempo \(t\), misurato in ore, conta 
\(N(t)=N_0e^{\frac{t}{3}}\)
 individui. Dopo quante ore il numero di batteri passa da \(10^6\) a \(10^9\)?\\
 Soluzione: \(\Delta N\cong N'(t)\Delta t=\dfrac{N_0}{3}e^{\frac{t}{3}}\Delta t
 \srarrow \Delta t\cong\dfrac{3\Delta N}{N_0e^{\frac{t}{3}}}=\dfrac{3\Delta N}
 {N(t)}\).\\
 Il numero iniziale di batteri è \(10^6=N(0)=N_0e^0\). Perciò:
 \(\Delta t\cong\dfrac{3(10^9-10^6)}{10^9}\), che, calcolato in ore,  
corrisponde 
a 
 3 ore meno \(11\) secondi circa.
 
 \begin{osservazione}
  Dunque, la risposta è che in quasi \(3\) ore il numero di batteri passa da un 
  milione a un miliardo, che è \(1000\) volte tanto. Possiamo pensare che 
  occorra lo stesso tempo per passare da \(1\) individuo a \(1000\), oppure da 
  \(1000\) individui a \(1\) milione?
 \end{osservazione}
 \begin{osservazione}
 Come mai in un caso del genere l'uso delle derivate non è indispensabile? 
 Perché la funzione esponenziale è l'unica funzione che ha per derivata...
\end{osservazione}
\begin{osservazione}
  Si tratta di un problema tipico sulla crescita esponenziale, di quelli già 
  risolti quando ancora non conoscevi l'esistenza delle derivate, riguardanti
  per esempio l'interesse composto o il decadimento radiattivo.
 \end{osservazione}

\end{esempio}
 
\begin{esempio}
Il costo marginale è l'aumento di costo che si ha quando si vuole produrre 
un'unità in più di un certo bene.\\
Supponi che per produrre un certo numero \(n\) di aghi il costo in euro sia
\(y=\sqrt{n}\). Calcola il costo marginale per produrne più di \(10.000\).\\
Soluzione: \(y=\sqrt{n}\srarrow y'=\dfrac{1}{2\sqrt{n}}\).
Se \(n=10000\), \(\Delta y \cong\dfrac{1}{2\sqrt{10000}}\Delta n=\dfrac{\Delta 
n}{200}\).
Il costo marginale, cioè per unità in più, è quindi dello \(0,5\%\).
\begin{osservazione}
 Anche in questo caso concreto, non è possibile pensare che \(\Delta n\) sia 
 un infinitesimo, dato che non ha senso calcolare il costo per frazioni 
 infinitesime di un ago.
\end{osservazione}
\end{esempio}
 
\begin{comment}
\begin{esempio}
 Una barra metallica è lunga \(10\) m a \(T=0 ^\circ C\) e al crescere della 
 temperatura si dilata secondo la legge \(l(T)=10(1+0,000024T)\). Di quanti 
gradi 
 occorre aumentare la temperatura perché aumenti la sua lunghezza di \(5\) 
cm?\\
 La risposta è la stessa in ogni caso, oppure dipende dal valore iniziale 
di 
 \(T\)?\\
 Risposta. \(\Delta T = \dfrac{\Delta l(T)}{l'(T)}=\dfrac{0,05}
 {10\cdot 0,000024}=208,3 ^\circ C\).\\
 Nella formula risolutiva non compare il simbolo \(T\), quindi la risposta 
non 
 dipende dalla temperatura iniziale. Ovviamente tutto questo deve avvenire 
nei 
 limiti del fenomeno, cioè finché non si raggiunge la temperatura di 
fusione.
 \begin{osservazione}
  Si tratta di un semplice esercizio di fisica: la legge coinvolta si chiama
  legge della dilatazione lineare, perché il suo grafico nel piano 
cartesiano è una retta. Poiché la legge è espressa da un polinomio di 
primo grado, la soluzione non contiene la variabile \(T\) e l'equazione alle 
differenze è
  esatta: non ci sono infinitesimi da trascurare.\\
  Non è indispensabile coinvolgere il calcolo infintesimale per un problema 
di
  primo grado come questo: avresti potuto risolverlo anche in terza media.
 \end{osservazione}
\end{esempio}
\end{comment}

