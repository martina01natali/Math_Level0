% (c) 2015 Daniele Zambelli daniele.zambelli@gmail.com
%  Bruno Stecca

\input{\folder differenziazione_grafici.tex}

% -----------------------------------------------
% % TODO Possibile schema:
% - Definizioni
% - - introduzione
% - - pendenza
% - - funzione derivata
% - Teoremi
% - - funzioni elementari algebriche (radice?)
% - - regole di derivazione
% - - funzioni trascendenti
% - Applicazioni
% -----------------------------------------------

\chapter{Analisi della curvatura}

\subsubsection{Applicazioni: cerchio osculatore}
I grafici delle pagine precedenti ci hanno mostrato alcune rette e tante curve. 
Calcolando le derivate abbiamo iniziato ad analizzare la forma delle 
curve-grafici delle funzioni regolari. Anche se più avanti renderemo più 
generali queste nozioni, al momento sappiamo che:
\begin{itemize}[noitemsep, nosep]
 \item con la derivata prima ricaviamo le pendenze di una curva, punto per 
punto;
\item con la derivata seconda mostriamo come variano le pendenze, perciò 
sappiamo dire se in un certo intervallo la curva mostra una concavità (le 
pendenze crescono quando \(x\) aumenta) oppure una convessità (il caso 
contrario).
\end{itemize}
La maggior parte dei grafici ha una forma curvilinea, per questo ora 
concentriamo lo studio sulla curvatura di un grafico. 

Immagina di essere in 
bicicletta lungo la curva, come un un puntino mobile su una mappa del 
navigatore: se c'è una forte curvatura il cambio di direzione è forte in un 
tratto breve, mentre avviene il contrario nei tratti a curvatura lieve. 
Quindi è importante considerare 
\begin{itemize}[nosep, noitemsep]
 \item l'angolo \(d\alpha\) fra due direzioni successive, le direzioni del 
movimento che si hanno percorrendo la curva;
\item la lunghezza \(ds\) del tratto di curva che si considera.
\end{itemize}
 \paragraph{Le direzioni: } sappiamo già esprimere una direzione lungo la 
curva: geometricamente è la pendenza, il coefficiente angolare della tangente 
alla curva nel punto considerato. Conoscendo l'angolo \(\alpha\), la pendenza 
\(f'(x)\) in quel punto è \(f'(x)=\tg(\alpha)\) e quindi 
\(\alpha=\arctan(f'(x))\).\\
Dal teorema dell'incremento si ha:
\[
 d\alpha \sim \Deriv{\arctan(f'(x))}dx=\dfrac{f''(x)}{1+(f'(x))^2}dx
\]

\paragraph{La lunghezza \(ds\):}
il tratto curvo \(ds\) è indistinguibile da un segmento rettilineo lungo la 
tangente, con la componente orizzontale che è \(dx\) e quella verticale 
che è \(dy\sim f'(x)dx\), per cui:
\[
ds=\sqrt{(dx)^2+(dy)^2}\sim\sqrt{(dx)^2+(f'(x)dx)^2}=dx\sqrt{1+(f'(x))^2}
\]
%TODO   disegno tipo p.396, magari in una semicirconf come un grafico già pronto
\begin{center}  disegno\end{center}


Dove il grafico mette in evidenza una forte curvatura, lì abbiamo un 
forte cambio di direzione in un tratto breve. Al contrario dove la curvatura è 
minore, il cambio di direzione è minore oppure avviene lungo un percorso più 
lungo. C'è quindi una proporzionalità inversa fra la curvatura \(k\) e la 
lunghezza \(ds\), a parità di angolo \(d\alpha\), e c'è una proporzionalità 
diretta fra \(k \) e \(d\alpha\), a parità di \(ds\): possiamo scrivere
\(k\propto\frac{d\alpha}{ds}\), quindi 
\[
 k\propto\frac{d\alpha}{ds}=\dfrac{\dfrac{f''(x)}{1+(f'(x))^2}dx}
{\sqrt{1+(f'(x))^2}dx} =
 \dfrac{f''(x)}{\quadra{1+(f'(x))^2}^{\frac{3}{2}}}
\]

Calcolando le derivate prima e seconda possiamo quindi ricavare 
una misura della curvatura di un grafico in ogni suo punto. 
\begin{esempio}
Analizziamo la curvatura del grafico della funzione \(f(x)=\dfrac{1}{x}\), che 
abbiamo già visto molte volte. Limitiamoci all'intervallo \(x>0\). Il grafico 
non ha quasi curvatura agli estremi dell'intervallo mentre ne ha una molto 
accentuata per \(y=x\). Vediamo l'espressione di \(k\), che essendo diversa da 
punto a punto, sarà \(k=k(x)\).
\begin{align*}
 f(x) &= \dfrac{1}{x}\quad \text{ con } x>0 \qquad f'(x)=\dfrac{-1}{x^2}\qquad 
f''(x)=\dfrac{2}{x^3}\\
k(x)&=\dfrac{f''(x)}{\quadra{1+(f'(x))^2}^{\frac{3}{2}}}=
\dfrac{\dfrac{2}{x^3}}{\quadra{1+\tonda{\dfrac{-1}{x^2}}^2}^{\frac{3}{2}}}=
\dfrac{\dfrac{2}{x^3}}{\quadra{1+\dfrac{1}{x^4}}^{\frac{3}{2}}}=
\dfrac{\dfrac{2}{x^3}}{\dfrac{[x^4+1]^\frac{3}{2}}{x^6}}=
\dfrac{2x^2}{[x^4+1]^\frac{3}{2}}
\end{align*}
Per fare dei confronti, prendiamo tre punti sulla curva e confrontiamo le 
curvature risultanti.
\begin{align*}
x_A&=0,5 \quad\longrightarrow \quad k(0,5)=\dfrac{2\cdot 
0,5^2}{\sqrt{(0,5^4+1)^3}}\simeq 0,456\\
x_B&=1 \quad  \longrightarrow \quad k(1)=\dfrac{2\cdot 
1^2}{\sqrt{(1^4+1)^3}}=\dfrac{\sqrt{2}}{2}\simeq 0,707\\
x_B&=4 \quad  \longrightarrow  \quad k(4)=\dfrac{2\cdot 
4^2}{\sqrt{(4^4+1)^3}}=\simeq 0,008
\end{align*}
Come si vede anche dal disegno, la curvatura è più accentuata nel punto di 
simmetria della curva, cioè \(x_B\). Allontanandosi da questo punto sia a 
destra che a sinistra, la curvatura diminuisce rapidamente.
\end{esempio}
\begin{esempio}
 Che curvatura ha una retta, per esempio \(y=3x-1\)\,? La risposta viene 
d'istinto, trattandosi di una retta. Calcoliamo:
\begin{align*}
 f(x) &= 3x-1 \qquad f'(x)=3 \qquad f''(x)=0\\
 k(x)&=\dfrac{f''(x)}{\quadra{1+(f'(x))^2}^{\frac{3}{2}}}=
\dfrac{0}{\quadra{1+3^2}^\frac{3}{2}}=0
 \end{align*}
\end{esempio}
La retta non ha curvatura in alcun punto, è banale. Ovviamente vale per ogni 
retta, anche quelle verticali, anche se in questi casi non possiamo 
applicare la formula.
\begin{esempio}
 Calcola la curvatura della funzione \(f(x)=\sqrt{9-x^2}\). Si tratta della 
semicirconferenza che ha il centro nell'origine e il raggio \(r=3\), nel 
semipiano \(y\geqslant 0\).
\begin{align*}
f(x) &= \sqrt{9-x^2} \qquad f'(x)=\dfrac{-x}{\sqrt{9-x^2}} 
\qquad f''(x)=\dfrac{-9}{\sqrt{(9-x^2)^3}}=-9\tonda{9-x^2}^{-\frac{3}{2}}\\
k(x)&=\dfrac{f''(x)}{\quadra{1+(f'(x))^2}^{\frac{3}{2}}}=
\dfrac{-9\tonda{9-x^2}^{-\frac{3}{2}}}
{\quadra{1+\tonda{\dfrac{-x}{\sqrt{9-x^2}}}^2}^\frac{3}{2}}=
\dfrac{-9\tonda{9-x^2}^{-\frac{3}{2}}}
{\quadra{\dfrac{9-x^2+x^2}{9-x^2}}^\frac{3}{2}}=\\
&=\dfrac{-9\tonda{9-x^2}^{-\frac{3}{2}}}
{9^\frac{3}{2}\tonda{9-x^2}^{-\frac{3}{2}}}=
-9\cdot9^{-\frac{3}{2}}=- 9^{1-\frac{3}{2}}=-9^{-\frac{1}{2}}=-\dfrac{1}{3}
\end{align*}
Da notare tre cose importanti:
\begin{itemize}[noitemsep, nosep]
\item il segno negativo è dato dal numeratore, 
che è una derivata seconda: 
indica che la curva rivolge la sua concavità verso il basso;
 \item \(k\) è costante, non dipende da \(x\), come giustamente deve essere 
per un arco di circonferenza;
\item il numero 3 che risulta al denominatore è proprio il raggio 
della circonferenza.
\end{itemize}
\end{esempio}


\paragraph{Il cerchio osculatore}
C'è un secondo modo, molto espressivo, di indicare la curvatura. Come si vede 
dall'ultimo esempio, se non consideriamo il segno, il risultato è il reciproco 
del raggio della circonferenza. Infatti abbiamo definito la curvatura \(k\) 
come \(k=\dfrac{d\alpha}{ds}\), il rapporto fra il cambio di direzione e la 
lunghezza del tratto percorso.

Nelle circonferenze, se \(\alpha\) si misura in radianti, vale la relazione 
\(ds=rd\alpha\), per cui il raggio è \(r=\dfrac{ds}{d\alpha}=\dfrac{1}{k}\). Da 
qui viene l'idea di esprimere la curvatura di un grafico in un suo punto 
attraverso il raggio del cerchio che in quel punto approssima la curva al 
meglio. Tale cerchio è chiamato \emph{cerchio osculatore}.

Essendo il reciproco di \(k\), il raggio del cerchio osculatore è:
\[
 r=\dfrac{\quadra{1+(f'(x))^2}^{\frac{3}{2}}}{f''(x)}
\]
e il segno che risulterà alla fine del calcolo ci indica se la circonferenza si 
trova sopra o sotto la curva, come puoi verificare negli ultimi esempi.

Non è difficile trovare il centro \(O\) del cerchio osculatore: 
se \(P\) è il punto della curva che stiamo esaminando, il raggio \(r=OP\) è un 
segmento che appartiene alla normale alla curva, passante per \(P\). Il calcolo 
si svolge nel prossimo esempio.

Quanto alla curvatura di una retta, che risulta \(0\) come abbiamo visto, il 
raggio del cerchio osculatore risulterà infinito, come è giusto se si 
immagina una circonferenza che ha il suo centro a distanza infinita dalla 
retta.

\begin{esempio}
 Trovare i cerchi osculatori della parabola \(y=x^2\) nel suo vertice 
\(V(0, 0)\) e nel punto \(P(1, 1)\).
\begin{center} \(f(x)=x^2\qquad f'(x)= 2x \qquad f''(x)=2\)\end{center}
Relativamente al vertice \(V\): \quad
\(r=\dfrac{\quadra{1+(f'(x))^2}^{\frac{3}{2}}}{f''(x)}=
 \dfrac{\quadra{1+(2\cdot 0)^2}^{\frac{3}{2}}}{2}=\dfrac{1}{2}\).\\
Dato che in \(V\) la parabola ha tangente orizzontale, il centro del cerchio 
osculatore si troverà sulla verticale di \(V\), cioè in \(O_V(0, 
\dfrac{1}{2})\).\\
Relativamente al punto \(P\): \quad
\(r=\dfrac{\quadra{1+(f'(x))^2}^{\frac{3}{2}}}{f''(x)}=
 \dfrac{\quadra{1+2^2}^{\frac{3}{2}}}{2}\simeq 5,59\).\\
 Dato che \(\alpha\), l'inclinazione della tangente, è 
\(\alpha=\arctan(f'(1))=\arctan(2)=63,4^\circ\), l'inclinazione della normale 
passante per \(P\) è \(\beta=\alpha+90^\circ=153,4^\circ\).\\
Le coordinate del centro \(O_P\) del cerchio osculatore in \(P\) saranno quindi:
\begin{align*}
x_{O_P} &= x_P+\cos\beta=x_P+\cos 153,4^\circ=1-0,89=0,11\\
y_{O_P} &=y_P+\sen\beta=y_P+\sen 153,4\circ=1+0,45=1,45.
\end{align*}
Da notare, infine, che le due circonferenze si trovano sopra la curva, dato che 
la curvatura \(k\) è positiva perché \(f''(x) > 0\).
\end{esempio}


\begin{comment}
% \affiancati{.50}{.50}{
% \disegno{\polregincirc{5}{6}}
% }{
% \disegno{\polregincirc{5}{12}}
% }
% 
% \agnesia

\section{Premessa}
\footnote{Per scrivere questo capitolo mi sono ispirato 
ai lavori di Giorgio Goldoni ``Il calcolo delle differenze e il calcolo 
differenziale''. 
Chi volesse approfondire l'argomento può acquistare il testo 
all'indirizzo: 
\href{https://www.unilibro.it/libri/f/autore/goldoni\_giorgio}
     {www.unilibro.it/libri/f/autore/goldoni\_giorgio}.}
I numeri iperreali ci permettono di affrontare una nuova classe di problemi 
molto importante: è formata da tutti i problemi legati alla pendenza (e 
quindi alla velocità e all'accelerazione).

I metodi che seguiremo faranno uso anche del \emph{principio di estensione 
naturale}: è possibile estendere agli iperreali tutte le proprietà che 
sono vere nei numeri reali.
Per esempio, il concetto di funzione che abbiamo già definito nei Reali, 
sarà applicato anche agli Iperreali: chiameremo sempre \(f\) anche
la funzione iperreale che ha per argomento un numero iperreale, 
e che ha, nei reali, lo stesso comportamento.

% Per esempio, il concetto di funzione che abbiamo già definito nei Reali, 
% sarà applicato anche agli Iperreali: se la funzione \(f\) ha per 
% argomento un numero iperreale, invece di scrivere \effestar per indicare 
% la funzione iperreale che ha, nei reali, lo stesso comportamento della
% funzione reale \(f\)
% 
% e non scriveremo \(f^*\) al 
% posto 
% di \(f\), per indicare \(f\) estesa, oppure \(f(x^*)\) o simili. 
% 
% Le nuove tecniche di calcolo che presentiamo si basano sulla conoscenza 
% dell'insieme degli Iperreali e sull'applicazione diffusa del 
% \emph{principio di estensione naturale} agli Iperreali delle proprietà 
% utili, che sono vere nell'insieme dei Reali.\\
% Per esempio, il concetto di funzione che abbiamo già definito nei Reali, 
% sarà applicato silenziosamente anche agli Iperreali e non scriveremo 
% \(f^*\) al posto 
% di \(f\), per indicare \(f\) estesa, oppure \(f(x^*)\) o simili. 
% Se ci sarà bisogno, chiariremo i dubbi che possono insorgere in questa 
% estensione o i casi particolari più interessanti.\\
% L'argomento del capitolo è centrato sull'uso degli infinitesimi, quindi 
% dei numeri iperreali. Come si vedrà, 
Il percorso tipico di soluzione di un problema sarà: 
\begin{enumerate}[noitemsep, nosep]
\item definire il problema nell'insieme dei reali;
\item trovare la soluzione mediante un calcolo con numeri iperreali;
\item applicare la funzione \(\pst{}\) alla soluzione, per convertirla in 
numero reale.
\end{enumerate}

\section{Definizioni}
\label{sec:differenziazione_definizioni}
Il problema di determinare la velocità istantanea ci ha portati a conoscere 
i numeri infinitesimi e, attraverso questi, l'insieme dei numeri iperreali.
Ora siamo in grado di cercare la risposta alla domanda rimasta in sospeso: 
come si determina la velocità istantanea?\\
La risposta, che conosciamo nelle forme moderne da più di 400 anni,
propone al nostro studio un nuovo potentissimo strumento di calcolo, 
adatto a risolvere problemi in ogni ambito scientifico: la derivata.

\subsection{Introduzione}
\label{subsec:differenziazione_introduzione}
Nel Settecento fiorirono alcune leggende su Galileo Galilei. Una di queste 
racconta che per dimostrare che i gravi cadono con la stessa velocità, 
gettò dalla Torre di Pisa due sfere di peso diverso, ma di uguali 
dimensioni: i due oggetti raggiunsero il suolo 
contemporaneamente.\\
La Torre di Pisa è alta circa \(56m\) e immaginiamo, per semplificare, 
che la distanza percorsa dai due oggetti sia di \(56m\) (ti lascio 
calcolare il percorso effettivo: tieni presente che al giorno d'oggi 
l'inclinazione della Torre è di \(4,8^\circ\)).\\
Oggi sappiamo che un oggetto in caduta libera nel vuoto, 
partendo da fermo, ha la seguente legge del moto:
\(s=\dfrac{1}{2}gt^2\). Come al solito, \(s\) è lo spazio in metri, 
\(t\) è il tempo in secondi, \(g=9,81 m/s^2\) è l'accelerazione di 
gravità, costante vicino alla superficie terrestre.

Per trovare la velocità media, basta dividere lo spazio percorso per il 
tempo impiegato: ottenendo una velocità di quasi \(17\,m/s\) che 
corrisponde a circa \(60 km/h\). 

% \newpage %------------------------------------------
% 
% Se cerchiamo la velocità media, basta dividere lo spazio percorso per il 
% tempo impiegato:
% \begin{align*}
%  s_{tot} &= 56m\\
%  s=\frac{1}{2}gt^2 \sRarrow t_{tot} &= \sqrt{\frac{2s_{tot}}{g}}=
%  \sqrt{\frac{2\times 56}{9,81}}=3,36 s\\
%  v_m &= \frac{s_{tot}}{t_{tot}}=\frac{56 m}{3,36 s}=16,67 m/s
% \end{align*}
% che corrisponde a circa \(60 km/h\).

\subsubsection{La velocità istantanea}
\label{subsubsec:differenziazione_velocita_istantanea}

Ma questo non ci dice niente riguardo la velocità che l'oggetto ha in un 
determinato istante: ad esempio quale sarà la velocità 
dopo 3 secondi dall'inizio della caduta?

Per rispondere a questa domanda possiamo considerare i due istanti: 
infinitamente vicini \(t_0 = 3 e t_1 = 3 + dt\) dove \(dt\) è un intervallo 
di tempo infinitesimo.

Applicando la legge del moto, possiamo calcolare lo spazio percorso nella 
caduta: 
\[ds(3, dt)=\frac{1}{2}g \cdot \tonda{3+dt}^2-\frac{1}{2}g \cdot 3^2= 
\frac{1}{2}g\tonda{9+6 dt+(dt)^2}-\frac{1}{2}g \cdot 9 = 
3 g\cdot dt+\frac{1}{2} g (dt)^2 =
g \cdot dt \tonda{3+ \frac{1}{2} dt}\] 
Lo spazio percorso in un tempo infinitesimo è ovviamente\dots infinitesimo.
Ma dividendolo per l'intervallo infinitesimo di tempo, otteniamo:
\[v(3, dt)=\frac{ds(3)}{dt}=
  \frac{g \cdot dt \tonda{3+ \frac{1}{2} dt}}{dt}=
  3g+\frac{1}{2}g dt\] 
Che non è un valore infinitesimo. 
Il risultato ottenuto, dipende dall'accelerazione \(g\) che per i nostri 
scopi possiamo considerare costante, e da \(dt\) che invece può variare. Per 
ogni valore di \(dt\) avremo una velocità diversa.
La cosa interessante è che tutti questi diversi valori appartengono alla 
galassia del numero \(3g\) e quindi hanno la stessa parte standard:
\[\pst{3g+\frac{1}{2}g dt} = 3g = 29,43m/s\]
% In pratica il calcolo precedente ci fornisce un insieme infinito di numeri 
% iperreali, ma un solo numero reale (anzi nei casi pratici noi siamo 
% interessati a numeri razionali).
% Dividendo il tutto per \(dt\) si ottiene la velocità istantanea, 
% che in questo caso cambia istante dopo istante: 
% \[v(t)=\frac{ds}{dt}=\frac{gtdt+\frac{1}{2}dt^2}{dt}=gt+\frac{1}{2}dt\] 
% L'espressione \(gt+\frac{1}{2}dt=9,81t+\frac{1}{2}dt\) diventa un numero 
% ben preciso per ogni valore di \(t\) e per ogni valore \(dt\), un iperreale
% finito che è la somma di un numero standard e di un numero infinitesimo.

% Non è facile definire con precisione un istante di tempo: possiamo 
% immaginare che sia un tempo assolutamente breve, un tempo di durata 
% infinitesima.\\ 
% Considerando valori di tempo istantanei siamo portati naturalmente a 
% passare da valori \(t_1, \ t_2, \ \dots\), nell'insieme dei reali, a valori 
% \(t_1, \ t_1+dt, \ t_1+2dt,\ \dots\ t_2, \ \dots\) nell'insieme degli
% iperreali. 
% È quindi il momento di usare le quantità infinitesime.\\ 
% Chiamiamo \(dt\) 
% % (in questo caso sarà \(dt > 0\)) 
% un intervallo di tempo infinitesimo, fra due istanti successivi \(t\) e 
% \(t+dt\). 
% Lo spazio percorso nella caduta, in quell'intervallo di tempo, 
% applicando la legge del moto, sarà: 
% \[ds=\frac{1}{2}g\tonda{t+dt}^2-\frac{1}{2}gt^2= 
% \frac{1}{2}g\tonda{t^2+2t\cdot 
% dt+(dt)^2}-\frac{1}{2}gt^2= gt\cdot dt+\frac{1}{2}(dt)^2\] 
% Dividendo il tutto per \(dt\) si ottiene la velocità istantanea, 
% che in questo caso cambia istante dopo istante: 
% \[v(t)=\frac{ds}{dt}=\frac{gtdt+\frac{1}{2}dt^2}{dt}=gt+\frac{1}{2}dt\] 
% L'espressione \(gt+\frac{1}{2}dt=9,81t+\frac{1}{2}dt\) diventa un numero 
% ben preciso per ogni valore di \(t\) e per ogni valore \(dt\), un iperreale
% finito che è la somma di un numero standard e di un numero infinitesimo.
% 
% Se fissiamo l'attenzione su un certo istante, supponiamo 
% \(\overline{t}= 3s\), la velocità \(v(3)=9,81\cdot 3+\frac{1}{2}dt\) 
% non ha comunque un valore completamente fissato 
% perché dipende ancora dalla durata infinitesima \(dt\). 
% I \(dt\) possibili sono infiniti, ma questo non cambia granché nel 
% risultato, che si attesta sostanzialmente attorno al valore 
% \(9,81\cdot 3=29,43m/s\).\\
% Conosciamo già la funzione che ci consente di ottenere la parte sostanziale 
% di un risultato iperreale, cioè di trascurare la sua variabilità 
% residua dovuta agli infinitesimi: si tratta di applicare la funzione  
% la parte standard.
% \[ \pst{9,81t+\frac{1}{2}dt}=9,81t.\] 
% Questa è la velocità istantanea che cerchiamo: dipende unicamente dal tempo 
% \(t\), dai secondi che passano a partire dall'istante del lancio.

\noindent \begin{minipage}{0.35\textwidth}
Ripetendo i calcoli per diversi valori di \(t\):
 \begin{center}
\begin{tabular}{ccc}\toprule
\(t (s)\)      & \(s (m)\) & \(v(m/s)\)  \\\midrule
\(0\)      & \(0\)                 & \(0\)  \\
\(1\)      & \(4,41\)              & \(9,81\) \\
\(2\)      & \(19,62\)             & \(19,62\) \\
\(3\)      & \dots                 & \dots \\
\(t\)      & \(s=\frac{1}{2}gt^2\) & \(v=9,81 \cdot t\) \\\bottomrule
\end{tabular}
\label{tab:diff_velocita}
\end{center}
 \end{minipage}
  \hfill
 \begin{minipage}{.63 \textwidth}
\affiancati{.49}{.49}{
\begin{inaccessibleblock}[Grafico tempo-spazio della caduta libera]
\begin{center} \scalebox{1}{\tempospaziocaduta} \end{center}
\end{inaccessibleblock}
\label{graf:tempospaziocaduta}
}{
\begin{inaccessibleblock}[Grafico tempo-velocità della caduta libera]
\begin{center} \scalebox{1}{\tempovelocitacaduta} \end{center}
\end{inaccessibleblock}
\label{graf:tempovelocitacaduta}
}
 \end{minipage}

Il grafico visualizza la legge oraria del 
moto di caduta libera sotto due aspetti diversi: il ramo di parabola 
(tempo/spazio) mostra che intervalli di tempo uguali a partire da istanti 
diversi corrispondono a spazi progressivamente sempre maggiori. 

Il grafico (tempo-velocità) mostra che la progressione della velocità 
è lineare: man mano che il tempo scorre la velocità cresce 
proporzionalmente, come già si vede dalla tabella.
La formula \(v=9,81\times t\) ci permette il calcolo della velocità per 
ogni valore reale di \(t\) e questa è la risposta alla domanda iniziale. 

\subsubsection{Il problema della pendenza}
\label{subsec:differenziazione_pendenza}
Per studiare il movimento, Galileo usava piani inclinati. 
Una sferetta rotola lungo un piano inclinato con accelerazione diversa a 
seconda dell'inclinazione del piano. 
Minore è la pendenza del piano, minore è l'accelerazione costante, 
minore è la velocità che la sferetta sviluppa a partire da ferma. 

La pendenza del piano e la velocità che si sviluppa, da zero al massimo, 
sono collegate. 
Se il piano è orizzontale, la sferetta ferma resta ferma e la velocità è 
\(v=0\), come la pendenza. 
Se il piano è verticale, la caduta è libera e la velocità progredisce 
come abbiamo visto nella tabella precedente. 
Ma cos'è la pendenza? 
\begin{definizione}
Il termine \emph{pendenza} indica quanto varia l'altezza rispetto allo 
spostamento orizzontale. 
\end{definizione}

% La pendenza media di una funzione in un punto \(x_0\) è data da:
% \[\text{pendenza}media = \frac{f(x_1) - f(x_0)}{x_1 - x_0}\]
% Cioè il rapporto tra la variazione della funzione e la variazione delicate 
% suo argomento.
Per variazione intendiamo un cambiamento di valore, da un valore iniziale a 
un valore finale. 
Molto spesso invece di variazione si usa il termine \emph{incremento}, o 
differenza.

\paragraph{Il Rapporto Incrementale e la pendenza media}
\label{paragraph:differenziazione_ri}
\begin{esempio}
Misurando le temperature in una recente notte invernale, abbiamo registrato:

\noindent \begin{minipage}{0.48\textwidth}
 \begin{center}
\begin{tabular}{ccccc}\toprule
h (ore) & \(T (^\circ \! C)\) & \(\Delta T\) & \(\Delta h\)
             & \(\frac{\Delta T}{\Delta h}\)\\\midrule
\(21\) & \(3\) & -- & -- & -- \\
\(22\) & \(0\) & \(-3\) & \(1\) &\(-3\) \\
\(23\) & \(0\) & 0 & \(1\) & 0 \\
\(24\) & \(-1\) & \(-1\) & \(1\) & \(-1\) \\
\(1\) & \(-3\) &\dots &\dots & \dots\\
\(3\)  & \(-4\) & \(-1\) & \(2\) &\(-0,5\)\\
\(4\)  & \(-5\) &\dots &\dots & \dots\\
\(5\)  & \(-5\) & \(0\)  & \(1\) &\(0\)\\
\(7\)  & \(-0\) & \(0\) & \(2\) & \(2,5\)
\\ \bottomrule
\end{tabular}
\label{tab:temperaturea}
\end{center}
\end{minipage}
 \hfill
\begin{minipage}{.48 \textwidth}
\begin{inaccessibleblock}[Grafico delle temperature per punti distanziati]
\begin{center} \scalebox{.8}{\temperaturea} \end{center}
\end{inaccessibleblock}
\label{graf:temperaturea}
\end{minipage}

È facile calcolare le variazioni \(\Delta T\), ora per ora. 
Alcune di queste sono riportate nella tabella precedente: 
completa le righe della tabella dove appaiono i puntini.

Fra le 22 e le 23 non c'è variazione: l'incremento è zero. \\
Alle 23 la temperatura inizia a calare e alle 24 è calata di 
\(1^\circ C\), quindi l'incremento \( \Delta T=-1\). \\
Fra le \(5\) e le \(7\) l' incremento \(\Delta T= 5 ^\circ C\). 
Ma stavolta la differenza è su un intervallo di \(2\) ore, 
quindi una variazione media di \(2,5 ^\circ C\) all'ora.

Mettendo in grafico i dati di questa funzione empirica, ne risulta una 
spezzata. 
L'inclinazione dei vari segmenti si può calcolare direttamente sul 
grafico con la formula del coefficiente angolare vista in geometria 
analitica e corrisponde ai risultati dell'ultima colonna, 
che in effetti forniscono le pendenze dei vari segmenti. 
Vi sono pendenze positive, negative e nulle, in 
corrispondenza di segmenti crescenti, decrescenti o orizzontali. 
\end{esempio}

% TODO: disegno di pendenza costante e pendenze variabili

Useremo il termine \emph{pendenza} in un suo significato più generale:
\begin{definizione}
La \emph{pendenza} è la rapidità con cui varia una certa funzione reale, 
rispetto alla variazione del suo argomento. 
\end{definizione}

% \(\dfrac{\Delta T}{\Delta h}\), il 
La pendenza media di una funzione in un intervallo è il 
rapporto fra gli incrementi della funzione e gli incrementi del suo argomento. 
% Questo rapporto si chiama Rapporto Incrementale.

L'incremento della funzione e quindi anche il Rapporto Incrementale 
dipendono: 
\begin{multicols}{3}
\begin{itemize} [nosep]
\item dalla funzione \(f\), 
\item dal punto \(x_0\), 
\item dall'incremento \(\Delta\).
\end{itemize}
\end{multicols}
\begin{definizione}
Il Rapporto Incrementale (\(RI\)) 
di una funzione \(f\) 
in un dato punto \(x_0\)
per un certo incremento \(\Delta\)
è il rapporto 
fra l'incremento di \(f\) e l'incremento di \(x\), quando \(x\) 
cambia valore da \(x_0\) a \(x_0+\Delta\).
\[RI(f,~x_0,~\Delta) = 
  \frac{f(x_0+\Delta)-f(x_0)}{(x_0+\Delta)- x_0}=
  \frac{\Delta f(x_0)}{\Delta}\]
\end{definizione}
% Di conseguenza anche il Rapporto Incrementale dipende da quei tre 
% parametri: \(RI(f,~x_0,~\Delta)\).

\subsection{Il Rapporto Differenziale e la pendenza in un punto}
\label{subsubsec:RD}

Immaginiamo ora di avere esigenze scientifiche molto raffinate e di 
predisporre un apparato che misura le temperature ogni secondo. 
I punti del grafico sarebbero così vicini che per distinguerli dovremmo 
usare una lente.
Potremmo anche tracciare il grafico della temperatura con un pennino che si 
sposta su un rullo di carta tracciando una linea continua.
% Allora il grafico appena visto avrebbe spigoli meno visibili e i vari 
% incrementi \(\Delta T\) sarebbero numeri vicini allo zero. 
% Potremmo conoscere il fenomeno con grande precisione e individuare molti 
% più dettagli, se potessimo visualizzare le diverse pendenze.
% 
% % \begin{minipage}{0.48\textwidth}
% % % \begin{center}
% % Immaginiamo ora di avere esigenze scientifiche molto raffinate e di 
% % predisporre un apparato che misura le temperature ogni secondo. 
% % Allora il grafico appena visto avrebbe spigoli meno visibili e i vari 
% % incrementi \(\Delta T\) sarebbero numeri vicini allo zero. 
% % Potremmo conoscere il fenomeno con grande precisione e individuare molti 
% % più dettagli, se potessimo visualizzare le diverse pendenze.
% % %\end{center}
% % \end{minipage}
% %  \hfill
% % \begin{minipage}{.48 \textwidth}
% %  \begin{center}
% % \scalebox{1}{ \temperatureb}
% % \label{graf:temperatureb}
% %  \end{center}
% % \label{graf:temperature}
% % \end{minipage}
% 
% Se abbiamo un pennino che traccia in ogni istante la temperatura otterremo 
% una linea continua.

\affiancati{.50}{.50}{
%grafico temperature smussato
\begin{inaccessibleblock}[Grafico delle temperature con punti molto vicini]
\begin{center} \scalebox{.8}{\temperatureb} \end{center}
\end{inaccessibleblock}
\label{graf:temperatureb}
}{
%grafico temperature smussato
\begin{inaccessibleblock}[Grafico continuo delle temperature]
\begin{center} \scalebox{.8}{\temperaturec} \end{center}
\end{inaccessibleblock}
\label{graf:temperaturec}
}

In una situazione ideale, non legata a strumenti di misura necessariamente 
imperfetti, consideriamo una funzione matematica e il suo grafico e andiamo 
alla ricerca della pendenza della funzione. 

Ora siamo interessati alla pendenza della funzione in un punto, non alla 
pendenza media in un intervallo, per questo opereremo nell'insieme degli 
iperreali (\(\IR\)), 
considerando incrementi infinitesimi. 
Per ragioni storiche questi incrementi infinitesimi si chiamano 
\emph{differenziali}.
\begin{definizione}
Il differenziale \(df\) (\emph{de effe}) di una funzione \(f\) è 
l'incremento infinitesimo, se esiste, 
che la funzione subisce a causa di una variazione infinitesima 
\(\epsilon\) diversa da zero, della sua variabile indipendente \(x\), 
a partire da un valore fissato \(x_0\):
\[df(x_0) = f \tonda{x_0 + \epsilon} - f(x_0)\]
\end{definizione}

% \newpage %------------------------------------------

% \pagebreak[4] %------------------------------------------

\begin{osservazione}
Per segnalare che il simbolo \(df\) (oppure \(dx\), ecc.) non è il prodotto 
fra due variabili 
(cioè \(df \ne d \cdot f,\ dx \ne d\cdot x,\ \dots\)) ma indica 
una differenza infinitesima, nella lettura si pronunciano:
\textit{de effe, de ics}.
\end{osservazione}

% Anche il differenziale \(df\) e il Rapporto Differenziale dipendono: 
% \begin{multicols}{3}
% \begin{itemize} [nosep]
% \item dalla funzione \(f\), 
% \item dal punto \(x_0\), 
% \item dall'incremento \(\epsilon\).
% \end{itemize}
% \end{multicols}
% Quindi il differenziale di una funzione \(df\) è esso stesso una funzione 
% con \(3\) argomenti:
% \(d(f,~x_0,~\epsilon)\). 
\noindent \begin{minipage}{.48 \textwidth}
\begin{esempio}
Vogliamo calcolare l'incremento della 
funzione:~\(f(x) = \dfrac{1}{4} x^2 -x -3\)
quando \(x\) parte da~\(7\) e ha un incremento di~\(\epsilon\).
\begin{align*}
  df(7) &= f(7+\epsilon) - f(7) = \\
        &= \dfrac{\tonda{7 +\epsilon}^2}{4}  -\tonda{7 +\epsilon} -3 - 
           \dfrac{7^2}{4}  +7 +3 =\\
        &= \dfrac{49 +14 \epsilon +\epsilon^2}{4} -10 -\epsilon - 
           \dfrac{49}{4} +10 =\\
        &= \dfrac{14 \epsilon +\epsilon^2}{4} -\epsilon 
        = \dfrac{10 \epsilon +\epsilon^2}{4} =\\
        &= 2,5 \epsilon + \dfrac{\epsilon^2}{4}
%         \quad \forall \epsilon 
%\sim 2,5 \epsilon
\end{align*}
\end{esempio}
\end{minipage}
 \hfill
\begin{minipage}{.48 \textwidth}
 \begin{center}
\differenziale
 \end{center}
\end{minipage}
L'incremento della funzione è un differenziale: un infinitesimo che è la 
somma di due infinitesimi di ordine diverso. 
% Il risultato non cambia 
% sostanzialmente, se al posto di \(\epsilon\) svolgiamo lo stesso calcolo 
% utilizzando un diverso infinitesimo \(\delta\). Per significare che il 
% risultato non dipende dalla scelta dell'infinitesimo, si indica: \(\forall 
% \epsilon\).

% Nell'espressione della funzione riconosciamo che si tratta di una 
% parabola, 
% una funzione il cui grafico ha pendenze sempre diverse, che aumentano (o 
% diminuiscono) gradualmente, allontanandosi dal vertice. Seguendo lo schema 
% usato per il grafico delle temperature, cerchiamo la pendenza per \(x=7\) 
% calcolando il rapporto fra gli incrementi, cioè in questo caso, fra i 
% differenziali.

La pendenza della funzione \(f\) nel punto \(x_0\) è legata al 
rapporto tra l'incremento della funzione e l'incremento della variabile 
\(x\) quando questo incremento è infinitesimo. 
Questo rapporto viene detto \emph{Rapporto Differenziale}.

Anche il differenziale \(df\) e il Rapporto Differenziale dipendono: 
\begin{multicols}{3}
\begin{itemize} [nosep]
\item dalla funzione \(f\), 
\item dal punto \(x_0\), 
\item dall'incremento \(\epsilon\).
\end{itemize}
\end{multicols}
% Quindi il differenziale di una funzione \(df\) è esso stesso una funzione 
% con \(3\) argomenti: \(d(f,~x_0,~\epsilon)\). 
\begin{definizione}
Il \emph{Rapporto Differenziale} (\(RD\)) di una funzione \(f\) è il 
quoziente fra l'incremento infinitesimo \(df\) della funzione 
e l'incremento infinitesimo di \(x\), quando \(x\) cambia valore da 
\(x_0\) a \(x_0+\epsilon\) con \(\epsilon \ne 0\).
\[RD(f, x_0, \epsilon) = 
\frac{f(x_0+\epsilon)-f(x_0)}{(x_0+\epsilon)- x_0} =
\frac{df(x_0)}{\epsilon}\]
\end{definizione}
La differenza tra Rapporto Differenziale e Rapporto Incrementale sta nel 
fatto che nel \(RI\) l'incremento è un numero non infinitesimo mentre 
nel \(RD\) l'incremento è un numero iperreale infinitesimo.

Continuando l'esempio precedente calcoliamo \(RD(f,~7,~\epsilon)\):
\[RD(f,~7,~\epsilon) =
%  \frac{df(7)}{\epsilon}=
 \frac{f(7+\epsilon) - f(7)}{\epsilon}=\dots=
 \frac{2,5 \epsilon + \dfrac{\epsilon^2}{4}}{\epsilon}=
 \frac{\cancel{\epsilon}\tonda{2,5 + \dfrac{\epsilon}{4}}}
 {\cancel{\epsilon}}=
 2,5 + \frac{\epsilon}{4}
%  , \quad \forall \epsilon \ne 0
\]

Ma il Rapporto Differenziale, data una funzione e un punto, dipende 
dall'incremento \(\epsilon\) e quindi cambia al cambiare dell'infinitesimo. 
E questo è un problema. 

Non solo, il Rapporto Differenziale è un numero iperreale, mentre noi 
abbiamo bisogno che la pendenza di una funzione reale sia un numero reale.

Si può risolvere il secondo problema utilizzando la funzione 
\emph{parte standard} che trasforma un iperreale finito nel più vicino
numero reale.

Se il numero reale così ottenuto non dipende dal valore dell'incremento 
infinitesimo, abbiamo ottenuto la pendenza della funzione in quel punto.

\subsubsection{Definizione di pendenza}
% Le pendenze del grafico di temperature sono numeri reali e  con questi è 
% facile stabilire se una temperatura è in crescita di più o di meno 
% rispetto ad un'altra.
% Ora invece abbiamo un risultato parzialmente incerto, perché la 
% pendenza risulta un numero finito iperreale, la somma di uno standard e di 
% un infinitesimo. Si tratta di un numero non completamente esprimibile in 
% cifre, perché la sua parte infinitesima ha un valore sconosciuto.
% Per fortuna, però, questa parte infinitesima è trascurabile e si può 
% ragionevolmente concludere che, anche considerando gli infiniti possibili 
% valori dell'infinitesimo \(\dfrac{\epsilon}{4}\), tuttavia il risultato è 
% sostanzialmente uguale a \(2,5\).
% 
% Tradotto in termini matematici, questo ragionamento corrisponde al calcolo 
% della parte standard.
% \[ \text{pendenza}=\pst{RD}=\pst{2,5 + \frac{\epsilon}{4}}=2,5\]
Possiamo tradurre quanto appena detto nella seguente definizione: 
\begin{definizione}[Pendenza in un punto]
La pendenza di una funzione \(f\), in un punto \(x_0\) del suo dominio, 
è la parte standard del Rapporto Differenziale di \(f\) calcolato in 
\(x_0\), se questa esiste ed è sempre la stessa qualsiasi sia 
l'incremento infinitesimo \(\epsilon\) diverso da zero.
\end{definizione}

La pendenza della funzione \(f\) nel punto \(x_0\) viene indicata con 
\(f'(x_0)\) e viene anche detta \emph{derivata della funzione 
\(f\) nel punto \(x_0\)}:
\[\text{pendenza}(f,~x_0) = f'(x_0)\]

\begin{osservazione}
Mentre il Rapporto Differenziale dipende dal valore dell'infinitesimo 
\(\epsilon\), la pendenza, se esiste, non dipende da \(\epsilon\).
\end{osservazione}

\begin{esempio}
Calcoliamo per la stessa parabola precedente la pendenza nel vertice. 
Sappiamo già che dovrà risultare zero.\\
\begin{align*}
 &f(x) = \dfrac{1}{4} x^2 -x -3 \quad x_0=2;\\
 &RD=\frac{df(2)}{\epsilon}=\frac{f(2+\epsilon) - f(2)}{\epsilon}=
  \frac{\tonda{2 +\epsilon}^2}{4}  -\tonda{2 +\epsilon} -3 - 
           \frac{2^2}{4}  +2 +3 =\\
  &= \dfrac{4 +4 \epsilon +\epsilon^2}{4} -5 -\epsilon - 1 +5 =
 +1+ \epsilon +\frac{\epsilon^2}{4} -5 -\epsilon - 1 +5 =
 \frac{\epsilon^2}{4}
  \end{align*}
Dato che il rapporto differenziale esiste e è finito per qualsiasi 
infinitesimo non nullo \(\epsilon\), calcoliamo la sua parte standard.
\[\text{pendenza}(f,~2) = f'(2) = 
  \pst{RD(f, 2, \epsilon)} = \pst{\frac{\epsilon^2}{4}} = 
  \frac{\pst{\epsilon^2}}{\pst{4}} = \frac{0}{4} = 0\]
Il risultato è sempre lo stesso per ogni \(\epsilon \ne 0\), abbiamo 
perciò la conferma: la pendenza in \(x=2\) è uguale a zero.
\end{esempio}

Per visualizzare la pendenza di una curva in un suo punto 
possiamo disegnare la retta che passa per quel punto e che ha per 
coefficiente angolare proprio quella pendenza. 
Questa retta è chiamata tangente.

\begin{definizione}
Chiamiamo \emph{tangente} ad una funzione \(f\) in un suo punto \(P\), 
la retta passante per \(P\) che ha per coefficiente angolare la 
pendenza della funzione in quel punto.
\[y = f'(x_P) \tonda{x -x_P} +f(x_P)\]
\end{definizione}

\subsubsection{Controesempi}
\label{subsubsec:controesempi}

Prima di procedere con lo studio della pendenza delle funzioni, vediamo 
i casi in cui non si può applicare la definizione di pendenza:
\begin{enumerate} [nosep]
\item se la funzione non è definita per tutti i valori dell'infinitesimo 
\(\epsilon \ne 0\);
\item se non è possibile calcolare la parte standard;
\item se la parte standard cambia al cambiare di \(\epsilon\).
\end{enumerate}

\paragraph{1.~Funzione non definita per ogni \(\epsilon\)}

Per poter parlare di pendenza in  \(x_0\), la funzione deve essere 
definita in tutta la monade di \(x_0\).
Un esempio può chiarire la situazione.

\begin{esempio}
Calcola la pendenza in \(x_0 = 0\) della funzione \(f(x) = \sqrt{x^3}+x\)

\affiancati{.59}{39}{
Il differenziale della funzione in zero:
\begin{align*}
df(0) &= f(0+\epsilon) -f(0) = f(\epsilon) -f(0) = \\
      &= \sqrt{\epsilon^3} + \epsilon - \sqrt{0} - 0 = 
         \sqrt{\epsilon^3} + \epsilon
\end{align*}
Ora questa funzione non è definita per ogni valore di \(\epsilon\) ma 
solo per \(\epsilon \geqslant 0\).
Quindi la funzione, in zero, non ha una pendenza.\\
Possiamo però accontentarci di un risultato meno forte di quello di 
pendenza. 
Dato che per ogni \(\epsilon \geqslant 0\) il differenziale è definito
Possiamo calcolare il rapporto differenziale solo per valori non negativi 
di \(\epsilon\):
\[RD = \dfrac{\sqrt{\epsilon^3} + \epsilon}{\epsilon} =
       \dfrac{\cancel{\epsilon} \tonda{\sqrt{\epsilon} + 1}}
             {\cancel{\epsilon}} = \sqrt{\epsilon} + 1
\]
}{
\begin{inaccessibleblock}
  [Grafico di una funzione definita solo per x>=0]
  \pendenzadestra
\end{inaccessibleblock}
}
E dato che questo valore è senz'altro finito, possiamo applicare la 
funzione parte standard che ci fornisce un risultato indipendente 
dall'infinitesimo \(\epsilon \geqslant 0\):
\[\text{pendenza destra} = \pst{\sqrt{\epsilon} + 1} = 1\]
In conclusione possiamo affermare che la funzione \(f(x) = \sqrt{x^3}+x\) 
in \(0\) non ha una pendenza ma solo la pendenza destra che è \(1\).
\end{esempio}

\paragraph{2.~Non esiste la parte standard del Rapporto Diffderenziale}

Per poter calcolare la parte standard di un numero iperreale, questo 
deve essere finito. 
Vediamo un esempio in cui il Rapporto Differenziale è infinito.

\begin{esempio}
% Calcola la pendenza in \(x_0 = 0\) della funzione \(f(x)=\sqrt[3]{3x}\)
Calcola la pendenza in \(x_0 = 0\) della funzione \(f(x)=\sqrt[3]{x+1}\)

\affiancati{.39}{.59}{
\begin{inaccessibleblock}
  [Grafico di una funzione con un punto a pendenza verticale]
  \rdinfinito
\end{inaccessibleblock}
}{
\begin{align*}
RD(f,~-1,~\epsilon) &= 
   \frac{\sqrt[3]{(-1+\epsilon)+1}-\sqrt[3]{-1+1}}{\epsilon}=\\
&= \frac{\sqrt[3]{\epsilon}}{\epsilon}=
   \frac{\cancel{\sqrt[3]{\epsilon}}}
        {\tonda{\sqrt[3]{\epsilon}}^{\cancel{3}^2}}=
   \frac{1}{\sqrt[3]{\epsilon^2}}
\end{align*}
Qualsiasi sia l'infinitesimo \(\epsilon \neq 0\), \(RD\) è un numero
infinito, non ha dunque parte standard perciò la funzione \(f(x)\) non 
ha pendenza in \(0\).

Nonostante questo possiamo vedere che ha una tangente: 
è la retta \(x=-1\).
}
\end{esempio}

\paragraph{3.~La parte standard cambia al cambiare di \(\epsilon\)}

La definizione non produce un risultato quando al variare 
dell'infinitesimo varia anche la parte standard del 
Rapporto Differenziale, vediamo un semplice esempio.
\begin{esempio}
\label{esempio:differenziazione_derimodulo}
Calcola la pendenza in \(x_0 = 0\) della funzione \(f(x) = \sqrt{3x^2}\)

Innanzitutto osserviamo che la funzione può essere scritta anche in un altro 
modo: 
\[f(x) = \sqrt{3x^2} = \sqrt{3} \abs{x}\]
\affiancati{.59}{.39}{
% Funzione: 
% \(f(x)= \sqrt{3} \abs{x} =\begin{cases}
%   -\sqrt{3} x \quad & \text{ per } x < 0\\
%   +\sqrt{3} x \quad & \mbox{ per } x \ge 0
% \end{cases}\)

\vspace{.5em}
Differenziale per \(x=0\):

\vspace{.5em}
\(df(0) = \abs{\sqrt{3} \tonda{0+\epsilon}} - \abs{\sqrt{3} \cdot 0} = 
          \sqrt{3} \abs{\epsilon}\);

\vspace{.5em}
Rapporto Differenziale:

\vspace{.5em}
\(RD(f,~0,~\epsilon) =
\dfrac{df(0)}{\epsilon}=\dfrac{\sqrt{3} \abs{\epsilon}}{\epsilon}=
\begin{cases}
 -\sqrt{3} \quad &\text{ per } \epsilon < 0\\
 +\sqrt{3} \quad &\mbox{ per } \epsilon > 0
\end{cases}\)
}{
\begin{inaccessibleblock}
  [Grafico di una funzione a V]
  \derivamodulouno
\end{inaccessibleblock}
}
% Il grafico di \(f(x) = \sqrt{3x^2}\) è formato da due semirette che si 
% uniscono nell'origine. 
% Per \(x < 0\) la semiretta ha coefficiente angolare \(m = -\sqrt{3}\) 
% per \(x > 0\) la semiretta ha coefficiente angolare \(m = +\sqrt{3}\). 

% Qual è la pendenza giusta per \(x=0\), nel punto cioè dove il grafico cambia 
% pendenza all'improvviso?

A seconda del segno di \(\epsilon\), la parte standard del Rapporto 
Differenziale può essere \(-\sqrt{3}\) o \(+\sqrt{3}\) e dato che non è la 
stessa per qualsiasi \(\epsilon \ne 0\) la pendenza di questa funzione non è 
definita in \(0\).

% Possiamo però accontentarci di un risultato meno forte di quello di 
% pendenza. Possiamo dire che: 
% \begin{itemize}
% \item per ogni \(\epsilon < 0\), cioè a sinistra di \(0\), la funzione ha 
% pendenza \(-\sqrt{3}\);
% \item per ogni \(\epsilon > 0\), cioè a destra di \(0\), la funzione ha 
% pendenza \(+\sqrt{3}\).
% \end{itemize}
% Il punto \(0\) è anche detto \emph{punto angoloso} della funzione. 
\end{esempio}

Nell'ultimo esempio la funzione presenta un cambio improvviso di pendenza:
Immediatamente a sinistra di \(x_0\) la funzione ha una pendenza diversa 
da quella che ha immediatamente a destra. 
Diremo che la funzione ha un \emph{punto angoloso} in \(x_0\).

\begin{definizione}[Punto angoloso]
Diremo che una funzione ha un punto angoloso se: 
\begin{itemize} [nosep]
\item 
la parte standard del \(RD\) calcolato con 
\(\epsilon < 0\) esiste ed è sempre la stessa,
\item 
la parte standard del \(RD\) calcolato con 
\(\epsilon > 0\) esiste ed è sempre la stessa,
\end{itemize}
e queste due parti standard sono diverse.
\end{definizione}

\begin{osservazione}

Nell'esempio precedente:
\begin{itemize} [nosep]
\item \(f\) è definita in tutto \(\R\),
\item \(f\) è continua in tutto \(\R\),
\item \(f\) non è derivabile in tutto \(\R\).
\end{itemize}
Perché una funzione sia derivabile in un intervallo, 
deve essere continua in quell'intervallo, 
ma non è detto che se è continua sia anche derivabile.
\end{osservazione}

\subsubsection{In conclusione:} 
La pendenza è un numero reale che consente di capire con quale 
velocità variano i dati di un fenomeno in un certo momento oppure quanto sia 
inclinato il grafico di una funzione in un certo punto.
Chiamiamo \(f'(x_0)\) la pendenza della funzione \(f\) nel punto \(x_0\).

\begin{procedura}
\label{proc:ricetta_pendenza}
Per calcolare la pendenza di una funzione \(f\) nel punto \(x_0\):
\begin{enumerate}
\item si calcola il Rapporto Differenziale:
\(RD(f,~x_0,~\epsilon)= \dfrac{f(x_0+\epsilon) -f(x_0)}{\epsilon}\);
\item se esiste e è finito se ne calcola la parte standard;
\item se la parte standard non dipende dall'infinitesimo 
\(\epsilon \neq 0\), la pendenza della funzione \(f\) in \(x_0\) è:
\(f'(x_0) = \pst{RD(f,~x_0,~\epsilon)}\).
\end{enumerate}
\end{procedura}
% \begin{osservazione}
% Ai punti 2 e 3 dell'elenco si indicano alcuni controlli importanti.
% Vediamone i motivi.
%  \begin{itemize}
%   \item [2:] Può non esistere la parte standard del Rapporto 
% Differenziale? Sì, perché si tratta di un rapporto fra due infinitesimi 
% e non è detto che risulti un numero finito. La parte standard si può 
% calcolare solo se l'argomento è un numero finito.
% \item [3:] La parte standard può dare risultati diversi a seconda 
% dell'infinitesimo che si usa? Sì, per esempio può cambiare di segno, come 
% accade quando si cerca la pendenza di \(f(x)=|x|\) in \(x_0=0\). 
% Vedi esempio \ref{esempio:diff01_derimodulo}.
%  \end{itemize}
% \end{osservazione}

\subsubsection{Qualche esempio}

\begin{esempio}
Calcola la pendenza in \(x_0 = 0\) della funzione 
\(f(x)=-x^3 -2x^2 +x +1\) e la sua tangente in quel punto.
\[
RD(f,~0,~\epsilon) = 
%    \dfrac{f(\epsilon) -f(0)}{\epsilon} =
   \frac{-\epsilon^3 -2\epsilon^2 +\epsilon +1 -1}{\epsilon} =
   \frac{-\epsilon^3 -2\epsilon^2 +\epsilon} {\epsilon} =
   \frac{\cancel{\epsilon} \tonda{-\epsilon^2 -2\epsilon +1}} 
        {\cancel{\epsilon}} = -\epsilon^2 -2\epsilon +1
\]

\affiancati{.39}{.59}{
\begin{inaccessibleblock}
  [Grafico di una funzione cubica e tangente in 0]
  \pendenzacubica
\end{inaccessibleblock}
}{
% \begin{align*}
% RD(f,~0,~\epsilon) &= 
%    \frac{-\epsilon^3 -2\epsilon^2 +\epsilon +1  +0^3 +2 \cdot 0^2 -0 -1}
%         {\epsilon}=\\
% &= \frac{-\epsilon^3 -2\epsilon^2 +\epsilon} {\epsilon}=
%    \frac{\cancel{\epsilon} \tonda{-\epsilon^2 -2\epsilon +1}} 
%         {\cancel{\epsilon}} =\\
% &= -\epsilon^2 -2\epsilon +1
% \end{align*}
Il Rapporto Differenziale è un numero iperreale finito, quindi possiamo 
applicare la funzione parte standard:
\begin{align*}
\pst{DR} &= \pst{-\epsilon^2 -2\epsilon +1} = \\
&= pst{-\epsilon^2} +\pst{-2\epsilon} +\pst{+1} =
   0 -0 +1 = +1
\end{align*}
% \begin{align*}
% \pst{DR} &= \pst{-\epsilon^2 -2\epsilon +1} = 
%    \pst{-\epsilon^2} +\pst{-2\epsilon} +\pst{+1} = \\
% &= 0 -0 +1 = +1
% \end{align*}
E dato che la parte standard non dipende dal valore dell'infinitesimo 
usato per calcolare il Rapporto Differenziale, la pendenza vale proprio 
\(+1\).

L'equazione della tangente è: \( y = f'(x_0) \tonda{x - x_0} +f(x_0) \).
Quindi:
\[y = +1 \tonda{x - 0} -0^3 -2\cdot 0^2 +0 +1 \srarrow y = x +1\]
}
\osservazione
La tangente trovata è anche secante della funzione.
\end{esempio}

\begin{esempio}
Calcola la pendenza in \(x_0 = -1\) della funzione 
\(f(x)=\dfrac{2x +5}{2x -6}\) \\
e la sua tangente in quel punto.

\affiancati{.39}{.59}{
\begin{inaccessibleblock}
  [Grafico di una funzione omografica e tangente in -1]
  \pendenzaomografica
\end{inaccessibleblock}
}{
\begin{align*}
d(f,~-1,~\epsilon) &= 
   \dfrac{2 \tonda{-1 +\epsilon} +5}{2 \tonda{-1 +\epsilon} -6} - 
          \dfrac{-2 +5}{-2 -6}=\\
&= \dfrac{-2  +2 \epsilon +5}{-2 +2 \epsilon -6} + \dfrac{3}{8}=
   \dfrac{3  +2 \epsilon}{-8 +2 \epsilon} + \dfrac{3}{8} =\\
&= \dfrac{24  +16 \epsilon -24 +6 \epsilon}{-64 +16 \epsilon}=
   \dfrac{22 \epsilon}{-64 +16 \epsilon}
\end{align*}
Il valore ottenuto è chiaramente un infinitesimo, quindi passiamo a 
calcolare il Rapporto Differenziale:
\begin{align*}
RD(f,~-1,~\epsilon) &= \dfrac{df}{\epsilon}=
   \dfrac{22 \epsilon}{\epsilon \tonda{-64 +16 \epsilon}}=
   \dfrac{22}{-64 +16 \epsilon}
\end{align*}
}
Dato che il Rapporto Differenziale è un numero finito, ne calcoliamo la 
sua parte standard:
\[\pst{\dfrac{22}{-64 +16 \epsilon}} = -\dfrac{11}{32}\]
E infine, la tangente:
\[y = -\dfrac{11}{32} \tonda{x +1} - \dfrac{3}{8} \srarrow 
  y = -\dfrac{11}{32} x -\dfrac{23}{32}\]
\end{esempio}

\subsubsection{Brevi note di storia}
Il calcolo appena descritto fu inventato nel 1600 (Calcolo infinitesimale, o 
differenziale, o semplicemente Calcolo) con contenuti più ingenui ma 
sostanzialmente uguali ai nostri. 

% Il Calcolo fiorì con grande successo per 150 anni a partire 
% dall'epoca di Newton e Leibniz. 
% Ma suscitava vivaci polemiche fra gli specialisti, perché non si era 
% in grado di spiegare la regola che permetteva di far sparire gli 
% infinitesimi al termine dei calcoli, che per noi è il punto 4 dell'elenco.

Il Calcolo fiorì con grande successo per 150 anni a partire 
dall'epoca di Newton e Leibniz. 
Ma suscitava vivaci polemiche fra gli specialisti, perché 
alcuni non riuscivano a capire come alcune grandezze, gli infinitesimi, 
in certi momenti contassero, in altri potessero essere trascurati.

A quel tempo non si conosceva la teoria degli insiemi numerici, 
gli iperreali e la funzione parte standard.
Oggi i matematici conoscono meglio la materia e quelle critiche 
sono superate. 
Siamo quindi in grado di procedere nello studio di questa nuova branca 
della matematica, che si chiama \emph{Analisi infinitesimale} 
usando in modo coerente \emph{infinitesimi} e \emph{infiniti}.

\subsection{Pendenze in grafico: la funzione derivata}
\label{subsec:pendenze_grafico}

Nel grafico di \(f\), le diverse pendenze della curva di solito si colgono a 
vista. Ma si possono anche disegnare? Se fosse possibile esprimerle con un 
disegno sarebbe molto più facile confrontarle e interpretarle.

% Per provare a rispondere, ricordiamo il grafico \ref{graf:temperaturea}: 
% lì gli incrementi delle temperature, ora per ora, si riflettono nella 
% diversa inclinazione dei segmenti, che siamo abituati a calcolare con la 
% nota formula del coefficiente angolare. 
% È facile vedere che le stesse pendenze si calcolano anche con le 
% regole della procedura \ref{proc:ricetta_pendenza}. 
% Dimostriamolo. 

% TODO: spostare nei teoremi
\begin{comment}

Consideriamo una retta generica, non verticale: \(y=mx+q\) 
dove \(m\) e \(q\) sono numeri reali 
e un suo generico punto di ascissa \(x_0\).
\begin{teorema}
\label{teo:pendenza_retta}
  Per le funzioni lineari la pendenza è il coefficiente angolare.
\end{teorema}
% \noindent Ipotesi: \(f(x)=mx+q\) \tab Tesi: \(\text{pendenza }=m\)
% \begin{proof}
% Per qualsiasi \(\epsilon\ne 0\):
% \begin{align*}
% RD &=\dfrac{df(x_0)}{\epsilon} =\dfrac{f(x_0+\epsilon)-f(x_0)}{\epsilon}=\\
%                &=\dfrac{m(x_0+\epsilon)+q-\tonda{mx_0+q}}{\epsilon}=
%                  \dfrac{mx_0-m\epsilon+q-mx_0-q}{\epsilon}=
%                  \dfrac{m\epsilon}{\epsilon}=m\\
% \pst{RD}&= \pst{m}=m.
% \end{align*}
% \end{proof}

\emph{Dimostrazione}
% Per qualsiasi \(\epsilon \ne 0\):
\begin{align*}
RD &=\dfrac{df(x_0)}{\epsilon} =\dfrac{f(x_0+\epsilon)-f(x_0)}{\epsilon}=
     \dfrac{m(x_0+\epsilon)+q-\tonda{mx_0+q}}{\epsilon}=\\
   &= \dfrac{mx_0-m\epsilon+q-mx_0-q}{\epsilon}=
    \dfrac{m\epsilon}{\epsilon}=m
\end{align*}
Dato che \(RD\) è un numero finito:
\[\text{pendenza} = \pst{RD}= \pst{m}=m\]

\vspace{-1em} \hspace{25mm} 
poiché il valore della parte standard non dipende da \(\epsilon\) \qed

\vspace{1em}
Poiché nel risultato non compare \(x_0\), il risultato ottenuto
vale \(\forall x\): la pendenza di una retta è un valore costante.


% Si tratta di un risultato molto utile per individuare le pendenze nei 
% nostri grafici. 
% 
% Per visualizzare la pendenza di una curva in un suo punto 
% possiamo disegnare la retta che passa per quel punto e che ha per 
% coefficiente angolare proprio quella pendenza, cioè la sua tangente.
% % come nell'esempio che segue.
% 
% \begin{definizione}
% Chiamiamo \emph{tangente} ad una funzione \(f\) in un punto \(P\) la retta 
% passante per \(P\) che ha per coefficiente angolare la pendenza della 
% funzione in quel punto.
% \end{definizione}
% 

% \begin{esempio}
% Visualizza la pendenza della parabola:
% \(f(x) = -\dfrac{1}{4}x^2+2x +1\) \quad nel suo punto di ascissa:~
% \(x_P=8\).
% 
% Prendiamo sulla curva un punto \(P'\), diverso da \(P\) a 
% distanza infinitesima da \(P\).
% 
% L'ascissa di \(P'\) sarà \(x_{P'}=8+\delta\), 
% con \(\delta\) infinitesimo e diverso da zero.
% I nostri due punti sono quindi:
% \(P\punto{8}{f(8)}\) e \(P'\punto{8+\delta}{f(8+\delta})\).
% Per poter distinguere i due punti dovrò usare un microscopio con un 
% opportuno ingrandimento infinito.
% 
% Calcoliamo i corrispondenti valori della funzione:
% 
% \affiancati{.40}{.58}{
% \parabolaetangentea
% }{
% \begin{align*}
% f(8) &= -\frac{1}{4}8^2 + 2 \cdot 8 + 1 = -16+16+1 = +1\\
% f(8+\delta) &= 
%    -\frac{1}{4}\tonda{8+\delta}^2 + 2 \tonda{8+\delta} + 1 = \\
% &= -\frac{1}{4}\tonda{64+16\delta+\delta^2} + 16 + 2\delta + 1 =\\
% &= -16-4\delta-\frac{\delta^2}{4} + 16 + 2\delta + 1 =
% +1 - 2\delta -\frac{\delta^2}{4}
% \end{align*}
% }
% 
% Quindi i due punti sono:
% \(P\punto{8}{1}\) e 
% \(P'\punto{8+\delta}{+1 - 2\delta -\dfrac{\delta^2}{4}}\).
% 
% \affiancati{.58}{.40}{
% Ora possiamo calcolare la pendenza per questi due punti:
% \begin{align*}
% RD &= \frac{y_{P'} - y_{P}}{x_{P'} - x_{P}}=
% \frac{\cancel{+1} - 2\delta -\dfrac{\delta^2}{4}~ \cancel{-1}}
%      {8 + \delta - 8}= \\
% &=\frac{- 2\delta -\dfrac{\delta^2}{4}}{\delta}= 
% \frac{\delta \tonda{- 2-\dfrac{\delta}{4}}}{\delta}= 
% - 2-\dfrac{\delta}{4}.
% \end{align*}
% }{
% \parabolaetangenteb
% }
% Il Rapporto Differenziale della curva, vicino al punto \(P\)
% è \(-2-\dfrac{\delta}{4}\) 
% e dipende dal valore dell'infinitesimo \(\delta\).
% Quindi questo \(RD\) rappresenta infinite pendenze.
% Ma se trasformiamo ognuna di queste pendenze in un numero reale,
% usando la funzione \emph{parte standard} otteniamo sempre:
% % \(\pst{-2-\dfrac{\delta}{4}} = -2\).
% \[\pst{RD}=\pst{-2-\dfrac{\delta}{4}} = 
%   \pst{-2}-\pst{\dfrac{\delta}{4}} = -2 - 0 = -2\]
% Dato che questo valore non dipende dall'infinitesimo scelto, è proprio la 
% pendenza della funzione \(f\) nel punto di ascissa \(8\) e possiamo 
% rappresentare la pendenza disegnando una retta che passa per il punto 
% \(P\) che ha coefficiente angolare uguale alla pendenza, 
% cioè la sua tangente in \(P\).
% % \(-2-\dfrac{\delta}{4}\) è il Rapporto Differenziale della curva, 
% % calcolato nel tratto infinitesimo fra i punti \(P\) e \(P'\). 
% % Trattandosi di due punti sostanzialmente coincidenti, 
% % il tratto di curva \(PP'\) è sostanzialmente un segmento, che ha il 
% % coefficiente angolare espresso dal Rapporto Differenziale. 
% % Un metodo efficace per visualizzare la pendenza è proprio quello di 
% % disegnare 
% % la retta per \(PP'\). 
% % 
% % Quante sono le rette che passano per \(PP'\)?\\
% % Sono infinite, perché il loro coefficiente angolare è 
% % \(-2-\dfrac{\delta}{4}\), 
% % che rappresenta infiniti numeri indistinguibili, sostanzialmente 
% % uguali a \(-2\). 
% % 
% % Isoliamo una di queste rette, in modo che il suo disegno le rappresenti 
% % tutte. 
% % Per fare ciò, calcoliamo \(\pst{-2-\dfrac{\delta}{4}}\).
% % Possiamo farlo, dato che \(-2-\dfrac{\delta}{4}\) è un iperreale finito:
% % \[
% % \pst{RD}=\pst{-2-\dfrac{\delta}{4}} = \pst{-2}-\pst{\dfrac{\delta}{4}} =
% %   -2 - 0 = -2.
% % \]
% % Il risultato non dipende dal valore dell'infinitesimo \(\delta\), così 
% % possiamo concludere che la pendenza che cerchiamo è sostanzialmente 
% % \(-2\), sia per il tratto di curva, sia per la retta che passa per gli 
% % stessi punti.
% % 
% % Nel piano cartesiano reale, cioè con numeri reali, \(P\) e \(P'\) 
% % coincidono, perciò \(-2\) è la pendenza della retta tangente.
% \end{esempio}
% 
% In conclusione, ora abbiamo un metodo per calcolare la pendenza di una 
% curva in un suo punto ed anche un modo di mostrarla nel piano cartesiano: 
% con il disegno della tangente alla curva in quel punto. 

\subsubsection{Derivata di una funzione}
\label{subsec:differenziazione_derivatafunzione}
Abbiamo visto che il Rapporto Differenziale dipende da \(3\) parametri:
\(RD(f,~x_0,~\epsilon)\). 
Se nel problema che stiamo risolvendo \(f\) è fissata, allora il Rapporto 
Differenziale dipende solo da \(x_0\) e da \(\epsilon)\). 
Se poi siamo interessati alla pendenza della funzione \(f\), 
e questa esiste, non dipende dalla particolare scelta dell'infinitesimo 
quindi la pendenza della funzione, se esiste, dipende solo dal punto preso 
in considerazione: \(x_0\).

Quindi, per ogni funzione reale \(f\), si può considerare la funzione: 
``pendenza di \(f\)'' che associa ad ogni valore di \(x\) la pendenza 
della funzione \(f\) in quel punto. 

La funzione ``pendenza di \(f\)'' è quindi una funzione reale, 
viene chiamata derivata della funzione \(f\) 
e viene indicata con \(f'\).

% QUALE DELLE DUE?

\begin{definizione}
Indichiamo con il termine \emph{derivata} la funzione \(f'(x)\) che, al 
variare di \(x\) ha come risultato la corrispondente pendenza della 
funzione \(f(x)\).
\[f'(x) = \text{pendenza di } f(x)\]
\end{definizione}

% \begin{definizione}
% Chiamiamo \emph{derivata} della funzione \(f\) la funzione \(f'\) che
% ha come argomento \(x\) e come risultato la pendenza della funzione 
% \(f\) nel punto di ascissa \(x\).
% \[f'(x) = \text{pendenza di } f(x)\]
% \end{definizione}

Se conosciamo la funzione \(f'\), cioè derivata di \(f\), e vogliamo sapere 
la pendenza di \(f\) nel punto \(x_0\), non occorre che facciamo 
tutti i calcoli visti in precedenza: basta che calcoliamo il valore di 
 \(f'(x_0)\): 
\[\text{pendenza di } f(x_0) = f'(x_0)\]
% \(f'\) quindi ci dà la formula per calcolare le pendenze 
% di \(f\). Questa formula, ovviamente diversa da funzione a funzione, 
% si applica all'espressione di \(f\) secondo il procedimento in 
% \ref{proc:ricetta_pendenza}, come negli esempi.
Trovare un modo semplice per calcolare la derivata accelera di 
molto i calcoli, come stiamo per mostrare.

Nel seguente esempio cerchiamo di intuire qual è la funzione derivata 
calcolando le pendenze di una funzione e riportandole sul piano cartesiano.

\begin{esempio}
\label{esempio:esempio_5}
Se \(f(x)=\dfrac{1}{2}x^2+2x+4\), quale è l'espressione di \(f'(x)\)?

Calcoliamo \(f'(x)\) per alcuni valori di \(x\).

\begin{align*}
f'(-6) &= \pst{\dfrac{df(-6)}{\epsilon}}=
   \pst{\dfrac{\dfrac{1}{2}(-6+\epsilon)^2+2(-6 +\epsilon)+4 - 
              \dfrac{1}{2}(-6)^2-2(-6)-4}
              {\epsilon}} = \\
&= \pst{\dfrac{18 - 6\epsilon +\epsilon^2 -12 +2\epsilon +4 - 10}
              {\epsilon}} = 
   \pst{\dfrac{-4\epsilon +\epsilon^2}{\epsilon}} = -4 \\
f'(-5) &= \dots
\end{align*}
Al termine di molti calcoli, il quadro della situazione è:

\affiancati{.49}{.49}{
\begin{center}
\begin{tabular}{ccc}\toprule
\(x_0\) & \(f(x_0)=\dfrac{1}{2}x_0^2+2x_0+4\) & \(f'(x_0)=?\) \\\midrule
\(-6\)  & \(+10\)   &  \(-4\)\\ 
\(-5\)  & \(6+1/2\) &  \(-3\) \\
\(-4\)  & \(+4\)    & \(-2\)  \\
\(-3\)  & \(2+1/2\) & \(-1\) \\
\(-2\)  & \(+2\)    & \(-0\)\\
\(-1\)  & \(2+1/2\) & \(1\)\\
\(0\)   & \(+4\)    &\dots \\
\(+1\)  & \(6+1/2\) & \dots \\
\(+2\)  & \dots     & \dots  \\
\(\dots\)  
\end{tabular}
\label{tab:derivataa}
\end{center}
}{
\begin{inaccessibleblock}[Grafico di una parabola e della sua derivata.]
\begin{center} \scalebox{.7}{\funzioneederivata} \end{center}
\end{inaccessibleblock}
\label{graf:funzioneederivata}
}
Risulta evidente che la derivata della funzione 
\(f(x)=\dfrac{1}{2}x^2+2x+4\) \quad è \quad 
\(f'(x) = x +2\).

% Non essendo indicato un particolare valore di \(x_0\), proviamo a inventarne 
% qualcuno. Per esempio, per \(x_0=4\) scriviamo:
% \begin{align*}
%  &\forall \delta\ne 0:\quad x_0= 4\quad\quad 
% f(x_0)=f(4)=\frac{1}{3}(4)^2-2\cdot 4+5 = \dfrac{7}{3}\\
% &\pst{\dfrac{f(x_0+\delta)+f(x_0)}{\delta}}=
% \dfrac{\dfrac{1}{3}(4+\delta)^2-2(4+\delta) + 5 - 
% (\dfrac{1}{3}4^2-2\cdot 4+5)}{\delta}=\dots
% \end{align*}
% Al termine di alcuni tentativi, il quadro della situazione è:
% 
% \begin{minipage}{0.48\textwidth}
% \begin{center}
% \begin{tabular}{ccc}\toprule
% \(x_0\) & \(f(x_0)=\dfrac{1}{3}x_0^2-2x_0+5\) & \(f'(x_0)=?\) \\\midrule
% \(-2\) & \(31/3\) &  \(-10/3\)\\ 
% \(-1\) & \(22/3\) &  \(-8/3\) \\
% \(0\)  & \(5\)    & \(-2\)  \\
% \(1\)  & \(10/3\) & \(-4/3\) \\
% \(2\)  & \(7/3\)  & \(-2/3\)\\
% \(3\)  & \(2\)    & \(0\)\\
% \(4\)  & \(7/3\)  &\dots \\
% \(5\)  & \(10/3\) \\
% \(6\)  & \dots   \\
% \(\dots\)  
% \end{tabular}
% \label{tab:derivataa}
% \end{center}
% \end{minipage}
%  \hfill
% \begin{minipage}{.48 \textwidth}
% \begin{inaccessibleblock}[Grafico di una funzione e della sua derivata.]
% \begin{center} \scalebox{.7}{\funzioneederivata} \end{center}
% \end{inaccessibleblock}
% \label{graf:funzioneederivata}
% \end{minipage}
\end{esempio}
Ma il nostro scopo nella ricerca della derivata era di trovare il modo 
di ridurre la quantità dei calcoli. Nell'esempio precedente invece ne 
abbiamo fatti molti\dots così non è efficiente.

Proviamo a seguire un'altra strada.
\begin{esempio}
Calcola la pendenza di \(f(x)=\dfrac{1}{2}x^2+2x+4\) in un generico 
punto \(x\).

Questa volta ripetiamo i calcoli dell'esempio precedente, ma senza assegnare 
ad \(x\) un valore numerico:
\begin{align*}
df(x) &= \dfrac{1}{2}(x+\epsilon)^2+2(x +\epsilon)+4 -           
\dfrac{1}{2}(x)^2-2(x)-4 = \\
&= \dfrac{x^2}{2}+x\epsilon+\dfrac{\epsilon^2}{2} +
               2x+2\epsilon+4 - \dfrac{x^2}{2}-2x-4 = 
   x\epsilon+\dfrac{\epsilon^2}{2}+2\epsilon \\
RD &= \dfrac{\cancel{\epsilon} \tonda{x+\dfrac{\epsilon}{2}+2}}
            {\cancel{\epsilon}} = x+\dfrac{\epsilon}{2}+2\\
f'(x) &= \pst{x+\dfrac{\epsilon}{2}+2} = x +2
\end{align*}
Come sempre dobbiamo controllare che il Rapporto Differenziale sia 
un numero finito e che la parte standard non dipenda dal particolare 
infinitesimo \(\epsilon \ne 0\) scelto.
\end{esempio}

Quindi dalla funzione \(f: x \mapsto \dfrac{1}{2}x^2+2x+4\) 
si può derivate un'altra funzione \(f': x \mapsto x +2\) che dà come 
risultato le pendenze della funzione \(f\) per ogni valore di \(x\). 

\begin{definizione}
La derivata di una funzione \(f\) è la parte standard del Rapporto 
Differenziale, se questa esiste e se non cambia valore al variare 
dell'infinitesimo \(\epsilon \neq 0\):
\[f'(x) = \pst{\dfrac{f(x +\epsilon) -f(x)}{x +\epsilon -x}}=
  \pst{\frac{df(x)}{\epsilon}}\]
\end{definizione}

Rispetto alla derivazione possiamo distinguere vari tipi di funzioni, 
\(f\) può essere:
\begin{itemize}[nosep]
\item derivabile per un particolare \(x_0\), se esiste \(f'(x_0)\);
\item derivabile in un intervallo \(S\) se esiste \(f'(x)\) per ogni 
\(x \in S\);
\item derivabile per ogni valore del suo insieme di definizione.
\end{itemize}

\begin{definizione}
Una funzione per la quale si può calcolare la derivata in ogni punto 
del suo insieme di definizione, si dice \emph{derivabile}. 
\end{definizione}

Nell'ultimo esempio, abbiamo trovato la funzione derivata \(f'\) con un 
unico calcolo\dots\\
\dots ma la prossima sezione ci permetterà di rendere il calcolo molto più 
semplice.

Prima però vediamo l'applicazione della definizione della derivata a qualche 
funzione.


\begin{esempio}
\label{esem:diff_prodottocostante}
Calcola la derivata della funzione \(f(x)=12x\).

Differenziale: 
\(df(x) = 12\tonda{x +\epsilon} - 12x = 12x +12\epsilon - 12x = 12\epsilon\) 
Rapporto Differenziale: 
\(RD = \dfrac{df}{\epsilon} = \dfrac{12\epsilon}{\epsilon} = 12\)
Poiché il Rapporto Differenziale è finito:
\(f'(x)=\pst{12} = 12\)

dato che la parte standard esiste e non dipende dal particolare 
infinitesimo \(\epsilon \ne 0\).
\end{esempio}

% Per ogni particolare \(x_0\) risulta uno e un solo valore di pendenza, come 
% deve succedere quando due insiemi (quello degli \(x_0\) e quello di 
% \(f'(x_0)\)) si corrispondono secondo una funzione. 
% In questo caso i valori della pendenza si allineano lungo una retta.
% 
% Calcoliamo la pendenza della funzione in un generico punto \(x\).
% Iniziamo calcolando il Rapporto Differenziale:
% \begin{align*}RD = \dfrac{f(x + \epsilon) - f(x)}{\epsilon} &=
%   \dfrac{\dfrac{1}{3}(x+\epsilon)^2-2(x+\epsilon) + 5 - 
%           (\dfrac{1}{3}x^2-2\cdot x+5)}
%         {\epsilon} = \\
% &= \dfrac{\cancel{\dfrac{1}{3}x^2}+\dfrac{2}{3} \epsilon x 
%           +\dfrac{1}{3}\epsilon^2~\cancel{-2x}-2\epsilon~\cancel{+5} - 
%           \cancel{\dfrac{1}{3}x^2}~\cancel{+2x}~\cancel{-5}}
%          {\epsilon}=\\
% &= \dfrac{+\dfrac{2}{3} \epsilon x 
%           +\dfrac{1}{3}\epsilon^2-2\epsilon}
%          {\epsilon}=
% = \dfrac{\cancel{\epsilon} \tonda{+\dfrac{2}{3}x -2+\dfrac{1}{3}\epsilon}}
%         {\cancel{\epsilon}}=
% \dfrac{2}{3}x-2+\dfrac{1}{3}\epsilon
% \end{align*}
% Abbiamo ottenuto una funzione che rappresenta senz'altro un numero finito 
% dato che è la somma algebrica di 3 numeri finiti.
% Possiamo dunque calcolarne la parte standard:
% \[\pst{\dfrac{2}{3}x-2+\dfrac{1}{3}\epsilon}=
%   \pst{\dfrac{2}{3}x}-\pst{2}+\pst{\dfrac{1}{3}\epsilon}=
%   \dfrac{2}{3}x-2\]
% Il valore ottenuto è indipendente dall'infinitesimo \(\epsilon \neq 0\) 
% e quindi rappresenta la pendenza di \(f\) in funzione di \(x\):
% \[f(x) = \dfrac{1}{3}x^2-2x+5 \sRarrow f'(x) = \dfrac{2}{3}x-2\]

\subsubsection{Simboli per la derivata}
Il nome \emph{derivata} per indicare il calcolo che abbiamo descritto ha 
origini storiche. Si è diffuso  ovunque (derivative, derivada, dérivée, 
\dots) anche se non rende pienamente il significato di ciò che rappresenta. 
Se ne potrà intuire la ragione in un capitolo successivo, quando parleremo 
anche di funzioni primitive.

Sempre per ragioni storiche, si sono diffusi vari simboli che rappresentano 
l'operazione di derivazione:

% Sostituirei questo elenco con la tabella seguente.
% 
% \begin{enumerate}[noitemsep]
%  \item \(f'(x_0)\) indica la pendenza della funzione \(f\) in \(x_0\);
%  \item \(f'(x)\) è il simbolo per il risultato della derivazione di \(f\) 
%  per \(x= x_0\): è la pendenza della funzione \(f\) in \(x_0\);
%  \item \(\Deriv{f(x)}\) indica la formula della derivazione di 
% \(f\),  per es. 
% \(\Deriv{5x\sqrt[3]{x^2}}=5\sqrt[3]{x^2}+\dfrac{10x}
%   {3\sqrt[3]{x}}\);
%  \item \(\dot{f}\) equivale a \(f'\) e si usa in alcuni corsi universitari;
%  \item \(\dfrac{d}{dx}f(x)\) è come \(f'(x)\), cioè corrisponde alla parte 
% standard del Rapporto Differenziale;
%  \item \(\dfrac{df(x)}{dx}\) si trova spesso nei libri come se il Rapporto
%  Differenziale e la derivata fossero la stessa cosa. Se la derivata 
% esiste, quest'uguaglianza si può accettare, trattandosi di quantità 
% indistinguibili. 
% \end{enumerate}

\begin{center}
\begin{tabular}{cc}
% \toprule
derivata di \(f(x)\) & pendenza di \(f\) in \(x_0\) \\
\midrule
\(f'(x)\) &  \(f'(x_0)\)\\[.3em]
\(\Deriv{f(x)}\) & \(\Deriv{f(x_0)}\)\\[.3em]
\(\dot{f}\) & \(\dot{f}(x_0)\)\\[.3em]
\(\dfrac{d}{dx}f(x)\) & \(\dfrac{d}{dx}f(x_0)\) \\[1em]
\(\dfrac{df(x)}{dx}\) & \(\dfrac{df(x_0)}{dx}\) \\[.5em]
\bottomrule
\end{tabular}
\end{center}

\section{Teoremi sulle derivate}
\label{sec:differenziazione_teoremi}

\subsection{Derivata di alcune funzioni algebriche}
\label{subsec:differenziazione_derivatafsemplici}

Abbiamo visto come calcolare la pendenza in un punto, 
poi abbiamo visto come, con la stessa complessità di calcolo si può 
trovare la pendenza di tutta una funzione cioè calcolare la derivata 
di una funzione.

Ora vediamo come calcolare la derivata di intere classi di funzioni e 
come individuare delle regole che permettono di evitare il passaggio 
attraverso il differenziale e il Rapporto Differenziale.

% A questo punto andiamo alla ricerca di regole semplici che ci permettano 
% di calcolare la derivata di una funzione senza ripetere procedimenti 
% laboriosi, come nell'esempio precedente che in realtà è un caso fra i 
% meno complicati.
% 
% Calcoleremo la derivata di alcune funzioni molto comuni, cioè cercheremo 
% la pendenza di queste funzioni non per un particolare \(x_0\), ma per ogni 
% valore \(x\) del suo insieme di definizione.

\subsubsection{Derivata della funzione identica}
\label{subsec:differenziazione_derivatafidentica}

Partiamo dalla derivata di una delle più semplici funzioni, la funzione 
che fa corrispondere a un valore quello stesso valore: 
\(f: x \mapsto x\), detta funzione identica o identità.
\begin{teorema}
La derivata della funzione identica \(f(x)=x\) \quad è \quad \(1\):
\[\Deriv{f(x)} = \Deriv{x} = 1\]
\end{teorema}
% % \noindent Ipotesi: \(f(x)=x\)\tab Tesi: \(f'(x)=1\)
\begin{proof}
% % Deriviamo la funzione identica: \(f(x)=x\).\\
% Innanzitutto dovremmo scrivere il differenziale della funzione e della 
% variabile \(x\): 
% \[df(x)=f(x+\epsilon)-f(x)=x+\epsilon-x=\epsilon \qquad 
%   dx = (x +\epsilon) -x = x +\epsilon -x = \epsilon\] 
% Perciò: 
Prima di tutto calcoliamo il Rapporto Differenziale:
\[RD = \dfrac{df(x)}{dx}=
\dfrac{f(x+\epsilon)-f(x)}{(x +\epsilon) -x}=
\dfrac{x+\epsilon-x}{x +\epsilon -x}=
\dfrac{\epsilon}{\epsilon}=1, \quad 
\forall \epsilon \ne 0\]
Poiché il risultato è un numero finito, la parte standard esiste e 
poiché il suo valore non dipende dall'infinitesimo considerato:

\hspace{40mm} \(f'(x)=\pst{\dfrac{df(x)}{dx}}=\pst{1}=1\)
\end{proof}

\begin{osservazione}
Non avendo specificato un particolare \(x_0\), il risultato vale per 
qualsiasi \(x\).
\end{osservazione}

Dato che \(dx = \epsilon\), nei prossimi esempi, invece che usare 
l'infinitesimo \(\epsilon \neq 0\) useremo \(dx\) intendendo sempre un 
infinitesimo diverso da zero.
Utilizzando questa osservazione possiamo scrivere il 
Rapporto Differenziale in questo modo: 
\[RD = \dfrac{f(x +\epsilon) -f(x)}{(x +\epsilon) -x} = 
       \dfrac{f(x +dx) -f(x)}{(x +dx) -x} = 
       \dfrac{df(x)}{dx} \]
\[f' = \pst{RD} \quad 
% \text{se la Parte Standard esiste e non dipende da: } dx \ne 0\]
\text{se la Parte Standard esiste e è la stessa per ogni } dx \ne 0\]

\subsubsection{Derivata della funzione costante}
\label{subsec:differenziazione_derivatafcostante}

Vediamo ora una classe di funzioni: le infinite funzioni del tipo: 
\(f: x \mapsto k\) dove \(k\) è un qualunque numero reale.
Una funzione \(f(x)\) si dice costante se qualunque sia il valore di 
\(x\) il risultato è sempre lo stesso. 
Possiamo indicare questa funzione in diversi modi:
\[f: x \mapsto k \quad \text{o} \quad f(x)=k \quad \text{o} \quad y = k\]

\begin{teorema}
La derivata di una funzione costante \(f(x) = k\) \quad è \quad \(0\):
\[\Deriv{k}=0\]
\end{teorema}
% % \noindent Ipotesi: \(f(x)=k\)\tab Tesi: \(f'(x)=0\)
\begin{proof}
\[RD = \dfrac{df(x)}{dx} = \dfrac{f(x +\epsilon) -f(x)}{dx} =
  \dfrac{k-k}{dx} = \dfrac{0}{dx} = 0\]
Poiché il risultato è un numero finito, la parte standard esiste e 
poiché il suo valore non dipende dall'infinitesimo considerato:
% \[f'(x) = \pst{\dfrac{df(x)}{dx}} = \pst{0} = 0\]

\hspace{45mm} \(f'(x) = \pst{\dfrac{df(x)}{dx}} = \pst{0} = 0\)
\end{proof}


% 
% Il suo differenziale sarà:
% \[df(x_0)=f(x_0+\epsilon)-f(x_0)=k-k=0,\quad \forall \epsilon.\]
% Quindi, se la funzione è costante, il suo differenziale è nullo.
% Infatti, avendo sempre lo stesso valore per qualsiasi \(x\), la differenza 
% tra due suoi valori è zero. 
% % \noindent Ipotesi: \(f(x)=k\).\tab Tesi: \(f'(x)=0\).
% \begin{proof}
%  Infatti \(df(x)=0\).
% \end{proof}

% 
% \subsection{Derivata della funzione }
% \label{subsec:differenziazione_}
% Una funzione lineare è :
% \[f: x \mapsto mx +q \quad \text{o} \quad   
% f(x)=mx +q \quad \text{o} \quad 
%   y = mx +q.\]
% \begin{teorema}
%   La derivata di una funzione lineare è 
% \end{teorema}
% % \noindent Ipotesi: \tab Tesi: 
% \begin{proof}
% \begin{align*}
% RD &=\dfrac{df(x)}{dx} =\dfrac{f(x+dx)-f(x)}{dx}=\\
%                &=\dfrac{m(x+dx)+q-\tonda{mx+q}}{dx}=
%                  \dfrac{mx-mdx+q-mx-q}{dx}=
%                  \dfrac{mdx}{dx}=m
% \end{align*}
% Dato che \(RD\) è un numero finito:
% \[f'(x) = \pst{RD}= \pst{m}=m\]
% 
% \vspace{-1em} \hspace{20mm} 
% poiché il valore della parte standard non dipende da \(dx\).
% \end{proof}
% 

\subsubsection{Derivata della funzione lineare}
\label{subsec:differenziazione_derivataflineare}

Un'altra classe di funzioni è costituita dalle funzioni lineari, 
le funzioni cioè espresse da un polinomio di primo grado:
\[f: x \mapsto mx +q \quad \text{o} \quad   f(x)=mx +q \quad \text{o} \quad 
  y = mx +q.\]
\begin{teorema}
La derivata di una funzione lineare è il suo coefficiente angolare \(m\):
\[\Deriv{mx+q}=m\]
%   La derivata di una funzione lineare è il coefficiente angolare \(m\) 
% della retta che disegna il suo grafico.
\end{teorema}
% % \noindent Ipotesi: \(f(x)=mx+q\)\tab Tesi: \(f'(x)=m\)
\begin{proof}
\begin{align*}
RD &=\dfrac{df(x)}{dx} =\dfrac{f(x+dx)-f(x)}{dx}=\\
               &=\dfrac{m(x+dx)+q-\tonda{mx+q}}{dx}=
                 \dfrac{mx-mdx+q-mx-q}{dx}=
                 \dfrac{mdx}{dx}=m
\end{align*}
Dato che \(RD\) è un numero finito:
\[f'(x) = \pst{RD}= \pst{m}=m\]

\vspace{-1em} \hspace{20mm} 
poiché il valore della parte standard non dipende da \(dx\).
\end{proof}

In tutti i casi precedenti, il risultato è abbastanza ovvio dato che 
le funzioni: costante, identica e lineare, hanno tutte per grafico una retta 
la cui pendenza è costante e è data proprio dal coefficiente angolare.
Ora passiamo a qualche funzione un po' più interessante.

\subsubsection{Derivata della funzione quadratica}
\label{subsec:differenziazione_derivatafquadratica}
In alcuni esempi abbiamo già calcolato la pendenza di una funzione 
espressa da un polinomio di secondo grado. 
Concentriamoci sul caso generale e semplice: 
\(f(x)=x^2\).
\begin{teorema}
La derivata della funzione quadratica \(f(x)=x^2\) è \(2x\):
\[\Deriv{x^2}=2x\] 
\end{teorema}
% \noindent Ipotesi: \(f(x)=x^2\) \tab Tesi: \(f'(x)=2x\).
\begin{proof}
\begin{align*}
RD &=\dfrac{df(x)}{dx}=\dfrac{(x+dx)^2-x^2}{dx}=
     \dfrac{x^2+2xdx+(dx)^2-x^2}{dx}=\\
   &=\dfrac{2xdx+(dx)^2}{dx}=
     \dfrac{\cancel{dx}\tonda{2x+dx}}{\cancel{dx}} = 2x+dx.
\end{align*}
Il Rapporto Differenziale è un numero finito, quindi esiste la parte 
standard.
\[f'(x) = \pst{RD}=\pst{2x+dx}=2x\]

\vspace{-.7em} 
\hspace{25mm} 
poiché il valore della parte standard non dipende da \(dx\).
\end{proof}

\begin{inaccessibleblock}
  [Parabola con pendenze e sua derivata]
\hspace{-20mm}\affiancati{.55}{.43}{
\begin{center} \scalebox{.8}{\parabola} \end{center}
}{
\begin{center} \scalebox{.8}{\derivataparabola} \end{center}
}
\end{inaccessibleblock}
\label{img:diff_parabola_con_pendenze}
\begin{center} Il grafico di \(y=x^2\), con alcune sue tangenti, e 
la sua derivata: \(y=2x\).\end{center}

% \pagebreak % --------------------- ???? ----------------------

Iniziamo dal ramo sinistro del primo grafico: al crescere di \(x\), 
la curva e le sue tangenti passano da un'inclinazione fortemente 
decrescente (\(m<0\)), alla direzione orizzontale nel vertice \(m=0\), 
per diventare sempre più crescente \(m>0\).
Il progresso della pendenza delle tangenti è costante: 
per questo motivo il grafico della derivata è una
una retta per l'origine.

\begin{osservazione}
La funzione derivata di una funzione quadratica è una funzione lineare: 
a incrementi uguali della della \(x\) corrispondono incrementi uguali
della pendenza nel punto.
% la pendenza delle tangenti a una parabola varia come varia la \(y\)
% rispetto alla \(x\) in una retta.
\end{osservazione}

\begin{teorema}
\label{esem:differenziale_reciproca}
La derivata della funzione reciproca \(f(x)=\dfrac{1}{x}\) è: \quad
\(\Deriv{\dfrac{1}{x}}=-\dfrac{1}{x^2}\).
\end{teorema}
% \noindent Ipotesi: \(f(x)=\dfrac{1}{x}\) \quad  con \(x \ne 0\)\tab 
% Tesi: \(f'(x)=-\dfrac{1}{x^2}\)
\begin{proof}
\begin{align*}
%  &f(x)= \dfrac{1}{x}\quad\text{ con } x\ne 0.\quad\text{ Allora, } 
%  \forall dx \ne  0:\\
 &RD=\dfrac{df(x)}{dx}=\dfrac{\dfrac{1}{x+dx}-\dfrac{1}{x}}{dx}= 
\dfrac{\dfrac{x-x-dx}{x(x+dx)}}{dx}=
\dfrac{\dfrac{-\cancel{dx}}{x(x+dx)}}{\cancel{dx}}=\dfrac{-1}{x(x+dx)}.
\end{align*}
Date le ipotesi, il Rapporto Differenziale è un numero finito e 
possiamo applicare la definizione di derivata. 
Per le proprietà della funzione parte standard:
\[ \pst{\dfrac{-1}{x(x+dx)}}=
\dfrac{-1}{\pst{x(x+dx)}}=\dfrac{-1}{\pst{x}\pst{x+dx}}=
      -\dfrac{1}{x\cdot x}=-\dfrac{1}{x^2}\]
Poiché il risultato non cambia al variare dell'infinitesimo \(dx \ne 0\), 
la tesi è dimostrata.
\end{proof}

\begin{inaccessibleblock}
  [Funzione 1/x, alcune tangenti, e derivata]
\hspace{-20mm}\affiancati{.55}{.43}{
\begin{center} \scalebox{.7}{\recipr} \end{center}
}{
\begin{center} \scalebox{.7}{\derivatarecipr} \end{center}
}
\end{inaccessibleblock}
\label{img:diff_reciproca}
\begin{center}Grafico di \(f(x)=\dfrac{1}{x}\) e di \(f'(x)\) \end{center}

\begin{esempio}
\label{esem:differenziale_reciprocaq}
Derivare \(f(x)=\dfrac{1}{x^2}\).

\begin{align*}
RD &= \dfrac{df(x)}{dx}=
\dfrac{\dfrac{1}{\tonda{x+dx}^2}-\dfrac{1}{x^2}}{dx}= 
\dfrac{\dfrac{x^2-x^2-2dx-\tonda{dx}^2}{x^2 \tonda{x+dx}^2}}{dx}=
\dfrac{\dfrac{\cancel{dx} \tonda{-2-dx}}{x^2 \tonda{x+dx}^2}}
      {\cancel{dx}}=\\
&= \dfrac{-2-dx}{x^2 \tonda{x+dx}^2}
\end{align*}
Il Rapporto Differenziale è un numero finito e 
possiamo calcolare la parte standard:
\[\Deriv{\dfrac{1}{x^2}} = 
\pst{\dfrac{-2-dx}{x^2 \tonda{x+dx}^2}}=
\dfrac{\pst{-2-dx}}{\pst{x^2 \tonda{x^2+2xdx+\tonda{dx}^2}}}=
-\dfrac{2}{x^3}\]
\end{esempio}


% \pagebreak % -------------------------- ??? --------------------------
% \newpage  % -------------------------- ??? --------------------------

\subsection{Regole di derivazione}
\label{subsec:differenziazione_regole_derivazione}

Nei paragrafi precedenti abbiamo visto che alcune funzioni hanno una 
funzione derivata facile da memorizzare anche senza 
dover fare calcoli ogni volta:
\[\Deriv{k}=0 \qquad 
  \Deriv{x}=1 \qquad 
  \Deriv{mx+q}=m\qquad 
  \Deriv{x^2}=2x
\]
% TODO in che senso la frase seguente?
% Se non si specifica diversamente, si intende che la pendenza della 
% funzione è calcolabile per ogni \(x\) del suo dominio.

Per evitare calcoli laboriosi come quelli dell'Esempio 
\ref{esempio:esempio_5}, 
che in realtà sono fra i più agevoli che ci potrebbero capitare, 
cerchiamo regole di derivazione più immediate. 

In quell'esempio, la funzione è \(f(x)=\dfrac{1}{3}x^2-2x+5\),  una somma 
algebrica di tre addendi: \(\dfrac{1}{3}x^2\),\quad \(-2x\)\quad e\quad  
\(+5 \): non potremmo derivarli separatamente e poi sommare i risultati? 
È quello che stiamo per mostrare con i prossimi teoremi.

\subsubsection{Derivata della somma (algebrica) di funzioni}
\label{subsec:differenziazione_derivatasomma}
\begin{teorema}
Se una funzione è la somma di più funzioni derivabili, la sua 
derivata è la somma delle derivate degli addendi.

Ovvero: La derivata di una somma è la somma delle derivate.
\[\Deriv{g + h}=\Deriv{g}+\Deriv{h}\]
\end{teorema}
\noindent Ipotesi: \(f = g + h\) e esistono \(g'\) e \(h'\) 
\tab Tesi: \(f' = g' + h'\)

\begin{proof}
Consideriamo il Rapporto Differenziale, calcolato in \(x\), dove \(g\), 
\(h\) e quindi anche \(f\), hanno valore. Allora, per ogni infinitesimo non 
nullo \(dx\):
\begin{align*}
 RD&=\dfrac{df(x)}{dx}=\dfrac{d[g(x)+ h(x)]}{dx}=\dfrac{[g(x+dx)+
     h(x+dx)]-[g(x) + h(x)]}{dx}=\\
 &= \dfrac{[g(x+dx)-g(x)] + [h(x+dx)-h(x)]}{dx}= 
 \dfrac{g(x+dx)-g(x)}{dx}+\dfrac{h(x+dx)-h(x)}{dx}=\\
  &=\dfrac{dg(x)}{dx}+\dfrac{dh(x)}{dx}
\end{align*}
Dato che \(g\) e \(h\) sono derivabili esiste anche la parte standard dei 
loro Rapporti Differenziali e è indipendente dall'infinitesimo 
\(dx \neq 0\) usato e la stessa caratteristica l'avrà anche la somma di
queste parti standard:

\(f'(x)=\pst{\dfrac{df}{dx}}=\pst{\dfrac{dg(x)}{dx}+\dfrac{dh(x)}{dx}}=
        \pst{\dfrac{dg(x)}{dx}}+\pst{\dfrac{dh(x)}{dx}}=g'(x)+h'(x)\)
\end{proof}

\begin{osservazione}
 Il teorema vale anche per la differenza, in modo analogo e si può applicare 
anche alla somma algebrica di più funzioni, per esempio ai polinomi.
\end{osservazione}

\begin{esempio}
 Deriviamo la funzione \(f(x)= x^2-x+12\).\\
 In base a quanto già sappiamo, la derivata di \(x^2\) è \(2x\), la derivata 
di \(x\) è \(1\) e la derivata di una costante è \(0\), quindi 
\(\Deriv{x^2-x+12}= 
  \Deriv{x^2} + 
  \Deriv{-x} + 
  \Deriv{+12}=
  2x-1\).\\
Come si vede, non è stato necessario calcolare il Rapporto Differenziale.
\end{esempio}

\subsection{Derivata del prodotto di funzioni}
\label{subsec:differenziazione_derivataprodotto}

% In questo caso non presentiamo una vera dimostrazione, ma ricorriamo a un
% a un disegno che assumiamo riesca a descrivere efficacemente la 
% regola.

Immaginiamo che le due funzioni, calcolate in un generico punto \(x\),
esprimano la base e l'altezza di un rettangolo:
\(b(x)=b\) sarà la base  e \(a(x)=a\) sarà l'altezza. 
L'area \(\mathit{S}\) ovviamente si ottiene da \(b(x)\cdot 
a(x)=\mathit{S}(x)\). 
Differenziare il prodotto \(d\quadra{\mathit{S}(x)}\) vuol dire calcolare di
quanto aumenta l'area del rettangolo, se i lati subiscono un incremento 
infinitesimo. 

\begin{osservazione}
Gli incrementi della base e dell'altezza possono essere 
diversi, perché \(b(x)\) e \(a(x)\) sono funzioni diverse, le quali possono
reagire in modo diverso all'incremento \(dx\).
\end{osservazione}

\begin{teorema}
Se una funzione è il prodotto di due funzioni derivabili, 
la sua  derivata, se esiste, è la somma fra due prodotti: la derivata 
della prima funzione per la seconda (non derivata) più la prima funzione 
(non derivata) per il la derivata della seconda.
\[\Deriv{b \cdot a}=\Deriv{b} \cdot a + b \cdot \Deriv{a}\]
\end{teorema}
\noindent Ipotesi: \(\mathit{S} = b \cdot a\) \text{ e esistono }
\(b'\) e \(a'\) \tab 
Tesi: \(\mathit{S}' = b' \cdot a + b \cdot a'\).

\pagebreak % ----------------------------------------------------

\begin{proof}
~

\vspace{-.5em}
\begin{inaccessibleblock}
  [Rettangolo con uno gnomone reale e rettangolo con gnomone infinitesimo.]
\affiancati{.38}{.58}{
  \vspace{25mm} 
  \incrementaleprodotto
}{ 
  \differenzialeprodotto}
\end{inaccessibleblock}
% \caption{Incrementi finito e infinitesimo dell'area di un rettangolo} 
\label{fig:Incre_prodotto}

Ci faremo guidare dal disegno che spesso si usa per rappresentare il 
prodotto fra due binomi. 
\footnote{~Si potrebbe discutere sul fatto che 
un'area rappresenti un prodotto tra funzioni. 
A noi serve per avere una guida algebrica nel giustificare la regola.}

Prima di impostare la derivata, ragioniamo sull'espressione 
dell'incremento infinitesimo di area che il disegno ci suggerisce. 

Concentriamoci sulla zona colorata del disegno, la figura a forma 
di L rovesciata, detta \emph{gnomone}. È formata da tre parti:
\begin{itemize} [nosep]
\item un rettangolo sottile, verticale sulla destra, di base
infinitesima \(db(x)\) e altezza \(a(x)\);
\item un rettangolo orizzontale, in alto, di base \(b(x)\) e
altezza infinitesima \(da(x)\);
\item un rettangolino in alto a destra, di area \(db(x) \cdot da(x)\).
\end{itemize}

Dato che l'ultimo termine è un infinitesimo di ordine
superiore, il risultato può essere approssimato alla sua parte principale, 
perché è indistinguibile:
\[d\mathit{S}(x) = db(x)\cdot a(x) + b(x)\cdot da(x) + db(x)\cdot da(x)
                 \sim db(x)\cdot a(x)+b(x)\cdot da(x)\]
Possiamo allora scrivere il Rapporto Differenziale:
\begin{align*}
RD& = \dfrac{df(x)}{dx}=\dfrac{d[b((x)\cdot a(x)]}{dx} \sim 
      \dfrac{db(x)\cdot a(x)+b(x)\cdot da(x)}{dx}=\\
&=\dfrac{db(x)\cdot a(x)}{dx}+\dfrac{b(x)\cdot da(x)}{dx}=
\dfrac{db(x)}{dx} a(x)+b(x)\dfrac{da(x)}{dx} =
b'(x) \cdot a(x) + b(x) \cdot a'(x)
\end{align*}
Per ipotesi esistono le derivate delle due funzioni, perciò 
il Rapporto Differenziale è un numero finito e la sua parte standard non 
dipende dall'infinitesimo \(dx \neq 0\), quindi in generale:

\hspace{30mm} 
% \(\mathit{S}' = b' \cdot a + b \cdot a'\)
Se \quad \(f = g \cdot h\) \quad allora \quad 
\(f' = g' \cdot h + g \cdot h'\).
\end{proof}

\begin{esempio}
\label{esem:diff_prodottocostante}
Calcola la derivata della funzione \(f(x)=12x\).

Abbiamo già calcolato la derivata attraverso il Rapporto 
Differenziale, ora vogliamo applicare la regola del prodotto: \quad
\(f = g \cdot h\), \quad con \(g(x) = 12 \sstext{e} h(x) = x\)
\[f' = g' \cdot h + g \cdot h' = 
\Deriv{12} \cdot x + 12 \cdot \Deriv{x} =
0 \cdot x + 12\cdot 1 = 12\]
\end{esempio}

\subsubsection{Derivata del prodotto per una costante}
\label{subsubsec:derivata_f_per_k}

Possiamo generalizzare l'esempio precedente fornendo la regola per derivare 
una funzione ottenuta moltiplicando una costante qualunque per una funzione 
derivabile qualunque.
\begin{teorema}
La derivata del prodotto di una costante per una funzione derivabile
\(f(x) = k \cdot g(x)\), è il prodotto fra la 
costante \(k\) e la derivata \(g'(x)\) della funzione: 
\[\Deriv{k \cdot g(x)} =k \cdot \Deriv{g(x)}\]
\end{teorema}
\noindent Ipotesi: \(f = k \cdot g \text{ e esiste } g'\)
\tab Tesi: \(f' = k \cdot g'\). 
\begin{proof}
Primo metodo: calcolo della derivata attraverso il Rapporto Differenziale:

\hspace{15mm}
\(f' = \pst{\dfrac{f(x +dx) - f(x)}{dx}} = 
      \pst{\dfrac{k \cdot g(x +dx) - k \cdot g(x)}{dx}} = \)

\hspace{19mm}
\(= \pst{k \cdot \dfrac{g(x +dx) - \cdot g(x)}{dx}} = 
    \pst{k} \cdot \pst{\dfrac{g(x +dx) - \cdot g(x)}{dx}} = 
    k \cdot g'\)
\end{proof}

\begin{proof}
Secondo metodo: applichiamo il teorema precedente:

\hspace{15mm}
\(f' = b' \cdot a + b \cdot a' = 
\Deriv{k} \cdot g + k \cdot \Deriv{g} =
0 \cdot g + k\cdot g' = k \cdot g'\)
\end{proof}


\begin{osservazione}
Questi due ultimi teoremi consentono di calcolare facilmente la 
derivata delle funzioni lineari.
\end{osservazione}

\begin{esempio}
\label{esem:diff_prodottocostante}
Calcola la derivata della funzione \(f(x)=-\dfrac{3}{2}x+5\).
\[\Deriv{-\dfrac{3}{2}x+5} = 
  -\dfrac{3}{2} \cdot \Deriv{x} + \Deriv{5} = 
  -\dfrac{3}{2} \cdot 1 + 0 = -\dfrac{3}{2}\]
\end{esempio}

E anche la sua generalizzazione:
\begin{esempio}
\label{esem:diff_prodottocostante}
Calcola la derivata della funzione \(f(x)=mx+q\).
\[\Deriv{mx+q} = m \cdot \Deriv{x} + \Deriv{q} = m \cdot 1 + 0 = m\]
\end{esempio}

\begin{esempio}
 %valore assoluto
È derivabile la funzione \(f(x)=-\dfrac{1}{2}|x-2|+3\)?

La funzione contiene un valore assoluto: 
% seguendo l’esempio 
% \ref{esempio:diff01_derimodulo} 
riscriviamola come una funzione definita 
a tratti:
\[f(x) = -\dfrac{1}{2}|x -2| +3 =
\sistema{
 +\dfrac{x-2}{2} +3 &\text{ per } x-2 < 0  \\[.5em]
 -\dfrac{x-2}{2} +3 &\text{ per } x-2 \ge 0
} 
=
\sistema{
 +\dfrac{1}{2}x +2 &\text{ per } x < 2 \\[.5em]
 -\dfrac{1}{2}x +4 &\text{ per } x \ge 2
}\]

\affiancati{.33}{.65}{
\begin{inaccessibleblock}
  [Funzione con modulo e sua derivata]
  \derivamodulodue
\end{inaccessibleblock}
}{
Si tratta di due semirette che si uniscono in \(\punto{2}{3}\). L'equazione
di ciascuna di loro è una funzione lineare e già sappiamo che per ognuno dei 
rami la pendenza esiste ed è il coefficiente angolare. Quindi il calcolo
è immediato:
\[f'(x)= \begin{cases}
 +\dfrac{1}{2} &\text{ per } x < 2 \\[.5em]
 -\dfrac{1}{2} &\text{ per } x > 2
\end{cases}\]
}

\vspace{.5em}
Quale è la pendenza della funzione per \(x=2\)?\\
Poiché che la parte standard del Rapporto Differenziale 
\begin{itemize} [nosep]
\item per tutti i valori negativi di \(dx\) vale \(+\dfrac{1}{2}\),
\item per tutti i valori positivi di \(dx\) vale \(-\dfrac{1}{2}\),
\end{itemize}
la funzione \(f\) non ha pendenza (o derivata) per \(x = 2\), 
cioè \(f'(2)\) non esiste. 
Vedi anche l'esempio \ref{esempio:differenziazione_derimodulo}.

Dato che a sinistra di \(x = 2\) la funzione ha pendenza \(+\dfrac{1}{2}\) 
e a destra ha pendenza \(-\dfrac{1}{2}\), possiamo affermare che in \(2\)
la funzione ha un \emph{punto angoloso}.
\end{esempio}

\subsubsection{Derivata di una funzione potenza con esponente naturale}
\label{subsubsec:derivata_f_potenza_N}

Abbiamo già visto le prime potenze di \(x\): 
\(\Deriv{x^0} = 0; \quad \Deriv{x^1} = 1; \quad \Deriv{x^2} = 2x\).
Come calcolare la derivata delle potenze successive? 
È possibile ricavare una regola per \(\Deriv{x^n}\)?

\begin{teorema}
\label{diff01_teoderpotenza}
La derivata della funzione potenza \(f(x)= x^n\), con esponente 
naturale, è: \(\Deriv{x^n}=~nx^{n-1}\).
\end{teorema}

\paragraph{Derivata della funzione potenza con il rapporto differenziale}

% \noindent Ipotesi: \(f(x)=x^n\), con \(n\in \N\) .\tab Tesi: 
% \(f'(x)=nx^{n-1}\). 
\begin{proof}
%  Una funzione potenza è una funzione che dà come risultato la potenza 
% della variabile indipendente:
% \[f: x \mapsto x^n \quad \text{o} \quad 
%   f(x)=x^n \quad \text{o} \quad 
%   y = x^n\]
Ricaviamo per gradi il differenziale della funzione potenza \(f(x)=x^n\)
per un generico \(x\).

% Iniziamo dai casi già noti \(f(x)=x\) e \(f(x)=x^2\) e esaminiamo i 
% successivi aumentando progressivamente l'esponente. 
% Ancora una volta non inseriamo l'indicazione relativa a \(x_0\), perché 
% ci siamo accorti che è inutile. 
% Infatti i risultati cambiano al cambiare di \(x_0\), il che equivale a 
% dire che dipendono da \(x\).
\begin{align*}
  d(x^0) &=(x+dx)^0-x^0 = 1 -1 = 0\\
  d(x^1) &=x+dx-x =dx\\
  d(x^2) &=(x+dx)^2-x^2 = x^2 +2xdx +(dx)^2 -x^2 = 2xdx +(dx)^2\\
  d(x^3) &=(x+dx)^3-x^3 = \quad{x^3+3x^2dx+3x(dx)^2+(dx)^3}-x^3=
                      3x^2dx+3x(dx)^2+(dx)^3\\
  d(x^4) &=(x+dx)^4-x^4 = 
           \quad{x^4+4x^3dx+6x^2(dx)^2+4x(dx)^3+(dx)^4}-x^4=\\
         &=4x^3dx+6x^2(dx)^2+4x(dx)^3+(dx)^4  \\
  d(x^5) &= (x+dx)^5-x^5 = \quad{x^5+5x^4dx+\dots +(dx)^5}-x^5=
         5x^4dx+\dots+(dx)^5\\
  \dots  &= \dots\\
  d(x^n) &= (x+dx)^n-x^n = \quad{x^n+nx^{n-1}dx+\dots +(dx)^n}-x^n=
         nx^{n-1}dx+\dots+(dx)^n
\end{align*}
Possiamo osservare che:
\begin{enumerate}
\item 
All'aumentare dell'esponente della potenza, il 
differenziale diventa un'espressione sempre più lunga e complicata.
\item 
Il termine non infinitesimo si annulla sempre quindi l'incremento della 
funzione è infinitesimo per ogni \(x\), queste funzioni 
sono dunque continue in tutto \(\R\).
\end{enumerate}
Risolviamo il problema rappresentato dal punto 1. ragruppando tutti gli 
infinitesimi di ordine superiore al primo, in un unico termine: \(o(dx)\).

Il punto 2. ci assicura che il Rapporto Differenziale sarà un numero 
finito di cui possiamo calcolare la parte standard.
E dato che questa non dipende dal particolare infinitesimo \(dx\) 
utilizzato, la parte standard del Rapporto Differenziale è la derivata della 
funzione.


% I vari differenziali hanno un'espressione sempre più lunga, ma se 
% consideriamo la parte principale di ogni espressione, cioè se trascuriamo 
% gli infinitesimi di ordine superiore, il risultato è molto semplice.
% 
% % \newpage %----------------------------------------
% Quindi se invece del valore esatto ci accontentiamo della parte principale, 
% dato che è indistinguibile, concludiamo:
% \nopagebreak
% \begin{align*}
%   d(x) &=x+dx-x =dx\\
%   d(x^2) &=2xdx +(dx)^2 \sim 2xdx\\
%   d(x^3) &=3x^2dx+3x(dx)^2+(dx)^3 \sim 3x^2dx\\
%   d(x^4) &=4x^3dx+6x^2(dx)^2+4x(dx)^3+(dx)^4 \sim 4x^3dx\\
%   d(x^5) &=5x^4dx+10x^3(dx)^2+10x^2(dx)^3+5x(dx)^4+(dx)^5 \sim 5x^4dx\\
% %   d(x^6) &=6x^5dx+\dots+(dx)^6 \sim 6x^5dx\\
% %   d(x^7) &=7x^6(dx)+\dots+(dx)^7 \sim 7x^6dx\\
%   \dots &= \dots\\
%   d(x^{10}) &=10x^9(dx)+\dots+(dx)^{10} \sim 10x^9dx\\
%   \dots &= \dots\\
%   d(x^n) &=nx^{n-1}(dx)+\dots+(dx)^{n} \sim nx^{n-1}dx\\    
% \end{align*}
% \nopagebreak
% \begin{align*}
%   d(x^0) &= 0\\
%   d(x^1) &= dx\\
%   d(x^2) &= 2xdx +o(dx)\\
%   d(x^3) &= 3x^2dx +o(dx)\\
%   d(x^4) &= 4x^3dx +o(dx)\\
%   d(x^5) &= 5x^4dx +o(dx)\\
% %   d(x^6) &= 6x^5dx +o(dx)\\
% %   d(x^7) &= 7x^6dx +o(dx)\\
%   \dots &= \dots\\
%   d(x^{10}) &= 10x^9dx +o(dx)\\
%   \dots &= \dots\\
%   d(x^n) &= nx^{n-1}dx +o(dx)\\    
% \end{align*}
\begin{multicols}{2}
\begin{itemize}
\item \(\Deriv{x^0} = \pst{\dfrac{0}{dx}} = 0\)
\item \(\Deriv{x^1} = \pst{\dfrac{dx}{dx}} = 1\)
\item \(\Deriv{x^2} = \pst{\dfrac{2xdx +o(dx)}{dx}} = 2x\)
\item \(\Deriv{x^3} = \pst{\dfrac{3xdx +o(dx)}{dx}} = 3x^2\)
\item \(\Deriv{x^4} = \pst{\dfrac{4xdx +o(dx)}{dx}} = 4x^3\)
% \item \(\Deriv{x^5} = \pst{\dfrac{5xdx +o(dx)}{dx}} = 5x^4\)
% \item \(\Deriv{x^6} = \pst{\dfrac{6xdx +o(dx)}{dx}} = 6x^5\)
% \item \(\Deriv{x^7} = \pst{\dfrac{7xdx +o(dx)}{dx}} = 7x^6\)
\item \(\dots = \dots\)
\item \(\Deriv{x^{10}} = \pst{\dfrac{10xdx +o(dx)}{dx}} = 10x^9\)
\item \(\dots = \dots\)
% \item \(\Deriv{x^n} = \pst{\dfrac{nxdx +o(dx)}{dx}} = nx^{n-1}\)
\end{itemize}
\end{multicols}
Possiamo concludere che:

\hspace{30mm} 
\(\Deriv{x^n} = \pst{\dfrac{d(x^n)}{dx}} =
\pst{\dfrac{nxdx +o(dx)}{dx}} = nx^{n-1}\)
% Ora che il meccanismo è chiaro, abbiamo la regola per il differenziale di 
% questa 
% funzione: \(d(x^n) \sim nx^{n-1}dx\) 
% e la sfruttiamo per ricavare il Rapporto Differenziale.
% \[ \forall dx\ne 0:\quad
%  RD =\dfrac{d(x^n)}{dx}\sim \dfrac{nx^{n-1}dx}{dx}=nx^{n-1}\]
% Il Rapporto Differenziale è indistinguibile da una quantità finita 
% indipendente da \(dx\), possiamo calcolarne la parte standard. 
% Questa operazione, fra l'altro, ci permette di eliminare gli infinitesimi 
% di ordine superiore e quindi di scrivere il segno di \(=\) al posto del 
% segno \(\sim\).
% \[ f'(x)=\pst{\dfrac{d(x^n)}{dx}}=\pst{nx^{n-1}}=nx^{n-1}\]
\end{proof}


\paragraph{Derivata della funzione potenza con il teorema del prodotto}
~

Lo stesso teorema può essere dimostrato usando il teorema della derivata 
del prodotto.

\begin{proof}
Nella seguente tabella ogni riga usa il risultato della riga precedente:

\begin{itemize} [nosep]
\item \(\Deriv{x^0} = 0\)
\item \(\Deriv{x^1} = 1\)
\item 
\(\Deriv{x^2} = \Deriv{x \cdot x} = 
  \Deriv{x} \cdot x +x \cdot \Deriv{x} = 1 \cdot x +x \cdot 1 = 2x\)
\item 
\(\Deriv{x^3} = \Deriv{x^2 \cdot x} = 
  \Deriv{x^2} \cdot x +x^2 \cdot \Deriv{x} = 2x \cdot x +x^2 \cdot 1 = 3x^2\)
\item 
\(\Deriv{x^4} = \Deriv{x^3 \cdot x} = 
  \Deriv{x^3} \cdot x +x^3 \cdot \Deriv{x}=3x^2 \cdot x +x^3 \cdot 1=4x^3\)
% \item 
% \(\Deriv{x^5} = \Deriv{x^4 \cdot x} = 
%   \Deriv{x^4} \cdot x +x^4 \cdot \Deriv{x}=
%     4x^3 \cdot x +x^4 \cdot 1=5x^4\)
\item \(\dots = \dots\)
\item 
\(\Deriv{x^{10}} = \Deriv{x^9 \cdot x} = 
  \Deriv{x^9} \cdot x +x^9 \cdot \Deriv{x}=9x^8 \cdot x +x^9 \cdot 1=10x^9\)
\item \(\dots = \dots\)
% \item \(\Deriv{x^n} = \pst{\dfrac{nxdx +o(dx)}{dx}} = nx^{n-1}\)
\end{itemize}
Possiamo concludere che:

\hspace{0mm} 
\(\Deriv{x^n} = \Deriv{x^{n-1} \cdot x} = 
  \Deriv{x^{n-1}} \cdot x +x^{n-1} \cdot \Deriv{x}= \)

\hspace{10.5mm} 
\(= \tonda{n-1}x^{n-2} \cdot x +x^{n-1} \cdot 1 
  = \tonda{n-1}x^{n-1} +x^{n-1} = nx^{n-1}\)
\end{proof}

% \begin{esempio}
% \label{esem:diff_quadrato}
% Calcola la derivata della funzione \(f(x)=x^2\).
% 
% Applicando la regola del prodotto:
% \(f = g \cdot h\) \quad con: \quad \(g(x) = x\) \quad e \quad \(h(x) = x\)
% \[f' = g' \cdot h + g \cdot h' = 
% \Deriv{x} \cdot x + x \cdot \Deriv{x} =
% 1 \cdot x + x \cdot 1 = 2x\]
% \end{esempio}
% 
% \begin{esempio}
% \label{esem:diff_quadrato}
% Calcola la derivata della funzione \(f(x)=x^3\).
% 
% Sempre applicando la regola del prodotto:
% \(f = g \cdot h\) \quad con: \quad \(g(x) = x^2\) \quad e \quad \(h(x) = x\)
% \[f' = g' \cdot h + g \cdot h' = 
% \Deriv{x^2} \cdot x + x^2 \cdot \Deriv{x} =
% 2x \cdot x + x^2 \cdot 1 = 3x^2\]
% \end{esempio}
% 
% \begin{esempio}
% \label{esem:diff_quadrato}
% Calcola la derivata della funzione \(f(x)=x^4\).
% 
% Sempre applicando la regola del prodotto:
% \(f = g \cdot h\) \quad con: \quad \(g(x) = x^3\) \quad e \quad \(h(x) = x\)
% \[f' = g' \cdot h + g \cdot h' = 
% \Deriv{x^3} \cdot x + x^3 \cdot \Deriv{x} =
% 3x^2 \cdot x + x^3 \cdot 1 = 4x^3\]
% \end{esempio}
% 
% 

% \begin{esempio}
%  Deriva la funzione \(f(x)=x^2\) con la regola del prodotto. 
% \begin{align*}
%  &f(x) = x^2=x\cdot x;\quad \text{ funzione 1: } a(x)=x\quad 
% \text{ funzione 2: 
% } 
% b(x)=x.\\
%  &\text{Derivate dei fattori: }a'(x)=1;\quad b'(x)=1.\\
%  &\text{Regola: } f'(x)=a'(x)\cdot b(x)+a(x)\cdot b'(x),\quad 
% \text{ quindi:}\\
%  &f'(x)=1\cdot x + x\cdot 1= 2x.
%  \end{align*}
%  \end{esempio}
%  Ora sviluppiamo alcuni casi successivi nella speranza di individuare una 
% regola generale.
%  \begin{esempio}
%  Deriva la funzione \(f(x)=x^3\) con la regola del prodotto. 
% \begin{align*}
%  &f(x) = x^3=x^2\cdot x; \quad \text{ funzione 1: } a(x)=x^2\quad \text{ 
% funzione 2: } 
% b(x)=x.\\
%  &\text{Derivate dei fattori: }a'(x)=2x;\quad b'(x)=1.\\
%  &\text{Regola: } f'(x)=a'(x)\cdot b(x)+a(x)\cdot b'(x),\quad 
% \text{ quindi:}\\
%  &f'(x)=2x\cdot x + x^2\cdot 1= 3x^2.
%  \end{align*}
%  \end{esempio}
%  
%  \begin{esempio}
%  Deriva la funzione \(f(x)=x^4\) con la regola del prodotto. 
% \begin{align*}
%  &f(x) = x^4=x^3 \cdot x;\quad \text{ funzione 1: } a(x)=x^3\quad \text{ 
% funzione 2: } b(x)=x.\\
%  &\text{Derivate dei fattori: }a'(x)=3x^2;\quad b'(x)=1.\\
%  &\text{Regola: } f'(x)=a'(x)\cdot b(x)+a(x)\cdot b'(x),\quad 
% \text{ quindi:}\\
%  &f'(x)=3x^2\cdot x + x^3\cdot 1= 4x^3.
%  \end{align*}
%  \end{esempio}

% In effetti c'è una regolarità nei risultati:
% \[\Deriv{x} = 1; \quad \Deriv{x^2} = 2x; \quad \Deriv{x^3} = 3x^2; \quad 
%   \Deriv{x^4} = 4x^3;\]
% e possiamo avanzare la congettura che 
% \[\Deriv{x^5} = 5x^4; \quad \Deriv{x^6} = 6x^5; \quad \dots\]
% e in generale:
% \[\Deriv{x^n} = nx^{n-1}\]
% \end{proof}

Come esempio di derivata della funzione potenza, consideriamo \(f(x)=x^3\)
e il suo grafico nel piano cartesiano. 

\begin{inaccessibleblock}
  [Cubica con pendenze e sua derivata]
\hspace{-20mm}\affiancati{.55}{.43}{
\begin{center} \scalebox{.6}{\cubica} \end{center}
}{
\begin{center} \scalebox{.6}{\derivatacubica} \end{center}
}
\end{inaccessibleblock}
\label{img:diff_cubica_con_pendenze}
\begin{center} Il grafico di \(y=x^3\), con alcune sue tangenti, e 
la sua derivata: \(y=3x^2\).\end{center}

I due rami del grafico sono simmetrici rispetto all'origine
% e quindi lo sono anche le pendenze delle tangenti
. 
Considerando le \(x\) crescenti, quindi da sinistra verso destra, le 
pendenze delle tangenti sono sempre positive, all'inizio molto 
accentuate, poi diminuiscono fino a \(m=0\). Poi riprendono a crescere, 
in maniera sempre più evidente. Il grafico della funzione \(f'(x)=3 x^2\) 
ha infatti la forma di una parabola simmetrica rispetto all'asse \(y\).

Ora disponiamo di uno strumento potente che, insieme ai precedenti, 
ci permette di calcolare la derivata delle funzioni polinomiali 
senza dover calcolare il Rapporto Differenziale e la parte standard. 

\begin{esempio}
\label{esem:diff_prodottocostante}
Calcola la derivata della funzione 
\(f(x)=-\dfrac{4}{15}x^5 +\dfrac{2}{3}x^3-13x\).
\begin{align*}
\Deriv{-\dfrac{4}{15}x^5+ \dfrac{2}{3}x^3-13x} &= 
-\dfrac{4}{15} \cdot \Deriv{x^5} + 
 \dfrac{2}{3} \Deriv{x^3} -13 \cdot \Deriv{x}=\\
&=-\dfrac{4}{15} \cdot 5x^4 + \dfrac{2}{3} 3x^2 -13 =
-\dfrac{4}{3} x^4 + 2x^2 -13
\end{align*}
\end{esempio}

% \begin {esempio}
% \(f(x)= -\dfrac{4}{15}x^5+ \dfrac{2}{3}x^3-13kx\).
%  \begin{align*}
%  &f(x) = -\dfrac{4}{15}x^5+ \dfrac{2}{3}x^3-13kx\\
%  &\text{funzione 1:} a(x)=-\dfrac{4}{15}x^5\quad \text{ funzione 2: } 
% b(x) =+ \dfrac{2}{3}x^3\quad \text{ funzione 3: } c(x) =-13kx.\\
%  &\text{Derivate degli addendi: }a'(x)=\dfrac{4}{3}x^4;\quad b'(x)=+2 x^2;
%  \quad c'(x)=-13k.\\
%  &\text{Quindi: } f'(x)=\dfrac{4}{3}x^4+2x^2-13k.
%  \end{align*}
% \end {esempio}

\subsection{Derivata del quoziente di funzioni}
\label{subsec:differenziazione_derivataquoziente}
\begin{teorema}
 Se una funzione derivabile è data dal rapporto fra due funzioni derivabili, 
con il denominatore non nullo, la sua derivata si ottiene calcolando
 la differenza fra due prodotti (la derivata del numeratore per il 
 denominatore meno il numeratore per la derivata del 
denominatore) e dividendo il risultato per il quadrato del denominatore.
\[\Deriv{\dfrac{\mathit{S}(x)}{b(x)}} =
\dfrac{\Deriv{\mathit{S}(x)} \cdot b(x)-\mathit{S}(x) \cdot \Deriv{b(x)}}
      {\tonda{b(x)}^2}\]
\end{teorema}
\noindent Ipotesi: \(a(x)=\dfrac{\mathit{S}(x)}{b(x)}\), 
con \(b(x)\neq 0\) \tab 
Tesi: 
\(a'(x) = \dfrac{\mathit{S}'(x) \cdot b(x)-\mathit{S}(x) \cdot b'(x)}
             {\tonda{b(x)}^2}\)

\begin{proof}
Ricorriamo alla geometria anche in questo caso: nel disegno 
\(\mathit{S}(x)\) è la funzione che genera i valori per l'area, \(b(x)\) 
e \(a(x)\) generano le possibili basi e altezze. 
Per prima cosa calcoliamo \(d\mathit{S}(x)\).

\noindent \begin{minipage}[]{.48 \textwidth}
Essendo \(\mathit{S}(x)=b(x) \cdot a(x)\) allora:\\ 
\(a(x)=\dfrac{\mathit{S}(x)}{b(x)}\), con \(b(x)\neq 0\).

L'incremento infinitesimo \(da(x)\) è visibile solo applicando un 
microscopio non standard: si tratta dell'altezza della fascia superiore 
colorata. 
Questo incremento si può calcolare dividendo il rettangolo superiore dello 
gnomone per la base del rettangolo.
Il rettangolo superiore dello gnomone si ottiene togliendo a tutto lo 
gnomone \(d\mathit{S}\), il rettangolo destro di area \(a\cdot db\) e 
il rettangolino, in alto a destra,
\end{minipage} 
 \hfill
 \begin{minipage}[]{.48 \textwidth}
 \begin{center}
 \begin{inaccessibleblock}
  [Altezza rettangolo con gnomone infinitesimo.]
  \differenzialerapporto
 \end{inaccessibleblock}
 \end{center}
\end{minipage}
anch'esso infinitesimo di area 
\(da \cdot db = d\tonda{\dfrac{\mathit{S}}{d}} \cdot db \). 
Dunque:
\begin{align*}
da(x) &= d\quadra{\frac{\mathit{S}(x)}{b(x)}}=
 \frac{d\mathit{S}(x) - a(x) \cdot db(x) - db(x) \cdot da(x)}{b(x)}=\\
 &=\frac{d\mathit{S}(x) - \dfrac{\mathit{S(x)}}{b(x)} \cdot db(x) - 
          db(x) \cdot da(x)}{b(x)}=\\
 &=\frac{\dfrac{d\mathit{S}(x) \cdot b(x) - 
                \mathit{S(x)} \cdot db(x) - 
                db(x) \cdot da(x) \cdot b(x)}
               {b(x)}}
        {b(x)}=\\
 &=\frac{d\mathit{S}(x) \cdot b(x) - \mathit{S(x)} \cdot db(x) -
              db(x) \cdot da(x) \cdot b(x)}
        {\tonda{b(x)}^2} \sim 
 \frac{d\mathit{S}(x) \cdot b(x) - \mathit{S(x)} \cdot db(x)}
      {\tonda{b(x)}^2}
\end{align*}
Come si vede, l'ultimo termine al numeratore nel risultato manca in quanto 
è infinitesimo di ordine superiore ed è quindi trascurabile rispetto agli 
altri termini. 
Vediamo ora il Rapporto differenziale e la derivata.
\begin{align*}
RD &=\dfrac{da(x)}{dx} = 
   \dfrac{d\mathit{S}(x) \cdot b(x) - 
          \mathit{S(x)} \cdot db(x) -
          db(x) \cdot da(x) \cdot b(x)}
         {\tonda{b(x)}^2}
   \cdot \frac{1}{dx} = \\
&= \dfrac{d\mathit{S}(x) \cdot b(x) - 
          \mathit{S(x)} \cdot db(x) -
          db(x) \cdot da(x) \cdot b(x)}
         {dx} 
   \cdot \dfrac{1}{\tonda{b(x)}^2} =\\
&= \tonda{\dfrac{d \mathit{S}(x) \cdot b(x)}{dx} -
          \frac{\mathit{S(x)} \cdot  db(x)}{dx} -
          \frac{db(x) \cdot da(x) \cdot b(x)}{dx}}
   \cdot \dfrac{1}{\tonda{b(x)}^2}= \\
&= \tonda{\dfrac{d \mathit{S}(x) \cdot b(x)}{dx} -
          \frac{\mathit{S(x)} \cdot db(x)}{dx} -
          \frac{db(x) \cdot da(x) \cdot b(x)}{dx}}
   \cdot \dfrac{1}{\tonda{b(x)}^2}= \\
&= \tonda{\dfrac{d \mathit{S(x)}}{dx}b(x) -
          \mathit{S(x)}\dfrac{db(x)}{dx}}
   \cdot \dfrac{1}{\tonda{b(x)}^2}
\end{align*}
\begin{align*}
 RD  &=\dfrac{da(x)}{dx}\sim  \dfrac{\dfrac{d\mathit{S}(x) \cdot b(x) - 
\mathit{S(x)} \cdot db(x)}{\tonda{b(x)}^2}}{dx}=
\dfrac{d\mathit{S}(x) \cdot b(x) - \mathit{S(x)} \cdot 
 db(x)}{dx}\cdot\dfrac{1}{\tonda{b(x)}^2}=\\
&=\tonda{\dfrac{d \mathit{S}(x) \cdot b(x)}{dx}-
  \frac{\mathit{S(x)} \cdot  db(x)}{dx} }\cdot\dfrac{1}{\tonda{b(x)}^2}= 
\dfrac{\dfrac{d \mathit{S(x)}}{dx}b(x)-
\mathit{S(x)}\dfrac{db(x)}{dx}}{\tonda{b(x)}^2}
\end{align*}
Poiché \(\mathit{S(x)}\) e \(b(x)\) sono per ipotesi derivabili, esistono le 
parti standard dei Rapporti Differenziali, indipendenti dai \(dx\) non nulli 
usati, e si ottiene così 
\(a'(x)=\dfrac{\mathit{S'(x)}\cdot b(x)-
      \mathit{S(x)} \cdot b'(x)}{(b(x))^2}\).

Tornando alla notazione abituale, la regola della derivata del rapporto di 
funzioni diventa:
\[f = \dfrac{g}{h},\quad\text{ con } h \ne 0 \qquad
 f' = \dfrac{g' \cdot h - g \cdot h'}{h^2}\]
% \begin{align*}
%  f(x) &=\dfrac{g(x)}{h(x)},\quad\text{ con } h(x)\ne 0.\\
%  f'(x) &=\dfrac{g'(x)\cdot h(x)-g(x)\cdot h'(x)}{(h(x))^2}.
% \end{align*}
\end{proof}
Ora che sappiamo derivare un rapporto fra funzioni, non avremo 
problemi nel seguente esercizio.
\begin{esempio}
 Calcola la pendenza della funzione \(f(x)=\dfrac{x^2}{x-2}\) nel punto 
di ascissa \(x=1\).

\affiancati{.38}{.60}{
\vspace{-.5em}
\begin{inaccessibleblock}
[Ramo di una funzione con tangente nel punto (1; -1)]
\derivaomografica
\end{inaccessibleblock}
}{
Applichiamo direttamente la regola 
\ref{subsec:differenziazione_derivataquoziente}.
Dato che il numeratore e il 
denominatore sono entrambe funzioni derivabili e che \(x=1\) non annulla il 
denominatore, si ha: 
\begin{align*}
f'(x)&=\dfrac{2x\cdot(x-2)-x^2\cdot 1}{(x-2)^2}=\dfrac{x^2-4x}{(x-2)^2}\\
f'(1)&=\dfrac{1^2-4\cdot 1}{(1-2)^2}=\dfrac{-3}{1}=-3
\end{align*}
\(f'(x)\) è la derivata, \(f'(1)\) è la derivata calcolata per 
\(x=1\), cioè la pendenza della curva nel punto \(\punto{1}{f(1)}\).

Per \(x = 1\), la curva ha pendenza \(m = -3\).
}

\subsection{Derivata del reciproco di una funzione}
\label{subsubsec:derivata_reciproco_f} 

\begin{teorema}
\label{teo:diff_reciproco_f}
La derivata del reciproco di una funzione è uguale alla derivata della 
funzione fratto il quadrato della funzione stessa.
\[\Deriv{\dfrac{1}{g(x)}} = \dfrac{g'(x)}{\tonda{g(x)}^2}\]
\end{teorema}
% \noindent Ipotesi: \(f(x)=x^z\), con \(z\in \Z\), \(x\ne 0\) \tab 
% Tesi: \(f'(x)=zx^{z-1}\).
\begin{proof}
Se \(f(x) = \dfrac{1}{g(x)}\) allora il differenziale diventa:
\[d \quadra{\frac{1}{g(x)}} = \frac{1}{g(x+\epsilon)} - \frac{1}{g(x)} =
  \frac{g(x) - g(x+\epsilon)}{g(x+\epsilon) \cdot g(x)} =
  - \frac{g(x+\epsilon) - g(x)}{g(x+\epsilon) \cdot g(x)}\]
Possiamo ora calcolare il Rapporto Differenziale:
\[RD = 
  -\frac{g(x+\epsilon) - g(x)}{g(x+\epsilon) \cdot g(x)} \cdot 
  \frac{1}{\epsilon} =
  -\frac{g(x+\epsilon) - g(x)}{\epsilon} \cdot 
  \frac{1}{g(x+\epsilon) \cdot g(x)}\]
Se come da ipotesi \(g\) è derivabile, il Rapporto differenziale è un numero 
finito e possiamo calcolarne la parte standard:

\hspace{20mm}
\(\pst{RD} = 
  \pst{-\dfrac{g(x+\epsilon) - g(x)}{\epsilon} \cdot 
  \dfrac{1}{g(x+\epsilon) \cdot g(x)}} =
%   \pst{-\dfrac{g(x+\epsilon) - g(x)}{\epsilon}} \cdot 
%   \pst{\dfrac{1}{g(x+\epsilon) \cdot g(x)}} =
  -\dfrac{g'(x)}{\tonda{g(x)}^2}\)
\end{proof}


\subsection{Derivata di una funzione potenza con esponente intero}
\label{subsubsec:derivata_f_potenza_Z} 

Abbiamo già incontrato le funzioni \(f: x \mapsto \frac{1}{x}\) e 
\(f: x \mapsto \frac{1}{x^2}\) calcolando le derivate 
usando la definizione.

Ora possiamo rivedere questo tipo di funzioni utilizzando il teorema 
\ref{teo:diff_reciproco_f}.

\begin{esempio}
Calcola le derivate di: 
\(\dfrac{1}{x}; \quad \dfrac{1}{x^2}; \quad 
\dfrac{1}{x^3}; \quad \dfrac{1}{x^4}\)

\affiancati{.49}{.49}{
Frazioni
\begin{align*}
\Deriv{\dfrac{1}{x}} &= -\frac{\Deriv{x}}{x^2} = -\frac{1}{x^2} \\
\Deriv{\dfrac{1}{x^2}} &= -\frac{\Deriv{x^2}}{x^4} = 
  -\frac{2x}{x^4} = -\frac{2}{x^3} 
\end{align*}
}{
\begin{align*}
\Deriv{\dfrac{1}{x^3}} &= -\frac{\Deriv{x^3}}{x^6} = 
  -\frac{3x^2}{x^6} = -\frac{3}{x^4} \\
\Deriv{\dfrac{1}{x^4}} &= -\frac{\Deriv{x^4}}{x^8} = 
  -\frac{4x^3}{x^8} = -\frac{4}{x^5} 
\end{align*}
}
\end{esempio}

Tenendo conto del significato delle potenze a esponente negativo, 
\(\dfrac{1}{x^n} = x^{-n}\), possiamo riscrivere i risultati precedenti: 
\[
\Deriv{x^{-1}} = -x^{-2} \qquad
\Deriv{x^{-2}} = -2x^{-3} \qquad
\Deriv{x^{-3}} = -3x^{-4} \qquad
\Deriv{x^{-4}} = -4x^{-5} 
\]
Osserviamo che si ottiene lo stesso risultalo estendendo il 
teorema sulle potenze a esponente naturale \ref{diff01_teoderpotenza},
possiamo avanzare la congettura che la regola di derivazione per le 
potenze a esponente naturale valga anche per le potenze a esponente 
intero.

% \paragraph{Esempi sulle funzioni potenza \(f(x)=x^z\), con \(z\) intero 
% negativo (e \(x\ne 0\)).}
% 
% \begin{esempio}
%  Applichiamo la regola del rapporto alla funzione: 
% \(f(x) = \dfrac{g(x)}{h(x)} = \dfrac{1}{x}\), con \(x\ne 0\):
% \[
% f'= \dfrac{g'\cdot h - g \cdot h'}{h^2} =
%     \dfrac{\Deriv{1} \cdot x - 1 \cdot \Deriv{x}}{x^2} =
%     \dfrac{0 \cdot x - 1 \cdot 1}{x^2} = -\dfrac{1}{x^2}.
% \]
% % in cui poniamo \(g(x)=1\) e \(h(x)=x\), per cui 
% % \(g'(x)=0\) e \(h'(x)=1\).
% % \[
% % f'(x)= \dfrac{g'(x)\cdot h(x)-g(x)\cdot h'(x)}{(h(x))^2}=\dfrac{0\cdot 
% % x-1\cdot 1}{x^2}=-\dfrac{1}{x^2}.
% % \]
% \end{esempio}
% 
% \begin{esempio}
% Proviamo ora con \(f(x)= \dfrac{1}{x^2}\), con \(x\ne 0\): 
% \[
% f'(x)= \dfrac{g'(x)\cdot h(x)-g(x)\cdot h'(x)}{(h(x))^2}=
% \dfrac{0\cdot x^2-1\cdot 2x}{x^4}=-\dfrac{2}{x^3}.
% \]
% \end{esempio}
% 
% \begin{esempio}
% \label{esempio:derivata_x_alla_z}
% Prendiamo ora il caso generale \(f(x)= x^{-n}=\dfrac{1}{x^n}\), 
% con \(n\in \N\) e con \(x\ne 0\).  
% \[
% f'(x)= \dfrac{g'(x)\cdot h(x)-g(x)\cdot h'(x)}{(h(x))^2}=
% \dfrac{0\cdot x^n-1\cdot nx^{n-1}}{x^{2n}}=-nx^{n-1-2n}=-nx^{-n-1}.
% \]
% \end{esempio}
% 
% Quest'ultimo esempio è la dimostrazione del teorema che segue.

\begin{teorema}
\label{teo:diff01_deri_x_alla_z}
La derivata della funzione potenza \(f(x)= x^z\), con esponente 
intero, è: 
\[\Deriv{x^z}=~zx^{z-1}\]
\end{teorema}
% \noindent Ipotesi: \(f(x)=x^z\), con \(z\in \Z\), \(x\ne 0\) \tab 
% Tesi: \(f'(x)=zx^{z-1}\).
\begin{proof}
Se \(z \geqslant 0\), vale il teorema già dimostrato. 
Se invece \(z < 0\) possiamo considerare un numero naturale \(n\) 
tale che \(n = -z\) e \(z = -n\), quindi: 
\(x^z = \dfrac{1}{x^n}\).
Applichiamo il teorema sulla funzione reciproca 
\ref{teo:diff_reciproco_f}:

\hspace{13mm}
\(\Deriv{z} = 
  \Deriv{\dfrac{1}{x^n}} = -\dfrac{\Deriv{x^n}}{x^{2n}} = 
  -\dfrac{nx^{n-1}}{x^{2n}} = -\dfrac{n}{x^{n+1}} = -n{x^{-n-1}} =
  z{x^{z-1}}\)
\end{proof}

Con un procedimento che richiede delle conoscenze che non abbiamo ancora si 
può dimostrare che la precedente regola vale per esponenti qualunque.

\begin{teorema}
\label{teo:diff01_deri_x_alla_r}
La derivata della funzione potenza \(f(x)= x^r\), con esponente 
reale, è: 
\[\Deriv{x^r}=~rx^{r-1}\]
\end{teorema}

% %\begin{figure}[h]
% \begin{inaccessibleblock}
%   %\begin{center}
%  \begin{minipage}[]{.40 \textwidth}
%  % \caption{}
%  \end{minipage} 
%  \hfill
%  \begin{minipage}[]{.58 \textwidth}
% Applichiamo direttamente la regola 
% \ref{subsec:differenziazione_derivataquoziente}.
% Dato che il numeratore e il 
% denominatore sono entrambe funzioni derivabili e che \(x=1\) non annulla il 
% denominatore, si ha: 
% 
% \begin{align*}
% f'(x)&=\dfrac{2x\cdot(x-2)-x^2\cdot 1}{(x-2)^2}=\dfrac{x^2-4x}{(x-2)^2}.\\
% f'(1)&=\dfrac{1^2-4\cdot 1}{(1-2)^2}=\dfrac{-3}{1}=-3.
% \end{align*}
% \(f'(x)\) è la derivata. \(f'(1)\) è la derivata calcolata per 
% \(x=1\), cioè la pendenza della curva nel punto \((1,\ f(1)).\)
% \end{minipage}
% %\end{center}
% \end{inaccessibleblock}
%\label{}
%\end{figure} 
% 
% Per \(x=1\), la curva ha pendenza \(m=-3\).
\end{esempio}

\subsubsection{Sintesi provvisoria}
\label{subsubsec:diff01_sintesiprovvisoria}
Abbiamo dimostrato le regole di derivazione per alcune funzioni elementari: 
\begin{enumerate} [noitemsep]
 \item \(\Deriv{k} = 0\) \tab  derivata di una 
costante;
 \item \(\Deriv{x} = 1\) \tab derivata della funzione identica;
 \item \(\Deriv{x^k} = k x^{ka-1}\) 
\tab  derivata della funzione potenza (\(k \in \R\)).
\end{enumerate}
\begin{osservazione}
\label{oss:regola3}
La regola 3 anticipa un risultato giustificato per ora solo con 
\(k \in \Z\), 
ma che in realtà ha valore con qualsiasi esponente reale 
(vedi pag.~\pageref{teo:funzione_potenza_generica}).
\end{osservazione}

Conosciamo anche le regole di derivazione per alcune funzioni che sono 
somma, prodotto o quoziente di altre:
\begin{enumerate} [noitemsep]
\item \(\Deriv{k\cdot f}=kf'\) \tab 
derivata del prodotto per una costante;
\item \(\Deriv{f\pm g}=f'\pm g'\) \tab 
derivata di una somma o differenza;
\item \(\Deriv{f\cdot g}= f'\cdot g+f\cdot g'\) \tab 
derivata del prodotto;
\item \(\Deriv{\dfrac{1}{g}}=\dfrac{g'}{g^2}\)\tab 
derivata  del reciproco (\(g \ne 0\));
\item \(\Deriv{\dfrac{f}{g}}=\dfrac{f'\cdot 
g-f\cdot g'}{g^2}\)\tab 
derivata  del rapporto (\(g \ne 0\)).
\end{enumerate}

\subsubsection{Derivata della funzione radice quadrata}
\label{subsubsec:f_radice}
\begin{teorema}
  La derivata della funzione \(f(x)=\sqrt{x}\), con \(x \geq 0\), è: 
\(\Deriv{\dfrac{1}{2\sqrt{x}}}\), con \(x > 0\).
\end{teorema}
% \noindent Ipotesi: \(f(x)=\sqrt{x}\), con \(x\geq 0\) \tab 
% Tesi:     \(f'(x)=\dfrac{1}{2\sqrt{x}}\).
\begin{proof}
Calcoliamo separatamente \(df(x)\), che richiede una razionalizzazione 
del numeratore. 
Perciò, \(\forall dx\ne 0\) e se \(x\ne 0\):
  \begin{align*}
   df(x) &= f(x+dx)-f(x)=\sqrt{x+dx}-\sqrt{x}=
          \tonda{\sqrt{x+dx}-\sqrt{x}}\cdot
          \frac{\sqrt{x+dx}+ \sqrt{x}}{\sqrt{x+dx}+\sqrt{x}}=\\
       &=\frac{x+dx-x}{\sqrt{x+dx}+\sqrt{x}}=
         \frac{dx}{\sqrt{x+dx}+\sqrt{x}}\\
   RD&=\dfrac{df(x)}{dx}=\dfrac{dx}{dx\tonda{\sqrt{x+dx}+\sqrt{x}}}=
       \dfrac{1}{\sqrt{x+dx}+\sqrt{x}}.
  \end{align*}
Poiché il Rapporto Differenziale  con \(x\ne 0\) è un numero finito, 
esiste la parte standard. Allora:
\begin{align*}
 \pst{\dfrac{df(x)}{dx}}&= \pst{\dfrac{1}{\sqrt{x+dx}+\sqrt{x}}}=
      \dfrac{1}{\pst{\sqrt{x+dx}+\sqrt{x}}}=
      \dfrac{1}{\pst{\sqrt{x+dx}}+\pst{\sqrt{x}}}=\\
      &=\dfrac{1}{\sqrt{x}+\sqrt{x}}=\dfrac{1}{2\sqrt{x}}.
\end{align*}
La parte standard non cambia al cambiare del \(dx\) usato, quindi si 
conclude che la derivata esiste e 
\[ f'(x)=\dfrac{1}{2\sqrt{x}}\quad (x>0)\]
\end{proof}

\begin{inaccessibleblock}
  [Funzione radice quadrata di x con alcune tangenti e la sua derivata]
\hspace{-20mm}\affiancati{.49}{.49}{
\begin{center} \scalebox{1}{\radquadn} \end{center}
}{
\begin{center} \scalebox{1}{\derivataradquad} \end{center}
}
\end{inaccessibleblock}
\label{img:diff_cubica_con_pendenze}
\begin{center}Grafico di \(f(x)=\sqrt{x}\) e di \(f'(x)\) \end{center}

% % \begin{figure}[h!]
% \begin{inaccessibleblock}
%   [m tangenti a radquad]
%  \begin{minipage}[]{.48\textwidth}
%  \begin{center}
% \scalebox{.9}{ \radquad}
%  \end{center}
%  \end{minipage} 
%  \hfill
%  \begin{minipage}[]{.48\textwidth}
%  \begin{center}
% \scalebox{.9}{  \tangentiradquad}
%  \end{center}
%  \end{minipage}
% \end{inaccessibleblock}

Le rette tangenti ai punti vicini all'origine hanno una pendenza elevata, 
che si attenua gradualmente man mano che \(x\) aumenta, fino ad 
assestarsi quasi orizzontalmente.

\begin{osservazione}
 Per confermare l'osservazione di pag.~\pageref{oss:regola3}  sulle 
funzioni potenza, adottiamo quella regola anche per il caso appena visto, 
che deriviamo di nuovo dopo averlo espresso come potenza.
\begin{align*}
 f(x)&=\sqrt{x}=x^{\frac{1}{2}},\quad \text{ con  } x\geq 0.\\
 f'(x)&=\dfrac{1}{2}x^{\frac{1}{2}-1}=\dfrac{1}{2}x^{-\frac{1}{2}}=
 \dfrac{1}{2x^{\frac{1}{2}}}=\dfrac{1}{2\sqrt{x}},\quad \text{ con  } x > 0.
\end{align*}
\end{osservazione}

\subsection{Differenziale e incremento}
\label{subsec:diff01_diff_inc}
Abbiamo visto più volte che il Rapporto Differenziale di una funzione e la 
sua derivata sono cose diverse: infatti, in una funzione derivabile,
\(f'(x_0)=\pst{RD}\). Applicando la parte standard si eliminano dal Rapporto 
Differenziale gli infinitesimi che il rapporto genera, così rimane solo il 
numero reale che si accetta come valore della pendenza.

Fra \(f'(x_0)\) e \(RD\) c'è solo una differenza di infinitesimi:
\(RD-f'(x_0)=\epsilon\). Concentrando i nostri ragionamenti su un 
particolare \(x_0\) e sviluppando le formule, si ha:\\
\(\dfrac{df(x_0)}{dx}-f'(x_0)=\epsilon\quad \rightarrow\quad
df(x_0)=f'(x_0)dx+\epsilon\cdot dx\quad \rightarrow\)\\
\(\rightarrow\quad f(x_0+dx)-f(x_0)=f'(x_0)dx+\epsilon\cdot dx
\rightarrow\quad f(x_0+dx)=f'(x_0)dx+f(x_0)+\epsilon\cdot dx\).\\
Se mancasse l'ultimo termine, l'equazione \(f(x_0+dx)=f'(x_0)dx+f(x_0)\) 
sarebbe quella di una retta: la si ottiene scrivendo \(x-x_0\) invece di 
\(dx\) e \(m\) invece di \(f'(x_0)\), cioè \(f(x)=m(x-x_0) +f(x_0)\). 
La retta, che passa per il punto della curva con ascissa \(x_0\) e ha la 
stessa pendenza della curva, è la sua tangente per quel punto. 

La conclusione è che \emph{a distanza \(dx\) dal punto che si vuole 
esaminare, fra la curva e la tangente c'è una differenza pari a 
\(\epsilon\cdot dx\), un infinitesimo di ordine superiore a \(dx\)}. 
Questa differenza è \emph{l'incremento infinitesimo della curva rispetto 
alla tangente}.

\begin{minipage}[]{.47\textwidth}
\begin{inaccessibleblock}
  [differenziale della tangente]
    \begin{center} \differenzialens \end{center}
\end{inaccessibleblock}
\end{minipage} 
  \hfill
\begin{minipage}[]{.47\textwidth} \vspace{2.5em}
Nel punto \(\punto{x_0}{f(x_0)}\) il grafico della funzione e la tangente
sono indistinguibili. 
Il campo visivo del primo microscopio mostra \({x_0}\) e \(dx\), 
perché l'ingrandimento è infinito. A questo livello 
microscopico la curvatura del grafico non esiste,  per cui la curva e 
la tangente sono sovrapposte. 
Nemmeno un secondo microscopio non standard puntato 
nel primo microscopio su \(x_0\), per cui 
l'ingrandimento complessivo diventa \(\infty^2\), può distinguere i due 
grafici. 
Ma un microscopio non standard, centrato nel primo microscopio a 
distanza infinitesima da \(x_0\), mostra la tangente e 
la curva come rette parallele, che distano fra loro 
\(\epsilon \cdot dx\), cioè un infinitesimo di un infinitesimo.
\end{minipage}
\label{}

\begin{teorema}
\label{teo:inc}
\emph{Teorema dell'incremento.} A distanza infinitesima dal punto di 
ascissa \(x_0\), la funzione derivabile \(f(x)\) ha un valore che è la 
somma fra il valore della sua tangente \(t(x)\) passante per 
\(\punto{x_0}{f(x_0)}\) e un infinitesimo dell'infinitesimo \(dx\):
\[f(x_0+dx)=\underbrace{f'(x_0)dx+f(x_0)}_{t(x)}+\epsilon\cdot dx\]
\end{teorema}
L'importante conseguenza di questo teorema è che se ci troviamo in 
difficoltà a calcolare l'incremento infinitesimo di 
\(f\), possiamo considerare al suo posto l'incremento lungo la 
tangente, perché l'errore che si commette è una quantità trascurabile anche 
rispetto a \(dx\).

% TODO Differenziale standard
% L'avevo eliminto, mi sembrava una questione troppo fine per gli
% studenti. L'ho reintrodotta (B)

\begin{osservazione}
Il termine \emph{differenziale}, che usiamo a proposito di 
\(df(x)\), corrisponde al concetto di incremento infinitesimo ed è 
stato recuperato dal matematico A.Robinson in coerenza con l'uso 
originario dell'epoca di Leibniz. 

Lo stesso termine, tuttavia, viene usato anche in analisi standard, ma con un 
significato diverso. In analisi standard il differenziale della funzione 
\(\Delta f(x)\) è il prodotto \(f'(x)\cdot \Delta x= \Delta t(x)\), cioè 
l'incremento che subisce la tangente \(t\) di \(f\), come se l'incremento lungo 
la tangente e quello lungo la funzione fossero praticamente identici.

\(\Delta f\) è una differenza espressa in numeri reali, una quantità 
piccola a piacere, ma finita: non è infinitesima, percò non è trascurabile, 
anzi può diventare rilevante.

\begin{inaccessibleblock}
  [differenziale della tangente]
  \begin{minipage}[]{.4\textwidth}
    \begin{center} \scalebox{1}{\falsodifferenziale} \end{center}
 \end{minipage} 
  \hfill
 \begin{minipage}[]{.55\textwidth}
 Allontanandosi da \(x_0\) di una quantità finita \(\Delta x\), le differenze 
della funzione \(\Delta f(x)\), calcolate a partire da \(x_0\), possono 
 essere anche molto diverse dalle differenze \(f'(x_0)\Delta x\), calcolate 
lungo la tangente.

Ai fini pratici l'equazione \(\Delta f= f'(x)\cdot \Delta 
x\) è comunque utile, soprattutto se si studiano i fenomeni naturali, perché le 
variazioni che si misurano in questi ambiti sono differenze finite. 
\end{minipage}
\end{inaccessibleblock}
\label{}
\end{osservazione}

\subsection{Derivata di funzioni composte}
\label{subsec:differenziazione_derivatacomposta}

% TODO: da mettere questo pezzo nel capitolo sulle funzioni???
% Direi di sì, ma poi deve essere ripreso qui perché si passa subito
%al calcolo dei differenziali.
% Lo stesso vale per le generalità sulla funzione inversa

A volte può essere utile considerare una funzione complessa come la 
composizione di funzioni più semplici. 

\begin{esempio}
Rappresenta la funzione  \(x \mapsto \dfrac{\tonda{3x -2}^2}{8}\)
come composizione di due funzioni più semplici. 

La funzione \(f\) esegue il quadrato del suo argomento e lo divide per 8:
\(f:~ x \mapsto \dfrac{x^2}{8}\).

Ma questa volta l'argomento della funzione \(f\) non è una costante, ma,
a sua volta, è il risultato della funzione \(g:~x \mapsto 3x -2\).

Quindi \(f\) è funzione di \(g\) e \(g\) è funzione di \(x\).
Lo schema seguente rappresenta la situazione.

\begin{center} \scalebox{1}{\boxfcompostaa} \end{center}
\label{gra:differenziazione_boxcompostaa}

Si tratta di una macchina che concatena due calcoli successivi. 
Immettiamo ad esempio il valore \(6\). 
La macchina \(g\) calcola, al suo interno, \(g(6)=3\cdot 6-2\) e dà come 
risultato \(16\).
Questo numero è l'input della macchina \(f\) che calcola 
\(f(16)=\frac{16^2}{8}\) e dà come risultato \(=32\)
che è il risultato di \(f(g(x))\).
\end{esempio}

Una macchina del genere si chiama \emph{funzione di funzione} o
\emph{funzione composta} e 
viene indicata con \(f(g(x))\) o con \(f \circ g\).

% \begin{center}
% \begin{minipage}[]{.48\textwidth}
% \begin{inaccessibleblock}
%   [box funzione composta]
%   \boxfcomposta
% \end{inaccessibleblock}
%  \end{minipage} 
%  \hfill
%  \begin{minipage}[]{.48\textwidth}
% Si tratta di una macchina che incatena due calcoli successivi.\\
% Immettiamo ad esempio il valore \(t=6\). 
% La macchina sviluppa al suo interno \(u(6)=3\cdot 6-2=16\) grazie a 
% \(u\) e infine produce \(v(16)=\dfrac{16^2}{8}=32\).
% Una catena del genere si chiama \emph{funzione di funzione} o
% \emph{funzione composta}: in questo caso \(f(g(x))\).\\
%  \end{minipage}
%  \end{center}
% \label{}

\vspace{1em}
Per quanto riguarda la derivata di una funzione composta, 
considerando che i valori \(x\) per arrivare alla trasformazione prevista 
da \(f\) devono prima essere trasformati da \(g\), come deriviamo  \(f\) 
rispetto a \(x\)? La formula sintetica è: 
\(f'(g(x))=\pst{\dfrac{df(g(x))}{dx}}\) e il punto da studiare è il calcolo 
del differenziale al numeratore.

Il differenziale \(df\) si potrebbe calcolare dalla definizione: 
\(df(g)=f(g+dg)-f(g)\) e questo obbliga poi a sviluppare \(dg\) e 
inglobarlo nel calcolo. \\
C'è una via più diretta e semplice: fare riferimento al 
Teorema~\ref{teo:inc}.
La conseguenza del teorema è che si può calcolare il differenziale della 
tangente invece di quello della funzione, perché si tralasciano quantità 
che comunque sono trascurabili rispetto a \(dx\). 

Quindi il differenziale della funzione: \(f \circ g\) 
è indistinguibile da \(f'(g)dg\) e 
il differenziale della funzione: \(g\) 
è indistinguibile da \(g'dx\). 

Il rapporto differenziale di \(f\) rispetto a \(x\) è: 
\[RD(f, x, dx) = \dfrac{df(g(x))}{dx} \sim 
          \dfrac{f'(g(x)) dg}{dx} \sim 
          \dfrac{f'(g(x)) g'(x) dx}{dx} = f'(g(x)) g'(x)\]
Ed essendo \(f\) e \(g\) derivabili, \(f'\) e \(g'\) sono funzioni reali
e quindi anche il loro prodotto è un numero reale e è quindi la derivata 
di \(\Deriv{f(g(x))}\).

\begin{esempio}
Calcolare \(f'(x)\), con \(f(x)=\sqrt{3-x^2}\)

\(f(x)\) è composta da due funzioni entrambe derivabili: 

Si può pensare formata così: 
\(g(x)=3-x^2\) \quad e \quad \(f(g(x))=\sqrt{g(x)}=\sqrt{3-x^2}\).

\begin{center} \scalebox{1}{\boxfcompostab} \end{center}
\label{gra:differenziazione_boxcompostaa}

\affiancati{.5}{.5}{
Riprendendo le funzioni di partenza:
\[f'(x)=\pst{\dfrac{-2x}{2\sqrt{25-x^2}}} = \dfrac{-x}{\sqrt{25-x^2}}\]

Se voglio calcolare la pendenza della funzione in \(x = 3\):
\[f'(3) = \dfrac{-3}{\sqrt{25-3^2}} = 
          \dfrac{-3}{\sqrt{16}} = -\dfrac{3}{4}\]
E la tangente quindi sarà:
\[y = f'(x_0) \tonda{x -x_0} + f(x_0) \srarrow
  y = -\dfrac{3}{4} \tonda{x -3} + 4 \srarrow
  y = -\dfrac{3}{4} x +\dfrac{25}{4}\]
}{
\derivacompostaa
}
\end{esempio}

\begin{teorema}
\label{teo:diff01_dericomp}
Se esistono le derivate \(g'(x)\) e \(f'(g(x))\) per il medesimo valore 
\(x\), la funzione composta \(f(g(x))\) è derivabile e la sua derivata si 
calcola così: \(f'(x)=f'(g(x))=f'(g)\cdot g'(x)\), 
cioè la derivata di una funzione composta è il prodotto delle derivate 
delle funzioni componenti, ciascuna rispetto alla propria variabile.
\end{teorema}
\noindent Ipotesi: \(f(x)=f(g(x))\), \(f\), \(g\) derivabili \tab 
Tesi: \(f'(x)=f'(g(x))=f'(g(x))\cdot g'(x)\)

\begin{proof}
Anzitutto, poiché esiste \(g'(x)\), \(dg(x)\) è infinitesimo per ogni 
infinitesimo \(dx \).\\
Poi, poiché esiste \(f'(g)\), dal teorema dell'incremento si deduce che per 
ogni 
infinitesimo \(dg\) c'è un infinitesimo \(\epsilon\) tale che 
\(df(g(x))=f'(g)\cdot dg + \epsilon\cdot dg.\)

Così, dividendo per qualsiasi infinitesimo non nullo \(dx\)
%{\footnotesize
 \begin{align*}
\Deriv{f(g(x))} &=\pst{\frac{df(x)}{dx}} = \pst{\frac{df(g)}{dx}}=\\
&=\pst{\frac{f'(g)\cdot dg + \epsilon\cdot dg}{d x}}=
\pst{\frac{f'(g)\cdot dg}{d x}}+ \pst{\frac{\epsilon\cdot dg}{dx}}=\\
&= f'(g)\cdot \pst{\frac{dg}{dx}}+ \pst{\epsilon}\cdot\pst{\frac{dg}{dx}}=
f'(x)\cdot g'(x).
\end{align*}
\end{proof}

Vogliamo rappresentare nella stessa immagine le funzioni 
\(g(x)\), \(f(g)\) e \(f(x) = f(g(x))\).
Andrebbero rappresentate tutte nello stesso piano cartesiano, ma 
sovrapporle renderebbe difficile la comprensione. 
Proviamo a vedere se riproducendo più versioni dello stesso piano, 
specchiato e ruotato, si riesce a seguire il meccanismo della funzione 
composta.

\noindent\begin{minipage}{.32\textwidth}
\begin{center}
Partiamo dalla funzione \(g\):

\disegno[5]{
  \begin{scope}[red!50!black] \funzioneg \end{scope}
}
\end{center}
\end{minipage}
\begin{minipage}{.32\textwidth}
\begin{center}
\(g(x)\) specchiata:

\disegno[5]{
  \begin{scope}[yscale=-1, red!50!black] \funzioneg \end{scope}
}
\end{center}
\end{minipage}
\begin{minipage}{.32\textwidth}
\begin{center}
poi ruotata di \(-90\text{°}\):

\disegno[5]{
  \begin{scope}[rotate=-90, yscale=-1, red!50!black]
    \funzioneg
  \end{scope}
}
\end{center}
\end{minipage}

\noindent\begin{minipage}{.32\textwidth}
\begin{center}
Aggiungiamo la funzione \(f(t)\) specchiata rispetto a \(y\).

\disegno[5]{
  \begin{scope}[rotate=-90, yscale=-1, red!50!black]
    \funzioneg
  \end{scope}
  \begin{scope}[xscale=-1, blue!50!black]
    \funzioneh
  \end{scope}
}
\end{center}
\end{minipage}
\begin{minipage}{.65\textwidth}
\begin{center}
Mettiamo tutto insieme e anche, \(f(x) = f(g(x))\): \\[.5em]

\derivatacomposta
\end{center}
\end{minipage}

\begin{osservazione}
 La regola della funzione composta si estende ai casi in cui le funzioni 
 in gioco sono tre, o più:
 \(\Deriv{f(g(h(x)))}=f'(g)\cdot g'(h)\cdot h'(x)\).
\end{osservazione}

\begin{esempio}
Calcola l'equazione della tangente alla funzione 
\(f:~x \mapsto \sqrt{x^2 -x -2}\) nel punto di ascissa \(x = 3\).

\affiancati{.49}{.49}{
\vspace*{-40mm}
Il punto di tangenza ha coordinate:
\[\punto{3}{\sqrt{3^2 -3 -2}} = \punto{3}{2}\]

Poniamo \(g(x)=x^2 -x -2\)\quad e\quad \(f(g)=\sqrt{g}\). 
Allora: 
\begin{align*}
\Deriv{f \circ g} &= f' \cdot g' = 
  \frac{1}{2 \sqrt{x^2 -x -2}} \cdot 2x -1= \\
  &=\frac{2x -1}{2 \tonda{x^2 -x -2}}
\end{align*}
}{
\vspace*{-20mm}
\hspace{-30mm}\derivacompostab
}

\vspace*{-40mm}
\noindent La pendenza in \(P\) è: \tab 
\(f'(3) = \dfrac{2 \cdot 3 -1}{2 \sqrt{3^2 -3 -2}}= \dfrac{5}{4}\) \\[1em]
e la tangente avrà equazione: \tab 
\(y = \dfrac{5}{4} \tonda{x -3} +2 \srarrow 
  y = \dfrac{5}{4} x -\dfrac{7}{4}\)
\end{esempio}

\begin{esempio}
Derivare \(f(x)=\dfrac{1}{\sqrt{2x-3}}\).

\affiancati{.49}{.49}{
Poniamo: 
\[h(x)=2x-3, \quad g(h)=\sqrt{h} \quad
\text{e} \quad f(g)=\dfrac{1}{g}\]
Allora: 
\begin{align*}
f'(g(h)) &= f'(g(f)) \cdot g'(h) \cdot h'(x) = \\
&= -\dfrac{1}{\sqrt{2x-3}} \cdot \dfrac{1}{2\sqrt{2x-3}} \cdot 2 = \\
&= -\dfrac{1}{2x-3}
\end{align*}
}{
\funzionecompostaederivata
}
\end{esempio}

% \begin{esempio}
%   Derivare \(f(x)=\tonda{-\dfrac{3}{2}x^3+2x^2-6}^5\).\\
%   Poniamo \(g(x)=-\dfrac{2}{3}x^3+2x^2-6\)\quad e\quad \(f(g)=g^5\). \\
%   Allora: 
%   \(f'(g)= 5g^4\)\quad e\quad \(g'(x)=-2x^2+4x\),\quad quindi: \\
%   \(f'(x)=f'(g)\cdot g'(x)=5g^4(-2x^2+4x)=
%   5\tonda{-\dfrac{2}{3}x^3+2x^2-6}^4(-2x^2+4x)\).
% \end{esempio}

\subsection{Derivata di funzioni inverse}
\label{subsec:differenziazione_derivatainverse}
Cosa si intende per funzioni inverse? 
Quando abbiamo introdotto il calcolo letterale abbiamo imparato a 
rovesciare alcune espressioni per trovare le formule inverse.

% ---------------------- Esempio con rette ----------------------------
Ad esempio, \(y = 3x -6\) e \(x = \dfrac{y}{3} +2\) 
sono formule inverse l'una dell'altra, ma non sono funzioni inverse.

Infatti se:
\begin{itemize} [noitemsep]
\item in una tabella forniamo alla prima funzione alcuni valori 
a \(x\) e calcoliamo i corrispondenti valori di \(y\);
\item in una tabella forniamo alla seconda funzione alcuni valori 
a \(y\) e calcoliamo i corrispondenti valori di \(x\);
\item rappresentiamo in un piano cartesiano le coppie ottenute nelle 
tabelle.
\end{itemize}
Otteniamo punti che appartengono al grafico della stessa funzione. 
Cioè le due equazioni hanno per soluzioni esattamente le stesse coppie di 
numeri.

\affiancati{.49}{.49}{
{
\newcommand{\mc}[3]{\multicolumn{#1}{#2}{#3}}
\begin{center}
\begin{tabular}{rrcrr}
\mc{2}{l}{\(y = 3x -6\)} & \qquad & \mc{2}{l}{\(x = \dfrac{y}{3} +2\)}\\
\mc{1}{r}{\textit{\(x\)}} & \mc{1}{r}{\textit{\(y\)}} & & 
\mc{1}{r}{\textit{\(y\)}} & \mc{1}{r}{\textit{\(x\)}}\\
\(0\)  & \(-6\) & & \(-6\) & \(0\)\\
\(+1\) & \(-3\) & & \(-3\) & \(+1\)\\
\(+2\) &  \(0\) & & \(0\) & \(+2\)\\
\(+3\) & \(+3\) & & \(+3\) & \(+3\)\\
\(+4\) & \(+6\) & & \(+6\) & \(+4\)\\
\(+5\) & \(+9\) & & \(+9\) & \(+5\)\\
\(+6\) & \(+12\) & & \(+12\) & \(+6\)
\end{tabular}
\end{center}
}
}{
\formulainversaa
}
Le due diverse espressioni esprimono la stessa retta, lo stesso grafico:
la stessa funzione.

Per ottenere la funzione inversa dobbiamo semplicemente scambiare, 
nell'equazione, la variabile \(x\) con la variabile \(y\).

\affiancati{.54}{.44}{
Quindi la funzione inversa di \\
\(y = 3x -6 \sstext{o} f(x) = 3x -6\) \quad è \\
\(x = 3y -6\) \quad che è uguale a \quad \(y = \dfrac{1}{3}x +2\) \\
o: \quad \(f^{-1}(x) = \dfrac{1}{3}x +2\).

Possiamo osservare che se componiamo le due funzioni otteniamo:
\begin{align*}
y &= f \circ f^{-1} = f(f^{-1}(x)) = 3 \tonda{f^{-1}(x)} -6 = \\
  &= 3 \tonda{\dfrac{1}{3}x +2} -6 = \dfrac{3}{3}x +6 -6 = x
\end{align*}
}{
\funzinversaa
}

La composizione delle due funzioni è una funzione il cui 
risultato è uguale all'argomento \(f(f^{-1}(x)) = x\).

Cosa si intende per funzioni inverse? 
Quando abbiamo introdotto il calcolo letterale abbiamo imparato a 
rovesciare alcune espressioni per trovare le formule inverse.

% ---------------------- Esempio con iperboli --------------------------
Ad esempio, \(y = \dfrac{6}{x} -2\) e \(x = \dfrac{6}{y -2}\) 
sono formule inverse l'una dell'altra, ma non sono funzioni inverse.

Infatti se:
\begin{itemize} [noitemsep]
\item in una tabella forniamo alla prima funzione alcuni valori 
a \(x\) e calcoliamo i corrispondenti valori di \(y\);
\item in una tabella forniamo alla seconda funzione alcuni valori 
a \(y\) e calcoliamo i corrispondenti valori di \(s\);
\item rappresentiamo in un piano cartesiano le coppie ottenute nelle 
tabelle.
\end{itemize}
Otteniamo punti che appartengono al grafico della stessa funzione. 
Cioè le due equazioni hanno per soluzioni esattamente le stesse coppie di 
numeri.

\affiancati{.49}{.49}{
{
\newcommand{\mc}[3]{\multicolumn{#1}{#2}{#3}}
\begin{center}
\begin{tabular}{rrcrr}
\mc{2}{l}{\(y = \dfrac{6}{x} -2\)} & \qquad & 
\mc{2}{l}{\(x = \dfrac{6}{y +2}\)}\\
\mc{1}{r}{\textit{\(x\)}} & \mc{1}{r}{\textit{\(y\)}} & & 
\mc{1}{r}{\textit{\(y\)}} & \mc{1}{r}{\textit{\(x\)}}\\
\(-6\)  & \(-3\) & & \(-6\) & \(-3/2\)\\
\(-3\) & \(-4\) & & \(-4\) & \(-3\)\\
\(-1\) &  \(-6\) & & \(-2\) & \(nd\)\\
\(0\) & \(nd\) & & \(0\) & \(+3\)\\
\(+1\) & \(+4\) & & \(+2\) & \(+3/2\)\\
\(+3\) & \(0\) & & \(+4\) & \(+1\)\\
\(+6\) & \(-1\) & & \(+6\) & \(+3/4\)
\end{tabular}
\end{center}
}
}{
\formulainversab
}

Le due diverse espressioni esprimono la stessa iperbole, lo 
stesso grafico: la stessa funzione.

Per ottenere la funzione inversa dobbiamo semplicemente scambiare, 
nell'equazione, la variabile \(x\) con la variabile \(y\).

\affiancati{.54}{.44}{
Quindi la funzione inversa di \\
\(y = \dfrac{6}{x} -2 \sstext{o} f(x) = \dfrac{6}{x} -2\) \quad è \\
\(x = \dfrac{6}{y} -2\) \quad che è uguale a 
\quad \(y = \dfrac{6}{x +2}\) \\
o: \quad \(f^{-1}(x) = \dfrac{6}{x +2}\).

Possiamo osservare che se componiamo le due funzioni otteniamo:
\begin{align*}
y &= f \circ f^{-1} = f(f^{-1}(x)) = \dfrac{6}{f^{-1}(x)} -2 = \\
  &= \dfrac{6}{\dfrac{6}{x +2}} -2 = 6 \cdot \dfrac{x +2}{6} -2 = x
\end{align*}
}{
\funzinversab
}

La composizione delle due funzioni è una funzione il cui 
risultato è uguale all'argomento: \(f(f^{-1}(x)) = x\).

Possiamo quindi dare una definizione precisa di funzione inversa:

\begin{definizione}
Un funzione \(y=f^{-1}(x)\) si dice inversa di una funzione \(y=f(x)\) se la 
composizione delle due funzioni 
è la funzione identica:
% dà come risultato l'argomento stesso: 
\[f(f^{-1}(x))=x\]
\end{definizione}

\begin{osservazione}
Data una qualsiasi funzione \(f\), non è scontato che la sua inversa 
esista \(\forall x\) nel dominio di \(f\). 
Per questo, quando si cerca l'inversa di una funzione, succede di dover 
restringere il dominio di questa. 
Per esempio, 
\(y=g(x)=\sqrt{x}\) è inversa di \(y=f(x)=x^2\),  perché 
\(f(g(x))=(g(x))^2=(\sqrt{x})^2=x\), ma la composizione è possibile 
solo se \(x \ge 0\), mentre \(f\) vale \(\forall x\). 
In questo caso, quindi, dobbiamo considerare per \(f\) il dominio più 
ristretto, perché al di fuori di questo l'inversa non esiste.

Una buona regola pratica per capire se esiste l'inversa di una funzione 
\(f\), è tagliare il grafico di \(f\) con una retta orizzontale: 
se la retta incrocia il grafico di \(f\) in più punti, \(f^{-1}\) non esiste.
\end{osservazione}
\begin{osservazione}
Si vede subito fin dai primi disegni che il grafico di \(f\) e il grafico 
della sua inversa \(f^{-1}\) sono simmetrici. L'asse di simmetria è la retta 
\(y=x\). 
\end{osservazione}

Ora vediamo come calcolare la derivata dell'inversa di una funzione di cui 
conosciamo la derivata. Consideriamo il caso semplice che segue.
\begin{esempio}
Derivare \(f(x)=\sqrt{x^2}\), con \(x \ge 0\).\\
Si dirà: non c'è problema, poiché \(\sqrt{x^2}=x\), allora 
\(\Deriv{\sqrt{x^2}}=\Deriv{x}\), perciò  \(f'(x)=1\).\\
Vero. Ma poniamo \(g(x)=x^2\), che è la funzione inversa della radice quadrata, 
allora \(f\) si può riscrivere come la funzione 
composta \(f(x)=f(g(x))=\sqrt{g(x)}=\sqrt{x^2}\).\\
Con la regola delle funzioni composte si ha:\\ 
\(f'(x)=f'(g)\cdot g'(x)= \dfrac{1}{2\sqrt{g(x)}}\cdot 2x =
\dfrac{1}{2\sqrt{x^2}}\cdot 2x=\dfrac{1}{2x}\cdot 2x = 1\).\\
Possiamo osservare che \(f'(g)\)  è il reciproco di \(g'(x)\) dato che 
il loro prodotto è \(1\). 
In questo caso, inoltre, bisogna segnalare che \(g'(x)\) esiste se 
\(x \ne 0\).
\end{esempio}
Dato che la composizione di una funzione con la sua inversa dà sempre la 
funzione identica e la derivata della funzione identica è \(1\), 
si intuisce che: \(\Deriv{f(g(x))} = f'(g) \cdot g'(x)=1\), allora 
\(g'(x) = \dfrac{1}{f'(g)}\).
% TODO: uniformare la riga precedente e tutto l'esempio al teorema seguente.
L'intuizione è corretta ed effettivamente questa regola vale. 
Occorre però precisare che la regola vale
\begin{enumerate} [noitemsep]
\item se esiste l'inversa della funzione da derivare;
\item se entrambe le funzioni sono derivabili;
\item se  \(f'(g)\ne 0\).
\end{enumerate}

\begin{inaccessibleblock}
[differenziale funzione inversa]
\begin{center}
\begin{minipage}[]{.55\textwidth}
\diffinversa
\end{minipage} 
\hfill
\begin{minipage}[]{.42\textwidth}
Se valgono tutte le condizioni, allora esistono la funzione
\(f\) e la sua inversa \(g=f^{-1}\). La funzione e la sua inversa, 
se esiste, hanno grafici simmetrici rispetto alla bisettrice \(y=x\).

Ogni punto \(\punto{x}{f^{-1}(x)}\) sulla curva della funzione inversa ha un
corrispondente \(\punto{y}{f(y)}\) sulla curva \(y=f(x)\), nella simmetria 
rispetto alla bisettrice. Guardiamo come si corrispondono i differenziali:
\(dx\) e \(dy\) di una curva sono invertiti rispetto ai differenziali 
dell'altra.
Quindi le derivate corrispondenti sono reciproche l'una con l'altra.
\end{minipage}
\end{center}
\end{inaccessibleblock}
\label{}

\begin{teorema}
\label{teo:derinversa}
% TODO: aggiustare le ipotesi.
% In realtà l'ipotesi sarebbe diversa, vedi Ruggero appunti per corso 
% Firenze
% Forse bisogna parlare di g inversa di f e non di g e f reciprocamente 
% inverse.
% Ruggero propone due versioni, ho scelto la prima, più semplice (dia 68) -B
Le derivate di due funzioni \(f\), \(g\), inverse l'una dell'altra, se 
esistono e sono diverse da zero, sono reciproche l'una rispetto all'altra.
\end{teorema}
\noindent Ipotesi: \(y=f(x)\), \(x=g(y)\), \(f\) e \(g\) derivabili, con 
\(f'\ne0\), \(g'\ne 0\).
\hspace{2cm} Tesi: \(f'(x)=\dfrac{1}{g'(y)}\).
\begin{proof}
Grazie alle proprietà della funzione \(\pst{}\), abbiamo:\\
\(f'(x)\cdot g'(y)=\pst{\dfrac{dy}{dx}}\cdot\pst{\dfrac{dx}{dy}}=
\pst{\dfrac{dy}{dx}\cdot\dfrac{dx}{dy}}=\pst{1}=1\)\\
per cui: \(f'(x)=\dfrac{1}{g'(y)}\).
\end{proof}
\begin{osservazione}
Dire che il Rapporto Differenziale \(\dfrac{dy}{dx}\) è reciproco di 
\(\dfrac{dx}{dy}\) non è banale come dire che la frazione \(\dfrac{3}{4}\) 
è reciproca di \(\dfrac{4}{3}\).
Una frazione è un rapporto fra numeri e genera un numero, il rapporto 
differenziale è un rapporto fra funzioni e genera una funzione. 
In più, in una frazione come la frazione \(\dfrac{3}{4}\) i numeri \(3\) 
e \(4\) sono indipendenti, invece il differenziale \(dy\) dipende da 
\(dx\) nel rapporto \(\dfrac{dy}{dx}\),  e \(dx\) dipende da \(dy\) nel 
rapporto inverso. \\
\end{osservazione}

% TODO: esempio di applicazione del teorema
% OK (B)

\begin {esempio}
A proposito della derivata delle funzioni potenza, abbiamo anticipato che la 
regola \(f'(x)= n x^{n-1}\) vale anche con esponenti razionali. 
Ora siamo in grado di giustificarlo applicando il teorema precedente.\\

\noindent Ipotesi: \(y=f(x)=\sqrt[m]{x}=x^\frac{1}{m}\), con \(m \in \N\).
\tab Tesi: \(f'(x)= \dfrac{1}{m} x^{\frac{1}{m}-1}\).
\begin{proof}
Poiché l'inversa \(g(y)=y^m\) ha derivata \(g'(y)=my^{m-1}\), ricordando il 
Teorema~\ref{teo:derinversa} calcoliamo:\\
\(f'(x)=\dfrac{1}{g'(y)}=\dfrac{1}{my^{m-1}}=
  \dfrac{1}{m(\sqrt[m]{x})^{m-1}}=
  \dfrac  {x^{-\frac{m-1}{m}}}{m}=\dfrac{x^{\frac{1}{m}-1}}{m}\),\\
ed è lo stesso risultato che si ottiene scrivendo \(f(x)=x^\frac{1}{m}\) e 
derivando in base alla regola delle funzioni potenza.

Il caso in cui \(y=f(x)=\sqrt[m]{x^n}=x^\frac{m}{n}\) si  dimostra in modo 
analogo, anche avvalendosi del Teorema \ref{teo:diff01_dericomp}.
Invece la dimostrazione valida per qualsiasi esponente reale è a 
pag.~\pageref{teo:funzione_potenza_generica}.
\end{proof}
\end{esempio}

\begin{esempio}
 Derivare \(y=f(x)=\sqrt[3]{x}\).\\
 Poiché l'inversa \(g(y)=y^3\) ha derivata \(g'(y)=3y^2\), per il 
Teorema~\ref{teo:derinversa}, calcoliamo:\\
\(f'(x)=\dfrac{1}{g'(y)}=\dfrac{1}{3y^2}=
 \dfrac{1}{3(\sqrt[3]{x})^2}=\dfrac{1}{3} x^{-\frac{2}{3}}\), 
con \(x\ne 0\).\\
Se esprimiamo \(f(x)\) come potenza, abbiamo:\\
\(f(x)=\sqrt[3]{x}=x^\frac{1}{3}\). 
Applicando la regola \(f'(x)= \alpha x^{\alpha-1}\) risulta: 
\(f'(x)=\dfrac{1}{3}x^{\frac{1}{3}-1}=\dfrac{1}{3}x^{-\frac{2}{3}}\)
e le due derivazioni danno lo stesso risultato.
\end{esempio}

Nel prossimo esercizio sfruttiamo sia la regola per la derivata di una 
funzione composta che la regola per la derivata della funzione inversa.
\begin{esempio}
Trova la derivata di \(f(x)=\dfrac{1}{\sqrt{5-x^2}}\).
\begin{enumerate}[noitemsep]
\item Usando il teorema \ref{teo:diff01_dericomp} e le regole 
precedenti:\\
\(f'(x)=\Deriv{\dfrac{1}{\sqrt{5-x}}}=
\Deriv{(5-x)^{-\frac{1}{2}}}=
-\dfrac{1}{2}(5-x)^{-\frac{3}{2}}(-1) = 
\dfrac{1}{2(\sqrt{5-x})^3}\).
\item Usando la regola del Teorema~\ref{teo:derinversa}:\\
Costruiamo la formula inversa con pochi passaggi algebrici: riavremo la 
stessa funzione, in cui \(y\) figura come variabile indipendente: 
\(x=f(y)\).\\
Quindi deriviamo: \(\Deriv{x}=x'=f'(y)=\pst{\dfrac{dx}{dy}}\).\\
\(f(x)=y=\dfrac{1}{\sqrt{5-x}}\srarrow y^2=\dfrac{1}{5-x} \srarrow 
y^{-2}=5-x\srarrow x=5-y^{-2}\)  (formula inversa)\\
\(x'=\pst{\dfrac{dx}{dy}}=2y^{-3}=\dfrac{2}{y^3}\) 
(derivata della funzione inversa)\\
\(\srarrow y'=\pst{\dfrac{dy}{dx}}=\dfrac{y^3}{2}=
\dfrac{1}{2(\sqrt{5-x})^3}\).
\end{enumerate}
\end{esempio}

% \section{Derivata di funzioni trascendenti}
% \label{sec:differenziazione_trascendenti}
% 
% \subsection{Derivata della funzione esponenziale}
% \label{subsec:differenziazione_derivatafesponenziale}
% 
% \subsection{Derivata della funzione logaritmo}
% \label{subsec:differenziazione_derivataflogaritmo}
% 
% \subsection{Derivata della funzione seno}
% \label{subsec:differenziazione_derivatafseno}
% 
% \subsection{Derivata della funzione coseno}
% \label{subsec:differenziazione_derivatafcoseno}
% 
% \subsection{Derivata della funzione tangente}
% \label{subsec:differenziazione_derivataftangente}
% 
% \section{Riassunto}
% \label{sec:differenziazione_sunto}
% 
% \subsection{Differenziale}
% \label{subsec:differenziazione_differenziale}
% 
% \subsection{Schema riassuntivo}
% \label{subsec:differenziazione_schemaderivate}
% 
% \section{Applicazioni delle derivate}
% \label{sec:differenziazione_applicazioni}
% 
% \subsection{Derivata e tangente}
% \label{subsec:differenziazione_derivataetangente}
% 
% \subsection{Derivata e normale}
% \label{subsec:differenziazione_derivataenormale}
% 
% \subsection{Derivata della derivata}
% \label{subsec:differenziazione_derivataseconda}
% 
% \subsection{Altre applicazioni}
% \label{subsec:differenziazione_altreapplicazioni}

\section{Derivare funzioni trascendenti}
\label{sec:diff01_deritrasc}
Finora abbiamo imparato a derivare le funzioni algebriche. In 
questa sezione ci occupiamo della derivata delle funzioni trascendenti.

\subsection{Derivata di \(f(x)=a^x\)}
\label{subsubsec:deri_a_alla_x}
Il grafico di una generica funzione esponenziale \(y=a^x\), confrontato con 
il grafico dell'andamento delle sue pendenze è una sorpresa rispetto ai 
confronti che abbiamo fatto per altre funzioni. 
Prendiamo ad esempio la funzione \(f(x) = 1,5^x\).
Possiamo osservare che:
\begin{enumerate} [nosep]
\item dato che la funzione è sempre crescente: 
la sua \emph{pendenza è sempre positiva};
\item per valori negativi (molto piccoli) dell'argomento, 
la \emph{pendenza è molto vicina a zero};
\item per valori positivi dell'argomento la \emph{pendenza cresce molto 
rapidamente}.
\end{enumerate}
Anche la funzione \(f'\) ha un andamento simile:
\begin{enumerate} [nosep]
\item la \emph{funzione è sempre positiva};
\item per valori negativi (molto piccoli) dell'argomento, 
la \emph{funzione è molto vicina a zero};
\item per valori positivi dell'argomento la \emph{funzione cresce molto 
rapidamente}.
\end{enumerate}

La funzione e la sua derivata si assomigliano, sono funzioni della stessa 
``famiglia''.
% I due grafici praticamente si accompagnano: 
% rivelano uguali pendenze in coppie di punti di uguale ordinata.

\begin{inaccessibleblock}
  [esponenziale e pendenze]
\hspace{-20mm}\affiancati{.55}{.43}{
\begin{center} \scalebox{.8}{   \esp} \end{center}
}{
\begin{center} \scalebox{.8}{\derivataesp} \end{center}
}
\end{inaccessibleblock}
\label{}
\begin{center} Il grafico di \(y=a^x\), con alcune sue tangenti, e 
la sua derivata.\end{center}

% \begin{inaccessibleblock}
%   [esponenziale e pendenze]
%   \begin{minipage}[]{.49\textwidth}
% \begin{center} \scalebox{.8}{   \esp} \end{center}
%  \end{minipage} 
%   \hfill
%  \begin{minipage}[]{.49\textwidth}
% \begin{center} \scalebox{.8}{\derivataesp} \end{center}
%  \end{minipage}
% \end{inaccessibleblock}
% \label{}
% \begin{center} Il grafico di \(y=a^x\), con alcune sue tangenti, e 
% la sua derivata.\end{center}

Anche se i due grafici non sono identici, le pendenze delle tangenti 
sembrano avere un andamento anch'esso esponenziale: la derivata della 
funzione ha un grafico che somiglia molto al grafico della funzione. 
Possiamo dimostrare il seguente teorema:



\begin{teorema}
La derivata della funzione esponenziale è proporzionale alla funzione 
stessa:
\[\Deriv{a^x}=k \cdot a^x\]
\end{teorema}
% \noindent Ipotesi: \(f(x)=a^x\). \tab \(f'(x)=k \cdot a^x\).
\begin{proof}
Calcoliamo il differenziale e il Rapporto Differenziale: 
\[df=a^{x+dx}-a^x=a^xa^{dx}-a^x=a^x\tonda{a^{dx}-1} \qquad
RD=\dfrac{df(x)}{dx}=\dfrac{\tonda{a^{dx}-1}}{dx}a^x\]
Per poter calcolare la derivata il Rapporto Differenziale deve essere 
finito.
\begin{itemize}
\item 
\(a^x\) è senz'altro finito, dato che è una funzione reale definita per ogni 
valore di \(x\);
\item 
e \(\dfrac{\tonda{a^{dx}-1}}{dx} = \dfrac{\tonda{a^{0+dx}-a^0}}{dx}\) 
è finito dato che la funzione esponenziale è continua.
\end{itemize}
Possiamo quindi calcolare la parte standard:
\[f'(x)= \pst{\dfrac{df(0)}{dx}\cdot a^x}=f'(0) \cdot f(x)\]
Quindi \emph{la derivata di una funzione esponenziale è uguale alla
funzione stessa, moltiplicata per la pendenza della funzione in} \(x=0\).
\end{proof}



% 
% Sviluppiamo matematicamente questa intuizione, ricordando le proprietà
% delle potenze.\\
% Differenziale di \(f(x)=a^x\): \quad
% \(df=a^{x+dx}-a^x=a^xa^{dx}-a^x=a^x\tonda{a^{dx}-1}\).\\
% \(RD=\dfrac{df(x)}{dx}=\dfrac{\tonda{a^{dx}-1}}{dx}a^x\).\\
% Per poter derivare occorre che il rapporto sia finito.
% 
% L'espressione del Rapporto Differenziale contiene sia \(a^x\), che è una 
% funzione esponenziale e per ogni fissato \(x\) genera un numero finito, 
% sia il fattore \(\dfrac{\tonda{a^{dx}-1}}{dx}\) che è da interpretare.\\
% % \(\dfrac{\tonda{a^{dx}-1}}{dx}= \dfrac{a^{(0+dx)}-a^0}{dx}=
% % \dfrac{d(a^x)}{dx}\bigg |_{x=0}= \dfrac{df(0)}{dx}\).\\
% \(\dfrac{\tonda{a^{dx}-1}}{dx}= \dfrac{a^{(0+dx)}-a^0}{dx}=
% \dfrac{d(a^0)}{dx}= \dfrac{df(0)}{dx}\).\\[3pt]
% Quindi il Rapporto Differenziale della funzione esponenziale risulta:\\
% \(RD=\dfrac{df(x)}{dx}=\dfrac{\tonda{a^{dx}-1}}{dx}a^x=
% \dfrac{df(0)}{dx}\cdot a^x\)\\[3pt]
% cioè il prodotto fra l'esponenziale stesso 
% \(a^x\) e il Rapporto Differenziale di \(a^x\) calcolato per \(x=0\).
% % Dobbiamo presumere che quest'ultimo sia un numero non infinito per 
% % proseguire nel ragionamento e in effetti lo potremo verificare fra poco.
% Se questo valore è finito, come vedremo tra poco, possiamo continuare 
% applicando la parte standard e otteniamo:\\[3pt]
% \(f'(x)= \pst{\dfrac{df(0)}{dx}\cdot a^x}=f'(0)\cdot f(x)\).
% 
% % \emph{La derivata di una funzione esponenziale è proporzionale alla
% % funzione stessa, attraverso un fattore che corrisponde alla derivata 
% % calcolata in} \(x=0\).
% \emph{La derivata di una funzione esponenziale è uguale alla
% funzione stessa, moltiplicata per la pendenza della funzione in} \(x=0\).

% Volevamo calcolare la derivata di \(f(x)=a^x\) e ci ritroviamo con
% un risultato che contiene la derivata stessa \(f'(0)\), insomma non si 
% direbbe che ci siano stati grandi progressi. 
Se esiste una funzione esponenziale che in zero ha pendenza \(1\), 
allora la derivata della funzione è uguale alla funzione stessa.
% Ma fingiamo per un attimo che \(f'(0)\) non incida sul risultato, cioè 
% che \(f'(0)=1\). 
% In questo modo la funzione e la sua derivata sarebbero proprio identiche 
% e i due grafici sarebbero sovrapponibili.\\

In conclusione: perché una funzione esponenziale generica \(a^x\) coincida 
con la sua derivata occorre che la base sia \(a=e\): \(f(x)=e^x\) è la 
funzione esponenziale pura.

\begin{teorema}
Esiste una funzione esponenziale la cui derivata coincide con la 
funzione stessa: \(\Deriv{a^x}=a^x\)
\end{teorema}
% \noindent Ipotesi: \(f(x)=a^x\). \tab 
%           Tesi: Esiste un \(a\) per cui \(f'(x)=a^x\).
\begin{proof}
Dato che \(f'(x) = f'(0) \cdot f(x)\), 
dobbiamo trovare una base per cui \(f'(0)=1\):
\[f'(0) = \pst{\dfrac{a^{dx}-1}{dx}} = 1 \srarrow 
\dfrac{a^{dx}-1}{dx} \approx 1 \srarrow 
a^{dx}-1 \approx dx \srarrow 
a^{dx} \approx dx +1 \]
Elevando entrambi i membri per \(\frac{1}{dx}\):
\[a \approx (1 +dx)^\frac{1}{dx} \approx \pst{(1 +dx)^\frac{1}{dx}} = e\]
\end{proof}

% Attraverso l'uso del numero \(e\) siamo finalmente in grado di derivare la 
% funzione esponenziale generica \(f(x)=a^x\) e così risolvere anche i dubbi 
% sulla finitezza del Rapporto Differenziale.
% 
% \(f'(0)=1 \srarrow \pst{\dfrac{a^{dx}-1}{dx}}=1\srarrow a^{dx}\sim dx +1
% \srarrow a\sim(1 +dx)^\frac{1}{dx}\).\\[3pt]
% Questa espressione individua un ben preciso numero, il Numero di Nepero 
% \(e=\pst{(dx+1)^\frac{1}{dx}}\).
% 
% In conclusione: perché una funzione esponenziale generica \(a^x\) coincida 
% con la sua derivata occorre che la base sia \(a=e\): \(f(x)=e^x\) è la 
% funzione esponenziale pura.

% Quindi \emph{la funzione  esponenziale pura \(f(x)=e^x\) 
% coincide con la sua derivata}:
% \[\Deriv{e^x} = e^x\]

% Quindi arriviamo al seguente teorema
Conseguenza dei due teoremi precedenti è il seguente
\begin{teorema}
La derivata della funzione esponenziale \(f(x)=e^x\) coincide con la 
funzione stessa: \(\Deriv{e^x}=e^x\)
\end{teorema}
% \noindent Ipotesi: \(f(x)=e^x\). \tab \(f'(x)=e^x\).
\begin{proof}
\[\Deriv{e^x} = \Deriv{e^0} \cdot e^x = 1 \cdot e^x\ = e^x\]
\end{proof}
È stato dimostrato che questa è l'unica funzione che coincide con la sua 
derivata.

\begin{teorema}
  La derivata della funzione esponenziale generica \(f(x)=a^x\), con 
\(a>0\), è:\\ 
\(f'(x)= a^x\ln{a}\).
\end{teorema}
% \noindent Ipotesi: \(f(x)=a^x\); \tab Tesi: \(f'(x)=a^x\ln{a}\).
\begin{proof}
Usiamo una trasformazione appresa con lo studio dei logaritmi e applichiamo 
il teorema a pag.~\pageref{teo:diff01_dericomp}:
\(f(x)~=~a^x~=~e^{\ln a^x}\). 
Se poniamo \(g(x)=\ln a^x=x\ln a\), si ottiene:
\[f(g(x))=e^{g(x)} \srarrow f'(g(x))=f'(g) \cdot g'(x) = 
e^{x\ln a} \cdot \ln a = e^{\ln a^x} \cdot \ln a = a^x\ln a\]
\end{proof}
\begin{esempio}
Calcola la derivata di \(f(x)=3e^{x-1}\).

Poniamo \(g(x)=x-1\), quindi \(f(x) = 3e^{g(x)}\): 
\[f'(x) = \Deriv{3e^{g(x)}} \cdot \Deriv{x-1} = 
          3e^{g(x)} \cdot 1 = 3e^{x-1}\]
\end{esempio}
\begin{esempio}
Calcola la derivata di \(f(x)=e^{x^2}\).

Poniamo \(g(x)=x^2\), quindi \(f(x)=e^{g(x)}\): 
\[f'(x) = \Deriv{e^{g(x)}} \cdot \Deriv{x^2} = 
          e^{g(x)} \cdot 2x = 2xe^{x^2}\]
\end{esempio}

\begin{esempio}
Calcola la derivata di \(f(x)=2^{x^2}\).

Poniamo \(g(x)=x^2\), quindi \(f(x)=2^{g(x)}\): 
\[f'(x) = \Deriv{2^{g(x)}} \cdot \Deriv{x^2} = 
          \tonda{2^{g(x)} \cdot \ln 2} \cdot 2x = 
          x \cdot \ln 2 \cdot 2^{x^2} \cdot 2 = 
          x \ln 2 \cdot 2^{x^2+1}\]
\end{esempio}

\subsection{Derivata di \(f(x)=\log_a x\)}
\label{}

Iniziamo con un caso particolare.
\begin{esempio}
Calcola la derivata di \(f(x)=e^{\ln x}\).

Poniamo \(g(x)=\ln x\), quindi \(f(x)=e^{g(x)}\): 
\[f'(x) = \Deriv{e^{g(x)}} \cdot \Deriv{\ln x} = 
e^{\ln x} \cdot \dots \stext{???}\]
Siamo passati dal cercare \(\Deriv{e^{\ln x}}\) al cercare 
\(\Deriv{\ln x}\) non abbiamo fatto un gran passo avanti!
Ma rifacendoci alla definizione di logaritmo possiamo osservare che 
\(e^{\ln x} = x\) che ha quindi come derivata: \(1\).
\end{esempio}

Ma ragioniamo un po' sulle due funzioni: \(g(x)=\ln x\) e \(f(x)=e^x\)
possiamo osservare che:

Per definizione, \(e^{\ln x} = x\) e, 
per le proprietà dei logaritmi, \(\ln e^x = x \ln e = x\)
quindi il logaritmo naturale \(g(x) = \ln x\) e 
la funzione esponenziale \(f(x) = e^x\) sono due funzioni inverse una 
dell'altra, dato che la loro composizione dà la funzione identica.

Possiamo quindi usare il teorema della derivata delle funzioni inverse per 
dimostrare il seguente
\begin{teorema}
La derivata della funzione logaritmo naturale è: 
\(\Deriv{\ln x} = \dfrac{1}{x}, \stext{con} x>0\).
\end{teorema}

\noindent\begin{minipage}[]{.43\textwidth}
\begin{inaccessibleblock}
[esponenziale e logaritmo]
\begin{center} \scalebox{.9}{\esplog} \end{center}
\end{inaccessibleblock}
\end{minipage} 
\hfill
\begin{minipage}[]{.55\textwidth}
% \noindent Ipotesi: \(f(x)=\ln x, \mbox{ con } x>0\) \tab
% \noindent Tesi: \(f'(x)=\dfrac{1}{x}\)
\begin{proof}
Usando la regola della derivazione delle funzioni inverse, \\
\[\tonda{\Deriv{f^{-1}(x)} = \dfrac{1}{\Deriv{(f^{-1}(x)}}}
\quad \text{otteniamo:}\]
\[\Deriv{\ln x}=\dfrac{1}{\Deriv{e^{g(x)}}} = 
  \dfrac{1}{e^{\ln x}} = \dfrac{1}{x}\]
Ovviamente la derivata del logaritmo esiste solo nel suo insieme di 
definizione.
% La dimostrazione è nei ragionamenti dell'esempio precedente, ai quali 
% bisogna aggiungere le precauzioni perché le due funzioni siano 
% invertibili e derivabili: poiché \(\ln x\) esiste per \(x>0\), 
% i ragionamenti valgono solo per \(x>0\)
\end{proof} 
\end{minipage}
\label{}
\\

Vediamo ora il caso generale, quando la base del logaritmo è genericamente 
\(a>0\).
\begin{teorema}
La derivata della funzione logaritmo in base \(a\) è: 
\(\Deriv{\log_a x}= \dfrac{1}{x\ln a}\).
\end{teorema}
% \noindent Ipotesi: \(f(x)=\log_a x\). \tab \(f'(x)=\dfrac{1}{x\ln a}\)
\begin{proof}
Si ottiene direttamente dalla formula del cambiamento di base:\\
\[\Deriv{\log_a x} = \Deriv{\dfrac{1}{\ln a}\ln x} = 
  \dfrac{1}{\ln a} \cdot \Deriv{\ln x} = \dfrac{1}{x \ln a}\]
\end{proof}

\begin{esempio}
Derivare la funzione \(f(x)=Log(x^2+1)^2\).\\

Possiamo vedere la funzione come composta da tre funzioni:

\(h(x)=(x^2+1) \srarrow g(x)=(h(x))^2\srarrow f(x)=Log(g(h((x)))\)

Applichiamo quindi i teoremi precedenti:
\[f'(x) =
% f'(g) \cdot g'(h) \cdot h(x) =
\dfrac{1}{\ln 10}
\dfrac{1}{g(h)}g'(h) h'(x)=
\dfrac{1}{(\ln 10)(x^2+1)^2} \cdot 2(x^2+1) \cdot 2x =
\dfrac{4x}{(\ln 10)(x^2+1)}\]
% Nota che \(g(x)=(x^2+1)^2\) è a sua volta una funzione composta del tipo 
% \(g(x)=\quadra{h(x)}^2\) e quindi è stata applicata la regola della 
% derivata di più funzioni composte.
\end{esempio}

Abbiamo ora tutti gli strumenti per convalidare l'osservazione al teorema 
\ref{diff01_teoderpotenza}, a proposito delle funzioni potenza.

\begin{teorema}
\label{teo:funzione_potenza_generica}
La derivata della funzione potenza \(f(x)=x^\alpha\) è: \hspace{5mm}
\(\Deriv{x^\alpha}=(\alpha-1)x^\alpha\), \(\forall\alpha \in \R\).
\end{teorema}
% \noindent Ipotesi: \(f(x)=x^\alpha\). \tab \(f'(x)=(\alpha-1)x^\alpha\), 
\(\forall \alpha\).
\begin{proof}
Combinando alcune delle regole precedenti, si ha:
\[f(x) = x^\alpha= e^{\ln x^\alpha}=e^{\alpha\ln x}\]
\[f'(x)=e^{\alpha\ln x}\alpha\dfrac{1}{x}=x^\alpha\dfrac{\alpha}{x}=
\alpha x^{\alpha-1}\]
Il teorema vale per qualsiasi esponente poiché non è stata fatta nessuna 
particolare ipotesi sull'esponente 
(intero o razionale, positivo o negativo, o irrazionale).
\end{proof}

\begin{esempio}
Derivare \(f(x)=x^{\sqrt{2}}\).

\[f'(x)=\sqrt{2}x^{\sqrt{2}-1}\]
\end{esempio}

\subsection{Derivata di funzioni circolari}
\label{subsec:differenziazione_circolari}

Prima di affrontare il problema della derivazione delle funzioni 
circolari, cerchiamo di chiarirci cosa succede al seno e al coseno 
quando l'arco aumenta di una quantità infinitesima.

\affiancati{.34}{.64}{
\hspace{-10mm}\scalebox{1}{\dsincos}
\label{fig:differenziazione_dsincos}
}{
Ovviamente alla scala naturale, gli archi (e angoli) \(x\) e \(x +dx\) 
non sono distinguibili: appaiono sovrapposti.
Per poterli distinguere dobbiamo operare un ingrandimento infinito nel 
punto \(A\).
Nel microscopio non standard ora è visibile \(dx = AB\) quindi i punti 
\(A\) e \(B\) appaiono distinti. 
Possiamo tracciare il triangolo rettangolo \(ABC\) con i cateti paralleli 
agli assi:
\begin{itemize} [nosep]
\item \(CB\) rappresenta il differenziale della funzione seno; 
\item \(AC\) rappresenta il differenziale della funzione coseno. 
\end{itemize}
Il triangolo infinitesimo \(ABC\) è simile al triangolo \(AOD\) e
l'angolo \(\widehat{ABC}\) è uguale a \(x\).
}

\subsubsection{Derivata di \(f(x)=\sen x\)}
\begin{teorema}
La derivata della funzione \(f(x)=\sen x\)~~è~~\(f'(x)=\cos x\): \quad 
\(\Deriv{\sen x} = \cos x\)
\end{teorema}
% \noindent Ipotesi: \(f(x)=\sen x\) \tab \(f'(x)=\cos x\)
\begin{proof}
Poiché l'angolo \(\widehat{ABC}\) è uguale a \(x\), possiamo calcolare 
il segmento orientato \(CB\) usando la trigonometria: 
\quad \(BC = AB \cdot \cos x\). 
Operando le sostituzioni visualizzate in figura, il Rapporto Differenziale 
risulta:
\[RD = \dfrac{d \sen x}{dx} = \dfrac{CB}{AB} = 
       \dfrac{AB \cdot \cos x}{AB} = 
       \dfrac{dx \cdot \cos x}{dx} = \cos x\]
Dato che \(\cos x\) è un numero reale, la sua parte standard esiste e 
coincide con il numero stesso, e questa è la derivata.
\end{proof}

% Anche per queste funzioni dobbiamo dapprima definire il differenziale. 
% Per una migliore comprensione, ci affidiamo soprattutto al piano cartesiano.
% 
% Abbiamo già visto (pag.~\pageref{limiti:par_f_seno}) che per angoli 
% infinitesimi il seno e l'angolo sono indistinguibili: \\
% \(\st\tonda{\frac{\sen \epsilon}{\epsilon}}=1\).
% Dall'analisi del disegno ricaviamo l'espressione del differenziale della 
% funzione seno: \\
% \(df(x)=d(\sen x)= \sen (x+dx) -\sen x\).
% 
% \begin{inaccessibleblock}
%   [differenziale del seno]
%   \begin{minipage}[]{.40\textwidth}
%    \dseno 
%  \end{minipage} 
%   \hfill
%  \begin{minipage}[]{.56\textwidth}
% Nell'ingrandimento al microscopio non standard, l'incremento infinitesimo 
% di arco \(\overset{\frown}{AB}\) (che corrisponde all'incremento di 
% angolo da \(x\) a \(x+dx\)) è racchiuso fra due raggi indistinguibili da 
% segmenti paralleli nei punti \(A\equiv \punto{x}{\sen x}\) e \(B\equiv 
% \punto{x+dx}{\sen(x+dx)}\). 
% L'arco, a sua volta, risulta indistinguibile dal segmento rettilineo 
% \(AB\).
% I segmenti che uniscono \(A\) e \(B\) con le loro proiezioni sull'asse 
% \(X\) sono verticali e paralleli, perciò \(ABC\) è un triangolo 
% rettangolo infinitesimo, simile al triangolo \(BOC\). La sua altezza 
% \(BC\) corrisponde a \(d\sen x\). 
%  \end{minipage}
% \end{inaccessibleblock}
% \label{fig_diff01dseno}\\
% 
% Risolviamo il triangolo rettangolo \(ABC\) rispetto al lato \(BC\):\\
% \(BC=AB\cdot \cos x \srarrow d(\sen x)= dx\cdot cos x\)  
% 
% \begin{teorema}
%   La derivata della funzione \(f(x)=\sen x\) è \(\Deriv{\sen 
% x}=\cos x\).
% \end{teorema}
% % \noindent Ipotesi: \(f(x)=\sen x\). \tab \(f'(x)=\cos x\).
% \begin{proof}
%  Il commento al disegno giustifica la tesi. 
% \end{proof}
% 
% \begin{osservazione}
% Si potrebbe criticare il metodo per la dimostrazione: chi assicura che 
% negli altri quadranti le relazioni fra le variabili non cambino? 
% Saremo troppo legati al disegno?\\
% Ci sono altri modi per dimostrare la tesi, più vincolati al calcolo
% e meno al disegno.  Per esempio, dalle formule di addizione abbiamo:
% \(RD=\sen(x+dx)=\sen x \cos dx + \sen dx \cos x\). Allora:\\
% \(\dfrac{\sen(x+dx)-\sen x}{dx}=\dfrac{\sen x \cos dx + \sen dx \cos x -
%   \sen x}{dx}=\\
% =\sen x \dfrac{\cos dx-1}{dx}+ \cos x\dfrac{\sen dx}{dx}\sim
% \sen x\cdot 0 + \cos x\cdot 1=\cos x\),\\
% in cui si fa uso delle forme indeterminate discusse a pag. 
% \pageref{subsubsec:insnum_fseno}. Quando poi, per ottenere la derivata, 
% si applica la parte standard, gli infinitesimi che vengono sottointesi dal 
% segno \(\sim\) si eliminano e si ha l'uguaglianza in tutti i passaggi.
% \end{osservazione}

\begin{osservazione}
Si può dimostrare lo stesso teorema anche usando la formula di addizione 
del seno:
\(\sen(\alpha+\beta) = 
\sen \alpha \cdot \cos \beta + \sen \beta \cdot \cos \alpha\) e 
i limiti presentati a pag. \pageref{subsubsec:insnum_fseno}. 
\begin{align*}
RD &= \dfrac{\sen(x+dx)-\sen x}{dx} =
      \dfrac{\sen x \cdot \cos dx + \sen dx \cdot \cos x - \sen x}{dx} = \\
   &= \sen x \dfrac{\cos dx-1}{dx}+ \cos x\dfrac{\sen dx}{dx}
\end{align*}
Il Rapporto differenziale ottenuto è un numero finito per cui possiamo 
calcolare la parte standard:
\begin{align*}
\Deriv{\sin x} &= \pst{RD} = 
    \pst{\sen x \dfrac{\cos dx-1}{dx} + \cos x\dfrac{\sen dx}{dx}} = \\
&=  \pst{\sen x} \cdot \pst{\dfrac{\cos dx-1}{dx}} + 
    \pst{\cos x} \cdot \pst{\dfrac{\sen dx}{dx}} = \\
&=  \sen x\cdot 0 + \cos x\cdot 1 = \cos x
\end{align*}
\end{osservazione}

\begin{osservazione}
Anche il grafico dell'andamento delle tangenti conferma la tesi\\
\begin{inaccessibleblock}
  [esponenziale e pendenze]
\affiancati{.55}{.43}{
\hspace*{-10mm} \scalebox{1}{\seno}
}{
\vspace*{-2mm} \hspace*{-17mm} \scalebox{1}{\derivataseno}
}
\end{inaccessibleblock}
\label{}
\begin{center} Il grafico di \(y = \sin x\), con alcune sue tangenti, e 
la sua derivata.\end{center}

% \begin{inaccessibleblock}
%   [differenziale del seno]
%   \begin{minipage}[]{.47\textwidth}
%     \begin{center} \senov \end{center}
%  \end{minipage} 
%   \hfill
%  \begin{minipage}[]{.47\textwidth}
%  \begin{center} \tangentiseno \end{center}
%  \end{minipage}
% \end{inaccessibleblock}
% \label{}
\end{osservazione}

\begin{esempio}
Quale pendenza ha il grafico di \(y=\sen x\) nell'origine?\\
\(f(x)= \sen x \srarrow f'(x)=\cos x\srarrow f'(0)=\cos 0=1\).\\
La tangente al grafico nell'origine è la retta \(y=x\).
\end{esempio}

\begin{esempio}
Derivare le funzioni: 
\(f(x)=\sen 2x; \quad g(x)= \sen x^2; \quad h(x) = \sen^2 x\).
\begin{itemize}
\item \(\Deriv{\sen 2x} = \cos 2x \cdot 2 = 2 \cos 2x\)
\item \(\Deriv{\sen x^2} = \cos x^2 \cdot 2x = 2x \cos x^2\)
\item \(\Deriv{\sen^2 x} = \tonda{2 \sen x} \cdot \cos x = 
                           2 \sen x \cdot \cos x\)
\end{itemize}
\end{esempio}

\begin{esempio}
Derivare \(f(x)=x^{\sen x}\).\\
Si tratta di una funzione di tipo nuovo, un misto fra una funzione potenza
e una funzione esponenziale. Si risolve con una trasformazione che abbiamo
già visto:
\[x^{\sen x}=e^{\ln \tonda{x^{\sen x}}}=e^{\sen x \cdot \ln x}\]
e con l'uso delle regole della funzione composta e del prodotto.
\[\Deriv{e^{\sen x \cdot \ln x}}=
  e^{\sen x \cdot \ln x}\tonda{\cos x \cdot \ln x +\dfrac{\sen x}{x}}=
  x^{\sen x}\tonda{\cos x \cdot \ln x +\dfrac{\sen x}{x}}\]
\end{esempio}

\subsubsection{Derivata di \(f(x)=\cos x\)}
Anche in questo teorema ci riferiamo alla 
figura di pag.\pageref{fig:differenziazione_dsincos}
\begin{teorema}
La derivata della funzione \(f(x)=\cos x\)~~è~~ \(f'(x)=-\sen x\): \quad 
\(\Deriv{\cos x} = -\sen x\)
\end{teorema}
% \noindent Ipotesi: \(f(x)=\cos x\) \tab \(f'(x)=-\sen x\)
\begin{proof}
Poiché l'angolo \(\widehat{ABC}\) è uguale a \(x\), possiamo calcolare 
il segmento orientato \(AC\) usando la trigonometria: 
\quad \(AC = - AB \cdot \cos x\). 
Operando le sostituzioni visualizzate in figura, il Rapporto Differenziale 
risulta:
\[RD = \dfrac{d \sen x}{dx} = \dfrac{AC}{AB} = 
       \dfrac{-AB \cdot \sen x}{AB} = 
       \dfrac{-dx \cdot \sen x}{dx} = -\sen x\]
Dato che \(-\sen x\) è un numero reale, la sua parte standard esiste e 
coincide con il numero stesso, e questa è la derivata.
\end{proof}

\begin{osservazione}
Si può dimostrare lo stesso teorema anche usando la formula di addizione 
del coseno:
\(\cos(\alpha+\beta) = 
\cos \alpha \cdot \cos \beta - \sen \alpha \cdot \sen \beta\) e 
i limiti presentati a pag. \pageref{subsubsec:insnum_fseno}. 
\begin{align*}
RD &= \dfrac{\cos(x+dx)-\cos x}{dx} =
      \dfrac{\cos x \cdot \cos dx - \sen x \cdot \sen dx - \cos x}{dx} = \\
   &= \cos x \dfrac{\cos dx -1}{dx} - \sen x \dfrac{\sen dx}{dx}
\end{align*}
Il Rapporto differenziale ottenuto è un numero finito per cui possiamo 
calcolare la parte standard:
\begin{align*}
\Deriv{\cos x} &= \pst{RD} = 
    \pst{\cos x \dfrac{\cos dx -1}{dx} - \sen x \dfrac{\sen dx}{dx}} = \\
&=  \pst{\cos x} \cdot \pst{\dfrac{\cos dx-1}{dx}} - 
    \pst{\sen x} \cdot \pst{\dfrac{\sen dx}{dx}} = \\
&=  \cos x \cdot 0 - \sen x \cdot 1 = -\sen x
\end{align*}
\end{osservazione}

\begin{osservazione}
Anche il grafico dell'andamento delle tangenti conferma la tesi\\
\begin{inaccessibleblock}
  [esponenziale e pendenze]
\affiancati{.55}{.43}{
\hspace*{-10mm} \scalebox{1}{\coseno}
}{
\vspace*{-2mm} \hspace*{-17mm} \scalebox{1}{\derivatacoseno}
}
\end{inaccessibleblock}
\label{}
\begin{center} Il grafico di \(y = \cos x\), con alcune sue tangenti, e 
la sua derivata.\end{center}
\end{osservazione}

% \begin{teorema}
% La derivata della funzione \(f(x)=\cos x\) è \(\Deriv{\cos x}=-\sen x\).
% \end{teorema}
% % \noindent Ipotesi: \(f(x)=\cos x\). \tab \(f'(x)=-\sen x\).
% \begin{proof}
% Il disegno con cui dimostrare la tesi è uguale a quello di pag.
% \pageref{fig_diff01dseno}. Lo puoi riprodurre, tenendo però l'attenzione
% concentrata sul segmento \(AC\).\\
% L'unica osservazione importante è che nel passare da \(x\) a \( x+dx\), 
% cioè mentre l'angolo cresce, il valore del coseno decresce. 
% Infatti, al contrario di quanto avviene per il seno, nel primo quadrante 
% si ha: \( \cos(x+dx)<\cos x\). 
% Questa è la ragione del segno meno nel risultato.
% \end{proof}
% 
% \begin{inaccessibleblock}
%   [differenziale del coseno]
%   \begin{minipage}[]{.47\textwidth}
%     \begin{center} \cosenov \end{center}
%  \end{minipage} 
%   \hfill
%  \begin{minipage}[]{.47\textwidth}
%  \begin{center} \tangenticoseno \end{center}
%  \end{minipage}
% \end{inaccessibleblock}
% \label{}

\begin{esempio}
Quale pendenza ha il grafico di \(y=\cos x\) per \(x=0\)?\\
\(f(x)= \cos x \srarrow f'(x)=-\sen x\srarrow f'(0)=-\sen 0=0\).\\
In \(x=0\) la tangente al grafico è orizzontale.
\end{esempio} 

\begin{esempio}
Derivare le funzioni: 
\(f(x)=4 \cos \dfrac{x}{2}; \quad h(x) = \cos^2 x; \quad g(x)= \cos x^2\).
\begin{itemize}
\item \(\Deriv{4 \cos \dfrac{x}{2}} = 
                  -4 \sen \dfrac{x}{2} \cdot \dfrac{1}{2} = 
                  -2 \sen \dfrac{x}{2}\)
\item \(\Deriv{\cos^2 x} = \tonda{2 \cos x} \cdot \tonda{-\sen x} = 
                           -2 \sen x \cdot \cos x\)
\item \(\Deriv{\cos x^2} = -\sen x^2 \cdot 2x = -2x \sen x^2\)
\end{itemize}
\end{esempio}

\begin{esempio}
Derivare \(f(x)=\cos^2 x +\sen^2 x\)\\
\(\Deriv{\cos^2 x +\sen^2 x} = 
    2 \cos x \cdot \tonda{-\sen x} + 2 \sen x \cdot \cos x = 
    - 2 \sen x \cdot \cos x + 2 \sen x \cdot \cos x = 0\)
\end{esempio}


\subsubsection{Derivata di \(f(x)=\tg x\)}
La funzione \(f(x)=\tg x\) non è definita per \(x = k \frac{\pi}{2}\), 
quindi anche la derivata non è definita per quei valori di \(x\).

\begin{teorema}
La derivata della funzione \(f(x)=\tg x\)~~è~~ 
\(f'(x)=\dfrac{1}{\cos^2 x}=1+tg^2 x\) % per \(x\ne \pm\frac{\pi}{2}\): 
\[\Deriv{\tg x} = \dfrac{1}{\cos^2 x}=1+tg^2 x\]
\end{teorema}
% \noindent Ipotesi: \(f(x)=\tg x\) \tab 
\(f'(x)=\dfrac{1}{\cos^2 x}=1+tg^2 x\)
\begin{proof}
Usiamo la relazione goniometrica che lega la tangente al seno e al coseno:
\(\tg x=\frac{\sin x}{\cos x}\) e 
il teorema della derivata di un rapporto:
\begin{align*}
\Deriv{\tg x} &= \Deriv{\dfrac{\sin x}{\cos x}} =
\dfrac{\Deriv{\sen x}\cdot \cos x-\sen x\cdot \Deriv{\cos x}} {\cos^2 x}=
\dfrac{\sen^2 x +\cos^2 x}{\cos^2 x} = \dfrac{1}{\cos^2 x}
\end{align*}
La seconda espressione della derivata della tangente si ottiene così:
\[\Deriv{\tg x} = 
\dfrac{\sen^2 x +\cos^2 x}{\cos^2 x} = 
\dfrac{\sen^2 x}{\cos^2 x} + \dfrac{\cos^2 x}{\cos^2 x} = \tg^2+1\]
\end{proof}

\begin{osservazione}
Si può dimostrare lo stesso teorema anche usando le proprietà geometriche 
delle funzioni circolari. Nel disegno è rappresentato un %TODO
incremento non infinitesimo dell'arco \(x\), osserviamo che:

\affiancati{.29}{.69}{
\hspace*{-7mm} \scalebox{1}{\dtana}
}{
\begin{enumerate}
\item \(OH\), \(OA\) e \(OB\) sono lunghi \(1\);
\item nel triangolo rettangolo \(OHC\): l'ipotenusa \(OC\) è lunga 
\(\dfrac{1}{\cos x}\);
\item \(OE\) è equivalente a \(OC\); 
\item i triangoli isosceli \(OAB\) e \(OCE\) sono simili e il rapporto 
di similitudine è \(\dfrac{1}{\cos x}\), ne consegue che:
\(EC = AB \dfrac{1}{\cos x}\).
\end{enumerate}
}

Se passiamo ad un incremento infinitesimo \(dx\), i lati \(OC\) e \(OD\) 
sono indistinguibili nella scala dei reali, per poterli separare dobbiamo 
usare microscopi non standard. Entrambi i microscopi della figura seguente 
hanno lo stesso ingrandimento infinito.

\affiancati{.49}{.49}{
Per il principio di transfer, le proprietà geometriche della figura 
rimangono le stesse anche se i triangoli hanno un angolo infinitesimo.
\begin{enumerate}
\item quindi \(CE\) è uguale a \(AB \cdot \dfrac{1}{\cos x}\);
\item l'arco \(AB\) è l'incremento dell'angolo che, essendo infinitesimo, 
è indistinguibile dal segmento \(AB\); 
\item Il triangolo rettangolo \(CED\)  è simile al triangolo \(OLB\) 
e l'angolo \(ECD\) è \(x\).
\end{enumerate}
}{
\hspace*{-7mm} \scalebox{1}{\dtanb}
}
\[d(tg(x)) = CD = CE\dfrac{1}{\cos x} = 
\dfrac{AB}{\cos x} \cdot \dfrac{1}{\cos x} \sim 
\dfrac{dx}{\cos^2 x}\]
Trovato il differenziale, è immediato calcolare il Rapporto Differenziale:
\[RD = \dfrac{d(tg(x))}{dx} =\dfrac{dx}{dx \cdot \cos^2 x} = 
       \dfrac{1}{\cos^2 x} \]
Il Rapporto differenziale ottenuto è la derivata cercata.
\end{osservazione}

\begin{osservazione}
Anche il grafico dell'andamento delle tangenti conferma la tesi\\
\begin{inaccessibleblock}
  [esponenziale e pendenze]
\affiancati{.55}{.43}{
\hspace*{-10mm} \scalebox{1}{\tangente}
}{
\vspace*{-2mm} \hspace*{-17mm} \scalebox{1}{\derivatatangente}
}
\end{inaccessibleblock}
\label{}
\begin{center} Il grafico di \(y = \tg x\), con alcune sue tangenti, e 
la sua derivata.\end{center}
\end{osservazione}

% \begin{inaccessibleblock}
%   [differenziale della tangente]
%   \begin{minipage}[]{.49\textwidth}
%     \begin{center} \tangentev \end{center}
%  \end{minipage} 
%   \hfill
%  \begin{minipage}[]{.49\textwidth}
%  \begin{center} \tangentitangente \end{center}
%  \end{minipage}
% \end{inaccessibleblock}
% \label{}

\begin {esempio}
Quale è la pendenza del grafico di \(y=\tg x\), per \(x=\dfrac{\pi}{4}\)? 
E per \(x=\dfrac{\pi}{2}\)?\\
\(f'(x)=1+tg^2 x\srarrow f'(\dfrac{\pi}{4})=1+\tg^2\dfrac{\pi}{4}=2\)\\
\(f'(x)=1+tg^2 x\srarrow f'(\dfrac{\pi}{2})=1+\tg^2\dfrac{\pi}{2}=\) ???\\
Per \(x\approx\dfrac{\pi}{2}\) il grafico della funzione cresce 
verticalmente, la sua pendenza è un numero infinito e la parte standard 
di un infinito non esiste. 
D'altra parte, se \(x\) è esattamente uguale a \(\dfrac{\pi}{2}\) ,
la tangente non è definita.
\end {esempio}
% 
% 
% 
% \begin{comment}
% 
% 

\section{Applicazioni}
\label{sec:diff01_applicazioni}
Si è tanto parlato delle tangenti ai grafici di funzione e delle loro 
pendenze, senza mai arrivare a definire l'effettiva equazione delle 
tangenti che interessano. 

\begin{definizione}
La \emph{tangente ad una curva in un punto} \(P\punto{x_P}{y_P}\) 
è la retta passante per quel punto e che ha la stessa pendenza della 
funzione in quel punto. Se \(f'(x_P)\) è definito:
\[y = f'(x_P)\tonda{x - x_P} +y_P\]
Se il Rapporto Differenziale è un infinito:
\[x = x_P\]
\end{definizione}


Hai già incontrato negli anni scorsi dei problemi in cui si chiedeva di 
calcolare la tangente ad una parabola in un suo punto. 
Il metodo di calcolo algebrico che usavi era macchinoso (sistema tra la 
funzione e un fascio di rette, trovare i valori del parametro che 
annullano il discriminante) e, sfortunatamente, vale solo per le coniche. 
Il metodo delle derivate, invece, si rivela molto più generale e rapido.

Poiché la tangente è una retta, la sua equazione è del tipo 
\(y-y_P=m(x-x_P)\), dove  \(\punto{x_P}{y_P}\) è il punto di tangenza 
e \(m\) è la pendenza della retta, 
La pendenza della tangente è uguale alla pendenza della funzione il quel 
punto, quindi sarà: \(m=f'(x_P)\).
Combinando queste due formule otteniamo quanto scritto nella prima parte 
della definizione. 

Ci sono casi in cui la derivata non è definita perché il Rapporto 
Differenziale è infinito in questo caso la tangente esiste, ma non può 
essere scritta come funzione di \(x\), ma con l'equazione delle rette 
parallele all'asse \(y\). 
Come si vede nella seconda parte della definizione.

\begin{esempio}
Trova le equazioni delle tangenti alla parabola \(f(x)=x^2\) nei suoi punti
\(V~\equiv~\punto{0}{f(0)}\) e \(B\equiv\punto{-6}{f(-6)}\).\\
Soluzione. Nel punto \(V\): \(f'(0)~=~2\cdot 0~=~0=m\). La tangente è 
orizzontale e coincide con l'asse \(X\): \(y=~m(x~-~x_0)~+~y_0~=~0\).\\
Nel punto \(B\): \(f'(-6)=2 (-6)=-12\). \(m=-12\), la tangente è inclinata
verso il basso:
\(y=m(x-x_0)+y_0= -12(x+6)+36\srarrow y=-12x-36\).
\end{esempio}

% \pagebreak %--------------------------------------------------

\begin{esempio}
Trova i punti di intersezione con gli assi della tangente
alla curva \(f(x)=2x^3-x\) nel punto \(\punto{2}{f(2)}\).

\noindent Calcolo del valore della funzione in \(2\): \tab 
\(f(2) = 2 \cdot 2^3-2 = 14\)\\
Calcolo della pendenza in \(2\): \tab 
\(f'(x) = 6x^2-1 \srarrow f'(2) = 6 \cdot 2^2-1= 23\) \\
Da cui l'equazione della tangente: \tab 
\(y=23(x-2)+14=23x-32\)\\[5pt]
L'intersezione con l'asse \(x\): \tab 
\(y=0 \srarrow x=\dfrac{32}{23} \srarrow \punto{\dfrac{32}{23}}{0}\)\\[2pt]
con l'asse \(y\): \tab 
\(x=0 \srarrow y=-32 \srarrow \punto{0}{-32}\)

\end{esempio}

\begin{esempio}
In quale punto del suo grafico la parabola \(y=4x^2-3x+6\) è inclinata di 
\(45^\circ\)?

Nel punto che cerchiamo, la parabola avrà un'inclinazione 
indistinguibile da quella della tangente.
Le rette inclinate di \(45^\circ\) hanno pendenza \(m=1\), come la 
bisettrice del primo-terzo quadrante.  
Dobbiamo quindi imporre alla derivata il valore \(1\).
\[\Deriv{4x^2-3x+6} = 8x-3 = 1 \srarrow x =\dfrac{1}{2} 
\sstext{e il punto è:} 
\punto{\dfrac{1}{2}}{\dfrac{11}{2}}\]
\end{esempio}

\begin{esempio}
È vero che l'iperbole equilatera di equazione \(xy=16\) ha per vertici i 
punti medi del segmento che gli assi staccano sulle tangenti ai vertici? 

Consideriamo per comodità solo il ramo destro del grafico.
Il vertice sarà un punto \(V\) di coordinate uguali, essendo l'iperbole
equilatera. \\
Quindi \(V=V\punto{4}{4}\),
perché la funzione è \(y=\dfrac{16}{x}\),\\ 
la sua derivata in \(V\) è \tab 
\(f'(4)= -\dfrac{16}{4^2}=-1\)\\
e l'equazione della tangente in \(V\) è \tab \(y=-1(x-4)+4=-x+8\)\\
La retta \(y=-x+8\) interseca gli assi in: \tab 
\(\punto{8}{0}\) e \(\punto{0}{8}\) \\
ed è facile verificare che il punto \(V\) è medio fra i due. \\
Per ragioni di simmetria accade lo stesso con il vertice opposto 
\(\punto{-4}{-4}\).
\begin{osservazione}
In realtà si tratta di una proprietà generale dell'iperbole equilatera.
Qualsiasi retta tangente al grafico stacca sugli assi coordinati dei 
segmenti che hanno il punto medio coincidente con il punto di tangenza. 
Non è difficile dimostrarlo usando l'equazione generica \(yx=k^2\) e per 
punto di tangenza le coordinate \(\punto{a}{\dfrac{k^2}{a}}\).
\end{osservazione}
\end{esempio}

\begin{esempio}
È vero che è inclinato di \(30^\circ\) il raggio  
della circonferenza \(x^2+y^2=20\) che unisce l'origine al suo punto di 
ascissa \(4\)?

Risposta: No, non è vero. \\
Il modo più elementare per verificarlo è calcolare l'ordinata
del punto e cercare l'angolo di inclinazione dell'ipotenusa coincidente 
con il raggio.

L'alternativa è calcolare la derivata. Ma noi abbiamo imparato a calcolare 
la derivata di una funzione, mentre \(x^2+y^2=20\) non è una funzione.

Dato che il punto considerato appartiene al primo quadrante, consideriamo 
la funzione che rappresenta la semicirconferenza superiore:
\(x^2+y^2=20 \srarrow y=\sqrt{20-x^2}\). 
La sua derivata è:
\[f'(4)= \Deriv{\sqrt{20-x^2}}\bigg|_{x=4} =
\dfrac{-2x}{2\sqrt{20-x^2}}\bigg|_{x=4}=
\dfrac{-4}{\sqrt{20-16}}=\dfrac{-4}{2}=-2\]
Dunque la tangente ha una pendenza pari a \(-2\). Poiché il raggio e la 
tangente sono perpendicolari, la retta che contiene questo raggio avrà 
pendenza \(-\dfrac{1}{-2}=\dfrac{1}{2}\).

Se \(\Delta x\) è positivo, l'angolo è uguale a 
\(\arctan \dfrac{\Delta y}{\Delta x} = 
  \arctan 0,5 \simeq 26,565^\circ \neq 30^\circ\).
\end{esempio}
 
\subsection{Derivata e normale}
\label{}
Come si vede dall'ultimo esempio, una volta che si sappia come calcolare 
la tangente ad una curva, il calcolo della normale risulta molto facile.
Poiché, in ogni punto di una funzione la tangente e la normale 
sono rette perpendicolari e quindi hanno i coefficienti angolari 
antireciproci, l'equazione della normale ad una curva \(y=f(x)\) in un punto 
\(\punto{x_P}{y_P}\)
sarà:\\
\(y=\dfrac{-1}{f'(x_P)}(x-x_P)+y_P\),\\
dove la pendenza della normale \(m_n=\frac{-1}{m_t}\) è appunto 
l'antireciproco della pendenza della tangente.

\begin{esempio}
Scrivi l'equazione della tangente e della normale alla curva di equazione
\(y=\dfrac{x^2-1}{\ln x -1}\) nel suo punto di ascissa \(1\). 

Calcoliamo la pendenza nel punto:
% \[f'(x)|_{x=1}=\dfrac{2x(\ln x-1)-(x^2-1)\dfrac{1}{x}}
\[f'(1)=\dfrac{2x(\ln x-1)-(x^2-1)\dfrac{1}{x}}
{(\ln x -1)^2}\bigg|_{x=1}=\dfrac{2\cdot 1(0-1)-(1-1)\cdot 
1}{(0-1)^2}=-2\]
Calcoliamo l'ordinata del punto: \tab
\(\dfrac{1^2-1}{\ln 1 -1}=0=y_0\)\\
L'equazione della tangente:
\(y=-2(x-1)=-2x+2\).\\
L'equazione della normale: 
\(y=\dfrac{1}{2}(x-1)=\dfrac{1}{2}x-\dfrac{1}{2}\).
\end{esempio}

\begin{esempio}
Scrivi l'equazione della tangente e della normale alla curva di equazione
\(f(x) = \dfrac{x^2+1}{\ln x +1}\) nel suo punto di ascissa \(1\).

\(y = f(1) = \dfrac{x^2+1}{\ln x +1} = \dfrac{1^2+1}{\ln 1 +1}= 
\dfrac{1+1}{0 +1}= 2\)\\
\(f'(1)=
\dfrac{2x(\ln x+1)-(x^2+1)\dfrac{1}{x}}{(\ln x +1)^2}\bigg|_{x=1}=
\dfrac{2\cdot 1(0+1)-(1+1)\cdot 1}{(0+1)^2}=0\).\\
La tangente è quindi la retta orizzontale: \(y=2\),
di conseguenza la normale è la retta verticale: \(x=1\).
\end{esempio}

\subsection{Derivata della derivata}
\label{}
Abbiamo già notato che la derivata  di una funzione dipende dal punto in 
cui si 
calcola e che, una volta stabilito questo punto, ha un unico risultato,
se esiste.
Quindi la derivata di una funzione è a sua volta una funzione e,
se ci sono le condizioni, può essere derivata a sua volta.
\begin{definizione}
 Se una funzione \(f(x)\) è derivabile, la sua derivata è la funzione \(f'(x)\).
 Se anche \(f'(x)\) è derivabile, allora esiste la funzione \(f''(x)\) ed è
 chiamata \emph{derivata seconda di \(f(x)\)}.
\end{definizione}
Le regole di calcolo della derivata seconda sono le stesse regole che 
abbiamo già visto, quindi la seconda derivazione, se è possibile, non comporta 
problemi diversi da quelli conosciuti.

Riferendoci a un generico grafico di funzione \(y=f(x)\), la derivata prima 
\(f'(x)\) ci consente di trovare le pendenze delle tangenti al grafico. La
derivata seconda \(f''(x)\) descrive con quanta rapidità (o lentezza) variano 
queste pendenze, perciò ci indica quanto siano aperte o chiuse le concavità 
che \(y=f(x)\) disegna nel piano cartesiano.

In particolare vedremo che:
\begin{itemize}
\item 
dove la derivata seconda è positiva la pendenza aumenta sempre di più 
quindi la concavità è verso l'alto;
\item 
dove la derivata seconda è negativa la pendenza diminuisce sempre di più 
quindi la concavità è verso il basso;
\end{itemize}

Se le condizioni sono favorevoli, esistono e sono calcolabili anche le 
derivate terze, quarte, ecc. di una funzione, anche se non sono essenziali 
per i nostri scopi. Il loro calcolo segue i metodi già visti.

\begin{esempio}
Calcola \(f''(1)\) di \(f(x)=2x^5-3x^4+x^3+5x^2-6x+9\).

Derivata prima: 
\(f'(x)=10 x^4-12 x^3+3x^2+10 x-6\).\\
Derivata seconda per \(x=1\): 
\(f''(x)|_{x=1} = (40x^3-36x^2+6x+10)|_{x=1}=40-36+6+10=20\).
\end{esempio}

\begin{esempio}
Calcola \(f''(x)\) di \(f(x)=\ln x\).

\(f'(x)=\dfrac{1}{x}\) e \(f''(x)=-\dfrac{1}{x^2}\)
\begin{osservazione}
La funzione \(\ln x\) esiste per \(x>0\). Le derivate prima e seconda 
esistono per \(x\ne 0\).
Osserviamo quindi che il campo di esistenza della funzione derivata (prima, 
seconda, terza \dots) può non coincidere con il campo di esistenza della 
funzione da derivare. 
% TODO: mi sembra di capire che la derivata esiste anche dove non esiste la 
% f(x).
\end{osservazione}
\end{esempio}

\begin{esempio}
Calcola le derivate successive di \(f(x)=\sen x\).\\
\(f'(x)=\cos x\) \hspace{1cm}  \(f''(x)=-\sen x\) \hspace{1cm}
\(f'''(x)=-\cos x\) \hspace{1cm} \(f^{IV}(x)=\sen x\) \dots
\end{esempio}

\end{comment}

