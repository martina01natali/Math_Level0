% (c) 2016 Daniele Zambelli - daniele.zambelli@gmail.com

% \newcommand{\gnomino}[7]{
%   % example:
%   % \disegno{\gnomino{0.7}{1.5}{f(a)}{a}{gray!50}{blue!50!black}{.69}}
%   \def \x{#1}
%   \def \y{#2}
%   \def \lx{#3}
%   \def \ly{#4}
%   \def \colorline{#5}
%   \def \colorpoint{#6}
%   \def \lbelow{#7}
%   \draw [thin, dashed, \colorline] 
%         (0, \y) node [left] {\(\ly\)} -- (\x, \y) --
%         (\x, 0) node [below=\lbelow] {\(\lx\)};
%   \filldraw (\x, \y) [\colorpoint] circle(1.5pt);
% }

\newcommand{\xbar}[7]{
  % example:
  % \disegno{\xbar{0.7}{1.5}{a_1}{gray!50}{blue!50!black}{red}{0}}
  \def \x{#1}
  \def \y{#2}
  \def \lx{#3}
  \def \colorline{#4}
  \def \colorpoint{#5}
  \def \colorlabel{#6}
  \def \lbelow{#7}
  \draw [dotted, \colorline]
        (\x, 0) node [\colorlabel, below=\lbelow] {\(\lx\)} -- (\x, \y);
  \filldraw (\x, \y) [\colorpoint] circle(1.5pt);
}

% \newcommand{\ybar}[4]{
%   % example:
%   % \disegno{\ybar{0.7}{1.5}{a_1}{gray!50}}
%   \def \x{#1}
%   \def \y{#2}
%   \def \ly{#3}
%   \def \colorline{#4}
%   \draw [dotted, \colorline]
%         (0, \y) node [left] {\(\ly\)} -- (\x, \y);
% }

\def \graficobasea{
    \def \xi{-0.3};
    \def \yi{4.7};
    \def \xf{10.3};
    \def \yf{-1.5};
    \def \xa{0.4};
    \def \ya{2.9};
    \coordinate (i) at (\xi, \yi);
    \coordinate (f) at (\xf, \yf);
    \coordinate (a) at (\xa, \ya);
    \coordinate (b) at (\xb, \yb);
    \coordinate (ctrli) at (3, -6);
    \coordinate (ctrlf) at (7, 15);
    \def \linea{(i) .. controls (ctrli) and (ctrlf) .. (f)}
    \rcom{0}{+10}{0}{+8}{gray!50, very thin, step=1}
    \begin{scope}
      \clip (\xi, -0.3) rectangle (\xf, 8);
      \draw [ultra thick, green!50!black] \linea;
    \end{scope}
    \xbar{\xa}{\ya}{a}{white!30!black}{blue!50!black}{black}{0.79}
    \xbar{\xb}{\yb}{b}{white!30!black}{blue!50!black}{black}{0.6}
}

\newcommand{\partizionen}{% Valutazione di f in alcuni punti con n fissato
  \disegno{
    \def \xb{8.4};
    \def \yb{4.25};
    \graficobasea
    \foreach \xp/\yp/\lab in {0.4/2.9/a_0, 1.4/1.67/a_1, 2.4/1.57/a_2, 
                              3.4/2.17/~, 4.4/3.15/~, 5.4/4.17/a_i, 
                              6.4/4.9/~, 7.4/5.06/~, 8.4/4.25/a_n}{
      \xbar{\xp}{\yp}{\lab}
            {white!30!black}{blue!50!black}{blue!50!black}{0}}
    \foreach \xp in {3.9, 6.9}{
      \node at (\xp, 0) [white!30!black, below, yshift=-6pt] {\(\dots\)};}
  }
}

\newcommand{\limitigraficoa}{% 
  \def \funzione{(x**2-6*x+5)/(x**2+2*x-3)}
  \disegno{
    \rcom{-18}{+15}{-10}{+10}{gray!50, very thin, step=1}
    \tkzInit[xmin=-18.3,xmax=+15.3,ymin=-10.3,ymax=+10.3]
    \tkzFct[domain=-18.3:-3.1, ultra thick, color=Maroon!50!black]
         {\funzione}
    \tkzFct[domain=-2.9:+15.3, ultra thick, color=Maroon!50!black]
         {\funzione}
  }
}

\newcommand{\limmicx}[8]{% 
  % interno del microscopio posto sull'asse x.
  \def \xa{#1} \def \xb{#2} \def \xc{#3} \def \xd{#4}
  \def \ya{#5} \def \yb{#6} \def \yc{#7}
  \def \lab{#8}
  \draw (\xa, \ya) -- (\xd, \ya);
  \draw (\xa, \ya) -- (\xd, \ya);
  \draw [Green!50!black, ultra thick] (\xc, \yb) -- (\xb, \yb);
  \draw [Maroon!50!black, ultra thick] (\xb, \ya) node [below] {\lab} -- 
                                       (\xb, \yb);
  \draw (\xb, \yb) -- (\xb, \yc);
}

\newcommand{\limiteseno}{% 
\disegno[20]{
  \rcom{-1.0}{+1.0}{-1.0}{+1.0}{gray!50, very thin, step=1}
  \draw [Maroon!50!black, ultra thick] (0, 0) circle (1);
  \draw [Green!50!black, ultra thick] (0, 0) -- (1.45, 0);
  \microscopio{(1, 0)}{.3}{40}{230}{.5}{(1.8, 1.2)}{\(\times \infty\)}
  \limmicx{1.14}{1.6}{1.07}{1.97}{.3}{.7}{1.08}{1}
  \draw (1.7, .5) node {$\delta$};
  \microscopio{(0, 0)}{.3}{120}{320}{.5}{(-.8, 1.2)}{\(\times  \infty\)}
  \limmicx{-0.12}{-0.5}{-0.05}{-0.95}{.3}{.7}{1.08}{0}
  \draw (-0.75, .5) node {$\sen \delta$};
  }
}

% La seguente non funzione, sballa il colore della griglia di rcom!!!!!
% \newcommand{\sincos}[3]{%
% \def \funzc{#1} 
% \def \funzl{#2} 
% \def \color{#3} 
% \disegno[7]{
%   \rcom{-6.5}{+6.5}{-1.0}{+1.0}{gray!50, very thin, step=1}
%     \tkzInit[xmin=-6.5,xmax=+6.5,ymin=-1.3,ymax=+1.3]
%     \tkzFct[domain=-6.5:+6.5, ultra thick, #3] 
%            {\funzc}
%     \node at (0, -1.5) {\funzl};
%   }
% }

\newcommand{\sinusoide}{%
\disegno[5]{
  \rcom{-6.5}{+6.5}{-1.0}{+1.0}{gray!50, very thin, step=1}
    \tkzInit[xmin=-6.8,xmax=+6.8,ymin=-1.3,ymax=+1.3]
    \tkzFct[domain=-6.8:+6.8, ultra thick, color=Blue!50!black]
         {sin(x)}
    \node at (0, -1.8) {$y=\sen x$};
  }
}

\newcommand{\cosinusoide}{%
\disegno[5]{
  \rcom{-6.5}{+6.5}{-1.0}{+1.0}{gray!50, very thin, step=1}
    \tkzInit[xmin=-6.8,xmax=+6.8,ymin=-1.3,ymax=+1.3]
    \tkzFct[domain=-6.8:+6.8, ultra thick, color=Red!50!black]
         {cos(x)}
    \node at (0, -1.8) {$y=\cos x$};
  }
}

\newcommand{\tangentoide}{%
\disegno[5]{
  \rcom{-3.5}{+3.5}{-4.0}{+4.0}{gray!50, very thin, step=1}
    \tkzInit[xmin=-3.8,xmax=+3.8,ymin=-4.3,ymax=+4.3]
    \tkzFct[domain=-3.8:-1.8, ultra thick, color=Green!50!black]
         {tan(x)}
    \tkzFct[domain=-1.4:+1.4, ultra thick, color=Green!50!black]
         {tan(x)}
    \tkzFct[domain=+1.6:+3.8, ultra thick, color=Green!50!black]
         {tan(x)}
    \node at (0, -4.8) {$y=\tan x$};
  }
}

\newcommand{\limitigraficob}{% 
  \def \funzione{(x**2-6*x+5)/(x**2+2*x-3)}
  \disegno{
    \rcom{-3}{+3}{-3}{+3}{gray!50, very thin, step=1}
    \tkzInit[xmin=-3.3,xmax=+3.3,ymin=-3.3,ymax=+3.3]
    \tkzFct[domain=-2.9:+0.91, thick, color=Maroon!50!black]
         {\funzione}
    \tkzFct[domain=+1.1:+3.3, thick, color=Maroon!50!black]
         {\funzione}
    \draw [Maroon!50!black] (1, -1) circle (2pt);
  }
}

\newcommand{\continuitagraficoa}{%7
  \disegno{
    \rcom{-5}{+7}{-7}{+7}{gray!50, very thin, step=1}
    \tkzInit[xmin=-5.3,xmax=+7.3,ymin=-7.3,ymax=+7.3]
    \tkzFct[domain=-5.3:2, ultra thick, color=Maroon!50!black]
         {.5*x - 2}
    \tkzFct[domain=2:+7.3, ultra thick, color=Maroon!50!black]
         {x**2-6*x+7}
  }
}

\newcommand{\continuitagraficoese}{%7
  \disegno{
    \rcom{-12}{+12}{-7}{+7}{gray!50, very thin, step=1}
    \tkzInit[xmin=-12.3,xmax=+12.3,ymin=-7.3,ymax=+7.3]
    \tkzFct[domain=-12.3:-2.1, ultra thick, color=Maroon!50!black]
         {x/(x + 2)-2}
    \tkzFct[domain=-1.9:+0.9, ultra thick, color=Maroon!50!black]
         {x/(x + 2)-2}
    \tkzFct[domain=1:+3.95, ultra thick, color=Maroon!50!black]
         {x**2-6*x+7}
    \tkzFct[domain=+4.05:5, ultra thick, color=Maroon!50!black]
         {x**2-6*x+7}
    \tkzFct[domain=+5:12.3, ultra thick, color=Maroon!50!black]
         {1/(x-4)+1}
    \filldraw [Maroon!50!black] (1, 2) circle (2pt);
    \draw [Maroon!50!black] (1, -1.7) circle (2pt);
    \draw [Maroon!50!black] (4, -1) circle (2pt);
  }
}
