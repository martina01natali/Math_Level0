% (c) 2016 Daniele Zambelli - daniele.zambelli@gmail.com

\newcommand{\gnomino}[7]{
  % example:
  % \disegno{\gnomino{0.7}{1.5}{f(a)}{a}{gray!50}{blue!50!black}{.69}}
  \def \x{#1}
  \def \y{#2}
  \def \lx{#3}
  \def \ly{#4}
  \def \colorline{#5}
  \def \colorpoint{#6}
  \def \lbelow{#7}
  \draw [thin, dashed, \colorline] 
        (0, \y) node [left] {\(\ly\)} -- (\x, \y) --
        (\x, 0) node [below=\lbelow] {\(\lx\)};
  \filldraw (\x, \y) [\colorpoint] circle(1.5pt);
}

\newcommand{\xbar}[7]{
  % example:
  % \disegno{\xbar{0.7}{1.5}{a_1}{gray!50}{blue!50!black}{red}{0}}
  \def \x{#1}
  \def \y{#2}
  \def \lx{#3}
  \def \colorline{#4}
  \def \colorpoint{#5}
  \def \colorlabel{#6}
  \def \lbelow{#7}
  \draw [dotted, \colorline]
        (\x, 0) node [\colorlabel, below=\lbelow] {\(\lx\)} -- (\x, \y);
  \filldraw (\x, \y) [\colorpoint] circle(1.5pt);
}

% \newcommand{\ybar}[4]{
%   % example:
%   % \disegno{\ybar{0.7}{1.5}{a_1}{gray!50}}
%   \def \x{#1}
%   \def \y{#2}
%   \def \ly{#3}
%   \def \colorline{#4}
%   \draw [dotted, \colorline]
%         (0, \y) node [left] {\(\ly\)} -- (\x, \y);
% }

\def \graficobasea{
    \def \xi{-0.3};
    \def \yi{4.7};
    \def \xf{10.};
    \def \yf{-0.3};
    \coordinate (i) at (\xi, \yi);
    \coordinate (f) at (\xf, \yf);
    \coordinate (ctrli) at (3, -6);
    \coordinate (ctrlf) at (7, 15);
    \def \linea{(i) .. controls (ctrli) and (ctrlf) .. (f)}
    \rcom{0}{+10}{0}{+8}{gray!50, very thin, step=1}
%       \draw [dashed, ultra thick, white!70!black] \linea;
    \begin{scope}
      \clip (\xi, -0.3) rectangle (\xf, 8);
      \draw [dashed, ultra thick, white!70!black] \linea;
    \end{scope}
}

\def \puntipartizionen{4.17, 5.66, 5.89, 5.28, 4.25, 3.22, 2.61, 2.84, 4.33}
\def \puntipartizionevuoti{(4, 4.88), (7, 2.19)}

\newcommand{\partizionen}{% Valutazione di f in alcuni punti con n fissato
  \def \fa{.07*x*x*x-1.05*x*x+4.15*x+1}
  \def \mix{0}
  \def \max{10}
  \def \miy{0}
  \def \may{8}
  \disegno{
    \rcom{\mix}{\max}{\miy}{\may}{gray!50, very thin, step=1}
    \tkzInit[xmin=\mix-0.3,xmax=\max+0.3,ymin=\miy-0.3,ymax=\may+0.3]
    \tkzFct[domain=\mix-0.3:\max+.3, ultra thick, brown!50!black] {\fa}
    \foreach \yp [count = \xp] in \puntipartizionen{
      \xbar{\xp}{\yp}{x_\xp}
            {white!30!black}{blue!50!black}{blue!50!black}{0}};
  }
}

% \newcommand{\partizionen}{% Valutazione di f in alcuni punti con n fissato
%   \disegno{
%     \def \xb{8.4};
%     \def \yb{4.25};
%     \graficobasea
%     \foreach \xp/\yp/\lab in {0.4/2.9/x_0, 1.4/1.67/x_1, 2.4/1.59/x_2, 
%                               3.4/2.25/x_3, 4.4/3.27/x_4, 5.4/4.4/x_5, 
%                               6.4/5.25/x_6, 7.4/5.5/x_7, 8.4/4.67/x_8,
%                               9.4/2.25/x_9}{
%       \xbar{\xp}{\yp}{\lab}
%             {white!30!black}{blue!50!black}{blue!50!black}{0}}
%     \node at (5, -1.3) {\footnotesize{Funzione valutata in alcuni punti}};
%   }
% }

\newcommand{\xbarx}[6]{
  % example:
  % \disegno{\xbarx{0.7}{1.5}{a}{gray!50}{blue!50!black}{red}}
  \def \x{#1}
  \def \funct{#2}
  \def \lx{#3}
  \def \colorline{#4}
  \def \colore{#5}
  \def \colorfill{#6}
  \draw [dotted, \colorline]
        (\x, 0) node [\colore, below] {\(\lx\)} -- (\x, \funct);
  \filldraw (\x, \funct) [\colore, fill=\colorfill] circle(1.5pt);
}

\newcommand{\puntigrafico}{% Solo i punti evidenziati nel grafico precedente
  \def \fa{.07*x*x*x-1.05*x*x+4.15*x+1}
  \def \mix{0}
  \def \max{10}
  \def \miy{0}
  \def \may{8}
  \disegno{
    \rcom{\mix}{\max}{\miy}{\may}{gray!50, very thin, step=1}
    \tkzInit[xmin=\mix-0.3,xmax=\max+0.3,ymin=\miy-0.3,ymax=\may+0.3]
%     \tkzFct[domain=\mix-0.3:\max+.3, ultra thick, brown!50!black] {\fa}
    \foreach \yp [count = \xp] in \puntipartizionen{
      \xbar{\xp}{\yp}{x_\xp}
            {white!30!black}{blue!50!black}{blue!50!black}{0}};
  }
}

\newcommand{\graficodiscontinuo}{% 
  \def \fa{-0.62*x*x+3.34*x+1.46}
  \def \fb{-1.03*x+9.4}
  \def \fc{1/(-(1.65*(x-3.65)+0.66)+10)+2.35}
  \def \mix{0}
  \def \max{10}
  \def \miy{0}
  \def \may{8}
  \def \xdisca{4}
  \def \xdiscb{7}
  \def \asint{9.25}
  \disegno{
    \rcom{\mix}{\max}{\miy}{\may}{gray!50, very thin, step=1}
    \tkzInit[xmin=\mix-0.3,xmax=\max+0.3,ymin=\miy-0.3,ymax=\may+0.3]
    \begin{scope} [thick, brown!50!black]
      \tkzFct[domain=\mix-0.3:\xdisca] {\fa}
      \tkzFct[domain=\xdisca:\xdiscb] {\fb}
      \tkzFct[domain=\xdiscb:\asint] {\fc}
      \tkzFct[domain=\asint+0.1:\max+0.3] {\fc}
    \end{scope}
    \foreach \yp [count = \xp] in \puntipartizionen{
      \xbar{\xp}{\yp}{x_\xp}
            {white!30!black}{blue!50!black}{blue!50!black}{0}};
    \foreach \punto in \puntipartizionevuoti{
      \filldraw \punto [fill=white] circle(1.5pt);}
  }
}

% \newcommand{\puntia}{%
%   \def \fa{.5*\x+1.5}
%   \def \fb{-.5*\x*\x+4*\x-2}
%   \def \fc{1./(\x-\xc)}
%   \def \mi{0}
%   \def \ma{10}
%   \def \xa{0.3}
%   \def \xb{2.8}
%   \def \xc{5.4}
%   \def \xd{9.8}
%   \rcom{\mi}{\ma}{0}{+8}{gray!50, very thin, step=1}
%   \foreach \yp [count = \xp] in {2, 2.5, 
%                                   5.5, 6, 5.5, 
%                                   1.667, 0.625, .385, .278}{
%     \xbarx{\xp}{\yp}{x_\xp}{gray!90}{brown!50!black}{brown!50!black}}
% }

% \newcommand{\puntigraficodiscontinuo}{%
%   \disegno{
%     \puntia
%     \foreach \xp/\lab in {\xa/a, \xd/b}{
%       \draw [dotted, gray!50]
%         (\xp, 0) node [black, below] {\(\lab\)} -- (\xp, 8.3);}
%     \node at (5, -1.3) 
%       {\footnotesize \emph{Funzione poco regolare valutata in alcuni punti}};
%   }
% }

% \newcommand{\graficodiscontinuo}{%
%   \disegno{
%     \puntia
%     \tkzInit[xmin=\mi-0.3,xmax=\ma+0.3,ymin=-0.3,ymax=+8.3]
%     \begin{scope} [thick, brown!50!black]
%       \tkzFct[domain=\mi-0.3:\xb] {\fa}
%       \tkzFct[domain=\xb:\xc] {\fb}
%       \tkzFct[domain=\xc+.1:\ma+0.3] {\fc}
%     \filldraw (\xb, 2.9) [fill=white] circle(1.5pt);
%     \filldraw (\xb, 5.28) circle(1.5pt);
%     \filldraw (\xc, 5.02) circle(1.5pt);
%     \end{scope}
%     \xbarx{\xa}{\fa}{a}{gray!90}{brown!50!black}{brown!50!black}
%     \xbarx{\xd}{.227}{b}{gray!90}{brown!50!black}{brown!50!black}
%     \node at (5, -1.3) 
%       {\footnotesize \emph{Grafico della funzione poco regolare}};
%   }
% }

\newcommand{\contprimo}{%
  \def \fa{-\x*\x+0*x+5}
  \def \fm{-2*\x+13}
  \def \mi{-2}
  \def \ma{6}
  \def \xc{1}
  \def \yc{4}
  \def \xdma{1.65}
  \def \ydma{3.7}
  \def \xdmb{2.9}
  \def \ydmb{1.22}
  \disegno{
    \rcom{\mi}{\ma}{0}{+8}{gray!50, very thin, step=1}
    \tkzInit[xmin=\mi-0.3,xmax=\ma+0.3,ymin=-0.3,ymax=+8.3]
    \tkzFct[domain=\mi-0.3:\ma+0.3, ultra thick, green!50!black] {\fa}
    \draw [thin, dashed, black] 
          (0, \yc) node [left] {\footnotesize \(\approx \quad ~\)} 
          node [above left] {\footnotesize \(f(\xc) \quad\)} 
          node [below left] {\footnotesize \(f(\xc + \epsilon)\)} -- 
          (\xc, \yc) -- (\xc, 0) 
          node [below] {\footnotesize \(\xc \approx \xc + \epsilon\)};
    \filldraw (\xc, \yc) [green!50!black] circle(1.5pt);
    \microscopio{(\xc, \yc)}{1}{40}{230}{2}{(\xc+4, \yc+3.8)}
                {\footnotesize \(\times \infty\)}
    \tkzFct[domain=\xc+1.46:\xc+3.22, ultra thick, green!50!black] {\fm}
    \draw [thin, dashed, black] (\xc + 0.8, \yc + \ydma) 
          node [left] {\footnotesize \(f(\xc)\)} -- 
          (\xc + \xdma, \yc + \ydma) -- (\xc + \xdma, \yc + 0.3) 
          node [below ] {\footnotesize \(\xc\)};
    \filldraw (\xc + \xdma, \yc + \ydma) [green!50!black] circle(1.5pt);
    \draw [thin, dashed, black] (\xc + 0.4, \yc + \ydmb) 
          node [left] {\footnotesize \(f(\xc + \epsilon)\)} -- 
          (\xc + \xdmb, \yc + \ydmb) -- (\xc + \xdmb, \yc + 0.3) 
          node [below] {\footnotesize \(\xc + \epsilon\)};
    \filldraw (\xc + \xdmb, \yc + \ydmb) [green!50!black] circle(1.5pt);
  }
}

\newcommand{\contsecondo}{%
  \def \fa{abs(x)/x}
  \def \mix{-2}
  \def \max{+2}
  \def \miy{-1.}
  \def \may{+1.}
  \disegno[10]{
    \rcom{\mix}{\max}{\miy}{\may}{gray!50, very thin, step=1}
    \tkzInit[xmin=\mix-0.3,xmax=\max+0.3,ymin=\miy-0.3,ymax=\may+0.3]
    \tkzFct[domain=\mix-0.3:0, ultra thick, green!50!black] {\fa}
    \tkzFct[domain=0:\max+0.3, ultra thick, green!50!black] {\fa}
    \filldraw [green!50!black, fill= white] 
      (0, -1) circle(1.5pt) (0, +1) circle(1.5pt);
  }
}

\newcommand{\fsegno}{%
  \def \fa{abs(x)/x}
  \def \mix{-2}
  \def \max{+2}
  \def \miy{-1.}
  \def \may{+1.}
  \disegno[10]{
    \rcom{\mix}{\max}{\miy}{\may}{gray!50, very thin, step=1}
    \tkzInit[xmin=\mix-0.3,xmax=\max+0.3,ymin=\miy-0.3,ymax=\may+0.3]
    \tkzFct[domain=\mix-0.3:0, ultra thick, green!50!black] {\fa}
    \tkzFct[domain=0:\max+0.3, ultra thick, green!50!black] {\fa}
    \filldraw (0, -1) [green!50!black, fill= white] circle(1.5pt);
    \filldraw (0, +1) [green!50!black, fill= white] circle(1.5pt);
    \filldraw (0, 0) [green!50!black] circle(1.5pt);
    \microscopio{(0, 0)}{.5}{20}{210}{1}{(+2, +1.7)}
                {\footnotesize \(\times \infty\)}
    \filldraw (1.4, 0.7) [green!50!black] circle(1.5pt)
      node [below, black] {\(0\)} node [left, black] {\(f(0)\)};
    \microscopio{(0, +1)}{.5}{+180}{20}{1}{(-2, +1.7)}
                {\footnotesize \(\times \infty\)}
    \draw [ultra thick, green!50!black] (-1.5, +0.7) 
      node [below, black] {\(0\)} -- (-0.44, +0.7); 
    \filldraw (-1.5, +0.7) [green!50!black, fill= white] circle(1.5pt);
    \microscopio{(0, -1)}{.5}{-30}{160}{1}{(+2.2, -2.5)}
                {\footnotesize \(\times \infty\)}
    \draw [ultra thick, green!50!black] (1.4, -1.7) 
      node [below, black] {\(0\)} -- (0.38, -1.7); 
    \filldraw (1.4, -1.7) [green!50!black, fill= white] circle(1.5pt);
  }
}

\newcommand{\contterzo}{%
  \def \fa{abs(x)/4}
  \def \fb{abs(x-1.2)/4+2.4}
  \def \mix{-4}
  \def \max{+4}
  \def \miy{-1}
  \def \may{+4}
  \def \xc{0}
  \def \yc{0}
  \def \xdma{1.65}
  \def \ydma{3.7}
  \def \xdmb{2.9}
  \def \ydmb{1.22}
  \disegno{
    \rcom{\mix}{\max}{\miy}{\may}{gray!50, very thin, step=1}
    \tkzInit[xmin=\mix-0.3,xmax=\max+0.3,ymin=\miy-0.3,ymax=\may+0.3]
    \tkzFct[domain=\mix-0.3:\max+0.3, ultra thick, green!50!black] {\fa}
    \microscopio{(\xc, \yc)}{1}{60}{250}{2}{(\xc+3, \yc+4.4)}
                {\footnotesize \(\times \infty\)}
    \tkzFct[domain=-0.8:+3.18, ultra thick, green!50!black] {\fb}
    \draw [thin, dashed, black] (1.2, 2.4) 
          node [below] {\(0\)} -- (-0.8, 2.4) 
          node [left] {\(f(0)\)};
  }
}

\newcommand{\continuitagraficoa}{%
  \def \xc{2}
  \def \yc{-1}
  \disegno{
    \rcom{-3}{+7}{-3}{+5}{gray!50, very thin, step=1}
    \tkzInit[xmin=-3.3,xmax=+7.3,ymin=-3.3,ymax=+5.3]
    \tkzFct[domain=-3.3:2, ultra thick, color=green!50!black]
         {.5*x - 2}
    \tkzFct[domain=2:+7.3, ultra thick, color=green!50!black]
         {x**2-6*x+7}
    \filldraw (\xc, \yc) [green!50!black] circle(1.5pt);
    \microscopio{(\xc, \yc)}{1}{100}{290}{2}{(+2.4, +3.8)}
                {\footnotesize \(\times \infty\)}
    \tkzFct[domain=-0.73:+1.04, ultra thick, color=green!50!black]
         {.5*x + 1.5}
    \tkzFct[domain=+1.:+1.97, ultra thick, color=green!50!black]
         {-2*x + 4}
    \draw [thin, dashed, black] (-0.8, 2) 
          node [left] {\(f(\xc)\)} -- 
          (1, 2) -- (1, - 0.1) 
          node [below ] {\(\xc\)};
    \filldraw (\xc - 1, \yc + 3) [green!50!black] circle(1.5pt);
  }
}

\newcommand{\contsecondoa}{%
  \def \fa{abs(x)/x}
  \def \mix{-2}
  \def \max{+2}
  \def \miy{-1.}
  \def \may{+1.}
  \disegno[10]{
    \rcom{\mix}{\max}{\miy}{\may}{gray!50, very thin, step=1}
    \tkzInit[xmin=\mix-0.3,xmax=\max+0.3,ymin=\miy-0.3,ymax=\may+0.3]
    \tkzFct[domain=\mix-0.3:0, ultra thick, green!50!black] {\fa}
    \tkzFct[domain=0:\max+0.3, ultra thick, green!50!black] {\fa}
    \filldraw [green!50!black, fill= white] 
      (0, -1) circle(1.5pt) (0, +1) circle(1.5pt);
    \node at (1, -0.5) {\(y = \dfrac{\abs{x}}{x}\)};
  }
}

\newcommand{\contrad}{%
  \def \fa{sqrt(x)}
  \def \mix{-4}
  \def \max{+4}
  \def \miy{-4}
  \def \may{+4}
  \disegno{
    \rcom{\mix}{\max}{\miy}{\may}{gray!50, very thin, step=1}
    \tkzInit[xmin=\mix-0.3,xmax=\max+0.3,ymin=\miy-0.3,ymax=\may+0.3]
    \tkzFct[domain=0:\max+0.3, ultra thick, green!50!black] {\fa}
    \node at (1.5, -1.5) {\(y = \sqrt{x}\)};
  }
}

\newcommand{\contip}{%
  \def \fa{1./x}
  \def \mix{-4}
  \def \max{+4}
  \def \miy{-4}
  \def \may{+4}
  \disegno{
    \rcom{\mix}{\max}{\miy}{\may}{gray!50, very thin, step=1}
    \tkzInit[xmin=\mix-0.3,xmax=\max+0.3,ymin=\miy-0.3,ymax=\may+0.3]
    \tkzFct[domain=\mix-0.3:-0.2, ultra thick, green!50!black] {\fa}
    \tkzFct[domain=+0.2:\max+0.3, ultra thick, green!50!black] {\fa}
    \node at (1.5, -1.5) {\(y = \dfrac{1}{x}\)};
  }
}

\newcommand{\continuitafcostante}{%
  \def \fa{+2.5}
  \def \fam{+5.3}
  \def \mix{0}
  \def \max{8}
  \def \miy{0}
  \def \may{7}
  \def \xp{3.5}
  \def \yp{2.5}
  \disegno{
    \rcom{\mix}{\max}{\miy}{\may}{gray!50, very thin, step=1}
    \tkzInit[xmin=\mix-0.3,xmax=\max+0.3,ymin=\miy-0.3,ymax=\may+0.3]
    \tkzFct[domain=\mix-0.3:\max+0.3, ultra thick, green!50!black] {\fa}
    \node at (-.5, 3.5) [black] {\(k\)};
    \microscopio{(\xp, \yp)}{1}{60}{250}{2}{(\xp+3, \yp+4.4)}
                {\footnotesize \(\times \infty\)}
    \tkzFct[domain=\xp-0.8:\xp+3.15, ultra thick, green!50!black] {\fam}
  }
}

\newcommand{\continuitafidentica}{%
  \def \fa{x}
  \def \fam{x-4}
  \def \mix{0}
  \def \max{8}
  \def \miy{0}
  \def \may{7}
  \def \xp{3.5}
  \def \yp{3.5}
  \disegno{
    \rcom{\mix}{\max}{\miy}{\may}{gray!50, very thin, step=1}
    \tkzInit[xmin=\mix-0.3,xmax=\max+0.3,ymin=\miy-0.3,ymax=\may+0.3]
    \tkzFct[domain=\mix-0.3:\max+0.3, ultra thick, green!50!black] {\fa}
    \microscopio{(\xp, \yp)}{1}{-60}{-200}{2}{(\xp+4.4, \yp-3.1)}
                {\footnotesize \(\times \infty\)}
    \tkzFct[domain=\xp+1.0:\xp+3.85, ultra thick, green!50!black] {\fam}
  }
}

\begin{comment}
\begin{tikzpicture}
  \draw
    (3,-1) coordinate (a) node[right] {a}
    -- (0,0) coordinate (b) node[left] {b}
    -- (2,2) coordinate (c) node[above right] {c}
    pic["$\alpha$", draw=orange, <->, angle eccentricity=1.2, angle radius=1cm]
    {angle=a--b--c};
\end{tikzpicture}
\end{comment}

\newcommand{\continuitasincosa}{%
  \def \fa{sqrt(36-x**2)}
%   \def \fa{-sqrt(1-x**2}
  \def \fam{x-4}
  \def \mix{0}
  \def \max{5}
  \def \miy{0}
  \def \may{5}
  \def \pp{(25:\max)}
  \def \pq{(45:\max)}
  \disegno{
    \rcom{\mix}{\max}{\miy}{\may}{white}
    \begin{scope}
      \clip (\mix-0.3, \miy-0.3) rectangle (\max+0.3, \may+0.3);
      \draw [thick, green!50!black] (0, 0) circle (\max);
    \end{scope}
    \draw (0, 0) -- \pp coordinate (p) node [right] 
      {\footnotesize\(P\)} 
                        node [right, yshift=12] {\footnotesize\(\Delta y\)}
          (0, 0) -- \pq coordinate (q) node [above, xshift=-3] 
      {\footnotesize\(Q\)} 
                        node [above, xshift=12] {\footnotesize\(\Delta x\)};
    \draw (p) -- (q) (p) |- (q);
    \draw (1, 0) coordinate (a) (0,0) coordinate (o)
      pic["\footnotesize$\theta$", draw=orange, ->, 
          angle eccentricity=1.2, angle radius=10mm]
      {angle=a--o--p}
      pic["\footnotesize$\Delta$", draw=orange, ->, 
          angle eccentricity=1.2, angle radius=12mm]
      {angle=p--o--q};
  }
}

\newcommand{\continuitasincosb}{%
  \def \fa{sqrt(36-x**2)}
  \def \fam{-2.1*x+17}
  \def \mix{0}
  \def \max{5}
  \def \miy{0}
  \def \may{5}
  \def \pp{(25:\max)}
  \def \pm{(6.44, 3.5)}
  \def \qm{(5.48, 5.5)}
  \def \ppm{(6.44-0.6*2.1, 3.5-0.6)}
  \def \qqm{(5.48-0.65*2.1, 5.5-0.65)}
%   \def \xpm{6.44}
%   \def \ypm{3.5}
%   \def \xqm{5.48}
%   \def \yqm{5.5}
  \disegno{
    \rcom{\mix}{\max}{\miy}{\may}{white}
    \begin{scope}
      \clip (\mix-0.3, \miy-0.3) rectangle (\max+3.2, \may+1.7);
      \draw [thick, green!50!black] (0, 0) circle (\max);
    \draw (0, 0) -- \pp coordinate (p) node [right] 
      {\footnotesize\(P \approx Q\)}; 
    \draw (1, 0) coordinate (a) (0,0) coordinate (o)
      pic["\footnotesize$\theta$", draw=orange, ->, 
          angle eccentricity=1.2, angle radius=10mm]
      {angle=a--o--p};
    \microscopio{(p)}{1}{55}{-120}{2}{(+4.1, +6.1)}
                {\footnotesize \(\times \infty\)}
    \tkzFct[domain=5.07:6.78, ultra thick, green!50!black] {\fam}
    \filldraw [green!50!black]
      \pm circle (1.5pt) 
          node [right, black] {\footnotesize\(P\)}
          node [right, black, yshift=15] {\footnotesize\(d y\)}
      \qm circle (1.5pt) 
          node [above, black] {\footnotesize\(Q\)}
          node [above, black, xshift=12] {\footnotesize\(d x\)}
      (5.4, 4.4) node [black] {\footnotesize\(\delta\)};
    \draw \pm |- \qm \pm -- \ppm \qm -- \qqm;
    \end{scope}
  }
}

\newcommand{\continuitafesp}{%
  \def \fa{1.6**x}
  \def \fam{.47*x+4.1}
  \def \mix{-4}
  \def \max{+4}
  \def \miy{0}
  \def \may{7}
  \def \xp{0}
  \def \yp{1}
  \disegno{
    \rcom{\mix}{\max}{\miy}{\may}{gray!50, very thin, step=1}
    \tkzInit[xmin=\mix-0.3,xmax=\max+0.3,ymin=\miy-0.3,ymax=\may+0.3]
    \tkzFct[domain=\mix-0.3:\max+0.3, ultra thick, green!50!black] {\fa}
    \microscopio{(\xp, \yp)}{1}{120}{-50}{2}{(\xp-3.8, \yp+4.1)}
                {\footnotesize \(\times \infty\)}
    \tkzFct[domain=\xp-3.55:\xp+0.08, ultra thick, green!50!black] {\fam}
  }
}

\newcommand{\continuitaintervallo}{%
  \def \fa{+.25*\x*\x-2*\x+4}
  \def \fb{-.25*\x*\x+2*\x}
  \def \fma{+0.75*x+5}
  \def \fmp{-0.50*x+9}
  \def \fmb{-1.25*x+15}
  \def \mix{0}
  \def \max{10}
  \def \miy{0}
  \def \may{7}
  \def \xa{2.5}
  \def \yaa{0.562}
  \def \yab{3.438}
  \def \xb{6.5}
  \def \yba{1.562}
  \def \ybb{2.438}
  \def \xp {5}
  \def \yp {3.75}
  \disegno{
    \rcom{\mix}{\max}{\miy}{\may}{gray!50, very thin, step=1}
    \tkzInit[xmin=\mix-0.3,xmax=\max+0.3,ymin=\miy-0.3,ymax=\may+0.8]
    \begin{scope} [ultra thick, green!50!black]
      \tkzFct[domain=\mix-0.3:\xa] {\fa}
      \tkzFct[domain=\xa:\xb] {\fb}
      \tkzFct[domain=\xb:\max+0.3] {\fa}
      \foreach \xp/\yp in {\xa/\yaa, \xb/\yba}{
        \filldraw [thick, fill=white] (\xp, \yp) circle(1.5pt);}
      \foreach \xp/\yp/\lab in {\xa/\yab/a, \xb/\ybb/b}{
        \draw [thick, dotted, black] (\xp, 0) -- (\xp, \may+0.3);
        \node [above, black] at (\xp, -1) {\(\lab\)};
        \filldraw [thick] (\xp, \yp) circle(1.5pt);
      }
    \end{scope}
    \microscopio{(\xa, \yab)}{1}{120}{-50}{2}{(\xa-2.4, \yab+4.6)}
                {\footnotesize \(\times \infty\)}
    \tkzFct[domain=\xa-1.7:\xa-0.1, ultra thick, green!50!black] {\fma}
    \microscopio{(\xp, \yp)}{1}{60}{-80}{2}{(\xp+2.4, \yp+4.1)}
                {\footnotesize \(\times \infty\)}
    \tkzFct[domain=\xp-1.75:\xp+1.83, ultra thick, green!50!black] {\fmp}
    \microscopio{(\xb, \ybb)}{1}{30}{-130}{2}{(\xb+3.3, \ybb+4.1)}
                {\footnotesize \(\times \infty\)}
    \tkzFct[domain=\xb+0.73:\xb+2.0, ultra thick, green!50!black] {\fmb}
      \foreach \xp/\yp in {.8/5.6, 8.5/4.375}{
        \filldraw [thick, green!50!black] (\xp, \yp) circle(1.5pt);}
  }
}

\newcommand{\continuitaintervalli}{%
  \def \fa{.5*\x+1.5}
  \def \fb{-.5*\x*\x+4*\x-2}
  \def \fc{1./(\x-\xc)}
  \def \mix{0}
  \def \max{10}
  \def \miy{0}
  \def \may{7}
  \def \xa{0.3}
  \def \xb{2.8}
  \def \xc{5.4}
  \def \xd{8.8}
  \disegno{
    \rcom{\mix}{\max}{\miy}{\may}{gray!50, very thin, step=1}
    \tkzInit[xmin=\mix-0.3,xmax=\max+0.3,ymin=\miy-0.3,ymax=\may+0.3]
    \begin{scope} [ultra thick, green!50!black]
      \tkzFct[domain=\mix-0.3:\xb] {\fa}
      \tkzFct[domain=\xb:\xc] {\fb}
      \tkzFct[domain=\xc+.1:\max+0.3] {\fc}
    \filldraw [thick, fill=white] (\xb, 2.9) circle(1.5pt);
    \foreach \xp/\yp/\lab in {\xa/1.65/a, \xb/5.28/b, 
                              \xc/5.02/c, \xd/.294/d}{
      \draw [thick, dotted, black] (\xp, 0) -- (\xp, \may+0.3);
      \node [above, black] at (\xp, -1) {\(\lab\)};
      \filldraw [thick] (\xp, \yp) circle(1.5pt);
    }
    \end{scope}
  }
}

\newcommand{\continuitaintervalloese}{%
  \def \fa{(x-1)/(x-5)}
  \def \mix{-2}
  \def \max{+10}
  \def \miy{-3}
  \def \may{+7}
  \def \xa{-1.5}
  \def \xb{+2.5}
  \def \xc{+3.5}
  \def \xd{+7.5}
  \disegno{
    \rcom{\mix}{\max}{\miy}{\may}{gray!50, very thin, step=1}
    \tkzInit[xmin=\mix-0.3,xmax=\max+0.3,ymin=\miy-0.3,ymax=\may+0.3]
    \begin{scope} [ultra thick, green!50!black]
      \tkzFct[domain=\mix-0.3:5] {\fa}
      \tkzFct[domain=5:\max+0.3] {\fa}
    \end{scope}
%     \filldraw [thick, fill=white] (\xb, 2.9) circle(1.5pt);
    \foreach \xp/\yp/\lab in {\xa/1.65/a, \xb/5.28/b, 
                              \xc/5.02/c, \xd/.294/d}{
      \draw [thick, dotted, black] (\xp, \miy-0.3) -- (\xp, \may+0.3);
      \node [above, black] at (\xp, -1) {\(\lab\)};
%       \filldraw [thick] (\xp, \yp) circle(1.5pt);
    }
  }
}

\newcommand{\limitigraficoa}{% 
  \def \funzione{(x**2-6*x+5)/(x**2+2*x-3)}
  \disegno{
    \rcom{-18}{+15}{-10}{+10}{gray!50, very thin, step=1}
    \tkzInit[xmin=-18.3,xmax=+15.3,ymin=-10.3,ymax=+10.3]
    \tkzFct[domain=-18.3:-3.1, ultra thick, color=orange!50!black]
         {\funzione}
    \tkzFct[domain=-2.9:+15.3, ultra thick, color=orange!50!black]
         {\funzione}
  }
}

\newcommand{\limitigraficob}{% 
  \def \funzione{(x**2-6*x+5)/(x**2+2*x-3)}
  \disegno{
    \rcom{-3}{+3}{-3}{+3}{gray!50, very thin, step=1}
    \tkzInit[xmin=-3.3,xmax=+3.3,ymin=-3.3,ymax=+3.3]
    \tkzFct[domain=-2.9:+0.91, ultra thick, color=orange!50!black]
         {\funzione}
    \tkzFct[domain=+1.1:+3.3, ultra thick, color=orange!50!black]
         {\funzione}
    \draw [orange!50!black] (1, -1) circle (2pt);
  }
}

\newcommand{\limitefinitodef}{%
  \def \fa{+.25*\x*\x-0.25*\x+0}
  \def \fb{+.25*\x*\x-2*\x+6}
  \def \fma{+0.75*x-3}
  \def \fmba{+2.*x-2.5}
  \def \fmbb{+0.25*x-1.5}
  \def \mix{-1}
  \def \max{9}
  \def \miy{-2}
  \def \may{7}
  \def \xa{2.}
  \def \ya{0.5}
  \def \xb{4.5}
  \def \yba{3.938}
  \def \ybb{2.062}
  \disegno{
    \rcom{\mix}{\max}{\miy}{\may}{gray!50, very thin, step=1}
    \tkzInit[xmin=\mix-0.3,xmax=\max+0.3,ymin=\miy-1.8,ymax=\may+1.3]
    \begin{scope} [ultra thick, orange!50!black]
      \tkzFct[domain=\mix-0.3:\xb] {\fa}
      \tkzFct[domain=\xb:\max+0.3] {\fb}
      \foreach \xp/\yp in {\xa/\ya, \xb/\yba, \xb/\ybb}{
        \filldraw [thick, fill=white] (\xp, \yp) circle(1.5pt);}
      \foreach \xp/\yp/\labx/\laby in {\xa/\ya/a/L, 
                                       \xb/\yba/b/N}{
        \draw [thick, dotted, black] (0, \yp) 
          node [left] {\(\laby\)} -- (\xp, \yp) -- (\xp, 0)
          node [below] {\(\labx\)};}
      \draw [thick, dotted, black] (0, \ybb) 
        node [left] {\(M\)} -- (\xb, \ybb);
    \end{scope}
    \microscopio{(\xa, \ya)}{1}{-130}{75}{2}{(\xa-3.0, \ya-4.3)}
                {\footnotesize \(\times \infty\)}
    \tkzFct[domain=\xa-2.67:\xa+0.53, ultra thick, orange!50!black] {\fma}
    \microscopio{(\xb, \yba)}{.7}{80}{-85}{2}{(\xb+2.0, \yba+4.2)}
                {\footnotesize \(\times \infty\)}
    \tkzFct[domain=\xb-0.85:\xb, ultra thick, orange!50!black] {\fmba}
    \microscopio{(\xb, \ybb)}{1}{-50}{145}{2}{(\xb+4., \ybb-3.8)}
                {\footnotesize \(\times \infty\)}
    \tkzFct[domain=\xb+2:\xb+4.2, ultra thick, orange!50!black] {\fmbb}
      \foreach \xp/\yp/\lab in {.8/-2.4/{\tonda{a;~L}}, 
                                \xb/6.5/{\tonda{b;~N}}, 
                                \xb+2/+.125/{\tonda{b;~M}}}{
        \filldraw [thick, orange!50!black, fill=white] 
          (\xp, \yp) circle(1.5pt) node [above] {\(\lab\)};}
  }
}

\newcommand{\limitenea}{% 
  \def \fa{0.5*x - 0.5}
  \def \fam{0.5*x + 2.5}
  \def \mix{-2}
  \def \max{+8}
  \def \miy{-2}
  \def \may{+5}
  \def \xp{4}
  \def \yp{1.5}
  \def \xpm{2}
  \def \ypm{3.5}
  \disegno{
    \rcom{\mix}{\max}{\miy}{\may}{gray!50, very thin, step=1}
    \tkzInit[xmin=\mix-0.3,xmax=\max+0.3,ymin=\miy-0.3,ymax=\may+0.3]
    \begin{scope}[ultra thick, orange!50!black] 
      \tkzFct[domain=\mix-0.3:\max+0.3] {\fa}
      \filldraw [opacity=.3] (\xp, \yp) circle (1.5pt);
    \end{scope}
    \microscopio{(\xp, \yp)}{1}{110}{-40}{2}{(\xp-0.2, \yp+4.0)}
                {\footnotesize \(\times \infty\)}
    \begin{scope}[ultra thick, orange!50!black] 
      \tkzFct[domain=\xp-3.55:\xp-3] {\fam}
      \tkzFct[domain=\xp-2.5:\xp-2.3] {\fam}
      \tkzFct[domain=\xp-1.5:\xp-0.05] {\fam}
    \end{scope}
      \filldraw [orange!50!black](\xpm, \ypm) circle (1.5pt);
  }
}

\newcommand{\limiteneb}{% 
  \def \fa{sqrt(-10./x**2 + 1)}
  \def \fama{-3*x - 11}
  \def \famb{+3*x - 11}
  \def \mix{-5}
  \def \max{+5}
  \def \miy{-5}
  \def \may{+2}
  \def \xca{-3.15}
  \def \xcb{+3.15}
  \disegno{
    \rcom{\mix}{\max}{\miy}{\may}{gray!50, very thin, step=1}
    \tkzInit[xmin=\mix-0.3,xmax=\max+0.3,ymin=\miy-0.3,ymax=\may+0.3]
    \begin{scope} [orange!50!black]
      \tkzFct[domain=\mix-0.3:\max+0.3, ultra thick, samples=1000] {\fa}
      \filldraw (\xca, 0) circle (1.5pt) node [above right=-.3] {\(c_1\)};
      \filldraw (\xcb, 0) circle (1.5pt) node [above left=-.3] {\(c_2\)};
    \end{scope}
    \microscopio{(\xca, 0)}{1}{-100}{+110}{2}{(\xca-1.2, 0-4.7)}
                {\footnotesize \(\times \infty\)}
    \microscopio{(\xcb, 0)}{.9}{-100}{+80}{2}{(\xcb+1.2, 0-4.7)}
                {\footnotesize \(\times \infty\)}
    \begin{scope}[orange!50!black] 
      \tkzFct[domain=\xca-.18:\xca+.45, ultra thick] {\fama}
      \filldraw (\xca+.47, -3) circle (1.5pt);
      \tkzFct[domain=\xcb-.45:\xcb+.18, ultra thick] {\famb}
      \filldraw (\xcb-.47, -3) circle (1.5pt);
    \end{scope}
  }
}

\newcommand{\limitesqrt}{% 
  \def \fa{sqrt(3*x - 12)}
  \def \fama{-3*x - 11}
  \def \famb{+3*x - 11}
  \def \mix{-0}
  \def \max{+10}
  \def \miy{0}
  \def \may{+5}
  \def \xp{4}
  \def \yp{0}
  \disegno{
    \rcom{\mix}{\max}{\miy}{\may}{gray!50, very thin, step=1}
    \tkzInit[xmin=\mix-0.3,xmax=\max+0.3,ymin=\miy-0.3,ymax=\may+0.3]
    \begin{scope} [ultra thick, orange!50!black]
      \tkzFct[domain=\mix-0.3:\max+0.3, samples=1000] {\fa}
      \draw (\xp, \yp) -- (\xp+.015, \yp+.2);
%       \filldraw (\xcb, 0) circle (1.5pt) node [above left=-.3] {\(c_2\)};
    \end{scope}
%     \microscopio{(\xca, 0)}{1}{-100}{+110}{2}{(\xca-1.2, 0-4.7)}
%                 {\footnotesize \(\times \infty\)}
%     \microscopio{(\xcb, 0)}{.9}{-100}{+80}{2}{(\xcb+1.2, 0-4.7)}
%                 {\footnotesize \(\times \infty\)}
%     \begin{scope}[orange!50!black] 
%       \tkzFct[domain=\xca-.18:\xca+.45, ultra thick] {\fama}
%       \filldraw (\xca+.47, -3) circle (1.5pt);
%       \tkzFct[domain=\xcb-.45:\xcb+.18, ultra thick] {\famb}
%       \filldraw (\xcb-.47, -3) circle (1.5pt);
%     \end{scope}
  }
}

\newcommand{\limiteiesea}{% 
  \def \fa{(x+2)/((x**2 +8*x +16))}
%   \def \fama{-3*x - 11}
%   \def \famb{+3*x - 11}
  \def \mix{-8}
  \def \max{+2}
  \def \miy{-12}
  \def \may{+2}
  \def \xp{-4}
  \def \yp{0}
  \disegno{
    \rcom{\mix}{\max}{\miy}{\may}{gray!50, very thin, step=1}
    \tkzInit[xmin=\mix-0.3,xmax=\max+0.3,ymin=\miy-0.3,ymax=\may+0.3]
    \begin{scope} [ultra thick, orange!50!black]
      \tkzFct[domain=\mix-0.3:\xp-0.1] {\fa}
      \tkzFct[domain=\xp+0.1:\max+0.3] {\fa}
%       \draw (\xp, \yp) -- (\xp+.015, \yp+.2);
%       \filldraw (\xcb, 0) circle (1.5pt) node [above left=-.3] {\(c_2\)};
    \end{scope}
%     \microscopio{(\xca, 0)}{1}{-100}{+110}{2}{(\xca-1.2, 0-4.7)}
%                 {\footnotesize \(\times \infty\)}
%     \microscopio{(\xcb, 0)}{.9}{-100}{+80}{2}{(\xcb+1.2, 0-4.7)}
%                 {\footnotesize \(\times \infty\)}
%     \begin{scope}[orange!50!black] 
%       \tkzFct[domain=\xca-.18:\xca+.45, ultra thick] {\fama}
%       \filldraw (\xca+.47, -3) circle (1.5pt);
%       \tkzFct[domain=\xcb-.45:\xcb+.18, ultra thick] {\famb}
%       \filldraw (\xcb-.47, -3) circle (1.5pt);
%     \end{scope}
  }
}

\newcommand{\limitesinunosux}{% 
  \def \fa{sin(5./x)}
%   \def \fa{sqrt(sin(5./x))}
  \def \mix{-6}
  \def \max{+6}
  \def \miy{-1}
  \def \may{+1}
  \def \da{0.4}
  \def \db{0.03}
  \disegno[10]{
    \rcom{\mix}{\max}{\miy}{\may}{gray!50, very thin, step=1}
    \tkzInit[xmin=\mix-0.3,xmax=\max+0.3,ymin=\miy-0.3,ymax=\may+0.3]
    \begin{scope} [thin, orange!50!black]
      \tkzFct[domain=\mix-0.3:-\da, samples=100, smooth] {\fa}
      \tkzFct[domain=-\da:-\db, samples=1000, smooth] {\fa}
      \tkzFct[domain=+\db:+\da, samples=1000, smooth] {\fa}
      \tkzFct[domain=+\da:\max+0.3, samples=100, smooth] {\fa}
      \filldraw [ultra thick] (-\db, -0.99) rectangle (+\db, +0.99);
    \end{scope}
    \node at (-3.5, +0.5) {\(y = \sen \dfrac{5}{x}\)};
  }
}

\newcommand{\limiteallinfinito}{% 
  \def \fa{(3*x**2-4*x+5)/(x**2-2*x+3)}
%   \def \fa{sqrt(sin(5./x))}
  \def \mix{-10}
  \def \max{+10}
  \def \miy{-3}
  \def \may{+5}
  \def \xta{-3}
  \def \yta{3}
  \def \xtb{3}
  \def \ytb{2}
  \def \xmica{-3}
  \def \ymica{.3}
  \def \xmicb{+3}
  \def \ymicb{-.5}
  \disegno{
    \rcom{\mix}{\max}{\miy}{\may}{gray!50, very thin, step=1}
    \tkzInit[xmin=\mix-0.3,xmax=\max+0.3,ymin=\miy-0.3,ymax=\may+0.3]
      \tkzFct[domain=\mix-0.3:\max+0.3, ultra thick, orange!50!black] {\fa}
    \draw (\xta, \yta) pic [rotate=180, xscale=.5, yscale=.5] 
      {telescopio=\(M\)};
    \microscopio{(\xta, \yta)}{1}{-90}{90}{2}{(0, 0)}{}
    \draw [ultra thick, orange!50!black] (\xta-1.97, \ymica) 
      node [black, left] {\(3\)} -- (\xta+1.97, \ymica);
    \microscopio{(\xmica, \ymica)}{1}{-120}{60}{2}{(\xmica+0.1, \ymica-4.5)}
      {\footnotesize \(\times \infty\)}
    \draw [dashed, black] (\xmica-3.47, \ymica-2.4) 
      node [black, left] {\(3\)} -- (\xmica+0.48, \ymica-2.4);
    \draw [ultra thick, orange!50!black] 
      (\xmica-3.47, \ymica-2.8) -- (\xmica+0.48, \ymica-2.8);
    \draw (\xtb, \ytb) pic [rotate=0, , xscale=.5, yscale=-.5] 
      {telescopio=\(M\)};
    \microscopio{(\xtb, \ytb)}{0.5}{-90}{90}{2}{(0, 0)}{}
    \draw [ultra thick, orange!50!black] (+1, \ymicb) -- (+5, \ymicb) 
      node [black, right] {\(3\)};
    \microscopio{(\xmicb, \ymicb)}{1}{-60}{120}{2}{(\xmicb+0.0, \ymicb-4.5)}
      {\footnotesize \(\times \infty\)}
    \draw [ultra thick, orange!50!black] 
      (\xmicb-0.48, \ymicb-2.4) -- (\xmicb+3.48, \ymicb-2.4);
    \draw [dashed, black] (\xmicb-0.48, \ymicb-2.8) 
      node [black, left] {\(3\)} -- (\xmicb+3.48, \ymicb-2.8);
  }
}

\begin{comment}

\newcommand{\telenonstandard}{% Telescopio per visualizzare A.
    \disegno[4]{
    \assecontrattini{-1}{+6}{0}{\(x\)}
    \draw (0, 0) [below] node{0} (1, 0) [below] node{1};
    \draw (-1, 1) pic [rotate=0, scale=.5] {telescopio=\(A\)};
    \microscopio{(-1, 1)}{2.5}{40}{-130}{3.5}{(7.5, 9.5)}{}
    \segmentocontrattini{0}{+6.2}{6}{1}
    \foreach \p/\l in {0/\dots, 1/A - 2, 2/A - 1, 
                       3/A, 4/A + 1, 5/A + 2, 6/\dots}{
      \draw (\p, 6) [left] node [rotate=90] {\footnotesize \(\l\)};}
%     \draw (0, 6) [left] node [rotate=90] {\dots} 
%           (1, 6) [left] node [rotate=90] {\(A - 1\)} 
%           (2, 6) [left] node [rotate=90] {\(A ~~\quad\)}
%           (3, 6) [left] node [rotate=90] {\(A + 1\)}
%           (5, 6) [left] node [rotate=90] {\(A + 3\)}
%           (6, 6) [left] node [rotate=90] {\dots};
    }
}

\newcommand{\limitesqrtsinunosux}{% 
  \def \fa{sqrt(sin(1./x))}
%   \def \fa{sqrt(sin(5./x))}
  \def \mix{-4}
  \def \max{+4}
  \def \miy{-1}
  \def \may{+1}
  \def \da{0.4}
  \def \db{0.03}
  \disegno[10]{
    \rcom{\mix}{\max}{\miy}{\may}{gray!50, very thin, step=1}
    \tkzInit[xmin=\mix-0.3,xmax=\max+0.3,ymin=\miy-0.3,ymax=\may+0.3]
    \begin{scope} [thin, orange!50!black]
      \tkzFct[domain=-1.7:-\da, samples=100, smooth] {\fa}
      \tkzFct[domain=-\da:-\db, samples=1000, smooth] {\fa}
      \tkzFct[domain=+\db:+\da, samples=1000, smooth] {\fa}
      \tkzFct[domain=+\da:\max+0.3, samples=100, smooth] {\fa}
      \filldraw [ultra thick] (-\db, +0.01) rectangle (+\db, +0.99);
%       \draw [ultra thick] (-0.018, 1) -- (-0.018, -1)
%                           (+0.018, 1) -- (+0.018, -1);
    \end{scope}
  }
}

\newcommand{\limitequadsqrtsinunosux}{% 
  \def \fa{(sqrt(sin(1./x)))**2} 
  % Qualche ottimizzazione del compilatore fa sì che non venga calcolata
  % correttamente: non viene calcolata prima la radice e poi il quadrato
  % ma (sqrt(sin(5./x)))**2 viene calcolata come: sin(5./x)
  \def \mix{-7}
  \def \max{+7}
  \def \miy{-1}
  \def \may{+1}
  \def \da{0.4}
  \def \db{0.03}
  \disegno[10]{
    \rcom{\mix}{\max}{\miy}{\may}{gray!50, very thin, step=1}
    \tkzInit[xmin=\mix-0.3,xmax=\max+0.3,ymin=\miy-0.3,ymax=\may+0.3]
    \begin{scope} [thin, orange!50!black]
      \tkzFct[domain=-1.7:-\da, samples=100, smooth] {\fa}
      \tkzFct[domain=-\da:-\db, samples=1000, smooth] {\fa}
      \tkzFct[domain=+\db:+\da, samples=1000, smooth] {\fa}
      \tkzFct[domain=+\da:\max+0.3, samples=100, smooth] {\fa}
      \filldraw [ultra thick] (-\db, +0.01) rectangle (+\db, +0.99);
%       \draw [ultra thick] (-0.018, 1) -- (-0.018, -1)
%                           (+0.018, 1) -- (+0.018, -1);
    \end{scope}
  }
}

\end{comment}

\newcommand{\limmicx}[8]{% 
  % interno del microscopio posto sull'asse x.
  \def \xa{#1} \def \xb{#2} \def \xc{#3} \def \xd{#4}
  \def \ya{#5} \def \yb{#6} \def \yc{#7}
  \def \lab{#8}
  \draw (\xa, \ya) -- (\xd, \ya);
  \draw (\xa, \ya) -- (\xd, \ya);
  \draw [Green!50!black, ultra thick] (\xc, \yb) -- (\xb, \yb);
  \draw [brown!50!black, ultra thick] (\xb, \ya) node [below] {\lab} -- 
                                       (\xb, \yb);
  \draw (\xb, \yb) -- (\xb, \yc);
}

\newcommand{\limiteseno}{% 
\disegno[20]{
  \rcom{-1.0}{+1.0}{-1.0}{+1.0}{gray!50, very thin, step=1}
  \draw [brown!50!black, ultra thick] (0, 0) circle (1);
  \draw [Green!50!black, ultra thick] (0, 0) -- (1.45, 0);
  \microscopio{(1, 0)}{.3}{40}{230}{.5}{(1.8, 1.2)}{\(\times \infty\)}
  \limmicx{1.14}{1.6}{1.07}{1.97}{.3}{.7}{1.08}{1}
  \draw (1.7, .5) node {$\delta$};
  \microscopio{(0, 0)}{.3}{120}{320}{.5}{(-.8, 1.2)}{\(\times  \infty\)}
  \limmicx{-0.12}{-0.5}{-0.05}{-0.95}{.3}{.7}{1.08}{0}
  \draw (-0.75, .5) node {$\sen \delta$};
  }
}

% La seguente non funzione, sballa il colore della griglia di rcom!!!!!
% \newcommand{\sincos}[3]{%
% \def \funzc{#1} 
% \def \funzl{#2} 
% \def \color{#3} 
% \disegno[7]{
%   \rcom{-6.5}{+6.5}{-1.0}{+1.0}{gray!50, very thin, step=1}
%     \tkzInit[xmin=-6.5,xmax=+6.5,ymin=-1.3,ymax=+1.3]
%     \tkzFct[domain=-6.5:+6.5, ultra thick, #3] 
%            {\funzc}
%     \node at (0, -1.5) {\funzl};
%   }
% }

\newcommand{\sinusoide}{%
\disegno[5]{
  \rcom{-6.5}{+6.5}{-1.0}{+1.0}{gray!50, very thin, step=1}
    \tkzInit[xmin=-6.8,xmax=+6.8,ymin=-1.3,ymax=+1.3]
    \tkzFct[domain=-6.8:+6.8, ultra thick, color=Blue!50!black]
         {sin(x)}
    \node at (0, -1.8) {$y=\sen x$};
  }
}

\newcommand{\cosinusoide}{%
\disegno[5]{
  \rcom{-6.5}{+6.5}{-1.0}{+1.0}{gray!50, very thin, step=1}
    \tkzInit[xmin=-6.8,xmax=+6.8,ymin=-1.3,ymax=+1.3]
    \tkzFct[domain=-6.8:+6.8, ultra thick, color=Red!50!black]
         {cos(x)}
    \node at (0, -1.8) {$y=\cos x$};
  }
}

\newcommand{\tangentoide}{%
\disegno[5]{
  \rcom{-3.5}{+3.5}{-4.0}{+4.0}{gray!50, very thin, step=1}
    \tkzInit[xmin=-3.8,xmax=+3.8,ymin=-4.3,ymax=+4.3]
    \tkzFct[domain=-3.8:-1.8, ultra thick, color=Green!50!black]
         {tan(x)}
    \tkzFct[domain=-1.4:+1.4, ultra thick, color=Green!50!black]
         {tan(x)}
    \tkzFct[domain=+1.6:+3.8, ultra thick, color=Green!50!black]
         {tan(x)}
    \node at (0, -4.8) {$y=\tan x$};
  }
}

\newcommand{\continuitagraficoese}{%7
  \disegno{
    \rcom{-12}{+12}{-7}{+7}{gray!50, very thin, step=1}
    \tkzInit[xmin=-12.3,xmax=+12.3,ymin=-7.3,ymax=+7.3]
    \tkzFct[domain=-12.3:-2.1, ultra thick, color=brown!50!black]
         {x/(x + 2)-2}
    \tkzFct[domain=-1.9:+0.9, ultra thick, color=brown!50!black]
         {x/(x + 2)-2}
    \tkzFct[domain=1:+3.95, ultra thick, color=brown!50!black]
         {x**2-6*x+7}
    \tkzFct[domain=+4.05:5, ultra thick, color=brown!50!black]
         {x**2-6*x+7}
    \tkzFct[domain=+5:12.3, ultra thick, color=brown!50!black]
         {1/(x-4)+1}
    \filldraw [brown!50!black] (1, 2) circle (2pt);
    \draw [brown!50!black] (1, -1.7) circle (2pt);
    \draw [brown!50!black] (4, -1) circle (2pt);
  }
}
