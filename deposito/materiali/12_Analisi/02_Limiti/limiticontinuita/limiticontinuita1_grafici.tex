% (c) 2016 Daniele Zambelli - daniele.zambelli@gmail.com

\newcommand{\gnomino}[7]{
  % example:
  % \disegno{\gnomino{0.7}{1.5}{f(a)}{a}{gray!50}{blue!50!black}{.69}}
  \def \x{#1}
  \def \y{#2}
  \def \lx{#3}
  \def \ly{#4}
  \def \colorline{#5}
  \def \colorpoint{#6}
  \def \lbelow{#7}
  \draw [thin, dashed, \colorline] 
        (0, \y) node [left] {\(\ly\)} -- (\x, \y) --
        (\x, 0) node [below=\lbelow] {\(\lx\)};
  \filldraw (\x, \y) [\colorpoint] circle(1.5pt);
}

\newcommand{\xbar}[7]{
  % example:
  % \disegno{\xbar{0.7}{1.5}{a_1}{gray!50}{blue!50!black}{red}{0}}
  \def \x{#1}
  \def \y{#2}
  \def \lx{#3}
  \def \colorline{#4}
  \def \colorpoint{#5}
  \def \colorlabel{#6}
  \def \lbelow{#7}
  \draw [dotted, \colorline]
        (\x, 0) node [\colorlabel, below=\lbelow] {\(\lx\)} -- (\x, \y);
  \filldraw (\x, \y) [\colorpoint] circle(1.5pt);
}

% \newcommand{\ybar}[4]{
%   % example:
%   % \disegno{\ybar{0.7}{1.5}{a_1}{gray!50}}
%   \def \x{#1}
%   \def \y{#2}
%   \def \ly{#3}
%   \def \colorline{#4}
%   \draw [dotted, \colorline]
%         (0, \y) node [left] {\(\ly\)} -- (\x, \y);
% }

\def \graficobasea{
    \def \xi{-0.3};
    \def \yi{4.7};
    \def \xf{10.};
    \def \yf{-0.3};
    \coordinate (i) at (\xi, \yi);
    \coordinate (f) at (\xf, \yf);
    \coordinate (ctrli) at (3, -6);
    \coordinate (ctrlf) at (7, 15);
    \def \linea{(i) .. controls (ctrli) and (ctrlf) .. (f)}
    \rcom{0}{+10}{0}{+8}{gray!50, very thin, step=1}
%       \draw [dashed, ultra thick, white!70!black] \linea;
    \begin{scope}
      \clip (\xi, -0.3) rectangle (\xf, 8);
      \draw [dashed, ultra thick, white!70!black] \linea;
    \end{scope}
}

\newcommand{\partizionen}{% Valutazione di f in alcuni punti con n fissato
  \disegno{
    \def \xb{8.4};
    \def \yb{4.25};
    \graficobasea
    \foreach \xp/\yp/\lab in {0.4/2.9/x_0, 1.4/1.67/x_1, 2.4/1.59/x_2, 
                              3.4/2.25/x_3, 4.4/3.27/x_4, 5.4/4.4/x_5, 
                              6.4/5.25/x_6, 7.4/5.5/x_7, 8.4/4.67/x_8,
                              9.4/2.25/x_9}{
      \xbar{\xp}{\yp}{\lab}
            {white!30!black}{blue!50!black}{blue!50!black}{0}}
    \node at (5, -1.3) {\footnotesize{Funzione valutata in alcuni punti}};
  }
}

\newcommand{\xbarx}[6]{
  % example:
  % \disegno{\xbarx{0.7}{1.5}{a}{gray!50}{blue!50!black}{red}}
  \def \x{#1}
  \def \funct{#2}
  \def \lx{#3}
  \def \colorline{#4}
  \def \colore{#5}
  \def \colorfill{#6}
  \draw [dotted, \colorline]
        (\x, 0) node [\colore, below] {\(\lx\)} -- (\x, \funct);
  \filldraw (\x, \funct) [\colore, fill=\colorfill] circle(1.5pt);
}

\newcommand{\puntia}{%
  \def \fa{.5*\x+1.5}
  \def \fb{-.5*\x*\x+4*\x-2}
  \def \fc{1./(\x-\xc)}
  \def \mi{0}
  \def \ma{10}
  \def \xa{0.3}
  \def \xb{2.8}
  \def \xc{5.4}
  \def \xd{9.8}
  \rcom{\mi}{\ma}{0}{+8}{gray!50, very thin, step=1}
  \foreach \yp [count = \xp] in {2, 2.5, 
                                  5.5, 6, 5.5, 
                                  1.667, 0.625, .385, .278}{
    \xbarx{\xp}{\yp}{x_\xp}{gray!90}{brown!50!black}{brown!50!black}}
}

\newcommand{\puntigraficodiscontinuo}{%
  \disegno{
    \puntia
    \foreach \xp/\lab in {\xa/a, \xd/b}{
      \draw [dotted, gray!50]
        (\xp, 0) node [black, below] {\(\lab\)} -- (\xp, 8.3);}
    \node at (5, -1.3) 
      {\footnotesize \emph{Funzione poco regolare valutata in alcuni punti}};
  }
}

\newcommand{\graficodiscontinuo}{%
  \disegno{
    \puntia
    \tkzInit[xmin=\mi-0.3,xmax=\ma+0.3,ymin=-0.3,ymax=+8.3]
    \begin{scope} [thick, brown!50!black]
      \tkzFct[domain=\mi-0.3:\xb] {\fa}
      \tkzFct[domain=\xb:\xc] {\fb}
      \tkzFct[domain=\xc+.1:\ma+0.3] {\fc}
    \filldraw (\xb, 2.9) [fill=white] circle(1.5pt);
    \filldraw (\xb, 5.28) circle(1.5pt);
    \filldraw (\xc, 5.02) circle(1.5pt);
    \end{scope}
    \xbarx{\xa}{\fa}{a}{gray!90}{brown!50!black}{brown!50!black}
    \xbarx{\xd}{.227}{b}{gray!90}{brown!50!black}{brown!50!black}
    \node at (5, -1.3) 
      {\footnotesize \emph{Grafico della funzione poco regolare}};
  }
}

\newcommand{\contprimo}{%
  \def \fa{-\x*\x+0*x+5}
  \def \fm{-2*\x+13}
  \def \mi{-2}
  \def \ma{6}
  \def \xc{1}
  \def \yc{4}
  \def \xdma{1.65}
  \def \ydma{3.7}
  \def \xdmb{2.9}
  \def \ydmb{1.22}
  \disegno{
    \rcom{\mi}{\ma}{0}{+8}{gray!50, very thin, step=1}
    \tkzInit[xmin=\mi-0.3,xmax=\ma+0.3,ymin=-0.3,ymax=+8.3]
    \tkzFct[domain=\mi-0.3:\ma+0.3, ultra thick, green!50!black] {\fa}
    \draw [thin, dashed, black] 
          (0, \yc) node [left] {\footnotesize \(\approx \quad ~\)} 
          node [above left] {\footnotesize \(f(\xc) \quad\)} 
          node [below left] {\footnotesize \(f(\xc + \epsilon)\)} -- 
          (\xc, \yc) -- (\xc, 0) 
          node [below] {\footnotesize \(\xc \approx \xc + \epsilon\)};
    \filldraw (\xc, \yc) [green!50!black] circle(1.5pt);
    \microscopio{(\xc, \yc)}{1}{40}{230}{2}{(\xc+4, \yc+3.8)}
                {\footnotesize \(\times \infty\)}
    \tkzFct[domain=\xc+1.46:\xc+3.22, ultra thick, green!50!black] {\fm}
    \draw [thin, dashed, black] (\xc + 0.8, \yc + \ydma) 
          node [left] {\footnotesize \(f(\xc)\)} -- 
          (\xc + \xdma, \yc + \ydma) -- (\xc + \xdma, \yc + 0.3) 
          node [below ] {\footnotesize \(\xc\)};
    \filldraw (\xc + \xdma, \yc + \ydma) [green!50!black] circle(1.5pt);
    \draw [thin, dashed, black] (\xc + 0.3, \yc + \ydmb) 
          node [left] {\footnotesize \(f(\xc + \epsilon)\)} -- 
          (\xc + \xdmb, \yc + \ydmb) -- (\xc + \xdmb, \yc + 0.3) 
          node [below] {\footnotesize \(\xc + \epsilon\)};
    \filldraw (\xc + \xdmb, \yc + \ydmb) [green!50!black] circle(1.5pt);
  }
}

\newcommand{\contsecondo}{%
  \def \fa{abs(x)/x}
  \def \mix{-2}
  \def \max{+2}
  \def \miy{-1.}
  \def \may{+1.}
  \disegno[10]{
    \rcom{\mix}{\max}{\miy}{\may}{gray!50, very thin, step=1}
    \tkzInit[xmin=\mix-0.3,xmax=\max+0.3,ymin=\miy-0.3,ymax=\may+0.3]
    \tkzFct[domain=\mix-0.3:0, ultra thick, green!50!black] {\fa}
    \tkzFct[domain=0:\max+0.3, ultra thick, green!50!black] {\fa}
    \filldraw (0, -1) [green!50!black, fill= white] circle(1.5pt);
    \filldraw (0, +1) [green!50!black, fill= white] circle(1.5pt);
  }
}

\newcommand{\fsegno}{%
  \def \fa{abs(x)/x}
  \def \mix{-2}
  \def \max{+2}
  \def \miy{-1.}
  \def \may{+1.}
  \disegno[10]{
    \rcom{\mix}{\max}{\miy}{\may}{gray!50, very thin, step=1}
    \tkzInit[xmin=\mix-0.3,xmax=\max+0.3,ymin=\miy-0.3,ymax=\may+0.3]
    \tkzFct[domain=\mix-0.3:0, ultra thick, green!50!black] {\fa}
    \tkzFct[domain=0:\max+0.3, ultra thick, green!50!black] {\fa}
    \filldraw (0, -1) [green!50!black, fill= white] circle(1.5pt);
    \filldraw (0, +1) [green!50!black, fill= white] circle(1.5pt);
    \filldraw (0, 0) [green!50!black] circle(1.5pt);
    \microscopio{(0, 0)}{.5}{20}{210}{1}{(+2, +1.7)}
                {\footnotesize \(\times \infty\)}
    \filldraw (1.4, 0.7) [green!50!black] circle(1.5pt)
      node [below, black] {\(0\)} node [left, black] {\(f(0)\)};
    \microscopio{(0, +1)}{.5}{+180}{20}{1}{(-2, +1.7)}
                {\footnotesize \(\times \infty\)}
    \draw [ultra thick, green!50!black] (-1.5, +0.7) 
      node [below, black] {\(0\)} -- (-0.44, +0.7); 
    \filldraw (-1.5, +0.7) [green!50!black, fill= white] circle(1.5pt);
    \microscopio{(0, -1)}{.5}{-30}{160}{1}{(+2.2, -2.5)}
                {\footnotesize \(\times \infty\)}
    \draw [ultra thick, green!50!black] (1.4, -1.7) 
      node [below, black] {\(0\)} -- (0.38, -1.7); 
    \filldraw (1.4, -1.7) [green!50!black, fill= white] circle(1.5pt);
  }
}

\newcommand{\contterzo}{%
  \def \fa{abs(x)/4}
  \def \fb{abs(x-1.2)/4+2.4}
  \def \mix{-4}
  \def \max{+4}
  \def \miy{-1}
  \def \may{+4}
  \def \xc{0}
  \def \yc{0}
  \def \xdma{1.65}
  \def \ydma{3.7}
  \def \xdmb{2.9}
  \def \ydmb{1.22}
  \disegno{
    \rcom{\mix}{\max}{\miy}{\may}{gray!50, very thin, step=1}
    \tkzInit[xmin=\mix-0.3,xmax=\max+0.3,ymin=\miy-0.3,ymax=\may+0.3]
    \tkzFct[domain=\mix-0.3:\max+0.3, ultra thick, green!50!black] {\fa}
    \microscopio{(\xc, \yc)}{1}{60}{250}{2}{(\xc+3, \yc+4.4)}
                {\footnotesize \(\times \infty\)}
    \tkzFct[domain=-0.8:+3.18, ultra thick, green!50!black] {\fb}
    \draw [thin, dashed, black] (1.2, 2.4) 
          node [below] {\(0\)} -- (-0.8, 2.4) 
          node [left] {\(f(0)\)};
  }
}

\newcommand{\continuitagraficoa}{%
  \def \xc{2}
  \def \yc{-1}
  \disegno{
    \rcom{-3}{+7}{-4}{+5}{gray!50, very thin, step=1}
    \tkzInit[xmin=-3.3,xmax=+7.3,ymin=-4.3,ymax=+5.3]
    \tkzFct[domain=-3.3:2, ultra thick, color=green!50!black]
         {.5*x - 2}
    \tkzFct[domain=2:+7.3, ultra thick, color=green!50!black]
         {x**2-6*x+7}
    \filldraw (\xc, \yc) [green!50!black] circle(1.5pt);
    \microscopio{(\xc, \yc)}{1}{100}{290}{2}{(+2.4, +3.8)}
                {\footnotesize \(\times \infty\)}
    \tkzFct[domain=-0.73:+1.04, ultra thick, color=green!50!black]
         {.5*x + 1.5}
    \tkzFct[domain=+1.:+1.97, ultra thick, color=green!50!black]
         {-2*x + 4}
    \draw [thin, dashed, black] (-0.8, 2) 
          node [left] {\(f(\xc)\)} -- 
          (1, 2) -- (1, - 0.1) 
          node [below ] {\(\xc\)};
    \filldraw (\xc - 1, \yc + 3) [green!50!black] circle(1.5pt);
  }
}

% \newcommand{\continuitagraficoa}{%
%   \def \xc{2}
%   \def \yc{-1}
%   \disegno{
%     \rcom{-5}{+7}{-7}{+7}{gray!50, very thin, step=1}
%     \tkzInit[xmin=-5.3,xmax=+7.3,ymin=-7.3,ymax=+7.3]
%     \tkzFct[domain=-5.3:2, ultra thick, color=green!50!black]
%          {.5*x - 2}
%     \tkzFct[domain=2:+7.3, ultra thick, color=green!50!black]
%          {x**2-6*x+7}
%     \filldraw (\xc, \yc) [green!50!black] circle(1.5pt);
%     \microscopio{(\xc, \yc)}{1}{100}{290}{2}{(+2.2, +3.8)}
%                 {\footnotesize \(\times \infty\)}
%     \tkzFct[domain=-0.73:+1.04, ultra thick, color=green!50!black]
%          {.5*x + 1.5}
%     \tkzFct[domain=+1.:+1.97, ultra thick, color=green!50!black]
%          {-2*x + 4}
%     \draw [thin, dashed, black] (-0.8, 2) 
%           node [left] {\footnotesize \(f(2)\)} -- 
%           (1, 2) -- (1, - 0.1) 
%           node [below ] {\footnotesize \(\xc\)};
%     \filldraw (\xc - 1, \yc + 3) [green!50!black] circle(1.5pt);
%   }
% }

\newcommand{\limitigraficoa}{% 
  \def \funzione{(x**2-6*x+5)/(x**2+2*x-3)}
  \disegno{
    \rcom{-18}{+15}{-10}{+10}{gray!50, very thin, step=1}
    \tkzInit[xmin=-18.3,xmax=+15.3,ymin=-10.3,ymax=+10.3]
    \tkzFct[domain=-18.3:-3.1, ultra thick, color=brown!50!black]
         {\funzione}
    \tkzFct[domain=-2.9:+15.3, ultra thick, color=brown!50!black]
         {\funzione}
  }
}

\newcommand{\limmicx}[8]{% 
  % interno del microscopio posto sull'asse x.
  \def \xa{#1} \def \xb{#2} \def \xc{#3} \def \xd{#4}
  \def \ya{#5} \def \yb{#6} \def \yc{#7}
  \def \lab{#8}
  \draw (\xa, \ya) -- (\xd, \ya);
  \draw (\xa, \ya) -- (\xd, \ya);
  \draw [Green!50!black, ultra thick] (\xc, \yb) -- (\xb, \yb);
  \draw [brown!50!black, ultra thick] (\xb, \ya) node [below] {\lab} -- 
                                       (\xb, \yb);
  \draw (\xb, \yb) -- (\xb, \yc);
}

\newcommand{\limiteseno}{% 
\disegno[20]{
  \rcom{-1.0}{+1.0}{-1.0}{+1.0}{gray!50, very thin, step=1}
  \draw [brown!50!black, ultra thick] (0, 0) circle (1);
  \draw [Green!50!black, ultra thick] (0, 0) -- (1.45, 0);
  \microscopio{(1, 0)}{.3}{40}{230}{.5}{(1.8, 1.2)}{\(\times \infty\)}
  \limmicx{1.14}{1.6}{1.07}{1.97}{.3}{.7}{1.08}{1}
  \draw (1.7, .5) node {$\delta$};
  \microscopio{(0, 0)}{.3}{120}{320}{.5}{(-.8, 1.2)}{\(\times  \infty\)}
  \limmicx{-0.12}{-0.5}{-0.05}{-0.95}{.3}{.7}{1.08}{0}
  \draw (-0.75, .5) node {$\sen \delta$};
  }
}

% La seguente non funzione, sballa il colore della griglia di rcom!!!!!
% \newcommand{\sincos}[3]{%
% \def \funzc{#1} 
% \def \funzl{#2} 
% \def \color{#3} 
% \disegno[7]{
%   \rcom{-6.5}{+6.5}{-1.0}{+1.0}{gray!50, very thin, step=1}
%     \tkzInit[xmin=-6.5,xmax=+6.5,ymin=-1.3,ymax=+1.3]
%     \tkzFct[domain=-6.5:+6.5, ultra thick, #3] 
%            {\funzc}
%     \node at (0, -1.5) {\funzl};
%   }
% }

\newcommand{\sinusoide}{%
\disegno[5]{
  \rcom{-6.5}{+6.5}{-1.0}{+1.0}{gray!50, very thin, step=1}
    \tkzInit[xmin=-6.8,xmax=+6.8,ymin=-1.3,ymax=+1.3]
    \tkzFct[domain=-6.8:+6.8, ultra thick, color=Blue!50!black]
         {sin(x)}
    \node at (0, -1.8) {$y=\sen x$};
  }
}

\newcommand{\cosinusoide}{%
\disegno[5]{
  \rcom{-6.5}{+6.5}{-1.0}{+1.0}{gray!50, very thin, step=1}
    \tkzInit[xmin=-6.8,xmax=+6.8,ymin=-1.3,ymax=+1.3]
    \tkzFct[domain=-6.8:+6.8, ultra thick, color=Red!50!black]
         {cos(x)}
    \node at (0, -1.8) {$y=\cos x$};
  }
}

\newcommand{\tangentoide}{%
\disegno[5]{
  \rcom{-3.5}{+3.5}{-4.0}{+4.0}{gray!50, very thin, step=1}
    \tkzInit[xmin=-3.8,xmax=+3.8,ymin=-4.3,ymax=+4.3]
    \tkzFct[domain=-3.8:-1.8, ultra thick, color=Green!50!black]
         {tan(x)}
    \tkzFct[domain=-1.4:+1.4, ultra thick, color=Green!50!black]
         {tan(x)}
    \tkzFct[domain=+1.6:+3.8, ultra thick, color=Green!50!black]
         {tan(x)}
    \node at (0, -4.8) {$y=\tan x$};
  }
}

\newcommand{\limitigraficob}{% 
  \def \funzione{(x**2-6*x+5)/(x**2+2*x-3)}
  \disegno{
    \rcom{-3}{+3}{-3}{+3}{gray!50, very thin, step=1}
    \tkzInit[xmin=-3.3,xmax=+3.3,ymin=-3.3,ymax=+3.3]
    \tkzFct[domain=-2.9:+0.91, thick, color=brown!50!black]
         {\funzione}
    \tkzFct[domain=+1.1:+3.3, thick, color=brown!50!black]
         {\funzione}
    \draw [brown!50!black] (1, -1) circle (2pt);
  }
}

\newcommand{\continuitagraficoese}{%7
  \disegno{
    \rcom{-12}{+12}{-7}{+7}{gray!50, very thin, step=1}
    \tkzInit[xmin=-12.3,xmax=+12.3,ymin=-7.3,ymax=+7.3]
    \tkzFct[domain=-12.3:-2.1, ultra thick, color=brown!50!black]
         {x/(x + 2)-2}
    \tkzFct[domain=-1.9:+0.9, ultra thick, color=brown!50!black]
         {x/(x + 2)-2}
    \tkzFct[domain=1:+3.95, ultra thick, color=brown!50!black]
         {x**2-6*x+7}
    \tkzFct[domain=+4.05:5, ultra thick, color=brown!50!black]
         {x**2-6*x+7}
    \tkzFct[domain=+5:12.3, ultra thick, color=brown!50!black]
         {1/(x-4)+1}
    \filldraw [brown!50!black] (1, 2) circle (2pt);
    \draw [brown!50!black] (1, -1.7) circle (2pt);
    \draw [brown!50!black] (4, -1) circle (2pt);
  }
}
