% (c) 2015 Daniele Zambelli daniele.zambelli@gmail.com

\section{Esercizi}

\subsection{Esercizi dei singoli paragrafi}

\subsubsection*{\numnameref{sec:cont_limiti}}

\begin{esercizio}\label{ese:03.1}
Calcola i seguenti limiti:
\begin{multicols}{2}
 \begin{enumeratea}
  \item \(\displaystyle \lim_{x \rightarrow 3} \tonda{x^2-4x+2}\)
  \item \(\displaystyle \lim_{x \rightarrow 1} \tonda{4x^2-5x+8}\)
  \item \(\displaystyle \lim_{x \rightarrow -1} \tonda{7x^2+3x+5}\)
  \item \(\displaystyle \lim_{x \rightarrow -2} \tonda{3x^2+2x+6}\)
  \item \(\displaystyle \lim_{x \rightarrow 0} \frac{x^2-4x+2}{x-1}\)
  \hfill \(-2\)
 \end{enumeratea}
 \end{multicols}
\end{esercizio}

\begin{esercizio}\label{ese:03.1}
Calcola i seguenti limiti:
\begin{multicols}{2}
 \begin{enumeratea}
  \item \(\displaystyle \lim_{x \rightarrow 2} \tonda{x^2-4}\)
  \hfill \(\quadra{0}\)
  \item \(\displaystyle \lim_{x \rightarrow 0} \frac{x^3-4x}{2x^2+3x}\)
  \hfill \(\quadra{-\frac{4}{3}}\)
  \item \(\displaystyle \lim_{x \rightarrow -1} \frac{x^3}{\tonda{x+1}^2}\)
  \hfill \(\quadra{\infty}\)
  \item \(\displaystyle \lim_{x \rightarrow -1} 
          \frac{\tonda{x+1}^2 \tonda{x-1}}{x^3+1}\)
  \hfill \(\quadra{0}\)
  \item \(\displaystyle \lim_{x \rightarrow 0} 
          \frac{x^3-2x^2+x}{2x^3+x^2-2x}\)
  \hfill \(\quadra{-\frac{1}{2}}\)
  \item \(\displaystyle \lim_{x \rightarrow 1} 
          \frac{x^2+2x+3}{\tonda{x-1}^2}\)
  \hfill \(\quadra{\infty}\)
  \item \(\displaystyle \lim_{x \rightarrow 0} 
          \frac{x^4-4x^3+x^2}{x^3+x^2+x}\)
  \hfill \(\quadra{0}\)
  \item \(\displaystyle \lim_{x \rightarrow -1} 
          \frac{x^3+x^2+x+1}{x^4+x^2-2}\)
  \hfill \(\quadra{0}\)
  \item \(\displaystyle \lim_{x \rightarrow 2} 
          \frac{\tonda{x+1}^2}{2-x}\)
  \hfill \(\quadra{\infty}\)
  \item \(\displaystyle \lim_{x \rightarrow 2} 
          \frac{x-2}{x^2-3x+2}\)
  \hfill \(\quadra{1}\)
  \item \(\displaystyle \lim_{x \rightarrow 0} 
          \frac{3x+2x^{-1}}{x+4x^{-1}}\)
  \hfill \(\quadra{\frac{1}{2}}\)
  \item \(\displaystyle \lim_{x \rightarrow 2} 
          \frac{x^2-3x+2}{x^2-2x}\)
  \hfill \(\quadra{\frac{1}{2}}\)
  \item \(\displaystyle \lim_{x \rightarrow 1} 
          \tonda{\frac{1}{1-x}-\frac{3}{1-x^3}}\)
  \hfill \(\quadra{-1}\)
  \item \(\displaystyle \lim_{x \rightarrow 2} 
          \frac{x^2-x-2}{x^2-2x}\)
  \hfill \(\quadra{\frac{3}{2}}\)
  \item \(\displaystyle \lim_{x \rightarrow 2} 
          \frac{x^2+5}{x^2-3}\)
  \hfill \(\quadra{9}\)
  \item \(\displaystyle \lim_{x \rightarrow 1} 
          \frac{3x^4-4x^3+1}{\tonda{x-1}^2}\)
  \hfill \(\quadra{6}\)
  \item \(\displaystyle \lim_{x \rightarrow -2} 
          \frac{3x+6}{x^3+8}\)
  \hfill \(\quadra{\frac{1}{4}}\)
  \item \(\displaystyle \lim_{x \rightarrow 2} 
          \frac{x+1}{x-1}\)
  \hfill \(\quadra{3}\)
  \item \(\displaystyle \lim_{x \rightarrow -2} 
          \frac{x^3+3x^2+2x}{x^2-x-6}\)
  \hfill \(\quadra{-\frac{2}{5}}\)
  \item \(\displaystyle \lim_{x \rightarrow 1} 
          \frac{x^2-2x+1}{x^3-x}\)
  \hfill \(\quadra{0}\)
  \item \(\displaystyle \lim_{x \rightarrow 4} 
          \frac{x^2+7x-44}{x^2-6x+8}\)
  \hfill \(\quadra{\frac{15}{2}}\)
  \item \(\displaystyle \lim_{x \rightarrow 1} 
          \frac{x^2-4}{x^2-3x+2}\)
  \hfill \(\quadra{\infty}\)
  \item \(\displaystyle \lim_{x \rightarrow 1} 
          \frac{x^3-5x+4}{x^3-1}\)
  \hfill \(\quadra{-\frac{2}{3}}\)
  \item \(\displaystyle \lim_{x \rightarrow 2} 
          \frac{x^2-4}{x-2}\)
  \hfill \(\quadra{4}\)
  \item \(\displaystyle \lim_{x \rightarrow 2} 
          \frac{x^2-4}{x^2-3x+2}\)
  \hfill \(\quadra{4}\)
  \item \(\displaystyle \lim_{x \rightarrow 1} 
          \tonda{\frac{1}{x^2-1}-\frac{2}{x^4-1}}\)
  \hfill \(\quadra{\frac{1}{2}}\)
 \end{enumeratea}
 \end{multicols}
\end{esercizio}

\begin{esercizio}\label{ese:03.1}
Calcola i seguenti limiti:
\begin{multicols}{2}
 \begin{enumeratea}
  \item \(\displaystyle \lim_{x \rightarrow \infty} 
          \frac{x^2-1}{2x^2+1}\)
  \hfill \(\quadra{\frac{1}{2}}\)
  \item \(\displaystyle \lim_{x \rightarrow -\infty} 
          \frac{x^3+x^2-4}{2x^3+x+11}\)
  \hfill \(\quadra{\frac{1}{2}}\)
  \item \(\displaystyle \lim_{x \rightarrow \infty} 
          \frac{3x^2+2x-1}{x^3-x+2}\)
  \hfill \(\quadra{0}\)
  \item \(\displaystyle \lim_{x \rightarrow \infty} 
          \tonda{\frac{x^3}{x^2+2}-x}\)
  \hfill \(\quadra{0}\)
  \item \(\displaystyle \lim_{x \rightarrow \infty} 
          \frac{x^2+3x-4}{3x^2-2x+5}\)
  \hfill \(\quadra{\frac{1}{3}}\)
  \item \(\displaystyle \lim_{x \rightarrow \infty} 
          \frac{x\tonda{x-1}\tonda{x-2}}{x^2+6x-9}\)
  \hfill \(\quadra{\infty}\)
  \item \(\displaystyle \lim_{x \rightarrow \infty} 
          \frac{\sqrt{x^2+9}}{x+3}\)
  \hfill \(\quadra{1}\)
  \item \(\displaystyle \lim_{x \rightarrow \infty} 
          \tonda{\frac{x^2+x-1}{2x^2-x+1}}^3\)
  \hfill \(\quadra{\frac{1}{8}}\)
  \item \(\displaystyle \lim_{x \rightarrow \infty} 
          \frac{x^2+2x+1}{5x}\)
  \hfill \(\quadra{\infty}\)
  \item \(\displaystyle \lim_{x \rightarrow -\infty} 
          \frac{x^3+x^4-1}{2x^5+x-x^2}\)
  \hfill \(\quadra{0}\)
  \item \(\displaystyle \lim_{x \rightarrow \infty} 
          \frac{\tonda{\sqrt{x^2+1}+x}^2}{\sqrt[3]{x^6+1}}\)
  \hfill \(\quadra{4}\)
  \item \(\displaystyle \lim_{x \rightarrow -\infty} 
          \frac{x^6+7x^4-40}{1-x-5x^7}\)
  \hfill \(\quadra{0}\)
  \item \(\displaystyle \lim_{x \rightarrow \infty} 
          \frac{\tonda{x+1}\tonda{x-2}}{3x^2+6x-5}\)
  \hfill \(\quadra{\frac{1}{3}}\)
  \item \(\displaystyle \lim_{x \rightarrow \infty} 
          \frac{\sqrt{x^2+1}}{x}\)
  \hfill \(\quadra{1}\)
  \item \(\displaystyle \lim_{x \rightarrow \infty} 
          \tonda{\frac{3x^2+2x+1}{x^2-3x+2}}^4\)
  \hfill \(\quadra{81}\)
  \item \(\displaystyle \lim_{x \rightarrow -\infty} 
          \frac{5x^3-x^2+x}{1-x-3x^2}\)
  \hfill \(\quadra{\infty}\)
  \item \(\displaystyle \lim_{x \rightarrow \infty} 
          \frac{1+x-3x^3}{1+x^2+3x^3}\)
  \hfill \(\quadra{-1}\)
  \item \(\displaystyle \lim_{x \rightarrow -\infty} 
          \tonda{\frac{x^3-8}{x^4+16}}^10\)
  \hfill \(\quadra{0}\)
  \item \(\displaystyle \lim_{x \rightarrow \infty} 
          \frac{\tonda{x+3}\tonda{x+4}\tonda{x+5}}{x^4+x-11}\)
  \hfill \(\quadra{0}\)
  \item \(\displaystyle \lim_{x \rightarrow -\infty} 
          \frac{8x-2x^5+x^6}{11x+5x^3+3x^5}\)
  \hfill \(\quadra{-\infty}\)
  \item \(\displaystyle \lim_{x \rightarrow \infty} 
          \tonda{\frac{x^3}{2x^2-1}-\frac{x^2}{2x+1}}\)
  \hfill \(\quadra{\frac{1}{4}}\)
  \item \(\displaystyle \lim_{x \rightarrow \infty} 
          \tonda{x^2-\frac{x^4-1}{x^2-2}}\)
  \hfill \(\quadra{-2}\)
  \item \(\displaystyle \lim_{x \rightarrow \infty} 
          \frac{\tonda{x-1}^{100}\tonda{6x+1}^{200}}{\tonda{3x+5}^{300}}\)
  \hfill \(\quadra{\tonda{\frac{4}{3}}^{100}}\)
  \item \(\displaystyle \lim_{x \rightarrow \infty} 
          \frac{\sqrt[4]{x^5}+\sqrt[5]{x^3}+\sqrt[6]{x^8}}
               {\sqrt[3]{x^4+2}}\)
  \hfill \(\quadra{1}\)
  \item \(\displaystyle \lim_{x \rightarrow -\infty} 
          \frac{x^2\tonda{2x+1}\tonda{3x-2}}
               {2x^2\tonda{5x-8}\tonda{x+6}}\)
  \hfill \(\quadra{\frac{3}{5}}\)
  \item \(\displaystyle \lim_{x \rightarrow \infty} 
          \tonda{\frac{2x^8+8x^6+6x^4}{4x^8-x^6+12x^4}}^5\)
  \hfill \(\quadra{\frac{1}{2}}\)
 \end{enumeratea}
 \end{multicols}
\end{esercizio}

\begin{esercizio}\label{ese:03.1}
Calcola i seguenti limiti:
\begin{multicols}{2}
 \begin{enumeratea}
  \item \(\displaystyle \lim_{x \rightarrow 0} 
          \frac{\sqrt{1+2x}-1}{3x}\)
  \hfill \(\quadra{\frac{1}{3}}\)
  \item \(\displaystyle \lim_{x \rightarrow 0} 
          \frac{\sqrt{1+x}-\sqrt{1-x}}{x}\)
  \hfill \(\quadra{1}\)
  \item \(\displaystyle \lim_{x \rightarrow 0} 
          \frac{x-\sqrt{x}}{\sqrt{x}}\)
  \hfill \(\quadra{-1}\)
  \item \(\displaystyle \lim_{x \rightarrow 5} 
          \frac{\sqrt{x-1}-2}{x^2-25}\)
  \hfill \(\quadra{\frac{1}{40}}\)
  \item \(\displaystyle \lim_{x \rightarrow 3} 
          \frac{\sqrt{x+6}-3}{x^3-5x^2+3x+9}\)
  \hfill \(\quadra{\infty}\)
  \item \(\displaystyle \lim_{x \rightarrow 9} 
          \frac{3-\sqrt{x}}{27-\sqrt{x^3}}\)
  \hfill \(\quadra{\frac{1}{27}}\)
  \item \(\displaystyle \lim_{x \rightarrow 0} 
          \frac{\sqrt[3]{1+x}-\sqrt[3]{1-x}}{x}\)
  \hfill \(\quadra{\frac{2}{3}}\)
  \item \(\displaystyle \lim_{x \rightarrow 1} 
          \frac{x^{\frac{2}{3}}-1}{x^{\frac{2}{5}}-1}\)
  \hfill \(\quadra{\frac{10}{9}}\)
  \item \(\displaystyle \lim_{x \rightarrow 1} 
          \frac{1-\sqrt[n]{x}}{1-\sqrt[m]{x}}\)
  \hfill \(\quadra{\frac{m}{n}}\)
  \item \(\displaystyle \lim_{x \rightarrow \infty} 
          \tonda{\sqrt{x-2}-\sqrt{x}}\)
  \hfill \(\quadra{0}\)
  \item \(\displaystyle \lim_{x \rightarrow \infty} 
          \tonda{\sqrt{x^2-x}-x}\)
  \hfill \(\quadra{\frac{1}{2}}\)
  \item \(\displaystyle \lim_{x \rightarrow -\infty} 
          \tonda{\sqrt{x^2-x}-x}\)
  \hfill \(\quadra{\infty}\)
  \item \(\displaystyle \lim_{x \rightarrow \infty} 
          \tonda{\sqrt{x-3}-\sqrt{x}}\)
  \hfill \(\quadra{0}\)
  \item \(\displaystyle \lim_{x \rightarrow \infty} 
          \sqrt{x}\tonda{\sqrt{x-3}-\sqrt{x}}\)
  \hfill \(\quadra{-\frac{3}{2}}\)
  \item \(\displaystyle \lim_{x \rightarrow \infty} 
          x\tonda{\sqrt{x^2+1}-x}\)
  \hfill \(\quadra{\frac{1}{2}}\)
  \item \(\displaystyle \lim_{x \rightarrow -\infty} 
          x\tonda{\sqrt{x^2+1}-x}\)
  \hfill \(\quadra{-\infty}\)
  \item \(\displaystyle \lim_{x \rightarrow \infty} 
          \tonda{\sqrt{x^2+1}-x}\)
  \hfill \(\quadra{0}\)
  \item \(\displaystyle \lim_{x \rightarrow -\infty} 
          \tonda{\sqrt{x^2+1}-x}\)
  \hfill \(\quadra{\infty}\)
  \item \(\displaystyle \lim_{x \rightarrow \infty} 
          \frac{\sqrt{x+2}-\sqrt{2}}{x}\)
  \hfill \(\quadra{0}\)
  \item \(\displaystyle \lim_{x \rightarrow \infty} 
          \frac{\sqrt{x+5}-\sqrt{5}}{\sqrt{x}-5}\)
  \hfill \(\quadra{1}\)
  \item \(\displaystyle \lim_{x \rightarrow \infty} 
          \frac{\sqrt{x^2+9}+\sqrt{x^2-9}}{6x}\)
  \hfill \(\quadra{\frac{1}{3}}\)
  \item \(\displaystyle \lim_{x \rightarrow \infty} 
          \frac{\sqrt{x^2+9}-\sqrt{x^2-9}}{6x}\)
  \hfill \(\quadra{0}\)
  \item \(\displaystyle \lim_{x \rightarrow \infty} 
          \frac{\sqrt{x-1}-2x}{x-7}\)
  \hfill \(\quadra{-2}\)
%   \item \(\displaystyle \lim_{x \rightarrow \infty} 
%           \frac{\sqrt{x}-6x}{3x+1}\)
%   \hfill \(\quadra{-2}\)
%   \item \(\displaystyle \lim_{x \rightarrow \infty} 
%           \frac{\sqrt{x^2+1}+\sqrt{x}}{\sqrt[4]{x^3+x}-x}\)
%   \hfill \(\quadra{-1}\)
%   \item \(\displaystyle \lim_{x \rightarrow \infty} 
%           \frac{\sqrt{x^2+1}+\sqrt{x}}{\sqrt[4]{x^2+1}-x}\)
%   \hfill \(\quadra{-1}\)
%   \item \(\displaystyle \lim_{x \rightarrow \infty} 
%           \frac{\sqrt[3]{x}-2\sqrt{x^3}}{\sqrt[4]{x^5}+x\sqrt{x}}\)
%   \hfill \(\quadra{-2}\)
 \end{enumeratea}
 \end{multicols}
\end{esercizio}

\subsection{Esercizi riepilogativi}

\begin{esercizio}\label{ese:03.1}
Ricavare dal grafico della funzione rappresentata le seguenti 
caratteristiche:
 \begin{enumeratea}
  \item  dominio;
  \item  punti di discontinuità e loro classificazione;
  \item  asintoti.
 \end{enumeratea}
\begin{center}\continuitagraficoese\end{center} % TODO
\end{esercizio}

\begin{esercizio}\label{ese:03.1}
Individua e classifica gli eventuali punti di discontinuità, in \(\R\), 
delle seguenti funzioni:
\begin{multicols}{2}
 \begin{enumerate} [label=\alph*]
  \item \(y = \dfrac{3 + x}{x^4 + 3 x^3}\)
   \hfill[\(0\): 2° tipo; \(-3\): 3° tipo]
  \item \(y = \dfrac{5}{3 + 5^{\frac{1}{x}}}\)
   \hfill[\(0\): 1° tipo]
  \item \(y = \dfrac{3 x - 3}{x^2 - 1}\)
   \hfill[\(-1\): 2° tipo; \(+1\): 3° tipo]
  \item \(y = \dfrac{1}{3^{\frac{1}{x}}-1}\)
   \hfill[\(0\): 1° tipo]
  \item \(y = \dfrac{x - 4}{x^2 - 16}\)
   \hfill[\(-4\): 2° tipo; \(+4\): 3° tipo]
  \item \(y = \dfrac{6}{2^{\frac{1}{x-3}} + 4}\)
   \hfill[\(+3\): 1° tipo]
 \end{enumerate}
\end{multicols}
\end{esercizio}

\begin{esercizio}\label{ese:03.1}
Determina per quale valore di \(k\) le seguenti funzioni sono continue:
 \begin{enumerate} [label=\alph*]
  \item 
  \(y=\begin{cases} 
    0 & \mbox{se } x \leqslant 1 \\ 
    x^2-2x+1 & \mbox{se } 1 \leqslant x \leqslant 3 \\ 
    3x+k & \mbox{se } x > 3
  \end{cases}\)
  \hfill [\(k=-5\)]
  \item 
  \(y=\begin{cases} 
    x^2-2x+k & \mbox{se } x \leqslant 0 \\ 
    \dfrac{2x-6}{x+3} & \mbox{se } x > 0
  \end{cases}\)
  \hfill [\(k=-2\)]
  \item 
  \(y=\begin{cases} 
    kx-2 & \mbox{se } x < 1 \\ 
    \dfrac{k}{x+1} & \mbox{se } x \geqslant> 0
  \end{cases}\)
  \hfill [\(k=+4\)]
 \end{enumerate}
\end{esercizio}

\begin{esercizio}\label{ese:03.1}
Studia la continuità delle seguenti funzioni e poi rappresentale:
 \begin{enumerate} [label=\alph*]
  \item 
  \(f(x)=\begin{cases} 
    -x^2+3 & \mbox{se } x < 0 \\ 
    e^x+2 & \mbox{se } x \geqslant 0
  \end{cases}\)
  \hfill [continua in \(\R\)]
  \item 
  \(f(x)=\begin{cases} 
    1-2^x & \mbox{se } 0 \leqslant x < 1 \\ 
    -\sqrt{x} & \mbox{se } 1 \leqslant x \leqslant 4
  \end{cases}\)
  \hfill [continua nell'I.D.]
  \item 
  \(f(x)=\begin{cases} 
    x^2 & \mbox{se } x \leqslant -1 \\ 
    2x+3 & \mbox{se } -1 < x < 0 \\
    4 & \mbox{se } x \geqslant 0
  \end{cases}\)
  \hfill [discontinua in 0]
 \end{enumerate}
\end{esercizio}

\begin{esercizio}\label{ese:03.1}
Studia la continuità delle seguenti funzioni e poi rappresentale:
\begin{multicols}{2}
 \begin{enumerate} [label=\alph*]
  \item \(y = x+\dfrac{1}{x}\) \hfill [\(\)]
  \item \(y = \dfrac{xe^x}{\sqrt{2x-1}}\) \hfill [\(\)]
  \item \(\dfrac{x^3+1}{x^2}\) \hfill [\(\)]
  \item \(\dfrac{x^2+4}{x-1}\) \hfill [\(\)]
 \end{enumerate}
\end{multicols}
\end{esercizio}

\begin{esercizio}\label{ese:03.1}
Disegnare il grafico di una funzione che abbia le seguenti caratteristiche:
 \begin{enumeratea}
  \item è definita in \(\tonda{-\infty;~-1} \cup \tonda{-1;~0} \cup 
                        \tonda{0;~+1} \cup \tonda{+1;~+\infty}\);
  \item ha come asintoti verticali solo le rette \(x= -1;~x=0;~x=1\);
  \item ha come asintoto orizzontale la retta \(y=0\);
  \item è positiva per \(-1<x<0 \quad e \quad x>1\);
  \item è simmetrica rispetto all’origine.
 \end{enumeratea}
\end{esercizio}
