% (c) 2015 Daniele Zambelli daniele.zambelli@gmail.com

\input{\folder limiticontinuita1_grafici.tex}

\chapter{Funzioni: continuità e limiti}

\section{Continuità}
\label{sec:cont_continuita}

Spesso per analizzare un fenomeno, e la funzione che lo rappresenta,  lo si 
valuta in vari punti di un intervallo \(\intervcc{a}{b}\) durante il 
quale si svolge.

\affiancati{.57}{.41}
{
Considerando i valori di \(f\) in un numero finito di punti non si può dire 
di conoscere appieno le caratteristiche del fenomeno, ma, se è 
\emph{sufficientemente regolare}, può darsi che i valori nei punti 
considerati diano già un'idea del fenomeno con una buona approssimazione.

Tanto è maggiore il numero dei punti considerati, tanto più ricca è 
l'informazione che si ricava. 

Non solo, ma si possono fare delle considerazioni, più o meno attendibili, 
sull'andamento globale del fenomeno confrontando a due a due i valori 
rilevati.
}
{\scalebox{1}{\partizionen}}

Se la funzione è poco regolare, tracciare solo i punti corrispondenti ad 
alcuni valori può servirci poco.
% Consideriamo la seguente funzione definita nell'intervallo 
% \(\intervcc{a}{b}\).

\affiancati{.49}{.49}
{\scalebox{1}{\puntigrafico}}
{\scalebox{1}{\graficodiscontinuo}}

Sono infinite le funzioni che passano per quell'insieme di punti. 
Se inoltre nel grafico ci sono dei salti, come nel secondo grafico
in \(x_4\) e \(x_7\), risulta difficile tracciarla 
basandosi solo sui punti individuati, anche aumentando il loro numero.

Abbiamo detto che la curva deve essere ``sufficientemente regolare'',
ma cosa significa essere \emph{sufficientemente regolare}?

Nella prossima sezione parleremo di un certo tipo di regolarità molto 
importante che è la continuità di una funzione in un intervallo.
% \vspace{.5em}

\subsection{Definizione di continuità in un punto}
\label{subsec:cont_definizione}

Intuitivamente possiamo dire che una funzione è 
\emph{continua in un intervallo} 
se è rappresentata da una linea senza interruzioni e salti.

Per precisare questo concetto, partiamo dal definire cos'è una funzione 
\emph{continua in un punto} interno al suo insieme di definizione.

\begin{definizione}
Diremo che una funzione è \textbf{continua in un punto \(c\)} non isolato, 
se è definita in \(c\) e, 
quando \(x\) è infinitamente vicino a \(c\), 
allora \(f(x)\) è infinitamente vicino a \(f(c)\), 

\vspace{1em}
e si scrive \(f\) è continua nel punto \(c\) se: \hspace{5mm}
\(\forall x \approx c \quad f(x) \approx f(c)\)

\vspace{.5em}
oppure: \hspace{51mm}
\(\forall \epsilon \approx 0 \quad f(c + \epsilon) \approx f(c)\)

\vspace{.5em}
oppure: \hspace{51mm}
\(\forall \epsilon \approx 0 \quad f(c + \epsilon) - f(c) \approx 0\)

\vspace{.5em}
o ancora: \hspace{49mm}
\(\forall x \stext{se } \pst{x} = c \sstext{allora} \pst{f(x)} = f(c)\)
\end{definizione}

\begin{esempio}
Data la funzione \(f(x)=-x^2+5\) dimostrare che \(f(x)\) è continua in~1.

\affiancati{.59}{.39}
{
La funzione è continua in~1 se per ogni \(x\) infinitamente vicino a~1, 
\(f(x)\) è infinitamente vicino a \(f(1)\); \\
cioè se: \quad 
\(\forall \epsilon \approx 0 \quad f(1 + \epsilon) - f(1) \approx 0\)

\emph{Dimostrazione}
\begin{align*}
f(1+\epsilon) - f(1) &= 
-\tonda{1+\epsilon}^2+5-\tonda{-1^2+5}=\\
&=\cancel{-1}-2\epsilon-\epsilon^2 \cancel{+5}~
  \cancel{+1}~\cancel{-5} = \\
&=-2\epsilon-\epsilon^2 = 
\epsilon \tonda{-2 - \epsilon}
\end{align*}
}{
\scalebox{1}{\contprimo}
}
Ora, il prodotto tra un infinitesimo e un finito è un infinitesimo, quindi, 
se la distanza tra \(x\) e \(1\) è infinitesima, anche la distanza tra 
\(f(x)\) e \(f(1)\) è infinitesima. \hfill \(\qed\) 
 
\end{esempio}

\begin{esempio}
Dimostrare che \(f(x)=\frac{\abs{x}}{x}\) non è continua in~0.

\affiancati{.59}{.39}
{
\emph{Dimostrazione}\\
Perché una funzione sia continua per un certo valore di \(x\) 
lì deve essere definita. 
Quindi \(f(x)\) non è continua dato che \(0\) non appartiene al suo insieme 
di definizione.

Per essere pignoli, non ha senso chiedersi se una funzione sia 
continua in un punto dove non è neppure definita!\hfill \qed
}{
\scalebox{.9}{\contsecondo}
}
\end{esempio}

% Si può anche osservare che non ha senso cercare se una
% funzione è continua in un punto dove non è definita:
% se non è definita in un punto, 
% lì non è né \emph{continua} né \emph{discontinua}, semplicemente non ha 
% alcun valore.

% \newpage %-------------------------------------------------

\begin{esempio}
Studia la continuità in zero della funzione \emph{segno}:
\(f(x)=\begin{cases} 
    -1 & \text{se } x < 0 \\ 
     0 & \text{se } x = 0 \\ 
    +1 & \text{se } x > 0
  \end{cases}
\)

\vspace{-.5em}
\affiancati{.39}{.59}
{
\scalebox{.9}{\fsegno}
}{
\emph{Svolgimento}\\
La funzione non è continua in \(0\) dato che ci sono dei valori 
\(x\) infinitamente vicini a \(0\) ma tali che \(f(x)\) non è infinitamente 
vicino a \(f(0)\). 

In questa funzione siamo addirittura nel caso in cui per \emph{nessun} 
valore di \(x\) infinitamente vicino a zero ma diverso da zero, 
\(f(x) \approx f(0)\).\hfill \(\qed\)
}
\end{esempio}

\affiancati{.59}{.39}
{
\begin{esempio}
 Dimostrare che \(f(x)=\frac{\abs{x}}{4}\) è continua in \(0\).

\emph{Dimostrazione}:
\[f(\epsilon) - f(0) = \frac{\abs{\epsilon}}{4} - \frac{\abs{0}}{4} = 
 \frac{\abs{\epsilon}}{4} \approx 0 \quad \forall \epsilon \approx 0\]
~ \hfill \(\qed\)
\end{esempio}
}{
\scalebox{1}{\contterzo}
}

\vspace{1em}

\begin{esempio} \label{limcont:ese_fparti}
Studia la continuità in \(x_0=2\) della funzione: 
\(f(x)=\begin{cases} 
    \dfrac{1}{2}x-2 & \text{se } x \leqslant 2 \\ 
    x^2-6x+7 & \text{se } x > 2
  \end{cases}\)

Per prima cosa dobbiamo verificare che la funzione sia definita per 
per \(x=2\) e trovare il suo valore: 
\(f(2) = \dfrac{1}{2} \cdot 2-2 = -1\)

Dobbiamo quindi verificare che \(f(x)\) sia infinitamente vicino 
a~\(f(2) = -1\), quando \(x\) è infinitamente vicino a~\(2\).

Dobbiamo distinguere i casi in cui ci avviciniamo a~2 da sinistra o da 
destra. In entrambi i casi consideriamo un qualunque infinitesimo 
\emph{positivo} \(\epsilon\) e esplicitiamo il segno:

\affiancati{.58}{.40}{
\begin{description}
 \item [da sinistra:]
 ~\\
\(f(2-\epsilon) - f(2) =
  \dfrac{1}{2} \cdot \tonda{2-\epsilon}-2-\tonda{-1} =\)\\
\(= 1 - \dfrac{\epsilon}{2} -2 +1 = -\dfrac{\epsilon}{2}\), \\ %\quad 
che è un infinitesimo \(\forall \epsilon \approx 0\).
 \item [da destra:]
 ~\\
\(f(2+\epsilon) - f(2) =\)\\
\(\tonda{2+\epsilon}^2-6 \cdot \tonda{2+\epsilon}+7-\tonda{-1} =\)\\
\(4+4\epsilon+\epsilon^2-12-6 \epsilon+7+1 =\)\\
\(\epsilon^2-2 \epsilon\),  \quad
che è un infinitesimo \(\forall \epsilon \approx 0\).\hfill \(\qed\)
\end{description}
}{
\begin{center}\continuitagraficoa\end{center}
}
\end{esempio}

\subsection{Funzioni continue}
\label{subsec:cont_definizione}

Dimostrare che una funzione è continua in un punto è piuttosto laborioso, 
pur non essendo complicato, ma quando sono interessato a studiare la 
continuità di una funzione in un intervallo sorge un ulteriore problema. 
Infatti in un intervallo, anche piccolo, i punti sono infiniti e dimostrare 
la continuità per ognuno di essi risulta piuttosto lungo\dots

Per superare questo scoglio, i matematici hanno pensato un approccio 
diverso:
generalizzare il problema studiando la continuità di intere funzioni.

A prima vista, questo può sembrare un modo per complicare il problema, 
invece permette di riconoscere la continuità di un gran numero di 
funzioni senza dover fare noiosi calcoli. La strada seguita consiste nei 
seguenti due passi:
\begin{itemize} [noitemsep]
\item dimostrare che alcune funzioni elementari sono continue nel loro 
insieme di definizione;
\item dimostrare che la combinazione di funzioni continue è ancora una 
funzione continua.
\end{itemize}
Di seguito vediamo qualche teorema che permette di stabilire la continuità di 
un gran numero di funzioni.

\subsubsection{Funzione continua}
\label{subsubsec:cont_funzionecontinua}

\begin{definizione}
Una \emph{funzione è continua} se è continua in ogni punto del suo 
insieme di definizione.
\end{definizione}

Quando voglio studiare se una funzione nel suo complesso è continua o no, lo 
dovrò fare tenendo conto del suo insieme di definizione: non ha senso 
domandarsi ad esempio se \(f(x) = \sqrt{x}\) è continua in \(-5\) dato che lì 
non è definita.

Alcuni grafici di funzioni continue:

\hspace{-12mm}\begin{minipage}{.32\textwidth}
\begin{center} \scalebox{.9}{\contsecondoa} \end{center}
\end{minipage}
\hfill
\begin{minipage}{.32\textwidth}
\begin{center} \contrad \end{center}
\end{minipage}
\hfill
\begin{minipage}{.32\textwidth}
\begin{center} \contip \end{center}
\end{minipage}


\subsubsection{Funzioni elementari}
\label{subsubsec:cont_funzionielementari}

Dimostriamo la continuità di alcune funzioni elementari.

\begin{teorema}[Continuità delle costanti]
Le funzioni costanti (\(f(x) = k\)) sono continue.
\end{teorema}

Ipotesi: \(f(x)=k\).\tab 
Tesi: \(f(x+\epsilon) \approx f(x) \quad \forall x\).

% \newpage %----------------------------------------------------

\noindent \emph{Dimostrazione}
Per la definizione di continuità vogliamo dimostrare che 

\affiancati{.49}{.49}{
\[\forall x \quad f(x + \epsilon) \approx f(x) \quad
  \forall \epsilon \approx 0\]
Essendo la funzione costante, 
\[f(x+\epsilon)=k\] 
che, ovviamente, è infinitamente vicino a \(f(x) = k\). 
In simboli:
\[f(x+\epsilon) = k \approx k = f(x) \quad \qed\] 
}{
\begin{center} \continuitafcostante \end{center}
}

\begin{teorema}[Continuità della funzione identica]
La funzione identica (\(f(x) = x\)) è continua.
\end{teorema}

\noindent Ipotesi: \(f(x)=x\).\tab 
Tesi: \(f(x+\epsilon) \approx f(x) \quad \forall x\).

\emph{Dimostrazione}
Per la definizione di continuità vogliamo dimostrare che 

\affiancati{.49}{.49}{
\[\forall x \quad f(x + \epsilon) \approx f(x) \quad
  \forall \epsilon \approx 0\]
Essendo la funzione identica: \(f(x+\epsilon)=x+\epsilon\). \\
Che è infinitamente vicino a \(f(x) = x\). 
In simboli:
\[f(x+\epsilon) = x+\epsilon \approx x = f(x) \quad \qed\] 
}{
\begin{center} \continuitafidentica \end{center}
}

% Da mettere nelle definizioni???
\newcommand{\overbow}[1]{
   \tikz [baseline = (N.base), every node/.style={}] {
      \node [inner sep = 0pt] (N) {$#1$};
      \draw [line width = 0.4pt] plot [smooth, tension=1.3] coordinates {
         ($(N.north west) + (0.1ex,0)$)
         ($(N.north)      + (0,0.5ex)$)
         ($(N.north east) + (0,0)$)
      };
   }
}

% % Altra soluzione:
% \newcommand{\tarc}{\mbox{\large$\frown$}}
% \newcommand{\arc}[1]{\stackrel{\tarc}{#1}}

\begin{teorema}[Continuità delle funzioni seno e coseno]
Le funzioni seno (\(f(x) = \sen x\)) e coseno (\(f(x) = \cos x\)) 
sono continue.
\end{teorema}

% \noindent Ipotesi: \(f(x)=\sen x\).\tab Tesi: \(f(x)\) è continua.
Il disegno riporta solo il primo quadrante, ma la dimostrazione è generale.

\noindent \emph{Dimostrazione}
Consideriamo un angolo \(\theta\) e un incremento \(\Delta\) 
di questo angolo. 

\affiancati{.59}{.39}{
Indichiamo con \(\Delta(x)\) l'incremento del coseno e con \(\Delta(y)\) 
l'incremento del seno corrispondenti all'incremento indicato 
con \(\Delta\): \quad
\(\Delta(x) = \Delta(\cos x) \stext{ e } \Delta(y) = \Delta(\sen x)\).

Il segmento \(PQ\) è la corda corrispondente all'arco \(\Delta\) e come tutte 
le corde è minore dell'arco su cui insiste:
% \[\text{Posto} \quad PQ = \sqrt{\tonda{\Delta x}^2 + \tonda{\Delta y}^2}\]
\[0 < \overline{PQ} < \overbow{PQ} = \Delta\]
 }{
\begin{center} \continuitasincosa \end{center}
 }
 
 \affiancati{.39}{.59}{
\begin{center} \continuitasincosb \end{center}
}{
Ora, se passiamo ad un incremento infinitesimo \(\delta\) otteniamo la stessa 
situazione:
\[0 < \overline{PQ} < \overbow{PQ} = \delta\]
E dato che \(dx\) e \(dy\) sono minori di \(PQ\) che è 
un infinitesimo, anche 
\(\Delta(\cos x) \stext{ e } \Delta(\sen x)\) sono infinitesimi. \qed
}

È stato dimostrato inoltre il seguente teorema relativo alle funzioni 
monotòne (crescenti o decrescenti):

\begin{teorema}[Continuità della funzione monotòne]
Una funzione reale \(f\) definita in un intervallo \(\intervcc{a}{b}\)
che sia monotòna e assuma tutti i valori compresi tra 
\(f(a) \stext{e} f(b)\) è continua.
\end{teorema}



\begin{comment}
A titolo di esempio dimostriamo la continuità della
funzione esponenziale. Dobbiamo mostrare che
lim x→c e x = e c .
 ̄  ̄e x
 − e c  ̄
  ̄ < ε, ε > 0
 ̄ x−c
  ̄ ε
 ̄e
 − 1 ̄ < = ε0
e
 c1 − ε0
 < e
 x−c
 < 1 + ε0
per 0 < ε0
 < 1 → ln ¡
1 − ε0
 ¢
 < x − c < ln ¡
1 + ε
0
 ¢
se
 δ = min ¡
− ln ¡
1 − ε0 ¢ , ln ¡
1 + ε0 ¢¢
 → −δ < x − c < δ

\end{comment}

\begin{comment}
\newpage %==========================================================

\begin{teorema}[Continuità della funzione logaritmo]
La funzione logaritmo (\(f(x) = \ln(x)\)) è continua.
\end{teorema}

\noindent Ipotesi: \(f(x)=\ln x\).\\ 
Tesi: 
\(f(x +\epsilon) \approx f(x) \stext{ cioè }
\ln(x +\epsilon) - \ln(x) \approx 0 \quad (\forall \epsilon \approx 0)\).

\emph{Dimostrazione}
\begin{align*}
\ln(x +\epsilon) - \ln(x) \stackrel{1}{=} 
\ln \frac{x + \epsilon}{x} \stackrel{2}{=} 
% \ln \tonda{\frac{\cancel{x}}{\cancel{x}} + \frac{\epsilon}{x}} 
% \stackrel{3}{=} 
\epsilon \cdot \frac{1}{\epsilon} \ln \tonda{1 + \frac{\epsilon}{x}} 
\stackrel{3}{=} 
\epsilon  \ln \tonda{1 + \frac{1}{x}\epsilon}^{\frac{1}{\epsilon}} 
\stackrel{4}{\approx} 
\epsilon \ln \tonda{e^{\frac{1}{x}}} 
% \stackrel{5}{\approx} 
% \epsilon \frac{1}{x} 
\stackrel{5}{\approx} 0
\end{align*}
~\qed

Dove i passaggi indicati hanno le seguenti giustificazioni.
\begin{enumerate} [nosep]
\item Proprietà dei logaritmi: differenza di logaritmi.
% \item Banale passaggio algebrico.
\item Moltiplicazione per \(1 = \frac{\epsilon}{\epsilon}\).
\item Proprietà dei logaritmi: costante per un logaritmo.
\item Vedi nel capitolo sulle funzioni esponenziali l'esempio:
``Una particolare funzione esponenziale'' 
% \item Per la definizione di logaritmo.
% \item Per qualunque \(x \in \R\) anche \(e^{\frac{1}{x}}\)
% è un numero reale e quindi finito. 
\item Per qualunque \(x \in \R\) anche \(\ln \tonda{e^{\frac{1}{x}}}\)
è un numero reale e quindi finito. 
Moltiplicando un finito per un infinitesimo si ottiene un infinitesimo.
\end{enumerate}

\begin{teorema}[Continuità della funzione esponenziale]
La funzione esponenziale (\(f(x) = e^x\)) è continua.
\end{teorema}

\noindent Ipotesi: \(f(x)=e^x\).\\ 
Tesi: 
\(f(x +\epsilon) \approx f(x) \stext{ cioè }
e^{x +\epsilon} - e^x \approx 0 \quad (\forall \epsilon \approx 0)\).

\emph{Dimostrazione}
\begin{align*}
e^{x +\epsilon} - e^x \stackrel{1}{=} 
e^{x +\epsilon} - e^x \stackrel{2}{=} 
\ln \tonda{\frac{\cancel{x}}{\cancel{x}} + \frac{\epsilon}{x}} 
\stackrel{3}{=} 
\epsilon \cdot \frac{1}{\epsilon} \ln \tonda{1 + \frac{\epsilon}{x}} 
\stackrel{4}{=} 
\epsilon  \ln \tonda{1 + \frac{1}{x}\epsilon}^{\frac{1}{\epsilon}} 
\stackrel{5}{\approx} \epsilon e^{\frac{1}{x}} \stackrel{6}{\approx} 0
\end{align*}
~\qed

Dove i passaggi indicati hanno le seguenti giustificazioni.
\begin{enumerate} [nosep]
\item Proprietà dei logaritmi: differenza di logaritmi.
\item Banale passaggio algebrico.
\item Moltiplicazione per \(1 = \frac{\epsilon}{\epsilon}\).
\item Proprietà dei logaritmi: costante per un logaritmo.
\item Vedi nel capitolo sulle funzioni esponenziali l'esempio:
``Una particolare funzione esponenziale'' 
\item Per qualunque \(x \in \R\) anche \(e^{\frac{1}{x}}\)
è un numero reale e quindi finito. 
Moltiplicando un finito per un infinitesimo si ottiene un infinitesimo.
\end{enumerate}

\end{comment}

\begin{comment}

\begin{teorema}[Continuità della funzione esponenziale in zero]
La funzione esponenziale (\(f(x) = e^x\)) è continua in \(0\).
\end{teorema}

\noindent Ipotesi: \(f(x)=e^x\).\\ 
Tesi: 
\(f(\epsilon) \approx f(0) \stext{ cioè }
e^\epsilon \approx 1 \quad (\forall \epsilon \approx 0)\).

\emph{Dimostrazione}

Dimostriamo che, intorno allo zero, diciamo nell'intervallo 
\(\intervaa{-1}{+1}\), 
la funzione \(y = e^x\) è compresa tra due funzioni continue che 
in zero valgono \(1\) e poi dimostriamo la tesi.

Scegliamo \\
come funzione minorante: \(g(x) = 1 + x\) \\
come funzione maggiorante: \(h(x) = \dfrac{1}{1 - x}\)

\newpage % ------------------------------------

\begin{enumerate}
\item Nel primo passo dimostriamo che \(e^x \geqslant 1 +x\).

\affiancati{.49}{.49}{
\(\tonda{1+\dfrac{x}{1}}^1 = 1 +x\)\\
\(\tonda{1+\dfrac{x}{2}}^2 = 
   1 +\cancel{2}\dfrac{x}{\cancel{2}} + \dfrac{x^2}{2^2} > 1 +x\)\\
\(\tonda{1+\dfrac{x}{3}}^3 = 
   1 +\cancel{3}\dfrac{x}{\cancel{3}} + 
     3\dfrac{x^2}{3^2} + \dfrac{x^3}{3^3} > 1 +x\)\\
\(\tonda{1+\dfrac{x}{4}}^4 = 
   1 +\cancel{4}\dfrac{x}{\cancel{4}} + 
     6\dfrac{x^2}{4^2} + \dots > 1 +x\)\\
\dots \\[.5em]
\(\tonda{1+\dfrac{x}{n}}^n = 
   1 +\cancel{n}\dfrac{x}{\cancel{n}} + \dots > 1 +x\)
}{
\begin{center} \continuitafespa \end{center}
}
Per la proprietà di transfer, passando dal numero finito \(n\) al numero 
infinito \(N\) avremo ancora:
\[\tonda{1+\dfrac{x}{N}}^N = 
   1 +\cancel{N}\dfrac{x}{\cancel{N}} + \dots \geqslant 1 +x\]
Ma \(\tonda{1+\dfrac{x}{N}}^N \approx e^x\) quindi:
\[e^x \geqslant 1 +x\]

% \[e^x \stackrel{1}{\approx} \tonda{1+\frac{x}{N}}^N \stackrel{2}{=}
% 1^N + \cancel{N} \cdot 1^{N-1} \cdot \frac{x}{\cancel{N}} + \dots 
% \leqslant 1 + x\]

\item Nel secondo passo dimostriamo che 
\(e^x \leqslant \dfrac{1}{1 - x}\)

\affiancati{.49}{.49}{
Abbiamo appena dimostrato che \(e\) elevato ad un qualunque esponente 
è maggiore di \(1\) più quell'esponente perciò \\ [.5em]
anche: \quad \(e^{-x} \geqslant 1 + \tonda{-x}\) \\[.5em]
cioè: \quad \(e^{-x} \geqslant 1 -x\) \\ 
e passando ai reciproci: \quad 
\(\dfrac{1}{e^{-x}} \leqslant \dfrac{1}{1 - x}\)\\
ma: \quad \(\dfrac{1}{e^{-x}} = e^x\) \\ 
Quindi: \quad \(e^x \leqslant \dfrac{1}{1 - x}\)
}{
\begin{center} \continuitafespb \end{center}
}
\item Ora dimostriamo la tesi:

Per il transfer quello che vale per \(x\) \emph{reale}, 
vale anche per \(x\) \emph{iperreale} 
e, in particolare, infinitesimo:

\affiancati{.49}{.49}{
\[1 + x \leqslant e^x \leqslant \dfrac{1}{1 - x}\]
per transfer: 
\[1 + \epsilon \leqslant e^\epsilon \leqslant \dfrac{1}{1 - \epsilon}\]
Ma questo ci dice che \(e^\epsilon\) è compreso tra due numeri infinitamente 
vicini a \(1\) quindi anche lui sarà infinitamente vicino a \(1\):
\[1 \approx 1 + \epsilon \leqslant e^\epsilon \leqslant 
  \dfrac{1}{1 - \epsilon} \approx 1
  \sRarrow e^\epsilon \approx 1\]

\vspace*{-2.6em} \qed
}{
\begin{center} \continuitafespc \end{center}
}
\end{enumerate}

A questo punto e semplice dimostrare che la funzione esponenziale è 
continua.
\begin{teorema}[Continuità della funzione esponenziale]
La funzione esponenziale (\(y=e^x\)) è continua.
\end{teorema}

\noindent Ipotesi: \(f(x)=e^x\).\tab 
Tesi: \(f(x+\epsilon) \approx f(x) \quad \forall x\).

\emph{Dimostrazione}
Usiamo la prima proprietà delle potenze:
\begin{align*}
f(x+\epsilon) &= 
e^{x+\epsilon} = e^{x} \cdot e^{\epsilon} = 
e^{x} \cdot \tonda{1 + \delta} = 
e^{x} + e^x \cdot \delta \approx 
e^{x} = f(x) \qed
\end{align*}

%---------- tentativi inutili
% \nopagebreak
% \samepage
% \filbreak
%----------

%----------
% 13
%  \setlength{\columnsep}{1.5pc}
% We want a rule between columns.
% 14
%  \setlength\columnseprule{.4pt}
% We also want to ensure that a new multicols envi-
% ronment finds enough space at the bottom of the
% page.
% 15
%  \setlength\premulticols{6\baselineskip}

%----------

\end{comment}

\noindent\begin{minipage}{\textwidth}
Oltre alle funzioni precedenti, anche altre funzioni elementari sono 
continue, il seguente elenco riporta le principali funzioni continue:

\noindent\begin{minipage}{1.05\textwidth}
\begin{multicols}{5}
\begin{itemize} [noitemsep]
 \item \(y=k\)
 \item \(y=x\)
 \item \(y=\frac{1}{x}\)  \textasteriskcentered
 \item \(y=\sqrt[n]{x}\)  \textasteriskcentered
 \item \(y=\abs{x}\)
 \item \(y=a^x\)
 \item \(y=\log_a x\)  \textasteriskcentered
 \item \(y=\sen x\)
 \item \(y=\cos x\)
 \item \(y=\tg x\)  \textasteriskcentered
\end{itemize}
\end{multicols}
\end{minipage}

\begin{osservazione}
Le funzioni segnate da ``\textasteriskcentered'' non sono definite su 
tutto \(\R\).
\end{osservazione}
\end{minipage}

\subsubsection{Composizione di funzioni}
\label{subsubsec:cont_composizionefunzioni}

Vediamo ora che anche componendo in alcuni modi funzioni continue otteniamo 
ancora funzioni continue.

\begin{teorema}[Somma di funzioni continue]
Se \(f\) e \(g\) sono funzioni continue, anche \(f+g\) è continua.
\end{teorema}

\noindent Ipotesi: 
\(f(x) \stext{e} g(x)\) sono continue
\tab Tesi: 
\(f(x)+g(x)\) è continua.

\emph{Dimostrazione}
Dato che sono continue, 
\(f(x+\epsilon) = f(x)+\alpha \stext{ e } g(x+\epsilon) = g(x)+\beta\) \\
e dato che la somma di infinitesimi è un infinitesimo:
\[f(x+\epsilon) + g(x+\epsilon) = 
\tonda{f(x)+\alpha} + \tonda{g(x)+\beta} = 
f(x)+g(x)+\tonda{\alpha + \beta} \approx f(x)+g(x) \qed\]

\begin{teorema}[Prodotto di funzioni continue]
Se \(f\) e \(g\) sono funzioni continue, anche \(f \cdot g\) è continua.
\end{teorema}

\noindent Ipotesi: 
\(f(x) \stext{e} g(x)\) sono continue
\tab Tesi: 
\(f(x) \cdot g(x)\) è continua.

\emph{Dimostrazione}
Dato che sono continue e dato che il prodotto tra un numero finito e un 
infinitesimo è un infinitesimo: 
\[f(x+\epsilon) \cdot g(x+\epsilon) = 
\tonda{f(x)+\alpha} \cdot \tonda{g(x)+\beta} = 
f(x) \cdot g(x) + f(x) \cdot \beta + g(x) \cdot \alpha + \alpha \cdot \beta
\approx f(x) \cdot g(x)\]
\qed

\begin{corollario}
 Ogni funzione polinomiale è continua.
\end{corollario}

\emph{Dimostrazione}
Dato che una funzione polinomiale si può ottenere partendo da funzioni 
costanti e da funzioni identiche attraverso moltiplicazioni e addizioni, 
la tesi consegue dai teoremi precedenti. \qed

\begin{esempio}
 Dimostrare che \(f(x)=2x^2 + 3\) è una funzione continua.

\emph{Dimostrazione}
\(f(x)=2x^2 + 3\) è continua perché è somma di due funzioni continue: 
% \begin{itemize}[noitemsep]
%  \item \(f(x)=2x^2 + 3\) è continua perché è somma di due funzioni 
% continue: 
 \begin{itemize}[noitemsep]
  \item \(2x^2\) è continua perché è prodotto di due funzioni continue:
  \begin{itemize}[nosep]
   \item \(2\) è continua perché è una costante;
   \item \(x^2\) è continua perché è prodotto di due funzioni continue:
   \begin{itemize}[nosep]
    \item \(x\) è continua perché è una funzione identica;
    \item \(x\) è continua perché è una funzione identica;
   \end{itemize}
  \end{itemize}
  \item \(3\) è continua perché è una costante. \qed
 \end{itemize}
% \end{itemize}
\end{esempio}

% \newpage %------------------------------------------------------

\begin{teorema}[Funzioni di funzioni]
Se \(f(x)\) e \(g(x)\) sono funzioni continue, anche \(f(g(x))\) è continua.
\end{teorema}

\noindent Ipotesi: 
\(f(x) \text{ e} g(x)\) sono continue
\tab Tesi: 
\(f(g(x))\) è continua.

\emph{Dimostrazione}
Dato che \(g\) è continua: 
\[f(g(x+\epsilon)) = f(g(x)+\alpha)\]
e dato che \(f\) è continua: 
\[f(g(x)+\alpha)=f(g(x))+\beta\]
quindi: 
\[f(g(x+\epsilon)) = f(g(x)+\alpha) = f(g(x))+\beta \approx f(g(x)) \qed\]

\subsection{Continuità in un intervallo}
\label{subsec:cont_definizione}

Nello sviluppo dell'analisi hanno grande importanza le funzioni 
\(f\) puntualmente continue in \emph{ciascun} punto di un intervallo chiuso 
\(\intervcc{a}{b}\).

Diremo che una funzione \(f\) è puntualmente continua nell'intervallo 
\(\intervcc{a}{b}\) se è \emph{continua in ogni punto \(c\)} dell'intervallo:
\[f(x) \approx f(c) \stext{per ogni} x \stext{tale che} 
x \in \intervcc{a}{b} \stext{e} x \approx c.\] 
In questo caso diremo semplicemente che 
\begin{center}
la funzione \(f\) è continua nell'intervallo \(\intervcc{a}{b}\). 
\end{center}
Questa affermazione equivale a dire che:

\affiancati{.39}{.59}{
\begin{center} \continuitaintervallo \end{center}
}{
\begin{itemize}
\item 
per i punti \(c\) interni all’intervallo \(\intervcc{a}{b}\) 
(i punti che appartengono all'intervallo aperto \(\intervaa{a}{b}\))
per ogni infinitesimo \(\epsilon\) si ha che 
\(f(c+\epsilon) \approx f(c)\);
\item 
per gli estremi dell’intervallo ci si accontenterà di dire che è continua a 
destra in \(a\) e a sinistra in \(b\), cioè, per 
ogni infinitesimo \emph{positivo} \(\epsilon\), si ha che 
\(f(a+\epsilon) \approx f(a)\) e \(f(b-\epsilon) \approx f(b)\). \\
Cioè con \(\epsilon > 0\):\\ 
\(\pst{f(a+\epsilon)} = f(a) \stext{ e } \pst{f(b-\epsilon)} = f(b)\).
\end{itemize}
}
 
\affiancati{.49}{.49}{
\begin{esempio}
~

La funzione rappresentata qui a fianco, è continua nell'intervallo 
chiuso \(\intervcc{b}{c}\) ma non lo è negli intervalli chiusi 
\(\intervcc{a}{b}\) e \(\intervcc{c}{d}\). 

Si può dire comunque che è continua negli intervalli aperti:
\(\intervca{a}{b}\) e \(\intervac{c}{d}\).
\end{esempio}
}{
\begin{center}\continuitaintervalli\end{center}
}

La definizione di continuità in un punto non ci è di grande aiuto per 
verificare se una funzione è continua in un intervallo, dato che dovremmo 
controllare la continuità in infiniti punti.

Si può seguire un'altra via: 

\begin{definizione}
Una funzione è continua in un intervallo se:
\begin{itemize} [noitemsep]
\item 
la funzione, ristretta a quell'intervallo, è continua e
\item 
è definita in tutti i punti dell'intervallo.
\end{itemize}
\end{definizione}

\begin{esempio}
Data la funzione \(f(x) = \dfrac{x-1}{x-5}\) stabilisci se la funzione è 
continua negli intervalli chiusi \(A=\intervcc{-1,5}{+2,5}\) e 
\(B=\intervcc{+3,5}{+7,5}\).

La funzione è continua (nel suo insieme di definizione) essendo una 
composizione di funzioni continue.

\affiancati{.49}{.49}{
Perché sia continua negli intervalli richiesti basta che questi intervalli 
siano sottoinsiemi del suo insieme di definizione.
\[\ID =~ \intervaa{-\infty}{+5} ~\cup~ \intervaa{+5}{+\infty} ~=~ 
\R \setminus \graffa{+5}\]
L'intervallo \(A=\intervcc{-1,5}{+2,5}\) è del tutto contenuto in \(\ID\) \\
L'intervallo \(B=\intervcc{+3,5}{+7,5}\) ha un elemento, \(+5\), 
che non appartiene a \(\ID\)

Perciò \(f(x)\) è continua nell'intervallo \(A\) e non è continua 
nell'intervallo \(B\).
}{
\begin{center} \continuitaintervalloese \end{center}
}
\end{esempio}

\begin{esempio}
Riprendendo la funzione dell'esempio \ref{limcont:ese_fparti}
possiamo dire che tutta la funzione è continua in \(\R\) dato che:
\begin{itemize}
\item è continua in \(\intervaa{-\infty}{2}\) 
poiché \(x \mapsto \dfrac{1}{2}x -2\) è una funzione polinomiale;
\item è continua in \(2\) avendolo dimostrato nell'esempio;
\item è continua in \(\intervaa{2}{+\infty}\)
poiché \(x \mapsto x^2 -6x +7\) è una funzione polinomiale.
\end{itemize}
\end{esempio}


% TODO: aggiungere un po' di esempi per fare andare a pagina nuova 
% i limiti.

\newpage %-----------------------------------------------

\section{Limiti}
\label{sec:cont_limiti}

\footnote{Per scrivere questo capitolo mi sono ispirato 
al testo di Howard Jerome Keisler, 
``Elementary Calculus: An Infinitesimal Approach''. 
Chi volesse approfondire l'argomento può scaricare il testo all'indirizzo: 
\href{https://www.math.wisc.edu/~keisler/calc.html}
     {www.math.wisc.edu/\(\sim\)keisler/calc.html}}
In alcuni problemi non siamo interessati a sapere come si comporta una 
funzione per un valore ben preciso (dove magari non è neppure definita), ci 
interessa di più sapere come si comporta quando la variabile \(x\) è 
\emph{sufficientemente vicina} a quel valore.

Detto altrimenti, cerchiamo di rispondere alla domanda: \\
\emph{Per un certo valore di \(x\) la funzione potrebbe anche \emph{non} 
essere definita, ma cosa succede quando \(x\) si avvicina abbastanza a quel 
valore?} \\
Vediamo un esempio.

\begin{esempio}
 Studia l'Insieme di Definizione della funzione: 
 \(f(x)=\dfrac{x^2-6x+5}{x^2+2x-3}\),\quad
poi studia come si comporta la funzione per valori infinitamente vicini 
agli estremi dell'\(\ID\)

\affiancati{.44}{.54}{
La funzione è fratta e quindi non è definita quando il denominatore vale 
zero:
\begin{align*}
&x^2+2x-3=0 \sRarrow \tonda{x+3}\tonda{x-1}=0 \\ 
&\sRarrow x_1=-3;~x_2=+1
\end{align*}
L'insieme di definizione è formato quindi dall'unione dei seguenti 
intervalli:
\begin{align*}
&\intervaa{-\infty}{-3} \quad \cup \quad 
\intervaa{-3}{+1}\quad \cup \quad 
\intervaa{+1}{-\infty} \\ 
&\text{oppure} \qquad \R \setminus \graffa{-3;~+1}
\end{align*}
}{
\begin{center}\scalebox{1}{\limitigraficoa}\end{center}
}

Calcoliamo alcuni valori della funzione ``vicini'' ai 4 estremi 
dell'Insieme di Definizione:

\vspace{1em}
\begin{minipage}{.24\textwidth}
\begin{center}
\(-\infty\)\\
\begin{tabular}{r|r}
\(x\) & \(f(x)\quad\)\\\hline
\(-100\) & \(+1.0825\) \\ 
\(-1000\) & \(+1.0080\) \\ 
\(-10000\) & \(+1.0008\) \\
\dots \\
&\\
&\\
&
\end{tabular}
\end{center}
\end{minipage}
\begin{minipage}{.24\textwidth}
\begin{center}
\(-3\)\\
\begin{tabular}{r|r}
\(x\) & \(f(x)\quad\)\\\hline
\(-3.100\) & \(+81.0\) \\ 
\(-3.010\) & \(+801.0\) \\ 
\(-3.001\) & \(+8001.0\) \\
\dots \\
\(-2.999\) & \(-7999.0\) \\ 
\(-2.990\) & \(-799.0\) \\ 
\(-2.900\) & \(-79.0\) \\
\end{tabular}
\end{center}
\end{minipage}
\begin{minipage}{.24\textwidth}
\begin{center}
\(+1\)\\
\begin{tabular}{r|r}
\(x\) & \(f(x)\quad\)\\\hline
\(+0.900\) & \(-1.0513\) \\ 
\(+0.990\) & \(-1.0050\) \\ 
\(+0.999\) & \(-1.0005\) \\
\dots \\
\(+1.001\) & \(-0.9995\) \\ 
\(+1.010\) & \(-0.9950\) \\ 
\(+1.100\) & \(-0.9512\) \\
\end{tabular}
\end{center}
\end{minipage}
\begin{minipage}{.24\textwidth}
\begin{center}
\(+\infty\)\\
\begin{tabular}{r|r}
\(x\) & \(f(x)\quad\)\\\hline
\(+100\) & \(+0.9223\) \\
\(+1000\) & \(+0.9920\) \\
\(+10000\) & \(+0.9992\) \\
\dots \\
&\\
&\\
&
\end{tabular}
\end{center}
\end{minipage}

\vspace{1em}
Possiamo osservare che:
\begin{itemize} [nosep]
 \item Quando \(x\) diventa sempre più negativo, 
%  (si avvicina a meno infinito), 
\(f(x)\) si avvicina a~\(1\) dall'alto.
 \item Quando \(x\) si avvicina a~\(-3\) da sinistra, 
\(f(x)\) diventa sempre più grande.
 \item Quando \(x\) si avvicina a~\(-3\) da destra, 
\(f(x)\) diventa sempre più negativo.
 \item Quando \(x\) si avvicina a~\(+1\) da sinistra, 
\(f(x)\)  si avvicina a~\(-1\) dal basso.
 \item Quando \(x\) si avvicina a~\(+1\) da destra, 
\(f(x)\)  si avvicina a~\(-1\) dall'alto.
 \item Quando \(x\) diventa sempre più grande, 
%  (si avvicina a più infinito), 
\(f(x)\) si avvicina a~\(1\) dal basso.
\end{itemize}
\end{esempio}

Osservando il grafico possiamo dire che:

\begin{itemize} [nosep]
 \item 
Quando \(x\) si avvicina a \(-3\), la funzione 
da un lato diventa sempre più grande, 
dall'altro sempre più piccola (sempre più negativa);
 \item 
quando \(x\) vale proprio \(-3\), la funzione non è definita.
 \item 
Quando \(x\) si avvicina a \(+1\), la funzione, 
da entrambi i lati si avvicina a \(-1\);
 \item 
quando \(x\) vale proprio \(+1\), la funzione non è definita.
\end{itemize}

\affiancati{.64}{.34}{
\begin{osservazione}
Nel tratto crescente della funzione, che appare come linea continua, in 
realtà manca un punto: il punto \(\punto{1}{-1}\). 
Questo punto mancante è invisibile essendo infinitamente piccolo, quindi il 
grafico della funzione appare continuo. 
Possiamo indicare un punto mancante in una linea usando un cerchietto vuoto 
come nell'ingrandimento a fianco.
\end{osservazione}
}{
\begin{center}\scalebox{0.9}{\limitigraficob}\end{center}
}

Alcuni matematici del 1800, rifiutandosi di usare infinitesimi e infiniti, 
hanno ricostruito la matematica creata nei secoli precedenti usando il 
concetto di \emph{limite}.
I \emph{limiti} studiano il comportamento di una funzione quando \(x\) è 
infinitamente vicino ad un certo valore ma diverso da quel valore.
La definizione di limite perfezionata da Weierstrass è piuttosto 
complessa e la vedremo alla fine del capitolo.
Noi che usiamo numeri infinitesimi e infiniti diremo che:

\begin{definizione}
Il numero reale \(L\) è il \textbf{limite} della funzione \(f(x)\) 
per \(x\) che tende a un valore \(c\) se \(f(x)\) è infinitamente vicino a 
\(L\) per tutti gli \(x\) infinitamente vicini a \(c\) e diversi da 
\(c\).
\[\lim_{x \to c} f(x) = L \stext{ se } \forall x \approx c \stext{ e } 
x \neq c, \quad f(x) \approx L\] \\
Dato che il numero reale infinitamente vicino ad un iperreale è proprio il 
risultato della parte standard, possiamo scrivere:
\[\lim_{x \to c} f(x) = \pst{f(c + \epsilon)} \]

\vspace{-1.5em}
\begin{center} 
se questa parte standard esiste e è sempre la stessa 
per ogni infinitesimo \(\epsilon \neq 0\).
\end{center}
\end{definizione}

\begin{esempio}
Consideriamo la funzione \(f\) il cui grafico è riportato a fianco.

\affiancati{.54}{.44}{
Possiamo osservare che la funzione è continua (perché?) e che non è 
definita per 
\(x = a \stext{ e } x = b\), quindi:
\[\ID = \R \setminus \graffa{a;~b}\]
Possiamo studiare il limite per \(x \to a\) e \(x \to b\). 

\[\lim_{x \to a}f(x) = L\]
dato che per ogni \(x\) infinitamente vicino a \(a\) e diverso da \(a\), 
\(f(x)\) è infinitamente vicino a \(L\). 
\[\lim_{x \to b}f(x) = \text{ non esiste}\]
}{
\scalebox{.8}{\limitefinitodef}
}
perché per alcuni \(x\) infinitamente vicini a \(b\), \(f(x)\) è 
infinitamente vicino a \(M\) e per alcuni \(x\) sempre infinitamente 
vicini a \(b\), \(f(x)\) è infinitamente vicino a \(N\).
\end{esempio}

La definizione di limite richiede soltanto il concetto di 
\emph{infinitamente vicino}, la sua formulazione con la parte standard 
fornisce un metodo per \emph{calcolarlo}:

\begin{procedura}
Per calcolare il limite finito di una funzione per \(x \to c\)

\begin{enumerate} [noitemsep]
\item il primo passo consiste nel calcolare il valore di 
\(f(c+\epsilon)\);
\item se è possibile, si calcola parte standard;
\item se il valore di \(\pst{f(c+\epsilon)}\) è indipendente dal valore di
\(\epsilon\), questo è il limite.
\end{enumerate}
\end{procedura}


\begin{esempio}
Riprendiamo il primo esempio presentato in questa sezione e calcoliamo il 
limite: \(\displaystyle \lim_{x \to 1}\dfrac{x^2-6x+5}{x^2+2x-3}\).

Studia il comportamento della funzione 
\(f(x) = \dfrac{x^2-6x+5}{x^2+2x-3}\)
vicino a \(2\). 
\[
f(1 + \epsilon) \stackrel{1}{=}
  \dfrac{\tonda{1+\epsilon}^2-6\tonda{1+\epsilon}+5}
        {\tonda{1+\epsilon}^2+2\tonda{1+\epsilon}-3} =
  \dfrac{\cancel{1}+2\epsilon+\epsilon^2~
            \cancel{-6}-6\epsilon~\cancel{+5}}
        {\cancel{1}+2\epsilon+\epsilon^2~
            \cancel{+2}+2\epsilon~\cancel{-3}}
\stackrel{2}{=}
  \dfrac{-4\epsilon+\epsilon^2}
        {+4\epsilon+\epsilon^2}
\stackrel{3}{=}
  \dfrac{\cancel{\epsilon}\tonda{-4+\epsilon}}
        {\cancel{\epsilon}\tonda{+4+\epsilon}}
\]
Dato che il valore ottenuto è finito 
(il quoziente di due finiti non infinitesimi) 
possiamo calcolare la parte standard:
\[
\lim_{x \to 2} \frac{x^2-6x+5}{x^2+2x-3} \stackrel{4}{=} 
\pst{\frac{-4+\epsilon} {+4+\epsilon}} \stackrel{5}{=}
\frac{\pst{-4+\epsilon}}{\pst{+4+\epsilon}} \stackrel{6}{=} 
\frac{-4}{+4} = -1
\]
Dato che il valore ottenuto è indipendente dall'infinitesimo \(\epsilon\), 
\(-1\) è il limite cercato.

Motivazione dei vari passaggi:
\begin{enumerate} [nosep]
\item calcoliamo \(f(1+\epsilon)\);
\item otteniamo il rapporto tra due infinitesimi;
\item possiamo raccogliere e semplificare \(\epsilon\);
\item dato che abbiamo ottenuto un valore finito possiamo applicare la parte 
standard;
\item dato che il denominatore non è infinitesimo, la parte standard della 
frazione è uguale al rapporto delle parti standard del numeratore e del 
denominatore;
\item la parte standard di un reale più un infinitesimo è il reale stesso.
\end{enumerate}
\end{esempio}

% \footnote{Possiamo inserire queste osservazioni nella definizione 
% ottenendo:
% \begin{definizione}
% Il numero reale \(l\) è il \textbf{limite} di una funzione reale \(f(x)\) 
% per \(x\) che tende a un valore reale \(c\), se, 
% quando \(x\) è un qualunque numero iperreale infinitamente vicino a \(c\), 
% ma diverso da \(c\), 
% allora il valore della corrispondente funzione iperreale \({}^*f(x)\) è 
% infinitamente vicino al valorereale \(l\).
% \end{definizione}}

\subsubsection{Casi in cui il limite finito non esiste}
\label{subsubsec:cont_limiti_nonesiste}

Il metodo precedente permette di calcolare il limite 
solo se ci sono alcune condizioni, ma in vari casi può fallire. 
I motivi per cui può fallire sono:
\begin{enumerate} [nosep]
\item la funzione \(f(c+\epsilon)\) non è definita per alcuni valori 
dell'infinitesimo \(\epsilon \neq 0\);
\item la funzione \(f(c+\epsilon)\) dà come risultato un valore infinito e 
quindi non ha parte standard; 
\item al variare dell'infinitesimo \(\epsilon\) la parte standard 
\(\pst{f(c+\epsilon)}\) non è sempre la stessa. 
\end{enumerate}

Nelle prossime sezioni vedremo quando è possibile, comunque, ricavare delle 
informazioni interessanti in alcune delle situazioni elencate sopra.

\subsection{Se non esiste \(f(c+\epsilon)\) per alcuni valori di 
\(\epsilon\)}
\label{subsec:cont_limiti_nonsempreesiste}

\affiancati{.49}{.49}{
\begin{center} \limitenea \end{center}
}{
Potremmo trovarci nella situazione nella quale, a distanza infinitesima 
non nulla da un certo valore \(c\), la funzione non sia definita.

In questo caso il limite non esiste.
}

Ma potremmo trovarci in una situazione fortunata se la funzione è sempre 
definita a sinistra o a destra del punto \(c\).

In questo caso potremmo parlare di limite sinistro o di limite destro della 
funzione.

\affiancati{.49}{.49}{
Ad esempio nella situazione rappresentata a fianco, la funzione è sempre 
definita a sinistra del punto \(c_1\) e a destra del punto \(c_2\).
Pur non esistendo i limiti in \(c_1\) e \(c_2\),
potremo dire che esiste il limite sinistro per \(x\) che tende a \(c_1\) e
il limite destro per \(x\) che tende a \(c_2\), e diremo:
\[\lim_{x \to c_1^-} f(x) = 0 \sstext{e} \lim_{x \to c_2^+} f(x) = 0\]
}{
\begin{center} \limiteneb \end{center}
}

\begin{definizione}
Il numero reale \(L\) è il \textbf{limite sinistro} della funzione \(f(x)\) 
per \(x\) che tende a un valore \(c\) se \(f(x)\) è infinitamente vicino a 
\(L\) per tutti gli \(x\) infinitamente vicini a \(c\) e minori di 
\(c\). \quad 
E si scrive:
\[\lim_{x \to c^-} f(x) = L \stext{ se e solo se } 
\forall x \approx c \stext{ e } x < c, \quad f(x) \approx L\] \\
Dato che il numero reale infinitamente vicino ad un iperreale è proprio il 
risultato della parte standard, possiamo scrivere:
\[\lim_{x \to c^-} f(x) = \pst{f(c + \epsilon)} \]

\vspace{-1.5em}
\begin{center} 
se questa parte standard esiste e è sempre la stessa 
per ogni infinitesimo \(\epsilon < 0\).
\end{center}
\end{definizione}

\begin{definizione}
In modo simmetrico si definisce il \textbf{limite destro}
% Il numero reale \(L\) è il \textbf{limite sinistro} della funzione 
% \(f(x)\) per \(x\) che tende a un valore \(c\) se \(f(x)\) è 
% infinitamente vicino a \(L\) per tutti gli \(x\) infinitamente 
% vicini a \(c\) e minori di \(c\). \quad 
% E si scrive:
\[\lim_{x \to c^+} f(x) = L \stext{ se e solo se } 
\forall x \approx c \stext{ e } x > c, \quad f(x) \approx L\] 
Oppure:
% Dato che il numero reale infinitamente vicino ad un iperreale è 
% proprio il risultato della parte standard, possiamo scrivere:
\[\lim_{x \to c^+} f(x) = \pst{f(c + \epsilon)} \]

\vspace{-1.5em}
\begin{center} 
se questa parte standard esiste e è sempre la stessa 
per ogni infinitesimo \(\epsilon > 0\).
\end{center}
\end{definizione}

\begin{esempio}
Studia il comportamento della funzione \(f(x)=\sqrt{3x-12}\) 
vicino a \(4\).

\affiancati{.49}{.49}{
Per prima cosa studiamo l'\(\ID\) della funzione. 
L'argomento della radice quadrata deve essere maggiore o uguale a zero: 
\[3x-12 \geqslant 0 \sRarrow 3x \geqslant 12 \sRarrow x \geqslant 4\]
Quindi: \(\ID = \intervca{4}{\infty}\)

}{
\begin{center} \limitesqrt \end{center}
}

% Possiamo calcolare il valore della funzione in \(4\):
% \(f(4) = \sqrt{3 \cdot 4 - 12} = \sqrt{12 - 12} = \sqrt{0} = 0\). 
% La funzione è quindi definita solo a destra del punto \(\punto{4}{0}\).
% Quindi possiamo studiare solo il limite destro per \(x \to 2\):
Calcoliamo il valore della funzione per valori dell'ascissa infinitamente 
vicini a \(4\):
\[f(4 + \epsilon) = \sqrt{3\tonda{4+\epsilon}-12} = 
\sqrt{\cancel{+12}+3 \epsilon~\cancel{-12}} = 
\sqrt{3 \epsilon}\]
La radice è definita solo per valori positivi di \(\epsilon\), quindi 
non esiste il limite, ma possiamo calcolare il limite destro:
\[\lim_{x \to 4^+}\sqrt{3x-12} = \pst{\sqrt{3 \epsilon}} = 
\sqrt{\pst{3 \epsilon}} = \sqrt{0} = 0\]
Poiché il risultato non dipende dal particolare infinitesimo scelto, il 
valore ottenuto è il limite cercato.
\end{esempio}

\subsection{Se non esiste la parte standard}
\label{subsec:cont_limiti_nonsempreesiste}

La funzione parte standard è definita solo per valori finiti 
dell'argomento.
Può succedere che \(f(c + \epsilon)\) sia un numero infinito. 
Questa situazione può essere interessante, così 
interessante da meritare un ampliamento della definizione.

\subsubsection{Infinito: \(\infty\)}

Tutti i numeri reali sono finiti, quando vogliamo descrivere il comportamento 
di una funzione che cresce sempre di più dobbiamo utilizzare un simbolo che 
non fa parte dell'insieme dei reali. 

Il simbolo \(\infty\) non indica un numero, ma un comportamento.
Precisiamo il significato:

\begin{definizione}
Useremo:
\begin{itemize}
\item \(+\infty\) per indicare il comportamento di una funzione che cresce 
sempre di più, 
cioè che assume valori maggiori di \(m\) qualunque sia \(m \in \R\);
\item \(-\infty\) per indicare il comportamento di una funzione che 
diminuisce sempre di più, 
cioè che assume valori minori di \(m\) qualunque sia \(m \in \R\);
\item in alcuni libri con \(\infty\) si intende \(+\infty\); 
\item in altri libri con \(\infty\) si intende un comportamento di una 
funzione che, in valore assoluto diventa grande più di ogni numero reale.
\end{itemize}
\end{definizione}

Dato che non c'è un uso concorde del simbolo ``\(\infty\)'' in questo testo 
eviteremo di usarlo.

Possiamo ora ampliare la definizione di limite:

\begin{definizione}
Diremo che il \textbf{limite} della funzione \(f(x)\) 
per \(x\) che tende a un valore \(c\) è \(+\infty\) 
se per ogni \(x\) infinitamente vicini a \(c\) e diverso da 
\(c\), \(f(x)\) è un infinito positivo. \quad 
E si scrive:
\[\lim_{x \rightarrow c} f(x) = +\infty \sstext{se e solo se} 
% {f(c + \epsilon) = M > 0} \quad 
{f(x) \stext{è un infinito positivo}} \quad 
\forall x \approx c \stext{ e } x \neq c\]
Simmetricamente diremo che il \textbf{limite} della funzione \(f(x)\) 
per \(x\) che tende a un valore \(c\) è \(-\infty\)  
se per ogni \(x\) infinitamente vicini a \(c\) e diverso da 
\(c\), \(f(x)\) è un infinito negativo. \quad 
E si scrive:
\[\lim_{x \rightarrow c} f(x) = -\infty \sstext{se e solo se} 
% {f(c + \epsilon) = M > 0} \quad 
{f(x) \stext{è un infinito negativo}} \quad 
\forall x \approx c \stext{ e } x \neq c\]
\end{definizione}

Vediamo ora come applicare questa nuova definizione.

\begin{esempio}
Studia il comportamento della funzione \(f(x)=\dfrac{x+2}{x^2+8x+16}\) 
vicino a \(-4\).

\affiancati{.59}{.39}{
Per prima cosa calcoliamo il valore (iperreale) della funzione in un punto 
infinitamente vicino a \(-4\).
\begin{align*}
f(-4 + \epsilon) &= 
  \dfrac{\tonda{-4 +\epsilon}+2}
        {\tonda{-4 +\epsilon}^2+8\tonda{-4 +\epsilon}+16} =\\ 
&=\dfrac{-4 +\epsilon +2}
        {\cancel{16}~\cancel{-8\epsilon} +\epsilon^2~
         \cancel{-32}~\cancel{+8\epsilon}~\cancel{+16}} =\\ 
&=\dfrac{-2 +\epsilon}{\epsilon^2}
\end{align*}
Il valore ottenuto è un infinito perché è il rapporto tra un 
\emph{non infinitesimo} e un \emph{infinitesimo}.

Dato che per qualunque valore di \(\epsilon\) il numeratore è negativo e 
il denominatore è positivo, il risultato è un infinito negativo:
\[\lim_{x \to -4} \dfrac{x+2}{x^2+8x+16} = -\infty\]
}{
\begin{center} \limiteiesea \end{center}
}
\end{esempio}

\subsection{Se \(\pst{f(c+\epsilon)}\) varia al variare di \(\epsilon\)}
\label{subsec:cont_limiti_nonsempreesiste}

Può capitare che al variare del valore \(x\) infinitamente vicino a \(c\) la 
parte standard della funzione \(f(x)\) assuma diversi valori. Ovviamente in 
questo caso il limite non esiste.

\subsubsection{Una funzione interessante: \(\sen \dfrac{1}{x}\)}

\noindent Una funzione importante per esplorare situazioni strane attorno 
allo zero è: \(f(x) = \sen \dfrac{1}{x}\).\\
Più \(\abs{x}\) diventa grande, più \(\dfrac{1}{x}\) si avvicina a \(0\) e 
quindi anche \(\sen \dfrac{1}{x}\) si avvicina all'asse \(x\).

Ma la parte interessante della funzione è quella vicino allo \(0\):
man mano che la variabile \(x\) si avvicina a \(0\), \(\dfrac{1}{x}\) diventa 
sempre più grande e quindi \(\sen \dfrac{1}{x}\) oscilla in modo sempre più 
fitto tra \(-1 \stext{e} +1\).
Il grafico seguente si riferisce alla funzione \(\sen \dfrac{5}{x}\), in 
questo modo il grafico è un po' stirato nel senso della larghezza e si può 
distinguere qualche oscillazione in più.
\begin{center} \limitesinunosux \end{center}
% \begin{center} \limitesqrtsinunosux \end{center}
% \begin{center} \limitequadsqrtsinunosux \end{center} % non funziona!!!
Il comportamento della funzione, attorno allo zero, si mantiene, per 
transfer, anche quando \(x\) è infinitamente vicino a zero.
Quindi al variare di \(\epsilon\) la funzione continuerà a oscillare tra 
\(-1 \stext{e} +1\) perciò, al variare di \(\epsilon\) otterrò diversi valori 
di \(\pst{\sen \dfrac{1}{\epsilon}}\).

Quindi concludiamo che il 
\(\displaystyle \lim_{x \to 0}\sen \dfrac{1}{x}\) \emph{non esiste}.

\subsubsection{Limite sinistro e destro}
A volte possiamo trovarci in una situazione più fortunata:
Potrebbe succedere che 
\begin{itemize} [nosep]
\item 
per tutti i valori di \(x\) minori di \(c\) e infinitamente vicini a \(c\), 
\(f(x)\) sia infinitamente vicino al valore \(M\);
\item 
per tutti i valori di \(x\) maggiori di \(c\) e infinitamente vicini a \(c\), 
\(f(x)\) sia infinitamente vicino al valore \(N\). 
\end{itemize}

In questi casi possiamo dire che, pur non esistendo il limite, esiste un 
\emph{limite sinistro} o un \emph{limite destro}.

Indicheremo il limite sinistro con un ``meno'' posto a esponente 
(\(x \to c^-\)) 
e il limite destro con un ``più'' posto a esponente (\(x \to c^+\)).

Possiamo ampliare le definizioni, già date,  di limiti sinistro e destro 
tenendo conto anche dei limiti infiniti:

\begin{definizione}
Diremo che il \emph{limite sinistro} per \(x\) che tende a \(c\) 
della funzione \(f(x)\) è:
\noindent \begin{itemize}
\item 
\(\displaystyle \lim_{x \to c^-}f(x) = \pst{f(c+\epsilon)}\)
se \(f(c+\epsilon)\) è un numero finito e la sua parte standard \\
\hspace*{37.5mm} è sempre la stessa per ogni infinitesimo \(\epsilon < 0\).
\item 
\(\displaystyle \lim_{x \to c^-}f(x) = -\infty\)
se \(f(c+\epsilon)\) è un numero infinito negativo 
per ogni infinitesimo \(\epsilon  < 0\).
\item 
\(\displaystyle \lim_{x \to c^-}f(x) = +\infty\)
se \(f(c+\epsilon)\) è un numero infinito positivo 
per ogni infinitesimo \(\epsilon  < 0\).
\end{itemize}
\end{definizione}

\begin{definizione}
Diremo che il \emph{limite destro} per \(x\) che tende a \(c\) 
della funzione \(f(x)\) è:
\noindent \begin{itemize}
\item 
\(\displaystyle \lim_{x \to c^+}f(x) = \pst{f(c+\epsilon)}\)
se \(f(c+\epsilon)\) è un numero finito e la sua parte standard \\
\hspace*{37.5mm} è sempre la stessa per ogni infinitesimo \(\epsilon > 0\).
\item 
\(\displaystyle \lim_{x \to c^+}f(x) = -\infty\)
se \(f(c+\epsilon)\) è un numero infinito negativo 
per ogni infinitesimo \(\epsilon > 0\).
\item 
\(\displaystyle \lim_{x \to c^+}f(x) = +\infty\)
se \(f(c+\epsilon)\) è un numero infinito positivo 
per ogni infinitesimo \(\epsilon > 0\).
\end{itemize}
\end{definizione}

% \begin{definizione}
% Diremo che il \emph{limite sinistro} per \(x\) che tende a \(c\) 
% della funzione \(f(x)\) è:
% \noindent \begin{itemize}
% \item 
% \(\displaystyle \lim_{x \to c^-}f(x) = \pst{f(c+\epsilon)}\)
% se \(f(c+\epsilon)\) è un numero finito e \\
% per ogni \(\epsilon \approx 0 \stext{e} \epsilon < 0\)
% la parte standard di \(f(c+\epsilon)\) è sempre la stessa.
% \item 
% \(\displaystyle \lim_{x \to c^-}f(x) = +\infty\)
% se \(f(c+\epsilon)\) è un numero infinito positivo 
% per ogni \(\epsilon \approx 0 \stext{e} \epsilon < 0\).
% \item 
% \(\displaystyle \lim_{x \to c^-}f(x) = -\infty\)
% se \(f(c+\epsilon)\) è un numero infinito negativo 
% per ogni \(\epsilon \approx 0 \stext{e} \epsilon < 0\).
% \end{itemize}
% \end{definizione}

% \begin{definizione}
% Diremo che il \emph{limite destro} per \(x\) che tende a \(c\) 
% della funzione \(f(x)\) è:
% \begin{itemize}
% \item 
% \(\displaystyle \lim_{x \to c^+}f(x) = \pst{f(c+\epsilon)}\)
% se \(f(c+\epsilon)\) è un numero finito e \\
% per ogni \(\epsilon \approx 0 \stext{e} \epsilon > 0\)
% la parte standard di \(f(c+\epsilon)\) è sempre la stessa.
% \item 
% \(\displaystyle \lim_{x \to c^+}f(x) = +\infty\)
% se \(f(c+\epsilon)\) è un numero infinito positivo 
% per ogni \(\epsilon \approx 0 \stext{e} \epsilon > 0\).
% \item 
% \(\displaystyle \lim_{x \to c^+}f(x) = -\infty\)
% se \(f(c+\epsilon)\) è un numero infinito negativo 
% per ogni \(\epsilon \approx 0 \stext{e} \epsilon > 0\).
% \end{itemize}
% \end{definizione}
A questo punto possiamo dare una nuova definizione di limite:
\begin{definizione}
Se il limite sinistro e il limite destro hanno lo stesso valore \(L\), 
diremo che questo è il limite della funzione.
\[\lim_{x \to c}f(x) = L \stext{ se e solo se }~
\lim_{x \to c^-}f(x) = \lim_{x \to c^+}f(x) = L\]
\end{definizione}

\begin{esempio}
Calcola il limite \(\displaystyle \lim_{x \to 0}\frac{-1}{x}\)
\begin{enumerate} %[nosep]
\item 
Calcoliamo il valore della funzione nei punti infinitamente vicini a \(0\):
~~\(f(0+ \epsilon) = \dfrac{-1}{\epsilon}\)

Il valore ottenuto è un infinito.
\item 
Possiamo allora passare alla seconda parte della definizione per vedere se 
esiste un limite infinito \dots 
Niente da fare: dato che il segno dell'infinito \(\dfrac{-1}{\epsilon}\) 
è opposto a quello di \(\epsilon\), al variare di \(\epsilon\) la 
funzione assumerà valori infiniti di segno diverso.
\item 
Possiamo studiare se esistono almeno i limiti sinistro o destro:

per ogni \(\epsilon\) negativo l'infinito che 
otteniamo è positivo, per ogni \(\epsilon\) positivo l'infinito che 
otteniamo è negativo. Quindi:
\[\lim_{x \to 0^-}\frac{-1}{x} = +\infty \sstext{e} 
\lim_{x \to 0^+}\frac{-1}{x} = -\infty\]
\end{enumerate}
\end{esempio}


\subsection{Limiti all'infinito}
\label{subsec:cont_limiti_allinfinito}

% TODO Immagini con telescopi

Abbiamo studiato il comportamento di una funzione che 
% cresce sempre più all'a
assume valori infiniti quando il suo argomento si avvicina ad un certo 
valore finito. 
Ora vogliamo studiare come si comporta una funzione quando il suo argomento 
è un infinito e scriveremo: 
\[\lim_{x \to -\infty} f(x) \sstext{o} \lim_{x \to +\infty} f(x)\]
Diamo per il limite all'infinito una definizione analoga a quella 
data per i limiti al finito.

\begin{definizione}
\emph{Limite} della funzione \(f(x)\) per \(x\) che tende a \(-\infty\).
\begin{itemize}
\item 
\(\displaystyle \lim_{x \to -\infty}f(x) = \pst{f(M)}\)
se \(f(M)\) è un numero finito e \\
per ogni \(M\) infinito negativo
la parte standard di \(f(M)\) è sempre la stessa.
\item 
\(\displaystyle \lim_{x \to -\infty}f(x) = -\infty\)
se \(f(M)\) è un infinito negativo 
per ogni infinito \(M < 0\).
\item 
\(\displaystyle \lim_{x \to -\infty}f(x) = +\infty\)
se \(f(M)\) è un infinito positivo 
per ogni infinito \(M < 0\).
\end{itemize}
\end{definizione}

\begin{definizione}
\emph{Limite} della funzione \(f(x)\) per \(x\) che tende a \(+\infty\).
\begin{itemize}
\item 
\(\displaystyle \lim_{x \to +\infty}f(x) = \pst{f(M)}\)
se \(f(M)\) è un numero finito e \\
per ogni \(M\) infinito positivo
la parte standard di \(f(M)\) è sempre la stessa.
\item 
\(\displaystyle \lim_{x \to +\infty}f(x) = -\infty\)
se \(f(M)\) è un infinito negativo 
per ogni infinito \(M > 0\).
\item 
\(\displaystyle \lim_{x \to +\infty}f(x) = +\infty\)
se \(f(M)\) è un infinito positivo 
per ogni infinito \(M > 0\).
\end{itemize}
\end{definizione}

\begin{esempio}
Calcola i limiti: 
\(\displaystyle \lim_{x \to -\infty}\frac{3x^2-4x+5}{x^2-2x+3}\)
\quad e \quad
\(\displaystyle \lim_{x \to +\infty}\frac{3x^2-4x+5}{x^2-2x+3}\)

Iniziamo calcolando \(f(M)\) con \(M\) infinito negativo:
\[f(M) = \dfrac{3M^2-4M+5}{M^2-2M+3} \stackrel{1}{\sim} 
\dfrac{3\cancel{M^2}}{\cancel{M^2}} = 3\]
Dato che il risultato ottenuto è un finito:
\[\lim_{x \to +\infty}\frac{3x^2-4x+5}{x^2-2x+3} = 
  \pst{f(M)} = \pst{3} \stackrel{2}{=} 3\]
% \[\lim_{x \to -\infty}\frac{3x^2-4x+5}{x^2-2x+3} \stackrel{1}{=} 
% \pst{\frac{3M^2-4M+5}{M^2-2M+3}} \stackrel{2}{=} 
% \pst{\frac{3\cancel{M^2}}{\cancel{M^2}}} = \pst{3} = 3\]
Dove i passaggi indicati hanno le seguenti giustificazioni:
\begin{enumerate} [nosep]
\item Sostituiamo numeratore e denominatore con espressioni indistinguibili 
(vedi la sezione sul confronto tra numeri iperreali).
\item La parte standard di un numero reale è il numero reale stesso.
\end{enumerate}
Possiamo osservare che il risultato ottenuto sopra non dipende dal segno di 
\(M\) quindi anche: 
\[\displaystyle \lim_{x \to +\infty}\frac{3x^2-4x+5}{x^2-2x+3} = 3\]
\begin{center} \limiteallinfinito \end{center}
\end{esempio}

\subsection{Calcolare limiti}
\label{subsec:cont_limiti_calcolo}

In questa sezione vengono presentati alcuni limiti, risolti e commentati.
Rivediamo i procedimenti di calcolo.
\begin{procedura}
Per calcolare il limite di una funzione per \(x\) che tende a un 
valore \(c\) finito iniziamo calcolando:
\(f(c + \epsilon)\)
\begin{itemize}
\item 
se \(\pst{f(c + \epsilon)}\) esiste e è sempre la stessa 
per ogni \(\epsilon \neq 0\): \quad 
\(\displaystyle \lim_{x \to c}f(x) = \pst{f(c + \epsilon)}\)
\item 
se \(f(c + \epsilon)\) è un infinito negativo per ogni \(\epsilon \neq 0\): 
\quad \(\displaystyle \lim_{x \to c}f(x) = -\infty\)
\item 
se \(f(c + \epsilon)\) è un infinito positivo per ogni \(\epsilon \neq 0\): 
\quad \(\displaystyle \lim_{x \to c}f(x) = +\infty\)
\item 
se \(f(c + \epsilon)\) ha un comportamento diverso a seconda del 
segno di \(\epsilon\), possiamo cercare i limiti sinistro o destro
restringendo i calcoli precedenti rispettivamente ai valori di 
\(\epsilon < 0\) e \(\epsilon > 0\). 
\end{itemize}
\end{procedura}

\begin{procedura}
Per calcolare il limite di una funzione per 
\(x \to -\infty\)
iniziamo calcolando:
\(f(M)\) con l'infinito \(M < 0\)
\begin{itemize}
\item 
se \(\pst{f(M)}\) esiste e è sempre la stessa per ogni \(M\): \quad 
\(\displaystyle \lim_{x \to -\infty}f(x) = \pst{f(M)}\)
\item 
se \(f(M)\) è un infinito negativo per ogni \(M\): 
\quad \(\displaystyle \lim_{x \to -\infty}f(x) = -\infty\)
\item 
se \(f(M)\) è un infinito positivo per ogni \(M\): 
\quad \(\displaystyle \lim_{x \to -\infty}f(x) = +\infty\)
\end{itemize}
\end{procedura}

\begin{procedura}
Per calcolare il limite di una funzione per 
\(x \to +\infty\)
iniziamo calcolando:
\(f(M)\) con l'infinito \(M > 0\)
\begin{itemize}
\item 
se \(\pst{f(M)}\) esiste e è sempre la stessa per ogni \(M\): \quad 
\(\displaystyle \lim_{x \to +\infty}f(x) = \pst{f(M)}\)
\item 
se \(f(M)\) è un infinito negativo per ogni \(M\): 
\quad \(\displaystyle \lim_{x \to +\infty}f(x) = -\infty\)
\item 
se \(f(M)\) è un infinito positivo per ogni \(M\): 
\quad \(\displaystyle \lim_{x \to +\infty}f(x) = +\infty\)
\end{itemize}
\end{procedura}

Nei prossimi esempi applichiamo le procedure per il calcolo del limite 
ai diversi casi che potrai incontrare.

\begin{esempio}
\textbf{Limite per \(x \to + \infty\)}

Studia il comportamento di 
\(\dfrac{3x^2-3x+7}{5x^2-6}\) per \(x \to +\infty\).\\
Calcoliamo \(f(x)\) in un generico infinito positivo~\(M\).
\[
f(M)= 
  \frac{3M^2-3M+7}{5M^2-6} \stackrel{1}{\sim} 
  \frac{3\cancel{M^2}}{5\cancel{M^2}} = \frac{3}{5}
\]
Dato che abbiamo ottenuto un valore finito:
\[\lim_{x \to +\infty}\frac{3x^2-3x+7}{5x^2-6} = 
  \pst{f(M)} = \pst{\frac{3}{5}} \stackrel{2}{=} \frac{3}{5}\]
Poiché la parte standard esiste e è sempre la stessa per ogni 
infinito \(M > 0\). 

Dove i passaggi hanno i seguenti motivi:
\begin{enumerate} [nosep]
\item Sostituiamo l'espressione ottenuta con una espressione 
  indistinguibile;
\item La parte standard di un valore reale è il reale stesso.
\end{enumerate}
\end{esempio}

\begin{esempio}
\textbf{Limite per \(x \to c\)}

Studia il comportamento di \(f(x) = x^2-4x+2\) vicino a \(+5\).\\
Calcoliamo \(f(x)\) in \(5+\epsilon\).
\[
f(5+\epsilon)=\tonda{5+\epsilon}^2-4\tonda{5+\epsilon}+2 = 
  25 + 10 \epsilon + \epsilon^2 - 15 - 4\epsilon + 2 =
  12 + 6 \epsilon +\epsilon^2
\]
Dato che abbiamo ottenuto un valore finito:
\[\lim_{x \to +5}f(x) = 
  \pst{f(5+\epsilon)} = \pst{12 + 6 \epsilon +\epsilon^2} \stackrel{1}{=} 
  \pst{12} + \pst{6 \epsilon} + \pst{\epsilon^2} \stackrel{2}{=} 12\]
Poiché la parte standard esiste e è sempre la stessa per ogni 
infinitesimo \(\epsilon \neq 0\). 

Dove i passaggi hanno i seguenti motivi:
\begin{enumerate} [nosep]
\item La parte standard di una somma è uguale alla somma delle parti 
standard;
\item La parte standard di un infinitesimo è zero.
\end{enumerate}
\end{esempio}

\begin{esempio}
\textbf{Metodo rapido}, utilizzabile nelle funzioni polinomiali. \\
Riprendiamo l'esercizio precedente ma applichiamo la relazione di 
indistinguibilità prima di effettuare calcoli:
\[
f(5+\epsilon)=
\tonda{5+\epsilon}^2-4\tonda{5+\epsilon}+2 \stackrel{1}{\sim} 
  \tonda{5}^2-4\tonda{5}+2 = 25 - 15 + 2 = 12
\]
Dato che abbiamo ottenuto un valore finito:
\[\lim_{x \to +5}f(x) = 
  \pst{f(5+\epsilon)} = \pst{12} \stackrel{2}{=} 12\]
Poiché la parte standard esiste e è sempre la stessa per ogni 
infinitesimo \(\epsilon \neq 0\). 

\begin{enumerate} [nosep]
 \item Sostituiamo l'espressione con una espressione indistinguibile;
 \item La parte standard di un valore reale è il reale stesso.
\end{enumerate}
\end{esempio}

\begin{esempio}
\textbf{Limite per \(x \to 0\)}

Studia il comportamento di \(\dfrac{x^2-4x+2}{x^2-4} \) vicino allo zero.\\
Calcoliamo il valore della funzione in un generico punto infinitamente 
vicino allo zero:
\[
f(0 + \epsilon) = f(\epsilon) = 
  \frac{\epsilon^2-4\epsilon+2}{\epsilon^2-4}
\]
Dato che abbiamo ottenuto un valore finito (il rapporto tra due finiti non 
infinitesimi:
\[\lim_{x \to 0}f(x) = 
  \pst{f(\epsilon)} = \pst{\frac{\epsilon^2-4\epsilon+2}{\epsilon^2-4}} 
\stackrel{1}{=} \frac{\pst{\epsilon^2-4\epsilon+2}}{\pst{\epsilon^2-4}} 
\stackrel{2}{=} 
  \frac{\pst{\epsilon^2}-\pst{4\epsilon}+\pst{2}}
  {\pst{\epsilon^2}-\pst{4}} =
\frac{+2}{-4} = -\frac{1}{2}\]
Poiché la parte standard esiste e è sempre la stessa per ogni 
infinitesimo \(\epsilon \neq 0\). 

Dove i passaggi hanno i seguenti motivi:
\begin{enumerate} [nosep]
 \item La parte standard di un quoziente è uguale al quoziente delle parti 
standard se il denominatore non è un infinitesimo;
 \item La parte standard di una somma è uguale alla somma delle parti standard.
\end{enumerate}
\end{esempio}

Dato che il valore della funzione che calcoliamo ci serve per ottenere, 
eventualmente, la parte standard, possiamo utilizzare la relazione di 
indistinguibilità consumando un po' meno inchiostro:

\begin{esempio}
\textbf{Limite per \(x \to 0\)}

Studia il comportamento di \(\dfrac{x^2-4x+2}{x^2-4} \) vicino allo zero.\\
Calcoliamo il valore della funzione in un generico punto infinitamente 
vicino allo zero:
\[
f(0 + \epsilon) = f(\epsilon) = 
  \frac{\epsilon^2-4\epsilon+2}{\epsilon^2-4} \stackrel{1}{\sim} 
  \frac{+2}{-4}
\]
Dato che abbiamo ottenuto un valore finito (il rapporto tra due finiti non 
infinitesimi:
\[\lim_{x \to 0}f(x) = 
  \pst{f(\epsilon)} = \pst{\frac{+2}{-4}} \stackrel{2}{=} 
  -\frac{1}{2}\]
Poiché la parte standard esiste e è sempre la stessa per ogni 
infinitesimo \(\epsilon \neq 0\). 

Dove i passaggi segnati hanno i seguenti motivi:
\begin{enumerate} [nosep]
 \item Applico la relazione di indistinguibilità;
 \item La parte standard di un reale è il reale stesso.
\end{enumerate}
\end{esempio}

Ora vediamo alcuni casi un po' più delicati: ricordiamoci che un numero 
iperreale diverso da zero non può mai essere dichiarato indistinguibile 
da zero.

\begin{esempio}
\textbf{Infinitesimo fratto finito non infinitesimo 
\(\tonda{\frac{i}{fni}}\).}

Studia il comportamento di \(\dfrac{x+5}{x-3}\) vicino a \(-5\).
\[
f(-5 + \epsilon) = \frac{-5+\epsilon+5}{-5+\epsilon-3} = 
  \frac{\epsilon}{-8+\epsilon} \stackrel{1}{\sim} 
  \frac{\epsilon}{-8} 
\]
Abbiamo ottenuto un valore finito quindi:
\[
\lim_{x \to -5} f(x) =
  \pst{\frac{\epsilon}{-8}} \stackrel{2}{=} 
  \pst{\delta} \stackrel{3}{=} 0
\]
Dove i passaggi hanno i seguenti motivi:
\begin{enumerate} [nosep]
 \item Sostituiamo l'espressione ottenuta con una espressione 
   indistinguibile.
 \item Un infinitesimo diviso un finito non infinitesimo dà come risultato 
un infinitesimo.
 \item La parte standard di un infinitesimo è zero.
\end{enumerate}
\end{esempio}

\begin{esempio}
\textbf{Finito non infinitesimo fratto infinitesimo 
\(\tonda{\frac{fni}{i}}\).}

Studia il comportamento di \(\dfrac{-2x+4}{x-4}\) vicino a \(+4\).
\[
f(4 +\epsilon) = \frac{-2\tonda{4+\epsilon}+4}{4+\epsilon-4} =
  \frac{-4 - 2 \epsilon}{\epsilon} \sim \frac{-4}{\epsilon} 
\]
Il risultato ottenuto non è un finito quindi non possiamo applicare la 
funzione parte standard.

Il segno dell'infinito ottenuto dipende dal segno di \(\epsilon\) quindi non 
esiste il limite, possiamo però calcolare i limiti sinistro e destro.
Dato che il rapporto \(\frac{-4}{\epsilon}\) ha segno opposto a quello di
\(\epsilon\):
\[
\lim_{x \to +4^-} f(x) = +\infty \sstext{e} 
\lim_{x \to +4^+} f(x) = -\infty
\]
\end{esempio}

\begin{esempio}
\textbf{Funzione irrazionale, con radici quadrate}.

Studia il comportamento di \(2x-\sqrt{4x^2-8x+3}\) per \(x \to +\infty\).
\[
f(M) = 2M-\sqrt{4M^2-8M+3} \stackrel{1}{\sim} 
  2M-\sqrt{4M^2} =
  2M-2M = 0
\]
Si dovrebbe concludere che:
\[\lim_{x \to +\infty} f(x) = 0\]
\begin{minipage}{.59\textwidth}
Ma questa volta nei ragionamenti fatti c'è un errore. Proviamo a calcolare 
la funzione per alcuni valori abbastanza grandi di \(x\).\\
I risultati dovrebbero avvicinarsi a zero, ma non è così, sembra si 
avvicinino, invece, a due. Dove abbiamo sbagliato?
\end{minipage}
\begin{minipage}{.39\textwidth}
\begin{center}
\begin{tabular}{r|r}
x & y\\\hline
100 & 2.00252 \\
1000 & 2.00025 \\
10000 & 2.00002 \\
\end{tabular}
\end{center}
\end{minipage}\\

Nel passaggio (1) abbiamo usato in modo improprio la relazione 
\emph{indistinguibile}: 
abbiamo ottenuto un'espressione indistinguibile da zero, ma ciò non è 
possibile (vedi la definizione di indistinguibile) \dots \\
Dobbiamo seguire un'altra strada.

Consideriamo l'espressione data come una frazione e razionalizziamo il 
numeratore:
\begin{align*}
f(M) &= \frac{\tonda{2M}-\sqrt{4M^2-8M+3}}{1} \cdot 
\frac{\tonda{2M}+\sqrt{4M^2-8M+3}}{\tonda{2M}+\sqrt{4M^2-8M+3}}=\\
&=\frac{\cancel{4M^2}-\cancel{4M^2}+8M-3}{2M+\sqrt{4M^2-8M+3}} \sim
\frac{8M}{2M+\sqrt{4M^2}} = \frac{8M}{2M+2M}=
   \frac{8\cancel{M}}{4\cancel{M}} = 2
\end{align*}
Dato che il valore ottenuto è finito e non dipende dal particolare infinito 
\(M > 0\):
\[\lim_{x \to +\infty} f(x) = \pst{2} = 2\]
In accordo con l'andamento della funzione calcolata in alcuni opportuni 
valori.
\end{esempio}

\begin{esempio}
\textbf{Funzione razionale}: metodo rapido ma non sempre funzionante.

Studia il comportamento di \(\dfrac{x^3+3x^2+2x}{x^2-x-6}\) per 
\(x \to +2\).
\[
f(2 + \epsilon) = \frac
  {\tonda{2+\epsilon}^3+3\tonda{2+\epsilon}^2+2\tonda{2+\epsilon}}
  {\tonda{2+\epsilon}^2-\tonda{2+\epsilon}-6} \stackrel{1}{\sim} 
  \frac{2^3+3\cdot 2^2+2\cdot 2}{2^2-2-6} =
  \frac{8+12+4}{4-2-6} = \frac{24}{-4} = -6
\]
Dato che il valore ottenuto è finito e non dipende dal particolare 
infinitesimo \(\epsilon\):
\[\lim_{x \to +2} f(x) = \pst{-6} \stackrel{2}{=} -6\]
Dove i passaggi hanno i seguenti motivi:
\begin{enumerate} [nosep]
 \item sostituiamo le espressioni tra parentesi con espressioni 
indistinguibili.
 \item La parte standard di un reale è il reale stesso.
\end{enumerate}
\end{esempio}

Ora vediamo un caso in cui il metodo precedente non funziona. Prima di 
affrontare l'esempio ricordiamoci che se non abbiamo maggiori informazioni 
sugli infinitesimi \(\alpha\) e \(\beta\), non possiamo calcolare 
\(\frac{\alpha}{\beta}\).

\begin{esempio}
\textbf{Funzione razionale}: metodo rapido ma inconcludente.

Studia il comportamento di \(\dfrac{x^3+3x^2+2x}{x^2-x-6}\) per 
\(x \to -2\).
\[
f(-2 + \epsilon) =\frac
  {\tonda{2+\epsilon}^3+3\tonda{2+\epsilon}^2+2\tonda{2+\epsilon}}
  {\tonda{2+\epsilon}^2-\tonda{2+\epsilon}-6} \stackrel{1}{\sim} 
  \frac
  {\tonda{-2}^3+3\tonda{-2}^2+2\tonda{-2}}
  {\tonda{-2}^2-\tonda{-2}-6} = 
  \frac{-8+12-4}{4+2-6} = \frac{0}{0}
\]
Ma qui ci scontriamo con 2 problemi: abbiamo usato la relazione di 
indistinguibile con lo zero, e abbiamo ottenuto una divisione per zero che 
non è definita. Potremmo essere un po' più pignoli ma ancora troppo 
grossolani osservando che quando \(x\) è infinitamente vicino a \(-2\) il 
numeratore e il denominatore sono infinitamente vicini a zero, sono cioè 
degli infinitesimi e quindi otteniamo: \(\frac{\alpha}{\beta}\), ma anche 
questa maggior precisione non è sufficiente.

\end{esempio}

\begin{esempio}
\textbf{Funzione razionale}: Metodo lungo ma sicuro.

Studia il comportamento di \(\dfrac{x^3+3x^2+2x}{x^2-x-6}\) per 
\(x \to -2\).

Dobbiamo rimboccarci le maniche e affrontare il calcolo algebrico 
(ricordandoci che: \(\tonda{x+a}^3= x^3+3x^2a+3xa^2+a^3\))

\begin{align*}
f(-2+\epsilon)& = \frac
  {\tonda{-2+\epsilon}^3+3\tonda{-2+\epsilon}^2+2\tonda{-2+\epsilon}}
  {\tonda{-2+\epsilon}^2-\tonda{-2+\epsilon}-6} =\\
  &=\frac{-8+12\epsilon-6\epsilon^2+\epsilon^3+
             3\tonda{4-4\epsilon+\epsilon^2}-4+2\epsilon}
             {4-4\epsilon+\epsilon^2-\tonda{-2+\epsilon}-6} =\\ 
  &=\frac{\cancel{-8}+\cancel{12\epsilon}-6\epsilon^2+\epsilon^3+
          \cancel{12}\cancel{-12\epsilon}+3\epsilon^2\cancel{-4}+2\epsilon}
             {\cancel{4}-4\epsilon+
              \epsilon^2\cancel{+2}-\epsilon\cancel{-6}}=\\ 
  &=\frac{\epsilon^3+2\epsilon}{\epsilon^2-5\epsilon} = 
    \frac{\cancel{\epsilon} \tonda{\epsilon^2+2}}
             {\cancel{\epsilon} \tonda{\epsilon-5}} \stackrel{1}{\sim} 
    \frac{+2}{-5}
\end{align*}
Dato che il valore ottenuto è finito e non dipende dal particolare 
infinitesimo \(\epsilon\):
\[
\lim_{x \rightarrow -2} f(x) = \pst{\frac{+2}{-5}} \stackrel{2}{=} 
  -\frac{2}{5}
\]
Dove i passaggi hanno i seguenti motivi:
\begin{enumerate} [nosep]
 \item Sostituiamo l'espressione con una indistinguibile.
 \item La parte standard di un reale è il reale stesso.
\end{enumerate}
\end{esempio}

\begin{osservazione}
Pensate alla complicazione dei calcoli se ci fosse un qualche \(x^4\) o
\(x^5\)\dots
Vedremo ora un altro modo di calcolare il limite che risulta meno 
complicato del precedente, ma ugualmente sicuro.
\end{osservazione}\\

Prima di affrontare il prossimo metodo ricordiamoci che possiamo 
rappresentare tutti gli infinitesimi di ordine superiore ad un certo 
infinitesimo \(\epsilon\) con il simbolo: \(o\tonda{\epsilon}\).

\begin{esempio}
\textbf{Funzione razionale}: altro metodo.

Studia il comportamento di \(\dfrac{x^3+3x^2+2x}{x^2-x-6}\) per 
\(x \to -2\).

\begin{align*}
f(-2+\epsilon)& = \frac
  {\tonda{-2+\epsilon}^3+3\tonda{-2+\epsilon}^2+2\tonda{-2+\epsilon}}
  {\tonda{-2+\epsilon}^2-\tonda{-2+\epsilon}-6} \stackrel{1}{=}\\ 
  &\stackrel{2}{=}
  \frac{-8+12\epsilon+12-12\epsilon-4+2\epsilon+o\tonda{\epsilon}}
       {4-4\epsilon+2-\epsilon-6+o\tonda{\epsilon}}=\\ 
  &=\frac{\cancel{-8}\,\cancel{+12\epsilon}\,
               \cancel{+12}\,\cancel{-12\epsilon}\,\cancel{-4}+2\epsilon+
               o\tonda{\epsilon}}
              {\cancel{4}-4\epsilon\,\cancel{+2}-\epsilon\,\cancel{-6}+
               o\tonda{\epsilon}}\stackrel{2}{\sim}
    \frac{+2\cancel{\epsilon}}{-5\cancel{\epsilon}} = -\frac{2}{5}
\end{align*}
Dato che il valore ottenuto è finito e non dipende dal particolare 
infinitesimo \(\epsilon\):
\[
\lim_{x \rightarrow -2} f(x) = \pst{\frac{+2}{-5}} \stackrel{3}{=} 
  -\frac{2}{5}
\]
Dove i passaggi hanno i seguenti motivi:
\begin{enumerate} [nosep]
 \item Eseguiamo i calcoli riunendo in un unico simbolo la somma di tutti 
 gli infinitesimi di ordine superiore a \(\epsilon\) e perciò trascurabili 
(si spera);
 \item Sostituiamo l'espressione con una indistinguibile.
 \item La parte standard di un reale è il reale stesso.
\end{enumerate}
\end{esempio}

\begin{osservazione}
Se si annullassero anche tutti i termini contenenti \(\epsilon\), dovremmo 
esplicitare anche \(\epsilon^2\) raccogliendo in \(o\tonda{\epsilon^2}\) 
gli infinitesimi di ordine superiore a \(\epsilon^2\). 
\end{osservazione}

\subsection{Limiti notevoli}
\label{subsec:cont_limiti_notevoli}

Ci sono alcuni limiti che hanno delle dimostrazioni particolari e che è 
utile conoscere per poter risolvere dei casi particolari.

\subsubsection{Funzioni goniometriche}

Se consideriamo le funzioni seno, coseno e tangente, possiamo vedere che 
alcuni limiti risultano banali, ma altri sono interessanti e possono essere 
dimostrati.

\begin{minipage}{.66\textwidth}
\begin{center} \sinusoide \end{center}
% \begin{center} \sincos{sin(x)}{$y=\sen x$}{Blue} \end{center}
\begin{center} \cosinusoide \end{center}
\end{minipage}
\hfill
\begin{minipage}{.33\textwidth}
\begin{center} \tangentoide \end{center}
\end{minipage}

\paragraph{Funzione seno}~
\label{limiti:par_f_seno}

Osservando il grafico della funzione seno, risulta abbastanza evidente che:
\[\lim_{x \rightarrow 0}{\sen x} = \pst{\sen \delta} = 0
\quad \text{e che} \quad 
\lim_{x \rightarrow \infty}{\sen x} = \pst{\sen M} = \text{ non esiste}\]
Infatti man mano che \(x\) aumenta, \(\sen x\)
continua a variare tra \(-1\) e \(+1\) e quindi 
\(\sen M\) non può essere infinitamente vicino ad un solo numero reale.\\

Per esaminare il comportamento della funzione \(sen x\) quando l'arco (e 
l'angolo \(\delta\)) si avvicina a zero, si può ripercorrere il geniale 
ragionamento di Archimede, che lo portò a determinare un valore di \(\pi\) 
poco diverso da quello odierno. Archimede immaginò la circonferenza racchiusa 
fra poligono regolare inscritto e uno circoscritto e determinò la lunghezza del 
lato in base al diametro. Raddoppiando progressivamente i lati (fino a 96 
lati!) e aggiornando i calcoli, si accorse che il rapporto fra i perimetri e i 
diametri si assestava progressivamente intorno al valore 
\(\frac{223}{71}\simeq 3,141\). Dato che la circonferenza è intermedia fra il 
poligono inscritto e quello circoscritto, questo numero, che noi chiamiamo 
\(\pi\), ci indica il rapporto fra una circonferenza e il 
suo diametro. Questo vale per tutti i cerchi, di qualsiasi raggio.\\

\affiancati{.50}{.50}{
\disegno{\polregincirc{4}{6}}
}{
\disegno{\polregincirc{4}{12}}
}

Se sfruttiamo il ragionamento di Archimede in modo non standard, possiamo dire 
che la misura della circonferenza è infinitamente vicina al perimetro dei 
poligoni di infiniti lati, a lei inscritti e circoscritti.\\
Isoliamo uno dei lati infinitesimi del poligono inscritto. La 
metà\footnote{Pare infatti che il nome della funzione \emph{seno} derivi da 
un'errata traduzione latina (\emph{sinus}) di un termine arabo indicante una 
\emph{mezza corda}.} di questo lato, \(\dfrac{l}{2}\), corrisponde graficamente 
al seno 
dell'angolo al centro infinitesimo \(sen \delta\):
\[sen \delta \approx \dfrac{1}{2}\dfrac{2\pi r}{N}\approx \dfrac{2p}{N}=
\dfrac{1}{2}\dfrac{Nl}{N}\quad \rightarrow\quad
\sin \delta \approx\dfrac{l}{2},\quad \text{con } N \text{ ipernaturale 
infinito.}  \]

Poiché stiamo parlando di un poligono inscritto di infiniti lati, con 
raggio fissato, \(l\) è necessariamente infinitesimo, come l'arco e l'angolo 
\(\delta\), e concludiamo: \emph{l'arco infinitesimo, che corrisponde a  
\(\delta\), e \(\sen\delta\) sono indistinguibili}.

Tutto questo porta facilmente a dimostrare il seguente limite notevole:
\[\lim_{x \rightarrow 0}\frac{\sen x}{x} = 
  \pst{\frac{\sen \delta}{\delta}} = 1\]
  
Possiamo anche rappresentarlo graficamente.

\begin{minipage}{.49\textwidth}
L'arco infinitesimo differisce dalla corda sottesa per infinitesimi di ordine 
superiore alle loro lunghezze. Perciò il loro rapporto è \(1\).
Se rappresentiamo in una circonferenza goniometrica un arco di ampiezza 
infinitesima e il corrispondente seno, appaiono non distinguibili ad ogni 
ingrandimento finito. 
Se li ingrandiamo con un microscopio non standard che permetta di 
osservare \(\delta\), possiamo vedere che sono ancora indistinguibili 
cioè differiscono per infinitesimi di ordine superiore. 

Quindi il loro rapporto è \(1\).
\end{minipage}
\hfill
\begin{minipage}{.49\textwidth}
\begin{center} \limiteseno \end{center}
\end{minipage}

Si coinclude che \(\sen \delta \text{ è diverso da } \delta\) solo per 
infinitesimi di ordine superiore a \(\delta\).

\paragraph{Funzione tangente}~

Anche nella funzione tangente valgono i seguenti limiti:
\[\lim_{x \rightarrow 0}{\tan x} = \pst{\tan \delta} = 0; \quad
\lim_{x \rightarrow \infty}{\tan x} = \pst{\tan M} = 
                                      \text{ non esiste}; \quad
\lim_{x \rightarrow 0}\frac{\tan x}{x} = \pst{\frac{\tan \delta}{\delta}} 
                                       = 1\]
per considerazioni analoghe a quelle fatte per la funzione seno.

\paragraph{Funzione coseno}~

Per la funzione coseno valgono i seguenti limiti:
\[\lim_{x \rightarrow 0}{\cos x} = \pst{\cos \delta} = 1; ~
\lim_{x \rightarrow 0}\tonda{1-\cos x} = \pst{1-\cos \delta} = 0; ~
\lim_{x \rightarrow \infty}{\cos x} = \pst{\cos M} = \text{ non es.}\]

Sono interessanti i seguenti due limiti che riguardano il coseno:
\paragraph{Confrontare \(1-\cos\delta\) e \(\delta\)}~
\label{senoverso1}

\begin{align*}
 \lim_{x \rightarrow 0} \frac{1-\cos x}{x} &=
 \pst{\frac{1-\cos \delta}{\delta}}
~ \stackrel{1}{=} ~  
 \pst{\frac{1-\cos \delta}{\delta} \cdot 
      \frac{1+\cos \delta}{1+\cos \delta}}
~ \stackrel{2}{=} ~ 
 \pst{\frac{1-\cos^2 \delta}{\delta \tonda{1+\cos \delta}}}~ 
\stackrel{3}{=} \\
& \stackrel{3}{=} ~
 \pst{\frac{\sen^2 \delta}{\delta \tonda{1+\cos \delta}}}
~ \stackrel{4}{=}
 \pst{\frac{\sen \delta}{\delta} \cdot 
      \frac{\sen \delta}{1+\cos \delta}}
~ \stackrel{5}{=}
 \pst{1 \cdot \frac{\delta}{2}} = 0
\end{align*}
Dove i passaggi hanno i seguenti motivi:
\begin{enumerate} [nosep]
 \item moltiplico la funzione per una frazione equivalente a 1;
 \item prodotto notevole;
 \item ricordando che \(\sen^2 x + \cos^2 x = 1\) e quindi: 
 \(\sen^2 x = 1 - \cos^2 x\);
 \item per quanto visto sopra: \(\frac{\sen \delta}{\delta} = 1\).
 \item un infinitesimo fratto un non infinitesimo è un infinitesimo 
e la sua parte standard è 0.
\end{enumerate}
Questo limite ci dice in sostanza che la quantità \(1- \cos \delta\) è 
infinitesima rispetto a \(\delta\).
\begin{osservazione}
Dato che \(\sen \delta \sim \delta\), i seguenti due limiti sono 
equivalenti:
\[\lim_{x \rightarrow 0} \frac{1-\cos x}{x} =
 \pst{\frac{1-\cos \delta}{\delta}} =
 \pst{\frac{1-\cos \delta}{\sen \delta}} =
 \lim_{x \rightarrow 0} \frac{1-\cos x}{\sen x}\]
\end{osservazione}

\paragraph{Confrontare \(1-\cos\delta\) e \(\delta^2\)}~
\label{senoverso2}

L'altro limite notevole, che si dimostra in modo analogo a quello 
precedente, è:
\begin{align*}
 \lim_{x \rightarrow 0} \frac{1-\cos x}{x^2} &=
 \pst{\frac{1-\cos \delta}{\delta^2}}
~ \stackrel{1}{=} ~  
 \pst{\frac{1-\cos \delta}{\delta^2} \cdot 
      \frac{1+\cos \delta}{1+\cos \delta}}
~ \stackrel{2}{=} ~ 
 \pst{\frac{1-\cos^2 \delta}{\delta^2 \tonda{1+\cos \delta}}}~ 
\stackrel{3}{=} \\
& \stackrel{3}{=} ~
 \pst{\frac{\sen^2 \delta}{\delta^2 \tonda{1+\cos \delta}}}
~ \stackrel{4}{=}
 \pst{\tonda{\frac{\sen \delta}{\delta}}^2 \cdot 
      \frac{1}{\tonda{1+\cos \delta}}}
~ \stackrel{5}{=}
 \pst{1 \cdot \frac{1}{2}} = \frac{1}{2}
\end{align*}
Dove i passaggi hanno i seguenti motivi:
\begin{enumerate} [nosep]
 \item moltiplico la funzione per una frazione equivalente a 1;
 \item prodotto notevole;
 \item ricordando che \(\sen^x + \cos^2 x = 1\);
 \item riscrivo l'espressione in un modo più comodo;
 \item ricordando i limiti visti precedentemente.
\end{enumerate}

\newpage
\paragraph{Da un libro di 300 anni fa}\hspace{-3mm}\footnote{
Maria Gaetana Agnesi (1718-1799) fu un'insigne matematica milanese. Nel 1748 
pubblicò le \emph{Instituzioni analitiche ad uso della gioventù italiana}, un 
manuale di matematica che per chiarezza e profondità si diffuse in tutta 
Europa. 
Il disegno è tratto dall'originale.}~

\affiancati{.59}{.39}{
\agnesia}{
Per intuire la portata di questi due ultimi limiti recuperiamo il 
brillante ragionamento di M.G.Agnesi, che immagina che l'angolo \(\delta\) e 
quindi l'arco \(CD\) siano infinitesimi. Il diametro \(GC\) è una 
quantità finita ed è base del triangolo rettangolo \(GCD\), quindi, per il 2° 
Teorema di Euclide:\\
\(GB:BD=BD:BC\).}
\(\dfrac{GB}{BD}\) è il rapporto fra una quantità finita ed una infinitesima 
ed è quindi un infinito. Allora deve essere altrettanto infinito il secondo 
rapporto di questa proporzione. La conseguenza di tutto questo è che il 
segmento \(BC= AC-AB=1-\cos\delta\) è infinitesimo rispetto a 
\(BD=\sen \delta\) ed è quindi infinitesimo del secondo ordine. Questo 
giustifica il valore finito del limite 
\(\lim_{x \rightarrow 0} \dfrac{1-\cos x}{x^2}\).


\subsubsection{Esponenziali e logaritmi}

\paragraph{Numero di Eulero (o di Nepero)}

Ricordiamo come è definita la costante di Eulero:
\[e = \pst{\tonda{1+\frac{1}{M}}^M} = 
\pst{\tonda{1+\delta}^\frac{1}{\delta}} 
\]
\[e=\sum_{n=0}^{\infty}{\frac{1}{n!}}=
\frac{1}{1}+\frac{1}{1}+\frac{1}{2}+\frac{1}{2\cdot3}+
\frac{1}{2\cdot3\cdot4}+\frac{1}{2\cdot3\cdot4\cdot5}+\dots\]
Le definizioni sono equivalenti, mentre la seconda definizione risulta 
molto efficiente per il calcolo, la 
prima ha delle interessanti applicazioni matematiche.

\begin{esempio}
\label{esempio:log}
Limite di una particolare funzione logaritmica:
\[
 \lim_{x \rightarrow 0} \dfrac{\ln\tonda{1+x}}{x} =
 \pst{\dfrac{\ln\tonda{1+\delta}}{\delta}}~ = ~  
 \pst{\frac{1}{\delta}\ln\tonda{1+\delta}} ~\stackrel{1}{=} ~
 \pst{\ln\tonda{1+\delta}^\frac{1}{\delta}}
~ \stackrel{2}{=} ~
\pst{\ln\tonda{e}} \stackrel{3}{=} ~ 1
\]
Dove i passaggi hanno i seguenti motivi:
\begin{enumerate} [nosep]
 \item per la proprietà dei logaritmi: prodotto di una funzione per 
 un logaritmo;
 \item l'argomento del logaritmo è proprio la definizione di \(e\);
 \item per la definizione di logaritmo: l'esponente da dare a \(e\) per 
ottenere \(e\) è \(1\).
\end{enumerate}
\end{esempio}

\begin{esempio}
Come nell'esempio precedente ma con una base generica:
\[
 \lim_{x \rightarrow 0} \dfrac{\log_a\tonda{1+x}}{x} =
 \pst{\dfrac{\log_a\tonda{1+\delta}}{\delta}}~\stackrel{1}{=} ~  
 \pst{\dfrac{\frac{\ln\tonda{1+\delta}}{\ln a}}{\delta}}~= ~
 \pst{\dfrac{\ln\tonda{1+\delta}}{\delta}\cdot \dfrac{1}{\ln a}}
 ~ \stackrel{2}{=} ~
 \dfrac{1}{\ln a}
\]
Dove i passaggi hanno i seguenti motivi:
\begin{enumerate} [nosep]
 \item cambio di base del logaritmo;
 \item per quanto visto nell'esempio \ref{esempio:log}.
\end{enumerate}
\end{esempio}

\begin{esempio}
Limite di una particolare funzione esponenziale:
\[
\lim_{x \rightarrow 0} \dfrac{e^x-1}{x} =
\pst{\dfrac{e^\epsilon-1}{\epsilon}}
~ \stackrel{1}{=} ~  
\pst{\dfrac{\delta}{\ln{(1 + \delta)}}}~ \stackrel{2}{=} ~ 1
\]
Dove i passaggi hanno i seguenti motivi:
\begin{enumerate} [nosep]
 \item operiamo una sostituzione: poniamo \(e^\epsilon-1=\delta\). 
Allora \(e^\epsilon = 1+\delta\) quindi \(\epsilon\) è l'esponente da dare 
a \(e\) per ottenere \(1+\delta\) cioè: 
\(\epsilon = \ln{\tonda{1+\delta}}\);
 \item per quanto visto nell'esempio \ref{esempio:log}.
\end{enumerate}
\end{esempio}

\begin{esempio}
Simile all'esempio precedente, ma con una base generica.
\begin{align*}
\lim_{x \rightarrow 0} \dfrac{a^x-1}{x} &=
\pst{\dfrac{a^\epsilon-1}{\epsilon}}
~ \stackrel{1}{=} ~  
\pst{\dfrac{\delta}{\log_a{(1 + \delta)}}}~ \stackrel{2}{=}\\
&\stackrel{2}{=} 
\pst{\frac{\delta}{\dfrac{\ln{(1 + \delta)}}{\ln a}}} ~=~
\pst{\frac{\delta \cdot \ln a}{\ln{(1 + \delta)}}} %\stackrel{3}{=}\\
\stackrel{3}{=} ~ 
\pst{1 \cdot \ln a}=\ln{a}
\end{align*}

% \newpage   %----------------------------------------------------
Dove i passaggi hanno i seguenti motivi:
\begin{enumerate} [nosep]
 \item ancora una sostituzione: poniamo
\(a^\epsilon-1=\delta\), allora \(a^\epsilon= 1 + \delta\) e 
\(\epsilon=\log_a(1 + \delta)\);
 \item applichiamo la formula del cambiamento di base di un logaritmo; 
 \item per quanto visto nell'esempio \ref{esempio:log}.
\end{enumerate}
\end{esempio}

\begin{esempio}
Limite di una particolare funzione esponenziale:
\[
 \lim_{x \rightarrow \infty} \tonda{1+\dfrac{k}{x}}^x =
 \pst{\tonda{1+\dfrac{k}{N}}^N}
~ \stackrel{1}{=} ~  
\pst{\tonda{1+\dfrac{1}{M}}^{kM}}
~ \stackrel{2}{=} ~
\pst{\quadra{\tonda{1+\dfrac{1}{M}}^M}^k}
~ \stackrel{3}{=} ~ e^k
\]
Dove i passaggi hanno i seguenti motivi:
\begin{enumerate} [nosep]
 \item se al posto di \(\dfrac{k}{N}\) scrivo \(\dfrac{1}{M}\) 
allora al posto di \(N\) dovrò scrivere \(N=kM\)\\ 
(infatti: \(\dfrac{k}{N}=\dfrac{1}{M} \sLRarrow kM=1N\));
 \item la potenza di potenza è una potenza che ha per base la stessa base 
e per \dots
 \item l'espressione tra parentesi quadre è proprio la definizione di \(e\).
\end{enumerate}
\end{esempio}

% \newpage %------------------------------------------------
\subsection{Limite \(\epsilon \stext{-} \delta\)}

Nella seconda metà del 1800 i matematici Cauchy, Bolzano e Weierstrass  
hanno messo a punto il concetto di limite con la seguente definizione:

\begin{definizione}
Il numero reale \(L\) è il \textbf{limite finito} della funzione \(f(x)\) 
per \(x\) che tende a un valore \(\mathbf{c}\) \textbf{finito} 
se per ogni numero reale \(\epsilon\) 
esiste un numero reale \(\delta\) tale che 
per ogni \(x\) che dista da \(c\) meno di \(\delta\) e 
è diverso da \(c\),
\(f(x)\) dista da \(L\) meno di \(\epsilon\).

\[\lim_{x \rightarrow c}f(x) = L \stext{se}
\forall \epsilon > 0 ~~ \exists \delta > 0 
\sstext{tale che} \forall x \sstext{con} 0 < \abs{x - c} < \delta
\stext{allora} 
\abs{f(x) - L} < \epsilon\]
\end{definizione}

\begin{definizione}
Il \textbf{limite} della funzione \(f(x)\) 
per \(x\) che tende a un valore \(\mathbf{c}\) \textbf{finito} 
è \(+\infty\)
se per ogni numero reale \(M\) 
esiste un numero reale \(\delta\) tale che 
per ogni \(x\) che dista da \(c\) meno di \(\delta\) e 
è diverso da \(c\),
\(f(x)\) è maggiore di \(M\).

\[\lim_{x \rightarrow c}f(x) = +\infty \stext{se}
\forall M \in \R ~~ \exists \delta > 0 
\sstext{tale che} \forall x \sstext{con} 0 < \abs{x - c} < \delta
\stext{allora} f(x) > M\]
\end{definizione}

\affiancati{.49}{.49}{
\image{\limiteedfin}{[scale=.4]{img/limiteedfin}}
}{
\image{\limiteedinf}{[scale=.4]{img/limiteedinf}}
}

In modo simile vengono definiti anche:
\begin{itemize} [noitemsep]
\item il limite infinito negativo per \(x\) che tende a un finito;
\item il limite finito per \(x\) che tende a un infinito positivo o 
a un infinito negativo;
\item il limite infinito per \(x\) che tende a un infinito positivo o 
a un infinito negativo;
\item i limiti sinistro e destro per \(x\) che tende a un finito.
\end{itemize}

La cosa interessante, per noi, è che i matematici hanno dimostrato che se un 
valore è il limite secondo la definizione dei Cauchy-Bolzano-Weierstrass 
lo è anche secondo la definizione non standard, e viceversa.
