% (c) 2015 Daniele Zambelli daniele.zambelli@gmail.com

\input{\folder limiticontinuita1_grafici.tex}

\chapter{Funzioni: limiti e continuità}

\section{Limiti}
\label{sec:cont_limiti}

\footnote{Per scrivere questo capitolo mi sono ispirato 
al testo di H. Jerome KeislerKeissler ``Elementary Calculus: An Infinitesimal 
Approach''. 
Chi volesse approfondire l'argomento può acquistare il testo 
all'indirizzo: 
\href{https://www.math.wisc.edu/~keisler/calc.html}
     {www.math.wisc.edu/~keisler/calc.html}}
In alcuni problemi non siamo interessati a sapere come si comporta una funzione 
per un valore ben preciso, ci interessa di più 
sapere come si comporta quando la variabile \(x\) è \emph{infinitamente 
vicina} a quel valore.
Per un certo valore di \(x\) potrebbe anche \emph{non} essere definita, ma 
cosa succede quando \(x\) si avvicina infinitamente a quel valore?

Vediamo un esempio:

\begin{esempio}
 Studia l'Insieme di Definizione della funzione: 
 \(\dfrac{x^2-6x+5}{x^2+2x-3}\)
 Poi studia come si comporta la funzione per valori infinitamente vicini 
agli estremi dell'Insieme di Definizione.

La funzione fratta non è definita quando il denominatore vale zero:
\[x^2+2x-3=0 \sRarrow \tonda{x+3}\tonda{x-1}=0 \sRarrow x_1=-3;~x_2=+1\]
L'insieme di definizione è formato quindi dai seguenti intervalli:
\[\intervaa{-\infty}{-3} \quad \cup \quad \intervaa{+1}{-\infty}\]
Calcoliamo alcuni valori della funzione ``vicini'' agli estremi 
dell'Insieme di Definizione.

\begin{minipage}{.24\textwidth}
\begin{center}
\(-\infty\)\\
\begin{tabular}{r|r}
x & y\\\hline
-100 & 1.0824 \\
-1000 & 1.0080 \\
-10000 & 1.0008 \\
&\\
&\\
&
\end{tabular}
\end{center}
\end{minipage}
\begin{minipage}{.24\textwidth}
\begin{center}
\(-3\)\\
\begin{tabular}{r|r}
x & y\\\hline
-2.9 & -79.0 \\
-2.99 & -799.0 \\
-2.999 & -7999.0 \\
-3.001 & 8001.0 \\
-3.01 & 801.0 \\
-3.1 & 80.9 \\
\end{tabular}
\end{center}
\end{minipage}
\begin{minipage}{.24\textwidth}
\begin{center}
\(+1\)\\
\begin{tabular}{r|r}
x & y\\\hline
0.9 & -1.0512\\
0.99 & -1.0050 \\
0.999 & -1.0005 \\
1.001 & -0.9995 \\
1.01 & -0.9950 \\
1.1 & -0.9512 \\
\end{tabular}
\end{center}
\end{minipage}
\begin{minipage}{.24\textwidth}
\begin{center}
\(+\infty\)\\
\begin{tabular}{r|r}
x & y\\\hline
100 & 0.9223 \\
1000 & 0.9920 \\
10000 & 0.9992 \\
&\\
&\\
&
\end{tabular}
\end{center}
\end{minipage}
Possiamo osservare che:
\begin{itemize} [nosep]
 \item 
Quando \(x\) si avvicina a meno infinito, \(f(x)\) si avvicina a~\(1\) dall'alto.
 \item 
Quando \(x\) si avvicina a~\(-3\) da sinistra, \(f(x)\) diventa sempre più 
piccolo.
 \item 
Quando \(x\) si avvicina a~\(-3\) da destra, \(f(x)\) diventa sempre più 
grande.
 \item 
Quando \(x\) si avvicina a~\(+1\) da sinistra, \(f(x)\)  si avvicina a~\(-1\) dal 
basso.
 \item 
Quando \(x\) si avvicina a~\(+1\) da sinistra, \(f(x)\)  si avvicina a~\(-1\) 
dall'alto.
 \item 
Quando \(x\) si avvicina a più infinito, \(f(x)\) si avvicina a~\(1\) dal basso.
\end{itemize}
\begin{center}\scalebox{.6}{\limitigraficoa}\end{center}
\end{esempio}

Osservando il grafico provate a descrivere a parole cosa succede: 
\begin{itemize} [nosep]
 \item quando \(x\) si avvicina a \(-\infty\); 
 \item quando si avvicina a \(-3\) e quando vale proprio \(-3\);
 \item quando si avvicina a \(+1\) e quando vale proprio \(+1\); 
 \item e, infine, quando si avvicina a \(+\infty\).
\end{itemize}
\vspace{1em}
Prima della nascita dell'analisi non standard, per trattare queste 
situazioni i matematici si sono inventati il concetto di 
\emph{limite}.

\begin{definizione}
\(l\) è il \textbf{limite} di una funzione \(f(x)\) per \(x\) 
che tende a un 
valore \(c\), se, quando \(x\) è infinitamente vicino a \(c\), 
ma diverso da \(c\), 
allora \(f(x)\) è infinitamente vicino a \(l\). E si scrive:

\[l=\lim_{x \rightarrow c} f(x) \Leftrightarrow 
\forall x \tonda{\tonda{x \approx c \wedge x \neq c} \Rightarrow 
\tonda{f(x) \approx l}}\]
\end{definizione}

Per calcolare il limite di una funzione per \(x\) che tende a un certo 
valore, \(c\), basta calcolare la parte standard del valore della funzione 
per un numero infinitamente vicino a \(c\):
\[l=\lim_{x \rightarrow c} f(x) = \pst{f(c+\epsilon)}\]

Nel caso \(c\) sia un numero finito basta sostituire la \(x\) con \(c+\epsilon\) 
e poi prendere la parte standard.

\begin{esempio}
Calcola: \quad \(\displaystyle \lim_{x \rightarrow 3} \tonda{x^2-4x+2}\)
\begin{align*}
\lim_{x \rightarrow 3} \tonda{x^2-4x+2} & \stackrel{1}{=} 
  \pst{\tonda{3+\epsilon}^2-4\tonda{3+\epsilon}+2} \stackrel{2}{=}\\  
  & =\pst{9 + 6 \epsilon + \epsilon^2 - 12 - 4\epsilon + 2} \stackrel{3}{=}
  \pst{9 - 12 + 2} = -1
\end{align*}
Dove le uguaglianze hanno i seguenti motivi:
\begin{enumerate} [nosep]
 \item sostituiamo \(x\) con \(\tonda{3+\epsilon}\);
 \item eseguiamo i calcoli algebrici;
 \item sostituiamo l'espressione con una espressione indistinguibile,
   eseguiamo i calcoli e calcoliamo la parte standard.
\end{enumerate}
\end{esempio}

Se \(c=0\) ci possiamo semplificare notevolmente i calcoli ricordando 
che~\(0 + \epsilon = \epsilon\).

\begin{esempio}
Calcola: \quad 
  \(\displaystyle \lim_{x \rightarrow 0} \frac{x^2-4x+2}{x^2-4}\)
\begin{align*}
\lim_{x \rightarrow 0} \frac{x^2-4x+2}{x^2-4} & \stackrel{1}{=} 
  \pst{\frac{\epsilon^2-4\epsilon+2}{\epsilon^2-4}} \stackrel{2}{=}  
  \pst{\frac{2}{-4}} = -\frac{1}{2}
\end{align*}
Dove le uguaglianze hanno i seguenti motivi:
\begin{enumerate} [nosep]
 \item sostituiamo \(x\) con \(\tonda{0+\epsilon}=\epsilon\);
 \item sostituiamo l'espressione ottenuta con una espressione 
   indistinguibile, eseguiamo i calcoli e calcoliamo la parte standard.
\end{enumerate}
\end{esempio}

Se \(c\) è un numero infinito sostituiremo \(x\) con un generico 
infinito~\(M\), svolti i calcoli, prenderemo la parte standard.

\begin{esempio}
Calcola: \quad 
  \(\displaystyle \lim_{x \rightarrow \infty} \frac{3x^2-3x+7}{5x^2-6}\)
\begin{align*}
\lim_{x \rightarrow \infty} \frac{3x^2-3x+7}{5x^2-6} & \stackrel{1}{=} 
  \pst{\frac{3M^2-3M+7}{5M^2-6}} \stackrel{2}{=}  
  \pst{\frac{3\cancel{M^2}}{5\cancel{M^2}}} = \frac{3}{5}
\end{align*}
Dove le uguaglianze hanno i seguenti motivi:
\begin{enumerate} [nosep]
 \item sostituiamo \(x\) con \(M\);
 \item sostituiamo l'espressione ottenuta con una espressione 
   indistinguibile, eseguiamo i calcoli e calcoliamo la parte standard.
\end{enumerate}
\end{esempio}

\begin{esempio}
Calcola: \quad 
  \(\displaystyle \lim_{x \rightarrow -5} \frac{x+5}{x-3}\)
\begin{align*}
\lim_{x \rightarrow -5} \frac{x+5}{x-3} & \stackrel{1}{=} 
  \pst{\frac{-5+\epsilon+5}{-5+\epsilon-3}} \stackrel{2}{=}  
  \pst{\frac{\epsilon}{\epsilon-8}} \stackrel{3}{=} 
  \pst{\frac{\epsilon}{-8}} \stackrel{4}{=} \pst{\delta} = 0
\end{align*}
Dove le uguaglianze hanno i seguenti motivi:
\begin{enumerate} [nosep]
 \item sostituiamo \(x\) con \(-5+\epsilon\);
 \item eseguiamo i calcoli;
 \item sostituiamo l'espressione ottenuta con una espressione 
   indistinguibile;
 \item un infinitesimo diviso un finito non infinitesimo dà come risultato 
un infinitesimo e la sua parte standard è zero.
\end{enumerate}
\end{esempio}

\begin{esempio}
Calcola: \quad 
  \(\displaystyle \lim_{x \rightarrow 4} \frac{2x+6}{x-4}\)
\begin{align*}
\lim_{x \rightarrow 4} \frac{2x+6}{x-4} & \stackrel{1}{=} 
  \pst{\frac{2\tonda{4+\epsilon}+6}{4+\epsilon-4}} \stackrel{2}{=}  
  \pst{\frac{14 + 2 \epsilon}{\epsilon}} \stackrel{3}{=} 
  \pst{\frac{14}{\epsilon}} \stackrel{4}{=} 
  \pst{M} \stackrel{5}{~\longrightarrow~} \infty
\end{align*}
Dove le uguaglianze hanno i seguenti motivi:
\begin{enumerate} [nosep]
 \item sostituiamo \(x\) con \(4+\epsilon\);
 \item eseguiamo i calcoli;
 \item sostituiamo l'espressione ottenuta con una espressione 
   indistinguibile;
 \item un finito fratto un infinitesimo dà come risultato un infinito; 
 \item gli infiniti non hanno parte standard quindi qui non possiamo usare 
l'``\(=\)''. Ma i matematici per indicare un infinito usano il simbolo 
``\(\infty\)'', quindi quando dobbiamo calcolare la parte standard di un 
numero non finito, tracceremo una freccia seguita da ``\(\infty\)''.
\end{enumerate}
\end{esempio}

\begin{esempio}
\textbf{Importante}, calcola: \quad 
  \(\displaystyle \lim_{x \rightarrow \infty} \tonda{2x-\sqrt{4x^2-8x+3}}\)
\begin{align*}
\lim_{x \rightarrow \infty} \tonda{2x-\sqrt{4x^2-8x+3}} & \stackrel{1}{=} 
  \pst{2M-\sqrt{4M^2-8M+3}} \stackrel{2}{=}  
  \pst{2M-\sqrt{4M^2}} \stackrel{3}{=} 
  \pst{2M-2M} = 0
\end{align*}
Dove le uguaglianze hanno i seguenti motivi:
\begin{enumerate} [nosep]
 \item sostituiamo \(x\) con \(M\);
 \item sostituiamo l'espressione ottenuta con una espressione 
   indistinguibile;
 \item eseguiamo i calcoli \dots
\end{enumerate}
\vspace{1em}
\begin{minipage}{.69\textwidth}
Ma questa volta nei ragionamenti fatti c'è un errore. Proviamo a calcolare 
la funzione per alcuni valori abbastanza grandi di \(x\).
I risultati dovrebbero avvicinarsi a zero, ma non è così, sembra si 
avvicinino a due. Dove abbiamo sbagliato?
\end{minipage}
\begin{minipage}{.39\textwidth}
\begin{center}
\begin{tabular}{r|r}
x & y\\\hline
100 & 2.00252 \\
1000 & 2.00025 \\
10000 & 2.00002 \\
\end{tabular}
\end{center}
\end{minipage}

Abbiamo usato in modo improprio la relazione \emph{indistinguibile}: 
abbiamo ottenuto un'espressione indistinguibile da zero, ma ciò non è 
possibile. Dobbiamo seguire un'altra strada.

Consideriamo l'espressione data come una frazione e razionalizziamo il 
numeratore:
\[\lim_{x \rightarrow \infty} \frac{2x-\sqrt{4x^2-8x+3}}{1}=
  \pst{\frac{\cancel{4M^2}-\cancel{4M^2}+8M-3}{2M+\sqrt{4M^2-8M+3}}}=\]
Questa volta possiamo sostituire l'espressione sotto radice senza ottenere 
zero:
\[=\pst{\frac{8M-3}{2M+\sqrt{4M^2-8M+3}}} =
   \pst{\frac{8M}{2M+\sqrt{4M^2}}}=\]
e svolgendo i calcoli otteniamo:
\[=\pst{\frac{8M}{2M+2M}}=
   \pst{\frac{8\cancel{M}}{4\cancel{M}}} = 2\]
\end{esempio}

\begin{esempio}
Calcola: \quad 
  \(\displaystyle \lim_{x \rightarrow -2} \frac{x^3+3x^2+2x}{x^2-x-6}\)
 
Seguendo il metodo proposto sopra otteniamo:

\begin{align*}
\lim_{x \rightarrow -2} \frac{x^3+3x^2+2x}{x^2-x-6} & \stackrel{1}{=} 
\pst{\frac
  {\tonda{-2+\epsilon}^3+3\tonda{-2+\epsilon}^2+2\tonda{-2+\epsilon}}
  {\tonda{-2+\epsilon}^2-\tonda{-2+\epsilon}-6}} \stackrel{2}{=}\\ 
  &=\pst{\frac{-8+12\epsilon-6\epsilon^2+\epsilon^3+
             3\tonda{4-4\epsilon+\epsilon^2}-4+2\epsilon}
             {4-4\epsilon+\epsilon^2-\tonda{-2+\epsilon}-6}} =\\ 
  &=\pst{\frac{\cancel{-8}+\cancel{12\epsilon}-6\epsilon^2+\epsilon^3+
          \cancel{12}\cancel{-12\epsilon}+3\epsilon^2\cancel{-4}+2\epsilon}
             {\cancel{4}-4\epsilon+
              \epsilon^2\cancel{+2}-\epsilon\cancel{-6}}}=\\ 
  &=\pst{\frac{\epsilon^3+2\epsilon}{\epsilon^2-5\epsilon}} = 
    \pst{\frac{\cancel{\epsilon} \tonda{\epsilon^2+2}}
             {\cancel{\epsilon} \tonda{\epsilon-5}}}  \stackrel{3}{=} 
    \pst{\frac{2}{-5}} = -\frac{2}{5}
\end{align*}
Dove le uguaglianze hanno i seguenti motivi:
\begin{enumerate} [nosep]
 \item sostituiamo \(x\) con \(-2+\epsilon\);
 \item eseguiamo tutti i calcoli e semplifichiamo;
 \item sostituiamo l'espressione con una indistinguibile.
\end{enumerate}
\end{esempio}

\begin{osservazione}
Pensate a cosa succederebbe se ci fosse un qualche \(x^4\) \dots
Vedremo ora un altro modo di calcolare il limite che risulta molto meno 
complicato.

Nei prossimi due esempi, che vanno studiati assieme, vedremo un metodo che 
permette di svolgere meno calcoli.
\end{osservazione}

% 
% Per essere precisi questo esempio andrebbe affrontato dopo il prossimo 
% paragrafo sulle funzioni continue. Lo abbiamo inserito qui per completare 
% la trattazione dei limiti.
% 
% Anticipiamo quindi due affermazioni che saranno giustificate più avanti:
% \begin{itemize} [nosep]
%  \item Le funzioni fratte sono continue dove sono definite.
%  \item Se una funzione è continua in un punto, allora il limite per \(x\) 
% che 
% tende a quel punto è uguale al valore che la funzione ha in quel punto.
% \end{itemize}


\begin{esempio}
\textbf{Importante 1}, calcola: \quad 
  \(\displaystyle \lim_{x \rightarrow +2} \frac{x^3+3x^2+2x}{x^2-x-6}\)
 
Seguendo il metodo proposto sopra otteniamo:

\begin{align*}
\lim_{x \rightarrow 2} \frac{x^3+3x^2+2x}{x^2-x-6} & \stackrel{1}{=} 
\tonda{\pst{\frac
  {\tonda{2+\epsilon}^3+3\tonda{2+\epsilon}^2+2\tonda{2+\epsilon}}
  {\tonda{2+\epsilon}^2-\tonda{2+\epsilon}-6}} \stackrel{2}{=}}\\ 
  &=\frac{2^3+3\cdot 2^2+2\cdot 2}{2^2-2-6} =
  \pst{\frac{8+12+4}{4-2-6}} \stackrel{3}{=} \pst{\frac{24}{-4}} =-6
\end{align*}
Dove le uguaglianze hanno i seguenti motivi:
\begin{enumerate} [nosep]
 \item sostituiamo \(x\) con un numero infinitamente vicino a \(+2\);
 \item sostituiamo le espressioni tra parentesi con espressioni 
indistinguibili;
 \item poiché i risultati ottenuti sono diversi da zero, il risultato è 
effettivamente indistinguibile. Quindi il risultato ottenuto è valido.
\end{enumerate}
\end{esempio}

\begin{esempio}
\textbf{Importante 2}, calcola: \quad 
  \(\displaystyle \lim_{x \rightarrow -2} \frac{x^3+3x^2+2x}{x^2-x-6}\)
 
Seguendo il metodo proposto sopra otteniamo:

\begin{align*}
\lim_{x \rightarrow 2} \frac{x^3+3x^2+2x}{x^2-x-6} & \stackrel{1}{=} 
\tonda{\pst{\frac
  {\tonda{-2+\epsilon}^3+3\tonda{-2+\epsilon}^2+2\tonda{-2+\epsilon}}
  {\tonda{-2+\epsilon}^2-\tonda{-2+\epsilon}-6}} \stackrel{2}{=}}\\ 
  &=\pst{\frac{-8+12-4}{4+2-6}} = \pst{\frac{0}{0}} 
  \text{ Indistinguibile non è applicabile.}
\end{align*}

È evidente che \(-2\) è uno zero sia del numeratore sia del denominatore 
quindi, per il teorema di Ruffini, entrambi questi polinomi sono divisibili 
per \(x+2\).
Possiamo scomporre i due polinomi, semplificarli procedendo nel seguente 
modo:

\begin{align*}
\lim_{x \rightarrow -2} \frac{x^3+3x^2+2x}{x^2-x-6} &=
\lim_{x \rightarrow -2} \frac{x \tonda{x+1} \cancel{\tonda{x+2}}}
          {\tonda{x-3} \cancel{\tonda{x+2}}} \stackrel{1}{=}
\lim_{x \rightarrow -2} \frac{x^2+x}{x-3}=\\
&=\tonda{\pst{\frac{\tonda{-2+\epsilon}^2-\tonda{2+\epsilon}}
                   {\tonda{-2+\epsilon}-3}}}\stackrel{2}{=}
\pst{\frac{\tonda{-2}^2-2}{-2-3}}=
\pst{\frac{2}{-5}} = -\frac{2}{5}
\end{align*}
Dove le uguaglianze hanno i seguenti motivi:
\begin{enumerate} [nosep]
 \item possiamo semplificare perché \(x\) è infinitamente vicino a \(-2\) ma è 
diverso da \(-2\);
 \item sostituiamo l'espressione con una indistinguibile operazione, questa 
volta, applicabile perché le due espressioni sono diverse da zero.
\end{enumerate}
\end{esempio}

\begin{osservazione}
Abbiamo ottenuto lo stesso valore ricavato più sopra facendo tutti i 
calcoli algebrici, ma i passaggi sono estremamente più semplici.
\end{osservazione}


\begin{esempio}
Calcola: \quad 
  \(\displaystyle \lim_{x \rightarrow \infty} \tonda{1+\dfrac{k}{x}}^x\)
  
Ricordiamo che, per definizione:  
\(\pst{\tonda{1+\epsilon)^{\frac{1}{\epsilon}}}} = 
  \pst{\tonda{1+\dfrac{1}{N}}^N} = e\)
\begin{align*}
 \lim_{x \rightarrow \infty} \tonda{1+\dfrac{k}{x}}^x =
 \pst{\tonda{1+\dfrac{k}{N}}^N}
~ \stackrel{1}{=} ~  
\pst{\tonda{1+\dfrac{1}{M}}^{kM}}
~ \stackrel{2}{=} ~
\pst{\quadra{\tonda{1+\dfrac{1}{M}}^M}}^k
~ \stackrel{3}{=} ~ e^k
\end{align*}
Dove le uguaglianze hanno i seguenti motivi:
\begin{enumerate} [nosep]
 \item uno sporco trucco: la sostituzione. Supponiamo
\(\frac{k}{N}=\dfrac{1}{M}\). Allora \(N=kM\);
 \item una potenza di potenza è una potenza che ha...
 \item per la definizione del numero \(e\) e per le proprietà della funzione 
\(\st()\).
\end{enumerate}
\end{esempio}

\begin{esempio}
Calcola: \quad 
  \(\displaystyle \lim_{x \rightarrow 0} \dfrac{a^x-1}{x}\)
\begin{align*}
\lim_{x \rightarrow 0} \dfrac{a^x-1}{x} =
\pst{\dfrac{a^\epsilon-1}{\epsilon}}
~ \stackrel{1}{=} ~  
\pst{\dfrac{\delta}{\log_a{(\delta+1)}}}
~ \stackrel{2}{=} ~
\pst{\frac{1}{\dfrac{\log_a{(\delta+1)}}{\delta}}}
~ \stackrel{3}{=} ~ 
\pst{\frac{1}{\dfrac{1}{\ln{a}}}}=\ln{a}
\end{align*}
Dove le uguaglianze hanno i seguenti motivi:
\begin{enumerate} [nosep]
 \item ancora una sostituzione: poniamo
\(a^\epsilon-1=\delta\). Allora \(\epsilon=\log_a(\delta+1)\);
 \item una capriola algebrica: oplà! 
 \item per le forme di indecisione discusse a proposito del numero di Nepero
e per il cambiamento di base;
\end{enumerate}
\end{esempio}

\begin{esempio}
Calcola: \quad 
  \(\displaystyle \lim_{x \rightarrow 0} \dfrac{a^x-1}{x}\)
\begin{align*}
\lim_{x \rightarrow 0} \dfrac{a^x-1}{x} =
\pst{\dfrac{a^\epsilon-1}{\epsilon}}
~ \stackrel{1}{=} ~  
\pst{\dfrac{\delta}{\log_a{(\delta+1)}}}
~ \stackrel{2}{=} ~
\pst{\frac{1}{\dfrac{\log_a{(\delta+1)}}{\delta}}}
~ \stackrel{3}{=} ~ 
\pst{\frac{1}{\dfrac{1}{\ln{a}}}}=\ln{a}
\end{align*}
Dove le uguaglianze hanno i seguenti motivi:
\begin{enumerate} [nosep]
 \item ancora una sostituzione: poniamo
\(a^\epsilon-1=\delta\). Allora \(\epsilon=\log_a(\delta+1)\);
 \item una capriola algebrica: oplà! 
 \item per le forme di indecisione discusse a proposito del numero di Nepero
e per il cambiamento di base;
\end{enumerate}
\end{esempio}

\begin{esempio}
Calcola: \quad 
  \(\displaystyle \lim_{x \rightarrow 0} \dfrac{1-\cos x}{\sin x}\)
 % limite notevole seno e coseno
\[\lim_{x \rightarrow 0} \dfrac{1-\cos x}{\sin x} =
  \pst{\dfrac{1-\cos \delta}{\sin \delta}}=0\]
Dove l'uguaglianza si giustifica per quanto detto a proposito dell'ordine 
degli infinitesimi, ma gli appassionati del calcolo possono provare a 
dividere il numeratore e il denominatore per \(x\)...
\end{esempio}



\section{Continuità}
\label{sec:cont_continuita}

\subsection{Definizione di continuità in un punto}
\label{subsec:cont_definizione}


\begin{definizione}
Diremo che una funzione è \textbf{continua} in un punto \(c\), 
se è definita in \(c\) e, 
quando \(x\) è infinitamente vicino a \(c\), 
allora \(f(x)\) è infinitamente vicino a \(f(c)\). E si scrive::

\[f \text{ è continua in } c \Leftrightarrow 
\forall x \tonda{\tonda{x \approx c} \Rightarrow 
\tonda{f(x) \approx f(c)}}\]

\end{definizione}

\begin{esempio}
 Data la funzione \(f(x)=x^2-3x\) dimostrare che \(f(x)\) è continua in~4.
 
 La funzione è continua in~4 se per ogni \(x\) infinitamente vicino a~4 
 \(f(x)\) è infinitamente vicino a \(f(4)\). Cioè se \(x -4=\epsilon\) allora
 \(f(x) -f(4) = \delta\) dove \(\epsilon\) e \(\delta\) sono due infinitesimi.
 
\emph{dimostrazione}

Da \(x-4=\epsilon\) si ricava che \(x=4+\epsilon\), quindi: 
\[f(x) -f(4) = f(4+\epsilon) -f(4) = 
\tonda{4+\epsilon}^2-3\tonda{4+\epsilon}-\tonda{4^2-3\cdot 4}=\]
\[=\cancel{16}+8\epsilon+\epsilon^2\cancel{-12} 
  -3\epsilon\cancel{-16}\cancel{+12} = 
  +8\epsilon+\epsilon^2 -3\epsilon = 
\epsilon \tonda{5 + \epsilon}\]

Ora, il prodotto tra un infinitesimo e un finito è un infinitesimo, quindi, se 
la distanza tra \(x\) e \(4\) è infinitesima, anche la distanza tra 
\(f(x)\) e \(f(4)\) è infinitesima. \hfill \textbf{qed} 
 
\end{esempio}

\begin{esempio}
 Dimostrare che \(f(x)=\frac{\abs{x}}{x}\) non è continua in~0.
 
\emph{dimostrazione}\\
Perché una funzione sia continua per un certo valore di \(x\) lì deve essere 
definita. 
\end{esempio}

\begin{esempio}
 Dimostrare che \(f(x)=\frac{\abs{x}}{4}\) è continua in~0.
 
\emph{dimostrazione}:
\[f(\epsilon) - f(0) = \frac{\abs{\epsilon}}{4} - \frac{\abs{0}}{4} = 
 \frac{\abs{\epsilon}}{4} \approx 0\]
\end{esempio}

\begin{esempio}
Studia la continuità della funzione 
\(y=\begin{cases} 
    \dfrac{1}{2}x-2 & \mbox{se }x \leqslant 2 \\ 
    x^2-6x+7 & \mbox{se }x > 2
  \end{cases}\)
\quad in \(x_0=2\)

La funzione è definita per \(x=2\) e vale:\\
\(f(2) = \dfrac{1}{2} \cdot 2-2 = -1\)\\
Dobbiamo  verificare che, se \(x\) è infinitamente vicino a~\(2\) allora 
\(f(x)\) sia infinitamente vicino a~\(-1\).

\begin{minipage}{.49\textwidth}
Dobbiamo distinguere i casi in cui ci avviciniamo a~2 da sinistra o da 
destra. In entrambi i casi consideriamo \(epsilon\) positivo e esplicitiamo 
il segno:
\begin{description}
 \item [da sinistra:]
 ~\\
\(f(2-\epsilon) - f(2) =
  \dfrac{1}{2} \cdot \tonda{2-\epsilon}-2-\tonda{-1} =\)\\
\(= 1 - \dfrac{\epsilon}{2} -2 +1 = -\dfrac{\epsilon}{2}\)\\
Che è un infinitesimo.
 \item [da destra:]
 ~\\
\(f(2+\epsilon) - f(2) =\)\\
\(\tonda{2+\epsilon}^2-6 \cdot \tonda{2+\epsilon}+7-\tonda{-1} =\)\\
\(4+4\epsilon+\epsilon^2-12-6 \epsilon+7+1 =\)\\
\(\epsilon^2-2 \epsilon =\)\\
Che è un infinitesimo.
\end{description}
\end{minipage}
\begin{minipage}{.49\textwidth}
\begin{center}\continuitagraficoa\end{center}
\end{minipage}

\end{esempio}

Data una funzione \(y=f(x)\) definita in \(c\), le seguenti 
affermazioni sono equivalenti:

\begin{enumerate}[noitemsep]
 \item \(f\) è continua in \(c\);
 \item se \(x \approx c\) allora \(f(x) \approx f(c)\);
 \item se \(\st(x) = c\) allora \(\st(f(x)) = f(c)\);
 \item \(\lim_{x \to c} f(x) = f(c)\);
 \item se \(x\) si allontana da \(c\) di un infinitesimo allora 
   \(f(x)\) si allontana da \(f(c)\) di un infinitesimo.
 \item se \(\Delta x\) è infinitesimo allora il corrispondente \(\Delta y\) 
   è infinitesimo.
\end{enumerate}

\begin{comment}
\begin{teorema}[Derivabilità e continuità]
Se una funzione è derivabile in un punto allora è continua in quel punto.
\end{teorema}

\noindent Ipotesi: 
\(f(x) \text{ è derivabile in } c\)
\tab Tesi: 
\(f(x) \text{ è continua in } c\).

\begin{proof}
TODO
\end{proof}
\end{comment}

\subsection{Definizione di continuità in un intervallo}
\label{subsec:cont_definizione}

Dimostrare che una funzione è continua in un punto è piuttosto laborioso, 
pur non essendo complicato, ma quando sono interessato a studiare la 
continuità di una funzione in un intervallo sorge un ulteriore problema. 
Infatti in un intervallo, anche piccolo, i punti sono infiniti e dimostrare 
la continuità per ognuno di essi risulta piuttosto lungo\dots.

Per superare questo scoglio, i matematici hanno pensato un approccio diverso:

\begin{itemize}
 \item dimostrare che alcune funzioni elementari sono continue;
 \item dimostrare che la combinazione di funzioni continue è ancora una 
funzione continua.
\end{itemize}

In questo modo si può riconoscere la continuità di un gran numero di funzioni 
senza fare noiosi calcoli. Di seguito vediamo qualcuno di questi teoremi.

\subsubsection{Funzioni elementari}
\label{subsubsec:cont_funzionielementari}

Dimostriamo la continuità di alcune funzioni elementari.

\begin{teorema}[Continuità delle costanti]
Le funzioni costanti sono continue.
\end{teorema}

\noindent Ipotesi: \(f(x)=k\).\tab Tesi: \(f(x)\) è continua.

\begin{proof}
Per la definizione di continuità vogliamo dimostrare che 
\[\forall x \text{ se } x_0 \approx x \text{ allora } f(x_0) \approx f(x)\]
Poniamo \(x_0=x+\epsilon\), essendo la funzione costante, anche 
\[f(x+\epsilon)=k\] 
che, ovviamente, è infinitamente vicino a \(k\). In simboli:
\[f(x+\epsilon) = k \approx k = f(x)\] 
\end{proof}

\begin{teorema}[Continuità della funzione identica]
La funzione identica (\(y=x\)) è continua.
\end{teorema}

\noindent Ipotesi: \(f(x)=x\).\tab Tesi: \(f(x)\) è continua.

\begin{proof}
Per la definizione di continuità vogliamo dimostrare che 
\[\forall x \text{ se } x_0 \approx x \text{ allora } f(x_0) \approx f(x)\]
Poniamo \(x_0=x+\epsilon\), \(f(x_0) = f(x+\epsilon)=x+\epsilon\). 
Dato che la differenza:
\[f(x+\epsilon)-f(x) = x+\epsilon-x= \epsilon\]
è un infinitesimo, allora i due valori sono infinitamente vicini. In simboli:
\[f(x+\epsilon) = x+\epsilon \approx x = f(x)\] 
\end{proof}

\begin{teorema}[Continuità della funzione seno]
La funzione seno (\(y=\sen x\)) è continua.
\end{teorema}

\noindent Ipotesi: \(f(x)=\sen x\).\tab Tesi: \(f(x)\) è continua.

\begin{proof}
Usando la formula del seno della somma di due angoli:
\[f(x+\epsilon) =
\sen{(x+\epsilon)} = \sen x \cos \epsilon - \cos x \sen \epsilon\]
Se un angolo è infinitamente vicino a zero avrà il coseno infinitamente vicino 
a uno e il seno infinitamente vicino a zero. Quindi:
\[\sen{(x+\epsilon)} = \sen x + \delta\]
Perciò:
\[f(x+\epsilon) =
\sen{(x+\epsilon)} = \sen{(x+\epsilon)} = 
\sen x + \delta \approx \sen x = f(x)\]
\end{proof}
% 
% \begin{teorema}[Continuità della funzione esponenziale]
% La funzione esponenziale (\(y=a^x\)) è continua.
% \end{teorema}
% 
% \noindent Ipotesi: \(f(x)=a^x\).\tab Tesi: \(f(x)\) è continua.
% 
% \begin{proof}
% Usando la prima proprietà delle potenze:
% \[f(x+\epsilon) =
% a^{x+\epsilon} = a^{x} \cdot a^{\epsilon} = a^{x} \cdot \tonda{1 + \delta} 
% \approx a^{x} = f(x)\]
% \end{proof}

%---------- tentativi inutili
% \nopagebreak
% \samepage
% \filbreak
%----------

%----------
% 13
%  \setlength{\columnsep}{1.5pc}
% We want a rule between columns.
% 14
%  \setlength\columnseprule{.4pt}
% We also want to ensure that a new multicols envi-
% ronment finds enough space at the bottom of the
% page.
% 15
%  \setlength\premulticols{6\baselineskip}

%----------

\noindent\begin{minipage}{\textwidth}
Oltre alle funzioni precedenti, anche altre funzioni elementari sono 
continue, il seguente elenco riporta le principali funzioni continue:

\noindent\begin{minipage}{1.05\textwidth}
\begin{multicols}{5}
\begin{itemize} [noitemsep]
 \item \(y=k\)
 \item \(y=x\)
 \item \(y=\frac{1}{x}\)  \textasteriskcentered
 \item \(y=\sqrt[n]{x}\)  \textasteriskcentered
 \item \(y=\abs{x}\)
 \item \(y=a^x\)
 \item \(y=\log_a x\)  \textasteriskcentered
 \item \(y=\sen x\)
 \item \(y=\cos x\)
 \item \(y=\tg x\)  \textasteriskcentered
\end{itemize}
\end{multicols}
\end{minipage}

\begin{osservazione}
Le funzioni segnate da ``\textasteriskcentered'' sono continue non su 
tutto \(\R\), 
ma solo \textbf{all'interno del loro Insieme di Definizione}.
\end{osservazione}
\end{minipage}

\subsubsection{Composizione di funzioni}
\label{subsubsec:cont_composizionefunzioni}

Vediamo ora che anche componendo in alcuni modi funzioni continue otteniamo 
ancora funzioni continue.

\begin{teorema}[Somma di funzioni continue]
Se \(f\) e \(g\) sono funzioni continue, anche \(f+g\) è continua.
\end{teorema}

\noindent Ipotesi: 
\(f(x) \text{ e} g(x)\) sono continue
\tab Tesi: 
\(f(x)+g(x)\) è continua.

\begin{proof}
Dato che sono continue: 
\[f(x+\epsilon) + g(x+\epsilon) = f(x)+\alpha + g(x)+\beta\]
Ma la somma di due infinitesimi è ancora un infinitesimo quindi:
\[f(x)+g(x)+\tonda{\alpha + \beta} \approx f(x)+g(x)\]
\end{proof}

\begin{teorema}[Prodotto di funzioni continue]
Se \(f\) e \(g\) sono funzioni continue, anche \(f \cdot g\) è continua.
\end{teorema}

\noindent Ipotesi: 
\(f(x) \text{ e} g(x)\) sono continue
\tab Tesi: 
\(f(x) \cdot g(x)\) è continua.

\begin{proof}
Dato che sono continue: 
\[f(x+\epsilon) \cdot g(x+\epsilon) = 
\tonda{f(x)+\alpha} \cdot \tonda{g(x)+\beta} = 
f(x) \cdot g(x) + f(x) \cdot \beta + g(x) \cdot \alpha + \alpha \cdot \beta
\approx f(x) \cdot g(x)\]

Dato che sia il prodotto tra un numero finito e un infinitesimo, sia il 
prodotto tra due infinitesimi sono infinitesimi e lo è anche la loro somma. 
\end{proof}

\begin{corollario}
 Ogni funzione polinomiale è continua.
\end{corollario}

\begin{proof}
Dato che una funzione polinomiale si può ottenere partendo da funzioni 
costanti e da funzioni identiche attraverso moltiplicazioni e addizioni, 
la tesi consegue dai teoremi precedenti. 
\end{proof}

\begin{esempio}
 Dimostrare che \(f(x)=2x^2 + 3\) è una funzione continua.

\begin{proof}

\(f(x)=2x^2 + 3\) è continua perché è somma di due funzioni continue: 
% \begin{itemize}[noitemsep]
%  \item \(f(x)=2x^2 + 3\) è continua perché è somma di due funzioni continue: 
 \begin{itemize}[nosep]
  \item \(y=2x^2\) è continua perché è prodotto di due funzioni continue:
  \begin{itemize}[nosep]
   \item \(y=2\) è continua perché è una costante;
   \item \(y=x^2\) è continua perché è prodotto di due funzioni continue:
   \begin{itemize}[nosep]
    \item \(y=x\) è continua perché è una funzione identica;
    \item \(y=x\) è continua perché è una funzione identica;
   \end{itemize}
  \end{itemize}
  \item \(y=3\) è continua perché è una costante;
 \end{itemize}
% \end{itemize}
\end{proof}

\end{esempio}

\begin{teorema}[Funzioni di funzioni]
Se \(f(x)\) e \(g(x)\) sono funzioni continue, anche \(f(g(x))\) è continua.
\end{teorema}

\noindent Ipotesi: 
\(f(x) \text{ e} g(x)\) sono continue
\tab Tesi: 
\(f(x) \star g(x) = f(g(x))\) è continua.

\begin{proof}
Dato che \(g\) è continua: 
\[f(g(x+\epsilon)) = f(g(x)+\alpha)\]
e dato che \(f\) è continua: 
\[f(g(x)+\alpha)=f(g(x))+\beta\]
quindi: 
\[f(g(x+\epsilon)) = f(g(x)+\alpha) = f(g(x))+\beta \approx f(g(x))\]
\end{proof}











