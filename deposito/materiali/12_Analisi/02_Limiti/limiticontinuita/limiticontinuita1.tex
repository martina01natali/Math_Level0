% (c) 2015 Daniele Zambelli daniele.zambelli@gmail.com

\input{\folder limiticontinuita1_grafici.tex}

\chapter{Funzioni: continuità e limiti}

\section{Continuità}
\label{sec:cont_continuita}

Spesso per analizzare un fenomeno, e la funzione che lo rappresenta,  lo si 
valuta in vari punti di un intervallo \(\intervcc{a}{b}\) durante il 
quale si svolge.

\affiancati{.57}{.41}
{
Considerando i valori di \(f\) in un numero finito di punti non si può dire 
di conoscere appieno le caratteristiche del fenomeno, ma, se è 
\emph{sufficientemente regolare}, può darsi che i valori nei punti 
considerati diano già un'idea del fenomeno con una buona approssimazione.

Tanto è maggiore il numero dei punti considerati, tanto più ricca è 
l'informazione che si ricava. 

Non solo, ma si possono fare delle considerazioni, più o meno attendibili, 
sull'andamento globale del fenomeno confrontando a due a due i valori 
rilevati.
}
{\scalebox{1}{\partizionen}}

Se la funzione è poco regolare, tracciare solo i punti corrispondenti ad 
alcuni valori può servirci poco.
Consideriamo la seguente funzione definita nell'intervallo 
\(\intervcc{a}{b}\).

\affiancati{.49}{.49}
{\scalebox{1}{\puntigraficodiscontinuo}}
{\scalebox{1}{\graficodiscontinuo}}

Se nel grafico ci sono dei salti, come 
tra \(x_2\) e \(x_3\) e tra \(x_5\) e \(x_6\), risulta difficile tracciarla 
basandosi solo sui punti individuati, anche aumentando il loro numero.

Abbiamo detto che la curva deve essere ``sufficientemente regolare'',
ma cosa significa essere \emph{sufficientemente regolare}?

Nella prossima sezione parleremo di un certo tipo di regolarità molto 
importante che è la continuità di una funzione in un intervallo.
% \vspace{.5em}

\subsection{Definizione di continuità in un punto}
\label{subsec:cont_definizione}

Intuitivamente possiamo dire che una funzione è 
\emph{continua in un intervallo} 
se è rappresentata da una linea senza interruzioni e salti

Per precisare questo concetto, partiamo da definire cos'è una funzione 
\emph{continua in un punto} interno al suo insieme di definizione.

\begin{definizione}
Diremo che una funzione è \textbf{continua in un punto \(c\)} non isolato, 
se è definita in \(c\) e, 
quando \(x\) è infinitamente vicino a \(c\), 
allora \(f(x)\) è infinitamente vicino a \(f(c)\). 

\vspace{1em}
E si scrive \(f\) è continua nel punto \(c\) se: \hspace{12mm}
\(\forall x \approx c \quad f(x) \approx f(c)\)

\vspace{.5em}
oppure: \hspace{58mm}
\(\forall \epsilon \approx 0 \quad f(c + \epsilon) \approx f(c)\)

\vspace{.5em}
oppure: \hspace{58mm}
\(\forall \epsilon \approx 0 \quad f(c + \epsilon) - f(c) \approx 0\)

\vspace{.5em}
o ancora: \hspace{56mm}
\(\forall x \approx c \quad \pst{f(x)} = f(c)\)
\end{definizione}

\begin{esempio}
Data la funzione \(f(x)=-x^2+5\) dimostrare che \(f(x)\) è continua in~1.

\affiancati{.59}{.39}
{
La funzione è continua in~1 se per ogni \(x\) infinitamente vicino a~1 
\(f(x)\) è infinitamente vicino a \(f(1)\), \\
cioè se \(\forall \epsilon \quad f(1 + \epsilon) - f(1) \approx 0\)

\emph{Dimostrazione}
\begin{align*}
f(1+\epsilon) - f(1) &= 
-\tonda{1+\epsilon}^2+5-\tonda{-1^2+5}=\\
&=\cancel{-1}-2\epsilon-\epsilon^2 \cancel{+5}~
  \cancel{+1}~\cancel{-5} = \\
&=-2\epsilon-\epsilon^2 = 
\epsilon \tonda{-2 - \epsilon}
\end{align*}
}{
\scalebox{1}{\contprimo}
}
Ora, il prodotto tra un infinitesimo e un finito è un infinitesimo, quindi, 
se la distanza tra \(x\) e \(1\) è infinitesima, anche la distanza tra 
\(f(x)\) e \(f(1)\) è infinitesima. \hfill \(\qed\) 
 
\end{esempio}

\begin{esempio}
Dimostrare che \(f(x)=\frac{\abs{x}}{x}\) non è continua in~0.

\affiancati{.59}{.39}
{
\emph{Dimostrazione}\\
Perché una funzione sia continua per un certo valore di \(x\) 
lì deve essere definita. 

Quindi \(f(x)\) non è continua dato che \(0\) non appartiene al suo insieme 
di definizione.

Per essere pignoli, non ha neppure molto senso chiedersi se una funzione sia 
continua in un punto dove non è neppure definita!
}{
\scalebox{1}{\contsecondo}
}
\end{esempio}

% \newpage

\begin{esempio}
Continuità in zero della funzione \emph{segno}:
\(f(x)=\begin{cases} 
    -1 & \text{se } x < 0 \\ 
     0 & \text{se } x = 0 \\ 
    +1 & \text{se } x > 0
  \end{cases}
\)

\affiancati{.59}{.39}
{
\emph{Dimostrazione}\\
La funzione non è continua in \(0\) dato che ci sono dei valori 
\(x\) infinitamente vicini a \(0\) ma tali che \(f(x)\) non è infinitamente 
vicino a \(f(0)\). 

Anzi, c'è un solo valore \(x\) infinitamente vicino a \(0\) che è anche 
infinitamente vicino a \(f(0)\), ed è \(0\) stesso, tutti gli altri 
\(x \approx 0\) hanno una distanza da \(f(0)\)  non infinitesima.
}{
\scalebox{1}{\fsegno}
}
\end{esempio}

\affiancati{.59}{.39}
{
\begin{esempio}
 Dimostrare che \(f(x)=\frac{\abs{x}}{4}\) è continua in \(0\).

\emph{dimostrazione}:
\[f(\epsilon) - f(0) = \frac{\abs{\epsilon}}{4} - \frac{\abs{0}}{4} = 
 \frac{\abs{\epsilon}}{4} \approx 0\]
\end{esempio}
}{
\scalebox{1}{\contterzo}
}

\begin{esempio}
Studia la continuità in \(x_0=2\) della funzione: 
\(y=\begin{cases} 
    \dfrac{1}{2}x-2 & \text{se } x \leqslant 2 \\ 
    x^2-6x+7 & \text{se } x > 2
  \end{cases}\)

Per prima cosa dobbiamo verificare che la funzione sia definita per 
per \(x=2\) e trovare il suo valore:\\
\(f(2) = \dfrac{1}{2} \cdot 2-2 = -1\)\\
Dobbiamo quindi verificare che, se \(x\) è infinitamente vicino a~\(2\) 
allora \(f(x)\) sia infinitamente vicino a~\(f(2) = -1\).

Dobbiamo distinguere i casi in cui ci avviciniamo a~2 da sinistra o da 
destra. In entrambi i casi consideriamo \(\epsilon\) positivo e esplicitiamo 
il segno:

\begin{minipage}{.49\textwidth}
\begin{description}
 \item [da sinistra:]
 ~\\
\(f(2-\epsilon) - f(2) =
  \dfrac{1}{2} \cdot \tonda{2-\epsilon}-2-\tonda{-1} =\)\\
\(= 1 - \dfrac{\epsilon}{2} -2 +1 = -\dfrac{\epsilon}{2}\)\\ 
Che è un infinitesimo.
 \item [da destra:]
 ~\\
\(f(2+\epsilon) - f(2) =\)\\
\(\tonda{2+\epsilon}^2-6 \cdot \tonda{2+\epsilon}+7-\tonda{-1} =\)\\
\(4+4\epsilon+\epsilon^2-12-6 \epsilon+7+1 =\)\\
\(\epsilon^2-2 \epsilon\) \quad
Che è un infinitesimo.
\end{description}
\end{minipage}
\begin{minipage}{.49\textwidth}
\begin{center}\continuitagraficoa\end{center}
\end{minipage}

\end{esempio}

Data una funzione \(y=f(x)\) definita in \(c\), le seguenti 
affermazioni sono equivalenti:

\begin{enumerate}[noitemsep]
 \item \(f\) è continua in \(c\);
 \item se \(x \approx c\) allora \(f(x) \approx f(c)\);
 \item se \(\st(x) = c\) allora \(\st(f(x)) = f(c)\);
%  \item \(\lim_{x \to c} f(x) = f(c)\);
 \item se \(x\) si allontana da \(c\) di un infinitesimo allora 
   \(f(x)\) si allontana da \(f(c)\) di un infinitesimo.
%  \item se \(\delta x\) è infinitesimo allora il corrispondente \(\delta y\) 
%    è infinitesimo.
\end{enumerate}

\begin{comment}
\begin{teorema}[Derivabilità e continuità]
Se una funzione è derivabile in un punto allora è continua in quel punto.
\end{teorema}

\noindent Ipotesi: 
\(f(x) \text{ è derivabile in } c\)
\tab Tesi: 
\(f(x) \text{ è continua in } c\).

\emph{Dimostrazione}
TODO

\end{comment}

\subsection{Funzioni continue}
\label{subsec:cont_definizione}

Dimostrare che una funzione è continua in un punto è piuttosto laborioso, 
pur non essendo complicato, ma quando sono interessato a studiare la 
continuità di una funzione in un intervallo sorge un ulteriore problema. 
Infatti in un intervallo, anche piccolo, i punti sono infiniti e dimostrare 
la continuità per ognuno di essi risulta piuttosto lungo\dots

Per superare questo scoglio, i matematici hanno pensato un approccio 
diverso:
generalizzare il problema studiando la continuità di intere funzioni.

A prima vista, questo può sembrare un modo per complicare il problema, 
invece permette di riconoscere la continuità di un gran numero di 
funzioni senza dover fare noiosi calcoli. La strada seguita consiste nei 
seguenti due passi:
\begin{itemize} [noitemsep]
\item dimostrare che alcune funzioni elementari sono continue nel loro 
insieme di definizione;
\item dimostrare che la combinazione di funzioni continue è ancora una 
funzione continua.
\end{itemize}
Di seguito vediamo qualche teorema che permette di stabilire la continuità di 
un gran numero di funzioni.

\subsubsection{Funzione continua}
\label{subsubsec:cont_funzionecontinua}

\begin{definizione}
Una \emph{funzione è continua} se è continua in ogni punto del suo 
insieme di definizione.
\end{definizione}

Quando voglio studiare se una funzione nel suo complesso è continua o no, lo 
dovrò fare tenendo conto del suo insieme di definizione: non ha senso 
domandarsi ad esempio se \(f(x) = \sqrt{x}\) è continua in \(-5\) dato che lì 
non è definita.

Alcuni grafici di funzioni continue:

\hspace{-12mm}\begin{minipage}{.32\textwidth}
\begin{center} \scalebox{.9}{\contsecondo} \end{center}
\end{minipage}
\hfill
\begin{minipage}{.32\textwidth}
\begin{center} \contrad \end{center}
\end{minipage}
\hfill
\begin{minipage}{.32\textwidth}
\begin{center} \contip \end{center}
\end{minipage}


\subsubsection{Funzioni elementari}
\label{subsubsec:cont_funzionielementari}

Dimostriamo la continuità di alcune funzioni elementari.

\begin{teorema}[Continuità delle costanti]
Le funzioni costanti (\(f(x) = k\)) sono continue.
\end{teorema}

Ipotesi: \(f(x)=k\).\tab Tesi: \(f(x)\) è continua.

% \newpage %----------------------------------------------------

\noindent \emph{Dimostrazione}
Per la definizione di continuità vogliamo dimostrare che 

\affiancati{.49}{.49}{
\[\forall x \text{ se } x_0 \approx x \text{ allora } f(x_0) \approx f(x)\]
Poniamo \(x_0=x+\epsilon\), essendo la funzione costante, anche 
\[f(x+\epsilon)=k\] 
che, ovviamente, è infinitamente vicino a \(f(x) = k\). 
In simboli:
\[f(x+\epsilon) = k \approx k = f(x) \quad \qed\] 
}{
\begin{center} \continuitafcostante \end{center}
}

\begin{teorema}[Continuità della funzione identica]
La funzione identica (\(f(x) = x\)) è continua.
\end{teorema}

\noindent Ipotesi: \(f(x)=x\).\tab Tesi: \(f(x)\) è continua.

\emph{Dimostrazione}
Per la definizione di continuità vogliamo dimostrare che 

\affiancati{.49}{.49}{
\[\forall x \text{ se } x_0 \approx x \text{ allora } f(x_0) \approx f(x)\]
Poniamo \(x_0=x+\epsilon\), \(f(x_0) = f(x+\epsilon)=x+\epsilon\). 
Dato che la differenza:
\[f(x+\epsilon)-f(x) = x+\epsilon-x= \epsilon\]
è un infinitesimo, allora i due valori sono infinitamente vicini. 
In simboli:
\[f(x+\epsilon) = x+\epsilon \approx x = f(x) \quad \qed\] 
}{
\begin{center} \continuitafidentica \end{center}
}

\begin{comment}
% TODO un disegno per ogni esempio!!!!!!!!!!!!!!!!!!!!!!!!!!!!!!!!

\affiancati{.49}{.49}{
}{
\begin{center}  \end{center}
}
\end{comment}

\begin{teorema}[Continuità della funzione seno]
La funzione seno (\(f(x) = \sen x\)) è continua.
\end{teorema}

\noindent Ipotesi: \(f(x)=\sen x\).\tab Tesi: \(f(x)\) è continua.

\begin{center} \continuitafseno \end{center}

\noindent \emph{Dimostrazione}
Usando la formula del seno della somma di due angoli e ricordando che 
\(\sen \epsilon \approx 0 \stext{e} \cos \epsilon \approx 1\),
abbiamo:
\[f(x+\epsilon) =
\sen{(x+\epsilon)} = \sen x \cos \epsilon + \cos x \sen \epsilon = 
\sen x \cdot (1 + \delta) + \cos x \cdot \gamma \approx
\sen x = f(x)\]

\begin{comment}
% TODO Dimostrazione circolare

\begin{teorema}[Continuità della funzione esponenziale]
La funzione esponenziale (\(y=a^x\)) è continua.
\end{teorema}

\noindent Ipotesi: \(f(x)=a^x\).\tab Tesi: \(f(x)\) è continua.

\affiancati{.49}{.49}{
\emph{Dimostrazione}
Usando la prima proprietà delle potenze:

\begin{align*}
f(x+\epsilon) &= 
a^{x+\epsilon} = a^{x} \cdot a^{\epsilon} = 
a^{x} \cdot \tonda{1 + \delta} = \\
&=a^{x} \cdot \tonda{1 + \delta} + a^x \cdot \delta \approx 
a^{x} = f(x) \qed
\end{align*}
}{
\begin{center} \continuitafesp \end{center}
}
\end{comment}

%---------- tentativi inutili
% \nopagebreak
% \samepage
% \filbreak
%----------

%----------
% 13
%  \setlength{\columnsep}{1.5pc}
% We want a rule between columns.
% 14
%  \setlength\columnseprule{.4pt}
% We also want to ensure that a new multicols envi-
% ronment finds enough space at the bottom of the
% page.
% 15
%  \setlength\premulticols{6\baselineskip}

%----------

\noindent\begin{minipage}{\textwidth}
Oltre alle funzioni precedenti, anche altre funzioni elementari sono 
continue, il seguente elenco riporta le principali funzioni continue:

\noindent\begin{minipage}{1.05\textwidth}
\begin{multicols}{5}
\begin{itemize} [noitemsep]
 \item \(y=k\)
 \item \(y=x\)
 \item \(y=\frac{1}{x}\)  \textasteriskcentered
 \item \(y=\sqrt[n]{x}\)  \textasteriskcentered
 \item \(y=\abs{x}\)
 \item \(y=a^x\)
 \item \(y=\log_a x\)  \textasteriskcentered
 \item \(y=\sen x\)
 \item \(y=\cos x\)
 \item \(y=\tg x\)  \textasteriskcentered
\end{itemize}
\end{multicols}
\end{minipage}

\begin{osservazione}
Le funzioni segnate da ``\textasteriskcentered'' sono continue non su 
tutto \(\R\), 
ma solo \textbf{all'interno del loro Insieme di Definizione}.
\end{osservazione}
\end{minipage}

\subsubsection{Composizione di funzioni}
\label{subsubsec:cont_composizionefunzioni}

Vediamo ora che anche componendo in alcuni modi funzioni continue otteniamo 
ancora funzioni continue.

\begin{teorema}[Somma di funzioni continue]
Se \(f\) e \(g\) sono funzioni continue, anche \(f+g\) è continua.
\end{teorema}

\noindent Ipotesi: 
\(f(x) \stext{e} g(x)\) sono continue
\tab Tesi: 
\(f(x)+g(x)\) è continua.

\emph{Dimostrazione}
Dato che sono continue, 
\(f(x+\epsilon) = f(x)+\alpha \stext{ e } g(x+\epsilon) = g(x)+\beta\) \\
e dato che la somma di infinitesimi è un infinitesimo:
\[f(x+\epsilon) + g(x+\epsilon) = 
\tonda{f(x)+\alpha} + \tonda{g(x)+\beta} = 
f(x)+g(x)+\tonda{\alpha + \beta} \approx f(x)+g(x) \qed\]

\begin{teorema}[Prodotto di funzioni continue]
Se \(f\) e \(g\) sono funzioni continue, anche \(f \cdot g\) è continua.
\end{teorema}

\noindent Ipotesi: 
\(f(x) \text{ e} g(x)\) sono continue
\tab Tesi: 
\(f(x) \cdot g(x)\) è continua.

\emph{Dimostrazione}
Dato che sono continue e dato che il prodotto tra un numero finito e un 
infinitesimo è un infinitesimo: 
\[f(x+\epsilon) \cdot g(x+\epsilon) = 
\tonda{f(x)+\alpha} \cdot \tonda{g(x)+\beta} = 
f(x) \cdot g(x) + f(x) \cdot \beta + g(x) \cdot \alpha + \alpha \cdot \beta
\approx f(x) \cdot g(x)\]
\qed

\begin{corollario}
 Ogni funzione polinomiale è continua.
\end{corollario}

\emph{Dimostrazione}
Dato che una funzione polinomiale si può ottenere partendo da funzioni 
costanti e da funzioni identiche attraverso moltiplicazioni e addizioni, 
la tesi consegue dai teoremi precedenti. \qed

\begin{esempio}
 Dimostrare che \(f(x)=2x^2 + 3\) è una funzione continua.

\emph{Dimostrazione}
\(f(x)=2x^2 + 3\) è continua perché è somma di due funzioni continue: 
% \begin{itemize}[noitemsep]
%  \item \(f(x)=2x^2 + 3\) è continua perché è somma di due funzioni 
% continue: 
 \begin{itemize}[nosep]
  \item \(2x^2\) è continua perché è prodotto di due funzioni continue:
  \begin{itemize}[nosep]
   \item \(2\) è continua perché è una costante;
   \item \(x^2\) è continua perché è prodotto di due funzioni continue:
   \begin{itemize}[nosep]
    \item \(x\) è continua perché è una funzione identica;
    \item \(x\) è continua perché è una funzione identica;
   \end{itemize}
  \end{itemize}
  \item \(3\) è continua perché è una costante. \qed
 \end{itemize}
% \end{itemize}


\end{esempio}

\begin{teorema}[Funzioni di funzioni]
Se \(f(x)\) e \(g(x)\) sono funzioni continue, anche \(f(g(x))\) è continua.
\end{teorema}

\noindent Ipotesi: 
\(f(x) \text{ e} g(x)\) sono continue
\tab Tesi: 
\(f(g(x))\) è continua.

\emph{Dimostrazione}
Dato che \(g\) è continua: 
\[f(g(x+\epsilon)) = f(g(x)+\alpha)\]
e dato che \(f\) è continua: 
\[f(g(x)+\alpha)=f(g(x))+\beta\]
quindi: 
\[f(g(x+\epsilon)) = f(g(x)+\alpha) = f(g(x))+\beta \approx f(g(x)) \qed\]

\subsection{Continuità in un intervallo}
\label{subsec:cont_definizione}

Nello sviluppo dell'analisi hanno grande importanza le funzioni 
\(f\) puntualmente continue in \emph{ciascun} punto di un intervallo chiuso 
\(\intervcc{a}{b}\).

Diremo che una funzione \(f\) è puntualmente continua nell'intervallo 
\(\intervcc{a}{b}\) se è \emph{continua in ogni punto \(c\)} dell'intervallo:
\[f(x) \approx f(c) \stext{per ogni} x \stext{tale che} 
x \in \intervcc{a}{b} \stext{e} x \approx c.\] 
In questo caso diremo semplicemente che 
\begin{center}
la funzione \(f\) è continua nell'intervallo \(\intervcc{a}{b}\). 
\end{center}
Questa affermazione equivale a dire che:

\affiancati{.39}{.59}{
\begin{center} \continuitaintervallo \end{center}
}{
\begin{itemize}
\item 
per i punti \(c\) interni all’intervallo \(\intervcc{a}{b}\) 
(i punti che appartengono all'intervallo aperto \(\intervaa{a}{b}\))
per ogni infinitesimo \(\epsilon\) si ha che 
\(f(c+\epsilon) \approx f(c)\);
\item 
per gli estremi dell’intervallo ci si accontenterà di dire che è continua a 
destra in \(a\) e a sinistra in \(b\), cioè, per 
ogni infinitesimo \emph{positivo} \(\epsilon\), si ha che 
\(f(a+\epsilon) \approx f(a)\) e \(f(b-\epsilon) \approx f(b)\) 
(cioè con \(\epsilon > 0 \stext{ e } \pst{f(a+\epsilon)} = f(a)
\stext{ e } \pst{f(b-\epsilon)} = f(b)\)).
\end{itemize}
}
 
\begin{esempio}
~

\affiancati{.49}{.49}{
La funzione rappresentata qui a fianco, è continua nell'intervallo 
chiuso \(\intervcc{b}{c}\) ma non lo è negli intervalli chiusi 
\(\intervcc{a}{b}\) e \(\intervcc{c}{d}\). 

Si può dire comunque che è continua negli intervalli aperti:
\(\intervca{a}{b}\) e \(\intervac{c}{d}\).
}{
\begin{center}\continuitaintervalli\end{center}
}
\end{esempio}

La definizione di continuità in un intervallo non ci è di grande aiuto per 
verificare se una funzione è continua in un intervallo, dato che dovremmo 
controllare la continuità in ogni suo punto e, per quanto piccolo, 
l'intervallo, di punti, ne ha infiniti.

Si può seguire un'altra via: 

\begin{definizione}
Una funzione è continua in un intervallo se:
\begin{itemize} [noitemsep]
\item 
la funzione è continua i quell'intervallo;
\item 
l'intervallo è un sottoinsieme dell'insieme di definizione della funzione.
\end{itemize}
O, detto in altro modo: 
\begin{itemize} [noitemsep]
\item 
la funzione è continua i quell'intervallo;
\item 
è definita in tutti i punti dell'intervallo.
\end{itemize}
\end{definizione}

\begin{esempio}
Data la funzione \(f(x) = \dfrac{x-1}{x-5}\) stabilisci se la funzione è 
continua negli intervalli chiusi \(A=\intervcc{-1,5}{+2,5}\) e 
\(B=\intervcc{+3,5}{+7,5}\).

La funzione è continua (nel suo insieme di definizione) essendo una 
composizione di funzioni continue.

\affiancati{.49}{.49}{
Perché sia continua negli intervalli richiesti basta che questi intervalli 
siano sottoinsiemi del suo insieme di definizione.
\[\ID =~ \intervaa{-\infty}{+5} ~\cup~ \intervaa{+5}{+\infty} ~=~ 
\R \setminus \graffa{+5}\]
L'intervallo \(A=\intervcc{-1,5}{+2,5}\) è del tutto contenuto in \(\ID\) \\
L'intervallo \(B=\intervcc{+3,5}{+7,5}\) ha un elemento, \(+5\), 
che non appartiene a \(\ID\)

Perciò \(f(x)\) è continua nell'intervallo \(A\) e non è continua 
nell'intervallo \(B\).
}{
\begin{center} \continuitaintervalloese \end{center}
}
\end{esempio}


\section{Limiti}
\label{sec:cont_limiti}

\footnote{Per scrivere questo capitolo mi sono ispirato 
al testo di Howard Jerome Keisler, 
``Elementary Calculus: An Infinitesimal Approach''. 
Chi volesse approfondire l'argomento può scaricare il testo all'indirizzo: 
\href{https://www.math.wisc.edu/~keisler/calc.html}
     {www.math.wisc.edu/\(\sim\)keisler/calc.html}}
In alcuni problemi non siamo interessati a sapere come si comporta una 
funzione per un valore ben preciso (dove magari non è neppure definita), ci 
interessa di più sapere come si comporta quando la variabile \(x\) è 
\emph{sufficientemente vicina} a quel valore.

Detto altrimenti, cerchiamo di rispondere alla domanda: \\
\emph{Per un certo valore di \(x\) la funzione potrebbe anche \emph{non} 
essere definita, ma cosa succede quando \(x\) si avvicina a quel valore?}

Vediamo un esempio:

\begin{esempio}
 Studia l'Insieme di Definizione della funzione: 
 \(f(x)=\dfrac{x^2-6x+5}{x^2+2x-3}\),\quad
poi studia come si comporta la funzione per valori infinitamente vicini 
agli estremi dell'Insieme di Definizione.

La funzione è fratta e quindi non è definita quando il denominatore vale 
zero:
\[x^2+2x-3=0 \sRarrow \tonda{x+3}\tonda{x-1}=0 \sRarrow x_1=-3;~x_2=+1\]
L'insieme di definizione è formato quindi dai seguenti intervalli:
\[\intervaa{-\infty}{-3} \quad \cup \quad 
\intervaa{-3}{+1}\quad \cup \quad 
\intervaa{+1}{-\infty} \qquad \text{oppure} \qquad
\R \setminus \graffa{-3;~+1}\]
Calcoliamo alcuni valori della funzione ``vicini'' ai 4 estremi 
dell'Insieme di Definizione.

\begin{minipage}{.24\textwidth}
\begin{center}
\(-\infty\)\\
\begin{tabular}{r|r}
\(x\) & \(f(x)\quad\)\\\hline
\(-100\) & \(+1.0824\) \\
\(-1000\) & \(+1.0080\) \\
\(-10000\) & \(+1.0008\) \\
\dots \\
&\\
&\\
&
\end{tabular}
\end{center}
\end{minipage}
\begin{minipage}{.24\textwidth}
\begin{center}
\(-3\)\\
\begin{tabular}{r|r}
\(x\) & \(f(x)\quad\)\\\hline
\(-2.9\) & \(-79.0\) \\
\(-2.99\) & \(-799.0\) \\
\(-2.999\) & \(-7999.0\) \\
\dots \\
\(-3.001\) & \(+8001.0\) \\
\(-3.01\) & \(+801.0\) \\
\(-3.1\) & \(+80.9\) \\
\end{tabular}
\end{center}
\end{minipage}
\begin{minipage}{.24\textwidth}
\begin{center}
\(+1\)\\
\begin{tabular}{r|r}
\(x\) & \(f(x)\quad\)\\\hline
\(+0.9\) & \(-1.0512\)\\
\(+0.99\) & \(-1.0050\) \\
\(+0.999\) & \(-1.0005\) \\
\dots \\
\(+1.001\) & \(-0.9995\) \\
\(+1.01\) & \(-0.9950\) \\
\(+1.1\) & \(-0.9512\) \\
\end{tabular}
\end{center}
\end{minipage}
\begin{minipage}{.24\textwidth}
\begin{center}
\(+\infty\)\\
\begin{tabular}{r|r}
\(x\) & \(f(x)\quad\)\\\hline
\(+100\) & \(+0.9223\) \\
\(+1000\) & \(+0.9920\) \\
\(+10000\) & \(+0.9992\) \\
\dots \\
&\\
&\\
&
\end{tabular}
\end{center}
\end{minipage}
Possiamo osservare che:
\begin{itemize} [nosep]
 \item Quando \(x\) diventa sempre più piccolo 
 (si avvicina a meno infinito), 
\(f(x)\) si avvicina a~\(1\) dall'alto.
 \item Quando \(x\) si avvicina a~\(-3\) da sinistra, 
\(f(x)\) diventa sempre più piccolo.
 \item Quando \(x\) si avvicina a~\(-3\) da destra, 
\(f(x)\) diventa sempre più grande.
 \item Quando \(x\) si avvicina a~\(+1\) da sinistra, 
\(f(x)\)  si avvicina a~\(-1\) dall'alto.
 \item Quando \(x\) si avvicina a~\(+1\) da destra, 
\(f(x)\)  si avvicina a~\(-1\) dal basso.
 \item Quando \(x\) diventa sempre più grande 
 (si avvicina a più infinito), 
\(f(x)\) si avvicina a~\(1\) dal basso.
\end{itemize}
\begin{center}\scalebox{.6}{\limitigraficoa}\end{center}
\end{esempio}

Osservando il grafico provate a descrivere a parole cosa succede:

\begin{minipage}{.64\textwidth}
\begin{itemize} [nosep]
 \item quando \(x\) si avvicina a \(-\infty\); 
 \item quando si avvicina a \(-3\) e quando vale proprio \(-3\);
 \item quando si avvicina a \(+1\) e quando vale proprio \(+1\); 
 \item e, infine, quando si avvicina a \(+\infty\).
\end{itemize}

\begin{osservazione}
Nel tratto crescente della funzione, che appare come linea continua, in 
realtà manca un punto: il punto \(\punto{1}{-1}\). 
Questo punto mancante è invisibile essendo infinitamente piccolo, quindi il 
grafico della funzione appare continuo. 
Possiamo indicare un punto mancante in una linea usando un cerchietto vuoto 
come nell'ingrandimento a fianco.
\end{osservazione}

\end{minipage}
\hfill
\begin{minipage}{.34\textwidth}
\begin{center}\scalebox{1.2}{\limitigraficob}\end{center}
\end{minipage}

\vspace{1em}
Un tempo si trattavano queste situazioni usando infinitesimi e infiniti in 
modo non del tutto rigoroso.
Alcuni matematici del 1800, rifiutandosi di usare infinitesimi e infiniti, 
hanno ricostruito la matematica creata nei secoli precedenti usando il 
concetto di
% Per sopperire a questa mancanza di precisione, i matematici hanno inventato 
% il concetto di limite.
% Prima della nascita dell'analisi non standard, per trattare queste 
% situazioni i matematici si sono inventati il concetto di 
\emph{limite}.

Il concetto di limite perfezionato da Weierstrass è piuttosto complesso e lo 
riprenderemo più avanti.
Noi che usiamo numeri infinitesimi e infiniti diremo che

\begin{definizione}
Il numero reale \(L\) è il \textbf{limite} della funzione \(f(x)\) 
per \(x\) che tende a un valore \(c\) se \(f(x)\) è infinitamente vicino a 
\(L\) per tutti gli \(x\) infinitamente vicini a \(c\) e diversi da \(c\).

E si scrive:
\[L = \lim_{x \rightarrow c} f(x) \sstext{se e solo se} 
L \approx {f(x)} \quad 
\forall x \approx c \stext{ e } x \neq c\]

In modo equivalente, usando la parte standard: 
\[\lim_{x \to c} f(x) = \pst{f(c+\epsilon)}\]
se questa parte standard esiste e è sempre la stessa 
per ogni \(\epsilon \approx 0 \stext{ e } \epsilon \neq 0\).
\end{definizione}

\begin{esempio}
Consideriamo la funzione \(f\) il cui grafico è riportato a fianco.

\affiancati{.49}{.49}{
Possiamo osservare che la funzione è continua (perché?) e che non è 
definita per \\
\(x = a \stext{ e } x = b\), quindi:
\[\ID = \R \setminus \graffa{a;~b}\]
Possiamo quindi studiare il limite per \(x \to a\) e \(x \to b\). 

\[\lim_{x \to a}f(x) = L\]
dato che per ogni \(x\) infinitamente vicino a \(a\) \(f(x)\) è infinitamente 
vicino a \(L\). 
\[\lim_{x \to b}f(x) = \text{ non esiste}\]

}{
\begin{center} \limitefinitodef \end{center}
}
perché per alcuni \(x\) infinitamente vicini a \(b\) \(f(x)\) è 
infinitamente vicino a \(M\) e per alcuni \(x\) infinitamente vicini a \(b\) 
\(f(x)\) è infinitamente vicino a \(N\).
\end{esempio}

\begin{esempio}
Riprendiamo il primo esempio presentato in questa sezione e calcoliamo il 
limite: \(\displaystyle \lim_{x \to 1}\dfrac{x^2-6x+5}{x^2+2x-3}\).
\begin{align*}
\lim_{x \to 1}\dfrac{x^2-6x+5}{x^2+2x-3} &=
\pst{\dfrac{\tonda{1+\epsilon}^2-6\tonda{1+\epsilon}+5}
           {\tonda{1+\epsilon}^2+2\tonda{1+\epsilon}-3}} = \\
&=\pst{\dfrac{\cancel{1}+2\epsilon+\epsilon^2~
              \cancel{-6}-6\epsilon~\cancel{+5}}
           {\cancel{1}+2\epsilon+\epsilon^2~
            \cancel{+2}+2\epsilon~\cancel{-3}}} =
\pst{\dfrac{-4\epsilon+\epsilon^2}
           {+4\epsilon+\epsilon^2}} = \\
&=\pst{\dfrac{\cancel{\epsilon} \tonda{-4+\epsilon}}
             {\cancel{\epsilon} \tonda{+4+\epsilon}}} =
\pst{\dfrac{-4+\epsilon}{+4+\epsilon}} =
\dfrac{\pst{-4+\epsilon}}{\pst{+4+\epsilon}} =
\dfrac{-4}{+4} = -1
\end{align*}


\end{esempio}

% \footnote{Possiamo inserire queste osservazioni nella definizione 
% ottenendo:
% \begin{definizione}
% Il numero reale \(l\) è il \textbf{limite} di una funzione reale \(f(x)\) 
% per \(x\) che tende a un valore reale \(c\), se, 
% quando \(x\) è un qualunque numero iperreale infinitamente vicino a \(c\), 
% ma diverso da \(c\), 
% allora il valore della corrispondente funzione iperreale \({}^*f(x)\) è 
% infinitamente vicino al valorereale \(l\).
% \end{definizione}}

\subsubsection{Casi in cui il limite finito non esiste}
\label{subsubsec:cont_limiti_nonesiste}

Il calcolo del limite presentato sopra permette di calcolare il limite 
finito solo se ci sono alcune condizioni, ma in vari casi può fallire, 
i motivi per cui può fallire sono:
\begin{enumerate} [nosep]
\item la funzione \(f(c+\epsilon)\) non è definita per alcuni valori di 
\(\epsilon = 0\); 
\item al variare dell'infinitesimo \(\epsilon\) la parte standard 
\(\pst{f(c+\epsilon)}\) non è sempre la stessa;
\item la funzione \(f(c+\epsilon)\) dà come risultato un valore infinito e 
quindi non ha parte standard. 
\end{enumerate}

Nelle prossime sezioni vedremo quando è possibile, comunque, ricavare delle 
informazioni interessanti in alcune delle situazioni elencate sopra.

\subsection{Se non esiste \(f(c+\epsilon)\) per ogni \(\epsilon\)}
\label{subsec:cont_limiti_nonsempreesiste}

\affiancati{.49}{.49}{
\begin{center} \limitenea \end{center}
}{
Potremmo trovarci nella situazione nella quale, a distanza infinitesima 
da un certo valore \(c\), la funzione non sia sempre definita.

In questo caso il limite non esiste.
}

Ma potremmo trovarci in una situazione fortunata se la funzione è sempre 
definita a sinistra o a destra del punto \(c\).

In questo caso potremmo parlare di limite sinistro o di limite destro della 
funzione.

\affiancati{.49}{.49}{
Ad esempio nella situazione rappresentata a fianco, la funzione è sempre 
definita a sinistra del punto \(c_1\) e a destra del punto \(c_2\).
Pur non esistendo i limiti in \(c_1\) e \(c_2\),
potremo dire che esiste il limite sinistro per \(x \to c_1\) e
il limite destro per \(x \to c_2\):
\[\lim_{x \to c_1^-} f(x) = 0 \sstext{e} \lim_{x \to c_1^+} f(x) = 0\]
}{
\begin{center} \limiteneb \end{center}
}

\begin{definizione}
Il numero reale \(L\) è il \textbf{limite sinistro} della funzione \(f(x)\) 
per \(x\) che tende a un valore \(c\) se \(f(x)\) è infinitamente vicino a 
\(L\) per tutti gli \(x\) infinitamente vicini a \(c\) e minori di 
\(c\). \quad 
E si scrive:
\[L = \lim_{x \rightarrow c^-} f(x) \sstext{se e solo se} 
L \approx {f(x)} \quad 
\forall x \approx c \stext{ e } x < c\]

In modo equivalente, usando la parte standard: 
\[\lim_{x \to c^-} f(x) = \pst{f(c+\epsilon)}\]
se questa parte standard esiste e è sempre la stessa 
per ogni \(\epsilon \approx 0 \stext{ e } \epsilon < 0\).
\end{definizione}
In modo simmetrico si definisce il \textbf{limite destro}.

\begin{esempio}
Studia il comportamento della funzione \(f(x)=\sqrt{3x-12}\) 
vicino al punto di ascissa \(4\).

\affiancati{.49}{.49}{
Per prima cosa studiamo l'\(\ID\) della funzione. 
L'argomento della radice quadrata deve essere maggiore di zero: 
\[3x-12>0 \sRarrow 3x>12 \sRarrow x>4\]
Quindi: \(\ID = \intervca{4}{\infty}\)

}{
\begin{center} \limitesqrt \end{center}
}

% Possiamo calcolare il valore della funzione in \(4\):
% \(f(4) = \sqrt{3 \cdot 4 - 12} = \sqrt{12 - 12} = \sqrt{0} = 0\). 
% La funzione è quindi definita solo a destra del punto \(\punto{4}{0}\).
% Quindi possiamo studiare solo il limite destro per \(x \to 2\):
Partiamo applicando la definizione del limite finito per \(x\) che tende a 
\(4\):
\[\lim_{x \to 4}\sqrt{3x-12} = \pst{3\tonda{4+\epsilon}-12} = 
\pst{\sqrt{\cancel{+12}+3 \epsilon~\cancel{-12}}} = 
\pst{\sqrt{3 \epsilon}} = \dots\]
A questo punto vediamo che non possiamo procedete con i calcoli per ogni 
valore di \(\epsilon\) quindi \emph{non esiste il limite}.

Però, dato che posso eseguire i calcoli \emph{per ogni} \(\epsilon > 0\) 
possiamo calcolare \emph{il limite destro}. 
Quindi sospendiamo il calcolo del limite, 
aggiustiamo l'intestazione e procediamo senza ripetere i calcoli già fatti:
\[\lim_{x \to 4^+}\sqrt{3x-12} = \dots =
\pst{\sqrt{3 \epsilon}} = 
\sqrt{\pst{3 \epsilon}} = \sqrt{0} = 0\]
\end{esempio}

\subsection{Se non esiste la parte standard}
\label{subsec:cont_limiti_nonsempreesiste}

La funzione parte standard è definita solo per valori finiti 
dell'argomento.
Può succedere che calcolando il limite, l'argomento della parte standard sia  
un numero infinito. Questa situazione può essere interessante, così 
interessante da meritare un ampliamento della definizione.

\subsubsection{Infinito: \(\infty\)}

Tutti i numeri reali sono finiti, quando vogliamo descrivere il comportamento 
di una funzione che cresce sempre di più dobbiamo utilizzare un simbolo che 
non fa parte dell'insieme dei reali. 

Il simbolo \(\infty\) non indica un numero, ma un comportamento.
Precisiamo il significato:

\begin{definizione}
Useremo:
\begin{itemize}
\item \(+\infty\) per indicare il comportamento di una funzione che cresce 
sempre di più, 
cioè che assume valori maggiori di \(m\) qualunque sia \(m \in \R\);
\item \(-\infty\) per indicare il comportamento di una funzione che 
diminuisce sempre di più, 
cioè che assume valori minori di \(m\) qualunque sia \(m \in \R\);
\item in alcuni libri con \(\infty\) si intende \(+\infty\); 
\item in altri libri con \(\infty\) si intende un comportamento di una 
funzione che, in valore assoluto diventa grande più di ogni numero reale.
\end{itemize}
\end{definizione}

Dato che non c'è un uso concorde del simbolo ``\(\infty\)'' cercheremo 
di precisarne sempre il significato.

Possiamo ora ampliare la definizione di limite:
\begin{definizione}
Diremo che il \textbf{limite} della funzione \(f(x)\) 
per \(x\) che tende a un valore \(c\) è \(+\infty\) 
se per ogni \(\epsilon \approx 0 \stext{ e } \epsilon \neq 0\) 
\(f(c + \epsilon)\) è numero infinito positivo. \quad 
E si scrive:
\[\lim_{x \rightarrow c} f(x) = +\infty \sstext{se e solo se} 
{f(c + \epsilon) = M > 0} \quad 
\forall \epsilon \approx 0 \stext{ e } \epsilon \neq 0\]
Simmetricamente diremo che il \textbf{limite} della funzione \(f(x)\) 
per \(x\) che tende a un valore \(c\) è \(-\infty\) 
se per ogni \(\epsilon \approx 0 \stext{ e } \epsilon \neq 0\) 
\(f(c + \epsilon)\) è numero infinito negativo. \quad 
E si scrive:
\[\lim_{x \rightarrow c} f(x) = -\infty \sstext{se e solo se} 
{f(c + \epsilon) = M < 0} \quad 
\forall \epsilon \approx 0 \stext{ e } \epsilon \neq 0\]
\end{definizione}

Vediamo ora come applicare questa nuova definizione.

\begin{esempio}
Studia il comportamento della funzione \(f(x)=\dfrac{x+2}{x^2+8x+16}\) 
vicino al punto di ascissa \(4\).

\affiancati{.59}{.39}{
Per prima cosa studiamo l'\(\ID\) della funzione \dots~non è definita per 
\(x = -4\).
Vogliamo studiare il comportamento vicino a questo valore.
Avendo a disposizione il suo grafico, possiamo intuirlo, ma noi vogliamo 
calcolarlo. 
Studiamo dunque il limite:
\begin{align*}
\lim_{x \to -4} \dfrac{x+2}{x^2+8x+16} &=
\pst{\dfrac{\tonda{-4 +\epsilon}+2}
           {\tonda{-4 +\epsilon}^2+8\tonda{-4 +\epsilon}+16}} =\\ 
&=\pst{\dfrac{-4 +\epsilon +2}
             {\cancel{16}~\cancel{-8\epsilon} +\epsilon^2~
              \cancel{-32}~\cancel{+8\epsilon}~\cancel{+16}}} =\\ 
&=\pst{\dfrac{-2 +\epsilon}{\epsilon^2}} 
= \pst{M}
\end{align*}
Si vede chiaramente, nell'ultima riga, che l'argomento della parte 
standard è un infinito perché abbiamo un finito non infinitesimo fratto un 
infinitesimo.
}{
\begin{center} \limiteiesea \end{center}
}
\end{esempio}
Per ogni \(\epsilon\), il numeratore è negativo e il denominatore è 
positivo, quindi il valore negativo risultato della frazione è un 
\emph{infinito negativo} che abbiamo indicato con \(M\). 
Quindi:
\[\lim_{x \to -4} \dfrac{x+2}{x^2+8x+16} = -\infty\]

\subsection{Se \(\pst{f(c+\epsilon)}\) varia al variare di \(\epsilon\)}
\label{subsec:cont_limiti_nonsempreesiste}

Può capitare che al variare del valore \(x\) infinitamente vicino a \(c\) la 
parte standard della funzione \(f(x)\) assuma diversi valori. Ovviamente in 
questo caso il limite non esiste.

\subsubsection{Una funzione interessante: \(\sen \dfrac{1}{x}\)}

\noindent Una funzione importante per esplorare situazioni strane attorno 
allo zero è: \(f(x) = \sen \dfrac{1}{x}\).\\
Più \(\abs{x}\) diventa grande, più \(\dfrac{1}{x}\) si avvicina a \(0\) e 
quindi anche \(\sen \dfrac{1}{x}\) si avvicina all'asse \(x\).

Ma la parte interessante della funzione è quella vicino allo \(0\):
man mano che la variabile \(x\) si avvicina a \(0\), \(\dfrac{1}{x}\) diventa 
sempre più grande e quindi \(\sen \dfrac{1}{x}\) oscilla in modo sempre più 
fitto tra \(-1 \stext{e} +1\).
Il grafico seguente si riferisce alla funzione \(\sen \dfrac{5}{x}\), in 
questo modo il grafico è un po' stirato nel senso della larghezza e si può 
distinguere qualche oscillazione in più.
\begin{center} \limitesinunosux \end{center}
% \begin{center} \limitesqrtsinunosux \end{center}
% \begin{center} \limitequadsqrtsinunosux \end{center} % non funziona!!!
Il comportamento della funzione, attorno allo zero, si mantiene, per 
transfer, anche quando \(x\) è infinitamente vicino a zero.
Quindi al variare di \(\epsilon\) la funzione continuerà a oscillare tra 
\(-1 \stext{e} +1\) perciò, al variare di \(\epsilon\) otterrò diversi valori 
di \(\pst{\sen \dfrac{1}{\epsilon}}\).

Quindi concludiamo che il 
\(\displaystyle \lim_{x \to 0}\sen \dfrac{1}{x}\) \emph{non esiste}.

\subsubsection{Limite sinistro e destro}
A volte possiamo trovarci in una situazione più fortunata:
Potrebbe succedere che 
\begin{itemize} [nosep]
\item 
per tutti i valori di \(x\) minori di \(c\) e infinitamente vicini a \(c\), 
\(f(x)\) sia infinitamente vicino al valore \(M\);
\item 
per tutti i valori di \(x\) maggiori di \(c\) e infinitamente vicini a \(c\), 
\(f(x)\) sia infinitamente vicino al valore \(N\). 
\end{itemize}

In questi casi possiamo dire che, pur non esistendo il limite, esiste un 
\emph{limite sinistro} o un \emph{limite destro}.

Indicheremo il limite sinistro con un \emph{meno} (\(c^-\)) posto a esponente 
e il limite destro con un più posto a esponente (\(c^+\)).

Possiamo quindi dare la definizione di limite sinistro:

\begin{definizione}
Diremo che il \emph{limite sinistro} per \(x\) che tende a \(c\) 
della funzione \(f(x)\) è:
\noindent \begin{itemize}
\item 
\(\displaystyle \lim_{x \to c^-}f(x) = \pst{f(c+\epsilon)}\)
se \(f(c+\epsilon)\) è un numero finito e \\
per ogni \(\epsilon \approx 0 \stext{e} \epsilon < 0\)
la parte standard di \(f(c+\epsilon)\) è sempre la stessa.
\item 
\(\displaystyle \lim_{x \to c^-}f(x) = +\infty\)
se \(f(c+\epsilon)\) è un numero infinito positivo 
per ogni \(\epsilon \approx 0 \stext{e} \epsilon < 0\).
\item 
\(\displaystyle \lim_{x \to c^-}f(x) = -\infty\)
se \(f(c+\epsilon)\) è un numero infinito negativo 
per ogni \(\epsilon \approx 0 \stext{e} \epsilon < 0\).
\end{itemize}
\end{definizione}

\begin{definizione}
Diremo che il \emph{limite destro} per \(x\) che tende a \(c\) 
della funzione \(f(x)\) è:
\begin{itemize}
\item 
\(\displaystyle \lim_{x \to c^+}f(x) = \pst{f(c+\epsilon)}\)
se \(f(c+\epsilon)\) è un numero finito e \\
per ogni \(\epsilon \approx 0 \stext{e} \epsilon > 0\)
la parte standard di \(f(c+\epsilon)\) è sempre la stessa.
\item 
\(\displaystyle \lim_{x \to c^+}f(x) = +\infty\)
se \(f(c+\epsilon)\) è un numero infinito positivo 
per ogni \(\epsilon \approx 0 \stext{e} \epsilon > 0\).
\item 
\(\displaystyle \lim_{x \to c^+}f(x) = -\infty\)
se \(f(c+\epsilon)\) è un numero infinito negativo 
per ogni \(\epsilon \approx 0 \stext{e} \epsilon > 0\).
\end{itemize}
\end{definizione}
A questo punto possiamo dare una nuova definizione di limite:
\begin{definizione}
Se il limite sinistro e il limite destro hanno lo stesso valore \(L\), 
diremo che questo è il limite della funzione.
\[\displaystyle \text{Se }~
\lim_{x \to c^-}f(x) = \lim_{x \to c^+}f(x) = L \stext{ allora }
\lim_{x \to c}f(x) = L \]
\end{definizione}

\begin{esempio}
Calcola il limite \(\displaystyle \lim_{x \to 0}\frac{-1}{x}\)
\begin{enumerate} %[nosep]
\item 
Possiamo partire auspicandoci di ottenere un limite finito:
\[\lim_{x \to 0}\frac{-1}{x} = \pst{\frac{-1}{\epsilon}} = \pst{M}\]
Niente da fare: non posso calcolare la parte standard di un numero infinito.
\item 
Possiamo allora passare alla seconda parte della definizione per vedere se 
esiste un limite infinito. 

Niente da fare: dato che il numero infinito \(M\) ha segno opposto a 
quello di \(\epsilon\), non ottengo sempre lo stesso valore per ogni 
infinitesimo \(\epsilon\).
\item 
Possiamo studiare se esistono almeno i limiti sinistro o destro: 
questa volta abbiamo fortuna. 

Per ogni \(\epsilon\) negativo l'infinito che 
otteniamo è positivo, per ogni \(\epsilon\) positivo l'infinito che 
otteniamo è negativo. Quindi:
\[\lim_{x \to 0^-}\frac{1}{x} = +\infty \sstext{e} 
\lim_{x \to 0^+}\frac{1}{x} = -\infty\]
\end{enumerate}
\end{esempio}

\subsection{Limiti all'infinito}
\label{subsec:cont_limiti_allinfinito}

% TODO Immagini con telescopi

Abbiamo studiato il comportamento di una funzione che 
% cresce sempre più all'a
che assume valori infiniti quando il suo argomento si avvicina ad un certo 
valore finito. 
Ora vogliamo studiare come si comporta una funzione quando il suo argomento 
è un infinito e scriveremo: 
\[\lim_{x \to -\infty} f(x) \sstext{o} \lim_{x \to +\infty} f(x)\]
Diamo per il limite all'infinito una definizione analoga a quella 
data per i limiti al finito.

\begin{definizione}
\emph{Limite} per \(x\) che tende a \(-\infty\) della funzione \(f(x)\).
\begin{itemize}
\item 
\(\displaystyle \lim_{x \to -\infty}f(x) = \pst{f(M)}\)
se \(f(M)\) è un numero finito e \\
per ogni \(M\) infinito negativo
la parte standard di \(f(M)\) è sempre la stessa.
\item 
\(\displaystyle \lim_{x \to -\infty}f(x) = -\infty\)
se \(f(M)\) è un numero infinito negativo 
per ogni \(M\) infinito negativo.
\item 
\(\displaystyle \lim_{x \to -\infty}f(x) = +\infty\)
se \(f(M)\) è un numero infinito positivo 
per ogni \(M\) infinito negativo.
\end{itemize}
\end{definizione}

\begin{definizione}
\emph{Limite} per \(x\) che tende a \(+\infty\) della funzione \(f(x)\).
\begin{itemize}
\item 
\(\displaystyle \lim_{x \to +\infty}f(x) = \pst{f(M)}\)
se \(f(M)\) è un numero finito e \\
per ogni \(M\) infinito positivo
la parte standard di \(f(M)\) è sempre la stessa.
\item 
\(\displaystyle \lim_{x \to +\infty}f(x) = -\infty\)
se \(f(M)\) è un numero infinito negativo 
per ogni \(M\) infinito positivo.
\item 
\(\displaystyle \lim_{x \to +\infty}f(x) = +\infty\)
se \(f(M)\) è un numero infinito positivo 
per ogni \(M\) infinito positivo.
\end{itemize}
\end{definizione}

\begin{esempio}
Calcola i limiti 
\(\displaystyle \lim_{x \to -\infty}\frac{3x^2-4x+5}{x^2-2x+3}\)
e 
\(\displaystyle \lim_{x \to +\infty}\frac{3x^2-4x+5}{x^2-2x+3}\)

Iniziamo dal primo auspicandoci di ottenere un limite finito:
\[\lim_{x \to -\infty}\frac{3x^2-4x+5}{x^2-2x+3} = 
\pst{\frac{3M^2-4M+5}{M^2-2M+3}} \stackrel{*}{=} 
\pst{\frac{3\cancel{M^2}}{\cancel{M^2}}} = \pst{3} = 3\]
\(*\) questa uguaglianza è giustificata dalla relazione di 
indistinguibilità (vedi il confronto tra infinitesimi).
Per il numeratore: 
\(3M^2-4M+5 \sim 3M^2\) infatti: \\ [.5em]
\(\dfrac{3M^2-4M+5 - 3M^2}{3M^2} = \dfrac{-4M+5}{3M^2} = 
\dfrac{-4M}{3M^2} + \dfrac{+5}{3M^2} = 
\dfrac{-4}{3M} + \dfrac{+5}{3M^2} = \alpha + \beta \approx 0\) \\ [.5em]
e: \\ [.5em]
\(\dfrac{3M^2-4M+5 - 3M^2}{3M^2-4M+5} = 
\dfrac{\dfrac{-4M+5}{M}}{\dfrac{3M^2-4M+5}{M}} = 
\dfrac{\dfrac{-4M}{M}+\dfrac{+5}{M}}
      {\dfrac{3M^2}{M}+\dfrac{-4M}{M}+\dfrac{+5}{M}} = 
\dfrac{-4+\alpha}{3M-4+\beta} \stackrel{*}{\approx} 0\)\\ [.5em]
\(*\) Dato che il numeratore è un finito e il denominatore un infinito.
\\ [.5em]
In modo analogo si dimostra che \(M^2-2M+3 \approx M^2\)
\end{esempio}


\subsection{Calcolare limiti}
\label{subsec:cont_limiti_calcolo}

\begin{procedura}
Per calcolare il limite di una funzione per \(x\) che tende a un certo 
valore, \(c\) finito, basta calcolare la parte standard del valore della 
funzione per un numero infinitamente vicino a \(c\):
\[l=\lim_{x \rightarrow c} f(x) = \pst{f(c+\epsilon)}\]
Se \(c\) è infinito basta calcolare la parte standard della funzione 
calcolata per un valore infinito:
\[l=\lim_{x \rightarrow \infty} f(x) = \pst{f(M)}\]
\end{procedura}

% Seguendo uno schema già incontrato: 
% il valore di partenza \(c\) è un numero reale, 
% poi si usano numeri iperreali \(x\) e \(f(x)\)
% per ottenere la soluzione \(l\) che è, di nuovo, un numero reale.

I prossimi esempi mostrano come è possibile applicare la procedura per il 
calcolo del limite a diversi casi che potrai incontrare.

\begin{esempio}
\textbf{Limite per \(x\) che tende a un numero infinito}:
sostituiremo \(x\) con un generico infinito~\(M\).
\begin{align*}
\lim_{x \rightarrow \infty} \frac{3x^2-3x+7}{5x^2-6} & \stackrel{1}{=} 
  \pst{\frac{3M^2-3M+7}{5M^2-6}} \stackrel{2}{=}  
  \pst{\frac{3\cancel{M^2}}{5\cancel{M^2}}} = \frac{3}{5}
\end{align*}
Dove le uguaglianze hanno i seguenti motivi:
\begin{enumerate} [nosep]
 \item sostituiamo \(x\) con \(M\), un generico iperreale infinito;
 \item sostituiamo l'espressione ottenuta con una espressione 
   indistinguibile, eseguiamo i calcoli e calcoliamo la parte standard.
\end{enumerate}
\end{esempio}

\begin{esempio}
\textbf{Limite per \(x\) che tende ad un valore finito \(c\)}:
in questo caso basta sostituire, nella funzione, la variabile \(x\) con 
\(c+\epsilon\) e poi calcolare la parte standard.
\begin{align*}
\lim_{x \rightarrow 3} \tonda{x^2-4x+2} & \stackrel{1}{=} 
  \pst{\tonda{3+\epsilon}^2-4\tonda{3+\epsilon}+2} \stackrel{2}{=}\\  
  & \stackrel{2}{=} \pst{9 + 6 \epsilon + \epsilon^2 - 12 - 4\epsilon + 2} 
  \stackrel{3}{=} \pst{9 - 12 + 2} \stackrel{4}{=} -1
\end{align*}
Dove le uguaglianze hanno i seguenti motivi:
\begin{enumerate} [nosep]
 \item sostituiamo \(x\) con \(\tonda{3+\epsilon}\);
 \item eseguiamo i calcoli algebrici;
 \item sostituiamo l'espressione con una espressione indistinguibile;
 \item eseguiamo i calcoli e calcoliamo la parte standard.
\end{enumerate}
\end{esempio}

\begin{esempio}
\textbf{Metodo rapido}, ma che non sempre funziona:
\begin{align*}
\lim_{x \rightarrow 3} \tonda{x^2-4x+2} & \stackrel{1}{=} 
  \pst{\tonda{3+\epsilon}^2-4\tonda{3+\epsilon}+2} \stackrel{2}{=}\\ 
  & \stackrel{2}{=}\pst{\tonda{3}^2-4\tonda{3}+2} \stackrel{3}{=}
  \pst{9 - 12 + 2} \stackrel{4}{=} -1
\end{align*}
Dove le uguaglianze hanno i seguenti motivi:
\begin{enumerate} [nosep]
 \item sostituiamo \(x\) con \(\tonda{3+\epsilon}\);
 \item sostituiamo l'espressione con una espressione indistinguibile;
 \item eseguiamo i calcoli algebrici;
 \item eseguiamo i calcoli e calcoliamo la parte standard.
\end{enumerate}
\end{esempio}

\begin{esempio}
\textbf{Limite per \(x\) che tende a 
\mathversion{bold}\(0\)\mathversion{normal}}:
in questo caso i calcoli risultano più semplici dato che
\(0 + \epsilon = \epsilon\).: 
\begin{align*}
\lim_{x \rightarrow 0} \frac{x^2-4x+2}{x^2-4} & \stackrel{1}{=} 
  \pst{\frac{\epsilon^2-4\epsilon+2}{\epsilon^2-4}} \stackrel{2}{=}  
  \pst{\frac{2}{-4}} = -\frac{1}{2}
\end{align*}
Dove le uguaglianze hanno i seguenti motivi:
\begin{enumerate} [nosep]
 \item sostituiamo \(x\) con \(\tonda{0+\epsilon}=\epsilon\);
 \item sostituiamo l'espressione ottenuta con una espressione 
   indistinguibile, eseguiamo i calcoli e calcoliamo la parte standard.
\end{enumerate}
\end{esempio}

Ora vediamo alcuni casi un po' più delicati: ricordiamoci che un numero 
iperreale diverso da zero non può mai essere indistinguibile da zero.

\begin{esempio}
\textbf{Infinitesimo fratto finito non infinitesimo (\emph{i/fni}).}
\begin{align*}
\lim_{x \rightarrow -5} \frac{x+5}{x-3} & \stackrel{1}{=} 
  \pst{\frac{-5+\epsilon+5}{-5+\epsilon-3}} \stackrel{2}{=}  
  \pst{\frac{\epsilon}{\epsilon-8}} \stackrel{3}{=} 
  \pst{\frac{\epsilon}{-8}} \stackrel{4}{=} \pst{\delta} = 0
\end{align*}
Dove le uguaglianze hanno i seguenti motivi:
\begin{enumerate} [nosep]
 \item sostituiamo \(x\) con \(-5+\epsilon\);
 \item eseguiamo i calcoli;
 \item sostituiamo l'espressione ottenuta con una espressione 
   indistinguibile;
 \item un infinitesimo diviso un finito non infinitesimo dà come risultato 
un infinitesimo e la sua parte standard è zero.
\end{enumerate}
\end{esempio}

\begin{esempio}
\textbf{Finito non infinitesimo fratto infinitesimo (\emph{fni/i}).}
\begin{align*}
\lim_{x \rightarrow 4} \frac{2x+6}{x-4} & \stackrel{1}{=} 
  \pst{\frac{2\tonda{4+\epsilon}+6}{4+\epsilon-4}} \stackrel{2}{=}  
  \pst{\frac{14 + 2 \epsilon}{\epsilon}} \stackrel{3}{=} 
  \pst{\frac{14}{\epsilon}} \stackrel{4}{=} 
  \pst{M} \stackrel{5}{~\longrightarrow~} \infty
\end{align*}
Dove le uguaglianze hanno i seguenti motivi:
\begin{enumerate} [nosep]
 \item sostituiamo \(x\) con \(4+\epsilon\);
 \item eseguiamo i calcoli;
 \item sostituiamo l'espressione ottenuta con una espressione 
   indistinguibile;
 \item un finito fratto un infinitesimo dà come risultato un infinito; 
 \item gli infiniti non hanno parte standard quindi qui non possiamo usare 
l'``\(=\)''. Ma i matematici per indicare un numero più grande di qualunque 
altro numero usano il simbolo ``\(\infty\)'', quindi invece dell'uguale 
tracceremo una freccia.
\end{enumerate}
\end{esempio}

\begin{esempio}
\textbf{Funzione irrazionale, con radici quadrate}.
\begin{align*}
\lim_{x \rightarrow \infty} \tonda{2x-\sqrt{4x^2-8x+3}} & \stackrel{1}{=} 
  \pst{2M-\sqrt{4M^2-8M+3}} \stackrel{2}{=} \\
  &=\pst{2M-\sqrt{4M^2}} \stackrel{3}{=} 
  \pst{2M-2M} = \pst{0} = 0
\end{align*}
Dove le uguaglianze hanno i seguenti motivi:
\begin{enumerate} [nosep]
 \item sostituiamo \(x\) con \(M\);
 \item sostituiamo l'espressione ottenuta con una espressione 
   indistinguibile;
 \item eseguiamo i calcoli \dots
\end{enumerate}
\vspace{1em}
\begin{minipage}{.69\textwidth}
Ma questa volta nei ragionamenti fatti c'è un errore. Proviamo a calcolare 
la funzione per alcuni valori abbastanza grandi di \(x\).\\
I risultati dovrebbero avvicinarsi a zero, ma non è così, sembra si 
avvicinino, invece, a due. Dove abbiamo sbagliato?
\end{minipage}
\begin{minipage}{.39\textwidth}
\begin{center}
\begin{tabular}{r|r}
x & y\\\hline
100 & 2.00252 \\
1000 & 2.00025 \\
10000 & 2.00002 \\
\end{tabular}
\end{center}
\end{minipage}\\

Abbiamo usato in modo improprio la relazione \emph{indistinguibile}: 
abbiamo ottenuto un'espressione indistinguibile da zero, ma ciò non è 
possibile (vedi la definizione di indistinguibile) \dots \\
Dobbiamo seguire un'altra strada.

Consideriamo l'espressione data come una frazione e razionalizziamo il 
numeratore:
\begin{align*}
\lim_{x \rightarrow \infty} 2x-\sqrt{4x^2-8x+3}=
&\pst{\frac{\tonda{2M}-\sqrt{4M^2-8M+3}}{1} \cdot 
\frac{\tonda{2M}+\sqrt{4M^2-8M+3}}{\tonda{2M}+\sqrt{4M^2-8M+3}}}=\\
&=\pst{\frac{\cancel{4M^2}-\cancel{4M^2}+8M-3}{2M+\sqrt{4M^2-8M+3}}}=
\end{align*}
Questa volta possiamo sostituire l'espressione sotto radice con 
un'espressione indistinguibile, senza ottenere zero:
\[=\pst{\frac{8M-3}{2M+\sqrt{4M^2-8M+3}}} =
   \pst{\frac{8M}{2M+\sqrt{4M^2}}}=\]
e svolgendo i calcoli otteniamo:
\[=\pst{\frac{8M}{2M+2M}}=
   \pst{\frac{8\cancel{M}}{4\cancel{M}}} = 2\]
Ottenendo un risultato in accordo con l'andamento della funzione.
\end{esempio}

\begin{esempio}
\textbf{Funzione razionale}: metodo rapido ma non sempre funzionante.
\begin{align*}
\lim_{x \rightarrow 2} \frac{x^3+3x^2+2x}{x^2-x-6} & \stackrel{1}{=} 
\pst{\frac
  {\tonda{2+\epsilon}^3+3\tonda{2+\epsilon}^2+2\tonda{2+\epsilon}}
  {\tonda{2+\epsilon}^2-\tonda{2+\epsilon}-6}} \stackrel{2}{=}\\ 
  &=\frac{2^3+3\cdot 2^2+2\cdot 2}{2^2-2-6} =
  \pst{\frac{8+12+4}{4-2-6}} \stackrel{3}{=} \pst{\frac{24}{-4}} =-6
\end{align*}
Dove le uguaglianze hanno i seguenti motivi:
\begin{enumerate} [nosep]
 \item sostituiamo \(x\) con un numero infinitamente vicino a \(+2\);
 \item sostituiamo le espressioni tra parentesi con espressioni 
indistinguibili;
 \item poiché i risultati ottenuti sono diversi da zero, il risultato è 
effettivamente indistinguibile. Quindi il risultato ottenuto è valido.
\end{enumerate}
\end{esempio}

Ora vediamo un caso in cui il metodo precedente non funziona. Prima di 
affrontare l'esempio ricordiamoci che se non abbiamo maggiori informazioni 
sugli infinitesimi \(\alpha\) e \(\beta\), non possiamo calcolare 
\(\frac{\alpha}{\beta}\).

\begin{esempio}
\textbf{Funzione razionale}: metodo rapido ma inconcludente.
\begin{align*}
\lim_{x \rightarrow -2} \frac{x^3+3x^2+2x}{x^2-x-6} & = 
\pst{\frac
  {\tonda{-2}^3+3\tonda{-2}^2+2\tonda{-2}}
  {\tonda{-2}^2-\tonda{-2}-6}} =\\ 
  &=\pst{\frac{-8+12-4}
              {4+2-6}} = \pst{\frac{0}{0}} = \dots
\end{align*}
Ma qui ci scontriamo con 2 problemi: abbiamo usato la relazione di 
indistinguibile con lo zero, e abbiamo ottenuto una divisione per zero che 
non è definita. Potremmo essere un po' più pignoli ma ancora troppo 
grossolani osservando che quando \(x\) si avvicina a \(-2\) il numeratore e 
il denominatore si avvicinano a zero, sono cioè degli infinitesimi e quindi 
otteniamo: \(\frac{\alpha}{\beta}\) ma anche questa maggior precisione non 
è sufficiente.

Metodo lungo ma sicuro:
dobbiamo rimboccarci le maniche e affrontare il calcolo algebrico 
ricordandoci che: \\
\(\tonda{x+a}^3= x^3+3x^2a+3xa^2+a^3\)

\begin{align*}
\lim_{x \rightarrow -2} \frac{x^3+3x^2+2x}{x^2-x-6} & \stackrel{1}{=} 
\pst{\frac
  {\tonda{-2+\epsilon}^3+3\tonda{-2+\epsilon}^2+2\tonda{-2+\epsilon}}
  {\tonda{-2+\epsilon}^2-\tonda{-2+\epsilon}-6}} \stackrel{2}{=}\\ 
  &\stackrel{2}{=}\pst{\frac{-8+12\epsilon-6\epsilon^2+\epsilon^3+
             3\tonda{4-4\epsilon+\epsilon^2}-4+2\epsilon}
             {4-4\epsilon+\epsilon^2-\tonda{-2+\epsilon}-6}} =\\ 
  &=\pst{\frac{\cancel{-8}+\cancel{12\epsilon}-6\epsilon^2+\epsilon^3+
          \cancel{12}\cancel{-12\epsilon}+3\epsilon^2\cancel{-4}+2\epsilon}
             {\cancel{4}-4\epsilon+
              \epsilon^2\cancel{+2}-\epsilon\cancel{-6}}}=\\ 
  &=\pst{\frac{\epsilon^3+2\epsilon}{\epsilon^2-5\epsilon}} = 
    \pst{\frac{\cancel{\epsilon} \tonda{\epsilon^2+2}}
             {\cancel{\epsilon} \tonda{\epsilon-5}}}  \stackrel{3}{=} 
    \pst{\frac{2}{-5}} = -\frac{2}{5}
\end{align*}
Dove le uguaglianze hanno i seguenti motivi:
\begin{enumerate} [nosep]
 \item sostituiamo \(x\) con \(-2+\epsilon\);
 \item eseguiamo tutti i calcoli e semplifichiamo;
 \item sostituiamo l'espressione con una indistinguibile.
\end{enumerate}
\end{esempio}

\begin{osservazione}
Pensate alla complicazione dei calcoli se ci fosse un qualche \(x^4\) o
\(x^5\)\dots
Vedremo ora un altro modo di calcolare il limite che risulta meno 
complicato.
\end{osservazione}\\

Prima di affrontare il prossimo metodo ricordiamoci che possiamo 
rappresentare tutti gli infinitesimi di ordine superiore ad un certo 
infinitesimo \(\epsilon\) nel simbolo: \(o\tonda{\epsilon}\).

\begin{esempio}
\textbf{Funzione razionale}: altro metodo.
\begin{align*}
\lim_{x \rightarrow -2} \frac{x^3+3x^2+2x}{x^2-x-6} & \stackrel{1}{=} 
\pst{\frac
  {\tonda{-2+\epsilon}^3+3\tonda{-2+\epsilon}^2+2\tonda{-2+\epsilon}}
  {\tonda{-2+\epsilon}^2-\tonda{-2+\epsilon}-6}} \stackrel{2}{=}\\ 
  &\stackrel{2}{=}\pst{\frac{-8+12\epsilon+12-12\epsilon-4+2\epsilon+
                             o\tonda{\epsilon}}
                           {4-4\epsilon+2-\epsilon-6+o\tonda{\epsilon}
                           }}=\\ 
  &=\pst{\frac{\cancel{-8}+\cancel{12\epsilon}+
               \cancel{12}\cancel{-12\epsilon}\cancel{-4}+2\epsilon+
               o\tonda{\epsilon}}
              {\cancel{4}-4\epsilon\cancel{+2}-\epsilon\cancel{-6}+
               o\tonda{\epsilon}}}\stackrel{3}{=}
    \pst{\frac{2\cancel{\epsilon}}{-5\cancel{\epsilon}}} = -\frac{2}{5}
\end{align*}
Dove le uguaglianze hanno i seguenti motivi:
\begin{enumerate} [nosep]
 \item sostituiamo \(x\) con \(-2+\epsilon\);
 \item eseguiamo i calcoli riunendo in un unico simbolo tutti gli 
infinitesimi di ordine superiore a \(\epsilon\) e perciò trascurabili 
(si spera);
 \item sostituiamo l'espressione con una indistinguibile poi calcoliamo la 
parte standard.
\end{enumerate}
\end{esempio}

% \begin{esempio}
% \textbf{Funzione razionale}: altro metodo.
% \begin{align*}
% \lim_{x \rightarrow 2} \frac{x^3+3x^2+2x}{x^2-x-6} & \stackrel{1}{=} 
% \tonda{\pst{\frac
%   {\tonda{-2+\epsilon}^3+3\tonda{-2+\epsilon}^2+2\tonda{-2+\epsilon}}
%   {\tonda{-2+\epsilon}^2-\tonda{-2+\epsilon}-6}} \stackrel{2}{=}}\\ 
%   &=\pst{\frac{-8+12-4}{4+2-6}} = \pst{\frac{0}{0}} 
%   \text{ Indistinguibile non è applicabile.}
% \end{align*}
% 
% È evidente che \(-2\) è uno zero sia del numeratore sia del denominatore 
% quindi, per il teorema di Ruffini, entrambi questi polinomi sono 
% divisibili per \(x+2\).
% Possiamo scomporre i due polinomi, semplificarli procedendo nel seguente 
% modo:
% 
% \begin{align*}
% \lim_{x \rightarrow -2} \frac{x^3+3x^2+2x}{x^2-x-6} &=
% \lim_{x \rightarrow -2} \frac{x \tonda{x+1} \cancel{\tonda{x+2}}}
%           {\tonda{x-3} \cancel{\tonda{x+2}}} \stackrel{1}{=}
% \lim_{x \rightarrow -2} \frac{x^2+x}{x-3}=\\
% &=\tonda{\pst{\frac{\tonda{-2+\epsilon}^2-\tonda{2+\epsilon}}
%                    {\tonda{-2+\epsilon}-3}}}\stackrel{2}{=}
% \pst{\frac{\tonda{-2}^2-2}{-2-3}}=
% \pst{\frac{2}{-5}} = -\frac{2}{5}
% \end{align*}
% Dove le uguaglianze hanno i seguenti motivi:
% \begin{enumerate} [nosep]
%  \item possiamo semplificare perché \(x\) è infinitamente vicino a \(-2\) 
% ma è diverso da \(-2\);
%  \item sostituiamo l'espressione con una indistinguibile operazione, 
% questa volta, applicabile perché le due espressioni sono diverse da zero.
% \end{enumerate}
% \end{esempio}

\begin{osservazione}
Abbiamo ottenuto lo stesso valore ricavato più sopra facendo tutti i 
calcoli algebrici, ma i passaggi sono più semplici.
\end{osservazione}


\subsection{Limiti notevoli}
\label{subsec:cont_limiti_notevoli}

Ci sono alcuni limiti che hanno delle dimostrazioni particolari e che è 
utile conoscere per poter risolvere dei casi particolari.

\subsubsection{Funzioni goniometriche}

Se consideriamo le funzioni seno, coseno e tangente, possiamo vedere che 
alcuni limiti risultano banali, ma altri sono interessanti e possono essere 
dimostrati.

\begin{minipage}{.66\textwidth}
\begin{center} \sinusoide \end{center}
% \begin{center} \sincos{sin(x)}{$y=\sen x$}{Blue} \end{center}
\begin{center} \cosinusoide \end{center}
\end{minipage}
\hfill
\begin{minipage}{.33\textwidth}
\begin{center} \tangentoide \end{center}
\end{minipage}

\paragraph{Funzione seno}~

Osservando il grafico della funzione seno, risulta abbastanza evidente che:
\[\lim_{x \rightarrow 0}{\sen x} = \pst{\sen \epsilon} = 0
\quad \text{e che} \quad 
\lim_{x \rightarrow \infty}{\sen x} = \pst{\sen M} = \text{ non esiste}\]
Infatti man mano che \(x\) aumenta, \(\sen \)
continua a variare tra \(-1\) e \(+1\) e quindi 
\(\sen M\) non può essere infinitamente vicino ad un solo numero reale.

Ma il seguente è un limite notevole di cui proponiamo una dimostrazione 
grafica:
\[\lim_{x \rightarrow 0}\frac{\sen x}{x} = 
  \pst{\frac{\sen \delta}{\delta}} = 1\]
\begin{minipage}{.49\textwidth}
Una circonferenza non ha differenze reali da un poligono regolare di 
infiniti lati quindi un arco infinitesimo differisce dalla corda 
sottesa per infinitesimi di ordine superiore alle loro lunghezze. Perciò il 
loro rapporto è \(1\).
Se rappresentiamo in una circonferenza goniometrica un arco di ampiezza 
infinitesima e il corrispondente seno, appaiono non distinguibili ad ogni 
ingrandimento finito. Se li ingrandiamo con un microscopio non standard con 
infiniti ingrandimenti, possiamo vedere che sono ancora indistinguibili 
cioè differiscono per infinitesimi di ordine superiore. 

Quindi il loro rapporto è \(1\).
\end{minipage}
\hfill
\begin{minipage}{.49\textwidth}
\begin{center} \limiteseno \end{center}
\end{minipage}
Questo significa che \(\sen \delta \text{ e } \delta\) sono indistinguibili 
cioè \(\sen \delta \text{ è diverso da } \delta\) solo per infinitesimi di 
ordine superiore a \(\delta\).

\paragraph{Funzione tangente}~

Anche nella funzione tangente valgono i seguenti limiti:
\[\lim_{x \rightarrow 0}{\tan x} = \pst{} = 0; \quad
\lim_{x \rightarrow \infty}{\tan x} = \pst{\tan M} = 
                                      \text{ non esiste}; \quad
\lim_{x \rightarrow 0}\frac{\tan x}{x} = \pst{\frac{\tan \delta}{\delta}} 
                                       = 1\]
per considerazioni analoghe a quelle fatte per la funzione seno.

\paragraph{Funzione coseno}~

Per la funzione coseno valgono i seguenti limiti:
\[\lim_{x \rightarrow 0}{\cos x} = \pst{\cos \delta} = 1; ~
\lim_{x \rightarrow 0}\tonda{1-\cos x} = \pst{1-\cos \delta} = 0; ~
\lim_{x \rightarrow \infty}{\cos x} = \pst{\cos M} = \text{ non es.}\]

Sono interessanti i seguenti due limiti che riguardano il coseno:

\begin{align*}
 \lim_{x \rightarrow 0} \frac{1-\cos x}{x} &=
 \pst{\frac{1-\cos \delta}{\delta}}
~ \stackrel{1}{=} ~  
 \pst{\frac{1-\cos \delta}{\delta} \cdot 
      \frac{1+\cos \delta}{1+\cos \delta}}
~ \stackrel{2}{=} ~ 
 \pst{\frac{1-\cos^2 \delta}{\delta \tonda{1+\cos \delta}}}~ 
\stackrel{3}{=} \\
& \stackrel{3}{=} ~
 \pst{\frac{\sen^2 \delta}{\delta \tonda{1+\cos \delta}}}
~ \stackrel{4}{=}
 \pst{\frac{\sen \delta}{\delta} \cdot 
      \frac{\sen \delta}{1+\cos \delta}}
~ \stackrel{5}{=}
 \pst{1 \cdot \frac{\delta}{2}} = 0
\end{align*}
Dove le uguaglianze hanno i seguenti motivi:
\begin{enumerate} [nosep]
 \item moltiplico la funzione per una frazione equivalente a 1;
 \item prodotto notevole;
 \item ricordando che \(\sen^x + \cos^2 x = 1\);
 \item un infinitesimo fratto un non infinitesimo è un infinitesimo 
e la sua parte standard è 0.
\end{enumerate}

\begin{osservazione}
Dato che \(\sen \delta \sim \delta\), i seguenti due limiti sono 
equivalenti:
\[\lim_{x \rightarrow 0} \frac{1-\cos x}{x} =
 \pst{\frac{1-\cos \delta}{\delta}} =
 \pst{\frac{1-\cos \delta}{\sen \delta}} =
 \lim_{x \rightarrow 0} \frac{1-\cos x}{\sen x}\]
\end{osservazione}

L'altro limite notevole, che si dimostra in modo analogo a quello 
precedente, è:
\begin{align*}
 \lim_{x \rightarrow 0} \frac{1-\cos x}{x^2} &=
 \pst{\frac{1-\cos \delta}{\delta^2}}
~ \stackrel{1}{=} ~  
 \pst{\frac{1-\cos \delta}{\delta^2} \cdot 
      \frac{1+\cos \delta}{1+\cos \delta}}
~ \stackrel{2}{=} ~ 
 \pst{\frac{1-\cos^2 \delta}{\delta^2 \tonda{1+\cos \delta}}}~ 
\stackrel{3}{=} \\
& \stackrel{3}{=} ~
 \pst{\frac{\sen^2 \delta}{\delta^2 \tonda{1+\cos \delta}}}
~ \stackrel{4}{=}
 \pst{\tonda{\frac{\sen \delta}{\delta}}^2 \cdot 
      \frac{1}{\tonda{1+\cos \delta}}}
~ \stackrel{5}{=}
 \pst{1 \cdot \frac{1}{2}} = \frac{1}{2}
\end{align*}
Dove le uguaglianze hanno i seguenti motivi:
\begin{enumerate} [nosep]
 \item moltiplico la funzione per una frazione equivalente a 1;
 \item prodotto notevole;
 \item ricordando che \(\sen^x + \cos^2 x = 1\);
 \item riscrivo l'espressione in un modo più comodo;
 \item ricordando i limiti visti precedentemente.
\end{enumerate}

\subsubsection{Esponenziali e logaritmi}

\paragraph{Numero di Eulero (o di Nepero)}

Ricordiamo come è definita la costante di Eulero:
\[e = \pst{\tonda{1+\frac{1}{M}}^M} = 
\pst{\tonda{1+\delta}^\frac{1}{\delta}} 
\]
\[e=\sum_{n=0}^{\infty}{\frac{1}{n!}}=
\frac{1}{1}+\frac{1}{1}+\frac{1}{2}+\frac{1}{2\cdot3}+
\frac{1}{2\cdot3\cdot4}+\frac{1}{2\cdot3\cdot4\cdot5}+\dots\]
Le definizioni sono equivalenti, mentre la seconda definizione risulta 
molto efficiente per il calcolo, la 
prima ha delle interessanti applicazioni matematiche.

\begin{esempio}
\label{esempio:log}
Limite di una particolare funzione logaritmica:
\begin{align*}
 \lim_{x \rightarrow 0} \dfrac{\ln\tonda{1+x}}{x} &=
 \pst{\dfrac{\ln\tonda{1+\delta}}{\delta}}~ = ~  
 \pst{\frac{1}{\delta}\ln\tonda{1+\delta}} ~\stackrel{1}{=} ~
 \pst{\ln\tonda{1+\delta}^\frac{1}{\delta}}
~ \stackrel{2}{=} ~
\pst{\ln\tonda{e}} \stackrel{3}{=} ~ 1
\end{align*}
Dove le uguaglianze hanno i seguenti motivi:
\begin{enumerate} [nosep]
 \item per una proprietà dei logaritmi;
 \item l'argomento del logaritmo è proprio la definizione di \(e\);
 \item l'esponente da dare a \(e\) per ottenere \(e\) è \(1\).
\end{enumerate}
\end{esempio}

\begin{esempio}
Come nell'esempio precedente ma con una base generica:
\begin{align*}
 \lim_{x \rightarrow 0} \dfrac{\log_a\tonda{1+x}}{x} &=
 \pst{\dfrac{\log_a\tonda{1+\delta}}{\delta}}~\stackrel{1}{=} ~  
 \pst{\dfrac{\frac{\ln\tonda{1+\delta}}{\ln a}}{\delta}}~\stackrel{1}{=} ~
 \pst{\dfrac{\ln\tonda{1+\delta}}{\delta}\cdot \dfrac{1}{\ln a}}
 ~ \stackrel{2}{=} ~
 \dfrac{1}{\ln a}
\end{align*}
Dove le uguaglianze hanno i seguenti motivi:
\begin{enumerate} [nosep]
 \item cambio di base del logaritmo;
 \item per quanto visto nell'esempio \ref{esempio:log}.
\end{enumerate}
\end{esempio}

\begin{esempio}
Limite di una particolare funzione esponenziale:
\begin{align*}
\lim_{x \rightarrow 0} \dfrac{e^x-1}{x} &=
\pst{\dfrac{e^\epsilon-1}{\epsilon}}
~ \stackrel{1}{=} ~  
\pst{\dfrac{\delta}{\ln{(\delta+1)}}}~ \stackrel{2}{=} ~ 1
\end{align*}
Dove le uguaglianze hanno i seguenti motivi:
\begin{enumerate} [nosep]
 \item ancora una sostituzione: poniamo \(e^\epsilon-1=\delta\). 
Allora \(e^\epsilon = 1+\delta\) quindi \(\epsilon\) è l'esponente da dare 
a \(e\) per ottenere \(1+\delta\) cioè: 
\(\epsilon = \ln{\tonda{1+\delta}}\);
 \item per quanto visto nell'esempio \ref{esempio:log}.
\end{enumerate}
\end{esempio}

\begin{esempio}
Simile all'esempio precedente, ma con una base generica.
\begin{align*}
\lim_{x \rightarrow 0} \dfrac{a^x-1}{x} &=
\pst{\dfrac{a^\epsilon-1}{\epsilon}}
~ \stackrel{1}{=} ~  
\pst{\dfrac{\delta}{\log_a{(\delta+1)}}}~ \stackrel{2}{=}\\
&\stackrel{2}{=} 
\pst{\frac{\delta}{\dfrac{\ln{(\delta+1)}}{\ln a}}} ~=~
\pst{\frac{\delta \cdot \ln a}{\ln{(\delta+1)}}} %\stackrel{3}{=}\\
\stackrel{3}{=} ~ 
\pst{1 \cdot \ln a}=\ln{a}
\end{align*}

% \newpage   %----------------------------------------------------
Dove le uguaglianze hanno i seguenti motivi:
\begin{enumerate} [nosep]
 \item ancora una sostituzione: poniamo
\(a^\epsilon-1=\delta\), allora \(\epsilon=\log_a(\delta+1)\);
 \item applichiamo la formula del cambiamento di base di un logaritmo; 
 \item per quanto visto nell'esempio \ref{esempio:log}.
\end{enumerate}
\end{esempio}

\begin{esempio}
Limite di una particolare funzione esponenziale:
\begin{align*}
 \lim_{x \rightarrow \infty} \tonda{1+\dfrac{k}{x}}^x =
 \pst{\tonda{1+\dfrac{k}{N}}^N}
~ \stackrel{1}{=} ~  
\pst{\tonda{1+\dfrac{1}{M}}^{kM}}
~ \stackrel{2}{=} ~
\pst{\quadra{\tonda{1+\dfrac{1}{M}}^M}^k}
~ \stackrel{3}{=} ~ e^k
\end{align*}
Dove le uguaglianze hanno i seguenti motivi:
\begin{enumerate} [nosep]
 \item se al posto di \(\dfrac{k}{N}\) scrivo \(\dfrac{1}{M}\) 
allora al posto di \(N\) dovrò scrivere \(N=kM\), 
% infatti:
% \(\dfrac{k}{N}=\dfrac{1}{M} \sLRarrow N=kM\);
 \item la potenza di potenza è una potenza che ha per base la stessa base 
e per \dots
 \item l'espressione tra parentesi quadre è proprio la definizione di \(e\).
\end{enumerate}
\end{esempio}

