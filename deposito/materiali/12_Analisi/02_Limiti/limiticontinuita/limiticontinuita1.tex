% (c) 2015 Daniele Zambelli daniele.zambelli@gmail.com

\input{\folder limiticontinuita1_grafici.tex}

\chapter{Funzioni: limiti e continuità}

\section{Limiti}
\label{sec:cont_limiti}

\footnote{Per scrivere questo capitolo mi sono ispirato 
al testo di Howard Jerome Keisler, 
``Elementary Calculus: An Infinitesimal Approach''. 
Chi volesse approfondire l'argomento può scaricare il testo all'indirizzo: 
\href{https://www.math.wisc.edu/~keisler/calc.html}
     {www.math.wisc.edu/\(\sim\)keisler/calc.html}}
In alcuni problemi non siamo interessati a sapere come si comporta una funzione 
per un valore ben preciso, ci interessa di più 
sapere come si comporta quando la variabile \(x\) è \emph{infinitamente 
vicina} a quel valore.
Per un certo valore di \(x\) la funzione potrebbe anche \emph{non} essere 
definita, ma cosa succede quando \(x\) si avvicina infinitamente a quel 
valore?

Vediamo un esempio:

\begin{esempio}
 Studia l'Insieme di Definizione della funzione: 
 \(\dfrac{x^2-6x+5}{x^2+2x-3}\)
 Poi studia come si comporta la funzione per valori infinitamente vicini 
agli estremi dell'Insieme di Definizione.

La funzione è fratta e quindi non è definita quando il denominatore vale 
zero:
\[x^2+2x-3=0 \sRarrow \tonda{x+3}\tonda{x-1}=0 \sRarrow x_1=-3;~x_2=+1\]
L'insieme di definizione è formato quindi dai seguenti intervalli:
\[\intervaa{-\infty}{-3} \quad \cup \quad \intervaa{+1}{-\infty}\]
Calcoliamo alcuni valori della funzione ``vicini'' ai 4 estremi 
dell'Insieme di Definizione.

\begin{minipage}{.24\textwidth}
\begin{center}
\(-\infty\)\\
\begin{tabular}{r|r}
x & y\\\hline
-100 & 1.0824 \\
-1000 & 1.0080 \\
-10000 & 1.0008 \\
\dots \\
&\\
&\\
&
\end{tabular}
\end{center}
\end{minipage}
\begin{minipage}{.24\textwidth}
\begin{center}
\(-3\)\\
\begin{tabular}{r|r}
x & y\\\hline
-2.9 & -79.0 \\
-2.99 & -799.0 \\
-2.999 & -7999.0 \\
\dots \\
-3.001 & 8001.0 \\
-3.01 & 801.0 \\
-3.1 & 80.9 \\
\end{tabular}
\end{center}
\end{minipage}
\begin{minipage}{.24\textwidth}
\begin{center}
\(+1\)\\
\begin{tabular}{r|r}
x & y\\\hline
0.9 & -1.0512\\
0.99 & -1.0050 \\
0.999 & -1.0005 \\
\dots \\
1.001 & -0.9995 \\
1.01 & -0.9950 \\
1.1 & -0.9512 \\
\end{tabular}
\end{center}
\end{minipage}
\begin{minipage}{.24\textwidth}
\begin{center}
\(+\infty\)\\
\begin{tabular}{r|r}
x & y\\\hline
100 & 0.9223 \\
1000 & 0.9920 \\
10000 & 0.9992 \\
\dots \\
&\\
&\\
&
\end{tabular}
\end{center}
\end{minipage}
Possiamo osservare che:
\begin{itemize} [nosep]
 \item 
Quando \(x\) si avvicina a meno infinito, \(f(x)\) si avvicina a~\(1\) dall'alto.
 \item 
Quando \(x\) si avvicina a~\(-3\) da sinistra, \(f(x)\) diventa sempre più 
piccolo.
 \item 
Quando \(x\) si avvicina a~\(-3\) da destra, \(f(x)\) diventa sempre più 
grande.
 \item 
Quando \(x\) si avvicina a~\(+1\) da sinistra, \(f(x)\)  si avvicina a~\(-1\) dal 
basso.
 \item 
Quando \(x\) si avvicina a~\(+1\) da sinistra, \(f(x)\)  si avvicina a~\(-1\) 
dall'alto.
 \item 
Quando \(x\) si avvicina a più infinito, \(f(x)\) si avvicina a~\(1\) dal basso.
\end{itemize}
\begin{center}\scalebox{.6}{\limitigraficoa}\end{center}
\end{esempio}

Osservando il grafico provate a descrivere a parole cosa succede:

\begin{minipage}{.64\textwidth}
\begin{itemize} [nosep]
 \item quando \(x\) si avvicina a \(-\infty\); 
 \item quando si avvicina a \(-3\) e quando vale proprio \(-3\);
 \item quando si avvicina a \(+1\) e quando vale proprio \(+1\); 
 \item e, infine, quando si avvicina a \(+\infty\).
\end{itemize}

\begin{osservazione}
Nel tratto crescente della funzione, che appare come linea continua, in 
realtà manca un punto: il punto \(\punto{1}{-1}\). 
Questo punto mancante è invisibile, quindi il grafico della funzione appare 
continuo. 
Possiamo indicare un punto mancante in una linea con un cerchietto vuoto 
come nell'ingrandimento a fianco.
\end{osservazione}

\end{minipage}
\hfill
\begin{minipage}{.34\textwidth}
\begin{center}\scalebox{1.2}{\limitigraficob}\end{center}
\end{minipage}

\vspace{1em}
Prima della nascita dell'analisi non standard, per trattare queste 
situazioni i matematici si sono inventati il concetto di 
\emph{limite}.

\begin{definizione}
Il \textbf{limite} di una funzione \(f(x)\) per \(x\) 
che tende a un valore \(c\) è \(l\),
se, quando \(x\) è infinitamente vicino a \(c\), 
ma diverso da \(c\), 
allora \(f(x)\) è infinitamente vicino a \(l\). E si scrive:
\[\lim_{x \rightarrow c} f(x) = l \Leftrightarrow 
\forall x \tonda{\tonda{x \approx c \wedge x \neq c} \Rightarrow 
\tonda{f(x) \approx l}}\]
\end{definizione}
\begin{osservazione}
~

Seguendo uno schema già incontrato: 
il valore di partenza \(c\) è un numero reale, 
poi si usano numeri iperreali \(x\) e \(f(x)\)
per ottenere la soluzione \(l\) che è, di nuovo, un numero reale.

\textbf{Esiste il limite} solo se per \textbf{qualunque} numero 
infinitamente 
vicino a \(c\), \(f(x)\) è infinitamente vicino ad \textbf{un solo} valore 
reale \(l\).
\footnote{Possiamo inserire queste osservazioni nella definizione ottenendo:
\begin{definizione}
Il numero reale \(l\) è il \textbf{limite} di una funzione reale \(f(x)\) 
per \(x\) che tende a un valore reale \(c\), se, 
quando \(x\) è un qualunque numero iperreale infinitamente vicino a \(c\), 
ma diverso da \(c\), 
allora il valore della corrispondente funzione iperreale \({}^*f(x)\) è 
infinitamente vicino a \(l\).
\end{definizione}}
\end{osservazione}

\subsection{Calcolare limiti}
\label{subsec:cont_limiti_calcolo}

\begin{procedura}
Per calcolare il limite di una funzione per \(x\) che tende a un certo 
valore, \(c\) finito, basta calcolare la parte standard del valore della 
funzione per un numero infinitamente vicino a \(c\):
\[l=\lim_{x \rightarrow c} f(x) = \pst{f(c+\epsilon)}\]
Se \(c\) è infinito basta calcolare la parte standard della funzione 
calcolata per un valore infinito:
\[l=\lim_{x \rightarrow \infty} f(x) = \pst{f(M)}\]
\end{procedura}

I prossimi esempi mostrano come è possibile applicare la procedura per il 
calcolo del limite a diversi casi che potrai incontrare.

\begin{esempio}
\textbf{Limite per \(x\) che tende ad un valore finito \(c\)}:
in questo caso basta sostituire, nella funzione, la variabile \(x\) con 
\(c+\epsilon\) e poi calcolare la parte standard.
\begin{align*}
\lim_{x \rightarrow 3} \tonda{x^2-4x+2} & \stackrel{1}{=} 
  \pst{\tonda{3+\epsilon}^2-4\tonda{3+\epsilon}+2} \stackrel{2}{=}\\  
  & \stackrel{2}{=} \pst{9 + 6 \epsilon + \epsilon^2 - 12 - 4\epsilon + 2} 
  \stackrel{3}{=} \pst{9 - 12 + 2} \stackrel{4}{=} -1
\end{align*}
Dove le uguaglianze hanno i seguenti motivi:
\begin{enumerate} [nosep]
 \item sostituiamo \(x\) con \(\tonda{3+\epsilon}\);
 \item eseguiamo i calcoli algebrici;
 \item sostituiamo l'espressione con una espressione indistinguibile;
 \item eseguiamo i calcoli e calcoliamo la parte standard.
\end{enumerate}
\end{esempio}

\begin{esempio}
\textbf{Metodo rapido}, ma che non sempre funziona:
\begin{align*}
\lim_{x \rightarrow 3} \tonda{x^2-4x+2} & \stackrel{1}{=} 
  \pst{\tonda{3+\epsilon}^2-4\tonda{3+\epsilon}+2} \stackrel{2}{=}\\ 
  & \stackrel{2}{=}\pst{\tonda{3}^2-4\tonda{3}+2} \stackrel{3}{=}
  \pst{9 - 12 + 2} \stackrel{4}{=} -1
\end{align*}
Dove le uguaglianze hanno i seguenti motivi:
\begin{enumerate} [nosep]
 \item sostituiamo \(x\) con \(\tonda{3+\epsilon}\);
 \item sostituiamo l'espressione con una espressione indistinguibile;
 \item eseguiamo i calcoli algebrici;
 \item eseguiamo i calcoli e calcoliamo la parte standard.
\end{enumerate}
\end{esempio}

\begin{esempio}
\textbf{Limite per \(x\) che tende a 
\mathversion{bold}\(0\)\mathversion{normal}}:
in questo caso i calcoli risultano più semplici dato che
\(0 + \epsilon = \epsilon\).: 
\begin{align*}
\lim_{x \rightarrow 0} \frac{x^2-4x+2}{x^2-4} & \stackrel{1}{=} 
  \pst{\frac{\epsilon^2-4\epsilon+2}{\epsilon^2-4}} \stackrel{2}{=}  
  \pst{\frac{2}{-4}} = -\frac{1}{2}
\end{align*}
Dove le uguaglianze hanno i seguenti motivi:
\begin{enumerate} [nosep]
 \item sostituiamo \(x\) con \(\tonda{0+\epsilon}=\epsilon\);
 \item sostituiamo l'espressione ottenuta con una espressione 
   indistinguibile, eseguiamo i calcoli e calcoliamo la parte standard.
\end{enumerate}
\end{esempio}

\begin{esempio}
\textbf{Limite per \(x\) che tende a un numero infinito}:
sostituiremo \(x\) con un generico infinito~\(M\).
\begin{align*}
\lim_{x \rightarrow \infty} \frac{3x^2-3x+7}{5x^2-6} & \stackrel{1}{=} 
  \pst{\frac{3M^2-3M+7}{5M^2-6}} \stackrel{2}{=}  
  \pst{\frac{3\cancel{M^2}}{5\cancel{M^2}}} = \frac{3}{5}
\end{align*}
Dove le uguaglianze hanno i seguenti motivi:
\begin{enumerate} [nosep]
 \item sostituiamo \(x\) con \(M\), un generico iperreale infinito;
 \item sostituiamo l'espressione ottenuta con una espressione 
   indistinguibile, eseguiamo i calcoli e calcoliamo la parte standard.
\end{enumerate}
\end{esempio}

Ora vediamo alcuni casi un po' più delicati: ricordiamoci che un numero 
iperreale diverso da zero non può mai essere indistinguibile da zero.

\begin{esempio}
\textbf{Infinitesimo fratto finito non infinitesimo (\emph{i/fni}).}
\begin{align*}
\lim_{x \rightarrow -5} \frac{x+5}{x-3} & \stackrel{1}{=} 
  \pst{\frac{-5+\epsilon+5}{-5+\epsilon-3}} \stackrel{2}{=}  
  \pst{\frac{\epsilon}{\epsilon-8}} \stackrel{3}{=} 
  \pst{\frac{\epsilon}{-8}} \stackrel{4}{=} \pst{\delta} = 0
\end{align*}
Dove le uguaglianze hanno i seguenti motivi:
\begin{enumerate} [nosep]
 \item sostituiamo \(x\) con \(-5+\epsilon\);
 \item eseguiamo i calcoli;
 \item sostituiamo l'espressione ottenuta con una espressione 
   indistinguibile;
 \item un infinitesimo diviso un finito non infinitesimo dà come risultato 
un infinitesimo e la sua parte standard è zero.
\end{enumerate}
\end{esempio}

\begin{esempio}
\textbf{Finito non infinitesimo fratto infinitesimo (\emph{fni/i}).}
\begin{align*}
\lim_{x \rightarrow 4} \frac{2x+6}{x-4} & \stackrel{1}{=} 
  \pst{\frac{2\tonda{4+\epsilon}+6}{4+\epsilon-4}} \stackrel{2}{=}  
  \pst{\frac{14 + 2 \epsilon}{\epsilon}} \stackrel{3}{=} 
  \pst{\frac{14}{\epsilon}} \stackrel{4}{=} 
  \pst{M} \stackrel{5}{~\longrightarrow~} \infty
\end{align*}
Dove le uguaglianze hanno i seguenti motivi:
\begin{enumerate} [nosep]
 \item sostituiamo \(x\) con \(4+\epsilon\);
 \item eseguiamo i calcoli;
 \item sostituiamo l'espressione ottenuta con una espressione 
   indistinguibile;
 \item un finito fratto un infinitesimo dà come risultato un infinito; 
 \item gli infiniti non hanno parte standard quindi qui non possiamo usare 
l'``\(=\)''. Ma i matematici per indicare un infinito usano il simbolo 
``\(\infty\)'', quindi quando dobbiamo calcolare la parte standard di un 
numero non finito, tracceremo una freccia seguita da ``\(\infty\)''.
\end{enumerate}
\end{esempio}

\begin{esempio}
\textbf{Funzione irrazionale, con radici quadrate}.
\begin{align*}
\lim_{x \rightarrow \infty} \tonda{2x-\sqrt{4x^2-8x+3}} & \stackrel{1}{=} 
  \pst{2M-\sqrt{4M^2-8M+3}} \stackrel{2}{=} \\
  &=\pst{2M-\sqrt{4M^2}} \stackrel{3}{=} 
  \pst{2M-2M} = 0
\end{align*}
Dove le uguaglianze hanno i seguenti motivi:
\begin{enumerate} [nosep]
 \item sostituiamo \(x\) con \(M\);
 \item sostituiamo l'espressione ottenuta con una espressione 
   indistinguibile;
 \item eseguiamo i calcoli \dots
\end{enumerate}
\vspace{1em}
\begin{minipage}{.69\textwidth}
Ma questa volta nei ragionamenti fatti c'è un errore. Proviamo a calcolare 
la funzione per alcuni valori abbastanza grandi di \(x\).\\
I risultati dovrebbero avvicinarsi a zero, ma non è così, sembra si 
avvicinino, invece, a due. Dove abbiamo sbagliato?
\end{minipage}
\begin{minipage}{.39\textwidth}
\begin{center}
\begin{tabular}{r|r}
x & y\\\hline
100 & 2.00252 \\
1000 & 2.00025 \\
10000 & 2.00002 \\
\end{tabular}
\end{center}
\end{minipage}\\

Abbiamo usato in modo improprio la relazione \emph{indistinguibile}: 
abbiamo ottenuto un'espressione indistinguibile da zero, ma ciò non è 
possibile (vedi la definizione di indistinguibile) \dots \\
Dobbiamo seguire un'altra strada.

Consideriamo l'espressione data come una frazione e razionalizziamo il 
numeratore:
\begin{align*}
\lim_{x \rightarrow \infty} 2x-\sqrt{4x^2-8x+3}=
&\pst{\frac{\tonda{2M}-\sqrt{4M^2-8M+3}}{1} \cdot 
\frac{\tonda{2M}+\sqrt{4M^2-8M+3}}{\tonda{2M}+\sqrt{4M^2-8M+3}}}=\\
&=\pst{\frac{\cancel{4M^2}-\cancel{4M^2}+8M-3}{2M+\sqrt{4M^2-8M+3}}}=
\end{align*}
Questa volta possiamo sostituire l'espressione sotto radice con 
un'espressione indistinguibile, senza ottenere zero:
\[=\pst{\frac{8M-3}{2M+\sqrt{4M^2-8M+3}}} =
   \pst{\frac{8M}{2M+\sqrt{4M^2}}}=\]
e svolgendo i calcoli otteniamo:
\[=\pst{\frac{8M}{2M+2M}}=
   \pst{\frac{8\cancel{M}}{4\cancel{M}}} = 2\]
Ottenendo un risultato in accordo con l'andamento della funzione.
\end{esempio}

\begin{esempio}
\textbf{Funzione razionale}: metodo rapido ma non sempre funzionante.
\begin{align*}
\lim_{x \rightarrow 2} \frac{x^3+3x^2+2x}{x^2-x-6} & \stackrel{1}{=} 
\tonda{\pst{\frac
  {\tonda{2+\epsilon}^3+3\tonda{2+\epsilon}^2+2\tonda{2+\epsilon}}
  {\tonda{2+\epsilon}^2-\tonda{2+\epsilon}-6}} \stackrel{2}{=}}\\ 
  &=\frac{2^3+3\cdot 2^2+2\cdot 2}{2^2-2-6} =
  \pst{\frac{8+12+4}{4-2-6}} \stackrel{3}{=} \pst{\frac{24}{-4}} =-6
\end{align*}
Dove le uguaglianze hanno i seguenti motivi:
\begin{enumerate} [nosep]
 \item sostituiamo \(x\) con un numero infinitamente vicino a \(+2\);
 \item sostituiamo le espressioni tra parentesi con espressioni 
indistinguibili;
 \item poiché i risultati ottenuti sono diversi da zero, il risultato è 
effettivamente indistinguibile. Quindi il risultato ottenuto è valido.
\end{enumerate}
\end{esempio}

Ora vediamo un caso in cui il metodo precedente non funziona. Prima di 
affrontare l'esempio ricordiamoci che se non abbiamo maggiori informazioni 
sugli infinitesimi \(\alpha\) e \(\beta\), non possiamo calcolare 
\(\frac{\alpha}{\beta}\).

\begin{esempio}
\textbf{Funzione razionale}: metodo rapido ma inconcludente.
\begin{align*}
\lim_{x \rightarrow -2} \frac{x^3+3x^2+2x}{x^2-x-6} & = 
\pst{\frac
  {\tonda{-2}^3+3\tonda{-2}^2+2\tonda{-2}}
  {\tonda{-2}^2-\tonda{-2}-6}} =\\ 
  &=\pst{\frac{-8+12-4}
              {4+2-6}} = \pst{\frac{0}{0}} = \dots
\end{align*}
Ma qui ci scontriamo con 2 problemi: abbiamo usato la relazione di 
indistinguibile con lo zero, e abbiamo ottenuto una divisione per zero che 
non è definita. Potremmo essere un po' più pignoli ma ancora troppo 
grossolani osservando che quando \(x\) si avvicina a \(2\) il numeratore e 
il denominatore si avvicinano a zero, sono cioè degli infinitesimi e quindi 
otteniamo: \(\frac{\alpha}{\beta}\) ma anche questa maggior precisione non 
è sufficiente.

Metodo lungo ma sicuro:
dobbiamo rimboccarci le maniche e affrontare il calcolo algebrico 
ricordandoci che: \\
\(\tonda{x+a}^3= x^3+3x^2a+3xa^2+a^3\)

\begin{align*}
\lim_{x \rightarrow -2} \frac{x^3+3x^2+2x}{x^2-x-6} & \stackrel{1}{=} 
\pst{\frac
  {\tonda{-2+\epsilon}^3+3\tonda{-2+\epsilon}^2+2\tonda{-2+\epsilon}}
  {\tonda{-2+\epsilon}^2-\tonda{-2+\epsilon}-6}} \stackrel{2}{=}\\ 
  &\stackrel{2}{=}\pst{\frac{-8+12\epsilon-6\epsilon^2+\epsilon^3+
             3\tonda{4-4\epsilon+\epsilon^2}-4+2\epsilon}
             {4-4\epsilon+\epsilon^2-\tonda{-2+\epsilon}-6}} =\\ 
  &=\pst{\frac{\cancel{-8}+\cancel{12\epsilon}-6\epsilon^2+\epsilon^3+
          \cancel{12}\cancel{-12\epsilon}+3\epsilon^2\cancel{-4}+2\epsilon}
             {\cancel{4}-4\epsilon+
              \epsilon^2\cancel{+2}-\epsilon\cancel{-6}}}=\\ 
  &=\pst{\frac{\epsilon^3+2\epsilon}{\epsilon^2-5\epsilon}} = 
    \pst{\frac{\cancel{\epsilon} \tonda{\epsilon^2+2}}
             {\cancel{\epsilon} \tonda{\epsilon-5}}}  \stackrel{3}{=} 
    \pst{\frac{2}{-5}} = -\frac{2}{5}
\end{align*}
Dove le uguaglianze hanno i seguenti motivi:
\begin{enumerate} [nosep]
 \item sostituiamo \(x\) con \(-2+\epsilon\);
 \item eseguiamo tutti i calcoli e semplifichiamo;
 \item sostituiamo l'espressione con una indistinguibile.
\end{enumerate}
\end{esempio}

\begin{osservazione}
Pensate alla complicazione dei calcoli se ci fosse un qualche \(x^4\) o
\(x^5\)\dots
Vedremo ora un altro modo di calcolare il limite che risulta meno 
complicato.
\end{osservazione}\\

Prima di affrontare il prossimo metodo ricordiamoci che possiamo 
raccogliere tutti gli infinitesimi di ordine superiore ad un certo 
infinitesimo \(\epsilon\) nel simbolo: \(o\tonda{\epsilon}\).

\begin{esempio}
\textbf{Funzione razionale}: altro metodo.
\begin{align*}
\lim_{x \rightarrow -2} \frac{x^3+3x^2+2x}{x^2-x-6} & \stackrel{1}{=} 
\pst{\frac
  {\tonda{-2+\epsilon}^3+3\tonda{-2+\epsilon}^2+2\tonda{-2+\epsilon}}
  {\tonda{-2+\epsilon}^2-\tonda{-2+\epsilon}-6}} \stackrel{2}{=}\\ 
  &\stackrel{2}{=}\pst{\frac{-8+12\epsilon+12-12\epsilon-4+2\epsilon+
                             o\tonda{\epsilon}}
                           {4-4\epsilon+2-\epsilon-6+o\tonda{\epsilon}
                           }}=\\ 
  &=\pst{\frac{\cancel{-8}+\cancel{12\epsilon}+
               \cancel{12}\cancel{-12\epsilon}\cancel{-4}+2\epsilon+
               o\tonda{\epsilon}}
              {\cancel{4}-4\epsilon\cancel{+2}-\epsilon\cancel{-6}+
               o\tonda{\epsilon}}}\stackrel{3}{=}
    \pst{\frac{2\cancel{\epsilon}}{-5\cancel{\epsilon}}} = -\frac{2}{5}
\end{align*}
Dove le uguaglianze hanno i seguenti motivi:
\begin{enumerate} [nosep]
 \item sostituiamo \(x\) con \(-2+\epsilon\);
 \item eseguiamo i calcoli riunendo in un unico simbolo tutti gli 
infinitesimi di ordine superiore e perciò trascurabili;
 \item sostituiamo l'espressione con una indistinguibile poi calcoliamo la 
parte standard.
\end{enumerate}
\end{esempio}

% \begin{esempio}
% \textbf{Funzione razionale}: altro metodo.
% \begin{align*}
% \lim_{x \rightarrow 2} \frac{x^3+3x^2+2x}{x^2-x-6} & \stackrel{1}{=} 
% \tonda{\pst{\frac
%   {\tonda{-2+\epsilon}^3+3\tonda{-2+\epsilon}^2+2\tonda{-2+\epsilon}}
%   {\tonda{-2+\epsilon}^2-\tonda{-2+\epsilon}-6}} \stackrel{2}{=}}\\ 
%   &=\pst{\frac{-8+12-4}{4+2-6}} = \pst{\frac{0}{0}} 
%   \text{ Indistinguibile non è applicabile.}
% \end{align*}
% 
% È evidente che \(-2\) è uno zero sia del numeratore sia del denominatore 
% quindi, per il teorema di Ruffini, entrambi questi polinomi sono divisibili 
% per \(x+2\).
% Possiamo scomporre i due polinomi, semplificarli procedendo nel seguente 
% modo:
% 
% \begin{align*}
% \lim_{x \rightarrow -2} \frac{x^3+3x^2+2x}{x^2-x-6} &=
% \lim_{x \rightarrow -2} \frac{x \tonda{x+1} \cancel{\tonda{x+2}}}
%           {\tonda{x-3} \cancel{\tonda{x+2}}} \stackrel{1}{=}
% \lim_{x \rightarrow -2} \frac{x^2+x}{x-3}=\\
% &=\tonda{\pst{\frac{\tonda{-2+\epsilon}^2-\tonda{2+\epsilon}}
%                    {\tonda{-2+\epsilon}-3}}}\stackrel{2}{=}
% \pst{\frac{\tonda{-2}^2-2}{-2-3}}=
% \pst{\frac{2}{-5}} = -\frac{2}{5}
% \end{align*}
% Dove le uguaglianze hanno i seguenti motivi:
% \begin{enumerate} [nosep]
%  \item possiamo semplificare perché \(x\) è infinitamente vicino a \(-2\) ma 
% è 
% diverso da \(-2\);
%  \item sostituiamo l'espressione con una indistinguibile operazione, questa 
% volta, applicabile perché le due espressioni sono diverse da zero.
% \end{enumerate}
% \end{esempio}

\begin{osservazione}
Abbiamo ottenuto lo stesso valore ricavato più sopra facendo tutti i 
calcoli algebrici, ma i passaggi sono più semplici.
\end{osservazione}


\subsection{Limiti notevoli}
\label{subsec:cont_limiti_notevoli}

Ci sono alcuni limiti che hanno delle dimostrazioni particolari e che è 
utile conoscere per poter risolvere dei casi particolari.

\subsubsection{Funzioni goniometriche}

Se consideriamo le funzioni seno, coseno e tangente, possiamo vedere che 
alcuni limiti risultano banali, ma altri sono interessanti e possono essere 
dimostrati.

\begin{minipage}{.66\textwidth}
\begin{center} \sinusoide \end{center}
% \begin{center} \sincos{sin(x)}{$y=\sen x$}{Blue} \end{center}
\begin{center} \cosinusoide \end{center}
\end{minipage}
\hfill
\begin{minipage}{.33\textwidth}
\begin{center} \tangentoide \end{center}
\end{minipage}

\paragraph{Funzione seno}~

Osservando il grafico della funzione seno, risulta abbastanza evidente che:
\[\lim_{x \rightarrow 0}{\sen x} = \pst{\sen \epsilon} = 0
\quad \text{e che} \quad 
\lim_{x \rightarrow \infty}{\sen x} = \pst{\sen M} = \text{ non esiste}\]
Infatti man mano che \(x\) aumenta, \(\sen \)
continua a variare tra \(-1\) e \(+1\) e quindi 
\(\sen M\) non può essere infinitamente vicino ad un solo numero reale.

Ma il seguente è un limite notevole di cui proponiamo una dimostrazione 
grafica:
\[\lim_{x \rightarrow 0}\frac{\sen x}{x} = 
  \pst{\frac{\sen \delta}{\delta}} = 1\]
\begin{minipage}{.49\textwidth}
Una circonferenza non ha differenze reali da un poligono regolare di 
infiniti lati quindi un arco infinitesimo differisce dalla corda 
sottesa per infinitesimi di ordine superiore alle loro lunghezze. Perciò il 
loro rapporto è \(1\).
Se rappresentiamo in una circonferenza goniometrica un arco di ampiezza 
infinitesima e il corrispondente seno, appaiono non distinguibili ad ogni 
ingrandimento finito. Se li ingrandiamo con un microscopio non standard con 
infiniti ingrandimenti, possiamo vedere che sono ancora indistinguibili 
cioè differiscono per infinitesimi di ordine superiore. 

Quindi il loro rapporto è \(1\).
\end{minipage}
\hfill
\begin{minipage}{.49\textwidth}
\begin{center} \limiteseno \end{center}
\end{minipage}

\paragraph{Funzione tangente}~

Anche nella funzione tangente valgono i seguenti limiti:
\[\lim_{x \rightarrow 0}{\tan x} = \pst{} = 0; \quad
\lim_{x \rightarrow \infty}{\tan x} = \pst{\tan M} = 
                                      \text{ non esiste}; \quad
\lim_{x \rightarrow 0}\frac{\tan x}{x} = \pst{\frac{\tan \delta}{\delta}} 
                                       = 1\]
per considerazioni analoghe a quelle fatte per la funzione seno.

\paragraph{Funzione coseno}~

Per la funzione coseno valgono i seguenti limiti:
\[\lim_{x \rightarrow 0}{\cos x} = \pst{\cos \delta} = 1; \qquad
\lim_{x \rightarrow 0}\tonda{1-\cos x} = \pst{1-\cos \delta} = 0; \qquad
\lim_{x \rightarrow \infty}{\cos x} = \pst{\cos M} = \text{ non esiste}\]

Sono interessanti i seguenti due limiti che riguardano il coseno:

\begin{align*}
 \lim_{x \rightarrow 0} \frac{1-\cos x}{x} &=
 \pst{\frac{1-\cos \delta}{\delta}}
~ \stackrel{1}{=} ~  
 \pst{\frac{1-\cos \delta}{\delta} \cdot 
      \frac{1+\cos \delta}{1+\cos \delta}}
~ \stackrel{2}{=} ~ 
 \pst{\frac{1-\cos^2 \delta}{\delta \tonda{1+\cos \delta}}}~ 
\stackrel{3}{=} \\
& \stackrel{3}{=} ~
 \pst{\frac{\sen^2 \delta}{\delta \tonda{1+\cos \delta}}}
~ \stackrel{4}{=}
 \pst{\frac{\sen \delta}{\delta} \cdot 
      \frac{\sen \delta}{1+\cos \delta}}
~ \stackrel{5}{=}
 \pst{1 \cdot \frac{\delta}{2}} = 0
\end{align*}
Dove le uguaglianze hanno i seguenti motivi:
\begin{enumerate} [nosep]
 \item moltiplico la funzione per una frazione equivalente a 1;
 \item prodotto notevole;
 \item ricordando che \(\sen^x + \cos^2 x = 1\);
 \item un infinitesimo fratto un non infinitesimo è un infinitesimo 
e la sua parte standard è 0.
\end{enumerate}

\begin{osservazione}
Dato che \(\sen \delta \sim \delta\), i seguenti due limiti sono 
equivalenti:
\[\lim_{x \rightarrow 0} \frac{1-\cos x}{x} =
 \pst{\frac{1-\cos \delta}{\delta}} =
 \pst{\frac{1-\cos \delta}{\sen \delta}} =
 \lim_{x \rightarrow 0} \frac{1-\cos x}{\sen x}\]
\end{osservazione}

L'altro limite notevole, che si dimostra in modo analogo a quello 
precedente, è:
\begin{align*}
 \lim_{x \rightarrow 0} \frac{1-\cos x}{x^2} &=
 \pst{\frac{1-\cos \delta}{\delta^2}}
~ \stackrel{1}{=} ~  
 \pst{\frac{1-\cos \delta}{\delta^2} \cdot 
      \frac{1+\cos \delta^2}{1+\cos \delta}}
~ \stackrel{2}{=} ~ 
 \pst{\frac{1-\cos^2 \delta}{\delta^2 \tonda{1+\cos \delta}}}~ 
\stackrel{3}{=} \\
& \stackrel{3}{=} ~
 \pst{\frac{\sen^2 \delta}{\delta^2 \tonda{1+\cos \delta}}}
~ \stackrel{4}{=}
 \pst{\frac{\sen \delta}{\delta} \cdot 
      \frac{\sen \delta}{\delta \tonda{1+\cos \delta}}}
~ \stackrel{5}{=}
 \pst{1 \cdot \frac{\cancel{\delta}}{2 \cancel{\delta}}} = \frac{1}{2}
\end{align*}

\subsubsection{Esponenziali e logaritmi}

\paragraph{Numero di Eulero (o di Nepero)}

Ricordiamo come è definita la costante di Eulero:
\[e = \pst{\tonda{1+\frac{1}{M}}^M} = 
\pst{\tonda{1+\delta}^\frac{1}{\delta}} 
\]
\[e=\sum_{n=0}^{\infty}{\frac{1}{n!}}=
\frac{1}{1}+\frac{1}{1}+\frac{1}{2}+\frac{1}{2\cdot3}+
\frac{1}{2\cdot3\cdot4}+\frac{1}{2\cdot3\cdot4\cdot5}+\dots\]
Mentre la seconda definizione risulta molto efficiente per il calcolo, la 
prima ha delle interessanti applicazioni matematiche.

\begin{esempio}
\label{esempio:log}
Limite di una particolare funzione logaritmica:
\begin{align*}
 \lim_{x \rightarrow 0} \dfrac{\ln\tonda{1+x}}{x} &=
 \pst{\dfrac{\ln\tonda{1+\delta}}{\delta}}~ = ~  
 \pst{\frac{1}{\delta}\ln\tonda{1+\delta}} ~\stackrel{1}{=} ~
 \pst{\ln\tonda{1+\delta}^\frac{1}{\delta}}
~ \stackrel{2}{=} ~
\pst{\ln\tonda{e}} \stackrel{3}{=} ~ 1
\end{align*}
Dove le uguaglianze hanno i seguenti motivi:
\begin{enumerate} [nosep]
 \item per una proprietà dei logaritmi;
 \item l'argomento del logaritmo è proprio la definizione di \(e\);
 \item l'esponente da dare a \(e\) per ottenere \(e\) è \(1\).
\end{enumerate}
\end{esempio}

\begin{esempio}
Come nell'esempio precedente ma con una base generica:
\begin{align*}
 \lim_{x \rightarrow 0} \dfrac{\log_a\tonda{1+x}}{x} &=
 \pst{\dfrac{\log_a\tonda{1+\delta}}{\delta}}~\stackrel{1}{=} ~  
 \pst{\dfrac{\frac{\ln\tonda{1+\delta}}{\ln a}}{\delta}}~\stackrel{1}{=} ~
 \pst{\dfrac{\ln\tonda{1+\delta}}{\delta}\cdot \dfrac{1}{\ln a}}
 ~ \stackrel{2}{=} ~
 \dfrac{1}{\ln a}
\end{align*}
Dove le uguaglianze hanno i seguenti motivi:
\begin{enumerate} [nosep]
 \item cambio di base del logaritmo;
 \item per quanto visto nell'esempio \ref{esempio:log}.
\end{enumerate}
\end{esempio}

\begin{esempio}
Limite di una particolare funzione esponenziale:
\begin{align*}
\lim_{x \rightarrow 0} \dfrac{e^x-1}{x} &=
\pst{\dfrac{e^\epsilon-1}{\epsilon}}
~ \stackrel{1}{=} ~  
\pst{\dfrac{\delta}{\ln{(\delta+1)}}}~ \stackrel{2}{=} ~ 1
\end{align*}
Dove le uguaglianze hanno i seguenti motivi:
\begin{enumerate} [nosep]
 \item ancora una sostituzione: poniamo \(e^\epsilon-1=\delta\). 
Allora \(e^\epsilon = 1+\delta\) quindi \(\epsilon\) è l'esponente da dare 
a \(e\) per ottenere \(1+\delta\) cioè: 
\(\epsilon = \ln{\tonda{1+\delta}}\);
 \item per quanto visto nell'esempio \ref{esempio:log}.
\end{enumerate}
\end{esempio}

\begin{esempio}
Simile all'esempio precedente, ma con una base generica.
\begin{align*}
\lim_{x \rightarrow 0} \dfrac{a^x-1}{x} &=
\pst{\dfrac{a^\epsilon-1}{\epsilon}}
~ \stackrel{1}{=} ~  
\pst{\dfrac{\delta}{\log_a{(\delta+1)}}}~ \stackrel{2}{=}\\
&\stackrel{2}{=} 
\pst{\frac{\delta}{\dfrac{\ln{(\delta+1)}}{\ln a}}} ~=~
\pst{\frac{\delta \cdot \ln a}{\ln{(\delta+1)}}} %\stackrel{3}{=}\\
\stackrel{3}{=} ~ 
\pst{1 \cdot \ln a}=\ln{a}
\end{align*}
Dove le uguaglianze hanno i seguenti motivi:
\begin{enumerate} [nosep]
 \item ancora una sostituzione: poniamo
\(a^\epsilon-1=\delta\), allora \(\epsilon=\log_a(\delta+1)\);
 \item applichiamo la formula del cambiamento di base di un logaritmo; 
 \item per quanto visto nell'esempio \ref{esempio:log}.
\end{enumerate}
\end{esempio}

\begin{esempio}
Limite di una particolare funzione esponenziale:
\begin{align*}
 \lim_{x \rightarrow \infty} \tonda{1+\dfrac{k}{x}}^x =
 \pst{\tonda{1+\dfrac{k}{N}}^N}
~ \stackrel{1}{=} ~  
\pst{\tonda{1+\dfrac{1}{M}}^{kM}}
~ \stackrel{2}{=} ~
\pst{\quadra{\tonda{1+\dfrac{1}{M}}^M}^k}
~ \stackrel{3}{=} ~ e^k
\end{align*}
Dove le uguaglianze hanno i seguenti motivi:
\begin{enumerate} [nosep]
 \item se al posto di \(\dfrac{k}{N}\) scrivo \(\dfrac{1}{M}\) 
allora al posto di \(N\) dovrò scrivere \(N=kM\), 
% infatti:
% \(\dfrac{k}{N}=\dfrac{1}{M} \sLRarrow N=kM\);
 \item la potenza di potenza è una potenza che ha per base la stessa base 
e per \dots
 \item l'espressione tra parentesi quadre è proprio la definizione di \(e\).
\end{enumerate}
\end{esempio}


\section{Continuità}
\label{sec:cont_continuita}

\subsection{Definizione di continuità in un punto}
\label{subsec:cont_definizione}


\begin{definizione}
Diremo che una funzione è \textbf{continua} in un punto \(c\), 
se è definita in \(c\) e, 
quando \(x\) è infinitamente vicino a \(c\), 
allora \(f(x)\) è infinitamente vicino a \(f(c)\). E si scrive::

\[f \text{ è continua in } c \Leftrightarrow 
\forall x \tonda{\tonda{x \approx c} \Rightarrow 
\tonda{f(x) \approx f(c)}}\]

\end{definizione}

\begin{esempio}
 Data la funzione \(f(x)=x^2-3x\) dimostrare che \(f(x)\) è continua in~4.
 
 La funzione è continua in~4 se per ogni \(x\) infinitamente vicino a~4 
 \(f(x)\) è infinitamente vicino a \(f(4)\). Cioè se \(x -4=\epsilon\) allora
 \(f(x) -f(4) = \delta\) dove \(\epsilon\) e \(\delta\) sono due infinitesimi.
 
\emph{dimostrazione}

Da \(x-4=\epsilon\) si ricava che \(x=4+\epsilon\), quindi: 
\[f(x) -f(4) = f(4+\epsilon) -f(4) = 
\tonda{4+\epsilon}^2-3\tonda{4+\epsilon}-\tonda{4^2-3\cdot 4}=\]
\[=\cancel{16}+8\epsilon+\epsilon^2\cancel{-12} 
  -3\epsilon\cancel{-16}\cancel{+12} = 
  +8\epsilon+\epsilon^2 -3\epsilon = 
\epsilon \tonda{5 + \epsilon}\]

Ora, il prodotto tra un infinitesimo e un finito è un infinitesimo, quindi, se 
la distanza tra \(x\) e \(4\) è infinitesima, anche la distanza tra 
\(f(x)\) e \(f(4)\) è infinitesima. \hfill \textbf{qed} 
 
\end{esempio}

\begin{esempio}
 Dimostrare che \(f(x)=\frac{\abs{x}}{x}\) non è continua in~0.
 
\emph{dimostrazione}\\
Perché una funzione sia continua per un certo valore di \(x\) lì deve essere 
definita. 
\end{esempio}

\begin{esempio}
 Dimostrare che \(f(x)=\frac{\abs{x}}{4}\) è continua in~0.
 
\emph{dimostrazione}:
\[f(\epsilon) - f(0) = \frac{\abs{\epsilon}}{4} - \frac{\abs{0}}{4} = 
 \frac{\abs{\epsilon}}{4} \approx 0\]
\end{esempio}

\begin{esempio}
Studia la continuità della funzione 
\(y=\begin{cases} 
    \dfrac{1}{2}x-2 & \text{se }x \leqslant 2 \\ 
    x^2-6x+7 & \\text{se }x > 2
  \end{cases}\)
\quad in \(x_0=2\)

La funzione è definita per \(x=2\) e vale:\\
\(f(2) = \dfrac{1}{2} \cdot 2-2 = -1\)\\
Dobbiamo  verificare che, se \(x\) è infinitamente vicino a~\(2\) allora 
\(f(x)\) sia infinitamente vicino a~\(-1\).

\begin{minipage}{.49\textwidth}
Dobbiamo distinguere i casi in cui ci avviciniamo a~2 da sinistra o da 
destra. In entrambi i casi consideriamo \(epsilon\) positivo e esplicitiamo 
il segno:
\begin{description}
 \item [da sinistra:]
 ~\\
\(f(2-\epsilon) - f(2) =
  \dfrac{1}{2} \cdot \tonda{2-\epsilon}-2-\tonda{-1} =\)\\
\(= 1 - \dfrac{\epsilon}{2} -2 +1 = -\dfrac{\epsilon}{2}\)\\
Che è un infinitesimo.
 \item [da destra:]
 ~\\
\(f(2+\epsilon) - f(2) =\)\\
\(\tonda{2+\epsilon}^2-6 \cdot \tonda{2+\epsilon}+7-\tonda{-1} =\)\\
\(4+4\epsilon+\epsilon^2-12-6 \epsilon+7+1 =\)\\
\(\epsilon^2-2 \epsilon =\)\\
Che è un infinitesimo.
\end{description}
\end{minipage}
\begin{minipage}{.49\textwidth}
\begin{center}\continuitagraficoa\end{center}
\end{minipage}

\end{esempio}

Data una funzione \(y=f(x)\) definita in \(c\), le seguenti 
affermazioni sono equivalenti:

\begin{enumerate}[noitemsep]
 \item \(f\) è continua in \(c\);
 \item se \(x \approx c\) allora \(f(x) \approx f(c)\);
 \item se \(\st(x) = c\) allora \(\st(f(x)) = f(c)\);
 \item \(\lim_{x \to c} f(x) = f(c)\);
 \item se \(x\) si allontana da \(c\) di un infinitesimo allora 
   \(f(x)\) si allontana da \(f(c)\) di un infinitesimo.
 \item se \(\Delta x\) è infinitesimo allora il corrispondente \(\Delta y\) 
   è infinitesimo.
\end{enumerate}

\begin{comment}
\begin{teorema}[Derivabilità e continuità]
Se una funzione è derivabile in un punto allora è continua in quel punto.
\end{teorema}

\noindent Ipotesi: 
\(f(x) \text{ è derivabile in } c\)
\tab Tesi: 
\(f(x) \text{ è continua in } c\).

\begin{proof}
TODO
\end{proof}
\end{comment}

\subsection{Definizione di continuità in un intervallo}
\label{subsec:cont_definizione}

Dimostrare che una funzione è continua in un punto è piuttosto laborioso, 
pur non essendo complicato, ma quando sono interessato a studiare la 
continuità di una funzione in un intervallo sorge un ulteriore problema. 
Infatti in un intervallo, anche piccolo, i punti sono infiniti e dimostrare 
la continuità per ognuno di essi risulta piuttosto lungo\dots.

Per superare questo scoglio, i matematici hanno pensato un approccio diverso:

\begin{itemize}
 \item dimostrare che alcune funzioni elementari sono continue;
 \item dimostrare che la combinazione di funzioni continue è ancora una 
funzione continua.
\end{itemize}

In questo modo si può riconoscere la continuità di un gran numero di funzioni 
senza fare noiosi calcoli. Di seguito vediamo qualcuno di questi teoremi.

\subsubsection{Funzioni elementari}
\label{subsubsec:cont_funzionielementari}

Dimostriamo la continuità di alcune funzioni elementari.

\begin{teorema}[Continuità delle costanti]
Le funzioni costanti sono continue.
\end{teorema}

\noindent Ipotesi: \(f(x)=k\).\tab Tesi: \(f(x)\) è continua.

\begin{proof}
Per la definizione di continuità vogliamo dimostrare che 
\[\forall x \text{ se } x_0 \approx x \text{ allora } f(x_0) \approx f(x)\]
Poniamo \(x_0=x+\epsilon\), essendo la funzione costante, anche 
\[f(x+\epsilon)=k\] 
che, ovviamente, è infinitamente vicino a \(k\). In simboli:
\[f(x+\epsilon) = k \approx k = f(x)\] 
\end{proof}

\begin{teorema}[Continuità della funzione identica]
La funzione identica (\(y=x\)) è continua.
\end{teorema}

\noindent Ipotesi: \(f(x)=x\).\tab Tesi: \(f(x)\) è continua.

\begin{proof}
Per la definizione di continuità vogliamo dimostrare che 
\[\forall x \text{ se } x_0 \approx x \text{ allora } f(x_0) \approx f(x)\]
Poniamo \(x_0=x+\epsilon\), \(f(x_0) = f(x+\epsilon)=x+\epsilon\). 
Dato che la differenza:
\[f(x+\epsilon)-f(x) = x+\epsilon-x= \epsilon\]
è un infinitesimo, allora i due valori sono infinitamente vicini. In simboli:
\[f(x+\epsilon) = x+\epsilon \approx x = f(x)\] 
\end{proof}

\begin{teorema}[Continuità della funzione seno]
La funzione seno (\(y=\sen x\)) è continua.
\end{teorema}

\noindent Ipotesi: \(f(x)=\sen x\).\tab Tesi: \(f(x)\) è continua.

\begin{proof}
Usando la formula del seno della somma di due angoli:
\[f(x+\epsilon) =
\sen{(x+\epsilon)} = \sen x \cos \epsilon + \cos x \sen \epsilon\]
Se un angolo è infinitamente vicino a zero avrà il coseno infinitamente vicino 
a uno e il seno infinitamente vicino a zero. Quindi:
\[\sen{(x+\epsilon)} = \sen x + \delta\]
Perciò:
\[f(x+\epsilon) =
\sen{(x+\epsilon)} = \sen{(x+\epsilon)} = 
\sen x + \delta \approx \sen x = f(x)\]
\end{proof}
% 
% \begin{teorema}[Continuità della funzione esponenziale]
% La funzione esponenziale (\(y=a^x\)) è continua.
% \end{teorema}
% 
% \noindent Ipotesi: \(f(x)=a^x\).\tab Tesi: \(f(x)\) è continua.
% 
% \begin{proof}
% Usando la prima proprietà delle potenze:
% \[f(x+\epsilon) =
% a^{x+\epsilon} = a^{x} \cdot a^{\epsilon} = a^{x} \cdot \tonda{1 + \delta} 
% \approx a^{x} = f(x)\]
% \end{proof}

%---------- tentativi inutili
% \nopagebreak
% \samepage
% \filbreak
%----------

%----------
% 13
%  \setlength{\columnsep}{1.5pc}
% We want a rule between columns.
% 14
%  \setlength\columnseprule{.4pt}
% We also want to ensure that a new multicols envi-
% ronment finds enough space at the bottom of the
% page.
% 15
%  \setlength\premulticols{6\baselineskip}

%----------

\noindent\begin{minipage}{\textwidth}
Oltre alle funzioni precedenti, anche altre funzioni elementari sono 
continue, il seguente elenco riporta le principali funzioni continue:

\noindent\begin{minipage}{1.05\textwidth}
\begin{multicols}{5}
\begin{itemize} [noitemsep]
 \item \(y=k\)
 \item \(y=x\)
 \item \(y=\frac{1}{x}\)  \textasteriskcentered
 \item \(y=\sqrt[n]{x}\)  \textasteriskcentered
 \item \(y=\abs{x}\)
 \item \(y=a^x\)
 \item \(y=\log_a x\)  \textasteriskcentered
 \item \(y=\sen x\)
 \item \(y=\cos x\)
 \item \(y=\tg x\)  \textasteriskcentered
\end{itemize}
\end{multicols}
\end{minipage}

\begin{osservazione}
Le funzioni segnate da ``\textasteriskcentered'' sono continue non su 
tutto \(\R\), 
ma solo \textbf{all'interno del loro Insieme di Definizione}.
\end{osservazione}
\end{minipage}

\subsubsection{Composizione di funzioni}
\label{subsubsec:cont_composizionefunzioni}

Vediamo ora che anche componendo in alcuni modi funzioni continue otteniamo 
ancora funzioni continue.

\begin{teorema}[Somma di funzioni continue]
Se \(f\) e \(g\) sono funzioni continue, anche \(f+g\) è continua.
\end{teorema}

\noindent Ipotesi: 
\(f(x) \text{ e} g(x)\) sono continue
\tab Tesi: 
\(f(x)+g(x)\) è continua.

\begin{proof}
Dato che sono continue: 
\[f(x+\epsilon) + g(x+\epsilon) = f(x)+\alpha + g(x)+\beta\]
Ma la somma di due infinitesimi è ancora un infinitesimo quindi:
\[f(x)+g(x)+\tonda{\alpha + \beta} \approx f(x)+g(x)\]
\end{proof}

\begin{teorema}[Prodotto di funzioni continue]
Se \(f\) e \(g\) sono funzioni continue, anche \(f \cdot g\) è continua.
\end{teorema}

\noindent Ipotesi: 
\(f(x) \text{ e} g(x)\) sono continue
\tab Tesi: 
\(f(x) \cdot g(x)\) è continua.

\begin{proof}
Dato che sono continue: 
\[f(x+\epsilon) \cdot g(x+\epsilon) = 
\tonda{f(x)+\alpha} \cdot \tonda{g(x)+\beta} = 
f(x) \cdot g(x) + f(x) \cdot \beta + g(x) \cdot \alpha + \alpha \cdot \beta
\approx f(x) \cdot g(x)\]

Dato che sia il prodotto tra un numero finito e un infinitesimo, sia il 
prodotto tra due infinitesimi sono infinitesimi e lo è anche la loro somma. 
\end{proof}

\begin{corollario}
 Ogni funzione polinomiale è continua.
\end{corollario}

\begin{proof}
Dato che una funzione polinomiale si può ottenere partendo da funzioni 
costanti e da funzioni identiche attraverso moltiplicazioni e addizioni, 
la tesi consegue dai teoremi precedenti. 
\end{proof}

\begin{esempio}
 Dimostrare che \(f(x)=2x^2 + 3\) è una funzione continua.

\begin{proof}

\(f(x)=2x^2 + 3\) è continua perché è somma di due funzioni continue: 
% \begin{itemize}[noitemsep]
%  \item \(f(x)=2x^2 + 3\) è continua perché è somma di due funzioni continue: 
 \begin{itemize}[nosep]
  \item \(y=2x^2\) è continua perché è prodotto di due funzioni continue:
  \begin{itemize}[nosep]
   \item \(y=2\) è continua perché è una costante;
   \item \(y=x^2\) è continua perché è prodotto di due funzioni continue:
   \begin{itemize}[nosep]
    \item \(y=x\) è continua perché è una funzione identica;
    \item \(y=x\) è continua perché è una funzione identica;
   \end{itemize}
  \end{itemize}
  \item \(y=3\) è continua perché è una costante;
 \end{itemize}
% \end{itemize}
\end{proof}

\end{esempio}

\begin{teorema}[Funzioni di funzioni]
Se \(f(x)\) e \(g(x)\) sono funzioni continue, anche \(f(g(x))\) è continua.
\end{teorema}

\noindent Ipotesi: 
\(f(x) \text{ e} g(x)\) sono continue
\tab Tesi: 
\(f(x) \star g(x) = f(g(x))\) è continua.

\begin{proof}
Dato che \(g\) è continua: 
\[f(g(x+\epsilon)) = f(g(x)+\alpha)\]
e dato che \(f\) è continua: 
\[f(g(x)+\alpha)=f(g(x))+\beta\]
quindi: 
\[f(g(x+\epsilon)) = f(g(x)+\alpha) = f(g(x))+\beta \approx f(g(x))\]
\end{proof}











