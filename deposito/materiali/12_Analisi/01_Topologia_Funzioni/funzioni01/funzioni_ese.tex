% (c) 2017 Leonardo Aldegheri
% (c) 2017 Carlotta Gualtieri

\begin{comment}
\section{TODO}

\section{Esercizi}

\subsection{Esercizi dei singoli paragrafi}

\subsubsection*{\numnameref{sec:01_}}

\begin{esercizio}
\label{ese:D.19}
testo esercizio
\end{esercizio}

\begin{esercizio}\label{ese:03.1}
Consegna:
 \begin{enumeratea}
  \item  
 \end{enumeratea}
\end{esercizio}

\subsection{Esercizi riepilogativi}

\begin{esercizio}
\label{ese:D.19}
testo esercizio
\end{esercizio}

\begin{esercizio}\label{ese:03.1}
Consegna:
 \begin{enumeratea}
  \item  
 \end{enumeratea}
\end{esercizio}
\end{comment}

\section{Esercizi}
%label{}
  \subsection{Esercizi dei singoli paragrafi}
  %label{}
  \subsubsection*{1.1 Definizione di funzione}
  %label{}
  \begin{itemize}
  \item[1.1)] Riflettendo sulla definizione di 
funzione rispondi argomentando alle seguenti domande:
  \begin{itemize}
  \item[a)] Quali tra i 
seguenti oggetti, che puoi rappresentare sul piano cartesiano, è una 
funzione: circonferenza, ellisse, parabola con asse verticale?
  \item[b)] Quali tra i 
seguenti oggetti, che puoi rappresentare sul piano cartesiano, sono funzioni: 
retta verticale, retta orizzontale, retta obliqua?
  \item[c)] Considera una 
parabola con asse verticale e una con asse orizzontale, quale delle due è una 
funzione?
  \end{itemize}
  \item[1.2)] Determina il dominio delle 
seguenti funzioni
  \begin{itemize}
  \item[a)] \(y= 4x^2+x+2\)   
   \hfill  [ \(D=R\) ]
  \item[b)] 
\(y=\frac{3x+2}{x-5}\)   \hfill   
   [ \(D=R-\{5\}\) ]
  \item[c)] \(y=\sqrt{9-x^2}   \) 
   \hfill   [ \(D=[-3, 3]\) ]
  \item[d)] 
\(y=\frac{4x}{\sqrt{x+2}}  \)  \hfill   
   [ \(D=]-2,+\infty[\) ]
  \item[e)] 
\(y=\frac{2+3x}{x^2+7x+12}\)   \hfill   
[\(D=\mathbb{R}-\{-3,-4\}\)]
  \item[f)] \(y=e^{x+3}\)\hfill   
   [\(D=\mathbb{R}\)]
  \item[g)] \(y=\log_2(x+3)\)  
\hfill  [\(D=]-3, +\infty[\) ]
  \item [h)] \(y=\ln(x^2+6x+8)\)  
  \hfill   
[\(D=]-\infty,-4[\cup]-2,+\infty[\)]
  
\item[i)]\(y=\frac{x^3+3x}{e^x+5}\)\hfill   
  [\(D=\mathbb{R}\)]
  
\item[l)]\(y=\frac{3x^2+4}{e^x-2}\)   \hfill  
   [\(D=\mathbb{R}-\{\ln2\}\)]
  \item[m)] 
\(y=\frac{4x-5}{2x-8}-\sqrt{x+3}\)  \hfill  
[\(D=[-3,4[\cup]4,+\infty[\)]
  \item[n)]\(y=\sqrt{\log(x+4)}\) 
   \hfill  [\(D=[-3,+\infty[\)]
  \item[o)]\(y=(x+5)^{x+3}\)  
   \hfill  [\(D=]-5,+\infty[\)]
  \item[p)] 
\(y=2\sin{x}+\cos{2x}\)   \hfill   
[\(D=\mathbb{R}\)]
  
\item[q)]\(y={2+\cos{3x}}{2\sin{x}}\)   \hfill  
   [\(D=\mathbb{R}-\{k\pi,\,k\in \mathbb{Z}\}\)]
  \item[r)] \(y=\arcsin(x+2)\)  
   \hfill  [\(D=[-3, -1]\)]
  
\item[s)]\(y=\arctan(\frac{3}{x+2}) \)  \hfill  
   [\(D=\mathbb{R}-\{-2\}\) ]\\

  \end{itemize}

\newpage
  
  \item[1.3)] Dall'analisi visiva del grafico deduci il 
dominio e codominio della funzione.
  \begin{figure}[htpb!]
  \centering

% \begin{minipage}{.49 \textwidth}
% \end{minipage}
% \hfill
% \begin{minipage}{.49 \textwidth}
% \end{minipage}

\includegraphics[width=0.45\textwidth]{img/funz_16.png} \quad 
\includegraphics[width=0.5\textwidth]{img/funz_17.png}  
  
\includegraphics[width=0.45\textwidth]{img/funz_17a.png} \quad
\includegraphics[width=0.45\textwidth]{img/funz_17b.png}   
  
\includegraphics[width=0.45\textwidth]{img/funz_17c.png} \quad
\includegraphics[width=0.45\textwidth]{img/funz_17d.png}
  %\caption{}
  %\label{fig:funz_14abc}
  \end{figure}
  \end{itemize}

\newpage

\subsubsection*{1.2 La rappresentazione di una funzione}
%label{}
\begin{itemize}
  \item[1.4)] Immagina l'evoluzione della temperatura ora per ora nella 
città di Firenze, in una giornata di agosto, costruiscine la rappresentazione 
tabulare e quella grafica.
  \item[1.5)] Rappresenta graficamente le seguenti funzioni:
  \begin{itemize}
  \item[a)] \(y=\sin2x\)
  \item[b)] \(y=3x+2\)
  \item[c)] \(y=x^2+2\)
  \item[d)] \(y=x^2+2x+3\)
  \item[e)] \(y=\log{(x+1)}\)
  \item[f)] \(y=e^x+2\)
  \item[g)] \(y=\frac{x+2}{2x-4}\)
  \item[h)] \(y=\sqrt{9-x}\)
  \item[i)] \(y=\sqrt{9-4x^2}\)
  \item[l)] \(y=\sqrt[3]{x}\)
  \item[m)] \(y=\vert x^2+4x+3\vert\)
  \end{itemize}
\end{itemize}

\subsubsection*{1.3 Le proprietà di una funzione}
%label{}
\begin{itemize}
  \item[1.6)] Stabilisci se le seguenti funzioni, rappresentate 
graficamente, sono iniettive, suriettive o biunivoce, motivando la risposta.
  \begin{multicols}{2} 
  \begin{itemize}
  \item[(a)] \(f:\mathbb{R}\to[0,+\infty[\)
  \item[(c)] \(f:\mathbb{R}\to]0,+\infty[\)
  \item[(b)] \(f:\mathbb{R}\to\mathbb{R}\)
  \item[(d)] \(f:\mathbb{R}\to\mathbb{R}\)
  \end{itemize}
  \end{multicols}
  
  \begin{figure}[htpb!]
  \centering
  
\includegraphics[width=0.65\textwidth]{img/funz_18.png}%\caption{}
  %\label{fig:funz_14abc}
  \end{figure}
\end{itemize}
\subsubsection*{1.4 Le caratteristiche di una funzione}
%label{}
\begin{itemize}
  \item[1.7)] Verifica se le seguenti funzioni sono pari o dispari.
  \begin{itemize}
  \item[a)] \(y=2x+3\)   \hfill   
  [né pari né dispari]
  \item[b)] \(y=x^3+x\)   \hfill  
   [dispari]
  \item[c)] \(y=\frac{x^2+3}{x}\)   \hfill  
  [dispari]
  \item[d)] \(y=\frac{3x^4}{2+x^2} \)   \hfill  
   [pari]
  \item[e)] \(y=\tan{x}+\sin{x} \)   \hfill   
 [dispari]
  \item[f)] \(y= \log{(x-1)}\)   \hfill   
[né pari né dispari]
  \end{itemize}
  
  \item[1.8)] Stabilisci se le seguenti funzioni sono periodiche, 
individuandone l'eventuale periodo.
  \begin{itemize}
  \item[a)] \(y=\sin4x\)   \hfill   
  [periodica \(T=\frac{\pi}{2}\)]
  \item[b)] \(y=\tan2x\)   \hfill   
  [periodica \(T=\frac{\pi}{2}\)]
  \item[c)] \(y=\cos3x+1\)   \hfill   
   [periodica \(T=\frac{2\pi}{3}\)]
  \item[d)] \(y=\cos x+x \)   \hfill  
   [non periodica]
  \item[e)] \(y=\tan{\frac{x}{5}} \)   \hfill   
   [periodica \(T=5\pi\)]
  \item[f)] \(y= \sin(x+1)\)   \hfill   
[periodica \(T=2\pi\)]
  \item[g)] \(y= \sin x+1\)   \hfill   
[periodica \(T=2\pi\)]
  \item[h)] \(y= \sin x+x\)   \hfill   
[non periodica]
  \item[i)] \(y= \sin{\frac{x}{4}}\)   \hfill   
  [periodica \(T=8\pi\)]
  \item[l)] \(y= 2^{\sin{x}}\)   \hfill   
[periodica \(T=2\pi\)]
  \item[m)] \(y=x\cos x\)   \hfill   [non 
periodica]
  \end{itemize}
\end{itemize}

\subsubsection*{1.5 La classificazione delle funzioni}
\begin{itemize}
  \item[1.9)] Classifica prima riguardo alle categorie algebrico o 
trascendente e poi rispetto alle categorie fratta o intera e razionale o 
irrazionale le seguenti funzioni.
  \begin{itemize}
  \item[a)] \(y=\frac{x^2+x+2}{3x}\)
  \item[b)] \(y=\frac{4}{\sqrt{x-6}}\)
  \item[c)] \(y=e^x+x\); \(y=\sqrt[3]{x^2+x}\)
  \item[d)] \(y=\frac{1}{3}\sqrt{3-x}\)
  \item[e)] \(y=\frac{\sin x}{4}\)
  \item[f)] \(y=\frac{x^2-4x}{3+\sqrt{2}}\)
  \end{itemize}
\end{itemize}

\subsubsection*{1.6 Funzioni inverse, composte e uguali}
%label{}
\begin{itemize}
  \item[1.10)] Stabilisci se le seguenti funzioni sono invertibili 
senza restrizioni, giustificando la risposta.  \begin{itemize}
  \item[a)] \(y=2x+3\) \hfill [invertibile]
  \item[b)] \(y=x^2+9x+18\) \hfill [non invertibile]
  \item[c)] \(y=e^x\) \hfill [invertibile]
  \item[d)] \(y=\ln{x}\) \hfill [invertibile]
  \item[e)] \(y=x^3+3x^2+x\) \hfill [non invertibile]
  \item[f)] \(y=\sin x\) \hfill [non invertibile]
  \end{itemize}
  
  \item[1.11)] Determina l'inversa della funzione data, specificando il 
suo dominio.  
  \begin{itemize}
  \item[a)] \(y=4x+3\) \hfill [\(y=\frac{x-3}{4},\, 
D=\mathbb{R}\)]
  \item[b)] \(y=e^{2x}\) \hfill [\(y=\frac{\ln x}{2},\, 
D=]0,+\infty[\)]
  \item[c)] \(y=x+1\) \hfill [\(y=x-1,\, D=\mathbb{R}\)]
  \item[d)] \(y=\ln(x+1)\) \hfill [\(y=e^x-1,\, 
D=\mathbb{R}\)]
  \end{itemize}
  
  
  \item[1.12)] Date le funzioni \(f(x)\) e \(g(x)\) determina le 
espressioni analitiche di \(f\circ g\) e di \(g\circ f\).% non riesco a rendere 
ben leggibile questo esercizio
  \begin{itemize}
  \item[a)] \(f(x)=\sqrt[3]{x+3}\)  \\  \(g(x)=\log_4{x}\)  
\\  \hfill   \([(f\circ g)(x)=\sqrt[3]{\log_4{x+3}}; (g\circ 
f)(x)=\log_4{(\sqrt[3]{x+3})}]\)
  
  \item[b)] \(f(x)=3^x\)  \\ \(g(x)=x-5\)   \\   
\hfill  \([(f\circ g)(x)=3^{x-5}; (g\circ f)(x)=3^x-5]\)
  
  \item[c)] \(f(x)=\cos3x\)  \\ \(g(x)=\sqrt{x^2+3}\)  \\ 
   \hfill  \([(f\circ g)(x)=\cos{3\sqrt{x^2+3}}; (g\circ 
f)(x)=\sqrt{(\cos{3x})^2+3}]\)
  
  \item[d)] \(f(x)=x^2+3\) \\  \(g(x)=2x+1\)  \\  
\hfill  \([(f\circ g)(x)=(2x+1)^2+3=4x^2+4x+4; (g\circ 
f)(x)=2(x^2+3)+1=2x^2+7]\)
  \end{itemize}
\end{itemize}
%mancano esercizi su limitatezza e monotonia (provvedo a creare dei grafici e 
% semmai si aggiungono nei prossimi giorni...










