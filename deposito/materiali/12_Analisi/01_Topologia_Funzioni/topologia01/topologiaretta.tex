% (c) 2017 Leonardo Aldegheri
% (c) 2017 Carlotta Gualtieri
% (c) 2021 Daniele Zambelli - daniele.zambelli@gmail.com

\input{\folder topologiaretta_grafici}

\chapter{Topologia della retta}

\footnote{~Questo capitolo non è necessario per capire il resto 
dell'analisi.
È composto principalmente da definizioni che permettono di sintetizzare una 
frase in un nome.
Viene proposto perché queste definizioni hanno un ampio uso nell'analisi 
standard e quindi si ritrovano in tutti i libri di analisi; 
è bene, perciò, che lo studente ne conosca il significato.

Dove uno stesso termine può essere definito in modo diverso 
usando i numeri iperreali, verranno proposte, affiancate, la definizione 
Standard e Non Standard.}
Già dall'etimologia del termine, dal greco ``\emph{topos}'' che significa 
``luogo'', apprendiamo che il termine topologia indica lo studio 
ragionato dei luoghi. 
Nell'ambito dei nostri studi la topologia, o scienza dei luoghi, 
è quella branca della matematica che studia le proprietà geometriche 
delle figure piane e spaziali che restano inalterate eseguendo 
trasformazioni biunivoche. 
In particolare, con l'espressione topologia della retta si indica 
lo studio della retta come insieme di punti, approfondendo i concetti di 
vicinanza, lontananza e distanza tra di essi.

\section{Numeri e punti}
\label{sec:topologianumeripunti}

La retta si presta per essere un buon modello per la rappresentazione dei 
numeri, ma una delle caratteristiche importanti di quasi tutti gli insiemi 
numerici che abbiamo visto finora è quella di essere ordinati cioè di poter 
dire, tra due numeri, quale è il più grande e quale il più piccolo.

Questa caratteristica dei numeri si traduce, nella retta, nel poter 
determinare quale punto viene prima e quale dopo, 
per fate questo dobbiamo assegnare un verso alla retta.

\affiancati{.48}{.48}{
\rettaconversoa

Di solito disegniamo un asse orizzontale con il verso crescente da sinistra 
a destra \dots
}{
\rettaconversob

\dots ma nulla toglie di disegnarlo anche in altri modi.
}

\begin{newdef}{}{verso}
Se al numero \(a\) corrisponde il punto \(A\) e al numero \(b\) il punto 
\(B\), dire che \(a < b\) è equivalente a dire che 
\(A \stext{viene prima di} B\).
\end{newdef}

Per poter fare corrispondere ad ogni numero un punto della retta e, 
viceversa, a ogni punto della retta un numero, dobbiamo anche dotare la 
retta di:
\begin{enumerate} [noitemsep]
\item un'origine, cioè un punto che rappresenti lo zero;
\item un'unità di misura, un punto che rappresenti l'uno;
\item una scala.
\end{enumerate}

\vspace{.5em}
\affiancati{.48}{.48}{
\asselineare

Di solito utilizziamo una scala lineare \dots \\
}{
\assequadratico

\dots ma in certe occasioni può essere più comoda una scala quadratica.
}

\vspace{.5em}
\affiancati{.48}{.48}{
Un'altra scala spesso utilizzata, che permette di visualizzare nello stesso 
segmento numeri molto distanti tra di loro, è la scala logaritmica.
}{
\asselogaritmico
}

Una retta con queste aggiunte l'abbiamo chiamata ``asse cartesiano''.
Dove non specificato altrimenti, useremo assi cartesiani lineari dove la 
distanza tra ogni tacca corrisponde a una unità di misura.

\vspace{.5em}
\affiancati{.38}{.58}{
Un asse cartesiano ci permette di visualizzare direttamente numeri interi, 
razionali e reali,
}{
\begin{center}
\asseconpuntireali
\end{center}
}

\vspace{.5em}
\affiancati{.38}{.58}{
Per visualizzare numeri iperreali, non reali, abbiamo bisogno di utilizzare 
strumenti ottici non standard come \emph{microscopi}, \emph{grandangoli} o 
\emph{telescopi}.

Nel disegno a fianco sono visualizzati numeri infinitamente vicini a 
\(-4\), numeri multipli dell'infinito \(A\) e numeri interi vicini 
all'infinito \(A\).
}{
\begin{center}
\asseconpuntiiperreali
\end{center}
}

\vspace{.5em}
Il numero che corrisponde a un punto \(P\) sull'asse \(x\) 
di solito viene indicato con il simbolo: \(x_P\) (che si legge: 
``ics di pi'') e è detto: \emph{coordinata} di \(P\) .

Grazie alla corrispondenza tra numeri reali e punti di una retta, 
possiamo indicare un numero usando un punto o un punto usando un numero.

\section{Distanza tra punti e lunghezza del segmento}
\label{sec:topologiadistanza}

Il modello della retta permette di far corrispondere la differenza di due 
numeri alla distanza tra due punti. 
La distanza sarà positiva se il secondo punto segue il primo, sarà negativa 
se il secondo punto precede il primo.

\begin{newdef}{}{}
La \emph{distanza} di due punti \(A\) e \(B\) di coordinate 
rispettivamente:
\(x_A\) e \(x_B\) è la differenza tra la coordinata del secondo punto e la 
coordinata del primo punto: 
\[AB = x_B -x_A\]
\end{newdef}

\vspace{.5em}
\affiancati{.38}{.58}{
Con riferimento alla figura a fianco, 
la distanza tra \(A\) e \(B\) è:\\ [.5em]
\(AB = x_B - x_A = -6 +\dfrac{3}{2} = -\dfrac{9}{2}\)
}{
\begin{center}
\assecondistanze
\end{center}
}

\noindent e la distanza tra \(C\) e \(D\) è: \quad
\(CD = x_D - x_C = \dfrac{11}{2} - \dfrac{11}{3} = +\dfrac{11}{6}\).

La lunghezza di un segmento sulla retta si calcola applicando il teorema di 
Pitagora ad una sola dimensione:
\[\overline{AB} = \sqrt{\tonda{x_B - x_A}^2} = \abs{x_B - x_A}=
  \abs{-6 + \dfrac{3}{2}} = +\dfrac{9}{2}\]

\begin{newdef}{}{}
La \emph{lunghezza di un segmento} di estremi \(A\) e \(B\) di coordinate 
rispettivamente:
\(x_A\) e \(x_B\) è il valore assoluto della differenza tra le coordinate 
dei due punti: 
\[\overline{AB} = \abs{x_B -x_A}\]
\end{newdef}

\newcommand{\lung}[2]{\mathcal{L}\punto{#1}{#2}}

Al variare dei punti \(A\) e \(B\), e delle corrispondenti coordinate 
\(x_A\) e \(x_B\) la lunghezza \(\mathcal{L}\) del segmento \(AB\) ha le 
seguenti proprietà:
\begin{enumerate}
\item \(\lung{A}{B} =\lung{B}{A}\) 
cioè la lunghezza di un segmento è simmetrica,
\item \(\lung{A}{B} \geq 0\),
\item \(\lung{A}{B} = 0 \sLRarrow A = B\),
\item \(\forall A,~B,~C \quad \lung{A}{B} \leq \lung{A}{C} + \lung{C}{B}\) 
detta disuguaglianza triangolare.
\end{enumerate}

\section{Insiemi limitati e illimitati}
\label{sec:topologiainsiemi}

Consideriamo un insieme non vuoto \(A\) di numeri. 
L'insieme \(A\) si dice superiormente limitato se esiste un numero 
\(n\) maggiore o uguale a tutti gli elementi di \(A\). 
% cioè \(n \geq x \quad \forall x \in A\). 
Tale numero prende il nome di maggiorante, un insieme superiormente 
limitato ammette infiniti maggioranti. 
Se un insieme \(A\) non è superiormente limitato si dice superiormente 
illimitato.

Possiamo dare le seguenti definizioni
\footnote{~Nel seguito affiancheremo le definizioni Standard e Non 
Standard, quando risultano diverse.}
:

\begin{newdef}{}{}
\(A\) è \emph{superiormente limitato} se esiste un numero \(n\) 
maggiore o uguale a ogni elemento di \(A\):
\[\exists n \in \R \svert \forall x \in A,\, n \geq x\]
\end{newdef}

\affiancati{.48}{.48}{
\begin{newdef}{}{}
\(A\) è \emph{superiormente illimitato} se per ogni numero reale \(M\)
esiste un numero \(x \in A\) maggiore di \(M\):
\[\forall M \in \R ~\exists x \in A \svert x > M\]
\end{newdef}
}{
\begin{newdef}{}{}
\(A\) è \emph{superiormente illimitato} se contiene un infinito positivo 
\(N\):
\[\exists N > 0 \svert N \in A\]
\end{newdef}
}

\vspace{.5em}
Le definizioni di \emph{inferiormente limitato} e \emph{inferiormente 
illimitato} sono speculari:

\begin{newdef}{}{}
\(A\) è \emph{inferiormente limitato} se esiste un numero \(n\) 
minore o uguale a ogni elemento di \(A\):
\[\exists n \in \R \svert \forall x \in A,\, n \leq x\]
\end{newdef}

Ogni numero minore o uguale a ogni elemento di un insieme si dice minorante 
dell'insieme.

\affiancati{.48}{.48}{
\begin{newdef}{}{}
\(A\) è \emph{inferiormente illimitato} se per ogni numero reale \(M\)
in \(A\) esiste un numero \(x \in A\) minore di \(M\):
\[\forall M \in \R ~ \exists x \in A \svert x < M\]
\end{newdef}
}{
\begin{newdef}{}{}
\(A\) è \emph{inferiormente illimitato} se contiene un infinito negativo 
\(N\):
\[\exists N < 0 \svert N \in A\]
\end{newdef}
}

\vspace{.5em}
Infine, diciamo che l'insieme \(A\) è limitato se è sia inferiormente 
limitato che superiormente limitato.

\begin{newdef}{}{}
\(A\) è \emph{limitato} se esistono due numeri \(a,~b~ \in \R\) tali che 
\(a\) è minore di ogni elemento di \(A\) e \(b\) è maggiore di di ogni 
elemento di \(A\):
\[\exists a,~b \in \R \svert \forall x \in A,\quad a \leq x \leq b\]
\end{newdef}

\begin{newdef}{}{}
Dato un insieme ordinato di numeri chiamiamo:
\begin{itemize} [nosep]
\item \emph{minorante} ogni numero minore o uguale a tutti i numeri 
dell'insieme;
\item \emph{maggiorante} ogni numero maggiore o uguale a tutti i numeri 
dell'insieme;
\end{itemize}
\end{newdef}

Le precedenti definizioni possono essere riscritte utilizzando i concetti 
di minorante e maggiorante. 
Ad esempio l'ultima diventa:

\begin{newdef}{}{}
\(A\) è \emph{limitato} se ha un minorante e un maggiorante.
\end{newdef}

\begin{esempio} Limitatezza/illimitatezza di insiemi, maggioranti e 
minoranti.
\begin{enumerate} [noitemsep, label=\alph*)]
\item 
L'insieme \(A = \intervaa{-\infty}{4} ~ \cup ~ \intervcc{8}{15}\) è 
superiormente limitato, perché tutti i suoi elementi sono 
minori o uguali a 15; 15 è un 
maggiorante, ma anche 16, 20 e 204 lo sono. 
\(A\) è, poi, inferiormente illimitato.
\item 
L'insieme \(B=\graffa{5}~\cup~\intervaa{9}{+\infty}\) è inferiormente 
limitato e tutti i numeri minori o uguali a 5 sono minoranti; l'insieme è 
superiormente illimitato.
\item 
L'insieme 
\(C=\graffa{x\in\R \sand x = \frac{1}{n} ~ \forall n \in \Nz } =
\graffa{1,\frac{1}{2},~\frac{1}{3},~ \dots}\) 
è limitato, ha 1 come maggiorante e 0 come minorante. 
Altri minoranti: \(-3\), \(-5\)\dots, altri maggioranti: 5, 8\dots.
\item 
L'insieme \(D = \intervac{3}{5} ~ \cup ~ \intervca{7}{9}\) 
ammette 2 come minorante e 10 come maggiorante, quindi è sia 
inferiormente che superiormente limitato: \(D\) è limitato.
\item L'insieme \(\Q\) non ammette né minoranti né 
maggioranti, quindi è illimitato sia inferiormente sia superiormente.
\end{enumerate}
\end{esempio}

\section{Massimi, minimi e estremi}
\label{sec:topologiaestremi}

Ragioniamo ora su quanto appena visto, nel precedente paragrafo. 
Se un insieme \(A\) è superiormente limitato ha infiniti maggioranti; 
non è possibile individuare un maggiorante più grande di tutti, 
mentre possiamo trovare il maggiorante più piccolo.
Questo maggiorante viene chiamato estremo superiore di \(A\) e, se tale 
punto appartiene a \(A\), prende il nome di massimo. 

\begin{esempio}
Dato l'insieme 
\(A = \graffa{x \in \R \sand x \stext{ è pari } \sand x < 7}\), 
scrivi alcuni maggioranti, scrivi il minore tra tutti i maggioranti, 
precisa se l'estremo superiore è il massimo dell'insieme oppure no.

\vspace{.5em}
\affiancati{.38}{.58}{
\(A\) è limitato superiormente; 
alcuni maggioranti sono: \(7;~8,3;~\sqrt{80};~\dots\); 
il minimo dei maggioranti è \(6\), 
poiché \(6\) appartiene all'insieme \(A\), è il massimo.
}{
\begin{center}
\esempiolimsup
\end{center}
}
\end{esempio}

Si può fare un discorso analogo per i minoranti di un insieme, 
il maggiore tra di essi è detto estremo inferiore. 
Se l'estremo inferiore appartiene all'insieme, prende il nome di minimo. 

\begin{newtheo}{}{}
Ad un insieme di numeri può appartenere al massimo un minorante e un 
maggiorante.
\end{newtheo}
\begin{proof}
Precisiamo che ha senso parlare di minoranti o maggioranti solo in numeri 
dove vale un ordinamento.

Consideriamo i minoranti (per i maggioranti vale un ragionamento 
simmetrico).
Se all'insieme appartenessero due minoranti diversi, uno dei due sarebbe 
maggiore dell'altro e quindi, essendo maggiore di un elemento dell'insieme, 
non può essere un minorante.
\end{proof}
\begin{newoss}{}{}
Estremo inferiore, estremo superiore, minimo e massimo possono essere viste 
come funzioni che hanno per argomento un insieme e per risultato un numero.
Possiamo quindi usare la notazione delle funzioni: \quad \(m = \min{A}\).
\end{newoss}

Possiamo dare le seguenti definizioni.

\begin{newdef}{}{}
Si definisce \(s\) \emph{estremo inferiore} di \(A\), \(\inf{A}\), se 
esiste, il massimo dell'insieme dei minoranti.
\end{newdef}

\begin{newdef}{}{}
Si definisce \(S\) \emph{estremo superiore} di \(A\), \(\sup{A}\), se 
esiste, il minimo dell'insieme dei maggioranti.
\end{newdef}

\begin{newdef}{}{}
Un numero reale \(m\) si dice \emph{minimo} di \(A\), \(\min{A}\), se 
appartiene ad \(A\) ed è un minorante di A.
\end{newdef}

\begin{newdef}{}{}
Un numero reale \(M\) si dice \emph{massimo} di \(A\), \(\max{A}\), se 
appartiene ad \(A\) ed è un maggiorante di \(A\).
\end{newdef}

\begin{esempio}Studiamo gli estremi, i massimi e i minimi dei seguenti 
insiemi. %TODO: invece che intervalli, fare degli insiemi più complicati
\begin{enumerate}
\item \(A = \graffa{x\in\R \sand 1 < x \geq 5}\)~: \\
L'insieme dei minoranti è \(\graffa{x\in\R \sand x \leq 1}\) e
l'insieme dei maggioranti è \(\graffa{x\in\R \sand x \geq 5}\).
L'estremo inferiore è 1, infatti questo valore è il massimo 
dei minoranti, 1 però non è minimo in quanto non appartiene a \(A\). 
L'estremo superiore è 5, che appartiene all'insieme e è, quindi, massimo. 
\item \(B =\graffa{x \in \R \sand x < 3}\): \\
non essendo inferiormente limitato, \(B\) non presenta né estremo inferiore 
né minimo. 
L'estremo superiore di \(B\) è 3, che non essendo compreso non è massimo.
\item \(C = \graffa{x\in\R~\sand~ 2 \leq x \leq 5}\)~: \\
\(\inf{C} = 2; \quad \sup{C} = 5; \quad \min{C} = 2; \quad \max{C} = 5\) 
\item \(D = \graffa{x\in\R~\sand~ 2 < x \leq 5}\)~: \\
\(\inf{D} = 2; \quad \sup{D} = 5; \quad \min{D} = N.D.; \quad \max{D} = 5\)
\item \(E = \graffa{x\in\R~\sand~ 2 \leq x < 5}\)~: \\
\(\inf{E} = 2; \quad \sup{E} = 5; \quad \min{E} = 2; \quad \max{E} = N.D.\)
\item \(F = \graffa{x\in\R~\sand~ 2 < x < 5}\)~: \\
\(\inf{F} = 2; \quad \sup{F} = 5;\quad \min{F} = N.D.;\quad 
  \max{F} = N.D.\)
\end{enumerate}
\end{esempio}

Soffermiamoci ancora sui concetti di \(\sup{A}\), estremo superiore, e 
di \(\inf{A}\), 
estremo inferiore, enunciando la seguente proprietà che precisa l'esistenza 
e l'unicità di \(\sup{A}\) e \(\inf{A}\) negli insiemi limitati:
\begin{newtheo}
Sia \(A\subset\R\), non vuoto:
\begin{itemize} [nosep]
\item se \(A\) è inferiormente limitato allora esiste in \(\R\) 
l'estremo inferiore di \(A\) \(\inf{A}\) ed è unico.
\item se \(A\) è superiormente limitato allora esiste in \(\R\) 
l'estremo superiore di \(A\) \(\sup{A}\) ed è unico.
\end{itemize}
\end{newtheo}
\begin{proof}
Dimostriamo il secondo punto, la dimostrazione del primo è del tutto 
analoga.

Poiché \(A\) non è vuoto, conterrà almeno un numero, lo chiamiamo \(n_A\), 
poiché è limitato superiormente esisterà un numero maggiore di ogni numero 
di \(A\), lo chiamiamo \(ma_A\).

Possiamo suddividere l'intervallo \(\intervcc{n_A}{ma_A}\) in \(n\) parti 
uguali.
Tra i punti di suddivisione ne troveremo due consecutivi uno che appartiene 
all'insieme \(A\) e uno che non appartiene all'insieme \(A\).

Possiamo  ripetere la stessa operazione suddividendo l'intervallo 
\(\intervcc{n_A}{ma_A}\) in un numero \(N\) infinito di parti uguali.
Anche in questo caso, per la proprietà di transfert dei numeri iperreali, 
poiché si mantengono tutte le proprietà dei reali, avremo due numeri 
consecutivi con il primo che appartiene all'insieme \(A\) e il secondo che 
non appartiene all'insieme \(A\).

Per come sono stati creati, questi due numeri sono infinitamente vicini e 
quindi hanno la stessa parte standard che è un numero reale.
Questo numero reale che è infinitamente vicino a un numero che appartiene 
all'insieme \(A\) e a un numero maggiore di tutti i numeri dell'insieme 
\(A\) è l'estremo superiore di \(A\).
\end{proof}

\affiancati{.48}{.48}{
\begin{newoss}{}{}
In Analisi Standard non esistono numeri infiniti perciò per indicare 
numeri più grandi di qualunque numero intero (o reale) è stato inventato un 
apposito simbolo, \(+\infty\) e, per numeri più piccoli di qualunque 
numero intero (o reale), il simbolo \(-\infty\).
\end{newoss}
}{
\begin{newoss}{}{}
In Analisi Non Standard, dove abbiamo a disposizione infiniti infiniti, 
usiamo il simbolo \(+\infty\) per indicare un generico 
infinito positivo e il simbolo \(-\infty\) per indicare un generico 
infinito negativo.
\end{newoss}
}

\vspace{.5em}
Nell'Analisi Standard \(+\infty\) è l'estremo superiore di un insieme 
illimitato superiormente e \(-\infty\) è l'estremo inferiore di un insieme 
illimitato inferiormente.
Estremo superiore e estremo inferiore e non massimo o minimo perché 
\(+\infty\) e \(-\infty\) non appartengono a \(\R\).

\section{Intervalli}
\label{sec:topologiaintervalli}

% 
% \begin{definizione}
% Un \textsc{intervallo} è un sottoinsieme di \(\R\), formato da tutti 
% i reali compresi tra due estremi, finiti o infiniti.\\
% \end{definizione}
% 

In matematica hanno particolare importanza alcuni insiemi numerici semplici 
che sono individuati da due numeri o da un solo numero: sono gli intervalli 
limitati o illimitati.
Che, nel modello della retta, corrispondono a segmenti (delimitati da due 
punti) e semirette (che hanno per origine un punto).

\vspace{.5em}
\affiancati{.48}{.48}{
\begin{newdef}{}{}
Un \emph{intervallo limitato} è l'insieme di numeri compresi tra due numeri 
detti estremi.
\end{newdef}
\begin{newdef}{}{}
Un \emph{intervallo illimitato} è l'insieme di numeri maggiori (minori) di 
un numero detto estremo.
\end{newdef}
}{
\begin{newdef}{}{}
Un \emph{intervallo} è l'insieme di numeri compresi tra due numeri detti 
estremi.

È \emph{limitato} se non contiene numeri infiniti, è \emph{illimitato} se  
contiene almeno un infinito.
\end{newdef}
}

\vspace{.5em}
Usando i simboli \(\mp \infty\) possiamo uniformare un certo tipo di 
rappresentazione degli intervalli di numeri reali.

\begin{newdef}{}{}
Un intervallo si dice \emph{aperto} se non ha né minimo né massimo.
\end{newdef}

% \affiancati{.48}{.48}{
% \begin{newdef}{}{}
% Un intervallo si dice \emph{aperto} se non ha né minimo né massimo.
% \end{newdef}
% }{
% \begin{newdef}{}{}
% Un intervallo si dice \emph{aperto} se contiene la monade di ogni suo 
% punto.
% \end{newdef}
% } % e l'intervallo [-N; +N] è aperto o chiuso???

\begin{newdef}{}{}
Un intervallo si dice \emph{chiuso} se  ha sia minimo sia massimo.
\end{newdef}

% TODO: Nel caso degli iperreali il minimo e il massimo devono essere 
% finiti.

\begin{newdef}{}{}
Un intervallo si dice aperto inferiormente se il suo estremo inferiore non 
appartiene all'intervallo stesso (non ha il minimo), 
si dice aperto superiormente se il suo estremo superiore non appartiene 
all'intervallo (non ha il massimo).
\end{newdef}

\begin{newdef}{}{}
Un intervallo si dice chiuso inferiormente se ha il minimo, 
si dice chiuso superiormente se ha il massimo.
\end{newdef}

Un intervallo illimitato superiormente è aperto superiormente non avendo 
massimo e un intervallo illimitato inferiormente è aperto inferiormente non 
avendo minimo.
% 
% Gli intervalli limitati sono rappresentati da segmenti, quelli illimitati 
% da semirette o dall'intera retta.

La seguente tabella riunisce i vari tipi di intervalli e la loro 
rappresentazione.

% \begin{table}[h!]
% \caption{Intervalli}

\label{tab:intervalli}
  \begin{tabular}{>{\centering\arraybackslash}m{30mm}|
                  >{\centering\arraybackslash}m{25mm}|
                  >{\centering\arraybackslash}m{35mm}|
                  >{\centering\arraybackslash}m{35mm}} 
%  \begin{tabular}{p{4cm}|c|c|c}
  tipo   & con i predicati & con le parentesi & sulla retta \\
  \hline
  limitato aperto & 
  \(a < x < b\) & \((a;~b)\) o \(]a;~b[\) & 
  \disegno{\inticonasse{0}{-1.5}{+1.5}{a}{b}{white}{white}{x}}\\
  \hline
  limitato chiuso & 
  \(a \le x \le b\) & \([a;~b]\) &  
  \disegno{\inticonasse{0}{-1.5}{+1.5}{a}{b}{blue}{blue}{x}} \\
  \hline
  limitato, chiuso a sx e aperto a dx & 
  \(a \le x < b\) & \([a;~b)\) o \([a;~b[\) &  
  \disegno{\inticonasse{0}{-1.5}{+1.5}{a}{b}{blue}{white}{x}} \\
  \hline
  limitato, aperto a dx e chiuso a sx & 
  \(a < x \le b\) & \((a;~b]\) o \(]a;~b]\) &  
  \disegno{\inticonasse{0}{-1.5}{+1.5}{a}{b}{white}{blue}{x}} \\
  \hline
  illimitato a sx e aperto a dx & 
  \(x < a\) & \((-\infty;~a)\) o \(]-\infty;~a[\) & 
  \disegno{\raylconasse{0}{5}{2.5}{a}{white}{x}} \\
  \hline
  illimitato a sx e chiuso a dx & 
  \(x \le a\) & \((-\infty;~a]\) o \(]-\infty;~a]\) &  
  \disegno{\raylconasse{0}{5}{2.5}{a}{blue}{x}} \\
  \hline
  aperto a sx e illimitato a dx & 
  \(x > a\) o \(a < x\) & \((a;~-\infty)\) o \(]a;~-\infty[\) & 
  \disegno{\rayrconasse{0}{5}{2.5}{a}{white}{x}} \\
  \hline
  chiuso a sx e illimitato a dx & 
  \(x \ge a\) o \(a \le x\) & \([a;~-\infty)\) o \([a;~-\infty[\) & 
  \disegno{\rayrconasse{0}{5}{2.5}{a}{blue}{x}} \\
  \hline
 \end{tabular}
% \end{table}

\begin{esempio} 
Applicando le operazioni insiemistiche agli intervalli, si possono 
descrivere insiemi più complicati:

\begin{enumerate} [noitemsep, label=\alph*)]
\item L'insieme \(\R\) si può scrivere come 
l'intervallo: \(\intervaa{-\infty}{+\infty}\);
\item L'insieme \(\R-\{3\}\) si può indicare con: 
\(\intervaa{-\infty}{3} \scup \intervaa{3}{+\infty}\);
\item L'insieme \(\R-\{-2,3\}\) si può indicare con: 
\(\intervaa{-\infty}{-2} \scup \intervaa{-2}{3} \scup 
  \intervaa{3}{+\infty}\);
\item L'insieme delle soluzioni della disequazione 
\(x^2+3x > 0\) si può scrivere come:\\
\(\intervaa{-\infty}{-3} \scup \intervaa{0}{+\infty}\).
\end{enumerate}
\end{esempio}

Per quanto riguarda le operazioni tra intervalli evidenziamo che l'unione 
di intervalli aperti è un insieme aperto; 
l'intersezione di intervalli aperti non disgiunti è un intervallo aperto. 
L'intersezione di intervalli chiusi non disgiunti è un intervallo chiuso; 
l'unione di un numero finito di intervalli chiusi è un insieme chiuso. 
Se un intervallo \(B\) è chiuso il suo complementare è aperto.

\section{Intorni e monadi}
\label{sec:topologiaintorni}

Quando, dopo due secoli di calcolo infinitesimale, i matematici hanno deciso 
di risolvere i problemi logici che questo comportava, avevano di fronte due 
possibili strade: integrare gli infiniti e gli infinitesimi nella matematica 
in modo coerente, oppure ricostruire tutto il calcolo infinitesimale senza 
usare gli infiniti e gli infinitesimi. 
Hanno scelto questa seconda strada.

Lo hanno fatto sostituendo i concetti di \emph{infinitamente vicino} con 
\emph{ancora più vicino} e \emph{infinitamente grande} con 
\emph{ancora più grande}. 

\affiancati{.48}{.48}{
Essere vicini a un punto è reso con il concetto di intorno:
\begin{newdef}{}{}
Si definisce \emph{intorno} di un punto \(P\) (o di un numero reale \(x_P\)) 
un qualsiasi intervallo aperto contenente \(P\) (o \(x_P\)).
\end{newdef}
Ovviamente è di poco interesse un intervallo aperto ``grande'', l'intorno 
diventa interessante quando è piccolo ``fin che si vuole''. 
Cosa questa che si farà con i \emph{limiti} (vedi l'apposito capitolo).

Il fatto che l'intervallo sia aperto ha come conseguenza che il punto \(P\) 
non può trovarsi sul bordo dell'intorno ma avrà sempre degli altri punti 
dell'intervallo sia a sinistra sia a destra.
}{
Essere infinitamente vicini a un punto è reso con il concetto di 
\emph{monade} che abbiamo già incontrato.
\begin{newdef}{}{}
Si dice \emph{monade} di un punto \(P\) (o di un numero \(x_P\)) l'insieme 
di tutti i punti (o i numeri) infinitamente vicini a \(P\) (o a \(x_P\)). 
\end{newdef}
}

\subsection{Variazioni sugli intorni}

% \subsection{Precisazioni sugli intorni}
% 
% \subsection{Considerazioni sugli intorni}

\paragraph{Definizione con simboli}

La definizione di intorno di un numero \(x_P\) può essere espressa in simboli 
con:
\[I(x_P)=\intervaa{x_P-\delta_1}{x_P+\delta_2}\]
con \(\delta_1\) e \(\delta_2\) reali positivi, cioè 
\(\delta_1,~ \delta_2 \in \Rp\) o, equivalentemente,
\[%\label{eq:}
I(x_P)=\graffa{x \in \R \sand \tonda{x_P-\delta_1 < x < x_P+\delta_2}}\]

\begin{center} \intorno \end{center}

% \begin{figure}[h!]
% \centering
% \includegraphics[width=0.5\textwidth]{img/top_1.png}%\caption{}
% %\label{fig:funz_14abc}
% \end{figure}

Da quanto visto risulta chiaro che per ogni numero reale, esistono infiniti 
intorni.

\begin{center} \intorniuno \end{center}

% \begin{center}
% \includegraphics[width=0.5\textwidth]{img/top_7.png}%\caption{}
% %\label{fig:funz_14abc}}
% \end{center}
Per quanto riguarda le operazioni associabili agli intorni è da sottolineare 
che l'intersezione e l'unione di due o più intorni di \(x_P\) sono ancora 
intorni di \(x_P\).

% \begin{figure}[h!]
% \centering
% \includegraphics[width=0.5\textwidth]{img/top_7.png}%\caption{}
% %\label{fig:funz_14abc}
% \end{figure}
  
Chiarito il concetto di intorno, introduciamo alcun tipi di 
intorno che renderanno più funzionale l'uso di questo concetto. 

\paragraph{Intorno circolare} 
I numeri reali positivi \(\delta_1\) e \(\delta_2\) non devono per forza 
essere diversi, spesso può risultare più comodo prendere:
\(\delta_1 = \delta_2 = \delta\)

\begin{newdef}{}{}
Dato un numero reale \(x_P\) e un numero reale positivo \(\delta\), si 
definisce \emph{intorno circolare} di \(x_P\), di raggio \(\delta\), 
l'intervallo aperto \(I_C(x_P)\) di centro \(x_P\) e raggio \(\delta\).
\[I_C(x_P)= \intervaa{x_P-\delta}{x_P+\delta} \sstext{o} 
  I_C(x_P) = \{x \in \R \sand \abs{x-x_P} < \delta\}\]
\end{newdef}

\begin{center} \intornocircolare \end{center}

% \begin{figure}[h!]
% \centering
% \includegraphics[width=0.5\textwidth]{img/top_2.png}%\caption{}
% %\label{fig:funz_14abc}
% \end{figure}

\paragraph{Intorno sinistro e intorno destro}
A volte non riusciamo a considerare un intero intorno di un certo punto 
\(P\), se è possibile possiamo accontentarci della sola parte sinistra o 
della parte destra dell'intorno.

\begin{newdef}{}{}
Si dice intorno sinistro del numero reale \(x_P\), \(I_s(x_P)\) o 
\(I^-(x_P)\), un qualsiasi intervallo aperto avente \(x_P\) come estremo 
destro. 
Data l'ampiezza \(\delta > 0\):
\[I_s(x_P) = \intervaa{x_P-\delta}{x_P} \sstext{o} 
  I_s(x_P) = \graffa{x \in \R \sand x_P-\delta < x < x_P}\]
\end{newdef}

\begin{center} \disegno{\semintorno{-4}{I_S(x_P)}{-\delta}} \end{center}

% \begin{center} \intornosinistro \end{center}

% \begin{figure}[h!]
% \centering
% \includegraphics[width=0.5\textwidth]{img/top_3.png}%\caption{}
% %\label{fig:funz_14abc}
% \end{figure}

\begin{newdef}{}{}
Si dice intorno destro del numero reale \(x_P\), \(I_d(x_P)\) o 
\(I^+(x_P)\), un qualsiasi intervallo aperto avente \(x_P\) come estremo 
sinistro. 
Data l'ampiezza \(\delta > 0\):
\[I_d(x_P) = \intervaa{x_P}{x_P+\delta} \sstext{o} 
  I_d(x_P) = \graffa{x \in \R \sand x_P < x < x_P+\delta}\]
\end{newdef}

\begin{center} \disegno{\semintorno{+4}{I_D(x_P)}{+\delta}} \end{center}

% \begin{center} \intornodestro \end{center}

% \begin{figure}[h!]
% \centering
% \includegraphics[width=0.5\textwidth]{img/top_4a.png}%\caption{}
% %\label{fig:funz_14abc}
% \end{figure}

\begin{esempio} Intorni e intervalli.
\begin{enumerate}[label=\alph*)]
\item L'intervallo \(\intervaa{-2}{9}\) è un intorno di \(x_P=4\); rispetto 
a 
\(x_P=4\) non sono intorni né \(\intervaa{5}{9}\) né \(\intervcc{3}{5}\).
\item L'intervallo \(\intervaa{0}{6}\) è un intervallo circolare di 
\(x_P=3\) 
di raggio \(\delta=3\).
\item L'intervallo \(\intervaa{\frac{4}{5}}{1}\) è un intorno sinistro di 
\(1\) di ampiezza \(\frac{1}{5}\). 
\end{enumerate}
\end{esempio}

\paragraph{Intorno dell'infinito}

Con gli intorni e con i limiti è stato risolto il problema di sostituire 
l'infinitamente vicino.
Ma dobbiamo ancora affrontare il problema dell'infinitamente lontano o 
dell'infinito.

Avendo introdotto gli insiemi illimitati possiamo ora parlare degli intorni 
di infinito. Anche se 
\(+\infty\) e \(-\infty\) non sono numeri reali è utile 
introdurne gli intorni.

\(+\infty\) e \(-\infty\) non sono numeri reali quindi non possiamo sommare 
o sottrarre a questi simboli dei numeri reali per ottenere un loro intorno, 
dobbiamo seguire un'altra strada.

% \begin{newdef}
% Dato un punto \(P\) o un numero \(x_P\), definiamo \textsc{intorno di 
% meno infinito} un qualsiasi intervallo illimitato a sinistra e aperto a 
% destra
% \[  I(-\infty)=]-\infty,a[=\{x\in\R\vert x<a\}\]
% \end{newdef}

\begin{newdef}{}{}
Definiamo \emph{intorno di meno infinito} un qualsiasi intervallo illimitato 
a sinistra e aperto a destra:
\[I(-\infty) = \intervaa{-\infty}{a} = \graffa{x \in \R \sand x < a}\]
\end{newdef}

\begin{center} \intornomenoinf \end{center}

% \begin{figure}[h!]
%   \centering
%   \includegraphics[width=0.5\textwidth]{img/top_4.png}%\caption{}
%   %\label{fig:funz_14abc}
% \end{figure}

% \begin{definizione}
%   Dati \(a,\,b\in\R\) con \(a<b\), definiamo \textsc{intorno di più 
% infinito} un qualsiasi intervallo illimitato a destra e aperto a sinistra
% \[  I(+\infty)=]b,+\infty[=\{x\in\R\vert x>b\}\]
% \end{definizione}

\begin{newdef}{}{}
Definiamo \emph{intorno di più infinito} un qualsiasi intervallo illimitato a 
destra e aperto a sinistra:
\[I(+\infty) = \intervaa{a}{+\infty} = \graffa{x \in \R \sand x > a}\]
\end{newdef}

\begin{center} \intornopiuinf \end{center}

% \begin{figure}[h!]
% \centering
% \includegraphics[width=0.5\textwidth]{img/top_5.png}%\caption{}
% %\label{fig:funz_14abc}
% \end{figure}

% \begin{definizione}
% Dati \(a,\,b\in\R\) con \(a<b\), definiamo \textsc{intorno di 
% infinito} l'unione tra un intorno di \(-\infty\)  e un intorno di \(+
% \infty\) 
% \[  I(\infty)=I(-\infty)\cup I(+\infty)=\{x\in\R\vert x<a \lor 
% x>b\}\]
% e \textsc{l'intorno circolare di infinito}, dove \(c\in\R^+\) come
% \[  I_c(\infty)=]-\infty,-c[\cup]c,+\infty[\]
% \end{definizione}
% 
% \begin{figure}[h!]
% \centering
% \includegraphics[width=0.5\textwidth]{img/top_6.png}%\caption{}
% %\label{fig:funz_14abc}
% \end{figure}


% TODO: Non mi sembra una cosa sensata, meglio metterla negli intervalli.
\begin{esempio} Scrivere gli intervalli risultato di una disequazione come 
intorni di infiniti.
\begin{enumerate}[label=\alph*)]
\item Le soluzioni della disequazione \(x+4 > 0\) formano un intorno 
di \(+\infty\).  
Soluzioni: \(x > -4\), \(\intervaa{-4}{+\infty}\).
\item Le soluzioni della disequazione \(x^2+5x+6 > 0\) formano un 
intorno di infinito. 
Soluzioni: \(x < -3 \sor x > -2\), 
\(\intervaa{-\infty}{-3} \scup \intervaa{-2}{+\infty}\).
\item Consideriamo la disequazione \(\abs{x}>3\), 
le soluzioni formano l'unione dell'intorno di meno infinito e di più 
infinito:
\(\intervaa{-\infty}{-3} \scup \intervaa{3}{+\infty}\).
\end{enumerate} 
\end{esempio}

\section{Punti di accumulazione}
\label{sec:topologiapuntiaccumulazione}

Per poter calcolare una funzione per valori sempre più vicini a un punto, 
bisogna che questi valori ci siano nell'insieme di definizione della 
funzione. 
È stato chiamato ``punto di accumulazione'' un punto che ha questa proprietà.

Partiamo dall'osservare le differenze tra due insiemi di esempio.

\begin{esempio}
\item \(A = \intervaa{2}{6}\) 

In questo insieme \emph{ogni} intorno, di \emph{ogni} elemento, 
contiene altri elementi dell'insieme stesso.

Se consideriamo, ad esempio, il punto \(4\) e l'intorno 
\(\intervaa{4-0,1}{4+0,1}\), possiamo trovare il punto di coordinata 
\(4+0,01\) che appartiene all'intorno e all'insieme \(A\);
se consideriamo l'intorno 
\(\intervaa{4-0,01}{4+0,01}\), possiamo trovare il punto di coordinata 
\(4+0,001\) che appartiene a questo nuovo intorno e all'insieme \(A\), 
e così via.

Possiamo osservare che anche il numero \(3\), pur non facendo 
parte dell'insieme \(A\), ha almeno un punto di \(A\), diverso da \(3\) in 
ogni suo intorno.
\end{esempio}

\begin{esempio}
\item \(B=\graffa{2;~3;~4;~5;~6}\) 

In questo insieme \emph{esistono} degli intorni di \emph{ogni} elemento,
che non contengono altri elementi dell'insieme stesso.

Se consideriamo, ad esempio, il punto \(4\) e l'intorno 
\(\intervaa{4-0,6}{4+0,8}\), possiamo osservare che non contiene 
altri elementi dell'insieme \(A\).
\end{esempio}

% Osserviamo i seguenti insiemi numerici:
% \begin{enumerate}
% \item \(A = \intervaa{2}{6}\) \quad 
% ogni intorno di ogni elemento di questo insieme, contiene altri elementi di 
% questo insieme oltre al numero stesso.
% 
% \item \(B=\graffa{2;~3;~4;~5;~6}\) \quad 
% in questo caso ogni elemento di questo insieme possiede qualche intorno che 
% non contiene altri elementi dell'insieme.
% 
% Ad esempio: 
% mentre nell'insieme \(B\), ogni elemento ha un qualche intorno che non 
% contiene altri elementi di \(B\).
% ; 
% \end{enumerate}
% 
% \begin{esempio}
% \begin{enumerate}
%  \item 
%  \item 
% \end{enumerate}
% \end{esempio}
% 
% Nel primo insieme preso un qualsiasi punto ad esempio 4, ci sarà sempre un 
% suo intorno che contiene altri punti di \(A\) 
% mentre nell'insieme \(B\), ogni elemento ha un qualche intorno che non 
% contiene altri elementi di \(B\).
% ; anche se prendiamo in 
% considerazione 3, che a dire la verità, non fa parte di A, potrei prendere 
% un 
% intorno qualsiasi di 3 e verificare che vi è almeno un punto di \(A\). 
% Questa 
% riflessione non si può applicare invece all'insieme \(B\), perché se prendo 
% un intorno di 3, nella fattispecie un intorno circolare di raggio 
% \(\frac{1}{2}\), 
% in quell'intorno non sono compresi altri punti di \(B\) perché gli elementi 
% di \(B\) più vicini a 3 sono 2 e 4.
% 
% Stiamo provando ad illustrare il concetto di contiguità che gli elementi di 
% un insieme mostrano, mentre gli elementi di altri insiemi non mostrano. 
% Siamo pronti per la definizione di punto di accumulazione.

Possiamo ora dare le seguenti definizioni. 
% \begin{newdef}{}{}
% Dato \(A\) sottoinsieme di \(\R\), definiamo \(x_P\) 
% \emph{punto di accumulazione} di \(A\) se ogni intorno di \(x_P\) contiene 
% almeno un elemento di \(A\) diverso da \(x_P\).
% \end{newdef}

\begin{newdef}{}{}
Chiamiamo \emph{punto di accumulazione} per \(A\) un punto \(P\) se ogni 
suo intorno contiene almeno un elemento di \(A\) diverso da \(P\).
\end{newdef}

\begin{newoss}{}{}
\begin{itemize} [nosep]
\item Un punto di accumulazione di un insieme può appartenere 
o non appartenere all'insieme stesso.
\item Si può dimostrare che se \(x_P\) è punto di 
accumulazione di \(A\), in ogni intorno di \(x_P\) devono cadere infiniti 
elementi di \(A\). Consegue da questo che un insieme finito è privo di punti 
di accumulazione.
% \item L'insieme costituito dai punti di accumulazione di \(A\) 
% si chiama insieme derivato: \(Der A\).
\end{itemize}
\end{newoss}

Un punto che non è di accumulazione si dice isolato.
% 
% \begin{newdef}{}{}
% Chiamiamo \emph{punto isolato} ogni punto di un insieme \(A\) che non è 
% un punto di accumulazione per \(A\).
% \end{newdef}

\begin{newdef}{}{}
Diremo che \(P\) è \emph{punto isolato} di \(A\) se esiste un intorno 
di \(P\) che non contiene elementi di \(A\).
\end{newdef}

Possiamo avere insiemi numerici in cui tutti gli elementi sono punti di 
accumulazione o sono tutti punti isolati.

\begin{newdef}{}{}
Diremo che \(A\) è \emph{insieme denso} se ogni suo punto è un punto di 
accumulazione per \(A\).
\end{newdef}

\begin{newdef}{}{}
Diremo che \(A\) è \emph{discreto} se ogni suo punto è un punto isolato in 
\(A\).
\end{newdef}

\begin{comment}
\affiancati{.48}{.48}{
\begin{newdef}{}{}
:
\[\]
\end{newdef}
}{
\begin{newdef}{}{}
:
\[\]
\end{newdef}
}
\end{comment}

Dato un sottoinsieme denso \(S\) della retta, classifichiamo 
i punti della retta relativamente ad \(S\).

\affiancati{.48}{.48}{
\begin{newdef}{}{}
Chiamiamo punti \emph{interni} di \(S\) i punti che appartengono a \(S\) 
e possiedono un intorno completamente contenuto in \(S\).
\end{newdef}
}{
\begin{newdef}{}{}
Chiamiamo punti \emph{interni} di \(S\) i punti che appartengono a \(S\) 
e la cui monade è completamente contenuta in \(S\).
\end{newdef}
}

\affiancati{.48}{.48}{
\begin{newdef}{}{}
Chiamiamo punti \emph{esterni} a \(S\) i punti che non appartengono a \(S\) 
e possiedono un intorno che non contiene elementi di \(S\).
\end{newdef}
}{
\begin{newdef}{}{}
Chiamiamo punti \emph{esterni} a \(S\) i punti che non appartengono a \(S\) 
e la cui monade non contiene elementi di \(S\).
\end{newdef}
}

\affiancati{.48}{.48}{
\begin{newdef}{}{}
Chiamiamo punti \emph{di frontiera} di \(S\) 
quelli per i quali tutti gli intorni contengono punti che appartengono a 
\(S\) e punti che non appartengono a \(S\).
\end{newdef}
}{
\begin{newdef}{}{}
Chiamiamo punti \emph{di frontiera} di \(S\) i punti la cui monade contiene 
punti che appartengono a \(S\) e punti che non appartengono a \(S\).
\end{newdef}
}

\begin{esempio}
Dato l'insieme \(A = \intervaa{4}{8}\) 
stabilisci le caratteristiche dei seguenti punti.
\begin{enumerate} [label=\alph*)]
\item \(x_P= 2\): \quad  \(x_P \notin A\), 
\(x_P\) non è di accumulazione per \(A\) \(x_P\) è esterno ad \(A\).
\item \(x_P= 4\): \quad  \(x_P \notin A\), 
\(x_P\) è di accumulazione per \(A\) \(x_P\) è di frontiera per \(A\).
\item \(x_P= 5\): \quad  \(x_P \in A\), 
\(x_P\) è di accumulazione per \(A\) \(x_P\) è interno ad \(A\).
\item \(x_P= 8\): \quad  \(x_P \notin A\), 
\(x_P\) è di accumulazione per \(A\) \(x_P\) è di frontiera per \(A\).
\item \(x_P= 9\): \quad  \(x_P \notin A\), 
\(x_P\) non è di accumulazione per \(A\) \(x_P\) è esterno ad \(A\).
\end{enumerate}
\end{esempio}

\begin{esempio}
Studio dei punti di accumulazione di \(A=\graffa{5;~6;~7;~8}\).

L'insieme, essendo finito, non ha punti di accumulazione; 
non possiamo parlare di punti interni o di frontiera.
\end{esempio}

\begin{esempio}
Studia i seguenti insiemi numerici.
\begin{enumerate} [label=\alph*)]
\item \(A = \intervaa{1}{6} \scup \intervac{6}{8}\)
\begin{itemize} [noitemsep]
\item l'estremo inferiore è 1, l'estremo superiore 8;
\item L'insieme dei minoranti è \(\intervac{-\infty}{1}\);
\item L'insieme dei maggioranti è \(\intervca{8}{+\infty}\) ;
\item non ha minimo, il massimo è 8;
\item 1 e 6 sono di accumulazione per \(A\) ma non appartengono a \(A\);
\item 8 è di accumulazione per \(A\) e appartiene a \(A\);
\item tutti i numeri dell'insieme \(\intervcc{1}{8}\)  sono di 
accumulazione per \(A\);
\item 1, 6 e 8 sono punti di frontiera.
\end{itemize}
\item \(\Q\) (insieme dei numeri razionali)
\begin{itemize} [noitemsep]
\item ogni numero razionale è di accumulazione per \(\Q\), 
ma anche ogni numero irrazionale è di accumulazione per \(\Q\);
\item tutti i punti della retta sono di accumulazione per \(\Q\); 
\item è illimitato sia inferiormente sia superiormente;
\item ogni suo punto è interno all'insieme.
\end{itemize}
\item \(B=\graffa{x = \dfrac{1}{n}, \quad \forall n \in \Nz}\)
\begin{itemize} [noitemsep]
\item tutti i punti dell'insieme sono isolati;
\item all'aumentare di \(n\) i numeri si addensano attorno a zero: 0 è un 
punto di accumulazione per l'insieme \(B\); 
\item è limitato sia inferiormente sia superiormente: 
\(\inf{B} = 0\), \(\sup{B} = 1\);
\item non ha minimo perché \(0 \notin B\), il massimo è 1 perché \(1 \in B\).
\end{itemize}
\end{enumerate}
\end{esempio}









