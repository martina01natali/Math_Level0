% (c) 2017 Daniele Zambelli - daniele.zambelli@gmail.com
% 
% Tutti i grafici per il capitolo relativo alla topologia della retta
% 
% 

\newcommand{\rettaconversoa}{% Retta dotata di un verso
  \disegno{
    \assex{-6}{+6}{0}
    \fill [fill=blue] (-3.5, 0) circle(2pt) 
      node[above] {\(a\)} node[below] {\(A\)}
      (0, 0) node[above] {\(<\)} node[below] {viene prima di}
      (+3.5, 0) circle(2pt) 
      node[above] {\(b\)} node[below] {\(B\)};
  }
}

\newcommand{\rettaconversob}{% Retta dotata di un verso da destra a sinistra
  \disegno{
    \assex{+6}{-6}{0}
    \fill [fill=blue] (-3.5, 0) circle(2pt) 
      node[above] {\(a\)} node[below] {\(A\)}
      (0, 0) node[above] {\(>\)} node[below] {viene dopo di}
      (+3.5, 0) circle(2pt) 
      node[above] {\(b\)} node[below] {\(B\)};
  }
}

\newcommand{\asselineare}{% Asse con scala lineare
  \disegno[7]{
    \assecontrattini{-4}{+4}{0}{x}
    \foreach \px in {-3, -2, -1, 0, +1, +2, +3}{
      \draw (\px, 0) node[below] {\footnotesize\(\px\)};
    }
  }
}

\newcommand{\assequadratico}{% Asse con scala quadratica
  \disegno[7]{
    \assecontrattini{-4}{+4}{0}{x}
    \foreach \px/\lx in {-3/-9, -2/-4, -1, 0, +1, +2/+4, +3/+9}{
      \draw (\px, 0) node[below] {\footnotesize\(\lx\)};
    }
  }
}

\newcommand{\asselogaritmico}{% Asse con scala logaritmica
  \disegno[7]{
    \assecontrattini{-4}{+4}{0}{x}
    \foreach \px in {-3, -2, -1, 0, +1, +2, +3}{
      \draw (\px, 0) node[below] {\footnotesize\(10^{\px}\)};
    }
  }
}

\newcommand{\asseconpuntireali}{% Retta con alcuni punti realievidenziati
  \disegno{
    \assecontrattini{-7}{+7}{0}{x}
    \foreach \px/\lx in {-6/-6, -3/-3, -1.5/{-1,5}, 0/0, 1/+1, 
                         2.236/+\sqrt{5}, 3.666/\frac{11}{3},
                         5.5/\frac{11}{2}}{
      \fill [fill=blue] (\px, 0) circle(2pt) node[below] {\footnotesize\(\lx\)};
%       \draw 
    }
  }
}

\newcommand{\asseconpuntiiperreali}{% Retta con alcuni punti iperreali
  \disegno{
    \assecontrattini{-7}{+7}{0}{x}
    \microscopio{(-5, 0)}{2}{120}{-50}{2}{(7.6, 10)}{\(\times \infty\)}
    \grandangolo{(0, 0)}{2}{120}{-50}{2}{(7.6, 10)}{\(\times \infty\)}
    \draw (+3, 1) pic [rotate=0, scale=.5] {telescopio=\(\times \infty\)};
    \microscopio{(+3, 1)}{2}{120}{-50}{2}{(7.6, 10)}{\(\times \infty\)}
  }
}

\begin{comment}

\end{comment}
