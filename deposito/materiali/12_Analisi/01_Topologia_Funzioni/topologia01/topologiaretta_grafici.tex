% (c) 2017 Daniele Zambelli - daniele.zambelli@gmail.com
% 
% Tutti i grafici per il capitolo relativo alla topologia della retta
% 
% 

\newcommand{\rettaconversoa}{% Retta dotata di un verso
  \disegno{
    \assex{-6}{+6}{0}
    \fill [fill=blue] (-3.5, 0) circle(2pt) 
      node[above] {\(a\)} node[below] {\(A\)}
      (0, 0) node[above] {\(<\)} node[below] {viene prima di}
      (+3.5, 0) circle(2pt) 
      node[above] {\(b\)} node[below] {\(B\)};
  }
}

\newcommand{\rettaconversob}{% Retta dotata di un verso da destra a sinistra
  \disegno{
    \assex{+6}{-6}{0}
    \fill [fill=blue] (-3.5, 0) circle(2pt) 
      node[above] {\(a\)} node[below] {\(A\)}
      (0, 0) node[above] {\(>\)} node[below] {viene dopo di}
      (+3.5, 0) circle(2pt) 
      node[above] {\(b\)} node[below] {\(B\)};
  }
}

\newcommand{\asselineare}{% Asse con scala lineare
  \disegno[7]{
    \assecontrattini{-4}{+4}{0}{x}
    \foreach \px in {-3, -2, -1, 0, +1, +2, +3}{
      \draw (\px, 0) node[below] {\footnotesize\(\px\)};
    }
  }
}

\newcommand{\assequadratico}{% Asse con scala quadratica
  \disegno[7]{
    \assecontrattini{-4}{+4}{0}{x}
    \foreach \px/\lx in {-3/-9, -2/-4, -1, 0, +1, +2/+4, +3/+9}{
      \draw (\px, 0) node[below] {\footnotesize\(\lx\)};
    }
  }
}

\newcommand{\asselogaritmico}{% Asse con scala logaritmica
  \disegno[7]{
    \assecontrattini{-4}{+4}{0}{x}
    \foreach \px in {-3, -2, -1, 0, +1, +2, +3}{
      \draw (\px, 0) node[below] {\footnotesize\(10^{\px}\)};
    }
  }
}

\newcommand{\asseconpuntireali}{% Retta con alcuni punti realievidenziati
  \disegno{
    \assecontrattini{-7}{+7}{0}{x}
    \foreach \px/\lx in {-6/-6, -3/-3, -1.5/{-1,5}, 0/0, 1/+1, 
                         2.236/+\sqrt{5}, 3.666/\frac{11}{3},
                         5.5/\frac{11}{2}}{
      \fill [fill=blue] (\px, 0) circle(2pt) node[below] {\footnotesize\(\lx\)};
%       \draw 
    }
  }
}

\newcommand{\asseconpuntiiperreali}{% Retta con alcuni punti iperreali
  \disegno{
    \assecontrattini{-7}{+7}{0}{x}
%     \draw (0, 0) node [below] {\footnotesize \(0\)}
%           (1, 0) node [below] {\footnotesize \(1\)};
    \foreach \px in {-7, -6, -5, -4, -3, -2, -1, 0, 
                     +1, +2, +3, +4, +5, +6, +7}{
      \draw (\px, 0) node[below] {\footnotesize\(\px\)};}
    \microscopio{(-4, 0)}{2}{110}{-70}{2}{(-3, 5)}
      {\footnotesize\(\times \frac{1}{\epsilon}\)}
    \segmentocontrattini{-7.1}{-3.75}{3.8}
    \foreach \px/\lx in {-7.1/{-4-2\epsilon}, -6.1/{-4-\epsilon}, 
                        -5.1/{-4}, -4.1/{-4+\epsilon}}{
      \draw (\px, 3.8) node [right, rotate=-90] {\footnotesize\(\lx\)};}
    \grandangolo{(0, 0)}{2}{90}{-90}{2}{(1.8, 5.8)}
      {\footnotesize\(\times \frac{1}{A}\)}
    \segmentocontrattini{-1.7}{+1.7}{4.1}
    \foreach \px/\lx in {-1.7/{-A}, -.7/{0}, 
                        +.3/{A}, +1.3/{2A}}{
      \draw (\px, 4.1) node [below] {\footnotesize\(\lx\)};}
    \draw (+3, 1) pic [rotate=0, scale=.5] 
      {telescopio=\footnotesize\(\times M\)};
    \microscopio{(+3, 1)}{2}{40}{-130}{2}{(6, 5)}{}
    \segmentocontrattini{+4.1}{7.5}{3.9}
    \foreach \px/\lx in {4.1/{M-1}, 5.1/{M}, 
                        6.1/{M+1}, 7.1/{M+2}}{
      \draw (\px, 3.9) node [right, rotate=-90] {\footnotesize\(\lx\)};}
  }
}

\newcommand{\assecondistanze}{% Retta con alcuni punti per le distanze
  \disegno{
    \assecontrattini{-7}{+7}{0}{x}
    \foreach \px/\nx/\lx in {-6/-6/B, -1.5/-\frac{3}{2}/A, 
                             3.666/+\frac{11}{3}/C, 5.5/+\frac{11}{2}/D}{
      \fill [fill=blue] (\px, 0) circle(2pt) 
        node[below] {\footnotesize\(\nx\)}
        node[above] {\footnotesize\(\lx\)};
    }
  }
}

\newcommand{\esempiolimsup}{% Insieme limitato superiormente
  \disegno{
    \assecontrattini{-5}{+9.3}{0}{x}
    \foreach \px in {-4, ..., 0, +1, +2, +3, +4, +5, +6, +7, +8}{
      \draw (\px, 0) node[below] {\footnotesize\(\px\)};}
    \foreach \px in {-5, +9}{
      \draw (\px, 0) node[below=.5em] {\footnotesize \dots};}
    \foreach \px/\nx/\lx in {-4, -2, ..., 6}{
      \fill [fill=blue] (\px, 0) circle(2pt);}
    \draw (7.5, 2) node {maggioranti};
    \foreach \px/\nx/\lx in {6, 6.4, ..., 9}{
      \draw [-{Latex[length=2mm, width=1mm]}, black] (\px, 1.3) -- (\px, 0);}
  }
}

\newcommand{\intor}[5]{% Intorno del punto P
% Esempio di chiamata:
% \disegno{\intor{-3.2}{+4}{I(x_P)}{\delta_1}{\delta_2}}
  \def \xa{#1}
  \def \xb{#2}
  \def \nomei{#3}
  \def \da{#4}
  \def \db{#5}
  \def \ya{+0.5}
  \def \yb{-1.3}
  \assex{-6.5}{+6.5}{0}
  \evidenziadafino{(\xa, 0)}{(\xb, 0)}{white}{white}
  \filldraw [blue] (0, 0) circle(3pt);
  \draw [latex - latex] (\xa, \yb) -- (\xb, \yb) 
    node [midway, below] {\(\nomei\)};
  \draw (\xa, \yb) node [above] {\(x_P -\da\)}
        (0, \yb) node [above] {\(x_P\)}
        (\xb, \yb) node [above] {\(x_P +\db\)};
  \draw [|-|] (\xa, \ya) -- (0, \ya) 
    node [midway, above] {\(\da\)};
  \draw (0, \ya) node [above] {\(P\)};
  \draw [|-|] (0, \ya) -- (\xb, \ya) 
    node [midway, above] {\(\db\)};
}

\newcommand{\intorno}{% Intorno del punto P
  \disegno{\intor{-3.2}{+4.5}{I(x_P)}{\delta_1}{\delta_2}}
}

\newcommand{\intorni}[2]{% Intorni 
  \def \x{#1}
  \def \dati{#2}
  \def \y{0}
  \def \xa{0}
  \def \xb{+3}
  \def \ya{0}
  \disegno[8]{
    \foreach \xa/\xb/\y/\labi  in \dati {
      \draw (-2, \y) node [left] {\(\labi\)};
      \assex{-1.5}{+5}{\y}
      \inti{\y}{\xa}{\xb}{\xa}{\xb}{white}{white}
      \filldraw [blue] (\x, \y) circle(3pt) node [above] {\(\x\)};}
  }
}

\newcommand{\intornocircolare}{% Intorno circolare del punto P
  \disegno{\intor{-4}{+4}{I_C(x_P)}{\delta}{\delta}}
}

\newcommand{\semintorno}[3]{% Semi intorno del punto P
  \def \xa{#1}
  \def \nomei{#2}
  \def \nomed{#3}
  \def \yb{-1.3}
  \assex{-7}{+7}{0}
  \evidenziafino{(0, 0)}{(\xa, 0)}{white}
  \filldraw [blue] (0, 0) circle(3pt);
  \draw [latex - latex] (0, \yb) -- (\xa, \yb) 
    node [midway, below] {\(\nomei\)};
  \draw (\xa, \yb) node [above] {\(x_P \nomed\)}
        (0, \yb) node [above] {\(x_P\)};
  \draw (0, 0) node [above] {\(P\)};
}

% \newcommand{\intornosinistro}{% Intorno sinistro del punto P
%   \disegno{\semintorno{-4}{I_S(x_P)}{-\delta}}
% }
% 
% \newcommand{\intornodestro}{% Intorno destro del punto P
%   \disegno{\semintorno{+4}{I_D(x_P)}{+\delta}}
% }

\newcommand{\intornomenoinf}{% Intorno di -\infty
  \disegno{
    \raylconasse{0}{10}{4}{a}{white}{x}
    \draw (-2, 0) node {\(-\infty \cdots\)} 
          (+12, 0) node {\(\cdots +\infty\)};
  }
}

\newcommand{\intornopiuinf}{% Intorno di -\infty
  \disegno{
    \rayrconasse{0}{10}{6}{a}{white}{x}
    \draw (-2, 0) node {\(-\infty \cdots\)} 
          (+12, 0) node {\(\cdots +\infty\)};
  }
}


