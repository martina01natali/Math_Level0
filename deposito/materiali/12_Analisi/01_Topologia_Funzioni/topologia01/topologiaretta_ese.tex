% (c) 2017 Leonardo Aldegheri
% (c) 2017 Carlotta Gualtieri

\begin{comment}
\section{Esercizi}

\subsection{Esercizi dei singoli paragrafi}

\subsubsection*{\numnameref{sec:01_}}

\begin{esercizio}
\label{ese:D.19}
testo esercizio
\end{esercizio}

\begin{esercizio}\label{ese:03.1}
Consegna:
 \begin{enumeratea}
  \item  
 \end{enumeratea}
\end{esercizio}

\subsection{Esercizi riepilogativi}

\begin{esercizio}
\label{ese:D.19}
testo esercizio
\end{esercizio}

\begin{esercizio}\label{ese:03.1}
Consegna:
 \begin{enumeratea}
  \item  
 \end{enumeratea}
\end{esercizio}
\end{comment}

\section{Esercizi}
%label{}
  \subsection{Esercizi dei singoli paragrafi}
  %label{}
  \subsubsection*{2.1 La topologia della retta}
  %label{}
  \begin{itemize}
  \item[2.1)] Rispondi alle seguenti domande, 
nella maniera più dettagliata e precisa possibile.
  \begin{itemize}
  \item[a)] Cosa si intende per 
topologia della retta?
  \item [b)] Qual è la 
principale identificazione che si fa nello studio della topologia della retta?
  \item[c)] Descrivi cosa si 
intende per distanza tra punti.
  \item[d)] Enuncia almeno tre 
proprietà della distanza, spiegandone il significato.
  \end{itemize}   
  \end{itemize}
  \subsubsection*{2.2 Gli intervalli}
  %label{}
  \begin{itemize}
  \item[2.2)] Data la rappresentazione algebrica 
dei seguenti intervalli esprimili sia graficamente che mediante le parentesi:
$$a)\, x\geq2\,\,\, \,\,\, \,\,\, \,\,\, b)\, 3<x<5  \,\,\,\,\,\, \,\,\,   
c)\, 4<x<7,\, x\neq5  \,\,\,  \,\,\, \,\,\,  d)\, x<-3 $$
$$ e) \, \frac{1}{2}<x\leq2 \,\,\, \,\,\,  \,\,\,  f)\, \frac{3}{4}\leq 
x\leq3\,\,\, \,\,\, \,\,\,  g)\, x\leq2\land x\geq4$$

  \item[2.3)] Risolvi le seguenti disequazioni, 
trovandone l'intervallo di soluzione
  \begin{itemize}
  \item[a)] $x^2+2>0$  
   \hfill  [ $]-\infty,+\infty[$ ]
  \item[b)] $3x^2-8x\geq0$   
   \hfill   [ $]-\infty,0]\,\cup\,[\frac{8}{3},+\infty[$ 
 ]
  \item[c)] $6x^2-5x+1>0$  
  \hfill   [ $]-\infty, 
\frac{1}{3}[\,\cup\,]\frac{1}{2}, +\infty[$  ]
  \item[d)] $-3x^2-2>0$  
  \hfill   [$\emptyset$]
  \item[e)] $x(x-2)(2x+3)<0$   
   \hfill   [ $]-\infty, -\frac{3}{2}[\,\cup\, ]0, 2[$ ]
  \item[f)] 
$\frac{2x+5}{x-3}\geq0$   \hfill   [ $ 
]-\infty,-\frac{5}{2}]\,\cup\,]3,+\infty[$ ]
  \item[g)] 
$\frac{(x+2)^2}{3-x}>0$  \hfill   [ $ ]-\infty, 
-2[\,\cup\,]-2, 3[$ ]
  \end{itemize}
  \end{itemize}   
  \subsubsection*{2.3 Gli intorni}
  %label{}
  \begin{itemize}
  \item[2.4)] Ricordando le diverse tipologie di 
intorno classifica gli intorni seguenti rispetto al punto indicato, 
determinando ampiezze e raggi.
  \begin{itemize}
  \item[a)] $]3, 7[$ rispetto a 
$x_0=5$   \hfill   [circolare di raggio 2]
  \item[b)] $]2, 5[$ rispetto a 
$x_0=2$   \hfill  [destro di ampiezza 3]
  \item[c)] $]3, 4[$ rispetto a 
$x_0=4$  \hfill  [sinistro di ampiezza 1]
  \item[d)] $]1/4, 3/4[$ 
rispetto a $x_0=\frac{1}{2}$   \hfill  [circolare di ampiezza $\frac{1}{4}$]
  \item[e)] $]3, 7/2[$ rispetto 
a $x_0=3$   \hfill  [destro di ampiezza $\frac{1}{2}$] 
  \item[f)] $]\frac{2}{5}, 2[$ 
rispetto a $x_0=2$  \hfill   [sinistro di ampiezza $\frac{8}{5}$]
  \item[g)] $]-\frac{8}{5}, 
-\frac{4}{5}[$ rispetto a $x_0= -\frac{6}{5}$   \hfill [circolare di ampiezza 
$\frac{2}{5}$]
  \end{itemize}
  \end{itemize}
  \subsubsection*{2.4 Insiemi limitati e illimitati}
  %label{}
  \begin{itemize}
  \item[2.5)]  Stabilisci se i seguenti insiemi 
sono superiormente/inferiormente limitati/illimitati
$$a)\, [3, 5]  \,\,\, \,\,\,   b)\, ]3, 5[  \,\,\, \,\,\,   c)\, ]-\infty, 
\frac{1}{2}[   \,\,\, \,\,\,  $$ 
  $$d)\,]-\infty, 3[\,\cup\,\{6\}  \,\,\,\,\,\,  e)\,  ]-\infty, 
-2]\,\cup\,[2, +\infty[ \,\,\,\,\,\,  f)\, [\sqrt{2}, +\infty[$$
  \end{itemize}
  \subsubsection*{2.5 Massimi minimi ed estremi}
  %label{}
  \begin{itemize}
  \item[2.6)] Individua, se esistono, massimi, 
minimi, estremi superiori ed inferiori dei seguenti intervalli
  \begin{itemize}
  \item[a)]  $A=[1.5] $  
\hfill   [$\min{A}=1$, $\inf{A}=1$, $\max{A}=5$, $\sup{A}=5$]
  \item[b)] $A=]3,6]$   \hfill   
  [$\min{A}$ non esiste, $\inf{A}=3$, $\max{A}=6$, $\sup{A}=6$]
  \item[c)] $A=]-2,+\infty[$  
\hfill   [$\min{A}$ non esiste, $\inf{A}=-2$, $\max$ e $\sup$ non esistono]
  \item[d)] 
$A=[2,5[\,\cup\,]6,8[$  \hfill [$\min{A}=2$, $\inf{A}=2$, $\max{A}$ non 
esiste, $\sup{A}=8$]
  \item[e)] $A=]-\infty, 4]$   
\hfill   [$\min$ e $\inf$ non esistono, $\max{A}=4$, $\sup{A}=4$]
  \end{itemize}
  \end{itemize}
  \subsubsection*{2.6 I punti di accumulazione}
  %label{}
  \begin{itemize}
  \item[2.7)] Stabilisci se i punti indicati 
sono di accumulazione per gli insiemi assegnati
  \begin{itemize}
  \item[a)]  $A=[3, 7]$ \hfill  
$x_0=2$, $x_1=3$, $x_2=5$
  \item[b)] $A=]4, 9[$  \hfill  
$x_0=4$, $x_1=5$, $x_2=11$
  \item[c)] $A=\{2,3,4,5\}$  
\hfill  $x_0=2$, $x_1=3$, $x_2=5$
  \item[d)] $\mathbb{R}$ \hfill  
$x_0=-2$, $x_1=0$, $x_2=\sqrt{3}$
  \item[e)] $\mathbb{N}$ \hfill  
$x_0=1$, $x_1=3$, $x_2=15$
  \item[f)] 
$A=\{x\in\mathbb{R}\vert x=\frac{3n}{n+1},\,n\in\mathbb{N}\}$ \hfill  
$x_0=3$, $x_1=4$ 
  \end{itemize}
  \end{itemize}
  










