% (c) 2015 Daniele Zambelli daniele.zambelli@gmail.com

\input{\folder teoremifc_grafici.tex}

\chapter{Teoremi sulle funzioni continue o derivabili}

% \emph{
% In questo capitolo vedremo un nuovo insieme numerico: 
% gli \emph{Iperinteri}.
% Questi numeri ci permetteranno di dimostrare con una certa facilità 
% alcuni teoremi. % sulle funzioni continue e derivabili.
% 
% Dopo aver ripreso le definizioni di continuità di una funzione, 
% dimostreremo alcuni teoremi fondamentali dell'analisi relativi 
% alle funzioni continue e alle funzioni derivabili.}

\section{Numeri iperinteri}
\label{subsec:cont_iperinteri}

Per affrontare alcuni dei prossimi argomenti, abbiamo bisogno di un altro 
strumento matematico: l'insieme dei numeri \emph{Iperinteri}.
Non è difficile visualizzare sulla retta dei numeri questo insieme.

Per dare una definizione rigorosa dei numeri Iperinteri abbiamo bisogno di 
usare la funzione \emph{parte intera} di un numero: che si indica con il 
simbolo:
\(\quadra{x}\). Qualche esempio:
\begin{center}
\begin{tabular}{ccccccccccccccc}
\(x\) & 
-3&-2,4&-2,01&-1,2&-1,03&-0,3&0,2&1&1,6&1,99&2&2.03&2.9&3,42\\
\hline
\(y=\quadra{x}\) & 
-3&-3  &-3   &-2  &-2   &-1  &0  &1&1  &1   &2&2   &2  &3
\end{tabular}
\end{center}

\begin{minipage}{.59\textwidth}
La parte intera di un numero \(x\) è il più grande numero intero \(n\) 
minore o uguale a \(x\). 

Attenzione che mentre per i numeri positivi il 
concetto è abbastanza naturale, per quelli negativi il concetto non è 
altrettanto immediato, vedi la tabella precedente:
\begin{itemize} [nosep]
\item per i numeri positivi, la parte intera si ottiene togliendo la parte 
decimale;
\item per i numeri negativi la parte intera è il numero stesso se la parte 
decimale è nulla, altrimenti è il numero meno la parte decimale e meno uno
(es. \(\quadra{-4,12}=-5\)).
\end{itemize}
\end{minipage}
\hfill
\begin{minipage}{.39\textwidth}
\begin{center}
\parteintera
\end{center}
\end{minipage}

\vspace{.5em}
Applicando la funzione parte intera ai numeri Iperreali otteniamo gli 
Iperinteri.

\begin{newdef}{}{iperinteri}
I numeri \textbf{Iperinteri} sono quei numeri Iperreali per cui vale 
l'uguaglianza: \qquad
\(x=\quadra{x}\)
\end{newdef}

Possiamo fare alcune osservazioni sugli Iperinteri:

\begin{enumerate} [noitemsep]
\item 
La somma algebrica di due numeri Iperinteri è un numero Iperintero.
\item 
Ogni numero Iperreale si trova tra due numeri Iperinteri:
\[\forall x \in \IR \quad \quadra{x} \leqslant x < \quadra{x}+1\]
\end{enumerate}

Possiamo usare gli Interi per dividere un intervallo Reale \([a;~b]\)
in \(n\) parti uguali. Ciascuna di queste \(n\) parti 
uguali è lunga \(l=\dfrac{b-a}{n}\).

Gli \(n\) sotto intervalli che si ottengono sono:
\[\left[a;~a+l \right[,~[a+l;~a+2l[,~\dots,~
[a+jl;~a+(j+1)l[,~\dots,~
[a+\tonda{n-1}l;~a+nl=b]\]

Gli estremi di questi intervalli sono chiamati \emph{punti di partizione} 
dell'intervallo:
\[a;~a+l;~a+2l;~a+3l;~\dots;~a+\tonda{n-1}l;~a+nl=b\]

Possiamo ora estendere questo procedimento ai numeri Iperreali.
Scegliamo un numero infinito iperintero \(H\) e dividiamo in parti 
uguali l'intervallo di numeri Iperreali \(\intervcc{a}{b}\). Ogni 
sotto intervallo avrà la stessa lunghezza infinitesima 
\(\delta=\dfrac{b-a}{H}\).

Gli \(H\) sotto intervalli che si ottengono sono:
\[\left[a;~a+\delta \right[,~[a+\delta;~a+2\delta[,~\dots,~
[a+j\delta;~a+(j+1)\delta[,~\dots,~
[a+\tonda{H-1}\delta;~a+H\delta=b]\]

e i punti di partizione sono:
\[a;~a+\delta;~a+2\delta;~\dots;~a+j\delta;~\dots;~a+H\delta=b\]
cioè i punti \(a+j\delta\) con \(j\) che varia da~0 a~\(H\).

Ogni numero iperreale \(x\) appartenente all'intervallo \(\intervca{a}{b}\)
apparterrà a uno dei sotto intervalli infinitesimi:
\[x \in \intervca{a+j\delta}{a+\tonda{j+1}\delta} \quad \Rightarrow \quad 
  a+j\delta \leqslant x < a+\tonda{j+1}\delta\]

Possiamo ora affrontare alcuni teoremi riguardanti le funzioni continue.

\section{Teoremi sulle funzioni continue}
\label{sec:cont_continuita}

\subsection{Definizione di continuità}
\label{sec:cont_definizione}

Richiamiamo le definizioni di continuità presentate nel capitolo sulla 
continuità e i limiti.

\begin{newdef}{}{}
Diremo che una funzione è \emph{continua in un punto \(c\)} non isolato, 
se è definita in \(c\) e, 
quando \(x\) è infinitamente vicino a \(c\), 
allora \(f(x)\) è infinitamente vicino a \(f(c)\), 

\vspace{.5em}
e si scrive \(f\) è continua nel punto \(c\) se: \hspace{5mm}
\(\forall x \approx c \quad f(x) \approx f(c)\)

% \vspace{.5em}
oppure: \hspace{51mm}
\(\forall \epsilon \approx 0 \quad f(c + \epsilon) \approx f(c)\)

% \vspace{.5em}
oppure: \hspace{51mm}
\(\forall \epsilon \approx 0 \quad f(c + \epsilon) - f(c) \approx 0\)

% \vspace{.5em}
o ancora: \hspace{49mm}
\(\forall x \stext{se } \pst{x} = c \sstext{allora} \pst{f(x)} = f(c)\)
\end{newdef}

\begin{newdef}{}{}
Una \emph{funzione è continua} se è continua in ogni punto del suo 
insieme di definizione.
\end{newdef}

\begin{newdef}{}{}
Una funzione è \emph{continua in un intervallo} se:
\begin{itemize} [noitemsep]
\item 
la funzione, ristretta a quell'intervallo, è continua e
\item 
è definita in tutti i punti dell'intervallo.
\end{itemize}
\end{newdef}

% \begin{newdef}{}{}
% Diremo che una funzione è \textbf{continua} in un punto \(c\), 
% se è definita in \(c\) e, 
% quando \(x\) è infinitamente vicino a \(c\), 
% allora \(f(x)\) è infinitamente vicino a \(f(c)\). E si scrive::
% 
% \[f \text{ è continua in } c \Leftrightarrow 
% \forall x \tonda{\tonda{x \approx c} \Rightarrow 
% \tonda{f(x) \approx f(c)}}\]
% \end{newdef}
% 
% Data una funzione \(y=f(x)\) definita nel punto \(c\), le seguenti 
% affermazioni sono equivalenti:
% 
% \begin{enumerate}[noitemsep]
%  \item \(f\) è continua in \(c\);
%  \item se \(x \approx c\) allora \(f(x) \approx f(c)\) 
%        (\(\approx\): infinitamente vicino);
%  \item se \(\st(x) = c\) allora \(\st(f(x)) = f(c)\);
%  \item \(\lim_{x \to c} f(x) = f(c)\);
%  \item se \(x\) si allontana da \(c\) di un infinitesimo allora 
%    \(f(x)\) si allontana da \(f(c)\) di un infinitesimo.
%  \item se \(\Delta x\) è infinitesimo allora il corrispondente \(\Delta y\) 
%    è infinitesimo.
% \end{enumerate}

\subsection{Alcuni teoremi delle funzioni continue}
\label{subsec:cont_iperinteri}

Il prossimo teorema riguarda gli zeri di una funzione. Con zero di una 
funzione 
si intende un valore della \(x\) che rende la funzione uguale a zero:

\begin{newdef}{}{}
 \(c\) è uno \textbf{zero} della funzione \(f(x)\) se \(f(c)=0\)
\end{newdef}

\begin{esempio}
Verifica che gli zeri della funzione \(f(x) = x^2 -5x -24\) \quad sono: 
\(-3 \stext{e} 8\).\\ [.5em]
\(f(-3) = (-3)^2 -5 (-3) -24 = 9 +15 -24 = 0\) \quad e \quad 
\(f(+8) = 8^2 -5 \cdot 8 -24 = 64 -40 -24 = 0\)
\end{esempio}


\begin{newtheo}{Teorema degli zeri}{}
Se una funzione \(f(x)\) è continua nell'intervallo chiuso
\(\intervcc{a}{b}\) e agli estremi dell'intervallo assume valori di segno 
opposto, allora ha (almeno) uno zero nell'intervallo 
aperto~\(\intervaa{a}{b}\).
\end{newtheo}

\noindent Ipotesi: \nopagebreak
\begin{enumerate}[nosep]
 \item \(f\) è una funzione continua nell'intervallo chiuso 
\(\intervcc{a}{b}\);
 \item \(f(a) \cdot f(b) < 0\) 
 (equivale a dire che \(f(a)\) e \(f(b)\) hanno valori discordi).
\end{enumerate}

\noindent Tesi: 

\(\exists~c \in \intervaa{a}{b} \stext{ tale che } f(c)=0\).

\begin{proof}
Supponiamo che \(f(a)<0\) e \(f(b)>0\), posto \(h > 1\) un numero naturale, 
dividiamo l'intervallo \(\intervcc{a}{b}\) in~\(h\) parti uguali 
di lunghezza \(l = \dfrac{b -a}{h}\).
\[a;~a+l;~a+2l;~\dots;~a+jl;~\dots;~a+hl=b\]
Esisterà un valore dell'indice, \(\overline{j}\) 
(``j segnato''), per il quale: \qquad
\(f(a+\overline{j}l) \leqslant 0 < f(a+(\overline{j}+1)l)\)\\
Ora suddividiamo l'intervallo in un numero iperintero infinito \(H\) 
otterremo infiniti intervalli di lunghezza \(\delta = \dfrac{b -a}{H}\):
\qquad \(a;~a+\delta;~a+2\delta;~\dots;~a+j\delta;~\dots;~a+H\delta=b\)\\
Per il principio di transfert anche in questo caso esisterà un valore 
iperreale \(k\) tale che:

\noindent\begin{minipage}{.58\textwidth}
\[f(a+k\delta) \leqslant 0 < f(a+(k+1)\delta)\]
I due valori: \(a+k\delta \stext{ e }a+(k+1)\delta\) sono infinitamente 
vicini e quindi hanno la stessa parte standard.
Dato che \(f\) è continua:
\[a+k\delta \approx a+(k+1)\delta \quad \Rightarrow \quad 
f(a+k\delta) \approx f(a+(k+1)\delta)\] 
Ma l'unico numero standard che è infinitamente vicino ad un numero
minore di zero e anche ad un numero maggiore di zero è zero. 
Quindi chiamando: \(c=\pst{a+k\delta}=\pst{a+(k+1)\delta}\): \quad 
\(f(c)=0\)
% \[f(c)=f(\pst{a+k\delta})=\pst{f(a+k\delta)}=0\]
\end{minipage}
\hfill
\begin{minipage}{.38\textwidth}
\begin{center} \tzeri \end{center}
\end{minipage}

In modo analogo si dimostra il caso in cui \(f(a)>0>f(b)\).
\end{proof}
Da notare che il teorema dimostra che in \(\intervcc{a}{b}\) c'è almeno uno 
zero della funzione, ma non dice nulla sul numero degli zeri.

La dimostrazione di questo teorema permette di dimostrarne facilmente degli 
altri. 

\begin{newcor}{Teorema dei valori intermedi}{}
Se una funzione è continua in un
intervallo \(\intervcc{a}{b}\) allora tra \(a\) e \(b\) assume tutti i 
valori compresi tra \(f(a)\) e \(f(b)\).
\end{newcor}

\begin{proof}
Scelto un qualsiasi valore \(h\) compreso tra \(f(a) \stext{ e } f(b)\), \\
costruiamo una nuova funzione: \quad \(g(x) = f(x)-h\).\\ 
\(g\) soddisfa tutte le ipotesi del teorema precedente per cui: \quad 
\(\exists~c \in \intervaa{a}{b} \text{ tale che } g(c)=0\)\\
e sostituendo la funzione \(g\) otteniamo: \quad 
\(g(c)=0 \Rightarrow f(c) -h=0 \sRarrow f(c)=h\).
\end{proof}

\begin{newcor}{}{}
Se una funzione è continua in un
intervallo \(\intervcc{a}{b}\) e 
\(f(x)\neq 0 \quad \forall x \in \intervcc{a}{b}\) 
allora:
\begin{enumerate}[nosep]
 \item se \(f(c)<0\) \quad per un qualche \quad \(c \in \intervcc{a}{b}\)
 \quad allora \quad \(f(x)<0 \quad \forall x \in \intervcc{a}{b}\);
 \item se \(f(c)>0\) \quad per un qualche \quad \(c \in \intervcc{a}{b}\)
 \quad allora \quad \(f(x)>0 \quad \forall x \in \intervcc{a}{b}\).
\end{enumerate}
\end{newcor}

\begin{proof}
Consideriamo il primo caso, \(f(c) < 0\):\\ 
se esistesse un valore \(d \in \intervcc{a}{b}\) per cui
\(f(d) > 0\) allora nell'intervallo \(\intervcc{c}{d}\) 
esiste un punto \(z \sRarrow f(z) = 0\)
essendo \(\intervcc{c}{d} \subseteq \intervcc{a}{b}\) 
ciò contraddirebbe l'ipotesi.
In modo analogo si dimostra il secondo caso.
\end{proof}

\subsection{Massimi e minimi}
\label{subsec:cont_massimiminimi}

Data una funzione definita in un certo intervallo, può darsi che questa 
funzione abbia un massimo o un minimo in questo intervallo.

\begin{newdef}{}{}
 Chiamiamo \textbf{massimo} di una funzione in un intervallo \(I\) un punto 
\(\punto{c}{f(c)}\) tale che, per ogni \(x\) appartenente all'intervallo, 
\(f(c)\) sia maggiore o uguale a \(f(x)\): \qquad
\(max(f,~I) = \punto{c}{f(c)} \sLRarrow \forall x \in I \quad
f(c) \geqslant f(x)\)
\end{newdef}

\begin{minipage}{.25\textwidth}
La definizione di \textbf{minimo} in un intervallo si ottiene facilmente 
modificando quella di massimo (scrivila tu e poi confrontala con quella 
scritta dagli altri tuoi compagni).

In un intervallo, una funzione potrebbe avere \emph{più} minimi o massimi 
oppure potrebbe \emph{non} avere minimi o massimi, vedi i grafici a fianco, 
considerando come intervallo tutto \(\R\).
\end{minipage}
\hfill
\begin{minipage}{.68\textwidth}
\begin{center} \contsinusoide \end{center}
\begin{center} \costante \contiperbole \end{center}
\end{minipage}

Un importante teorema afferma che ogni funzione continua in un intervallo 
chiuso è limitata.

\begin{newtheo}{Teorema di Weierstrass}{}
Supponiamo che una funzione \(f(x)\) sia continua nell'intervallo chiuso
\(\intervcc{a}{b}\) allora, in questo intervallo assume un valore massimo e 
un valore minimo.
\end{newtheo}

% \newpage %------------------------------------------------------------

\noindent Ipotesi: \quad 
\(f\) è una funzione continua nell'intervallo chiuso \(\intervcc{a}{b}\)

\noindent Tesi: 
\begin{enumerate}[nosep]
 \item \(\exists~m \in \intervcc{a}{b} \text{ tale che }
         f(m) \leqslant f(x) \quad \forall x \in \intervcc{a}{b}\);
 \item \(\exists~M \in \intervcc{a}{b} \text{ tale che }
         f(M) \geqslant f(x) \quad \forall x \in \intervcc{a}{b}\).
\end{enumerate}

\begin{proof}
Dimostriamo la prima tesi, la seconda si dimostra in modo analogo.

\vspace{-10mm}                           % Perchè???????????????????????
\begin{minipage}{.56\textwidth}
Se dividiamo l'intervallo \(\intervcc{a}{b}\) in \(h\) parti uguali 
con \(h\) numero naturale, 
otteniamo i punti di partizione \(t_i\). Confrontando i valori che assume 
la funzione in ognuno di questi punti otterremo che uno di questi valori è 
maggiore o uguale a tutti gli altri.

Operiamo ora una divisione infinita dell'intervallo \(\intervcc{a}{b}\)
ottenendo i punti di partizione: 
\[a;~a+\delta;~a+2\delta;~\dots;~a+j\delta;~\dots;~a+H\delta=b\]
\end{minipage}
\hfill
\begin{minipage}{.42\textwidth}
\vspace*{-25mm}                           % Perchè???????????????????????
\begin{center} \scalebox{.9}{\tweierstrass} \end{center}
\end{minipage}

\vspace{-10mm}                           % Perchè???????????????????????
Per il principio di tranfer posso confrontare tra di loro tutti i valori 
della funzione in questi punti e troverò che uno di questi è maggiore o 
uguale a tutti gli altri, supponiamo che questo sia vero per il punto: 
\(a+\overline{j}\delta\):
\[f(a+\overline{j}\delta) \geqslant f(a+j\delta) 
  \text{ per ogni } j \text{ Iperintero}\]
Considerando la parte standard 
se \(c=\pst{a+\overline{j}\delta}\) e \(d=\pst{a+j\delta}\)
ne deriva che: \\
\(f(c) \geqslant f(d) \quad \forall d \in I\), quindi esiste (almeno) un 
massimo dell'intervallo.
\end{proof}

\section{Teoremi sulle funzioni derivabili}
\label{subsec:cont_definizione}

\subsection{Continuità e derivabilità}
\label{subsec:cont_contderiv}


\begin{newtheo}{Derivabilità e continuità}{}
Se una funzione è derivabile in un punto allora, in quel punto, è continua.
\end{newtheo}

\noindent Ipotesi: 
\(f(x) \text{ è derivabile in } c\)
\tab Tesi: 
\(f(x) \text{ è continua in } c\).

\begin{proof}
Se \(\pst{\dfrac{dy}{dx}}=m \quad \text{allora} \quad 
\dfrac{dy}{dx}=a\) \quad con a iperreale finito.\\
Da questo deriva che \(dy= a \cdot dx\) e quindi in corrispondenza di 
un incremento infinitesimo di \(x\) anche la funzione ha una 
variazione infinitesima.
\end{proof}
\begin{newoss}{}{}
Il precedente teorema afferma che se una funzione è derivabile, in un 
intervallo incluso nel suo Insieme di Definizione, allora è continua in 
quell'intervallo, ma non dice niente del viceversa: potrebbe essere 
continua nell'intervallo ma non derivabile.
\end{newoss}

\subsection{Alcuni teoremi}
\label{subsec:cont_teoremi}

Ora affrontiamo una sequenza di teoremi collegati tra di loro.

\begin{newtheo}{Teorema di Fermat}{}
% Se una funzione è continua, 
Se una funzione è 
definita in un intervallo chiuso, 
ha un massimo (minimo) in un punto \(c\) interno all'intervallo 
dove la funzione è derivabile, 
allora in quel punto ha derivata della funzione è nulla.
\end{newtheo}

\vspace{-30mm}                           % Perchè???????????????????????
\begin{minipage}{.54\textwidth}
\noindent Ipotesi:
\begin{enumerate}[nosep]
%  \item \(f\) è una funzione continua nell'intervallo chiuso 
% \(\intervcc{a}{b}\)
 \item \(f\) è una funzione definita nell'intervallo chiuso 
\(\intervcc{a}{b}\)
 \item \(c\) appartiene all'intervallo aperto \(\intervaa{a}{b}\)
 \item \(f(c)\) è un massimo (minimo);
 \item \(f\) è derivabile in \(c\)
\end{enumerate}

\noindent Tesi: 

\hspace{4mm} la derivata \(f'(c)=0\)
\end{minipage}
\hfill
\begin{minipage}{.42\textwidth}
\begin{center} \tfermat \end{center}
\end{minipage}


\vspace{-10mm}                           % Perchè???????????????????????
\begin{proof}
Diamo la dimostrazione nel caso la funzione per \(x = c\) abbia un massimo, 
la dimostrazione nel caso di minimo è analoga.

Consideriamo un valore \(\Delta \neq 0\) abbastanza piccolo 
in modo che \(c+\Delta\) appartenga ancora all'intervallo \([a;~b]\).

Poiché \(f(c)\) è un massimo: \quad 
\(f(c+\Delta) \leqslant f(c) \sRarrow f(c+\Delta) - f(c) \leqslant 0\)

Dividendo entrambi i membri per \(\Delta\) otteniamo:
\begin{multicols}{2}
\begin{itemize} [nosep]
\item se \(\Delta < 0\): \quad 
\(\dfrac{f(c+\Delta) - f(c)}{\Delta} \geqslant 0\) \\
a sinistra di \(c\) la funzione è crescente.
\item se \(\Delta > 0\): \quad 
\(\dfrac{f(c+\Delta) - f(c)}{\Delta} \leqslant 0\) \\
a destra di \(c\) la funzione è decrescente.
\end{itemize}
\end{multicols}

Queste disuguaglianze continuano a valere anche se \(\Delta\) è un 
infinitesimo:
\begin{multicols}{2}
\begin{itemize} [nosep]
\item se \(\delta < 0\): \quad 
\(\dfrac{f(c+\delta) - f(c)}{\delta} \geqslant 0\)
\item se \(\delta > 0\): \quad 
\(\dfrac{f(c+\delta) - f(c)}{\delta} \leqslant 0\)
\end{itemize}
\end{multicols}

Prendendo le parti standard delle espressioni otteniamo le derivate 
sinistra e destra:
\begin{itemize} [nosep]
\item se \(\delta < 0\): \quad 
\(f'_-(c)=\pst{\dfrac{f(c+\delta) - f(c)}{\delta}} \geqslant 0\)
\item se \(\delta > 0\): \quad 
\(f'_+(c)=\pst{\dfrac{f(c+\delta) - f(c)}{\delta}} \leqslant 0\)
\end{itemize}

% Poiché, per ipotesi, la funzione in \(c\) è derivabile le due parti standard
% devono coincidere (perché per essere derivabile, la parte standard non deve 
% dipendere dal valore dell'infinitesimo scelto) e l'unico numero reale 
% infinitamente vicino sia a un numero negativo sia a uno positivo è zero.

% \[f'_+(c)=\pst{\frac{f(c+\delta) - f(c)}{\delta}} \leqslant 0\] 
% Ora possiamo ripetere le stesse considerazioni prendendo un valore 
% \(\Delta\) 
% negativo.
% Poiché \(f(c)\) è un massimo: 
% \[f(c+\Delta) \leqslant f(c) \Rightarrow f(c+\Delta) - f(c) \leqslant 0\]
% Questa volta dividendo entrambi i membri per \(\Delta\) dobbiamo tener 
% conto che \(\Delta\) è negativo quindi dobbiamo invertire il verso del 
% predicato:
% \[\frac{f(c+\Delta) - f(c)}{\Delta} \geqslant 0\]
% Questa disuguaglianza continua a valere anche se \(\Delta\) è un 
% infinitesimo:
% \[\frac{f(c+\delta) - f(c)}{\delta} \geqslant 0\] 
% e prendendo la parte standard dell'espressione otteniamo:
% \[f'_-(c)=\pst{\frac{f(c+\delta) - f(c)}{\delta}} \geqslant 0\] 

Poiché per ipotesi la funzione è derivabile in \(c\), il valore 
dell'espressione: \(\pst{\dfrac{f(c+\delta) - f(c)}{\delta}}\) non dipende 
dal valore dell'infinitesimo \(\delta\) perciò: \(f'_-(c) = f'_+(c)\). 
Quindi:
\[0 \leqslant f'_-(c) = f'_+(c) \leqslant 0\]
Da cui si ricava la tesi:
\hspace{26mm}\(f'(c) = 0\)
\end{proof}

Conseguenza dei teoremi di Weierstrass e Fermat è che se una funzione è 
continua in un intervallo chiuso allora in questo intervallo ha almeno un 
punto di massimo (o di minimo) che può trovarsi:\\
a) in un estremo; \quad b) in un punto non derivabile; \quad
c) in un punto la cui derivata vale zero.\\
\begin{center} \estremo \nonder \derzero \end{center}

\vspace{-15mm}                           % Perchè???????????????????????

\begin{newtheo}{Teorema di Rolle}{}
Supponiamo che una funzione \(f(x)\) continua nell'intervallo chiuso
\(\intervcc{a}{b}\),
sia derivabile nell'intervallo aperto
\(\intervaa{a}{b}\) 
e, agli estremi, assuma lo stesso valore: \(f(a) = f(b)\) allora
esiste un punto \(c\) dell'intervallo \(\intervaa{a}{b}\) nel quale 
la derivata è nulla: \qquad \(f'(c)=0\).

\end{newtheo}

\vspace*{-20mm}                           % Perchè???????????????????????
\noindent \begin{minipage}{.49\textwidth}
\noindent Ipotesi:
\begin{enumerate}[nosep]
 \item \(f\) è continua nell'intervallo chiuso \(\intervcc{a}{b}\)
 \item \(f\) è derivabile nell'intervallo aperto \(\intervaa{a}{b}\)
 \item \(f(a)=f(b)\)
\end{enumerate}

\noindent Tesi: 

\hspace{4mm}\(\exists~c \in \intervaa{a}{b} \text{ tale che } f'(c)=0\);

\vspace{1em}
(Dire che esiste un punto non esclude che ne esista anche un altro e un 
altro ancora e\dots)
\end{minipage}
\hfill
\begin{minipage}{.49\textwidth}
\vspace{-15mm}
\begin{center} \trolle \end{center}
\end{minipage}

\vspace{-15mm}                           % Perchè???????????????????????
\begin{proof}
Dato che valgono le ipotesi del teorema di Weierstrass, 
nell'intervallo \(\intervcc{a}{b}\) 
esisterà un massimo \(M\) o un minimo \(m\) (o entrambi).
Si possono distinguere 3 casi:
\begin{enumerate} %[nosep]
\item Se \(M = m = f(a) = f(b)\), la funzione è costante. 
In questo caso la dimostrazione è banale 
poiché~\(f'(c)=0 \quad \forall c \in \intervaa{a}{b}\)

\item Se la funzione ha un massimo in un punto~\(c\):~\(f(c)=M\) 
(o un minimo:~\(f(c)=m\)), 
soddisfa tutte le ipotesi del teorema di Fermat: 
\begin{enumerate}[noitemsep]
\item \(f\) è una funzione continua nell'intervallo chiuso 
\(\intervcc{a}{b}\);
\item \(c\) appartiene all'intervallo aperto \(\intervaa{a}{b}\);
\item \(f(c)\) è un massimo (o un minimo);
\item \(f\) è derivabile in \(c\).
\end{enumerate}
Quindi in \(c\) ha derivata nulla: \(f'(c)=0\).
% 
% \item Se la funzione ha un minimo in~\(c\) ovvero~\(f(c)=m\).
% Si dimostra in modo analogo al punto 2.
\end{enumerate}
In ogni caso esiste almeno un punto in cui ha derivata nulla.
\end{proof}

Il prossimo teorema assomiglia a quello di Rolle, ma qui \(f(a) \neq f(b)\).

Prima di tutto richiamiamo il concetto di \emph{pendenza media}: la pendenza 
media della funzione nell'intervallo \(\intervcc{a}{b}\) 
è data dal rapporto incrementale della funzione: \\
\(\text{pendenza media} = \dfrac{f(b)-f(a)}{b-a}\).


\begin{newtheo}{Teorema di Lagrange o della pendenza media}{}
% La pendenza media di una funzione \(f\) in un intervallo 
% \(\intervcc{a}{b}\) è data da: \quad 
% \(\text{pendenza media} = \dfrac{f(b)-f(a)}{b-a}\)

Se una funzione \(f\) è continua nell'intervallo chiuso \(\intervcc{a}{b}\) 
e è derivabile nell'intervallo aperto \(\intervaa{a}{b}\) allora
esiste (almeno) un punto \(c\) dell'intervallo \(\intervaa{a}{b}\) nel quale 
la derivata ha lo stesso valore della 
pendenza media: \qquad 
\(f'(c) = \dfrac{f(b)-f(a)}{b-a}\).

\end{newtheo}

\vspace*{-35mm}                           % Perchè???????????????????????
\begin{minipage}{.54\textwidth}
\noindent Ipotesi:
\begin{enumerate}[nosep]
 \item \(f\) è continua 
 nell'intervallo chiuso \(\intervcc{a}{b}\)
 \item \(f\) è derivabile 
 nell'intervallo aperto \(\intervaa{a}{b}\)
\end{enumerate}

\noindent Tesi: 

\hspace{4mm}\(\exists~c \in \intervaa{a}{b} \stext{ tale che } 
f'(c)=\dfrac{f(b)-f(a)}{b-a}\);
\end{minipage}
\hfill
\begin{minipage}{.42\textwidth}
\begin{center} \tlagrange \end{center}
\end{minipage}

\vspace{-15mm}                           % Perchè???????????????????????

\begin{proof}
% Chiamiamo \(m\) la pendenza media: \(m=\dfrac{f(b)-f(a)}{b-a}\). \\
La funzione lineare che congiunge i due punti \(A\) e \(B\) è:

\(l(x) = \dfrac{f(b)-f(a)}{b-a}(x-a) + f(a)\)\\
Costruiamo una nuova funzione togliendo da \(f(x)\) la funzione 
\(l(x)\): \quad 
\(h(x) = f(x) - l(x)\) \\
La nuova funzione \(h\) soddisfa tutte le ipotesi del teorema di Rolle:
\begin{enumerate}[nosep]
\item \(h\) è continua nell'intervallo chiuso \(\intervcc{a}{b}\)
poiché è somma di due funzioni continue;
\item \(h\) è derivabile nell'intervallo aperto \(\intervaa{a}{b}\)
poiché è somma di due funzioni derivabili;
\item agli estremi dell'intervallo la funzione ha lo stesso 
valore: \(h(a) = h(b) = 0\):

\(h(a) = f(a) - (\dfrac{f(b)-f(a)}{b-a}(a-a) + f(a)) = f(a)-0-f(a) = 0\)

e 

\(h(b) = %f(b) - (f(a) + m(b-a))=
f(b)-\dfrac{f(b)-f(a)}{\cancel{b-a}} \cdot 
\cancel{\tonda{b-a}}-f(a)=f(b)-f(b)+f(a)-f(a)=0\)
\end{enumerate}

Quindi esiste un punto \(c\) dell'intervallo \(\intervaa{a}{b}\) tale che:
\quad \(h'(c)=0\) \\
Sostituendo la funzione \(h\) con la sua definizione otteniamo:\\[4mm]
\hspace*{15mm}\(h'(c) = f'(c)-l'(c)= f'(c)-\dfrac{f(b)-f(a)}{b-a}=0 
~~\sRarrow ~~
f'(c)=\dfrac{f(b)-f(a)}{b-a}\)
% E sostituendo \(m\) otteniamo la tesi:
% \[f'(c)=\dfrac{f(b)-f(a)}{b-a}\]
\end{proof}

% \pagebreak %-------------------------------------------------

\begin{newcor}{}{}
Derivata e andamento di una funzione. Se in una funzione \(f'(x)>0\) per ogni 
punto di un certo intervallo \(I\), allora la funzione è crescente in tutto 
l'intervallo.
\end{newcor}
% 
% \newcommand{\sand}{~ \wedge ~}
% \newcommand{\sor}{~ \vee ~}
% \newcommand{\sRarrow}{~ \Rightarrow ~}

\bigskip
\begin{proof}
Consideriamo due punti qualunque dell'intervallo \(I\) con \(x_1 > x_0\), 
\\[2mm]
per il teorema della pendenza media: \hspace{22mm} 
\(\exists c \in I \text { tale che } 
  f'(c) = \dfrac{f(x_1)-f(x_0)}{x_1-x_0}\) \\
% dato che per ipotesi: \hspace{39mm}
% \(\forall c \in I \Rightarrow f'(c) > 0\) \\[1mm]
% si ha:\hspace{63mm}
% \(\dfrac{f(x_1)-f(x_0)}{x_1-x_0}>0 \)\\[1mm]
dato che per ipotesi la derivata è positiva, anche: \hspace{3.5mm}
% \(\forall c \in I \Rightarrow f'(c) > 0\) \\[1mm]
% si ha:\hspace{63mm}
\(\dfrac{f(x_1)-f(x_0)}{x_1-x_0}>0 \)\\[3mm]
E poiché il denominatore è positivo, si ha: \hspace{14.5mm}
\(f(x_1)-f(x_0)>0 \sRarrow f(x_1)>f(x_0)\)
\end{proof}

\bigskip
Un teorema che fornisce uno strumento per semplificare il calcolo dei 
limiti che si presentano nella forma indeterminata: 
\(\dfrac{\epsilon}{\delta}\) % o \(\dfrac{M}{N}\)
, è il seguente.


\begin{newtheo}{De L'H\^opital}{}

Date due funzioni \(f(x)\) e \(g(x)\) derivabili in \(x = c\) 
e che in \(c\) valgono zero, allora: \qquad 
\(\displaystyle \lim_{x \to c} \dfrac{f(x)}{g(x)} = 
  \dfrac{f'(c)}{g'(c)}\).
%   \lim_{x \to c} \dfrac{f'(c)}{g'(c)}\).
% Date due funzioni \(f(x)\) e \(g(x)\) definite in un intervallo \(I\) 
% aperto che contiene \(c\) eventualmente possono non essere definite in 
% \(c\).
% 
% Se \(f\) e \(g\) sono derivabili nell'insieme \(I-\graffa{c}\) e 
% \(g' \neq 0\) Entrambe le funzioni tendono a zero o all'infinito per 
% \(x \to c\) e
% esiste \(\dfrac{f'(c)}{g'(c)}\) allora:
% \(\displaystyle \lim_{x \to c} \dfrac{f(x)}{g(x)}=\dfrac{f'(c)}{g'(c)}\).

\end{newtheo}

\noindent Ipotesi: 
\begin{enumerate}[nosep]
 \item \(f(c) = 0 \stext{e} g(c) = 0\);
 \item \(f\) e \(g\) sono derivabili in \(c\);
\end{enumerate}

\noindent Tesi: 

\(\displaystyle \lim_{x \to c}\dfrac{f(x)}{g(x)} = 
  \dfrac{f'(c)}{g'(c)}\); 
%   \lim_{x \to c}\dfrac{f'(c)}{g'(c)}\); 

\begin{proof}
Consideriamo un infinitesimo \(\epsilon \neq 0\):
\begin{align*}
\lim_{x \to c} \dfrac{f(c)}{g(c)} &\stackrel{1}{=}
\pst{\dfrac{f(c+\epsilon)}{g(c+\epsilon)}}\stackrel{2}{=}
\pst{\dfrac{f(c+\epsilon) - f(c)}{g(c+\epsilon) - f(c)}}\stackrel{3}{=} \\
&=\pst{\dfrac{\dfrac{f(c+\epsilon) - f(c)}{\epsilon}}
             {\dfrac{g(c+\epsilon) - f(c)}{\epsilon}}}\stackrel{4}{=}
\dfrac{\pst{\dfrac{f(c+\epsilon) - f(c)}{\epsilon}}}
        {\pst{\dfrac{g(c+\epsilon) - f(c)}{\epsilon}}}\stackrel{5}{=} 
\dfrac{f'(c)}{g'(c)}
\end{align*}
Dove i passaggi hanno le seguenti giustificazioni:
\begin{enumerate} [nosep]
\item definizione di limite al finito;
\item dato che \(f(c) = 0 \stext{e} g(c) = 0\);
\item dividiamo il numeratore e il denominatore per lo stesso infinitesimo 
\(\epsilon\);
\item per una proprietà della parte standard;
\item dato che le funzioni sono derivabili in \(c\) le parti 
standard non dipendono dal particolare incremento infinitesimo scelto e 
hanno come risultato le derivate delle funzioni.
\end{enumerate}
\end{proof}

Più in generale: 
date due funzioni \(f(x)\) e \(g(x)\) derivabili in \(x = c\) 
e che in \(c\) valgono zero, allora: \quad 
\(\displaystyle \lim_{x \to c} \dfrac{f(x)}{g(x)} = 
  \lim_{x \to c} \dfrac{f'(c)}{g'(c)}\).

\bigskip
Il teorema di De L'H\^opital si estende anche al caso in cui in \(c\) le 
funzioni abbiano un valore infinito.

% TODO
% \section{Applicazioni}
% in questa sezione presentiamo alcuni esercizi nei quali viengono 
% utilizzati i teoremi presentati nel capitolo
% \subsection{???} 
% OPPURE
% \begin{esempio}
% 
% \end{esempio}








