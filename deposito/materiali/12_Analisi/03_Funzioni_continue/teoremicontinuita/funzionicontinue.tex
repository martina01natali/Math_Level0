% (c) 2015 Daniele Zambelli daniele.zambelli@gmail.com

\input{\folder funzionicontinue_grafici.tex}

\chapter{Teoremi sulle funzioni continue}

\section{Numeri iperinteri}
\label{subsec:cont_iperinteri}

Per affrontare alcuni dei prossimi argomenti, abbiamo bisogno di un altro 
strumento matematico: l'insieme dei numeri \emph{Iperinteri}.
Non è difficile visualizzare sulla retta dei numeri questo insieme.

Per dare una definizione rigorosa dei numeri Iperinteri abbiamo bisogno di 
usare la funzione \emph{parte intera} di un numero: che si indica con il 
simbolo:
\(\quadra{x}\). Qualche esempio:
\begin{center}
\begin{tabular}{ccccccccccccccc}
\(x\) & 
-3&-2,4&-2,01&-1,2&-1,03&-0,3&0,2&1&1,6&1,99&2&2.03&2.9&3,42\\
\hline
\(y=\quadra{x}\) & 
-3&-3  &-3   &-2  &-2   &-1  &0  &1&1  &1   &2&2   &2  &3
\end{tabular}
\end{center}

\begin{minipage}{.59\textwidth}
La parte intera di un numero \(x\) è il più grande numero intero \(n\) 
minore o uguale a \(x\). Attenzione che mentre per i numeri positivi il 
concetto è abbastanza naturale, per quelli negativi il concetto non è 
altrettanto immediato, vedi la tabella precedente.

Vale la pena di sottolineare che, mentre per i numeri positivi, la parte 
intera si ottiene togliendo la parte decimale, per i numeri negativi la 
parte intera è un po' meno intuitiva: \(\quadra{-4,12}=-5\).
\end{minipage}
\hfill
\begin{minipage}{.39\textwidth}
\begin{center}
\parteintera
\end{center}
\end{minipage}

Applicando la funzione parte intera ai numeri Iperreali otteniamo gli 
Iperinteri.

\begin{definizione}
 I numeri \textbf{Iperinteri} sono quei numeri Iperreali per cui vale 
l'uguaglianza:
 \[x=\quadra{x}\]
\end{definizione}

Possiamo fare alcune osservazioni sugli Iperinteri:

\begin{enumerate}
 \item 
La somma algebrica di due numeri Iperinteri è un numero Iperintero.
 \item 
Ogni numero Iperreale si trova tra due numeri Iperinteri:
\[\forall x \in \IR \quadra{x} \leqslant x < \quadra{x}+1\]
\end{enumerate}

Possiamo usare gli Interi per dividere un intervallo Reale \([a;~b]\)
in \(n\) parti uguali. Ciascuna di queste \(n\) parti 
uguali è lunga \(l=\frac{b-a}{n}\).

Gli \(n\) sotto intervalli che si ottengono sono:
\[\left[a;~a+l \right[,~[a+l;~a+2l[,~\dots,~
[a+jl;~a+(j+1)l[,~\dots,~
[a+\tonda{n-1}l;~a+nl=b]\]

Gli estremi di questi intervalli sono chiamati punti di partizione 
dell'intervallo:
\[a;~a+l;~a+2l;~a+3l;~\dots;~a+\tonda{n-1}l;~a+nl=b\]

Possiamo ora estendere questo procedimento ai numeri Iperreali.
Scegliamo un numero infinito iperintero \(H\) e dividiamo in parti 
uguali l'intervallo di numeri Iperreali \(\intervcc{a}{b}\). Ogni 
sotto intervallo avrà la stessa lunghezza infinitesima 
\(\delta=\frac{b-a}{H}\).

Gli \(H\) sotto intervalli che si ottengono sono:
\[\left[a;~a+\delta \right[,~[a+\delta;~a+2\delta[,~\dots,~
[a+j\delta;~a+(j+1)\delta[,~\dots,~
[a+\tonda{H-1}\delta;~a+H\delta=b]\]

e i punti di partizione sono:
\[a;~a+\delta;~a+2\delta;~\dots;~a+j\delta;~\dots;~a+H\delta=b\]
cioè i punti \(a+j\delta\) con \(j\) che varia da~0 a~\(H\).

Ogni numero iperreale \(x\) appartenente all'intervallo \(\intervca{a}{b}\)
apparterrà a uno dei sotto intervalli infinitesimi:
\[x \in \intervca{a+j\delta}{a+\tonda{j+1}\delta} \quad \Rightarrow \quad 
  a+j\delta \leqslant x < a+\tonda{j+1}\delta\]

Possiamo ora affrontare alcuni teoremi riguardanti le funzioni continue.

\section{Continuità}
\label{sec:cont_continuita}

\subsection{Definizione}
\label{sec:cont_definizione}

Richiamiamo il significato di continuità:

\begin{definizione}
Diremo che una funzione è \textbf{continua} in un punto \(c\), 
se è definita in \(c\) e, 
quando \(x\) è infinitamente vicino a \(c\), 
allora \(f(x)\) è infinitamente vicino a \(f(c)\). E si scrive::

\[f \text{ è continua in } c \Leftrightarrow 
\forall x \tonda{\tonda{x \approx c} \Rightarrow 
\tonda{f(x) \approx f(c)}}\]
\end{definizione}

Data una funzione \(y=f(x)\) definita nel punto \(c\), le seguenti 
affermazioni sono equivalenti:

\begin{enumerate}[noitemsep]
 \item \(f\) è continua in \(c\);
 \item se \(x \approx c\) allora \(f(x) \approx f(c)\) 
       (\(\approx\): infinitamente vicino);
 \item se \(\st(x) = c\) allora \(\st(f(x)) = f(c)\);
 \item \(\lim_{x \to c} f(x) = f(c)\);
 \item se \(x\) si allontana da \(c\) di un infinitesimo allora 
   \(f(x)\) si allontana da \(f(c)\) di un infinitesimo.
 \item se \(\Delta x\) è infinitesimo allora il corrispondente \(\Delta y\) 
   è infinitesimo.
\end{enumerate}

\subsection{Alcuni teoremi delle funzioni continue}
\label{subsec:cont_iperinteri}

Il prossimo teorema riguarda gli zeri di una funzione. Con zero di una 
funzione 
si intende un valore della \(x\) che rende la funzione uguale a zero:

\begin{definizione}
 \(c\) è uno \textbf{zero} della funzione \(f(x)\) se \(f(c)=0\)
\end{definizione}

\begin{teorema}[Teorema degli zeri]
Supponiamo che una funzione \(f(x)\) sia continua nell'intervallo chiuso
\(\intervcc{a}{b}\) e agli estremi dell'intervallo assuma valori di segno 
opposto allora la funzione ha uno zero nell'intervallo 
aperto~\(\intervaa{a}{b}\).
\end{teorema}

\noindent Ipotesi: \nopagebreak
\begin{enumerate}[nosep]
 \item \(f\) è una funzione continua;
 \item \(f\) è definita nell'intervallo chiuso \(\intervcc{a}{b}\);
 \item \(f(a) \cdot f(b) < 0\) 
 (equivale a dire che \(f(a)\) e \(f(b)\) hanno valori discordi).
\end{enumerate}

\noindent Tesi: 

\(\exists~c \in \intervaa{a}{b} \text{ tale che } f(c)=0\).

\begin{proof}
Supponiamo che \(f(a)<0\) e \(f(b)>0\), posto \(H\) un numero iperintero 
infinito, dividiamo l'intervallo iperreale~\(\intervcc{a}{b}\) in~\(H\) 
parti uguali

\begin{minipage}{.58\textwidth}
\[a;~a+\delta;~a+2\delta;~\dots;~a+j\delta;~\dots;~a+H\delta=b\]
Chiamiamo \(k\) l'indice per il quale:
\[f(a+k\delta) \leqslant 0 < f(a+(k+1)\delta)\]
Dato che \(f\) è continua:
\[a+k\delta \approx a+(k+1)\delta \quad \Rightarrow \quad 
f(a+k\delta) \approx f(a+(k+1)\delta)\] 
Ma l'unico numero standard che è infinitamente vicino ad un numero
minore di zero e anche ad un numero maggiore di zero è zero. 
Quindi chiamando:\(c=\pst{a+k\delta}=\pst{a+(k+1)\delta}\)
\[f(c)=f(\pst{a+k\delta})=\pst{f(a+k\delta)}=0\]
In modo analogo si dimostra il caso in cui \(f(a)>0>f(b)\).
\end{minipage}
\hfill
\begin{minipage}{.38\textwidth}
\begin{center} \tzeri \end{center}
\end{minipage}

\end{proof}

Da notare che il teorema dimostra che in \(\intervcc{a}{b}\) c'è almeno uno 
zero della funzione, ma non dice nulla sul numero degli zeri.

La dimostrazione di questo teorema permette di dimostrarne facilmente degli 
altri: 

\begin{corollario}
Il teorema dei valori intermedi dice che se una funzione è continua in un
intervallo \(\intervcc{a}{b}\) allora tra \(a\) e \(b\) assume tutti i 
valori 
compresi tra \(f(a)\) e \(f(b)\).
\end{corollario}

\begin{proof}
Suggerimento: 
supponiamo che \(f(a) \leqslant h < f(b)\), 
costruiamo una nuova funzione: \(g(x)=f(x)-h\).
La funzione \(g\) soddisfa tutte le ipotesi del teorema precedente per cui:
\[\exists~c \in \intervaa{a}{b} \text{ tale che } g(c)=0\]
e sostituendo la funzione \(g\) otteniamo:
\[g(c)=0 \Rightarrow f(c)-h=0 \Rightarrow f(c)=h\].
\end{proof}

\begin{corollario}
Se una funzione è continua in un
intervallo \(\intervcc{a}{b}\) e 
\(f(x)\neq 0 \quad \forall x \in \intervcc{a}{b}\) 
allora:
\begin{enumerate}[nosep]
 \item se \(f(c)<0\) \quad per un qualunque \quad \(c \in \intervcc{a}{b}\)
 \quad allora \quad \(f(x)<0 \quad \forall x \in \intervcc{a}{b}\);
 \item se \(f(c)>0\) \quad per un qualunque \quad \(c \in \intervcc{a}{b}\)
 \quad allora \quad \(f(x)>0 \quad \forall x \in \intervcc{a}{b}\).
\end{enumerate}
\end{corollario}

\begin{proof}
Suggerimento: 
consideriamo il primo caso: \(f(c)>0\) 
se esistesse un valore \(d \in \intervcc{a}{b}\) per cui
\(f(d)>0\) allora nell'intervallo \(\intervcc{c}{d}\) 
esiste un punto \(e \Rightarrow f(e)=0\)
essendo \(\intervcc{c}{d} \subseteq \intervcc{a}{b}\) 
ciò contraddirebbe l'ipotesi.
In modo analogo si dimostra il secondo caso.
\end{proof}

\section{Massimi e minimi}
\label{sec:cont_massimiminimi}

Data una funzione definita in un certo intervallo, può darsi che questa 
funzione abbia un massimo o un minimo in questo intervallo.

\begin{definizione}
 Chiamiamo \textbf{massimo} di una funzione in un intervallo \(I\) un punto 
\(\punto{c}{f(c)}\) tale che, per ogni \(x\) appartenente all'intervallo, 
\(f(c)\) sia maggiore o uguale a \(f(x)\):
\[\punto{c}{f(c)} \text{ è un massimo se } \forall x \in I \quad
f(c) \geqslant f(x)\]
\end{definizione}

\begin{minipage}{.25\textwidth}
La definizione di \textbf{minimo} in un intervallo si ottiene facilmente 
modificando quella di massimo (scrivila tu e poi confrontala con quella 
scritta dagli altri tuoi compagni).

In un intervallo, una funzione potrebbe avere \emph{più} minimi o massimi 
oppure potrebbe \emph{non} avere minimi o massimi, vedi i grafici a fianco.
\end{minipage}
\hfill
\begin{minipage}{.68\textwidth}
\begin{center} \contsinusoide \end{center}
\begin{center} \costante \contiperbole \end{center}
\end{minipage}

\newpage %------------------------------------------------------------

Un importante teorema è il seguente.

\begin{teorema}[Teorema di Weierstrass]
Supponiamo che una funzione \(f(x)\) sia continua nell'intervallo chiuso
\(\intervcc{a}{b}\) allora, in questo intervallo assume un valore massimo e 
un valore minimo.
\end{teorema}
\vspace{-30mm}                           % Perchè???????????????????????
\begin{minipage}{.56\textwidth}
\noindent Ipotesi:
\begin{enumerate}[nosep]
 \item \(f\) è una funzione continua;
 \item \(f\) è definita nell'intervallo chiuso \(\intervcc{a}{b}\)
\end{enumerate}

\noindent Tesi: 
\begin{enumerate}[nosep]
 \item \(\exists~c \in \intervcc{a}{b} \text{ tale che }
         f(c) \geqslant f(x) \quad \forall x \in \intervcc{a}{b}\);
 \item \(\exists~c \in \intervcc{a}{b} \text{ tale che }
         f(c) \leqslant f(x) \quad \forall x \in \intervcc{a}{b}\).
\end{enumerate}
\end{minipage}
\hfill
\begin{minipage}{.42\textwidth}
\begin{center} \tweierstrass \end{center}
\end{minipage}

\vspace{-10mm}                           % Perchè???????????????????????
\begin{proof}
Dimostriamo la prima tesi.

Se dividiamo l'intervallo \(\intervcc{a}{b}\) in \(h\) parti uguali, 
otteniamo i punti di partizione \(t_i\). Confrontando i valori che assume 
la funzione in ognuno di questi punti otterremo che uno di questi valori è 
maggiore o uguale a tutti gli altri.

Operiamo ora una divisione infinita dell'intervallo \(\intervcc{a}{b}\)
ottenendo i punti di partizione: 
\[a;~a+\delta;~a+2\delta;~\dots;~a+j\delta;~\dots;~a+H\delta=b\]
Per il principio di tranfer posso confrontare tra di loro tutti i valori 
della funzione in questi punti e troverò che uno di questi è maggiore o 
uguale a tutti gli altri, supponiamo che questo punto sia: \(a+k\delta\):
\[f(a+k\delta) \geqslant f(a+j\delta) \text{ per ogni j Iperintero}\]
Considerando la parte standard 
se \(c=\pst{a+k\delta}\) e \(d=\pst{a+j\delta}\)
ne deriva che:
\[f(c) \geqslant f(d)\]

In modo analogo si dimostra la seconda tesi.
\end{proof}


\section{Teoremi sulle funzioni derivabili}
\label{subsec:cont_definizione}

\subsection{Continuità e derivabilità}
\label{subsec:cont_contderiv}

\begin{teorema}[Derivabilità e continuità]
Se una funzione è derivabile in un punto allora, in quel punto, è continua.
\end{teorema}

\noindent Ipotesi: 
\(f(x) \text{ è derivabile in } c\)
\tab Tesi: 
\(f(x) \text{ è continua in } c\).

\begin{proof}

Se \(\pst{\dfrac{dy}{dx}}=m \quad \text{allora} \quad 
\dfrac{dy}{dx}=a \quad \text{con a finito} \)\\
Da questo deriva che \(dy= a \cdot dx\) e quindi in corrispondenza di 
un incremento infinitesimo di \(x\) anche la funzione ha una 
variazione infinitesima.
\end{proof}
\begin{osservazione}
Il precedente teorema afferma che se una funzione è derivabile, in un 
intervallo incluso nel suo Insieme di Definizione, allora è continua in 
quell'intervallo, ma non dice niente del viceversa: Potrebbe essere 
continua nell'intervallo ma non derivabile.
\end{osservazione}

\subsection{Alcuni teoremi}
\label{subsec:cont_teoremi}

Ora affrontiamo una sequenza di teoremi collegati tra di loro.
\begin{teorema}[Teorema di Fermat]
% Se una funzione è continua, 
Se una funzione è  
definita in un intervallo chiuso, 
ha un massimo (minimo) in un punto interno all'intervallo
e in quel punto è derivabile, 
allora in quel punto ha derivata nulla.
\end{teorema}

\vspace{-30mm}                           % Perchè???????????????????????
\begin{minipage}{.54\textwidth}
\noindent Ipotesi:
\begin{enumerate}[nosep]
%  \item \(f\) è una funzione continua nell'intervallo chiuso 
% \(\intervcc{a}{b}\)
 \item \(f\) è una funzione definita nell'intervallo chiuso 
\(\intervcc{a}{b}\)
 \item \(c\) appartiene all'intervallo aperto \(\intervaa{a}{b}\)
 \item \(f(c)\) è un massimo (minimo);
 \item \(f\) è derivabile in \(c\)
\end{enumerate}

\noindent Tesi: 

la derivata \(f'(c)=0\)
\end{minipage}
\hfill
\begin{minipage}{.42\textwidth}
\begin{center} \tfermat \end{center}
\end{minipage}


\vspace{-10mm}                           % Perchè???????????????????????
\begin{proof}
Consideriamo un valore \(\Delta\) positivo abbastanza piccolo 
in modo che \(c+\Delta\) appartenga ancora all'intervallo \([a;~b]\).
Poiché \(f(c)\) è un massimo: 
\[f(c+\Delta) \leqslant f(c) \Rightarrow f(c+\Delta) - f(c) \leqslant 0\]
Dividendo entrambi i membri per \(\Delta\) otteniamo:
\[\frac{f(c+\Delta) - f(c)}{\Delta} \leqslant 0\]
Questa disuguaglianza continua a valere anche se \(\Delta\) è un 
infinitesimo:
\[\frac{f(c+\delta) - f(c)}{\delta} \leqslant 0\] 
e prendendo la parte standard dell'espressione otteniamo:
\[f'_+(c)=\pst{\frac{f(c+\delta) - f(c)}{\delta}} \leqslant 0\] 
Ora possiamo ripetere le stesse considerazioni prendendo un valore 
\(\Delta\) 
negativo.
Poiché \(f(c)\) è un massimo: 
\[f(c+\Delta) \leqslant f(c) \Rightarrow f(c+\Delta) - f(c) \leqslant 0\]
Questa volta dividendo entrambi i membri per \(\Delta\) dobbiamo tener 
conto che \(\Delta\) è negativo quindi dobbiamo invertire il verso del 
predicato:
\[\frac{f(c+\Delta) - f(c)}{\Delta} \geqslant 0\]
Questa disuguaglianza continua a valere anche se \(\Delta\) è un 
infinitesimo:
\[\frac{f(c+\delta) - f(c)}{\delta} \geqslant 0\] 
e prendendo la parte standard dell'espressione otteniamo:
\[f'_-(c)=\pst{\frac{f(c+\delta) - f(c)}{\delta}} \geqslant 0\] 

Ma poiché per ipotesi la funzione è derivabile in \(c\), il valore 
dell'espressione: \(\pst{\frac{f(c+\delta) - f(c)}{\delta}}\) non dipende 
dal valore dell'infinitesimo \(\delta\) perciò: \(f'_-(c) = f'_+(c)\). 
Quindi:
\[0 \leqslant f'_-(c) = f'_+(c) \leqslant 0\]
Da cui si ricava la tesi:
\[f(c) = 0\]
\end{proof}

Conseguenza dei teoremi di Weierstrass e Fermat è che se una funzione è 
continua in un intervallo chiuso allora in questo intervallo ha almeno un 
punto di massimo (o di minimo) che può trovarsi:\\
a) in un estremo; \quad b) in un punto non derivabile; \quad
c) in un punto la cui derivata vale zero.\\
\begin{center} \estremo \nonder \derzero \end{center}

\vspace{-15mm}                           % Perchè???????????????????????
\begin{teorema}[Teorema di Rolle]
Supponiamo che una funzione \(f(x)\) continua nell'intervallo chiuso
\(\intervcc{a}{b}\),
sia derivabile nell'intervallo aperto
\(\intervaa{a}{b}\) 
e, agli estremi, assuma lo stesso valore: \(f(a) = f(b)\) allora
esiste un punto \(c\) dell'intervallo \(\intervaa{a}{b}\) nel quale 
la derivata è nulla: \(f'(c)=0\).
\end{teorema}

\vspace*{-35mm}                           % Perchè???????????????????????
\begin{minipage}{.54\textwidth}
\noindent Ipotesi:
\begin{enumerate}[nosep]
 \item \(f\) è continua nell'intervallo chiuso \(\intervcc{a}{b}\)
 \item \(f\) è derivabile nell'intervallo aperto \(\intervaa{a}{b}\)
 \item \(f(a)=f(b)\)
\end{enumerate}

\noindent Tesi: 

\(\exists~c \in \intervaa{a}{b} \text{ tale che } f'(c)=0\);
\end{minipage}
\hfill
\begin{minipage}{.42\textwidth}
\begin{center} \trolle \end{center}
\end{minipage}

\vspace{-15mm}                           % Perchè???????????????????????
\begin{proof}
Dato che valgono le ipotesi del teorema di Weierstrass, 
nell'intervallo \(\intervcc{a}{b}\) 
esisterà un massimo \(M\) e un minimo \(m\).
Si possono distinguere 3 casi:
\begin{enumerate} %[nosep]
 \item Se \(M = m = f(a) = f(b)\), la funzione è costante. 
 In questo caso la dimostrazione è banale 
 poiché~\(f'(c)=0 \forall c \in \intervaa{a}{b}\)
 \item Se \(M > f(a)\) vuol dire che la funzione ha un massimo 
 in~\(c\) ovvero~\(f(c)=M\). 
 La funzione \(f\) nel punto \(c\) soddisfa tutte le ipotesi del teorema 
 di Fermat, 
\begin{enumerate}[noitemsep]
 \item \(f\) è una funzione continua nell'intervallo chiuso 
  \(\intervcc{a}{b}\);
 \item \(c\) appartiene all'intervallo aperto \(\intervaa{a}{b}\);
 \item \(f(c)\) è un massimo;
 \item \(f\) è derivabile in \(c\).
\end{enumerate}
 quindi in \(c\) la funzione ha derivata nulla: \(f'(c)=0\).
 \item Se \(m < f(a)\) vuol dire che la funzione ha un minimo 
 in~\(c\) ovvero~\(f(c)=m\).
 Si dimostra in modo analogo al punto 2.
\end{enumerate}
\end{proof}

\begin{teorema}[Teorema di Lagrange o della pendenza media]
La pendenza media di una funzione \(f\) in un intervallo 
\(\intervcc{a}{b}\) è data da:
\[\text{pendenza media} = \frac{f(b)-f(a)}{b-a}\]
Se una funzione \(f\) è continua nell'intervallo chiuso \(\intervcc{a}{b}\) e
è derivabile nell'intervallo aperto \(\intervaa{a}{b}\) allora
esiste un punto \(c\) dell'intervallo \(\intervaa{a}{b}\) nel quale 
la derivata è ha lo stesso valore della 
pendenza media:~\(f'(c)=\text{pendenza media}\).
\end{teorema}

\vspace*{-35mm}                           % Perchè???????????????????????
\begin{minipage}{.54\textwidth}
\noindent Ipotesi:
\begin{enumerate}[nosep]
 \item \(f\) è continua 
 nell'intervallo chiuso \(\intervcc{a}{b}\)
 \item \(f\) è derivabile 
 nell'intervallo aperto \(\intervaa{a}{b}\)
\end{enumerate}

\noindent Tesi: 

\(\exists~c \in \intervaa{a}{b}\text{ tale che } 
f'(c)=\dfrac{f(b)-f(a)}{b-a}\);
\end{minipage}
\hfill
\begin{minipage}{.42\textwidth}
\begin{center} \tlagrange \end{center}
\end{minipage}

\vspace{-15mm}                           % Perchè???????????????????????

\begin{proof}
Chiamiamo \(m\) la pendenza media: \(m=\dfrac{f(b)-f(a)}{b-a}\). 
La funzione lineare che congiunge i due 
punti \(\punto{a}{f(a)}\) e \(\punto{b}{f(b)}\) è:
\[l(x) = f(a) + m(x-a)\]
Costruiamo una nuova funzione uguale alla distanza in verticale tra \(f(x)\) 
e \(l(x)\):
\[h(x) = f(x) - l(x)\]
Questa nuova funzione soddisfa tutte le ipotesi del teorema di Rolle:
\begin{enumerate}[nosep]
 \item \(h\) è una funzione continua 
 nell'intervallo chiuso \(\intervcc{a}{b}\)
 poiché è somma di due funzioni continue;
 \item \(h\) è una funzione derivabile 
 nell'intervallo aperto \(\intervaa{a}{b}\)
 poiché è somma di due funzioni derivabili;
 \item \(h(a)=h(b)\)  poiché:
 
 \(h(a)=f(a) - (f(a) + m(a-a))=f(a)-f(a)-0=0\)
 
 e 
 
 \(h(b)=f(b) - (f(a) + m(b-a))=
 f(b)-f(a)-\dfrac{f(b)-f(a)}{b-a}\cdot \tonda{b-a}=\)\\
 \(=f(b)-f(a)-f(b)+f(a)=0\)
\end{enumerate}
 
 Quindi esiste un punto \(c\) dell'intervallo \(\intervaa{a}{b}\) tale che:
 \[h'(c)=0\]
 Sostituendo la funzione \(h\) con la sua definizione otteniamo:
 \[h'(c) = f'(c)-l'(c)= f'(c)-m=0 \Rightarrow f'(c)=m\]
 E sostituendo \(m\) otteniamo la tesi:
 \[f'(c)=\dfrac{f(b)-f(a)}{b-a}\]
\end{proof}

\begin{corollario}
 Derivata e andamento di una funzione. Se in una funzione \(f'(x)>0\) per ogni 
punto di un certo intervallo \(I\), allora la funzione è crescente in tutto 
l'intervallo.
\end{corollario}
% 
% \newcommand{\sand}{~ \wedge ~}
% \newcommand{\sor}{~ \vee ~}
% \newcommand{\sRarrow}{~ \Rightarrow ~}

\begin{proof}
Consideriamo due punti \(x_0 < x_1 ~ \in I\), 
per il teorema della pendenza media:
\[\exists c \in I \text { tale che } f'(c) = \dfrac{f(x_1)-f(x_0)}{x_1-x_0}\]
Dato che 
\[\forall c \in I \Rightarrow f'(c) > 0\]
si ha:
\[\dfrac{f(x_1)-f(x_0)}{x_1-x_0}>0 \]
E poiché il denominatore è positivo, \(x_1 - x_0 > 0\)
\[f(x_1)-f(x_0)>0 \sRarrow f(x_1)>f(x_0)\]
\end{proof}

Un teorema che fornisce uno strumento per semplificare il calcolo dei 
limiti che si presentano nella forma indeterminata: 
\(\dfrac{\epsilon}{\delta}\) o \(\dfrac{M}{N}\), è il seguente:

\begin{teorema}[De L'H\^opital]

Date due funzioni \(f(x)\) e \(g(x)\) definite in un intervallo \(I\) 
aperto che contiene \(c\) eventualmente possono non essere definite in 
\(c\).

Se \(f\) e \(g\) sono derivabili nell'insieme \(I-\graffa{c}\) e 
\(g' \neq 0\) Entrambe le funzioni tendono a zero o all'infinito per 
\(x \to c\) e
esiste \(\dfrac{f'(c)}{g'(c)}\) allora:
\(\displaystyle \lim_{x \to c} \dfrac{f(x)}{g(x)}=\dfrac{f'(c)}{g'(c)}\).
\end{teorema}

\noindent Ipotesi: 
\begin{enumerate}[nosep]
 \item \(f\) \(g\) sono definite e derivabili in \(I-\graffa{c}\);
 \item \(g'(x)\neq 0 ~ \forall x \in I-\graffa{c}\); 
 \item \((\displaystyle \lim_{x \to c}{f(x)}=
          \lim_{x \to c}{g(x)}=0\) o (\(\infty\))
 \item \(\exists \dfrac{f'(c)}{g'(c)}\); 
\end{enumerate}

\noindent Tesi: 

\(\displaystyle \lim_{x \to c}{f(x)}{g(x)}=\dfrac{f'(c)}{g'(c)}\); 

% \begin{proof}
% 
% \end{proof}












