% (c) 2017 Daniele Zambelli - daniele.zambelli@gmail.com
% 
% I grafici per i teoremi sulle funzioni continue
% 
% 

\newcommand{\parteintera}{% 
  % prima funzione.
  \disegno[4]{
  \rcom{-5}{+5}{-5}{+5}{gray!50, very thin, step=1}
%   \tkzInit[xmin=-5.3, xmax=+7.3, ymin=-7.3, ymax=+7.3]
    \foreach \pi in {-5, ..., +4}
    {\draw [Maroon!50!black, ultra thick] (\pi, \pi) -- (\pi+1, \pi);
    \filldraw [Maroon!50!black] (\pi, \pi) circle (1.7pt);
    \draw [Maroon!50!black] (\pi+1, \pi) circle (1.7pt);}
    \def \pi {5}
    \draw [Maroon!50!black, ultra thick] (\pi, \pi) -- (\pi+.3, \pi);
    \filldraw [Maroon!50!black] (\pi, \pi) circle (1.7pt);
  }
}
