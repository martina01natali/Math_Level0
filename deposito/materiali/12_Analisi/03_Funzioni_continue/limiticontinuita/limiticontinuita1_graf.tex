% (c) 2016 Daniele Zambelli - daniele.zambelli@gmail.com
% 
% Tutti i grafici per il capitolo relativo alle funzioni continue
%

%%%%%%%%%%%%%%%%%%%%%%%%%%%%%%%%%%%%%%%%%%%%%%%%%%%%%%%%%%%%%%%
%
% Welcome to Overleaf --- just edit your LaTeX on the left,
% and we'll compile it for you on the right. If you give
% someone the link to this page, they can edit at the same
% time. See the help menu above for more info. Enjoy!
%
% Note: you can export the pdf to see the result at full
% resolution.
%
%%%%%%%%%%%%%%%%%%%%%%%%%%%%%%%%%%%%%%%%%%%%%%%%%%%%%%%%%%%%%%%
% Drawing a tree several times using the PGF 3.0 pic feature
% Authors: Kate, Ignasi
% \documentclass[tikz]{standalone}
%%%<
% \usepackage{verbatim}
%%%>
% \begin{comment}
% :Title: Drawing trees
% :Tags: Random;Pic;PGF 3.0;Decorative drawings
% :Author: Ignasi
% :Slug: tree-pic
% 
% PGF 3.0 introduces `pic`, a "short picture" which can be inserted
% in any place of a tikzpicture.
% 
% From the manual: A “pic” is a “short picture” (hence the short name...) that
% can be inserted anywhere in TikZ picture where you could also insert a node.
% Similarly to nodes, pics have a “shape” (called `type` to avoid confusion) that
% someone has defined. Each time a pic of a specified type is used, the type’s
% code is executed, resulting in some drawings to be added to the current picture.
% The syntax for adding nodes and adding pics to a picture are also very similar.
% The core difference is that pics are typically more complex than nodes and may
% consist of a whole bunch of nodes themselves together with complex paths
% joining them.
% 
% Here we use the pic feature for drawing a tree several times.
% 
% The code was written by Kate and Ignasi and published on TeX.SE.
% 
% \usetikzlibrary{decorations.pathmorphing,calc,shapes,shapes.geometric,patterns}
% \tikzset{
%   treetop/.style = {decoration={random steps, segment length=0.4mm}, decorate},
%   trunk/.style   = {decoration={random steps, segment length=2mm,
%                     amplitude=0.2mm}, decorate}}
% 
% \tikzset{
%    my tree/.pic={
%      \foreach \w/\f in {0.3/30,0.2/50,0.1/70} {
%        \fill [brown!\f!black, trunk] (-\w/2,0) rectangle +(\w,3);
%      }
%      \foreach \n/\f in {1.4/40,1.2/50,1/60,0.8/70,0.6/80,0.4/90} {
%        \fill [green!\f!black, treetop](0,3) ellipse (\n/1.5 and \n);
%      }
%    }
% }
% \begin{document}
% \begin{tikzpicture}
%   \shade[bottom color=cyan!60!black, top color=blue!20!white] (0,0)
%     rectangle (10,10);
%   \pic at (2,2)[rotate=45] {my tree};
%   \pic at (4,2.5)  {my tree};
%   \pic at (6,1.75) {my tree};
% \end{tikzpicture}
% \end{document}
% %--------------------------
% Nuovi comandi grafici:

%--------------------------
% Inizializza Tikz con unità di misura 5mm, e realizza un disegno:
\newcommand{\disegno}[1]{
\begin{tikzpicture}[x=5mm, y=5mm, smooth]
  #1
\end{tikzpicture}
}

\newcommand{\vettoreNOTEXTdraw}[3]{% Disegna un vettore.
  \def \inizio{#1}
  \def \fine{#2}
  \def \parametri{#3}
  \draw [#3] [->] (\inizio) --++ (\fine);
}
% Esempio
% 

\newcommand{\vettoredraw}[4]{% Disegna un vettore con un nome.
  \def \inizio{#1}
  \def \fine{#2}
  \def \etichetta{#3}
  \def \parametri{#4}
  \coordinate [label=$\Vec{\etichetta}$] (v) at ($((\inizio)+0.5*(\fine)$);
  \draw [#4] [->] (\inizio) --++ (\fine);
}
% Esempio
% 

\newcommand{\vettoreABdraw}[5]{% Disegna un vettore con un nome.
  \def \etichettainizio{#1}
  \def \inizio{#2}
  \def \etichettafine{#3}
  \def \fine{#4}
  \def \parametri{#5}
  \coordinate [label=$\Vec{\etichettainizio\etichettafine}$] (N) at 
              ($(0.5*(\inizio)+0.5*(\fine)$);
  \coordinate [label=$\etichettainizio$] (A) at (\inizio);
  \coordinate [label=$\etichettafine$] (B) at (\fine);
  \draw [#5] [->] (\inizio) -- (\fine);
}
% Esempio
% 

\newcommand{\rcom}[5]{% Riferimento Cartesiano Ortogonale Monometrico: x-y.
  %% \disegno{\RCOM{-7}{+7}{-11}{+10}{gray!50, very thin, step=1}}
% \begin{tikzpicture}[x=5mm, y=5mm, smooth]
  \def \xmi{#1}
  \def \xma{#2}
  \def \ymi{#3}
  \def \yma{#4}
  \def \griglia{#5}
\draw[#5] (\xmi-0.3, \ymi-0.3) grid (\xma+0.3, \yma+0.3); % Griglia
\begin{scope}[-{Stealth[length=2mm, open, round]}, black] % Assi
 \draw (\xmi-0.3, 0) -- (\xma+0.3, 0) node [below] {$x$};
 \draw (0, \ymi-0.3) -- (0, \yma+0.3) node [left] {$y$};
\end{scope}
% \end{tikzpicture}

% \newcommand{\RCOM}[5]{% Coordinate cartesiane ortogonali: x-y.
% \begin{tikzpicture}[x=5mm, y=5mm, smooth]
%   \def \xmi{#1}
%   \def \xma{#2}
%   \def \ymi{#3}
%   \def \yma{#4}
%   \def \griglia{#5}
%   \coordinate [label=$x$] (x) at (\xma, 0);     
%   \draw [-{Stealth[length=2mm, open, round]}] (\xmi, 0) -- (\xma, 0);
%   \coordinate [label=$y$] (y) at (0, \yma);	
%   \draw [-{Stealth[length=2mm, open, round]}] (0, \ymi) -- (0, \yma);
%   \draw [#5] (\xmi, \ymi) grid (\xma, \yma);
% \end{tikzpicture}
}
% Esempio
% 

\newcommand{\rcomvario}[7]{% Rif. Cart. Ort. Mon.: v_ind-v_dip.
  %% \disegno{\RCOMvario{0}{6}{0}{6}{gray!50, very thin, step=1}{t}{v}}
  \def \xmi{#1}
  \def \xma{#2}
  \def \ymi{#3}
  \def \yma{#4}
  \def \griglia{#5}
  \def \vind{#6}
  \def \vdip{#7}
  \draw [#5] (\xmi, \ymi) grid (\xma, \yma);
  \coordinate [label=$\vind$] (x) at (\xma, 0);     
  \draw [-{Stealth[length=2mm, open, round]}] (\xmi, 0) -- (\xma, 0);
  \coordinate [label=$\vdip$] (y) at (0, \yma);     
  \draw [-{Stealth[length=2mm, open, round]}] (0, \ymi) -- (0, \yma);
}
Esempio


\newcommand{\testodraw}[3]{% Scrive un testo.
	\def \testo{#1}
	\def \posizione{#2}
	\def \parametri{#3}
	\coordinate [label={ #3 $\testo$}] (T) at (\posizione);
}
% Esempio
% 

\newcommand{\angolodraw}[4]{% Disegna un arco.
	\def \centro{#1}
	\def \angoloi{#2}
	\def \raggio{#3}
	\def \angolof{#4}
	\draw (\centro) ++(\angoloi:\angolof) arc (\angoloi:\raggio:\angolof);
}
% Esempio
% 

\newcommand{\molla}[4]{% Disegna una molla.
	\def \posizione{#1}
	\def \lunghezza{#2}
	\def \ampiezza{#3}
	\def \angolo{#4}
	\draw [rotate=\angolo, shift={(\posizione)}]
	(0,-0.5*\ampiezza) --(0,0.5*\ampiezza);
	\draw plot [variable=\t, domain=45:7245, smooth, rotate=\angolo, 
	shift={(\posizione)}] 
	({\lunghezza*0.0001388*\t}, {\ampiezza*(sin(\t)^2-0.5)});
}
% Esempio
% 
% \end{comment}











\begin{comment}


\newcommand{\telescopio}{% 
    % Telescopio per iperreali.
    \disegno{
%       \draw (0, 0) -- (5, 0) (0, 1) -- (5, 1);
%       \draw [fill=black!50, ultra thick] (5, 0) -- (6.5, 0) (5, 1) -- (6.5, 1);
%       \draw (5, 0) -- (6.5, 0) (5, 1) -- (6.5, 1);
    }
}

\newcommand{\iperinteri}{% 
    % Alcuni punti di 2^x.
    \disegno{
      \rcom{-10}{+10}{-10}{10}{gray!50, very thin, step=1}
      \foreach \pi in {-10, -9,...,+9}
      \filldraw [Maroon!50!black, ultra thick] (\pi, \pi) circle (2pt)
                                               (\pi, \pi) -- (\pi+1, \pi) ;
    
    }
}
\end{comment}

