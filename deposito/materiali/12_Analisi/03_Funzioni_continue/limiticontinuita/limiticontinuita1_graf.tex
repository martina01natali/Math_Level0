% (c) 2016 Daniele Zambelli - daniele.zambelli@gmail.com

\newcommand{\graficoa}{% 
  \def \funzione{(x**2-6*x+5)/(x**2+2*x-3)}
  \disegno{
    \rcom{-18}{+15}{-10}{+10}{gray!50, very thin, step=1}
    \tkzInit[xmin=-18.3,xmax=+15.3,ymin=-10.3,ymax=+10.3]
    \tkzFct[domain=-18.3:-3.1, ultra thick, color=Maroon!50!black]
         {\funzione}
    \tkzFct[domain=-2.9:+15.3, ultra thick, color=Maroon!50!black]
         {\funzione}
  }
}

\newcommand{\graficob}{%7
  \disegno{
    \rcom{-5}{+7}{-7}{+7}{gray!50, very thin, step=1}
    \tkzInit[xmin=-5.3,xmax=+7.3,ymin=-7.3,ymax=+7.3]
    \tkzFct[domain=-5.3:2, ultra thick, color=Maroon!50!black]
         {.5*x - 2}
    \tkzFct[domain=2:+7.3, ultra thick, color=Maroon!50!black]
         {x**2-6*x+7}
  }
}








\begin{comment}


\newcommand{\telescopio}{% 
    % Telescopio per iperreali.
    \disegno{
%       \draw (0, 0) -- (5, 0) (0, 1) -- (5, 1);
%       \draw [fill=black!50, ultra thick] (5, 0) -- (6.5, 0) (5, 1) -- (6.5, 1);
%       \draw (5, 0) -- (6.5, 0) (5, 1) -- (6.5, 1);
    }
}

\newcommand{\iperinteri}{% 
    % Alcuni punti di 2^x.
    \disegno{
      \rcom{-10}{+10}{-10}{10}{gray!50, very thin, step=1}
      \foreach \pi in {-10, -9,...,+9}
      \filldraw [Maroon!50!black, ultra thick] (\pi, \pi) circle (2pt)
                                               (\pi, \pi) -- (\pi+1, \pi) ;
    
    }
}
\end{comment}

