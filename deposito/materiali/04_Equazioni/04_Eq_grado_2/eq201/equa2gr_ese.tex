% (c)~2014 Claudio Carboncini - claudio.carboncini@gmail.com
% (c)~2014 Dimitrios Vrettos - d.vrettos@gmail.com
% (c) 2015 Daniele Zambelli daniele.zambelli@gmail.com

\section{Esercizi}

\subsection{Esercizi dei singoli paragrafi}

% \subsubsection*{\numnameref{sec:01_}}

% \begin{esercizio}
% \label{ese:D.1}
% testo esercizio
% \end{esercizio}
% 
% \begin{esercizio}\label{ese:03.1}
% Consegna:
%  \begin{enumeratea}
%   \item  
%  \end{enumeratea}
% \end{esercizio}
% 
% \subsection{Esercizi riepilogativi}
% 
% \begin{esercizio}
% \label{ese:D.2}
% testo esercizio
% \end{esercizio}
% 
% \begin{esercizio}\label{ese:03.2}
% Consegna:
%  \begin{enumeratea}
%   \item  
%  \end{enumeratea}
% \end{esercizio}

\begin{esercizio}[\Ast]
 \label{ese:3.1}
Risolvi le seguenti equazioni pure e confronta i risultati con quelli 
trovati dai tuoi compagni.
\begin{multicols}{3}
 \begin{enumeratea}
 \item~\(x^{2}-1 = 0\)
 \item~\(x^{2}=\dfrac{49}{25}\)
 \item~\(2x^{2} - 32 = 0\)
 \item~\(x^{2}-25=0\)
 \item~\(16 x^{2}=1\)
 \item~\(3x^{2}+3=0\)
 \item~\(x^{2}-9=0\)
 \item~\(25=9 x^{2}\)
 \item~\(x^{2} - 3 = 0\)
 \item~\(x^{2} + 36 = 0\)
 \item~\(4 - x^{2} = 0\)
 \item~\(x^{2} + 4 = 0\)
 \item~\(x^{2} = 49\)
 \item~\(4 - 9 x^{2} = 0\)
 \item~\(5 x^{2} - 3 = 0\)
 \item~\(4 x^{2} - 9 = 0\)
 \item~\(9 x^{2} - 25 = 0\)
 \item~\(6 x^{2} = 0\)
 \item~\(2 x^{2} - 1 = 0\)
 \item~\(4 x^{2} + 16 = 0\)
 \item~\(1 + x^{2} = 50\)
 \item~\(3 x^{2} - 1 = 0\)
 \item~\(27 x^{2} - 3 = 0\)
%  \item~\(7 x^{2} = 28\)
 \end{enumeratea}
 \end{multicols}
\end{esercizio}

% \begin{esercizio}[\Ast]
%  \label{ese:3.3}
% Risolvi le seguenti equazioni di secondo grado pure.
% \begin{multicols}{3}
%  \begin{enumeratea}
%  \item~\(4 x^{2} - 4 = 0\) \hfill\(\left[...\right]\)
%  \item~\(5 x^{2} - 125 = 0\) \hfill\(\left[...\right]\)
%  \item~\(0,04 x^{2} = 1\) \hfill\(\left[x_{1,2} = \pm 5\right]\)
%  \item~\(x^{2} - 0,01 = 0\) \hfill\(\left[...\right]\)
%  \item~\(0,5 x^{2} - 4,5 = 0\) \hfill\(\left[...\right]\)
%  \item~\(0,09 x^{2} = 0,01\) \hfill\(\left[x_{1,2} = \pm \dfrac{1}{3}\right]\)
%  \item~\(\dfrac{1}{2} x^{2} - 2 = 0\) \hfill\(\left[...\right]\)
%  \item~\(x^{2} - \dfrac{9}{4} = 0\) \hfill\(\left[...\right]\)
%  \item~\(x^{2} - \dfrac{1}{6} = 0\) \hfill\(\left[...\right]\)
%  \item~\(121 x^{2} - \dfrac{1}{169} = 0\) 
%   \hfill\(\left[x_{1,2} = \pm \dfrac{\sqrt{6}}{6}\right]\)
%  \item~\(x^{2} + \dfrac{9}{4} = 0\) \hfill\(\left[...\right]\)
%  \item~\(4 \left(x^{2}-\dfrac{3}{4}\right)= 13\) 
%   \hfill\(\left[x_{1,2} = \pm 2\right]\)
%  \end{enumeratea}
%  \end{multicols}
% \end{esercizio}

% \begin{esercizio}[\Ast]
%  \label{ese:3.4}
% Risolvi le seguenti equazioni di secondo grado pure.
% \begin{multicols}{2}
%  \begin{enumeratea}
%  \item~\(x^{2} - \sqrt{3} = 0\) \hfill\(\left[...\right]\)
%  \item~\(- 9 x^{2} = - 1\) \hfill\(\left[...\right]\)
%  \item~\(4 x^{2} = - 9\) \hfill\(\left[\emptyset\right]\)
%  \item~\(x^{2} + 6 = 42\) \hfill\(\left[...\right]\)
%  \item~\(5 - 125 x^{2} = 0\) \hfill\(\left[...\right]\)
%  \item~\(18 - x^{2} = 0\) \hfill\(\left[x_{1,2} = \pm 3 \sqrt{2}\right]\)
%  \item~\((x + 3)^{2} = 6 x + 34\) \hfill\(\left[...\right]\)
%  \item~\((x + 1)^{2} = 25\) \hfill\(\left[...\right]\)
%  \item~\((x - \sqrt{3}) (x + \sqrt{3}) = 13\) 
%   \hfill\(\left[x_{1,2} = \pm \sqrt{10}\right]\)
%  \item~\((x + \sqrt{2})^{2} = 2 \sqrt{2} x\) \hfill\(\left[...\right]\)
%  \item~\((x - 2)^{2} + (1 - x)^{2} = 1 - 6x\) \hfill\(\left[...\right]\)
%  \item~\((\sqrt{2} x - \sqrt{3}) (\sqrt{2} x + \sqrt{3}) = 0\) 
%   \hfill\(\left[x_{1,2} = \pm \dfrac{\sqrt{6}}{2}\right]\)
%  \end{enumeratea}
%  \end{multicols}
% \end{esercizio}

\begin{esercizio}[\Ast]
\label{ese:3.5}
Risolvi le seguenti equazioni spurie e confronta i risultati con quelli 
trovati dai tuoi compagni.
\begin{multicols}{3}
 \begin{enumeratea}
 \item~\(x^{2} - 3 x = 0\)
 \item~\(3 x^{2} - 2 x = 0\)
 \item~\(7 x^{2} + 2 x = 0\) 
%  \item~\(x^{2} + 2 x = 0\)
 \item~\(x^{2} + 5 x = 0\)
 \item~\(x^{2} - x = 0\)
 \item~\(18 x^{2} - 36 x = 0\)
 \item~\(2x^{2} + \dfrac{1}{6}x = 0\)
 \item~\(x^{2} + \dfrac{1}{2} x = 0\)
 \item~\(1000 x - 2000 x^{2} = 0\)
 \item~\(9x^{2} + 16x = 0\)
 \item~\(5x = 25x^{2}\)
 \item~\(3 x^{2} - 2 x = 4 x\)
 \item~\(81x^{2} = 9x\)
 \item~\(7x^{2} - 2x = 0\)
 \item~\(\dfrac{1}{10} x^{2} - \dfrac{1}{2} x = 0\)
 \item~\(\dfrac{11}{3} x^{2} = - 2 x\)
 \item~\(0,5 x^{2} + 0,1 x = 0\)
 \item~\(6 x^{2} = 5 x\)
 \item~\((x - 1) (x + 3) = 3 x^{2} - 3\)
 \item~\(x^{2} + \sqrt{2} x = 0\) 
 \item~\((3 x - 2)^{2} - 4 = 6 x^{2}\)
 \item~\(5 \sqrt{2} x^{2} - 2 \sqrt{2} x = 0\) 
 \item~\(\dfrac{1}{2} x - \dfrac{1}{4} x^{2} = 0\)
 \item~\(\dfrac{1}{2} ( x - 2 )^{2} - x = 2\)
 \end{enumeratea}
 \end{multicols}
\end{esercizio}

% \begin{esercizio}[\Ast]
%  \label{ese:3.8}
% Risolvi le seguenti equazioni di secondo grado spurie.
% \begin{multicols}{3}
%  \begin{enumeratea}
%  \item~\(\sqrt{2} x^{2} + \sqrt{3} x = 0\) \hfill\(\left[...\right]\)
%  \item~\(- 2x^{2} + 4x = 0\) \hfill\(\left[...\right]\)
%  \item~\(\dfrac{1}{6} x^{2} + \dfrac{1}{4} x = 0\) \hfill\(\left[...\right]\)
%  \end{enumeratea}
%  \end{multicols}
% \end{esercizio}
% 
% \begin{esercizio}[\Ast]
%  \label{ese:3.9}
% Risolvi le seguenti equazioni di secondo grado spurie.
% \begin{multicols}{3}
%  \begin{enumeratea}
%  \item~\(3x^{2} - \dfrac{4}{3} x = 0\) 
%   \hfill\(\left[x_{1} = 0 \vee x_{2} = \dfrac{4}{9}\right]\)
%  \item~\((x - 2)^{2} = 4\) \hfill\(\left[...\right]\)
%  \item~\((x + 1)^{2} = 1\) 
%   \hfill\(\left[x_{1} = 0 \vee x_{2} = - 2\right]\)
%  \item~\((x + \sqrt{2})^{2} = 2\) \hfill\(\left[...\right]\)
%  \item~\(77 x - 11 x^{2} = 0\) 
%   \hfill\(\left[x_{1} = 0 \vee x_{2} = 7\right]\)
%  \item~\(\dfrac{3}{4} x^{2} - \dfrac{3}{2} x = 0\) \hfill\(\left[...\right]\)
%  \end{enumeratea}
%  \end{multicols}
% \end{esercizio}
% % 
% \begin{esercizio}[\Ast]
%  \label{ese:3.11}
% Risolvi le seguenti equazioni di secondo grado spurie.
%  \begin{enumeratea}
%  \item~\((x - 2)^{2} + (1 - x)^{2} = 5\)
%   \hfill\(\left[...\right]\)
%  \item~\((x -2)^{3} -4 (2 x -1) = (x +2) \left(x^{2} -2 x +4\right) -12\)
%   \hfill\(\left[...\right]\)
%  \item~\((\sqrt{2} + x)^{3} - (\sqrt{3} + x)^{3} = 2 \sqrt{2} - 3\sqrt{3}\)
%   \hfill\(\left[x_{1} = 0 \vee x_{2} = - (\sqrt{2} + \sqrt{5})\right]\)
%  \item~\((\sqrt{2} x - \sqrt{3}) (\sqrt{2} x + \sqrt{3}) + (\sqrt{3} x + 
% \sqrt{3})^{2} + (x - 1)^{2} = 1\)
%   \hfill\(\left[...\right]\)
%  \item~\(\left(x^{2} + \sqrt{2} \right) (\sqrt{3} - 1) + (2 x +\sqrt{3}) 
% (\sqrt{2} - 1) - \sqrt{2} + \sqrt{3} = 0\)
%   \hfill\(\left[...\right]\)
%  \end{enumeratea}
% \end{esercizio}

% \subsection*{3.2 - Risoluzione di un'equazione completa}
\subsection*{\numnameref{sec:eq2gr_completa}}

\begin{esercizio}[\Ast]
 \label{ese:3.12}
Risolvi le seguenti equazioni di secondo grado complete.
\begin{multicols}{2}
 \begin{enumeratea}
 \item~\(x^{2}-5 x + 6=0\)
  \hfill\(\left[x_{1} = 2 \vee x_{2} = 3\right]\)
 \item~\(x^{2} + x-20=0\)
  \hfill\(\left[x_{1} =-5 \vee x_{2} = 4\right]\)
 \item~\(2 x^{2}-6 x-6=0\)
  \hfill\(\left[x_{1,2} = \dfrac{3 \pm \sqrt{21}}{2}\right]\)
 \item~\(x^{2}-3 x + 6=0\)
  \hfill\(\left[\emptyset\right]\)
 \item~\(- x^{2} + x + 42=0\)
  \hfill\(\left[x_{1} =-6 \vee x_{2} = 7\right]\)
 \item~\(- x^{2} + 10 x-25=0\)
  \hfill\(\left[x_{1} = x_{2} = 5\right]\)
 \item~\(- 2 x^{2} + 7 x-5=0\)
  \hfill\(\left[x_{1} = 1 \vee x_{2} = \dfrac{5}{2}\right]\)
 \item~\(3 x^{2} + 2 x-1=0\)
  \hfill\(\left[x_{1} =-1 \vee x_{2} = \dfrac{1}{3}\right]\)
 \item~\(x^{2}-4 x + 9 = 0\)
  \hfill\(\left[\emptyset\right]\)
 \item~\(x^{2}-4 x-9 = 0\)
  \hfill\(\left[x_{1,2} = 2 \pm \sqrt{13}\right]\)
 \item~\(2 x^{2}-\sqrt{5} x-1 = 0\)
  \hfill\(\left[x_{1,2} = \dfrac{\sqrt{5} \pm \sqrt{13}}{4}\right]\)
 \item~\(x^{2}-3 x-2=0\)
  \hfill\(\left[x_{1,2} = \dfrac{3 \pm \sqrt{17}}{2}\right]\)
 \item~\(x^{2}-5 x + 3 = 0\)
  \hfill\(\left[x_{1,2} = \dfrac{5 \pm \sqrt{13}}{2}\right]\)
 \item~\(x^{2}-2 \sqrt{3} x-4=0\)
  \hfill\(\left[x_{1,2} = \sqrt{3} \pm \sqrt{7}\right]\)
 \item~\(x^{2} + 6 x-2 = 0\)
  \hfill\(\left[x_{1,2} =-3 \pm \sqrt{11}\right]\)
 \item~\(- x^{2} + 4 x-7=0\)
  \hfill\(\left[\emptyset\right]\)
%  \item~\(2 x^{2}-\sqrt{5} x-1=0\)
%   \hfill\(\left[x_{1} =-\sqrt{2} \vee x_{2} = \dfrac{3 \sqrt{2}}{2}\right]\)
 \item~\(- \dfrac{4}{3} x^{2}-x + \dfrac{3}{2}=0\)
  \hfill\(\left[x_{1} =-\dfrac{3}{2} \vee x_{2} = \dfrac{3}{4}\right]\)
 \item~\(- \dfrac{4}{5} x^{2} + \dfrac{1}{2} x-\dfrac{1}{20}=0\)
  \hfill\(\left[x_{1} = \dfrac{1}{8} \vee x_{2} = \dfrac{1}{2}\right]\)
%  \item~\(x^{2}-\sqrt{5} x-\sqrt{5}=0\)
%   \hfill\(\left[x_{1,2} = \dfrac{\sqrt{5} \pm \sqrt{5 + 4 
% \sqrt{5}}}{2}\right]\)
%  \item~\(x^{2} + (\sqrt{2}-\sqrt{3}) x-\sqrt{6} = 0\)
%   \hfill\(\left[x_{1} =-\sqrt{2} \vee x_{2} = \sqrt{3}\right]\)
 \item~\((x-2) (3-2 x) = x-2\)
  \hfill\(\left[x_{1} = 1 \vee x_{2} = 2\right]\)
 \end{enumeratea}
 \end{multicols}
\end{esercizio}

\begin{esercizio}[\Ast]
 \label{ese:3.17}
Risolvi le seguenti equazioni di secondo grado complete.
\begin{multicols}{2}
 \begin{enumeratea}
 \item~\((x + 5)^{2} = 5 (4 x + 5)\)
  \hfill\(\left[x_{1} = 0 \vee x_{2} = 10\right]\)
 \item~\(x^{2} + 6 x-3 = 0\)
  \hfill\(\left[x_{1,2} =-3 \pm 2 \sqrt{3}\right]\)
 \item~\(x^{2}-3 x-\dfrac{5}{2} = 0\)
  \hfill\(\left[x_{1,2} = \dfrac{3 \pm \sqrt{19}}{2}\right]\)
 \item~\(2 x^{2}-3 x + 1 = 0\)
  \hfill\(\left[x_{1} = 1 \vee x_{2} = \dfrac{1}{2}\right]\)
 \item~\(\dfrac{4}{3} x^{2}-\dfrac{1}{3} x-1 = 0\)
  \hfill\(\left[x_{1} = 1 \vee x_{2} =-\dfrac{3}{4}\right]\)
 \item~\(3 x^{2} + x-2 = 0\)
  \hfill\(\left[x_{1} =-1 \vee x_{2} = \dfrac{2}{3}\right]\)
 \item~\(3 x^{2}-\dfrac{2}{3} x-1 = 0\)
  \hfill\(\left[x_{1,2} = \dfrac{1 \pm 2 \sqrt{7}}{9}\right]\)
%  \item~\(\sqrt{2} x^{2}-x-3 \sqrt{2} = 0\)
%   \hfill\(\left[x_{1} =-\sqrt{2};~x_{2} = \dfrac{3 \sqrt{2}}{2}\right]\)
%  \item~\((3 x + 1)^{2}-(2 x + 2)^{2} = 0\)
%   \hfill\(\left[x_{1} =-\dfrac{3}{5} \vee x_{2} = 1\right]\)
%  \item~\((x + 200)^{2} + x + 200 = 2\)
%   \hfill\(\left[x_{1} =-202 \vee x_{2} =-199\right]\)
 \item~\(3 x^{2}-2 x-2 = 0\)
  \hfill\(\left[x_{1,2} = \dfrac{1 \pm \sqrt{7}}{3}\right]\)
 \item~\(4 x^{2}-8 x + 3 = 0\)
  \hfill\(\left[x_{1} = \dfrac{1}{2} \vee x_{2} = \dfrac{3}{2}\right]\)
%  \item~\(x^{2}-(\sqrt{2} + \sqrt{3}) x + \sqrt{6} = 0\)
%   \hfill\(\left[x_{1} = \sqrt{2} \vee x_{2} = \sqrt{3}\right]\)
%  \item~\((x^{2} + x + 1) (x^{2}-x-1) = (x^{2}-1)^{2}\)
%   \hfill\(\left[x_{1,2} = 1 \pm \sqrt{3}\right]\)
%  \item~\(7 x^{2}-2 x-5 = 0\)
%   \hfill\(\left[x_{1} = 1 \vee x_{2} =-\dfrac{5}{7}\right]\)
 \end{enumeratea}
 \end{multicols}
\end{esercizio}

\begin{esercizio}[\Ast]
\label{ese:3.21}
Risolvi, applicando quando possibile la formula ridotta o ridottissima.
\begin{multicols}{2}
 \begin{enumeratea}
 \item~\(40 x^{2} + 80 x-30 = 0\)
  \hfill\(\left[x_{1,2} = \dfrac{- 2 \pm \sqrt{7}}{2}\right]\)
 \item~\(5 x^{2}-4 x + 1 = 0\)
  \hfill\(\left[\emptyset\right]\)
 \item~\(5 x^{2}-4 x-9 = 0\)
  \hfill\(\left[x_{1} =-1 \vee x_{2} = \dfrac{9}{5}\right]\)
 \item~\(\dfrac{3}{2} x^{2} + 2 x-\dfrac{3}{4} = 0\)
  \hfill\(\left[x_{1,2} = \dfrac{- 4 \pm \sqrt{34}}{6}\right]\)
 \item~\(6 x^{2}-4 x-2 = 0\)
  \hfill\(\left[x_{1} = 1 \vee x_{2} =-\dfrac{1}{3}\right]\)
%  \item~\(90 x^{2}-180 x-270 = 0\)
%   \hfill\(\left[x_{1} = 3 \vee x_{2} =-1\right]\)
 \item~\(\dfrac{3}{2} x^{2}-4 x + 2 = 0\)
  \hfill\(\left[x_{1} = 2 \vee x_{2} = \dfrac{2}{3}\right]\)
 \item~\(\dfrac{4}{3} x^{2}-6 x + 6 = 0\)
  \hfill\(\left[x_{1} = 3 \vee x_{2} = \dfrac{3}{2}\right]\)
 \item~\(x^{2}-6 x + 1 = 0\)
  \hfill\(\left[x_{1,2} = 3 \pm 2 \sqrt{2}\right]\)
 \item~\(3 x^{2}-12 x-3 = 0\)
  \hfill\(\left[x_{1,2} = 2 \pm \sqrt{5}\right]\)
 \item~\(7 x^{2}-6 x + 8 = 0\)
  \hfill\(\left[\emptyset\right]\)
 \item~\(3 x^{2}-18 x + 27 = 0\)
  \hfill\(\left[x_{1,2} =3\right]\)
 \item~\(9 x^{2}-12 x + 4 = 0\)
  \hfill\(\left[x_{1,2} = \dfrac{2}{3}\right]\)
 \item~\(4 x^{2}-32 x + 16 = 0\)
  \hfill\(\left[x_{1,2} = 4 \pm 2 \sqrt{3}\right]\)
 \item~\(3 x^{2} + 10 x + 20 = 0\)
  \hfill\(\left[\emptyset\right]\)
 \end{enumeratea}
 \end{multicols}
\end{esercizio}

% \subsection*{Altri esercizi sulle equazioni di 2° grado}

\begin{esercizio}[\Ast]
\label{ese:3.25}
Risolvi le seguenti equazioni di secondo grado.
\begin{multicols}{2}
 \begin{enumeratea}
 \item~\(3 x-x^{2} = x^{2} + 3 (x-2)\)
  \hfill\(\left[x_{1,2} = \pm \sqrt{3}\right]\)
 \item~\(2 (x-1) (x + 1) = 2\)
  \hfill\(\left[x_{1,2} =\pm \sqrt{2}\right]\)
 \item~\((2 x-1) (4-x)-11 x = (1-x)^{2}\)
  \hfill\(\left[\emptyset\right]\)
 \item~\(2x^{2} = x + x^{2}-(x + \sqrt{x}) (x-\sqrt{x})\)
  \hfill\(\left[...\right]\)
 \item~\((x-3)^{2} = 9-6 x\)
  \hfill\(\left[x_{1,2}= 0\right]\)
 \item~\(\dfrac{3 x-2}{2} = x^{2}-2\)
  \hfill\(\left[x_{1} = 2 \vee x_{2} =-\dfrac{1}{2}\right]\)
 \item~\((2 x-3) (2 x + 3) = 27\)
  \hfill\(\left[x_{1,2} =\pm 3\right]\)
 \item~\(\dfrac{x-3}{2}-\dfrac{x^{2} + 2}{3} = 1 + x\)
  \hfill\(\left[\emptyset\right]\)
 \item~\((x-2)^{3}-x^{3} = x^{2}-4\)
  \hfill\(\left[x_{1,2} = \dfrac{6 \pm 2 \sqrt{2}}{7}\right]\)
 \item~\(\dfrac{(x-1)^{2}}{2}-\dfrac{2 x-5}{3} =-\dfrac{5}{3} x\)
  \hfill\(\left[\emptyset\right]\)
 \item~\((2-x)^{3}-(2-x)^{2} = \dfrac{3-4 x^{3}}{4}\)
  \hfill\(\left[\emptyset\right]\)
 \end{enumeratea}
 \end{multicols}
\end{esercizio}

\begin{esercizio}[\Ast]
 \label{ese:3.30}
Risolvi le seguenti equazioni di secondo grado.
 \begin{enumeratea}
 \item~\(9 x^{2} + 12 x + 1 = 0\)
  \hfill\(\left[x_{1,2} = \dfrac{- 2 \pm \sqrt{3}}{3}\right]\)
 \item~\(\dfrac{x-2}{3}-(3 x + 3)^{2} = x\)
  \hfill\(\left[x_{1} =-1 \vee x_{2} =-\dfrac{29}{27}\right]\)
 \item~\((3 x + 1) \left(\dfrac{5}{2} + x \right) = 2 x-1\)
  \hfill\(\left[x_{1} =-1 \vee x_{2} =-\dfrac{7}{6}\right]\)
 \item~\((3 x-2)^{2} + (5 x-1)^{2} = (3 x-2) (5 x-1)\)
  \hfill\(\left[\emptyset\right]\)
 \item~\((x-2)^{3}-1 = x^{3} + 12 x-11\)
  \hfill\(\left[x_{1,2} = \dfrac{\pm \sqrt{3}}{3}\right]\)
 \item~\(x (1-5 x) = [ 3-(2 + 5 x) ] x-(x^{2}-1)\)
  \hfill\(\left[x_{1,2} =\pm 1\right]\)
 \item~\((x + 1)^{3}-(x + 2)^{2} = \dfrac{2 x^{3}-1}{2}\)
  \hfill\(\left[x_{1,2} = \dfrac{1 \pm \sqrt{21}}{4}\right]\)
 \item~\((x + 2)^{3} + 4 x^{2} = (x-2)^{3} + 16\)
  \hfill\(\left[x_{1} = x_{2} = 0\right]\)
 \item~\(3 \left(x +\sqrt{2} \right)^{2}-18 \left(x +\sqrt{2}\right) +27 = 0\)
  \hfill\(\left[x_{1,2}= 3-\sqrt{2}\right]\)
 \item~\((4-3 x)^{3} + 27 x^{3} = 64 + 24 x\)
  \hfill\(\left[x_{1} = 0 \vee x_{2} = \dfrac{14}{9}\right]\)
 \item~\(\left(\dfrac{x-1}{3}-\dfrac{x}{6} \right)^{2} = (x + 1)^{2}\)
  \hfill\(\left[x_{1} =-\dfrac{8}{5} \vee x_{2} =-\dfrac{4}{7}\right]\)
 \item~\((\sqrt{3} x + 1)^{2} + (\sqrt{3} x-1)^{2}-3 (\sqrt{3}x + 1) 
(\sqrt{3} x-1) = 0\)
  \hfill\(\left[x_{1,2} = \pm \sqrt{\dfrac{5}{3}}\right]\)
 \item~\(\dfrac{(2 x + 1) (x-2)}{3} + \dfrac{(x + \sqrt{5}) (x -\sqrt{5})}{2} 
= \dfrac{(x-1)^{2}}{6}\)
  \hfill\(\left[x_{1,2} = \dfrac{1 \pm \sqrt{31}}{3}\right]\)
 \item~\(\left(\dfrac{1}{2} x +1 \right)^{3} = \left(\dfrac{1}{2} x-1\right) 
\left(\dfrac{1}{2} x + 1 \right)^{2}\)
  \hfill\(\left[x_{1,2}=-2\right]\)
 \item~\(\dfrac{(3 x-1)^{2}}{3}-\dfrac{(1-2 x)^{2}}{5} + \dfrac{3 x (x-1)}{5} 
+ \dfrac{(1 + x)^{2}}{3} = 0\)
  \hfill\(\left[\emptyset\right]\)
 \item~\(\dfrac{1}{\sqrt{10}} x^{2} + 1 = \left(\dfrac{1}{\sqrt{2}} 
+\dfrac{1}{\sqrt{5}} \right) x\)
  \hfill\(\left[...\right]\)
 \item~\((3 x-1)^{2} + (2 x + 1)^{2} = (3 x-1) (2 x + 1)\)
  \hfill\(\left[\emptyset\right]\)
 \item~\((x + 1)^{4}-(x + 1)^{3} = x^{3} (x + 4)-x (x + 1)^{2} + 3 x\)
  \hfill\(\left[x_{1} = 0 \vee x_{2} = \dfrac{1}{5}\right]\)
 \item~\(\left(\dfrac{1}{2} x^{2} + 1 \right)^{3} + \dfrac{1}{6} x^{3} = 
\left(\dfrac{1}{2} x^{2}-1 \right)^{3} + \dfrac{1}{6} (x + 1)^{3} + 
\dfrac{3}{2} x^{4}\)
  \hfill\(\left[x_{1,2} = \dfrac{- 3 \pm \sqrt{141}}{6}\right]\)
 \item~\(\dfrac{x-2}{2} \cdot \dfrac{x + 2}{3} + \dfrac{1}{3} 
\left[\dfrac{1}{2}-\left(x + \dfrac{1}{2} \right) \right] + 4 \left(x 
-\dfrac{1}{2} 
\right) \left(x + \dfrac{1}{2} \right) + \dfrac{5}{3} = 0\)
  \hfill\(\left[0;~\dfrac{2}{25}\right]\)
 \item~\((2-3 x)^{2}-1 = 8 (1-2 x) + (2 x + 1)^{2}-1\)
  \hfill\(\left[-1;~+1\right]\)
 \item~\(x^{2} + \left(\sqrt{3}-\sqrt{2} \right) x-\sqrt{6} = 0\)
  \hfill\(\left[-\sqrt{3};~ + \sqrt{2}\right]\)
%  \item~\(\dfrac{2 \sqrt{3} x + 1}{\sqrt{2}}-\left(x-\sqrt{3} \right)^{2} = 
% \dfrac{1-3 \sqrt{2} x}{\sqrt{2}} + \sqrt{3} x \left(\sqrt{2}+ 2 \right)\)
%   \hfill\(\left[...\right]\)
%  \item~\(\sqrt{3} (2 x-30)^{2}-2 \sqrt{27} (60-4 x) = 0\)
%   \hfill\(\left[...\right]\)
 \item~\(\left(2 x + \dfrac{1}{2} \right)^{2}-\dfrac{1}{2} 
        \left(\dfrac{1}{2} x-1 \right)^{2} + \left(x-\dfrac{1}{2} \right) 
        \left(x + \dfrac{1}{2} \right) = 0\)
  \hfill\(\left[-\dfrac{2}{3};~\dfrac{2}{13}\right]\)
 \end{enumeratea}
\end{esercizio}

% \begin{esercizio}[\Ast]
%  \label{ese:3.33}
% Risolvi le seguenti equazioni di secondo grado.
%  \begin{enumeratea}
%  \item~\(\dfrac{x^{2}-16}{9} + \dfrac{(x-1)^{2}}{3} = \dfrac{x (x -2)}{9} 
% + \left(x-\dfrac{5}{2} \right) \left(x + \dfrac{1}{3}\right)\)
%   \hfill\(\left[...\right]\)
%  \end{enumeratea}
% \end{esercizio}
% 
%  \begin{esercizio}[\Ast]
% \label{ese:3.34}
% Risolvi le seguenti equazioni di secondo grado.
%  \begin{enumeratea}
%  \item~\(\dfrac{(x-1) (x + 2)}{2} + \dfrac{(x + 2) (x-3)}{3} =\dfrac{(x-3) 
% (x + 4)}{6}\)
%   \hfill\(\left[...\right]\)
%  \item~\(\left(2 x-\dfrac{1}{2} \right)^{2} + \left(\dfrac{x-1}{2} 
% -\dfrac{x}{3} \right) x =-x^{2} + \dfrac{2}{3} 
% \left(x-\dfrac{1}{2}\right) x-\dfrac{1}{2} x + 
% \dfrac{1}{9}\)
%   \hfill\(\left[...\right]\)
%  \item~\(\dfrac{1}{4} (2 x-1)^{2}-\dfrac{1}{3} (x-1)^{2} +\dfrac{(x-2) (x 
% + 2)}{2}-\dfrac{1}{6} x + \dfrac{1}{6} = 0\)
%   \hfill\(\left[...\right]\)
%  \item~\(\dfrac{1}{2} (2 x-1) (x + 1) + \dfrac{1}{3} \left(x^{2}-5\right) + 
% 2 x (x-1) (x + 1) = 2 (x + 2)^{3}-(2 x-1)^{2}\)
%   \hfill\(\left[...\right]\)
%  \end{enumeratea}
% \end{esercizio}
% 
% \paragraph{3.33.} a)~\(\emptyset\),\quad b)~\(x_{1} = 9 \vee x_{2} = 
% 15\),\quad 
% c)~\(x_{1} =-\dfrac{2}{3} \vee x_{2} = \dfrac{2}{13}\),\quad d)~\(x_{1,2} = 
% \dfrac{31 \pm \sqrt{433}}{24}\)
% 
% \paragraph{3.34.} a)~\(x_{1,2} = \pm \dfrac{\sqrt{6}}{2}\),\quad b)~\(x_{1,2} 
% = \dfrac{10 \pm \sqrt{10}}{54}\),\quad c)~\(x_{1,2} = \dfrac{3 \pm 
% \sqrt{331}}{14}\),\quad d)~\(x_{1,2} = \dfrac{- 177 \pm \sqrt{14849}}{80}\)
% 
% \begin{esercizio}[\Ast]
% \label{ese:3.35}
% Risolvi le seguenti equazioni di secondo grado.
%  \begin{enumeratea}
%  \item~\(\dfrac{3 x-1}{\sqrt{5}-\sqrt{3}} + \dfrac{\left(x-\sqrt{3}\right) 
% \left(x + \sqrt{3} \right)}{\sqrt{3}}-\dfrac{\left(x -\sqrt{3} 
% \right)^{2}}{\sqrt{3}} = \dfrac{x^{2}}{\sqrt{5}-\sqrt{3}} + 2 x-2 
% \sqrt{3}\)
%   \hfill\(\left[...\right]\)
%  \item~\(\left(x + \dfrac{1}{2} \right)^{2}-\dfrac{3 x^{2}-7 x + 2}{2} - 
% \dfrac{x}{4} + \dfrac{5 x-13}{2} = \dfrac{2}{3} x (1-x) +\dfrac{73}{12} 
% x-\dfrac{15}{12}\)
%   \hfill\(\left[...\right]\)
%  \item~\(\dfrac{(x^{2} + 2 x + 1)^{2}}{4} + \dfrac{(x + 1)^{2}}{2}+ 
% \dfrac{(x^{4}-1)}{8}-(2 x^{2}-2 x + 1)^{2} + 9 x^{3}\left(\dfrac{3}{8} x-1 
% \right) 
% + \dfrac{1}{4} x^{2} (x^{2} + 20)= 0\)
%   \hfill\(\left[...\right]\)
%  \end{enumeratea}
% \end{esercizio}
% 
% \begin{esercizio}[\Ast]
%  \label{ese:3.36}
% Risolvi le seguenti equazioni con opportune sostituzioni.
% \begin{multicols}{2}
% \begin{enumeratea}
% \item\((4x + 3)^{2} = 25\)
% \item\((x-5)^{2} + 9 = 0\)
% \item\((3x-1)^{2}-36 = 0\)
% \item\(4 (2x + 1)^{2} = 36\)
% \end{enumeratea}
% \end{multicols}
% \end{esercizio}
% 
% \begin{esercizio}[\Ast]
% \label{ese:3.37}
% Risolvi le seguenti equazioni con opportune sostituzioni.
% \begin{multicols}{2}
% \begin{enumeratea}
% \item\((3x-5)^{2}-49 = 0\)
% \item\(3 (2x + 5)^{2}-4 (2x + 5) = 0\)
% \item\((3 \cdot 10^{3} x-10)^{2}-5 (3 \cdot 10^{3} x-10) = -6\)
% \item\((x-1)^{2}-(\sqrt{3} + \sqrt{5}) (x-1) + \sqrt{15} =0\)
% \end{enumeratea}
% \end{multicols}
% \end{esercizio}
% 
% \paragraph{3.35.} a)~\(x_{1,2} = \dfrac{3 \pm \sqrt{5}}{2}\),\quad 
% b)~\(x_{1,2} =\pm 
% 6\),\quad c)~\(x_{1,2} = 3 \pm \dfrac{\sqrt{138}}{4}\)
% 
% \paragraph{3.36.} a)~\(x_{1} =-2 \vee x_{2} = \dfrac{1}{2}\),\quad 
% b)~\(\emptyset\),\quad c)~\(x_1=-\dfrac{5}{3} \vee x_2=\dfrac{7}{3}\),\quad 
% d)~\(x_1=-2 
% \vee x_2=1\)
% 
% \paragraph{3.37.} a)~\(x_{1} = 4 \vee x_{2} =-\dfrac{2}{3}\),\quad 
% b)~\(x_{1} 
% =-\dfrac{5}{2} \vee x_{2} =-\dfrac{11}{6}\),\quad c)~\(x_{1} =\dfrac{1}{250} 
% \vee 
% x_{2} =\dfrac{13}{3000}\),\quad d)~\(x_{1} = 1 + \sqrt{3} \vee x_{2} = 1 + 
% \sqrt{5}\)
% 
% \begin{esercizio}[\Ast]
% \label{ese:3.38}
% Risolvi le seguenti equazioni con opportune sostituzioni.
% \begin{multicols}{2}
% \begin{enumeratea}
% \item\(3 (1-2x)^{2}-2 (1-2x)-1 = 0\)
% \item\(\dfrac{4}{3} (x-2)^{2}-6 (x-2) + 6 = 0\)
% \item\(\dfrac{1}{2} \left(x-\dfrac{1}{2} \right)^{2}-2 \left(x 
% -\dfrac{1}{2} 
% \right) = 0\)
% \item\(2 (x^{2}-1)^{2} + 3 (x^{2}-1)-5 = 0\)
% \item\(3 (34 x-47)^{2}-2 (34 x-47) = 1\)
% \end{enumeratea}
% \end{multicols}
% \end{esercizio}
% 
% \paragraph{3.38.} a)~\(x_{1} =0 \vee x_{2} =\dfrac{2}{3}\),\quad b)~\(x_{1} = 
% 5 \vee 
% x_{2} = \dfrac{7}{2}\),\quad c)~\(x_{1} =\dfrac{1}{2} \vee x_{2} 
% =\dfrac{9}{2}\),\quad 
% d)~\(x{1,2}= \pm \sqrt{2}\),\quad e)~\(x_{1} = \dfrac{24}{17} \vee x_{2} = 
% \dfrac{70}{51}\)


% \subsection*{3.3 - Discussione e risoluzione di equazioni numeriche 
% frazionarie}

\newpage %--------------------------------------------

\subsection*{\numnameref{sec:eq2gr_frazionarie}}

\begin{esercizio}[\Ast]
\label{ese:3.39}
Determina l'Insieme Soluzione delle seguenti equazioni fratte.
\begin{multicols}{2}
\begin{enumeratea}
\item\(\dfrac{3}{x}-2 = x\)
  \hfill\(\left[x_{1} =-3 \vee x_{2} = 1\right]\)
\item\(\dfrac{4-3 x}{x}=\dfrac{3-2 x}{x^{2}}\)
  \hfill\(\left[x_{1,2}= 1\right]\)
\item\(\dfrac{1}{x} = \dfrac{1}{x + 1}-1\)
  \hfill\(\left[\emptyset\right]\)
\item\(\dfrac{x}{2} = \dfrac{x + 2}{x-2} + 1\)
  \hfill\(\left[x_{1} = 0 \vee x_{2} = 6\right]\)
\item\(\dfrac{3}{x-1}-\dfrac{1}{x} + \dfrac{1}{2} = 0\)
  \hfill\(\left[x_{1} =-1 \vee x_{2} =-2\right]\)
\item\(\dfrac{3 x}{x^{2}-9} + \dfrac{x}{2 x-6}=1\)
  \hfill\(\left[x_{1,2} = \dfrac{9 \pm 3 \sqrt{17}}{2}\right]\)
\item\(\dfrac{x + 9}{x-3}=2-\dfrac{x-3}{x + 9}\)
  \hfill\(\left[\emptyset\right]\)
\item\(\dfrac{x}{x + 1} = \dfrac{4}{x + 2}\)
  \hfill\(\left[x_{1,2} = 1 \pm \sqrt{5}\right]\)
\item\(\dfrac{2 x + 1}{x} = \dfrac{x}{2 x + 1}\)
  \hfill\(\left[x_{1} =-1 \vee x_{2} =-\dfrac{1}{3}\right]\)
\item\(\dfrac{4-x}{18-2 x^{2}} + \dfrac{2}{3-x} = \dfrac{6 x}{4 x +12}\)
  \hfill\(\left[\emptyset\right]\)
\item\(\dfrac{6}{9 x^{2}-12 x + 4} + \dfrac{2}{6x -1} =0\)
  \hfill\(\left[...\right]\)
\item\(x-1-\dfrac{1}{x-1} = \dfrac{6}{6-6 x}\)
  \hfill\(\left[\emptyset\right]\)
\item\(\dfrac{x-4}{x-2} + \dfrac{x-1}{x^{2}-5 x + 6}-\dfrac{4 -2 x}{3-x} = 0\)
  \hfill\(\left[x =-1\right]\)
\item\(\dfrac{x-3}{x-1}-\dfrac{4}{3} + \dfrac{x-1}{x + 1}=0\)
  \hfill\(\left[x_{1,2} = 3 \pm \sqrt{10}\right]\)
\end{enumeratea}
\end{multicols}
\end{esercizio}

\begin{esercizio}[\Ast]
 \label{ese:3.43}
Determina l'Insieme Soluzione delle seguenti equazioni fratte.
\begin{enumeratea}
\item\(\dfrac{4 x-3}{x^{2}-4}-\dfrac{3 x}{x-2} = \dfrac{4}{2-x}-\dfrac{4 
x}{2 + x}\)
  \hfill\(\left[x_{1} = 1 \vee x_{2} = 5\right]\)
\item\(\dfrac{3 x + 2}{2 x^{2}-2 x-12}-\dfrac{3-x}{4 x-12} = - \dfrac{3}{x + 
2}\)
  \hfill\(\left[x_{1} =-19 \vee x_{2} = 2\right]\)
\item\(\dfrac{6 x-6}{x^{2}-4 x + 3} + \dfrac{x^{2}-x-6}{x-3}=-2\)
  \hfill\(\left[x_{1} =-3 \vee x_{2} = 2\right]\)
\item\(\dfrac{x-1}{x} + \dfrac{1}{x + 1} + \dfrac{2 + x}{x^{2} + x} =0\)
  \hfill\(\left[x_{1,2}=-1\right]\)
\item\(3 \left(x-\dfrac{1}{3} \right) + \dfrac{9}{3x-1} = 10\)
  \hfill\(\left[...\right]\)
\item\(\dfrac{x + 1}{\sqrt{2}-x} = \dfrac{x-2}{x-2 \sqrt{2}}\)
  \hfill\(\left[x_{1} = 0;~x_{2} = \dfrac{1 + 3 \sqrt{2}}{2}\right]\)
\item\(\dfrac{1}{x^{2} + x-2}-\dfrac{1}{x^{3}-2 x^{2} + x}=\dfrac{1}{3 
x^{2}-3 x}\)
  \hfill\(\left[x_{1,2} =-\dfrac{1}{2} \vee x_{2} = 4\right]\)
\item\(\dfrac{1}{2 x-4}-\dfrac{2}{x + 1}-\dfrac{1}{x-1}=\dfrac{1}{x^{2}-3 x 
+ 2}\)
  \hfill\(\left[x=\dfrac{7}{5}\right]\)
\item\(\dfrac{2 x}{x^{2} + 2 x-8}-\dfrac{2 x + 7}{x^{2}-3 x-4}= 0\)
  \hfill\(\left[x_{1} =-2 \vee x_{2} = \dfrac{28}{17}\right]\)
\item\(\dfrac{1-x}{x^{2}-4 x + 3}-\dfrac{4}{9-x^{2}} + \dfrac{x- 3}{x^{2} + 
4 x + 
3} =-\dfrac{5}{3-x}\)
  \hfill\(\left[x_{1} =-5 \vee x_{2} =-\dfrac{1}{5}\right]\)
\item\(\dfrac{3}{(3 x-6)^{2}}-\dfrac{x^{2}-4}{(3 x-6)^{4}} = 0\)
  \hfill\(\left[x = \dfrac{28}{13}\right]\)
\end{enumeratea}
\end{esercizio}

% \begin{esercizio}[\Ast]
%  \label{ese:3.45}
% Determina l'Insieme Soluzione delle seguenti equazioni fratte.
% \begin{multicols}{2}
% \begin{enumeratea}
% \item\(\dfrac{4 x-7}{x + 2} + \dfrac{1-6 x^{2}}{x^{2}-5 x + 6} =\dfrac{x}{2 
% x^{2}-2 
% x-12}-2\)
% \item\(\dfrac{1}{x-2} + \dfrac{2}{(x-2)^{2}} = \dfrac{3}{(x-2)^{3}}\)
% \item\(\dfrac{1}{x + 3}-\dfrac{5 (x + 2)}{(x + 3)^{2}} = \dfrac{5 x- 1}{(x 
% + 
% 3)^{3}}\)
% \end{enumeratea}
% \end{multicols}
% \end{esercizio}
% 
% \begin{esercizio}[\Ast]
%  \label{ese:3.46}
% Determina l'Insieme Soluzione delle seguenti equazioni fratte.
% \begin{multicols}{2}
% \begin{enumeratea}
% \item\(\dfrac{2 x}{x^{2}-2 x + 1} = \dfrac{- 7}{3 x^{2}-21 x + 18}+ 
% \dfrac{2 
% x}{x^{2}-3 x + 2}\)
% \item\(\dfrac{5 x-3}{x^{2}-5 x} + \dfrac{2}{x} = \dfrac{3 x}{x^{2}+ 3 
% x}-\dfrac{2}{x 
% + 3}-\dfrac{4}{5-x}\)
% \item\(\dfrac{x-9}{4 x-x^{2}}-\dfrac{3 x + 2}{2-x} = \dfrac{x-5}{x + 2} + 
% \dfrac{2 
% x^{4} + 6 x^{3}}{x (x-4) (x^{2}-4)}\)
% \item\(\dfrac{3 (x + 1)}{x-1}=1-\dfrac{2 x-3}{x}\)
% \end{enumeratea}
% \end{multicols}
% \end{esercizio}
% 
% \paragraph{3.45.} a)~\(\emptyset\),\quad b)~\(x_{1} =-1 \vee x_{2} = 
% 3\),\quad 
% c)~\(x_{1} =-5 \vee x_{2} =-1\),\quad
% 
% \paragraph{3.46.} a)~\(x_{1} =-14 \vee x_{2} =-1\),\quad b)~\(x_{1,2} = 
% \dfrac{- 1 
% \pm \sqrt{313}}{4}\),\quad c)~\(\emptyset\)
% 
% \begin{esercizio}[\Ast]
%  \label{ese:3.47}
% Determina l'Insieme Soluzione delle seguenti equazioni fratte.
% \begin{multicols}{2}
% \begin{enumeratea}
% \item\(\dfrac{3-3 x}{x^{2}-1} + \dfrac{8 x}{2-2 x} = 0\)
% \item\(\dfrac{1}{x^{2}-9} + \dfrac{2}{x-3} + \dfrac{2 x}{3 x + 9} 
% -\dfrac{31}{3 
% x^{2}-27} = \dfrac{1}{3}\)
% \item\(\dfrac{\dfrac{1}{1 + x}-\dfrac{1}{1-x}}{\dfrac{2}{x-1} +\dfrac{2}{x 
% + 1}} = 
% \dfrac{2 x}{1-x}-\dfrac{2 x}{1 + x}\)
% \item\(\dfrac{x + 1}{x-2 \sqrt{3}}-\dfrac{1-x}{x + 2 \sqrt{3}} 
% =\dfrac{x^{2} + 
% 8}{x^{2}-12}\)
% \item\(\dfrac{2 x + 1}{1 + x} + \dfrac{5}{1-x}-\dfrac{2}{x^{2}-1}= 0\)
% \end{enumeratea}
% \end{multicols}
% \end{esercizio}
% 
% \begin{esercizio}[\Ast]
%  \label{ese:3.48}
% Determina l'Insieme Soluzione delle seguenti equazioni fratte.
% \begin{multicols}{2}
% \begin{enumeratea}
% \item\(\left(\dfrac{x + 1}{x} \right)^{2}-\dfrac{2 (3 x-1)}{x^{2}} = 5\)
% \item\(\dfrac{(x-2)^{2}}{x^{2}-1}-\dfrac{x + 2}{x + 1} +\dfrac{x}{2 x + 2} 
% = 0\)
% \item\(- \dfrac{x^{2}}{x + 2} + \dfrac{2 x}{x-2} =-\dfrac{x + 
% x^{3}}{x^{2}-4}\)
% \item\(\dfrac{5}{x + 1} + \dfrac{2 x}{x-2} = \dfrac{6 x^{2}-10}{x^{2}-x-2}\)
% \item\(\dfrac{x + 1}{x-2}-\dfrac{3 x}{x + 3} = \dfrac{x^{2} + 2 x}{x^{2} + 
% x-6}\)
% \end{enumeratea}
% \end{multicols}
% \end{esercizio}
% 
% \paragraph{3.47.} a)~\(x_{1,2} = \dfrac{- 7 \pm \sqrt{97}}{8}\),\quad 
% b)~\(x_{1} =-1 
% \vee x_{2} = 1\),\quad c)~\(x_{1} =-\dfrac{1}{3} \vee x_{2} = 
% \dfrac{1}{3}\),\protect 
% \\ \quad d)~\(x_{1} = \sqrt{6}-\sqrt{2} \vee x_{2} = \sqrt{2}-\sqrt{6}\)
% 
% \paragraph{3.48.} a)~\(x_{1} =-\dfrac{3}{2} \vee x_{2} = 
% \dfrac{1}{2}\),\quad 
% b)~\(x_{1,2} = \dfrac{11 \pm \sqrt{73}}{2}\),\quad c)~\(x_{1} = 0 \vee x_{2} 
% =-\dfrac{5}{4}\),\quad d)~\(x_{1} = 0 \vee x_{2} = \dfrac{7}{4}\),\quad 
% e)~\(x_{1} 
% =-\dfrac{1}{3} \vee x_{2} = 3\)

\begin{esercizio}[\Ast]
 \label{ese:3.49}
È vero che in\(\insR\) le equazioni~\(\dfrac{3}{1 + x^{2}} = \dfrac{3}{x^{4} 
+2 x^{2} + 1}\) e~\(\dfrac{2 x + 14}{x^{3}-x^{2} + 4 x-4}-\dfrac{4}{x-1} 
=\dfrac{2}{x^{2} + 4}\) sono equivalenti?
\end{esercizio}

\begin{esercizio}[\Ast]
 \label{ese:3.50}
Verifica che il prodotto delle soluzioni di~\(\dfrac{x}{1-x^{3}} 
+ \dfrac{2 x-2}{x^{2} + x + 1}=0\) valga~\(1\)
\end{esercizio}

\begin{esercizio}[\Ast]
 \label{ese:3.51}
Sull'asse reale rappresenta il Dominio e l'Insieme Soluzione 
dell'equazione\\ 
\(\dfrac{x + 2}{x}=2+\dfrac{x}{x + 2}\)
\end{esercizio}

\begin{esercizio}[\Ast]
 \label{ese:3.52}
Stabilisci se esiste qualche numero reale per cui la somma delle due 
frazioni \\
\(f_{1}=\dfrac{2-x}{x + 2}\) e\(f_{2}=\dfrac{x + 1}{x-1}\) è uguale 
a~\(\dfrac{9}{5}\)
\end{esercizio}

\begin{esercizio}[\Ast]
 \label{ese:3.53}
È vero che l'espressione~\(E=\dfrac{4 x}{1-x^{2}} + \dfrac{1-x}{1 + 
x}-\dfrac{1 + x}{1-x}\) non assume mai il valore~\(-1\)?
\end{esercizio}

% \subsection*{3.4 - Discussione e risoluzione di equazioni letterali}
% 
% \begin{esercizio}[\Ast]
%  \label{ese:3.54}
% Risolvi ed eventualmente discuti le seguenti equazioni letterali.
% \begin{multicols}{2}
% \begin{enumeratea}
% \item\(x^{2}-a x = 0\)
% \item\(ax^{2}-4a^{3} = 0\)
% \item\(x^{2} + (x-a)^{2} = 2 a x\)
% \item\((2 x-a) x = a x\)
% \end{enumeratea}
% \end{multicols}
% \end{esercizio}
% 
% \begin{esercizio}[\Ast]
%  \label{ese:3.55}
% Risolvi ed eventualmente discuti le seguenti equazioni letterali.
% \begin{multicols}{2}
% \begin{enumeratea}
% \item\(x^{2}-a x-6 a^{2} = 0\)
% \item\((a-3) x^{2}-a x + 3 = 0\)
% \item\(a x^{2}-a^{2} x + x^{2} + x-a x-a = 0\)
% \item\(\dfrac{x}{a} + \dfrac{x^{2}}{a-1} = 0\)
% \end{enumeratea}
% \end{multicols}
% \end{esercizio}
% 
% \paragraph{3.54.} a)~\(x_{1} = 0 \vee x_{2} = a\),\quad b)~\(a = 0 
% \rightarrow 
% \insR;~ a \neq 0 \rightarrow x_{1}=-2 a \vee x_{2} = 2 a\),\quad 
% c)~\(x_{1,2} = 
% \dfrac{2\pm\sqrt{2}}{2} a\),\quad d)~\(x_{1} = 0 \vee x_{2} = a\)
% 
% \paragraph{3.55.} a)~\(x_{1} =-2 a \vee x_{2} = 3 a\),\quad b)~\(x_{1} = 1 
% \vee 
% x_{2} = \dfrac{3}{a-3}\),\quad c)~\(x_{1} = a \vee x_{2} =-\dfrac{1}{a + 
% 1}\),\protect\\ \quad d)~\(a \neq 0 \wedge a \neq 1 \rightarrow x_{1} = 0 
% \vee x 
% _{2} = \dfrac{1-a}{a}\)
% 
% \begin{esercizio}[\Ast]
%  \label{ese:3.56}
% Risolvi ed eventualmente discuti le seguenti equazioni letterali.
% \begin{multicols}{2}
% \begin{enumeratea}
% \item\(\dfrac{x}{a + 1} + \dfrac{x^{2}}{a-1} = 0\)
% \item\(\dfrac{2 x}{3 + k x}-\dfrac{x}{3-k x}=0\)
% \item\(\dfrac{m-n}{m n} x^{2}=\dfrac{2 m^{2} n}{m^{2}-n^{2}}-\dfrac{m n}{m 
% + n}\)
% \item\(\dfrac{m x-x^{2}}{m^{2}-3 m + 2}-\dfrac{x}{2-m} -\dfrac{m + 
% 1}{m-1}=0\)
% \end{enumeratea}
% \end{multicols}
% \end{esercizio}
% \newpage
% \begin{esercizio}[\Ast]
%  \label{ese:3.57}
% Risolvi ed eventualmente discuti le seguenti equazioni letterali.
% \begin{multicols}{2}
% \begin{enumeratea}
% \item\(\dfrac{x^{2} + 2 t x}{t^{2}-t x}-2=\dfrac{3 t}{t-x}+ \dfrac{x + 
% t}{t}\)
% \item\(\dfrac{x-1}{k + 1}-\dfrac{x^{2} + 1}{k^{2}-1} =\dfrac{2 k}{1-k^{2}}\)
% \item\(2 \cdot \sqrt{m}-x=\dfrac{m-1}{x}\)
% \end{enumeratea}
% \end{multicols}
% \end{esercizio}
% 
% \paragraph{3.56.} a)~\(a \neq-1 \wedge a \neq 1 \rightarrow x_{1} = 0 \vee 
% x_{2} 
% = \dfrac{1-a}{a + 1}\),\quad b)~\(x_{1} = 0 \vee x_{2} = 
% \dfrac{1}{k}\),\quad 
% c)~\(x_{1,2} = \pm \dfrac{m}{m-n}\),\quad d)~\(x_{1} = m-2 \vee x_{2} = m + 
% 1\)
% 
% \paragraph{3.57.} a)~\(x_{1,2} =-3t\),\quad b)~\(x_{1} =-1;~x_{2} = k\),\quad 
% c)~\(x_{1,2} = \sqrt{m} \pm 1\)
% 
% \begin{esercizio}
%  \label{ese:3.58}
% È vero che l'equazione~\(1-\dfrac{1}{k + x}-\dfrac{1}{k-x}=0\) ammette due 
% soluzioni 
% reali coincidenti se\(k = 2\)?
% \end{esercizio}
% 
% \begin{esercizio}
%  \label{ese:3.59}
% Nell'equazione~\((a-1) \cdot (x + a)=\dfrac{x + a}{x-1} \cdot [ x (a +1)-2 
% a ]\), 
% dopo aver completato la discussione, stabilisci per quali valori di\(a\) le
% radici che si ottengono dall'equazione completa sono entrambe positive.
% \end{esercizio}
% 
% \begin{esercizio}
%  \label{ese:3.60}
% È vero che l'equazione~\(3 k x^{2} + (x-k)^{2} + 2 k (k + x)=0\) ammette 
% radici 
% reali opposte se\(k <-\dfrac{1}{3}\)?
% \end{esercizio}
% 
% \begin{esercizio}
%  \label{ese:3.61}
% Per quali valori di\(b\) l'equazione~\(\dfrac{5 x^{2}-4 (b + 
% 1)}{b^{2}-4}-\dfrac{3 
% x-1}{b + 2 }=\dfrac{3-2 x}{2-b}-\dfrac{3 x}{b^{2}-4}\) ha una soluzione 
% negativa?
% \end{esercizio}
% 
% \begin{esercizio}
%  \label{ese:3.62}
% Per l'equazione~\((x-k-1)^{2}=(k + 1) \cdot (k-2 x + x^{2})\), completate 
% le 
% implicazioni:
% 
%\(k = 0 \rightarrow\) equazione~\(\ldots\ldots\ldots\ldots \ldots \IS = 
% \ldots 
% \ldots \ldots \ldots \ldots \ldots \ldots\)
% 
%\(k =-1 \rightarrow\) equazione~\(\ldots\ldots\ldots\ldots \ldots x_{1,2} = 
% \ldots 
% \ldots \ldots \ldots \ldots \ldots \ldots\)
% 
%\(k =\ldots \ldots\) equazione pura;~ due soluzioni reali \ldots\ldots\ldots 
% se 
% \ldots\ldots \ldots\(x_{1} = \ldots \ldots \vee x_{2} = \ldots \ldots\)
% 
% \end{esercizio}
% 
% \begin{esercizio}
%  \label{ese:3.63}
% Stabilisci per quali valori del parametro\(m\) l'equazione~\(\dfrac{m + 
% 2}{x-2} + m 
% x=2\) ammette soluzioni reali distinte. Se\(m =-2\) sono accettabili le 
% radici 
% reali trovate?
% \end{esercizio}
% 
% \begin{esercizio}
%  \label{ese:3.64}
% Dopo aver discusso l'equazione parametrica\(\dfrac{x + 1}{b-1} + 
% \dfrac{b-1}{x + 
% 1}=\dfrac{3 x^{2} +2-b x}{b x + b-1-x}\), determina per quale valore del 
% parametro 
% le soluzioni sono accettabili.
% \end{esercizio}
% 
% \begin{esercizio}
%  \label{ese:3.65}
% Le soluzioni dell'equazione~\((x + b)^{2} = (b + 1)^{2}\) con\(b \neq-1\) 
% sono:
% 
% \boxA \;~\(x_{1} =-1\vee x _{2} = 1\)
% \boxB \;~\(x_{1} =-2 b-1 \vee x_{2} = 1\)
% \boxC \;~\(x_{1} = x_{2} = 1\)
% \boxD \;~\(x_{1} = 1-2 b \vee x_{2} = 1\)
% \end{esercizio}
% 
% \begin{esercizio}
%  \label{ese:3.66}
% Per quali valori di\(k\) l'equazione~\(x^{2}-(2 k + 1) x + 3 k + 1=0\) 
% ammette 
% soluzioni reali coincidenti?
% \end{esercizio}

% \subsection*{3.5 - Relazioni tra soluzioni e coefficienti}
\subsection*{\numnameref{sec:eq2gr_coefficienti}}

\begin{esercizio}
 \label{ese:3.67}
Completare la seguente tabella.

 \begin{tabular*}{.9\textwidth}{@{\extracolsep{\fill}}*{5}{l}}
 \toprule
 Equazione &~Discriminante&~\(\IS\subset\insR\)? &~\(x_1 + x_2\) &~\(x_1 \cdot 
x_2\)\\
\midrule
\(5 x^{2} + 2 x-1 = 0\)&\(\Delta=\ldots \ldots\) & & &\\
\(- 3 x^{2} + 1 = 0\)&\(\Delta=\ldots \ldots\) & & &\\
\(6 x^{2} + 7 x = 0\)&\(\Delta=\ldots \ldots\) & & &\\
\(- x^{2} + x-1 = 0\)&\(\Delta=\ldots \ldots\) & & &\\
\(x^{2} + 2 x + 1 = 0\)&\(\Delta=\ldots \ldots\) & & &\\
\(2 x^{2}-7 x + 1 = 0\)&\(\Delta=\ldots \ldots\) & & &\\
\bottomrule
 \end{tabular*}

\end{esercizio}

\begin{esercizio}
 \label{ese:3.68}
Senza risolvere le equazioni determina somma e prodotto dello loro radici.
\begin{multicols}{2}
\begin{enumeratea}
\item\(x^{2} + 4ax + a = 0\)
\item\(2x^{2}-\sqrt{2} x + 1 = 0\)
\item\(2x^{2} + 6kx + 3k^{2} = 0\)
\item\(3 \sqrt{3} x^{2}-6 \sqrt{3} x + 2 = 0\):
\item\(\sqrt{2} x^{2} + (\sqrt{3}-\sqrt{2}) x + 4 = 0\):
\item\((\sqrt{5} + \sqrt{3}) x^{2}-(\sqrt{5}-\sqrt{3}) x + 1= 0\)
\end{enumeratea}
\end{multicols}
\end{esercizio}

\begin{esercizio}
 \label{ese:3.69}
Dell'equazione~\(3 \sqrt{2} x^{2}-5 x + \sqrt{2} = 0\) 
è nota la radice~\(x_{1} = \dfrac{1}{\sqrt{2}}\) 
senza risolvere l'equazione determinare l'altra radice.
\end{esercizio}

\begin{esercizio}
 \label{ese:3.70}
Si determini la relazione che lega i coefficienti della generica equazione 
di secondo grado alla somma dei quadrati delle radici. Si vuole esprimere,
attraverso i coefficiente~\(a\),~\(b\),~\(c\) dell'equazione la 
quantità~\(x_{1}^{2} + x_{2}^{2}\) 
Si tenga presente la seguente identità~\(x_{1}^{2} + x_{2}^{2} = 
(x_{1} + x_{2} )^{2}-2 x_{1} x_{2}\)
\end{esercizio}

\begin{esercizio}
 \label{ese:3.71}
Senza risolvere le equazioni~\(5 x^{2} + 2 x-1 = 0\), 
~\(-x^{2} + x-1 = 0\) e~\(2 x^{2}-7 x +1 = 0\) stabilisci quale ha come 
soluzioni due numeri reali positivi e quale due numeri reali reciproci.
\end{esercizio}

\begin{esercizio}
 \label{ese:3.72}
Un'equazione di secondo grado ha il primo coefficiente uguale a\(- 
\dfrac{3}{2}\) 
sapendo che l'insieme soluzione è~\(\IS = \left\{-\dfrac{3}{4};~\sqrt{2} 
\right\}\)
determinate i suoi coefficienti\(b\) e\(c\)
\end{esercizio}

\begin{esercizio}
 \label{ese:3.73}
Dell'equazione~\(a x^{2} + b x + c = 0\) la somma delle soluzioni 
è\(\dfrac{21}{5}\) e una soluzione è~\(x_{1} = 3,2\) determinare~\(x_{2}\)
\end{esercizio}

\begin{esercizio}
 \label{ese:3.74}
Determinate i coefficienti \emph{a}, \emph{b}, \emph{c} di un'equazione di 
secondo grado sapendo che~\(x_{1} = 1-\sqrt{2}\), il prodotto delle soluzioni 
è~\(-1\) e la somma del secondo con il terzo coefficiente è\(9\)
\end{esercizio}

\begin{esercizio}
 \label{ese:3.75}
Determinate i coefficienti \(b\) e \(c\) dell'equazione~\(x^{2} + b x + c = 0\) 
sapendo che una radice è tripla dell'altra e la loro somma è 20.
\end{esercizio}

\begin{esercizio}[\Ast]
 \label{ese:3.76}
Dopo aver completato la discussione dell'equazione parametrica\(\dfrac{x + 
1}{b-1} + \dfrac{b-1}{x + 1}=\dfrac{3 x^{2} + 2-b x}{b x + b-1-x}\), 
determina se 
esiste qualche valore del parametro per cui\(x_{1} + x_{2} = x_{1} \cdot 
x_{2}\)
\end{esercizio}

\begin{esercizio}
 \label{ese:3.77}
Determina, se possibile, due numeri aventi somma e prodotto indicati.
\begin{multicols}{2}
\begin{enumeratea}
\item\(s = 3 \text{ e } p = 5\)
\item\(s = 7 \text{ e } p = 2\)
\item\(s =-3 \text{ e } p =-8\)
\item\(s =-5 \text{ e } p = 4\)
\end{enumeratea}
\end{multicols}
\end{esercizio}

\begin{esercizio}
 \label{ese:3.78}
Determina, se possibile, due numeri aventi somma e prodotto indicati.
\begin{multicols}{2}
\begin{enumeratea}
\item\(s = \dfrac{1}{2} \text{ e } p = \dfrac{2}{3}\)
\item\(s = \sqrt{2} \text{ e } p = 2\)
\item\(s = \sqrt{7}-1 \text{ e } p = 6\)
\item\(s = a + 1 \text{ e } p= a^{2}\)
\end{enumeratea}
\end{multicols}
\end{esercizio}

\begin{esercizio}
 \label{ese:3.79}
Scrivi un'equazione di secondo grado che ammette come radici le soluzioni 
indicate.
\begin{multicols}{2}
\begin{enumeratea}
\item\(x_{1} =-2 \vee x_{2} = 5\)
\item\(x_{1} = 7 \vee x_{2} = 2\)
\item\(x_{1} =-\dfrac{1}{2} \vee x_{2} = \dfrac{3}{4}\)
\item\(x_{1} = \dfrac{2}{3} \vee x_{2} = \dfrac{1}{3}\)
\item\(x_{1} = \sqrt{2} \vee x_{2} = \sqrt{5}\)
\item\(x_{1} = \dfrac{1 + \sqrt{2}}{2} \vee x_{2} = \dfrac{1-\sqrt{2}}{2}\)
\end{enumeratea}
\end{multicols}
\end{esercizio}

\begin{esercizio}
 \label{ese:3.80}
Nell'equazione~\(2 x^{2} + 6 k x + 3 k^{2} = 0\) determinare i valori di\(k\) 
per 
cui tra le radici reali distinte sussista la relazione\(x_{1} + x_{2} = 
x_{1} 
\cdot x_{2}\)
\end{esercizio}

\begin{esercizio}
 \label{ese:3.81}
Determinate il perimetro del rombo avente\(area = 24\unit{m^{2}}\), sapendo 
che 
la somma delle misure delle sue diagonali è\(14\unit{m}\)
\end{esercizio}

\begin{esercizio}
\label{ese:3.82}
Costruire i due triangoli isosceli aventi\(area = 120\unit{m^{2}}\) sapendo 
che 
\(31\unit{m}\) è la somma delle misure della base con l'altezza.
\end{esercizio}

\begin{esercizio}
 \label{ese:3.83}
Il triangolo rettangolo\(ABC\) ha l'ipotenusa\(AC\) di\(40\unit{cm}\) e l'altezza 
\(BH\) ad essa relativa di\(19,2\unit{cm}~\)Determinate la misura delle 
proiezioni 
dei cateti sull'ipotenusa.
\end{esercizio}

% \subsection*{3.6 - Scomposizione del trinomio di secondo grado}
\subsection*{\numnameref{sec:eq2gr_scomposizione_trinomio}}

\begin{esercizio}[\Ast]
 \label{ese:3.84}
Scomponi in fattori i seguenti trinomi di secondo grado.
\begin{multicols}{2}
\begin{enumeratea}
\item\(x^{2}-5 x-14\) \hfill\(\left[(x + 2) (x-7)\right]\)
\item\(2 x^{2} + 6 x-8\) \hfill\(\left[2 (x-1) (x + 4)\right]\)
\item\(- 3 x^{2} + \dfrac{39}{2} x-9\) 
 \hfill\(\left[-3 \left(x-\dfrac{1}{2} \right) (x-6)\right]\)
\item\(- 2 x^{2} + 7 x + 4\) \hfill\(\left[...\right]\)
\item\(4 x^{2} + 4 x-15\) 
 \hfill\(\left[4 \left(x-\dfrac{3}{2} \right) 
         \left(x + \dfrac{5}{2} \right)\right]\)
\item\(3 x^{2} + 3 x-6\) \hfill\(\left[...\right]\)
\item\(4 x^{2}-9 x + 2\) 
 \hfill\(\left[4 (x-2) \left(x-\dfrac{1}{4} \right)\right]\)
\item\(2 x^{2} + 2 x - \dfrac{3}{2}\) \hfill\(\left[...\right]\)
\item\(3 x^{2} + 5 x - 2\) 
 \hfill\(\left[3 \left(x-\dfrac{1}{3} \right) (x + 2)\right]\)
\item\(4 x^{2}-24 x + 20\) 
 \hfill\(\left[...\right]\)
\item\(2 x^{2}-\dfrac{4}{3} x - \dfrac{16}{3}\) 
 \hfill\(\left[2 (x-2) \left(x + \dfrac{4}{3} \right)\right]\)
% \item\(\dfrac{4}{3} x^{2} + \dfrac{11}{3} x - \dfrac{7}{2}\) 
%  \hfill\(\left[...\right]\)
\end{enumeratea}
\end{multicols}
\end{esercizio}

\begin{esercizio}[\Ast]
 \label{ese:3.87}
Scomponi in fattori i seguenti trinomi di secondo grado.
\begin{enumeratea}
\item\(3 x^{2}-6 x-12\) 
 \hfill\(\left[3 \left(x-1-\sqrt{5} \right) 
         \left(x-1 + \sqrt{5} \right)\right]\)
\item\(2 x^{2}-8 x + 2\) 
 \hfill\(\left[...\right]\)
\item\(- \dfrac{1}{2} x^{2} + x + \dfrac{3}{8}\) 
 \hfill\(\left[- \dfrac{1}{2} \left(x-1-\dfrac{\sqrt{7}}{2} \right) 
         \left(x- 1 + \dfrac{\sqrt{7}}{2} \right)\right]\)
\item\(- \dfrac{3}{4} x^{2}-\dfrac{9}{2} x - \dfrac{45}{8}\) 
 \hfill\(\left[- \dfrac{3}{4} \left(x + 3-\dfrac{\sqrt{6}}{2} \right) 
         \left(x+ 3 + \dfrac{\sqrt{6}}{2} \right)\right]\)
\end{enumeratea}
\end{esercizio}

\begin{comment}

\subsection*{3.7 - Regola di Cartesio}

\begin{esercizio}
 \label{ese:3.88}
Determina il segno delle soluzioni delle equazioni senza risolverle 
se\(\Delta 
\geq 0\)
\begin{multicols}{2}
\begin{enumeratea}
\item\(x^{2}-5 x + 6=0\)
\item\(- x^{2} + x + 42=0\)
\item\(x^{2} + x-20=0\)
\item\(3 x^{2} + 2 x-1=0\)
\end{enumeratea}
\end{multicols}
\end{esercizio}

\begin{esercizio}
\label{ese:3.89}
Determina il segno delle soluzioni delle equazioni senza risolverle 
se\(\Delta 
\geq 0\)
\begin{multicols}{2}
\begin{enumeratea}
\item\(2 x^{2}-\sqrt{5} x-1=0\)
\item\(3 x^{2} + 5 x + 1 = 0\)
\item\(- x^{2}-x + 1 = 0\)
\item\(- 5 x + 1-x^{2} = 0\)
\end{enumeratea}
\end{multicols}
\end{esercizio}

\begin{esercizio}
\label{ese:3.90}
Determina il segno delle soluzioni delle equazioni senza risolverle 
se\(\Delta 
\geq 0\)
\begin{multicols}{2}
\begin{enumeratea}
\item\(- 1-x^{2}-2 x = 0\)
\item\(1 + x + 2 x^{2} = 0\)
\item\(x^{2}-4 \sqrt{2} x + 2 = 0\)
\item\(- \dfrac{1}{2} x^{2} + x + \dfrac{3}{8} = 0\)
\end{enumeratea}
\end{multicols}
\end{esercizio}

\subsection*{3.8 - Equazioni parametriche}
\begin{esercizio}
 \label{ese:3.91}
Assegnata l'equazione~\((1-k) x^{2} + (k-2) x + 1 = 0\), stabilire i valori 
da 
assegnare al parametro\(k\) affinché le soluzioni reali distinte abbiano la 
somma 
positiva.

\emph{Svolgimento guidato}

Nel testo del problema vi sono due richieste: a) le soluzioni siano reali 
distinte e b) abbiano somma positiva.

Il problema si formalizza attraverso il sistema
\(\sistema{ \Delta > 0 \\- \dfrac{b}{a} > 0 }
\Rightarrow \left \{\begin{array}{l} (k-2)^{2}-4 (1-k) > 0 
\\-\dfrac{k-2}{1-k} > 
0 }\) risolviamo la prima disequazione:\(k^{2} > 0 
\rightarrow 
\IS_{1} = \left \{ \right .k \in \insR | k \neq 0 \}\) e la seconda 
disequazione 
studiando il segno del numeratore e del denominatore:
\(\begin{array}{l} N: -k + 2 > 0 \Rightarrow k < 2 \\
D: 1-k > 0 \Rightarrow k < 1 \end{array}\) da cui con la tabella dei segni
\begin{center}
\input{\folder lbr/fig002_seg.pgf}
\end{center}
ricaviamo\(\IS_{2} = \{k \in \insR | k \ldots \ldots \ldots \vee k > 
\ldots 
\ldots \ldots \}\)
Dal grafico a destra inoltre otteniamo\(\IS = \IS_{1} \cap \IS_{2} = \{ k 
\in 
\insR | k
\ldots \ldots \ldots \vee 0 < k < \ldots \ldots \vee k \ldots \ldots 
\ldots \}\)
\end{esercizio}

\begin{esercizio}
 \label{ese:3.92}
Assegnata l'equazione~\((k + 1) x^{2} + (k + 3) x + k = 0\) stabilire per 
quale 
valore di\(k\) una sua soluzione è\(x =-1\) In tale caso determinare 
l'altra 
soluzione.

\emph{Traccia di svolgimento}:
Ricordiamo che un valore numerico è soluzione di un'equazione se 
sostituito 
all'incognita trasforma l'equazione in una uguaglianza vera.
Per questo motivo, sostituendo all'incognita il valore assegnato, il 
parametro 
\(k\) dovrà verificare l'uguaglianza:\((k + 1) (- 1)^{2} + (k + 3) (- 1) + 
k = 
0 
\Rightarrow\ldots\ldots\ldots\ldots\)
Sostituendo il valore di\(k\) trovato, l'equazione diventa:\(3 x^{2} + 5 
x + 
2 = 
0\) l'altra soluzione può essere trovata o con la formula risolutiva, 
oppure
ricordando che\(x_{1} + x_{2}=- \dfrac{b}{a}=- \dfrac{5}{3}\) da cui 
\(x_{2}=\ldots\ldots\) o anche\(x_{1} \cdot 
x_{2}=\dfrac{c}{a}=\dfrac{2}{3}\) 
da cui 
\(x_{2}=\ldots\ldots\)
\end{esercizio}

\begin{esercizio}
 \label{ese:3.93}
Giustificare la verità della seguente proposizione: ``per qualunque 
valore 
assegnato al parametro\(m\) l'equazione~\((m-1) x^{2} + 2 m x + m + 1 = 
0\)
ha soluzioni reali distinte''.
Determinare inoltre\(m\) affinché: a)\(x_{1} + x_{2} = 1-\sqrt{3}\)\quad 
b)\( 
x_{1} \cdot x_{2} = \dfrac{12}{5}\)\quad c)\(x_{1} + x_{2} = 1-x_{1} 
\cdot 
x_{2}\)
\end{esercizio}

\begin{esercizio}
 \label{ese:3.94}
Nell'equazione~\(7 x^{2} + (k-5) x-(k + 2) = 0\) determinare\(k\) 
affinché le 
soluzioni siano reali;~ distingui i casi ``reali coincidenti'' e ``reali 
distinte''.
Nel primo caso determina\(x_{1} = x_{2} = \ldots \ldots\) nel secondo 
caso, 
determina\(k\) affinché
\begin{enumeratea}
\item il prodotto delle soluzioni sia\(- \dfrac{8}{3}\)
\item una soluzione sia nulla;~
\item le soluzioni siano una il reciproco dell'altra, cioè:\(x_{1} = 
\dfrac 
{1} 
{x_{2}}\)
\item la somma dei reciproci delle soluzioni sia\(\dfrac{1} {2}\)
\item la somma delle soluzioni superi il loro prodotto di\(2\)
\end{enumeratea}
\end{esercizio}

\begin{esercizio}
 \label{ese:3.95}
Verificare che nell'equazione~\((2 m-3) x^{2}-(m + 2) x + 3 m-2 = 0\) si 
hanno due 
valori del parametro per cui le soluzioni sono reali coincidenti. 
Determina i 
due valori.
\end{esercizio}

\begin{esercizio}
 \label{ese:3.96}
Nell'equazione~\(x^{2}-2 (k + 2) x + (k^{2}-3 k + 2) = 0\) 
determinare\(k\) 
affinché le soluzioni siano reali, con somma positiva e prodotto negativo.

\emph{Traccia di svolgimento}:
Il problema richiede tre condizioni alle quali deve soddisfare 
contemporaneamente il parametro, pertanto si formalizza con il sistema
\(\left \{ \begin{array}{l} \Delta \geq 0 \\- \dfrac{b}{a} > 0\\
\dfrac{c}{a} < 0 }\)
\end{esercizio}
\end{comment}

\begin{esercizio}
 \label{ese:3.105}
 Per quale valore di \(k \in \insR\) l'equazione~\(kx^{2}-x + k = 0\) non 
ammette soluzioni reali?
\[\boxA\;~ k \leq-\dfrac{1}{2} \vee k \geq + \dfrac{1}{2}\quad\boxB\;~ - 
\dfrac{1}{2} 
< k < \dfrac{1}{2}\quad\boxC\;~ k <-\dfrac{1}{2} \vee k > 
\dfrac{1}{2}\quad\boxD\;~ - 
\dfrac{1}{2} \leq k \leq \dfrac{1}{2}\]
\end{esercizio}

\begin{esercizio}
 \label{ese:3.111}
Se l'equazione~\((k + 1) x^{2}-kx-4 = 0\) ha una soluzione uguale a \(2\) 
quanto 
vale l'altra soluzione?
\[\boxA\quad x = 0\qquad\boxB\quad x =-2\qquad\boxC\quad x = 
\dfrac{1}{2}\qquad\boxD\quad x = 2\]
\end{esercizio}

\begin{esercizio}[\Ast]
 \label{ese:3.97}
Data l'equazione~\(x^{2}-2 x-k = 0\) determinare\(k\) in modo che
\begin{enumeratea}
\item le soluzioni siano reali e distinte \quad\((\Delta>0)\)
\item la somma delle soluzioni sia \(10 \quad (x_{1} + x_{2} = 10)\)
\item il prodotto delle soluzioni sia \(10 \quad (x_{1} \cdot x_{2} = 10)\)
\item una soluzione sia uguale a \(0\) \quad (sostituire\(0\) alla\(x\));~
\item le radici siano opposte \quad \((x_{1} + x_{2} = 0)\)
\item le radici siano reciproche \quad \((x_{1} \cdot x_{2} = 1)\)
\item le radici siano coincidenti \quad \((\Delta=0)\)
\item la somma dei quadrati delle radici sia \(12 \quad \left(x_{1}^{2} + 
x_{2}^{2} = (x_{1} + x_{2})^{2}-2x_{1} x_{2} = 12\right)\)
\item la somma dei reciproci delle radici sia \(-4 \quad 
\left(\dfrac{1}{x_{1}} + 
\dfrac{1}{x_{2}} = \dfrac{x_{1} +x_{2}}{x_{1} x_{2}} =-4 \right)\)
\item la somma dei cubi delle radici sia \(1\) \protect\\ \(\left( 
x_{1}^{3} + 
x_{2}^{3} = (x_{1} + x_{2})^{3}-3x_{1}^{2} x_{2}-3x_{1} x_{2}^{2} = (x_{1} 
+ 
x_{2})^{3}-3x_{1} x_{2} (x_{1} + x_{2}) = 1\right)\)
\item le radici siano entrambe negative \(\left(\sistema{ 
x_{1} 
\cdot x_{2} > 0 \\x_{1} + x_{2} < 0 }\right)\)
\end{enumeratea}
\end{esercizio}

\begin{flushright}
\(\left[a)~k >-1,\quad b)~\emptyset,\quad c)~k =-10,\quad 
        d)~k = 0,\quad e)~ \emptyset ,\quad f)~ k =-1 ,\quad 
        g)~ k =-1 ,\quad \right.\)
       \(\left. h)~ k = 4 ,\quad i)~ k = \dfrac{1}{2} ,\quad 
        j)~ k =-\dfrac{7}{6} ,\quad k)~\emptyset\right]\)
\end{flushright}

\begin{esercizio}[\Ast]
 \label{ese:3.98}
Data l'equazione~\(x^{2}-kx -1 = 0\) determinare \(k\) in modo che
\begin{enumeratea}
\item le soluzioni siano coincidenti;~
\item la somma delle radici sia \(8\)
\item le radici siano opposte;~
\item una radice sia \(- \dfrac{1}{3}\)
\item il prodotto delle radici sia \(-1\)
\end{enumeratea}
\end{esercizio}

\begin{flushright}
\(\left[a)~\emptyset ,\quad b)~k = 8 ,\quad c)~ k = 0,\quad 
d)~k = \dfrac{8}{3} ,\quad e)~\forall k \in \insR\right]\)
\end{flushright}

\begin{esercizio}[\Ast]
 \label{ese:3.99}
Data l'equazione~\(x^{2} + (k + 1) x + k = 0\) determinate \(k\) affinché 
l'equazione
\begin{enumeratea}
\item abbia una soluzione uguale a zero;~
\item abbia soluzioni opposte;~
\item non abbia soluzioni reali;~
\item abbia le radici reciproche;~
\item abbia le radici positive (regola di Cartesio).
\end{enumeratea}
\end{esercizio}

\begin{flushright}
\(\left[a)~ k = 0 ,\quad b)~ k =-1 ,\quad c)~ \emptyset ,\quad 
d)~ k = 1 ,\quad e)~ \emptyset \right]\)
\end{flushright}

\begin{esercizio}[\Ast]
 \label{ese:3.100}
Data l'equazione~\(x^{2}-kx + 6 = 0\) determinate \(k\) affinché
\begin{enumeratea}
\item abbia la somma delle radici uguale a \(7\)
\item abbia le radici reali e opposte;~
\item abbia la somma dei reciproci delle radici uguale a \(-6\)
\item abbia una radice uguale a \(- \dfrac{3}{2}\)
\end{enumeratea}
\end{esercizio}

\begin{flushright}
\(\left[a)~ k = 7 ,\quad b)~ \emptyset ,\quad c)~ k =-36 ,\quad 
d)~ k =-\dfrac{11}{2} \right]\)
\end{flushright}

% \newpage %-----------------------------------------

\begin{esercizio}[\Ast]
 \label{ese:3.101}
Data l'equazione~\(x^{2} + (k + 1) x + k^{2} = 0\) determinare \(k\) 
affinché
\begin{enumeratea}
\item abbia come soluzione \(-1\)
\item abbia una soluzione doppia \((x_1 =x_2)\)
\item abbia le radici reciproche;~
\item abbia una radice l'opposto della reciproca dell'altra 
\(\left(x_1=-\dfrac{1}{x_2}\rightarrow x_1 \cdot x_2=-1\right)\)
\item abbia una radice nulla.
\end{enumeratea}
\end{esercizio}

\begin{flushright}
\(\left[a)~ k = 0 \vee k = 1 ,\quad b)~ k =-\dfrac{1}{3} \vee k = 1 
,\quad c)~ k = \pm 1 ,\quad d)~ \emptyset ,\quad e)~ k=0 \right]\)
\end{flushright}

\begin{esercizio}[\Ast]
 \label{ese:3.102}
Data l'equazione~\(kx^{2}-2kx + k-2 = 0\) determinare \(k\) affinché
\begin{enumeratea}
\item abbia una radice nulla;~
\item abbia la somma dei reciproci delle radici uguale a \(1\)
\item abbia la somma dei quadrati delle radici uguale a \(4\)
\item abbia la somma delle radici che superi di \(5\) il loro prodotto.
\end{enumeratea}
\end{esercizio}

\begin{flushright}
\(\left[a)~ k = 2 ,\quad b)~ k = -2 ,\quad c)~ k = 2 ,\quad 
d)~ k = \dfrac{1}{2} \right]\)
\end{flushright}

\begin{esercizio}[\Ast]
 \label{ese:3.103}
Data l'equazione~\(x (x-a) = \dfrac{a + x}{a + 2}\) determinate \(a\) 
affinché:
\begin{multicols}{2}
\begin{enumeratea}
\item \(x_1 = 1\);
\item l'equazione sia di primo grado;
\item \(x_1 = 1 / x_2\);
\item \(x_1 + x_2 = 2 \cdot x_1 x_2\);
\item \(x_1^2 + x_2^2 = 0\);
\item \(x_1 + x_2 = - x_1 x_2\);
\item le soluzioni siano reali e distinte;
\item l'equazione sia spuria;
\item \(x_1^3 + x_2^3 = 0\);
\item le soluzioni siano reali e discordi;
\item \(1/x_1^3 + 1/x_2^3 = 1\);
\end{enumeratea}
\end{multicols}
\end{esercizio}

\begin{flushright}
\(\left[a)~ a =-1 \pm \sqrt{2} ,\quad b)~ \emptyset ,\quad c)~ a 
=-1 ,\quad d)~ a_{1.2} =\dfrac{- 2 \pm \sqrt{3}}{2} ,\quad e)~ \emptyset 
,\quad f)~ \emptyset \right]\)
\end{flushright}

\begin{esercizio}[\Ast]
 \label{ese:3.104}
Data l'equazione~\(kx^{2}-(2k + 1) x + k-5 = 0\) determinare il valore 
di~\(k\) 
per 
il quale:
\begin{multicols}{2}
\begin{enumeratea}
\item l'equazione abbia soluzioni reali;
\item \(x_1 x_2=-2\);
\item \(x_1 + x_2 = 1\);
\item \(x_1 = -2\);
\item \(x_1 = -x_2\);
\item \(1/x_1 + 1/x_2 = 3\);
\item \(x_1 = 1/x_2\);
\item \(x_1 = -1/x_2\);
\item \(x_1^2 + x_2^2 = 4\);
\item le radici siano concordi;
\item le radici siano entrambe negative;
\item \(x_1 + x_2 = - x_1 x_2\).
\end{enumeratea}
\end{multicols}
\end{esercizio}

\begin{flushright}
\(\left[a)~ k \geq-\dfrac{1}{24} ,\quad b)~ k = \dfrac{5}{3} ,\quad 
c)~ k=-1  non accettabile,\quad d)~ k = \dfrac{1}{3} ,\quad e)~ k 
=-\dfrac{1}{2} non accettabile,\quad \right.\)
\(\left.f)~ k = 16 ,\quad g)~ \emptyset ,\quad 
i)~ k = \dfrac{7 \pm \sqrt{51}}{2} ,\quad j)~ - \dfrac{1}{24} \leq k < 0 
\vee k 
> 5 ,\quad k)~ - \dfrac{1}{24} \leq k < 0 \right]\)
\end{flushright}

\begin{esercizio}
 \label{ese:3.106}
Per quale valore di\(k \in \insR\) l'equazione~\(x^{2} + (k-2) x + 1 = 0\) 
ammette 
due soluzioni reali e distinte?
\[\boxA\quad k > 4\qquad\boxB\quad k = 0 \vee k = 4\qquad\boxC\quad 0 < k < 
4\qquad\boxD\quad k < 0 \vee k > 4\]
\end{esercizio}

\begin{esercizio}
 \label{ese:3.107}
Per quale valore di\(k\) l'equazione~\((k-1) x^{2} + kx + (k + 1) = 0\) ha 
una 
soluzione nulla?
\[\boxA\quad k = 1\qquad\boxB\quad k = -1\qquad\boxC\quad k 
=0\qquad\boxD\quad 
\text{nessun valore di }k\]
\end{esercizio}

\begin{esercizio}
 \label{ese:3.108}
Per quale valore di\(k\) l'equazione~\(kx^{2} + \dfrac{1}{2} x + 1 = 0\) ha 
due 
soluzioni identiche?
\[\boxA\quad k = \dfrac{1}{4}\qquad\boxB\quad k = 
\dfrac{1}{16}\qquad\boxC\quad k 
= 2\qquad\boxD\quad \text{nessun valore di }k\]
\end{esercizio}

\begin{esercizio}
 \label{ese:3.109}
Per quale valore di\(k\) l'equazione~\((k + 3) x^{2}-2x + k = 0\) ammette 
due 
soluzioni reciproche?
\[\boxA\quad k = 0\qquad\boxB\quad k = -3\qquad\boxC\quad \text{qualsiasi 
valore 
di }k\qquad\boxD\quad \text{nessun valore di }k\]
\end{esercizio}

\begin{esercizio}
 \label{ese:3.110}
Per quale valore di\(k\) l'equazione~\((k + 1) x^{2}-kx-4 = 0\) ha una 
soluzione 
uguale a\(2\)?
\[\boxA\quad k = 4\qquad\boxB\quad k =-2\qquad\boxC\quad k = 
0\qquad\boxD\quad k 
=-1\]
\end{esercizio}

\begin{multicols}{2}

%===================================

% \subsection*{3.9 - Problemi di secondo grado}
\subsection*{\numnameref{sec:eq2gr_problemi}}

\begin{esercizio}[\Ast]
 \label{ese:3.112}
Il quadrato di un numero reale supera la metà del numero stesso di~\(5\)
Determina i numeri reali che rendono vera la proposizione enunciata.
\hfill\(\left[-2;~5/2\right]\)
\end{esercizio}

\begin{esercizio}[\Ast]
 \label{ese:3.113}
Il prodotto della metà di un numero relativo con il suo successivo è~\(666\)
Quali numeri verificano questa proprietà?
\hfill\(\left[36;~-37\right]\)
\end{esercizio}

\begin{esercizio}
 \label{ese:3.114}
Trova un numero positivo che addizionato al proprio quadrato dia come 
somma~\(156\)
\hfill\(\left[...\right]\)
\end{esercizio}

\begin{esercizio}
 \label{ese:3.115}
Un numero addizionato al quadrato della sua metà, dà come risultato~\(120\)
Trova il numero.
\hfill\(\left[...\right]\)
\end{esercizio}

\begin{esercizio}
 \label{ese:3.116}
Verifica che non esiste alcun numero reale tale che il quadrato del suo
doppio uguagli la differenza tra il triplo del suo quadrato e il quadrato
della somma del numero con~\(3\)
\hfill\(\left[...\right]\)
\end{esercizio}

\begin{esercizio}[\Ast]
 \label{ese:3.117}
Due numeri naturali hanno rapporto~\(2/3\) e somma dei loro 
quadrati~\(3757\)
Individua i numeri che verificano questa proprietà.
\hfill\(\left[51;~34\right]\)
\end{esercizio}

\begin{esercizio}[\Ast]
 \label{ese:3.118}
La somma dei quadrati di due numeri pari consecutivi è~\(580\) Quali sono i
due numeri?
\hfill\(\left[16;~18\right]\)
\end{esercizio}

\begin{esercizio}[\Ast]
 \label{ese:3.119}
Di due numeri naturali consecutivi si sa che la somma dei loro reciproci
è~\(9/20\) Quali sono i due numeri?
\hfill\(\left[4;~5\right]\)
\end{esercizio}

\begin{esercizio}[\Ast]
 \label{ese:3.120}
Di cinque numeri interi consecutivi si sa che la differenza tra il quadrato
della somma degli ultimi due numeri e la somma dei quadrati dei primi tre è
\( 702\) Qual è il più piccolo di questi numeri?
\hfill\(\left[17\right]\)
\end{esercizio}

 \begin{esercizio}[\Ast]
 \label{ese:3.121}
La somma delle età di un padre con quella del figlio è~\(34\) Sapendo che
l'età del padre aumentata di~\(8\) anni dà il quadrato dell'età del figlio,
trovare le due età.
\hfill\(\left[28;~6\right]\)
\end{esercizio}

\begin{esercizio}[\Ast]
 \label{ese:3.122}
Determina due numeri naturali sapendo che la somma tra il doppio del minore
ed il triplo del maggiore è~\(42\) e che il rapporto tra la loro somma e il 
loro
prodotto è~\(5/12\)
\hfill\(\left[3;~12\right]\)
\end{esercizio}

\begin{esercizio}[\Ast]
 \label{ese:3.123}
Trova l'età di una persona sapendo che fra tre anni la sua età sarà
uguale al quadrato della quinta parte dell'età che aveva tre anni fa.
\hfill\(\left[33\right]\)
\end{esercizio}

\begin{esercizio}[\Ast]
 \label{ese:3.124}
Trova due numeri pari consecutivi tali che la somma del quadrato del minore
con il loro prodotto sia~\(544\)
\hfill\(\left[16;~18\right]\)
\end{esercizio}

\begin{esercizio}[\Ast]
 \label{ese:3.125}
Trova due numeri naturali sapendo che il minore supera di~\(2\) la terza 
parte
del maggiore e che il quadrato del maggiore supera di~\(68\) il quadrato del
doppio del minore.
\hfill\(\left[8;~18\right]\)
\end{esercizio}

\begin{esercizio}[\Ast]
 \label{ese:3.126}
Da un segmento di~\(25\unit{cm}\) ne vogliamo ottenere due in modo che la 
somma 
dei loro quadrati sia~\(337\)
\hfill\(\left[9;~16\right]\)
\end{esercizio}

\begin{esercizio}[\Ast]
 \label{ese:3.127}
In una frazione il numeratore e il denominatore hanno somma~\(14\), mentre 
la
somma dei loro quadrati è~\(106\) Qual è la frazione?
\hfill\(\left[5/9;~9/5\right]\)
\end{esercizio}

\begin{comment}

\begin{esercizio}[\Ast]
 \label{ese:3.128}
Due navi partono contemporaneamente da uno stesso porto e arrivano alla
stessa destinazione dopo aver percorso sulla stessa rotta a velocità
costante~\(720\unit{miglia}\) Sapendo che una delle due navi viaggia con 
una 
velocità
di 1 nodo (1 miglio all'ora) superiore a quella dell'altra nave e che 
perciò
arriva 3 ore prima a destinazione, determina le velocità in nodi delle due
navi.
\hfill\(\left[15;~16\right]\)
\end{esercizio}

\begin{esercizio}
 \label{ese:3.129}
Due navi che viaggiano su rotte perpendicolari a velocità costante si
incontrano in mare aperto. Sapendo che una delle navi viaggia a 15 nodi (1
nodo = 1 miglio all'ora), dopo quanto tempo le due navi si trovano alla
distanza di 40 miglia?
\hfill\(\left[...\right]\)
\end{esercizio}

\begin{esercizio}
 \label{ese:3.130}
Luca e Carlo bevono due aranciate in bottiglia. Nel tempo in cui Luca beve
11 sorsi, Carlo ne beve 8, ma due sorsi di Carlo equivalgono a tre di 
Luca.
Quando Carlo inizia a bere Luca ha già preso 4 sorsi. Dopo quanti sorsi di
Carlo le due bibite hanno lo stesso livello?
\hfill\(\left[...\right]\)
\end{esercizio}

\begin{esercizio}
 \label{ese:3.131}
Un maratoneta durante un allenamento fa due giri di un percorso 
di~\(22\unit{km} 
\)
mantenendo in ciascun giro una velocità costante ma nel secondo giro la
velocità è inferiore di~\(0,5\unit{km/h}\) rispetto al primo giro. A 
quali 
velocità
ha corso se ha impiegato complessivamente 2 ore e un quarto?
\hfill\(\left[...\right]\)
\end{esercizio}

\begin{esercizio}[\Ast]
 \label{ese:3.132}
Un capitale di 12000 euro è depositato in banca a un certo tasso di 
interesse
annuale. Alla scadenza del primo anno gli interessi maturati vengono
ridepositati sullo stesso conto. Alla scadenza del secondo anno si ritira 
la
somma di 12854,70 euro. Qual è stato il tasso di interesse?
\hfill\(\left[3,5\%\right]\)
\end{esercizio}

\begin{esercizio}
 \label{ese:3.133}
In un rettangolo, se si aumenta di 2 metri la base e si riduce di un metro
l'altezza, la sua area aumenta di 4 metri quadrati. Se invece si riduce di
un metro la base e si aumenta di 2 metri l'altezza, l'area aumenta di 22
metri quadrati. Quali sono le dimensioni del rettangolo?
\hfill\(\left[...\right]\)
\end{esercizio}

\begin{esercizio}[\Ast]
 \label{ese:3.134}
Una ditta spende mensilmente 73500 in stipendi per i propri dipendenti.
Aumentando di 5 il numero dei dipendenti, ma riducendo l'orario di lavoro,
diminuisce a ciascuno lo stipendio di 200 e spende solamente 2500 in più 
per
gli stipendi. Quanti dipendenti aveva inizialmente la ditta e quanto
guadagnava ognuno di essi?
\hfill\(\left[35;~2100\right]\)
\end{esercizio}

\end{comment}

\begin{esercizio}[\Ast]
 \label{ese:3.135}
Da un cartoncino rettangolare (~\(ABCD\), come in figura) si vuole 
ritagliare 
un
quadrato (~\(DEFG\)) in modo che le due parti ottenute siano equivalenti.
Determinare la misura del lato del quadrato sapendo che
\(\overline {EC} = 6\unit{cm}\) e\(\overline {AG} = 4\unit{cm}\)
\begin{center}
 \input{\folder lbr/fig007_rett.pgf}
\end{center}
\hfill\(\left[...\right]\)
\end{esercizio}

\begin{esercizio}[\Ast]
 \label{ese:3.136}
Un terreno a forma rettangolare di\(6016\unit{m^2}\) viene recintato con un 
muro 
lungo~\(350\unit{m}\) Quali sono le dimensioni del rettangolo?
\hfill\(\left[47;~128\right]\)
\end{esercizio}

\begin{esercizio}[\Ast]
 \label{ese:3.137}
Determinare sul segmento~\(AB\) di misura~\(5\unit{m}\) un punto~\(P\) tale 
che 
il rettangolo delle due parti sia equivalente al quadrato di 
lato~\(2\unit{m}\) Rappresenta con un disegno le soluzioni.
\hfill\(\left[1\unit{cm};~4\unit{cm}\right]\)
\end{esercizio}

\begin{esercizio}[\Ast]
 \label{ese:3.138}
Calcolare perimetro e area del triangolo~\(ABC\) isoscele sulla base~\(AB\) 
sapendo
che la differenza tra la base e l'altezza ad essa relativa 
è~\(0,5\unit{m}\) 
e 
tale
è anche la differenza tra il lato~\(CB\) e la base stessa.
\hfill\(\left[2p=25\unit{m};~A=30\unit{m^2}\right]\)
\end{esercizio}

\begin{esercizio}[\Ast]
 \label{ese:3.139}
La superficie del rettangolo~\(ABCD\) supera di~\(119\unit{m^2}\) la 
superficie 
del quadrato
costruito sul lato minore~\(AD~\)Determinare il perimetro e la misura della
diagonale sapendo che i~\(7/10\) del lato maggiore AB sono uguali 
ai~\(12/5\) 
del
lato minore.
\hfill\(\left[2p=62\unit{m};~d=25\unit{m}\right]\)
\end{esercizio}

\begin{esercizio}[\Ast]
 \label{ese:3.140}
Nel trapezio rettangolo~\(ABCD\), il rapporto tra la base maggiore~\(AB\) e 
la 
base
minore~\(CD\) è~\(8/5\), il lato obliquo forma con~\(AB\) un angolo di\( 
45\grado~\)Determinare il perimetro sapendo che l'area è\(312\unit{m^2}\)
\hfill\(\left[2p = 64 + 12 \sqrt{2}\right]\)
\end{esercizio}

\begin{esercizio}[\Ast]
 \label{ese:3.141}
Determina il perimetro di un rombo che ha l'area di\(24\unit{m^2}\) e il 
rapporto 
tra le diagonali~\(4/3\)
\hfill\(\left[40\unit{m}\right]\)
\end{esercizio}

\begin{esercizio}[\Ast]
 \label{ese:3.142}
Un rettangolo~\(ABCD\) ha il perimetro di~\(48\unit{cm}\) e l'area di\( 
128\unit{cm^2}~\)A una certa
distanza\(x\) dal vertice~\(A\) sui due lati~\(AD\) e~\(AB\) si prendono 
rispettivamente i
punti~\(P\) e\( Q~\)Alla stessa distanza\( x\) dal vertice~\(C\) sui 
lati~\(CB\) 
e~\(CD\) si
prendono rispettivamente i punti\( R\) e\( S\) Sapendo che il rapporto tra 
l'area
del rettangolo~\(ABCD\) e l'area del quadrilatero~\(PQRS\) è~\(32/23\) 
calcola 
la
distanza\(x\)
\hfill\(\left[6\unit{cm}\right]\)
\end{esercizio}

\begin{esercizio}
 \label{ese:3.143}
Un trapezio rettangolo ha la base minore di~\(9|unit{cm}\), l'altezza 
i~\(2/9\) 
della base
maggiore e l'area di\(20 + 9 \sqrt{2}\unit{cm^{2}}~\)Determina la misura 
della 
base maggiore.
\hfill\(\left[...\right]\)
\end{esercizio}

\begin{esercizio}[\Ast]
 \label{ese:3.144}
Da un quadrato di~\(32\unit{cm}\) di lato vengono ritagliati due triangoli 
rettangoli
come descritti in figura. Calcola la misura di \(x\),
inferiore alla metà del lato del quadrato, in modo che l'area totale dei
due triangoli evidenziati sia pari a~\(344\unit{cm^2}\)
\begin{center}
 \input{\folder lbr/fig009_quad.pgf}
\end{center}
\vspace*{-2em}
\hfill\(\left[...\right]\)
\end{esercizio}

\begin{esercizio}[\Ast]
 \label{ese:3.145}
Il rettangolo~\(ABCD\) ha l'area di~\(558\unit{cm^2}\) e il lato~\(DC\) 
di\( 
18\unit{cm}\) Lo si
vuole trasformare in un nuovo rettangolo~\(AEFG\) accorciando l'altezza di 
una 
quantità~\(5x\) e allungando la base di una quantità~\(4x\) in modo che il 
nuovo 
rettangolo~\(AEFG\) che abbia l'area di~\(228\unit{cm^2}~\).
Determina la quantità\( x\) necessaria a compiere la trasformazione 
richiesta.
\hfill\(\left[5\unit{cm}\right]\)
\end{esercizio}
% 
% \begin{esercizio}[\Ast]
%  \label{ese:3.146}
% Il rettangolo~\(AEFG\) ha l'area di~\(768\{cm^2\}\) e l'altezza~\(AG\) 
% di\( 
% 24\unit{cm}\). Si vuole allungare l'altezza di una quantità
% \( x\) e accorciare la base di una quantità doppia~\(2x\) in modo da 
% ottenere 
% un
% secondo rettangolo~\(ABCD\) che abbia l'area 
% di~\(702\unit{cm^2}~\)Determina~\(x\).
% \hfill\(\left[3\unit{cm}\right]\)
% \end{esercizio}

\begin{esercizio}
 \label{ese:3.147}
Un trapezio isoscele di area~\(144\unit{cm^2}\) ha la base maggiore che 
supera 
di~\(10\unit{cm}\)
la base minore che a sua volta supera di~\(10\unit{cm}\) l'altezza. 
Determina 
il 
perimetro del trapezio.
\hfill\(\left[...\right]\)
\end{esercizio}
% 
% \begin{esercizio}[\Ast]
%  \label{ese:3.148}
% Il rettangolo~\(ABCD\) ha l'area di~\(240\unit{cm^2}\) e l'altezza~\(AD\) 
% di\( 
% 12\unit{cm}\) Si
% vuole trasformare il rettangolo in un triangolo~\(AEF\) allungando 
% l'altezza 
% di 
% una quantità~\(3x\) e accorciando la
% base di una quantità\( x\) (vedi figura) in modo che il nuovo 
% triangolo~\(AEF\) 
% abbia l'area di~\(162\unit{cm^2}\)
% \begin{center}
%  \input{\folder lbr/fig008_rett.pgf}
% \end{center}
% \hfill\(\left[2;~14\text{ non accettabile}\right]\)
% \end{esercizio}

\begin{esercizio}[\Ast]
 \label{ese:3.149}
La piramide di Cheope è a base quadrata ed ha una superficie totale pari a
\( 135700\unit{m^2}\) Sapendo che l'apotema della piramide 
misura~\(180\unit{m}\), 
si calcoli la lunghezza del lato di base.
\hfill\(\left[230\unit{m}\right]\)
\end{esercizio}

\begin{esercizio}[\Ast]
 \label{ese:3.150}
Un container a forma di parallelepipedo a base quadrata ha una superficie
totale pari a~\(210\unit{m^2}\) L'altezza è il doppio del lato di base 
diminuito 
di \(2\) metri. Trovare la lunghezza del lato di base.
\hfill\(\left[5\unit{m}\right]\)
\end{esercizio}

\begin{comment}

\subsection*{3.10 - Problemi con un parametro}

\begin{esercizio}
 \label{ese:3.151}
Sul prolungamento dei lati~\(AB\),~\(BC\),~\(CD\),~\(DA\) del 
quadrato~\(ABCD\) 
prendi rispettivamente i punti\( Q\),\( R\),\( S\),~\(P\) in modo che\( 
QB=RC=SD=PA~\)Dimostra che~\(PQRS\) è un quadrato;~ nell'ipotesi che 
sia\(AB 
= 
3\unit{m}\) determina\(\overline {AP}\) in modo che l'area di~\(PQRS\) 
sia\( 
k\), 
con\( k\) reale positivo.
\begin{center}
 \input{\folder lbr/fig011_seg.pgf}
\end{center}
\emph{Svolgimento}:
per dimostrare che~\(PQRS\) è un quadrato dobbiamo dimostrare che i lati 
sono
congruenti e che gli angoli sono retti. Se si pone\(\overline{AP} = x\) 
con\(x > 
0\)
\(\Area(PQRS)= \overline {PQ}^{2} = \overline {PA}^{2} + 
\overline{AQ}^{2}\)per il 
teorema di Pitagora.
Verifica che si ottiene l'equazione risolvente\(2 x^{2} + 6 x + (9-k) = 
0\) 
Poiché vogliamo soluzioni reali positive, discuti l'equazione con il 
metodo di 
Cartesio. Il discriminante è\(\Delta = 36-8 (9-k)\) pertanto l'equazione 
ammette 
soluzioni reali per\(k \geq \dfrac{9}{2}~\)Dal segno dei coefficienti, 
essendo i 
primi due coefficienti positivi si ha una permanenza e quindi una radice 
negativa che non è accettabile. Per ottenere una soluzione positiva ci 
deve 
essere una variazione di segno negli ultimi due coefficienti, in altre 
parole 
\(9-k\) deve essere negativo cioè\(9-k < 0 \rightarrow k > 9~\)Pertanto 
il 
problema 
ha soluzioni per\(k > 9\)
\end{esercizio}

\begin{esercizio}
 \label{ese:3.152}
Nel trapezio rettangolo~\(ABCD\) di base maggiore~\(BC\), la 
diagonale~\(AC\) è 
bisettrice dell'angolo\(B \widehat {C} D~\)Posto\(\overline {AB} = 
1\unit{m}\), 
determina la base maggiore in modo che sia~\(2k\) il perimetro del 
trapezio. 
Imposta dati e obiettivo del problema.
\begin{center}
 \input{\folder lbr/fig012_trap.pgf}
\end{center}
\emph{Svolgimento}: poniamo\(\overline {BC} = x~\)Dall'informazione che 
la 
diagonale~\(AC\) è bisettrice dell'angolo\(B \widehat {C} D\), possiamo 
dimostrare che~\(ADC\) è un triangolo isoscele sulla base~\(AC\) 
L'equazione 
risolvente sarà determinata dalla relazione tra i lati che esprime il 
perimetro 
del trapezio. Dobbiamo quindi esprimere\(\overline {DC}\) in funzione 
di\( 
x\) 
Traccia l'altezza~\(DH\) del triangolo isoscele~\(ADC\) e dopo aver 
dimostrato 
la similitudine di~\(ABC\) con~\(DHC\), osserva che si ha\(DC : AC = HC : 
BC\) 
poiché\(HC = \dfrac{1}{2} AC\) si ha\(\dfrac{1}{2} \overline {AC}^{2} = 
\overline 
{DC} \cdot \overline {BC}\) da cui si può ricavare la misura di~\(DC = 
\dfrac{1}{2} \dfrac{AC^{2}}{BC}~\)Dato che\(\overline {AC}^{2}=1+x^2\), 
per 
il 
teorema di Pitagora applicato al triangolo~\(ABC\), quindi~\(DC = 
\dfrac{1 + 
x^2} 
{2 x}\) L'equazione parametrica risolvente è\(2 x^{2} + x \cdot (1-2 k) + 
1 
= 0\) 
con\(x > 0\) che può essere discussa con il metodo di Cartesio.
\end{esercizio}

\begin{esercizio}
 \label{ese:3.153}
Il quadrilatero~\(ABCD\) ha le diagonali perpendicolari ed è inscritto in 
una 
circonferenza;~ sapendo che\(\overline{AB} = 5a\[\overline {AE} = 3 a\]2 
p_{BCA} = \dfrac{5}{2} \cdot \overline {BD}\), essendo\( E\) punto 
d'incontro 
delle diagonali, determinate la misura delle diagonali. Poni\(\overline 
{CE} = 
x\)
\end{esercizio}

\begin{esercizio}
 \label{ese:3.154}
Il rettangolo~\(ABCD\) ha i lati~\(AB\) e~\(BC\) che misurano 
rispettivamente\( 
a\) e~\(3a\) (con~\(a\geq 0\)). Prolunga il lato~\(AB\) di due segmenti 
congruenti~\(BN\) e~\(AM\) e sia\( V\) il punto di intersezione delle 
retta\( MD 
\) e~\(CN~\)Posto\(\overline {BN} = x\), determina la misura della base\( 
MN\) 
del 
triangolo\( MVN\) in modo che la sua area sia\( k\) volte l'area del 
rettangolo 
assegnato.
\end{esercizio}

\begin{esercizio}
 \label{ese:3.155}
Due numeri reali hanno come somma~\(a\) con\((a \in \insR_{0})\) 
determinare 
i 
due numeri in modo che il loro prodotto sia\( k\) con\((k \in 
\insR_{0})\) 
Quale 
condizione si deve porre sull'incognita? Per quale valore del parametro i 
due 
numeri soluzione sono uguali?
\end{esercizio}

\begin{esercizio}
 \label{ese:3.156}
In un triangolo rettangolo l'altezza~\(AH\) relativa all'ipotenusa~\(BC\) 
misura 
~\(1\unit{m}\) e\(A \widehat {B} C = 60\grado~\)Determinare sulla 
semiretta~\(AH 
\), esternamente al triangolo, un punto~\(P\) in modo che sia\( k\) la 
somma 
dei 
quadrati delle distanze di~\(P\) dai vertici del triangolo. Quale 
condizione 
va 
imposta al parametro\( k\) perché il problema abbia significato?
\end{esercizio}

\begin{esercizio}
 \label{ese:3.157}
\(\overline {AB} = 16 a\)\(\overline {BC} = 2 a \sqrt{14}\) rappresentano 
le 
misure 
dei lati del rettangolo~\(ABCD\) determinare un punto~\(P\) del 
segmento~\(AB\) 
tale che la somma dei quadrati delle sue distanze dai vertici~\(C\) 
e~\(D\) 
sia 
uguale al quadrato della diagonale~\(DB\) Posto\(\overline {AP} = x\) 
quale 
delle 
seguenti condizioni deve rispettare la soluzione? Dopo aver risolto il 
problema 
spiegare il significato delle soluzioni ottenute.
\end{esercizio}

\begin{esercizio}
 \label{ese:3.158}
Ad una sfera di raggio~\(1\unit{m}\) è circoscritto un cono il cui volume 
è\( 
k 
\) volte il volume della sfera. Determina l'altezza del cono.
\begin{center}
 \input{\folder lbr/fig013_cono.pgf}
\end{center}

\emph{Dati}:\( \overline {OC} = 1\),\( \overline {OC} = \overline 
{OH}\),\( OC 
\perp VB\),\protect\\ \( \overline {BC} = \overline {BH}\(,\) \overline 
{AH} = 
\overline {HB}\(,\) VH \perp AB\(,\protect\\ \) \text{Volume(cono)} = k 
\cdot 
\text{Volume(sfera)}\)

\emph{Obiettivo}:\(\overline {VH}\)

\emph{Svolgimento}: Poniamo\(\overline {VO} = x\) con\(x > 0\) da 
cui\(\overline 
{VH}= \overline {VO} + \overline {OH} = x + 1\)

Ricordiamo che\(V\text{(cono)} = \dfrac{1}{3} \pi \overline {HB}^{2} 
\cdot 
\overline {VH}\) e\(V\text{(sfera)} = \dfrac{4}{3} \pi \overline 
{CO}^{3}~\)Per 
impostare l'equazione risolvente dobbiamo cercare di esprimere\(\overline 
{HB}^{2}\) in funzione di\( x\) Verifica che dalla similitudine di\( 
VOC\) 
con\( 
VHB\) si deduce:\(\overline {HB} : \overline {OC} = \overline {VH} : 
\overline{VC}\) quindi\(\overline {HB} = \dfrac{\overline {OC} \cdot 
\overline
{VH}}{\overline {VC}}\) dobbiamo ancora ricavare\(\overline {VC}\) che 
per 
il 
teorema di Pitagora su\( VCO\) è \ldots Sostituendo tutti gli elementi 
trovati 
nella relazione che lega il volume del cono con il volume della sfera, 
verifica 
che si ottiene\(x^{2} + 2 x (1-2 k) + 4 k = 0\) con\(x > 0\), da 
discutere con 
il 
metodo di Cartesio.
\end{esercizio}

\end{comment}

\end{multicols}

\begin{comment}
\begin{esercizio}[\Ast]
 \label{ese:3.159}
 Scheda di ripasso sulle equazioni
\begin{enumerate}
	\item L'equazione~\(25x^{2} + 1 = 0\) ha per soluzioni:

\boxA\quad\( x = \pm 5\)\qquad \boxB\quad\(x = \pm 
\dfrac{1}{5}\)\qquad\boxC\quad 
\(x = 4 \vee x = 1\)\qquad\boxD\quad non ha soluzioni reali

	\item L'equazione~\(16x^{2} + x = 0\) ha per soluzioni:

\boxA\quad\( x = 4 \vee x = 1\)\quad \boxB\quad\(x = \pm 
\dfrac{1}{4}\)\quad\boxC\quad\(x =-\dfrac{1}{16} \vee x = 
0\)\quad\boxD\quad 
non ha 
soluzioni reali

	\item L'equazione~\(4x^{2}-9x = 0\) ha per soluzioni:

\boxA\quad\( x = \pm \dfrac{3}{2}\)\qquad \boxB\quad\(x = \pm 
\dfrac{9}{4}\)\qquad\boxC\quad\(x = \dfrac{3}{2} \vee x = 
0\)\qquad\boxD\quad\(x = 
\dfrac{9}{4} \vee x = 0\)

	\item L'equazione~\(9x^{2} + 6x + 1 = 0\) ha per soluzioni:

\boxA\quad\(x = \pm 3\)\qquad\boxB\quad\(x = \pm 
\dfrac{1}{3}\)\qquad\boxC\quad\(x 
=-\dfrac{1}{3}\) doppia\qquad\boxD\quad non ha soluzioni reali

	\item L'equazione~\(x^{2}-6x + 36 = 0\) ha per soluzioni:

\boxA\quad\(x = \pm 6\)\qquad\boxB\quad\(x = \pm 
\sqrt{6}\)\qquad\boxC\quad\(x 
=6\) 
doppia\qquad\boxD\quad non ha soluzioni reali

	\item Quale di queste equazioni ammette una soluzione doppia\( 
x=3\)?

\boxA\quad\(2x^{2}-12x + 18 = 0\)\quad\boxB\quad\(9-x^{2} = 
0\)\quad\boxC\quad 
\(x^{2} + 6x + 9 = 0\)\quad\boxD\quad~\(3x^{2} + 9x = 0\)

	\item Quale equazione di secondo grado si ottiene con soluzioni\( 
x_1=1 
\) e\( x_2=3\)?

\boxA\quad\(x^{2} + x-1 = 0\)\quad\boxB\quad\(x^{2}-4x + 3 = 
0\)\quad\boxC\quad 
\(x^{2}-4x-3 = 0\)\quad\boxD\quad\( x^{2} + 4x-3 = 0\)

	\item Il polinomio\(x^{2} + 5x + 6\) può essere scomposto in:

\boxA\;~\((x + 2) (x-3)\)\quad\boxB\;~\((x + 5) (x + 
1)\)\quad\boxC\;~\((x-2) 
(x-3)\)\quad\boxD\;~ nessuna delle risposte precedenti

	\item Una delle soluzioni dell'equazione~\(x^{2}-(\sqrt{2} + 1) x + 
\sqrt{2} = 0\) è\(\sqrt{2}\), quanto vale l'altra?

\boxA\quad\( - \sqrt{2}\)\qquad 
\boxB\quad\(\dfrac{1}{\sqrt{2}}\)\qquad\boxC\quad 
\(\sqrt{2} + 1\)\qquad\boxD\quad\(1\)

	\item Per quale valore di\( k\) l'equazione~\((2k-1) x^{2} + (2k + 
1) x 
+ 
k-2 = 0\) diventa di I° grado?

\boxA\quad\(k = \dfrac{1}{2}\)\qquad \boxB\quad\(k 
=-\dfrac{1}{2}\)\qquad\boxC\quad 
\(k = 2\)\qquad\boxD\quad\(k = 0\)

	\item L'equazione~\(4m^{2} x^{2}-5mx + 1 = 0\) con parametro\( m\) 
ha 
per 
soluzioni:

\boxA\;~\(x = m \vee x = 4m\)\quad \boxB\;~\(x = \dfrac{1}{m} \vee x = 
\dfrac{1}{4m}\)\quad\boxC\;~\(x = 64m \vee x = 1\)\quad\boxD\;~\(x = m 
\vee x 
= 
\dfrac{1}{4}\)

	\item L'equazione di secondo grado\(x^{2} + (a + 1) x + a = 0\) 
con~\(a\) 
parametro reale ha come soluzioni:

\boxA\;~\(x = 1 \vee x = a\)\quad \boxB\;~\(x = a-1 \vee x = 
1\)\quad\boxC\;~\(x =-a 
\vee x =-1\)\quad\boxD\;~\(x = a + 1 \vee x = a\)

	\item L'equazione~\(x^{2} + (t-2) = 0\) con\( t\) parametro reale 
ammette 
soluzioni reali per:

\boxA\quad\(t \leq 2\)\qquad \boxB\quad\(t \geq 2\)\qquad\boxC\quad\(t < 
2\)\qquad\boxD\quad nessuna delle risposte precedenti

	\item Quanto vale il prodotto delle soluzioni dell'equazione 
\(x^{2}-6a^{2} x + 8a^{4} = 0\)?

\boxA\quad\(8a^{4}\)\qquad \boxB\quad\(8a^{2}\)\qquad\boxC\quad 
\(6a^{2}\)\qquad\boxD\quad non esiste

	\item Il polinomio\(x^{2} + (m-2) x-2m\) con\( m\) parametro reale 
può 
essere scomposto in:

\boxA\;~\((x + m) (x + 1)\)\quad\boxB\;~\((x + m) 
(x-2)\)\quad\boxC\;~\((x + m) 
(x + 
2)\)\quad\boxD\;~\((x-m) (x-2)\)

	\item L'equazione~\(x^{2} + (k-1) x = 0\) con\( k\) parametro reale:

\boxA\;~ non ha soluzioni reali\quad\boxB\;~ ha una soluzione uguale a 
zero

\boxC\;~ due soluzioni reali coincidenti per\( 
k=0\)\quad\boxD\;~soluzioni 
reali e 
distinte per\( k=1\)

	\item L'equazione~\(x^{2} + 2x + k-2 = 0\) con\( k\) parametro 
reale:

\boxA\quad ha due soluzioni reali coincidenti per\( k=3\)

\boxB\quad ha due soluzioni reali coincidenti per\( k=1\)

\boxC\quad ha una soluzione nulla per\(k =-2\)

\boxD\quad ha soluzioni reali e distinte per\(k \neq 3\)

	\item L'equazione~\(x^{2} + m^{2} + 1 = 0\) con\(m\) parametro 
reale:

\boxA\;~ ammette due soluzioni reali e opposte\quad\boxB\;~ ammette due 
soluzioni 
coincidenti

\boxC\;~ non ammette soluzioni reali\quad\boxD\;~ ammette due soluzioni 
negative

	\item L'equazione~\(2x^{2} + k^{2} = 0\) con\(k\) parametro reale 
ammette:

\boxA\;~ due soluzioni reali e distinte\quad\boxB\;~ due soluzioni reali 
solo se 
\(k>0\)

\boxC\;~ soluzioni coincidenti per\(k = 0\)\quad\boxD\;~ nessuna delle 
risposte 
precedenti è corretta

	\item L'equazione~\(tx^{2}-1 = 0\)

\boxA\;~ ha come soluzioni\(x_{1} = 0 \vee x_{2} = 1-t\)\quad\boxB\;~ 
ammette 
sempre soluzioni reali

\boxC\;~ ammette soluzioni reali per\(t > 0\)\quad\boxD\;~ ha come 
soluzioni\(x = 
\pm t\)

\end{enumerate}
\end{esercizio}

\paragraph{3.159.} 1.D - 2.C - 3.D - 4.C - 5.D - 6.A - 7.B - 8.D - 9.D - 
10.A - 11.B - 12.C - 13.A - 14.A - 15.B - 16.B - 17.A - 18.C - 19.C - 20.C



(c)~2014 Claudio Carboncini - claudio.carboncini@gmail.com
(c)~2014 Dimitrios Vrettos - d.vrettos@gmail.com
(c) 2015 Daniele Zambelli daniele.zambelli@gmail.com

\section{Esercizi}

\subsection{Esercizi dei singoli paragrafi}

\subsubsection*{\numnameref{sec:01_}}

\begin{esercizio}
\label{ese:D.19}
testo esercizio
\end{esercizio}

\begin{esercizio}\label{ese:03.1}
Consegna:
 \begin{enumeratea}
  \item  
 \end{enumeratea}
\end{esercizio}

\subsection{Esercizi riepilogativi}

\begin{esercizio}
\label{ese:D.19}
testo esercizio
\end{esercizio}

\begin{esercizio}\label{ese:03.1}
Consegna:
 \begin{enumeratea}
  \item  
 \end{enumeratea}
\end{esercizio}

\end{comment}

\subsection*{\numnameref{sec:eq2gr_sistemi}}

\begin{esercizio}[\Ast]
 \label{ese:6.1}
Risolvere i seguenti sistemi di secondo grado.
\begin{multicols}{2}
 \begin{enumeratea}
 \item~\(\sistema{x^2+2y^2=3\\x+y=2}\)
\hfill\(\left[\left(1;~1\right) \vee 
        \left(\dfrac 5 3;~\dfrac{1}{3}\right)\right]\)
 \item~\(\sistema{4x^2+2y^2-6=0\\x=y}\)
\hfill\(\left[\left(1;~1\right)\vee \left(-1;~-1\right)\right]\)
 \item~\(\sistema{y^2-3y=2xy\\y=x-3}\)
\hfill\(\left[\left(3;~0\right)\vee \left(-6;~-9\right)\right]\)
 \item~\(\sistema{xy-x^2+2y^2=y-2x\\x+y=0}\)
\hfill\(\left[\left(0;~0\right)\right]\)
 \item~\(\sistema{{x+2y=-1}\\{x+5y^2=23}}\)
\hfill\(\left[\left(-\dfrac{29} 5,\dfrac{12} 5\right);~   
              (3,-2)\right]\)
 \item~\(\sistema{{x-5y=2}\\{x^2+2y^2=4}}\)
\hfill\(\left[\left(-\dfrac{46}{27},-\dfrac{20}{27}\right);~
(2,0)\right]\)
 \item~\(\sistema{3x-y=2\\x^2+2xy+y^2=0}\)
\hfill\(\left[\left(\dfrac 1 2;~-\dfrac 1 2\right)\right]\)
 \item~\(\sistema{x^2+y^2=1\\x+3y=10}\)
\hfill\(\left[\emptyset\right]\)
 \item~\(\sistema{x^2+y^2=2\\x+y=2}\)
\hfill\(\left[\punto{1}{1}\right]\)
 \item~\(\sistema{3x+y=2\\x^2-y^2=1}\)
\hfill\(\left[\emptyset\right]\)
%  \item~\(\sistema{x+y=0\\x^2+y^2-x-10=0}\)
% \hfill\(\left[\left(-2;~2\right)\vee \left(\dfrac 5 2;~-\dfrac 5 
% 2\right)\right]\)
%  \item~\(\sistema{x^2-4y^2=0\\4x-7y=2}\)
% \hfill\(\left[\left(4;~2\right)\vee \left(\dfrac 4{15};~-\dfrac 
% 2{15}\right)\right]\)
 \end{enumeratea}
 \end{multicols}
\end{esercizio}

\begin{esercizio}[\Ast]
 \label{ese:6.1}
Risolvere i seguenti sistemi di secondo grado.
 \begin{enumeratea}
 \item~\(\sistema{x^2-4xy+4y^2-1=0\\x=y+2}\)
\hfill\(\left[\left(3;~1\right)\vee \left(5;~3\right)\right]\)
 \item~\(\sistema{2x^2-6xy=x\\3x+5y=-2}\)
\hfill\(\left[\left(0;~-\dfrac 2 5\right)\vee 
       \left(-\dfrac 1 4;~-\dfrac 1 4\right)\right]\)
 \item~\(\sistema{x+y=1\\x^2+y^2-3x+2y=3}\)
\hfill\(\left[\left(0;~1\right)\vee \left(\dfrac 7 2;~-\dfrac 5 
2\right)\right]\)
 \item~\(\sistema{x^2+y^2=25\\4x-3y+7=0}\)
\hfill\(\left[\left(-4;~-3\right) \vee 
       \left(\dfrac{44}{25};~\dfrac{117}{25}\right)\right]\)
 \item~\(\sistema{x^2-4xy+4y^2-1=0\\x=y+2}\)
\hfill\(\left[\left(3;~1\right)\vee \left(5;~-3\right)\right]\)
 \item~\(\sistema{2x^2+xy-7x-2y=-6\\2x+y=3}\)
\hfill\(\left[\left(1;~-3\right)\vee \left(-8;~-\dfrac{15} 2\right)\right]\)
%  \item~\(\sistema{x+y=0\\x^2+y^2-x-10=0}\)
% \hfill\(\left[\left(-2;~2\right)\vee \left(\dfrac 5 2;~-\dfrac 5 
% 2\right)\right]\)
%  \item~\(\sistema{x^2-4y^2=0\\4x-7y=2}\)
% \hfill\(\left[\left(4;~2\right)\vee \left(\dfrac 4{15};~-\dfrac 
% 2{15}\right)\right]\)
 \end{enumeratea}
\end{esercizio}

\begin{esercizio}[\Ast]
 \label{ese:6.7}
Risolvere i seguenti sistemi di secondo grado.
 \begin{enumeratea}
 \item~\(\sistema{3x^2-4y^2-x=0\\x-2y=1}\)
\hfill\(\left[\left(-1;~-1\right)\vee \left(\dfrac 1 2;~-\dfrac 1 
4\right)\right]\)
 
\item~\(\sistema{5x^2-y^2+4y-2x+2=0\\x-y=1}\)
\hfill\(\left[\left(-\dfrac 3 2;~-\dfrac 5 2\right) \vee 
       \left(\dfrac 1 2;~-\dfrac 1 2\right)\right]\)
 
\item~\(\sistema{x+2y=3\\x^2-4xy+2y^2+x+y-1=0}\)
\hfill\(\left[\left(1;~1\right)\vee \left(\dfrac{10} 7;
    \dfrac{11}{14}\right)\right]\)
 \item~\(\sistema{x-2y-7=0\\x^2-xy=4}\)
\hfill\(\left[\left(1;~-3\right)\vee \left(22;~\dfrac{15}{2}\right)\right]\)
% \hfill\(\left[\forall (x,y)\in \insR \times \insR:\,y=-2x+3\right]\) 
% ????????
\item~\(\sistema{x^2+2y^2-3xy-x+2y-4=0\\2x-3y+4=0}\)
\hfill\(\left[\left(4;~4\right)\vee \left(-5;~-2\right)\right]\)
 \item~\(\sistema{x-2y=1\\x^2+y^2-2x=1}\)
\hfill\(\left[\left(1+\dfrac{2\sqrt{10}} 5;~\dfrac{\sqrt{10}} 5\right)\vee 
\left(1-\dfrac{2\sqrt{10}} 5;~-\dfrac{\sqrt{10}} 5\right)\right]\)
 
\item~\(\sistema{x+y=1\\x^2+y^2-2xy-2y-2=0}\)
\hfill\(\left[\left(\dfrac{1+\sqrt{13}} 4;~\dfrac{3-\sqrt{13}} 4\right)\vee 
\left(\dfrac{1-\sqrt{13}} 4;~\dfrac{3+\sqrt{13}} 4\right)\right]\)
%  
% \item~\(\sistema{9x^2-12xy+4y^2-2x+6y=8\\x-2y=2}\)
% \hfill\(\left[\left(\dfrac{-9+\sqrt{241}} 
% 8;~\dfrac{-25+\sqrt{241}}{16}\right)\vee 
% \left(\dfrac{9-\sqrt{241}} 8;~\dfrac{-25-\sqrt{241}}{16}\right)\right]\)
%  \item~\(\sistema{3x+y=4\\x^2-y^2=1}\)
% \hfill\(\left[\left(\dfrac{6-\sqrt 2} 4;~\dfrac{-2+3\sqrt 2} 4\right)\vee 
% \left(\dfrac{6+\sqrt 2} 4;~\dfrac{-2-3\sqrt 2} 4\right)\right]\)
%  \item~\(\sistema{\dfrac 1 2(2y-x)(y+x)-(x+y)^2+\dfrac 3 
% 2x(x+y+1)+2(y-1)=0\\\dfrac 2 3(x-3)^2+4\left(x-\dfrac 3 
% 2\right)=2({xy}+1)}\)
% \hfill\(\left[\left(\dfrac{6-8\sqrt 
% 3}{13};~\dfrac{17+12\sqrt 3}{26}\right)\vee\left(\dfrac{6+ 8\sqrt 
% 3}{13};~\dfrac{17-12\sqrt 3}{26}\right)\right]\)
 \end{enumeratea}
\end{esercizio}

\begin{comment}

\begin{esercizio}[\Ast]
 \label{ese:6.8}
Risolvere i seguenti sistemi, dopo aver eseguito la discussione sul 
parametro.
 \begin{enumeratea}
 \item~\(\sistema{x+y=3 \\x^2+y^2=k }\)
\hfill\(\left[k\ge \dfrac 9 2.~\left(\dfrac{3-\sqrt{2k-9}} 2;~ 
\dfrac{3+\sqrt{2k-9}} 2\right)\vee \left(\dfrac{3+\sqrt{2k-9}} 2;~ 
\dfrac{3-\sqrt{2k-9}} 2\right)\right]\)
 \item~\( \sistema{ky+2x=4\\xy=2}\)
\hfill\(\left[...\right]\)
 \item~\( \sistema{y=kx-1 \\y^2-kx^2+1=0}\)
\hfill\(\left[...\right]\)
 \item~\(\sistema{y=kx-2k \\x^2-2y-x=2}\)
\hfill\(\left[\forall k\in \insR:~(2;~ 0)\vee (2k-1;~ 2k^2-3k)\right]\)
 \item~\(\sistema{y=x+k \\y=3x^2+2x}\)
 \item~\( \sistema{y=-x+k \\x^2-y^2-1=0}\)
 \item~\(\sistema{y+x-k=0 
\\{xy}+2{kx}-3{ky}-6k^2=0}\)
 \item~\( \sistema{y-x+k=0 
\\y-x^2+4x-3=0}\)
 \end{enumeratea}
\end{esercizio}

\begin{esercizio}[\Ast]
 \label{ese:6.10}
Trovare le soluzioni dei seguenti sistemi frazionari.
\begin{multicols}{2}
 \begin{enumeratea}
 
\item~\(\sistema{x^2+y^2=4\\\dfrac{x+2y}{x-1}=2}}
\right.\)
 


\item~\(\sistema{\dfrac{x+2y}{x-y}=4\\x^2+y^2+3x-2y=1}\)
 
\item~\(\sistema{\dfrac{2x+y}{x+2y}=3\\xy+3y=1}\)
 \item~\(\sistema{\dfrac{3x-2y} 
x=\dfrac{1-x}{y-1}\\2x-y=1}\)
 \end{enumeratea}
 \end{multicols}
\end{esercizio}

\paragraph{6.9.} a)~\(k\ge -\dfrac 1{12}:~\left(\dfrac{-1-\sqrt{12k+1}} 
6;~ 
\dfrac{6k-1-\sqrt{12k+1}} 6\right) \vee \left(\dfrac{-1+\sqrt{12k+1}} 6;~ 
\dfrac{6k-1+\sqrt{12k+1}} 6\right)\),\protect\\
\quad d)~\(\forall k\in \insR:~(3k;~ -2k)\)

\paragraph{6.10.} a)~\(x\neq 1:~\left(2;~0\right)\vee \left(-\dfrac 6 
5;~-\dfrac 8 
5\right)\),\quad b)~\(x\neq y:~\left(\dfrac 2 5;~\dfrac 1 5\right)\vee 
\left(-2;~-1\right)\),\quad c)~\(x\neq -2y:~\emptyset\),\protect\\
\quad d)~\(x\neq 0\wedge y\neq 1:~(4;~7)\)

\begin{esercizio}[\Ast]
 \label{ese:6.11}
Trovare le soluzioni dei seguenti sistemi frazionari.
\begin{multicols}{2}
 \begin{enumeratea}
 \item~\(\sistema{\dfrac{x+y}{x-2}=y+\dfrac 1 
3\\y=2x+2}\)
 
\item~\(\sistema{
\dfrac{2x+1}{y-2}=\dfrac{y-1}{x+1}\\2x+2y=3
}\)
 

\item~\(\sistema{\dfrac{y-1}{x+y}=x\\x-y=0}\)
 
\item~\(\sistema{\dfrac{x+1}{2y-1}=y\\2y-x=-4}\)
 \end{enumeratea}
 \end{multicols}
\end{esercizio}

\begin{esercizio}
 \label{ese:6.12}
Risolvere i seguenti sistemi di secondo grado in tre incognite.
\begin{multicols}{2}
 \begin{enumeratea}
 


\item~\(\sistema{x-3y-z=-4\\3x+2y+z=6\\4x^2+2xz+y^2=6}\)
 
\item~\(\sistema{x+y=5\\2x-y+3z=9\\x^2-y+z^2=1}\)
 
\item~\(\sistema{x-y+z=1\\2x-y+z=0\\x^2-y+z=3}\)
 

\item~\(\sistema{x-y+2z=3\\2x-2y+z=1\\x^2-y^2+z=12}\)
 \end{enumeratea}
 \end{multicols}
\end{esercizio}

\paragraph{6.11.} a)~\(x\neq 2:~\left(-1;~0\right)\vee \left(\dfrac{10} 
3;~\dfrac{26} 3\right)\),\quad b)~\(x\neq -1\wedge y\neq 2:~\left(-\dfrac 
5 
2;~4\right)\),\quad c)~\(x\neq -y:~\emptyset\),\protect\\
\quad d)~\(y\neq \dfrac 1 2:~\left(2;~-1\right)\vee \left(9;~\dfrac 5 
2\right)\)

\paragraph{6.12.} a)~\(\left(1;~2;~-1\right)\),\quad 
b)~\(\emptyset\),\quad 
c)~\(\forall z \in \insR (-1;~z-2;~z)\),\quad d)~\(\left(-\dfrac{47} 
3;~-\dfrac{46} 
3;~\dfrac 5 3\right)\)

\begin{esercizio}[\Ast]
 \label{ese:6.13}
Risolvere i seguenti sistemi di secondo grado in tre incognite.
\begin{multicols}{2}
 \begin{enumeratea}
 
\item~\(\sistema{2x-3y=-3\\5y+2z=1\\x^2+y^2+z^2=1}\)
 


\item~\(\sistema{x-2y+z=3\\x+2y+z=3\\x^2+y^2+z^2=29}\)
 
\item~\(\sistema{x+y-z=0\\x-y+3z=9\\x^2-y+z=12}\)
 

\item~\(\sistema{x-y=1\\x+y+z=0\\x^2+xy-z=0}\)
 

\item~\(\sistema{x-y-z=-1\\x+y+z=1\\x+y^2+z^2=32}\)
 \end{enumeratea}
 \end{multicols}
\end{esercizio}

\paragraph{6.13.} a)~\(\emptyset\),\quad b)~\(\left(5;~0;~-2\right)\vee 
\left(-2;~0;~5\right)\),\quad c)~\(\left(-4;~\dfrac{25} 2;~\dfrac{17} 
2\right)\vee 
\left(3;~-\dfrac 3 2;~\dfrac 3 2\right)\),\quad 
d)~\(\left(-1;~-2;~3\right)\vee 
\left(\dfrac 1 2;~-\dfrac 1 2;~0\right)\),\quad 
e)~\(\left(0;~\dfrac{3\sqrt 
7+1} 
2;~-\dfrac{3\sqrt 7-1} 2\right)\vee \left(0;~-\dfrac{3\sqrt 7-1} 
2;~\dfrac{3\sqrt 7+1} 
2\right)\)

\subsection*{5.2 - Equazioni riconducibili al prodotto di due o più 
fattori}

\end{comment}

\subsection*{\numnameref{sec:eq2gr_scomponibili}}

\begin{esercizio}[\Ast]
 \label{ese:5.1}
Trovare gli zeri dei seguenti polinomi.
\begin{multicols}{2}
 \begin{enumeratea}
 \item~\(x^3+5x^2-2x-24\) \hfill \(\left[-4;~-3;~2\right]\)
 \item~\(6x^3+23x^2+11x-12\) \hfill \(\left[\dfrac 1 2;~-3;~-\dfrac 4 
3\right]\)
 \item~\(8x^3-40x^2+62x-30\) \hfill \(\left[\dfrac 5 2;~1;~\dfrac 3 
2\right]\)
 \item~\(x^3+10x^2-7x-196\) \hfill \(\left[4;~-7\right]\)
 \item~\(x^3+\dfrac 4 3x^2-\dfrac{17} 3x-2\) 
  \hfill \(\left[-3;~-\dfrac 1 3;~+2\right]\)
 \item~\(x^3-\dfrac 1 3x^2-\dfrac{38} 3x+\dfrac{56} 3\) 
  \hfill \(\left[-4;~+\dfrac 7 3;~+2\right]\)
 \item~\(3x^3-\dfrac 9 2x^2+\dfrac 3 2x\) \hfill \(\left[0;~+\dfrac 1 
2;~+1\right]\)
 \item~\(3x^3-9x^2-9x-12\) \hfill \(\left[+4\right]\)
 \item~\(\dfrac 6 5x^3+\dfrac{42} 5x^2+\dfrac{72} 5x+12\) \hfill 
\(\left[-5\right]\)
 \item~\(4x^3-8x^2-11x-3\) \hfill \(\left[3;~-\dfrac 1 2\right]\)
 \item~\(\dfrac 3 2x^3-4x^2-10x+8\) \hfill \(\left[4;~\dfrac 2 
3;~-2\right]\)
 \item~\(\dfrac 3 2x^3-4x^2-10x+8\) \hfill \(\left[2;1;-\dfrac 1 2\right]\)
 \item~\(-3x^3+9x-6\) \hfill \(\left[1;~-2\right]\)
 \item~\(\dfrac 1 2x^3-3x^2+6x-4\) \hfill \(\left[2\right]\)
 \item~\(4x^3+4x^2-4x-4\) \hfill \(\left[1;~-1\right]\)
 \item~\(\dfrac 2 5x^3+\dfrac 8 5x^2+\dfrac{14} 5x-4\) \hfill 
\(\left[5;~1;~-2\right]\)
 \item~\(-6x^3-30x^2+192x-216\) \hfill \(\left[2;~-9\right]\)
 \item~\(x^3-2x^2-x+2\) \hfill \(\left[1;~-1;~2\right]\)
 \item~\(9x^3-7x+2\) \hfill \(\left[-1;~\dfrac 1 2;~\dfrac 2 3\right]\)
 \item~\(x^3-7x^2+4x+12\) \hfill \(\left[-1;~6;~2\right]\)
 \item~\( x^3+10x^2-7x-196 \) \hfill \(\left[...\right]\)
 \item~\( 400x^3-1600x^2\) \hfill \(\left[...\right]\)
%  \item~\(x^6-5x^5+6x^4+4x^3-24x^2+16x+32 \) \hfill \(\left[...\right]\)
 \item~\( 8x^3-14{ax}^2-5a^2x+2a^3 \) \hfill \(\left[...\right]\)
 \item~\( x^4-x^3-x^2-x-2 \) \hfill \(\left[...\right]\)
 \item~\( 3x^5-19x^4+42x^3-42x^2+19x-3 \) \hfill \(\left[...\right]\)
%  \item~\( {ax}^3-(a^2+1-a)x^2-(a^2+1-a)x+a \) \hfill \(\left[...\right]\)
 \end{enumeratea}
 \end{multicols}
\end{esercizio}

% \subsection*{6.2 - Sistemi simmetrici}
\subsection*{\numnameref{sec:eq2gr_sistemi_simmetrici}}

\begin{esercizio}[\Ast]
 \label{ese:6.14}
Risolvere i seguenti sistemi simmetrici di secondo grado.
\begin{multicols}{2}
 \begin{enumeratea}
 \item~\(\sistema{x+y=4\\{xy}=3}\)
  \hfill\(\left[(3;~1)\vee(1;~3)\right]\)
 \item~\(\sistema{x+y=1\\{xy}=7 }\)
  \hfill\(\left[\emptyset\right]\)
 \item~\(\sistema{x+y=5\\{xy}=6 }\)
  \hfill\(\left[(3;~2)\vee(2;~3)\right]\)
 \item~\(\sistema{x+y=-5\\{xy}=-6 }\)
  \hfill\(\left[(1;~-6)\vee(-6;~1)\right]\)
 \item~\(\sistema{x+y=3\\{xy}=2 }\)
  \hfill\(\left[(2;~1)\vee(1;~2)\right]\)
 \item~\(\sistema{x+y=3\\{xy}=-4}\)
  \hfill\(\left[(4;~-1)\vee(-1;~4)\right]\)
 \item~\(\sistema{x+y=-4\\{xy}=4 }\)
  \hfill\(\left[(-2;~-2)\right]\)
 \item~\(\sistema{x+y=6\\{xy}=9 }\)
  \hfill\(\left[(3;~3)\right]\)
 \item~\(\sistema{x+y=2\\{xy}=10 }\)
  \hfill\(\left[\emptyset\right]\)
 \item~\(\sistema{x+y=7\\{xy}=12 }\)
  \hfill\(\left[(4;~3)\vee(3;~4)\right]\)
 \item~\(\sistema{x+y=-1\\{xy}=2 }\)
  \hfill\(\left[\emptyset\right]\)
 \item~\(\sistema{x+y=12\\{xy}=-13 }\)
  \hfill\(\left[(13;~-1)\vee(-1;~13)\right]\)
%  \item~\(\sistema{
%   x+y=\dfrac 6 5\\{xy}=\dfrac 9{25}
%  }\)
%   \hfill\(\left[\left(\dfrac 3 5;~\dfrac 3 5\right)\right]\)
%  \item~\(\sistema{x+y=4\\{xy}=50 }\)
%   \hfill\(\left[\emptyset\right]\)
%  \item~\(\sistema{x+y=-5\\{xy}=-14 }\)
%   \hfill\(\left[(2;~-7)\vee(-7;~2)\right]\)
%  \item~\(\sistema{x+y=5\\{xy}=-14}\)
%   \hfill\(\left[(7;~-2)\vee(-2;~7)\right]\)
%  \item~\(\sistema{x+y=\dfrac 1 4\\{xy}=-\dfrac 3 
% 8}\)
%   \hfill\(\left[\left(\dfrac 3 4;~-\dfrac 1 2\right)\vee
%          \left(-\dfrac 1 2;~\dfrac 3 4\right)\right]\)
%  \item~\(\sistema{x+y=2\\{xy}=-10}\)
%   \hfill\(\left[\left(1+\sqrt{11};~1-\sqrt{11}\right)\vee
%          \left(1-\sqrt{11};~1+\sqrt{11}\right)\right]\)
 \end{enumeratea}
 \end{multicols}
\end{esercizio}

\begin{comment}

\begin{esercizio}[\Ast]
 \label{ese:6.18}
Risolvere i seguenti sistemi simmetrici di secondo grado.
\begin{multicols}{2}
 \begin{enumeratea}
 \item~\(\sistema{x+y=4\\{xy}=0 }\)
  \hfill\(\left[\right]\)
 \item~\(\sistema{x+y=\dfrac 5 2\\{xy}=-\dfrac 7 
2}\)
 \item~\(\sistema{x+y=-5\\{xy}=2 }\)
 \item~\(\sistema{x+y=\dfrac 4 3\\{xy}=-\dfrac 1 2 
}\)
 \end{enumeratea}
 \end{multicols}
\end{esercizio}

\begin{esercizio}[\Ast]
 \label{ese:6.19}
Risolvere i seguenti sistemi simmetrici di secondo grado.
\begin{multicols}{2}
 \begin{enumeratea}
 \item~\(\sistema{x+y=\dfrac 5 2\\{xy}=-\dfrac 9 
2}\)
 \item~\(\sistema{x+y=2\\{xy}=-\dfrac 1 
3}\)
 \item~\(\sistema{x+y=1\\{xy}=-3}\)
 \item~\(\sistema{x+y=4\\{xy}=-50}\)
 \end{enumeratea}
 \end{multicols}
\end{esercizio}

\paragraph{6.18.} a)~\((0;~4)\vee(4;~0)\),\;~ b)~\(\left(\dfrac 7 
2;~-1\right)\vee\left(-1;~\dfrac 7 2\right)\),\;~ 
c)~\(\left(\dfrac{-5+\sqrt{17}} 
2;~\dfrac{-5-\sqrt{17}} 2\right)\vee \left(\dfrac{-5-\sqrt{17}} 
2\\y=\dfrac{-5+\sqrt{17}} 2\right)\),\quad d)~\(\left(\dfrac{4+\sqrt{34}} 
6;~\dfrac{4-\sqrt{34}} 6\right)\vee \left(\dfrac{4-\sqrt{34}} 
6\\y=\dfrac{4+\sqrt{34}} 6\right)\)

\paragraph{6.19.} a)~\(\left(\dfrac{5+\sqrt{97}} 4\\y=\dfrac{5-\sqrt{97}} 
4\right)\vee \left(\dfrac{5-\sqrt{97}} 4\\y=\dfrac{5+\sqrt{97}} 
4\right)\),\quad 
b)~\(\left(\dfrac{3+{2\sqrt 3}} 3;~\dfrac {3-{2\sqrt 3}} 3\right)\vee 
\left(\dfrac 
{3-{2\sqrt 3}} 3;~\dfrac{3+{2\sqrt 3}} 3\right)\),\quad 
c)~\(\left(\dfrac{1+\sqrt{13}} 2;~\dfrac{1-\sqrt{13}} 2\right)\vee 
\left(\dfrac{1-\sqrt{13}} 2;~\dfrac{1+\sqrt{13}} 2\right)\),\quad 
d)~\(\left(2+3\sqrt 
6;~2-3\sqrt 6\right)\vee \left(2-3\sqrt 6;~2+3\sqrt 6\right)\)

\begin{esercizio}[\Ast]
\label{ese:6.20}
Risolvere i seguenti sistemi riconducibili al sistema simmetrico 
fondamentale.
\begin{multicols}{2}
 \begin{enumeratea}
 \item~\(\sistema{x+y=1 \\x^2+y^2=1}\)
 \item~\(\sistema{x+y=2 \\x^2+y^2=2}\)
 \item~\(\sistema{x+y=3 \\x^2+y^2=5}\)
 \item~\(\sistema{x+y=2 \\x^2+y^2+x+y=1}\)
 \item~\(\sistema{x+y=4\\x^2+y^2=8}\)
 \item~\(\sistema{x+y=2 \\x^2+y^2-3xy=4}\)
 \end{enumeratea}
 \end{multicols}
\end{esercizio}

\begin{esercizio}[\Ast]
\label{ese:6.21}
Risolvere i seguenti sistemi riconducibili al sistema simmetrico 
fondamentale.
\begin{multicols}{2}
 \begin{enumeratea}
 
\item~\(\sistema{{x+y=-12}\\{x^2+y^2=72}}\)
 
\item~\(\sistema{{2x+2y=-2}\\{(y-x)^2-{xy}=101}}\)
 


\item~\(\sistema{{-4x-4y=-44}\\{2x^2+2y^2-3{xy}=74}}\)
 \item~\(\sistema{x+y=3 
\\x^2+y^2-4x-4y=5}\)
 \end{enumeratea}
 \end{multicols}
\end{esercizio}

\begin{esercizio}[\Ast]
 \label{ese:6.22}
Risolvere i seguenti sistemi riconducibili al sistema simmetrico 
fondamentale.
\begin{multicols}{2}
 \begin{enumeratea}
 \item~\(\sistema{x+y=7 \\x^2+y^2=29}\)
 \item~\(\sistema{2x+2y=-2\\4x^2+4y^2=52}\)
 \item~\(\sistema{\dfrac{x+y} 2=\dfrac 3 
4\\3x^2+3y^2=\dfrac{15} 
4}\)
 \item~\(\sistema{x+y=-3 
\\x^2+y^2-5xy=37}\)
 \end{enumeratea}
 \end{multicols}
\end{esercizio}

\begin{esercizio}[\Ast]
\label{ese:6.23}
Risolvere i seguenti sistemi riconducibili al sistema simmetrico 
fondamentale.
\begin{multicols}{2}
 \begin{enumeratea}
 \item~\(\sistema{x+y=-6 \\x^2+y^2-xy=84}\)
 \item~\(\sistema{x+y=-5 
\\x^2+y^2-4xy+5x+5y=36}\)
 \item~\(\sistema{x+y=-7 
\\x^2+y^2-6xy-3x-3y=44}\)
 \item~\(\sistema{x^2+y^2=-1\\x+y=6 }\)
 \end{enumeratea}
 \end{multicols}
\end{esercizio}

\begin{esercizio}[\Ast]
\label{ese:6.24}
Risolvere i seguenti sistemi riconducibili al sistema simmetrico 
fondamentale.
\begin{multicols}{2}
 \begin{enumeratea}
 \item~\(\sistema{x^2+y^2=1\\x+y=-7}\)
 \item~\(\sistema{x^2+y^2=18\\x+y=6 }\)
 \item~\(\sistema{x^2+y^2-4xy-6x-6y=1\\x+y=1 
}\)
 \item~\(\sistema{x^2+y^2=8\\x+y=3}\)
 \end{enumeratea}
 \end{multicols}
\end{esercizio}
\newpage
\begin{esercizio}[\Ast]
\label{ese:6.25}
Risolvere i seguenti sistemi riconducibili a sistemi simmetrici.
\begin{multicols}{3}
 \begin{enumeratea}
 \item~\(\sistema{{x-y=1}\\{x^2+y^2=5}}\)
 \item~\(\sistema{{\dfrac 1 x+\dfrac 1 y=-12}\\{{xy}=\dfrac 
1{35}}}\)
 \item~\(\sistema{{-2x+y=3}\\{{xy}=1}}\)
 \end{enumeratea}
 \end{multicols}
\end{esercizio}

\begin{esercizio}[\Ast]
 \label{ese:6.26}
Risolvere i seguenti sistemi di grado superiore al secondo.
\begin{multicols}{2}
 \begin{enumeratea}
 \item~\(\sistema{{x+y=-1}\\{x^3+y^3=-1}}\)
 
\item~\(\sistema{{{xy}=-2}\\{x^2+y^2=13}}\)
 

\item~\(\sistema{{x+y=-2}\\{x^3+y^3-{xy}=-5}}\)
 \item~\(\sistema{{x+y=8}\\{x^3+y^3=152}}\)
 \end{enumeratea}
 \end{multicols}
\end{esercizio}

\begin{esercizio}[\Ast]
 \label{ese:6.27}
Risolvere i seguenti sistemi di grado superiore al secondo.
\begin{multicols}{2}
 \begin{enumeratea}
 \item~\(\sistema{x^3+y^3=9\\x+y=3}\)
 \item~\(\sistema{x^3+y^3=-342\\x+y=-6}\)
 
\item~\(\sistema{{x^3-y^3=351}{{xy}=-14}}\)
 \item~\(\sistema{x^3+y^3=35\\x+y=5}\)
 \end{enumeratea}
 \end{multicols}
\end{esercizio}

\begin{esercizio}[\Ast]
 \label{ese:6.28}
Risolvere i seguenti sistemi di grado superiore al secondo.
\begin{multicols}{2}
 \begin{enumeratea}
 \item~\(\sistema{x^4+y^4=2\\x+y=0}\)
 \item~\(\sistema{x^4+y^4=17\\x+y=-3}\)
 \item~\(\sistema{x^3+y^3=-35\\xy=6}\)
 \item~\(\sistema{x^3+y^3=-26\\xy=-3}\)
 \end{enumeratea}
 \end{multicols}
\end{esercizio}

\begin{esercizio}[\Ast]
 \label{ese:6.29}
Risolvere i seguenti sistemi di grado superiore al secondo.
\begin{multicols}{2}
 \begin{enumeratea}
 \item~\(\sistema{{x+y=3}\\{x^4+y^4=17}}\)
 
\item~\(\sistema{{x+y=-1}\\{8x^4+8y^4=41}}\)
 \item~\(\sistema{{x+y=3}\\{x^4+y^4=2}}\)
 \item~\(\sistema{{x+y=5}\\{x^4+y^4=257}}\)
 \end{enumeratea}
 \end{multicols}
\end{esercizio}

\begin{esercizio}[\Ast]
 \label{ese:6.30}
Risolvere i seguenti sistemi di grado superiore al secondo.
\begin{multicols}{2}
 \begin{enumeratea}
 \item~\(\sistema{x^4+y^4=2\\xy=1}\)
 \item~\(\sistema{x^4+y^4=17\\xy=-2}\)
 
\item~\(\sistema{{x+y=-1}\\{x^5+y^5=-211}}\)
 \item~\(\sistema{x^5+y^5=64\\x+y=4}\)
 \end{enumeratea}
 \end{multicols}
\end{esercizio}

\begin{esercizio}[\Ast]
 \label{ese:6.31}
Risolvere i seguenti sistemi di grado superiore al secondo.
\begin{multicols}{2}
 \begin{enumeratea}
 \item~\(\sistema{x^5+y^5=-2882\\x+y=-2}\)
 \item~\(\sistema{x^5+y^5=2\\x+y=0}\)
 \item~\(\sistema{x^5+y^5=31\\xy=-2}\)
 \item~\(\sistema{x^4+y^4=337\\xy=12}\)
 \end{enumeratea}
 \end{multicols}
\end{esercizio}
\newpage
\begin{esercizio}[\Ast]
 \label{ese:6.32}
Risolvere i seguenti sistemi di grado superiore al secondo.
\begin{multicols}{2}
 \begin{enumeratea}
 \item~\(\sistema{x^3+y^3=\dfrac{511} 
8\\xy=-2}\)
 \item~\(\sistema{x^2+y^2=5\\xy=2 }\)
 \item~\(\sistema{x^2+y^2=34\\xy=15 }\)
 \item~\(\sistema{xy=1 \\x^2+y^2+3xy=5}\)
 \end{enumeratea}
 \end{multicols}
\end{esercizio}

\begin{esercizio}[\Ast]
 \label{ese:6.33}
Risolvere i seguenti sistemi di grado superiore al secondo.
\begin{multicols}{2}
 \begin{enumeratea}
 \item~\(\sistema{xy=12 \\x^2+y^2=25}\)
 \item~\(\sistema{xy=1 \\x^2+y^2-4xy=-2}\)
 \item~\(\sistema{x^2+y^2=5\\xy=3 }\)
 \item~\(\sistema{x^2+y^2=18\\xy=9 }\)
 \end{enumeratea}
 \end{multicols}
\end{esercizio}

\begin{esercizio}[\Ast]
 \label{ese:6.34}
Risolvere i seguenti sistemi di grado superiore al secondo.
\begin{multicols}{2}
 \begin{enumeratea}
 \item~\(\sistema{x^2+y^2+3xy=10\\xy=6 }\)
 \item~\(\sistema{x^2+y^2+5xy-2x-2y=3\\xy=1 
}\)
 \item~\(\sistema{x^2+y^2-6xy+3x+3y=2\\xy=2 
}\)
 \item~\(\sistema{x^2+y^2=8\\xy=-3}\)
 \end{enumeratea}
 \end{multicols}
\end{esercizio}

\begin{esercizio}[\Ast]
\label{ese:6.35}
Risolvere i seguenti sistemi di grado superiore al secondo.
 \begin{enumeratea}
 \item~\(\sistema{x^2+y^2+5xy+x+y=-6\\xy=-2 
}\)
 \item~\(\sistema{{x+y=-\dfrac 1 
3}\\{x^5+y^5=-\dfrac{31}{243}}}\)
 \item~\(\sistema{x^2+y^2+5xy+x+y=-\dfrac{25} 4\\xy=-2 
}\)
 \item~\(\sistema{{x+y=1}\\{x^5+y^5=-2}}\)
 

\item~\(\sistema{{x+y=1}\\{x^5+y^5+7{xy}=17}}\)
 \end{enumeratea}
\end{esercizio}

\subsection*{6.3 - Sistemi omogenei di quarto grado}

\begin{esercizio}[\Ast]
 \label{ese:6.36}
Risolvi i seguenti sistemi omogenei.
\begin{multicols}{2}
 \begin{enumeratea}
 
\item~\(\sistema{x^2-2xy+y^2=0\\x^2+3xy-2y^2=0}\)
 
\item~\(\sistema{3x^2-2xy-y^2=0\\2x^2+xy-3y^2=0}\)
 \item~\(\sistema{x^2-6{xy}+8y^2=0 \\x^2+4{xy}-5y^2=0 
}\)
 
\item~\(\sistema{2x^2+xy-y^2=0\\4x^2-2xy-6y^2=0}\)
 \end{enumeratea}
\end{multicols}
\end{esercizio}

\paragraph{6.20.} a)~\((1,0)\vee(0,1)\),\quad b)~\((1,1)\),\quad 
c)~\((1,2)\vee(2,1)\),\quad d)~\(\emptyset\),\quad e)~\((2,2)\),\quad 
f)~\((0,2)\vee(2,0)\)

\paragraph{6.21.} a)~\((-6,-6)\),\quad b)~\((-5,4)\vee(4,-5)\),\quad 
c)~\((3,8)\vee(8,3)\),\quad d)~\((-1,4)\vee(4,-1)\)

\paragraph{6.22.} a)~\((2,5)\vee(5,2)\),\quad 
b)~\((-3,2)\vee(2,-3)\),\quad 
c)~\((\dfrac 1 2;~1)\vee(1;~\dfrac 1 2)\),\quad d)~\((-4;~1)\vee(1;~-4)\)

\paragraph{6.23.} a)~\((-8;~2)\vee(2;~-8)\),\quad 
b)~\((-6;~1)\vee(1;~-6)\),\quad 
c)~\(\left(-\dfrac 1 2;~-\dfrac{13} 2\right)\vee \left(-\dfrac{13} 
2;~-\dfrac 1 
2\right)\),\quad d)~\(\emptyset\)

\paragraph{6.24.} a)~\(\emptyset\),\;~ b)~\((3;~3)\),\;~ 
c)~\(\left(\dfrac{1+\sqrt 5} 
2;~\dfrac{1-\sqrt 5} 2\right)\vee \left(\dfrac{1-\sqrt 5} 
2;~\dfrac{1+\sqrt 5} 
2\right)\),\;~ d)~\(\left(\dfrac{3+\sqrt 7} 2;~\dfrac{3-\sqrt 7} 
2\right)\vee 
\left(\dfrac{3-\sqrt 7} 2;~\dfrac{3+\sqrt 7} 2\right)\)

\paragraph{6.25.} a)~\( (-1;~ -2)\vee(2;~1)\),~b)~\(\left(-\dfrac 1 
7;~-\dfrac 1 
5\right)\vee\left(-\dfrac 1 5;~-\dfrac 1 
7\right)\),~c)~\(\left(\dfrac{-3-\sqrt{17}} 
4;~{\dfrac{3-\sqrt{17}} 2}\right)\vee \left(\dfrac{-3+\sqrt{17}} 
4;~{\dfrac{3+\sqrt{17}} 2}\right)\)

\paragraph{6.26.} a)~\((-1;~0)\vee(0;~-1)\),\quad 
b)~\(\left(\dfrac{-3-\sqrt{17}} 
2;~\dfrac{-3+\sqrt{17}} 2\right)\vee\left(\dfrac{-3+\sqrt{17}} 
2;~\dfrac{-3-\sqrt{17}} 2\right)\),\protect\\
\quad c)~\(\left(\dfrac{-5-\sqrt{10}} 5;~\dfrac{-5+\sqrt{10}} 
5\right)\vee 
\left(\dfrac{-5+\sqrt{10}} 5;~\dfrac{-5-\sqrt{10}} 5\right)\),\quad 
d)~\((3;~5)\vee(5;~3)\)

\paragraph{6.27.} a)~\((1;~2)\vee(2;~1)\),\quad 
b)~\((-7;~1)\vee(1;~-7)\),\quad 
c)~\((2;~-7)\vee(7;~-2)\),\quad d)~\((2;~3)\vee(3;~2)\)

\paragraph{6.28.} 
a)~\((-1;~1)\vee(1;~-1)\),\;~b)~\((-2;~-1)\vee(-1;~-2)\),\;~ 
c)~\((-3;~-2)\vee(-2;~-3)\),\;~ d)~\((-3;~1)\vee(1;~-3)\)

\paragraph{6.29.} a)~\((1;~2)\vee(2;~1)\),\quad b)~\(\left(-\dfrac 3 
2;~\dfrac 1 
2\right)\vee\left(\dfrac 1 2;~-\dfrac 3 2\right)\),\quad 
c)~\(\emptyset\),\quad 
d)~\((1;~4)\vee(4;~1)\)

\paragraph{6.30.} a)~\((1;~1)\),\quad 
b)~\((1;~-2)\vee(-2;~1)\vee(-1;~2)\vee(2;~-1)\),\quad 
c)~\((-3;~2)\vee(2;~-3)\),\quad 
d)~\((2;~2)\)

\paragraph{6.31.} a)~\((-5;~3)\vee(3;~-5)\),\quad b)~\(\emptyset\),\quad 
c)~\((-1;~2)\vee(2;~-1)\),\quad 
d)~\((-4;~-3)\vee(-3;~-4)\vee(3;~4)\vee(4;~3)\)

\paragraph{6.32.} a)~\(\left(-\dfrac 1 2;~4\right)\vee\left(4;~-\dfrac 1 
2\right)\),\quad b)~\((-2;~-1)\vee(-1;~-2)\vee(1;~2)\vee(2;~1)\),\quad 
c)~\((-5;~-3)\vee(-3;~-5)\vee(3;~5)\vee(5;~3)\),\quad 
d)~\((-1;~-1)\vee(1;~1)\)

\paragraph{6.33.} a)~\((-4;~-3)\vee(-3;~-4)\vee(3;~4)\vee(4;~3)\),\quad 
b)~\((-1;~-1)\vee(1;~1)\),\quad c)~\(\emptyset\),\quad 
d)~\((-3;~-3)\vee(3;~3)\)

\paragraph{6.34.} a)~\(\emptyset\),\quad b)~\((1;~1)\),\quad 
c)~\((-3-\sqrt 
7;~-3+\sqrt 
7)\vee(-3+\sqrt 7;~-3-\sqrt 7)\vee(1;~2)\vee(2;~1)\),\protect\\
\quad d)~\(\left(\dfrac{\sqrt{14}+\sqrt 2} 2;~\dfrac{\sqrt 2-\sqrt{14}} 
2\right)\vee\left(\dfrac{\sqrt{14}-\sqrt 2} 2;~\dfrac{\sqrt 2+\sqrt{14}} 
2\right)\vee\left(\dfrac{\sqrt{14}-\sqrt 2} 2;~-\dfrac{\sqrt 2+\sqrt{14}} 
2\right)\vee\left(\dfrac{\sqrt{14}+\sqrt 2} 2;~-\dfrac{\sqrt 2-\sqrt{14}} 
2\right)\)

\paragraph{6.35.} a)~\((-2;~1)\vee(1;~-2)\vee(-\sqrt 
{2};~\sqrt{2})\vee(\sqrt 
{2};~-\sqrt{2})\),\quad b)~\(\left(-\dfrac 2 3;~\dfrac 1 
3\right)\vee\left(\dfrac 1 
3;~-\dfrac 2 3\right)\),\protect\\
\quad c)~\(\left(\dfrac{-1+\sqrt{33}} 4;~\dfrac{-1-\sqrt{33}} 
4\right)\vee 
\left(\dfrac{-1-\sqrt{33}} 4;~\dfrac{-1+\sqrt{33}} 4\right)\),\quad 
d)~\(\emptyset\),\quad e)~\((-1;~2)\vee(2;~-1)\)

\paragraph{6.36.} a)~\((0;~0)\),\quad b)~\((t;~t)\),\quad 
c)~\((0;~0)\),\quad 
d)~\((t;~-t)\)

\begin{esercizio}[\Ast]
\label{ese:6.37}
Risolvi i seguenti sistemi omogenei.
\begin{multicols}{2}
 \begin{enumeratea}
 
\item~\(\sistema{x^2-5xy+6y^2=0\\x^2-4xy+4y^2=0}\)
%  
\item~\(\sistema{x^2-5xy+6y^2=0\\x^2+2xy-8y^2=0}\)
%  
\item~\(\sistema{x^2+xy-2y^2=0\\x^2+5xy+6y^2=0}\)
%  
% 
\item~\(\sistema{x^2+7xy+12y^2=0\\2x^2+xy+6y^2=0}\)
 \end{enumeratea}
\end{multicols}
\end{esercizio}

\begin{esercizio}[\Ast]
\label{ese:6.38}
Risolvi i seguenti sistemi omogenei.
\begin{multicols}{2}
 \begin{enumeratea}
 

\item~\(\sistema{x^2+6xy+8y^2=0\\2x^2+12xy+16y^2=0}\)
 \item~\(\sistema{-4x^2-7{xy}+2y^2=0 \\12x^2+21{xy}-6y^2=0 
}\)
 
\item~\(\sistema{x^2+2xy+y^2=0\\x^2+3xy+2y^2=0}\)
 

\item~\(\sistema{x^2+4xy=0\\x^2+2xy-4y^2-4=0}\)
 \end{enumeratea}
\end{multicols}
\end{esercizio}

\begin{esercizio}[\Ast]
\label{ese:6.39}
Risolvi i seguenti sistemi omogenei.
\begin{multicols}{2}
 \begin{enumeratea}
 
\item~\(\sistema{x^2-8xy+15y^2=0\\x^2-2xy+y^2=1}\)
 \item~\(\sistema{4x^2-y^2=0\\x^2-y^2=-3}\)
 
\item~\(\sistema{x^2+3xy+2y^2=0\\x^2-3xy-y^2=3}\)
 

\item~\(\sistema{x^2-4xy+4y^2=0\\2x^2-y^2=-1}\)
 \end{enumeratea}
\end{multicols}
\end{esercizio}

\begin{esercizio}[\Ast]
\label{ese:6.40}
Risolvi i seguenti sistemi omogenei.
\begin{multicols}{2}
 \begin{enumeratea}
 
\item~\(\sistema{6x^2+5xy+y^2=12\\x^2+4xy+y^2=6}\)
 
\item~\(\sistema{x^2-xy-2y^2=0\\x^2-4xy+y^2=6}\)
 
\item~\(\sistema{x^2+y^2=3\\x^2-xy+y^2=3}\)
 
\item~\(\sistema{x^2-3xy+5y^2=1\\x^2+xy+y^2=1}\)
 \end{enumeratea}
\end{multicols}
\end{esercizio}

 \begin{esercizio}[\Ast]
\label{ese:6.41}
Risolvi i seguenti sistemi omogenei.
\begin{multicols}{2}
 \begin{enumeratea}
 

\item~\(\sistema{x^2+y^2=5\\x^2-3xy+y^2=11}\)
 

\item~\(\sistema{x^2+5xy+4y^2=10\\x^2-2xy-3y^2=-11}\)
 \item~\(\sistema{4x^2-xy-y^2=-\dfrac 1 
2\\x^2+2xy-y^2=\dfrac 
1 
4}\)
 
\item~\(\sistema{x^2-xy-8y^2=-8\\x^2-2y^2-xy=16}\)
 \end{enumeratea}
\end{multicols}
 \end{esercizio}

\begin{esercizio}
\label{ese:6.42}
Risolvi i seguenti sistemi omogenei.
\begin{multicols}{2}
 \begin{enumeratea}
 

\item~\(\sistema{x^2-6xy-y^2=10\\x^2+xy=-2}\)
 
\item~\(\sistema{4x^2-3xy+y^2=32\\x^2+3y^2-9xy=85}\)
 
\item~\(\sistema{x^2+3xy+2y^2=8\\3x^2-y^2+xy=-4}\)
 


\item~\(\sistema{x^2+5xy-7y^2=-121\\3xy-3x^2-y^2=-7}\)
 \end{enumeratea}
\end{multicols}
\end{esercizio}

\paragraph{6.37.} a)~\((2t;~t)\),\quad b)~\((2t;~t)\),\quad 
c)~\((-2t;~t)\),\quad 
d)~\((0;~0)\)

\paragraph{6.38.} a)~\((-4t;~t)\vee(-2t;~t)\),\quad 
b)~\((k;~4k)\vee(k;~-\dfrac{1 
2}k)\),\quad c)~\((-t;~t)\),\quad d)~\((-4;~1)\vee(4;~-1)\)

\paragraph{6.39.} a)~\(\left(-\dfrac 3 2;~-\dfrac 1 
2\right)\vee\left(\dfrac 3 
2;~\dfrac 1 2\right)\vee\left(-\dfrac 5 4;~-\dfrac 1 
4\right)\vee\left(\dfrac 5 
4;~\dfrac 1 4\right)\),\quad b)~\((1;~2)\vee(-1;~-2)\vee (-1;~2)\vee 
(1;~-2)\),\protect\\
\quad c)~\((-1;~1)\vee (1;~-1)\left(-\dfrac{2\sqrt 3} 3;~\dfrac{\sqrt 3} 
3\right)\vee\left(\dfrac{2\sqrt 3} 3;~-\dfrac{\sqrt 3} 3\right)\),\quad 
d)~\(\emptyset\)

\paragraph{6.40.} a)~\((1;~1)\vee(-1;~-1)\vee\left(\sqrt 6;~-4\sqrt 
6\right)\vee\left(-\sqrt 6;~4\sqrt 6\right)\),\quad 
b)~\((1;~-1)\vee(-1;~1)\),\protect\\
\quad c)~\((\sqrt 3;~0)\vee (-\sqrt 3;~0)\vee (0;~\sqrt 3)\vee (0;~-\sqrt 
3)\),\quad 
d)~\((1;~0)\vee (-1;~0)\vee\left(\dfrac{\sqrt 3} 3;~\dfrac{\sqrt 3} 
3\right)\vee\left(-\dfrac{\sqrt 3} 3;~-\dfrac{\sqrt 3} 3\right)\)

\paragraph{6.41.} a)~\((1;~-2)\vee(-1;~2)\vee (-2;~1)\vee (2;~-1)\),\quad 
b)~\((2;~-3)\vee(-2;~3)\),\protect\\
\quad c)~\(\left(\dfrac 1 2;~1\right)\vee \left(-\dfrac 1 
2;~-1\right)\),\quad 
d)~\((4;~-2)\vee(-4;~2)\vee (6;~2)\vee (-6;~-2)\)

\paragraph{6.42.} a)~\((-1;~3)\vee(1;~-3)\),\quad 
b)~\((1;~-4)\vee(-1;~4)\vee 
(-1;~-7)\vee (1;~7)\),\protect\\
\quad c)~\((0;~2)\vee (0;~-2)\vee \left(\dfrac{10} 3;~-\dfrac{14} 
3\right)\vee 
\left(-\dfrac{10} 3;~\dfrac{14} 3\right)\),\quad d)~\((2;~5)\vee 
(-2;~-5)\vee 
\left(-\dfrac{18} 7;~-\dfrac{37} 7\right)\vee \left(\dfrac{18} 
7;~\dfrac{37} 7\right)\)

\begin{esercizio}[\Ast]
\label{ese:6.43}
Risolvi i seguenti sistemi particolari.
\begin{multicols}{2}
 \begin{enumeratea}
 


\item~\(\sistema{x^2-5xy-3y^2=27\\-2x^2-2y^2+4xy=-50}\)
 
\item~\(\sistema{9x^2+5y^2=-3\\x^2+4xy-3y^2=8}\)
 


\item~\(\sistema{2x^2-4xy-3y^2=18\\xy-2x^2+3y^2=-18}\)
 \item~\(\sistema{x^2+2xy=-\dfrac 7 
4\\x^2-4xy+4y^2=\dfrac{81} 
4}\)
 \end{enumeratea}
\end{multicols}
\end{esercizio}
\newpage
\begin{esercizio}[\Ast]
\label{ese:6.44}
Risolvi i seguenti sistemi particolari.
\begin{multicols}{2}
 \begin{enumeratea}
 


\item~\(\sistema{x^2+4xy+4y^2-16=0\\x^2-xy+4y^2-6=0}\)
 
\item~\(\sistema{x^2-2xy+y^2-1=0\\x^2-2xy-y^2=1}\)
 \end{enumeratea}
\end{multicols}
\end{esercizio}

\begin{esercizio}
\label{ese:6.45}
Risolvi i seguenti sistemi particolari.
\begin{multicols}{2}
 \begin{enumeratea}
 \item~\(\sistema{x^2-y^2=0\\2x+y=3}\)
 
\item~\(\sistema{(x-2y)(x+y-2)=0\\3x+6y=3}\)
 
\item~\(\sistema{(x+y-1)(x-y+1)=0\\x-2y=1}\)
 
\item~\(\sistema{(x-3y)(x+5y-2)=0\\)x-2)(x-y+4)=0}\)
 \end{enumeratea}
\end{multicols}
\end{esercizio}

\begin{esercizio}[\Ast]
\label{ese:6.46}
Risolvi i seguenti sistemi particolari.
\begin{multicols}{2}
 \begin{enumeratea}
 
\item~\(\sistema{(x^2-3x+2)(x+y)=0\\x-y=2}\)
 
\item~\(\sistema{(x-y)(x+y+1)(2x-y-1)=0\\)x-3y-3)(x+y-2)=0
}\)
 \item~\(\sistema{(4x^2-9y^2)(x^2-2xy+y^2-9)=0 \\2x-y=2 
}\)
 

\item~\(\sistema{x^2+6xy+9y^2-4=0\\)x^2-y^2)(2x-y-4)=0}\)
 \end{enumeratea}
\end{multicols}
\end{esercizio}

\begin{esercizio}[\Ast]
 \label{ese:6.47}
Risolvi i seguenti sistemi particolari.
 \begin{enumeratea}
 
\item~\(\sistema{x^2-2xy-8y^2=0\\)x+y)(x-3)=0}\)
 \item~\(\sistema{(2x^2-3xy+y^2)(x-y-1)=0 
\\)x^2-4xy+3y^2)(12x^2-xy-y^2)=0 }\)
 \item~\(\sistema{(x-2y-2)(x^2-9y^2)=0 
\\)4x^2-4xy+y^2)(y+2)(x-y)=0 
}\)
 \item~\(\sistema{x^4-y^4=0 \\x^2-(y^2-6y+9)=0 
}\)
 \end{enumeratea}
\end{esercizio}

\begin{esercizio}[\Ast]
 \label{ese:6.48}
Risolvi i seguenti sistemi particolari.
 \begin{enumeratea}
 \item~\(\sistema{(y^2-4y+3)(x^2+2x-15)=0 
\\)x^2-3xy+2y^2)(9x^2-6xy+y^2)=0 }\)
 \item~\(\sistema{(x-y)(x+4y-4)(x+y-1)(3x-5y-2)=0 
\\)3x+y-3)(x^2-4y^2)=0 }\)
 \end{enumeratea}
\end{esercizio}

\paragraph{6.43.} a)~\((3;~-2)\vee(-3;~2)\vee\left(\dfrac{34} 7;~-\dfrac 
1 
7\right)\vee \left(-\dfrac{34} 7;~\dfrac 1 7\right)\),\quad 
b)~\(\emptyset\),\quad 
c)~\((-3;~0)\vee(3;~0)\),\protect\\
\quad d)~\(\left(\dfrac 1 2;~-2\right)\vee\left(-\dfrac 1 
2;~2\right)\vee\left(\dfrac 
7 4;~-\dfrac{11} 8\right)\vee\left(-\dfrac 7 4;~\dfrac{11} 8\right)\)

\paragraph{6.44.} a)~\((-2;~-1)\vee(2;~1)\),\quad b)~\((-1;~0)\vee(1;~0)\)

\paragraph{6.45.} a)~\((1;~1)\vee(3;~-3)\),\quad 
b)~\((3;~-1)\vee\left(\dfrac 
1 2;~\dfrac 
1 4\right)\),\quad c)~\((1;~0)\vee(-3;~-2)\),\protect\\
\quad d)~\((2;~0)\vee\left(2;~\dfrac 2 3\right)\vee(-6;~-2)\vee(-3;~1)\)

\paragraph{6.46.} a)~\((1;~-1)_\text{doppia} \vee(2;~0)\),\quad 
b)~\((0;~-1)_\text{doppia}\vee\left(-\dfrac 3 2;~-\dfrac 3 
2\right)\vee(1;~1)_\text{doppia}\),\quad c)~\((5;~8)\vee\left(\dfrac 3 
2;~1\right)\vee(-1;~-4)\vee\left(\dfrac 3 4;~-\dfrac 1 2\right)\),\quad 
d)~\((1;~-1)\vee(2;~0)\vee(-1;~1)\vee\left(\dfrac 1 2;~\dfrac 1 
2\right)\vee\left(-\dfrac 1 2;~-\dfrac 1 2\right)\vee\left(\dfrac{10} 
7;~-\dfrac 8 
7\right)\)

\paragraph{6.47.}a)~\((0;~0)_\text{doppia}\vee\left(3;~-\dfrac 3 
2\right)\vee(3;~\dfrac 3 4)\),\quad b)~\((t;~t)\vee\left(\dfrac 3 
2;~\dfrac 
1 
2\right)\vee\left(\dfrac 1 5;~-\dfrac 4 5\right)\vee\left(-\dfrac 1 
2;~-\dfrac 3 
2\right)\),\protect\\
\quad 
c)~\((0;~0)_\text{tripla}\vee(-2;~-2)_\text{doppia}\vee\left(-\dfrac 
2 
3,-\dfrac 4 3\right)_\text{doppia}\vee(6;~-2)\vee(-6;~-2)\),\quad 
d)~\(\left(\dfrac 3 
2;~\dfrac 3 2\right)\vee\left(-\dfrac 3 2;~\dfrac 3 2\right)\)

\paragraph{6.48.} 
a)~\((1;~1)\vee(2;~1)\vee(3;~3)_\text{doppia}\vee(6;~3)\vee\left(\dfrac 1 
3;~1\right)\vee(1;~3)\vee(-5;~-5)\vee\left(-5;~-\dfrac 5 
2\right)\vee\left(3;~\dfrac 3 
2\right)\vee(-5;~-15)\vee(3;~9)\),\quad 
b)~\((0;~0)_\text{doppia}\vee\left(\dfrac 3 
4;~\dfrac 3 4\right)\vee(1;~0)\vee\left(\dfrac 8{11,}\dfrac 
9{11}\right)\vee\left(\dfrac{17}{18};~\dfrac 1 6\right)\vee\left(\dfrac 4 
3;~\dfrac 2 
3\right)\vee\protect\\
(-4;~2)(2;~-1)\vee\left(\dfrac 2 3;~\dfrac 1 3\right)\vee\left(\dfrac 
4{11};~-\dfrac 
2{11}\right)\vee(4;~2)\)

\subsection*{6.4 - Problemi che si risolvono con sistemi di grado 
superiore al 
primo}
\begin{multicols}{2}
\begin{esercizio}[\Ast]
 \label{ese:6.49}
La differenza tra due numeri è\(\dfrac {11} 4\) e il loro 
prodotto\(\dfrac 
{21} 8\) 
Trova i due numeri.
\end{esercizio}

\begin{esercizio}[\Ast]
 \label{ese:6.50}
Trovare due numeri positivi sapendo che la metà del primo supera di~\(1\) 
il 
secondo e che il quadrato del secondo supera di~\(1\) la sesta parte del 
quadrato del primo.
\end{esercizio}

\begin{esercizio}[\Ast]
 \label{ese:6.51}
Data una proporzione tra numeri naturali conosciamo i due medi che 
sono~\(5\) e 
\( 16\) Sappiamo anche che il rapporto tra il prodotto degli estremi e la 
loro 
somma è uguale a\(\dfrac {10} 3\) Trovare i due estremi.
\end{esercizio}

\begin{esercizio}[\Ast]
 \label{ese:6.52}
La differenza tra un numero di due cifre con quello che si ottiene 
scambiando le 
cifre è uguale a~\(36\) La differenza tra il prodotto delle cifre e la 
loro 
somma è uguale a~\(11\) Trovare il numero.
\end{esercizio}

\begin{esercizio}[\Ast]
 \label{ese:6.53}
Oggi la differenza delle età tra un padre e sua figlia è~\(26\) anni, 
mentre 
due 
anni fa il prodotto delle loro età era~\(56~\)Determina l'età del padre e 
della 
figlia.
\end{esercizio}

\begin{esercizio}[\Ast]
 \label{ese:6.54}
La somma delle età di due fratelli oggi è~\(46\) anni, mentre fra due 
anni 
la 
somma dei quadrati delle loro età sarà~\(1250\) Trova l'età dei due 
fratelli.
\end{esercizio}

\begin{esercizio}[\Ast]
 \label{ese:6.55}
Nella produzione di un oggetto la macchina A impiega 5 minuti in più 
rispetto 
alla macchina B. Determinare il numero di oggetti che produce ciascuna 
macchina 
in 8 ore se in questo periodo la macchina A ha prodotto 16 oggetti in 
meno 
rispetto alla macchina B.
\end{esercizio}

\begin{esercizio}[\Ast]
 \label{ese:6.56}
In un rettangolo la differenza tra i due lati è uguale a\(2\unit{cm}\) Se 
si 
diminuiscono entrambi i lati di~\(1\unit{cm}\) si ottiene un'area di 
\(0,1224\unit{m^2}~\)Calcolare il perimetro del rettangolo.
\end{esercizio}

\begin{esercizio}[\Ast]
 \label{ese:6.57}
Trova due numeri sapendo che la somma tra i loro quadrati è~\(100\) e il 
loro 
rapporto\( \dfrac 3 4\)
\end{esercizio}

\begin{esercizio}[\Ast]
 \label{ese:6.58}
Ho comprato due tipi di vino. In tutto 30 bottiglie. Per il primo tipo ho 
speso 
54 € e per il secondo 36 €. Il prezzo di una bottiglia del secondo tipo 
costa 
2,5 € in meno di una bottiglia del primo tipo. Trova il numero delle 
bottiglie 
di ciascun tipo che ho acquistato e il loro prezzo unitario.
\end{esercizio}

\begin{esercizio}[\Ast]
 \label{ese:6.59}
In un triangolo rettangolo di area\(630\unit{m^2}\), l'ipotenusa misura 
\(53\unit{m}~\)Determinare il perimetro.
\end{esercizio}

\begin{esercizio}[\Ast]
 \label{ese:6.60}
Un segmento di\(35\unit{cm}\) viene diviso in due parti. La somma dei 
quadrati 
costruiti su ciascuna delle due parti è\(625\unit{{cm}^2}\) Quanto misura 
ciascuna parte?
\end{esercizio}

\begin{esercizio}[\Ast]
 \label{ese:6.61}
Se in un rettangolo il perimetro misura~\(16,8\unit{m}\) e l'area\( 
17,28\unit{m^2}\), quanto misura la sua diagonale?
\end{esercizio}

\begin{esercizio}[\Ast]
 \label{ese:6.62}
In un triangolo rettangolo la somma dei cateti misura~\(10,5\unit{cm}\), 
mentre 
l'ipotenusa è~\(7,5\unit{cm}\) Trovare l'area.
\end{esercizio}

\begin{esercizio}[\Ast]
 \label{ese:6.63}
Quanto misura un segmento diviso in due parti, tali che una parte è\( 
\dfrac 3 4 
\) dell'altra, sapendo che la somma dei quadrati costruiti su ognuna 
delle 
due 
parti è uguale a\(121\unit{{cm}^2}\)?
\end{esercizio}

\begin{esercizio}[\Ast]
 \label{ese:6.64}
In un trapezio rettangolo con area di\(81\unit{m^2}\) la somma della base 
minore 
e dell'altezza è\(12\unit{m}\) mentre la base minore è\(\dfrac 1 5\) 
della 
base 
maggiore. Trovare il perimetro del rettangolo.
\end{esercizio}

\begin{esercizio}[\Ast]
 \label{ese:6.65}
La differenza tra le diagonali di un rombo è\(8\unit{cm}\), mentre la sua 
area è 
\(24\unit{{cm}^2}~\)Determinare il lato del rombo.
\end{esercizio}

\begin{esercizio}[\Ast]
 \label{ese:6.66}
Sappiamo che in un trapezio rettangolo con area di\(40\unit{{cm}^2}\) la 
base 
minore è\(7\unit{cm}\), mentre la somma della base maggiore e 
dell'altezza 
è 
\(17\unit{cm}\) Trovare il perimetro del rettangolo.
\end{esercizio}

\begin{esercizio}[\Ast]
 \label{ese:6.67}
Un rettangolo ha l'area uguale a quella di un quadrato. L'altezza del 
rettangolo 
è\(16\unit{cm}\), mentre la sua base è di\(5\unit{cm}\) maggiore del lato 
del 
quadrato. Determinare il lato del quadrato.
\end{esercizio}

\begin{esercizio}[\Ast]
 \label{ese:6.68}
La differenza tra i cateti di un triangolo rettangolo è\(7k\), mentre la 
sua 
area 
è\(60 k^2~\)Calcola il perimetro. (\(k>0\))
\end{esercizio}

\begin{esercizio}[\Ast]
 \label{ese:6.69}
L'area di un rettangolo che ha come lati le diagonali di due quadrati 
misura\(90 
k^2\) La somma dei lati dei due quadrati misura\(14k~\)Determinare i lati 
dei 
due 
quadrati. (\( k>0\))
\end{esercizio}

\begin{esercizio}[\Ast]
 \label{ese:6.70}
Nel rettangolo ABCD la differenza tra altezza e base è\(4k\) Se 
prolunghiamo 
la 
base AB dalla parte di B di\(2k\) fissiamo il punto E e congiungiamo B 
con 
E. 
Trovare il perimetro del trapezio AECD sapendo che la sua area è\(28k^2\) 
con 
\(k>0\)
\end{esercizio}

\begin{esercizio}[\Ast]
 \label{ese:6.71}
In un triangolo isoscele la base è\( \dfrac 2 3\) dell'altezza e l'area 
è\(12k^2\) 
Trova il perimetro del triangolo.
\end{esercizio}
\end{multicols}

\paragraph{6.49.}\(\left(-\dfrac 3 4;~-\dfrac 7 2\right)\vee \left(\dfrac 
7 
2;~-\dfrac 
3 4\right)\)
\paragraph{6.50.}\((12;~5)\)

\paragraph{6.51.}\((4;~20)\vee (20,4)\)

\paragraph{6.52.}\(73\)

\paragraph{6.53.}\((30;~4)\)

\paragraph{6.54.}\((23;~23)\)

\paragraph{6.55.}\((32;~48)\)

\paragraph{6.56.}\(2p=144\unit{cm}\)

\paragraph{6.57.}\((-6;~-8)\vee (6,8)\)

\paragraph{6.58.}\((12;~18)\)

\paragraph{6.59.}\(2p=126\unit{m}\)

\paragraph{6.60.}\([15\unit{cm}\) e\(20\unit{cm}]\)

\paragraph{6.61.}\(\text{Diagonale }=6\unit{m}\)

\paragraph{6.62.}\(\Area=13,5\unit{{cm}^2}\)

\paragraph{6.63.}\(15,4\unit{cm}\)

\paragraph{6.64.}\(2p_1=42\vee 2p_2=57+3\sqrt{145}\)

\paragraph{6.65.}\(2\sqrt{10}\unit{cm}\)

\paragraph{6.66.}\(2p=24+2\sqrt{13}\)

\paragraph{6.67.}\(20\unit{cm}\)

\paragraph{6.68.}\(2p=40k\)

\paragraph{6.69.}\(l_1=5k\vee l_2=9k\)

\paragraph{6.70.}\(2p=15+k\sqrt{53}\)

\paragraph{6.71.}\(2p=4k(1+\sqrt{10})\)

\end{comment}


