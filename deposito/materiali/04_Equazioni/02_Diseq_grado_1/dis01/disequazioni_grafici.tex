% (c) 2017 Daniele Zambelli - daniele.zambelli@gmail.com
% 
% Tutti i grafici per il capitolo relativo alle disequazioni di primo grado
%

\newcommand{\disrettaa}{% Retta con alcuni punti evidenziati
  \disegno[5]{
    \rcom{-3}{+3}{-5}{+5}{gray!50, very thin, step=1}
    \tkzInit[xmin=-3.3,xmax=+3.3,ymin=-5.3,ymax=0]
    \tkzFct[domain=-3.3:+3.3, ultra thick, color=red!50!black]{-4*x+4}
    \tkzInit[xmin=-3.3,xmax=+3.3,ymin=0,ymax=+5.3]
    \tkzFct[domain=-3.3:+3.3, ultra thick, color=blue!50!black]{-4*x+4}
    \node at (-1, 4) {\(f(x)\)};
  }
}

\newcommand{\disrettab}{% Retta con alcuni punti evidenziati
  \disegno[5]{
    \rcom{-4}{+2}{-5}{+5}{gray!50, very thin, step=1}
    \tkzInit[xmin=-4.3,xmax=+2.3,ymin=-5.3,ymax=0]
    \tkzFct[domain=-4.3:+2.3, ultra thick, color=red!50!black]{3./2*x+3}
    \node at (-3, -3) {\(g(x)\)};
    \tkzInit[xmin=-4.3,xmax=+2.3,ymin=0,ymax=+5.3]
    \tkzFct[domain=-4.3:+2.3, ultra thick, color=blue!50!black]{3./2*x+3}
  }
}

\newcommand{\rettdec}[2][x]{% Retta decrescente
  \def \val{#2}
  \asse{-3}{+3}{0}{\(#1\)}
  \draw [-] [ultra thick,blue!50!black] (-3, 1.2) -- (0, 0);
  \draw [-] [ultra thick,red!50!black] (0, 0) -- (+3, -1.2);
  \draw [black] (0, -.3) -- (0, +.8) node [above] {\(\val\)};
}

\newcommand{\rettcre}[2][x]{% Retta crescente
  \def \val{#2}
  \asse{-3}{+3}{0}{\(#1\)}
  \draw [-] [ultra thick, red!50!black] (-3, -1.2) -- (0, 0);
  \draw [-] [ultra thick, blue!50!black] (0, 0) -- (+3, +1.2);
  \draw [black] (0, -.3) -- (0, +.8) node [above] {\(\val\)};
}

\newcommand{\rettpos}[1][x]{% Retta positiva
  \asse{-3}{+3}{0}{\(#1\)}
  \draw [-] [ultra thick,blue!50!black] (-3.3, +1) -- (+3.3, +1);
}

\newcommand{\rettneg}[1][x]{% Retta negativa
  \asse{-3}{+3}{0}{\(#1\)}
  \draw [-] [ultra thick, red!50!black] (-3.3, -1) -- (+3.3, -1);
}

\newcommand{\rettadec}[2][x]{% Retta decrescente
  \disegno[5]{\rettdec[#1]{#2}}
}

\newcommand{\rettacre}[2][x]{% Retta decrescente
  \disegno[5]{\rettcre[#1]{#2}}
}

\newcommand{\segni}[2]{% Retta crescente
  \node [xshift=-30, yshift=-2, above] at (0, 0) {\(#1\)};
  \draw (0, .4) circle (3pt);
  \node [xshift=30, yshift=-2, above] at (0, 0) {\(#2\)};% xshift=+30: ERR!
}

\newcommand{\segnidec}[2][x]{% Retta decrescente
  \disegno[5]{
    \rettdec[#1]{#2}
    \segni{+}{-}
  }
}

\newcommand{\segnicre}[2][x]{% Retta crescente
  \disegno[5]{
    \rettcre[#1]{#2}
    \segni{-}{+}
  }
}

\newcommand{\segnipos}[1][x]{% segni retta positiva
  \disegno[5]{
    \rettpos[#1]
  \node [xshift=-30, yshift=-2, above] at (0, 0) {\(+\)};
  \node [xshift=30, yshift=-2, above] at (0, 0) {\(+\)};% xshift=+30: ERR!
  }
}

\newcommand{\segnineg}[1][x]{% segni retta negativa
  \disegno[5]{
    \rettneg[#1]
  \node [xshift=-30, yshift=-2, above] at (0, 0) {\(-\)};
  \node [xshift=30, yshift=-2, above] at (0, 0) {\(-\)};% xshift=+30: ERR!
  }
}

\makeatletter
\newcommand{\lengtha}[1]{%
    \@tempcnta\z@
    \@for\@tempa:=#1\do{\advance\@tempcnta\@ne}
    \the\@tempcnta
}
\makeatother

\newcommand{\trespolosegnidis}[6]{% Grafo calcolo del segno con tre assi
  % Esempio di chiamata:
  %% \trespolosegni{5}{1.2}{3}{f(x), g(x), f(x)\cdot g(x)}{2}{2, 5}
  \def \dimx{#1}
  \def \dimy{#2}
  \def \numassi{#3}
  \def \labassi{#4}
  \def \numpunti{#5}
  \def \labpunti{#6}
  \foreach \li [count = \i] in \labassi{
    \assex{-\dimx}{\dimx}{-\i*\dimy}
    \node at (-\dimx, -\i*\dimy) (posy) [above left=-.2] {\(\li\)};
  }
  \pgfmathparse{\dimx*2 / (\numpunti+1)} \let\largx\pgfmathresult
  \foreach \li [count = \i] in \labpunti{
    \draw (\i*\largx-\dimx, -\numassi * \dimy) -- (\i*\largx-\dimx, 0)
          node [above] {\(\li\)};
  }
}

\newcommand{\segniprodottoa}{% Segno del prodotto di due funzioni
  \foreach \x/\y in {+1.67/-.6, -1.67/-1.7, -1.67/-2.8, +1.67/-2.8}
    \cerchietto{\x}{\y}
  \foreach \x/\s in {-3.5/-, 0/-, 3.5/+}
    \node at (\x, -1.1) [above] {\(\s\)};
  \foreach \x/\s in {-3.5/-, 0/+, 3.5/+}
    \node at (\x, -2.2) [above] {\(\s\)};
  \foreach \x/\s in {-3.5/+, 0/-, 3.5/+}
    \node at (\x, -3.3) [above] {\(\s\)};
}

\newcommand{\segnoprodottoa}{% Segno del prodotto di due funzioni
  \disegno{
    \trespolosegnidis{5}{1.1}
                     {3}{x-4, x+2, \tonda{x-4}\tonda{x+2}}
                     {2}{-2, +4}
    \segniprodottoa
  }
}

\newcommand{\segnoprodottoas}{
% Segno del prodotto di due funzioni indicate in modo simbolico
  \disegno{
    \trespolosegnidis{5}{1.1}
                     {3}{f_1(x), f_2(x), f(x) = f_1(x) \cdot f_2(x)}
                     {2}{-2, +4}
    \segniprodottoa
  }
}

\newcommand{\segniquozientea}{% Segni del quoziente di due funzioni
  \foreach \x/\y in {+1.67/-.6, +1.67/-2.8}
    \cerchietto{\x}{\y}
  \foreach \x/\y in {-1.67/-1.7, -1.67/-2.8}
    \crocetta{\x}{\y}
  \foreach \x/\s in {-3.5/+, 0/+, 3.5/-}
    \node at (\x, -1.1) [above] {\(\s\)};
  \foreach \x/\s in {-3.5/-, 0/+, 3.5/+}
    \node at (\x, -2.2) [above] {\(\s\)};
  \foreach \x/\s in {-3.5/-, 0/+, 3.5/-}
    \node at (\x, -3.3) [above] {\(\s\)};
}

\newcommand{\segnoquozientea}{% Segno del quoziente di due funzioni
  \disegno{
    \trespolosegnidis{5}{1.1}
                     {3}{-3x-4, 2x+7, \tonda{-3x-4} / \tonda{2x+7}}
                     {2}{-\frac{7}{2}, -\frac{4}{3}}
    \segniquozientea
  }
}

\newcommand{\segnoquozienteas}{
% Segno del quoziente di due funzioni indicate in modo simbolico
  \disegno{
    \trespolosegnidis{5}{1.1}
                     {3}{num(x), den(x), f(x) = num(x) / den(x)}
                     {2}{-\frac{7}{2}, -\frac{4}{3}}
    \segniquozientea
  }
}

\newcommand{\segnimista}{% Segni del prodotto e quoziente di più funzioni
  \foreach \x/\y in {+0/-0.6, 2.67/-1.7, -2.67/-2.8,
                     +0/-6.1, 2.67/-6.1, -2.67/-6.1}
    \cerchietto{\x}{\y};
  \foreach \x/\y in {-5.33/-5.0, 5.33/-3.9,
                     -5.33/-6.1, 5.33/-6.1}
    \crocetta{\x}{\y};
  \foreach \x/\s in {-6.67/-, -4/-, -1.33/-, +1.33/+, +4/+, +6.67/+}
    \node at (\x, -1.1) [above] {\(\s\)};
  \foreach \x/\s in {-6.67/+, -4/+, -1.33/+, +1.33/+, +4/-, +6.67/-}
    \node at (\x, -2.2) [above] {\(\s\)};
  \foreach \x/\s in {-6.67/-, -4/-, -1.33/+, +1.33/+, +4/+, +6.67/+}
    \node at (\x, -3.3) [above] {\(\s\)};
  \foreach \x/\s in {-6.67/-, -4/-, -1.33/+, +1.33/+, +4/+, +6.67/+}
    \node at (\x, -3.3) [above] {\(\s\)};
  \foreach \x/\s in {-6.67/-, -4/-, -1.33/-, +1.33/-, +4/-, +6.67/+}
    \node at (\x, -4.4) [above] {\(\s\)};
  \foreach \x/\s in {-6.67/-, -4/+, -1.33/+, +1.33/+, +4/+, +6.67/+}
    \node at (\x, -5.5) [above] {\(\s\)};
  \foreach \x/\s in {-6.67/+, -4/-, -1.33/+, +1.33/-, +4/+, +6.67/-}
    \node at (\x, -6.6) [above] {\(\s\)};
}

\newcommand{\segnomista}{% Segno del prodotto e quoziente di più funzioni
  \disegno{
    \trespolosegnidis{8}{1.1}
                     {6}{x, -2x+1, 2x+1, x-2, x+5, f(x)}
                     {5}{-5, -\frac{1}{2}, 0, +\frac{1}{2}, 2}
    \segnimista
  }
}

\newcommand{\segnimistadisa}{% Segni del prodotto e quoziente di più funzioni
  \foreach \x/\y in {+3/-0.6, 0/-1.7,
                     +3/-3.9, 0/-3.9}
    \cerchietto{\x}{\y};
  \foreach \x/\y in {-3/-2.8, -3/-3.9}
    \crocetta{\x}{\y};
  \foreach \x/\s in {-4.5/-, -1.5/-, +1.5/-, +4.5/+}
    \node at (\x, -1.1) [above] {\(\s\)};
  \foreach \x/\s in {-4.5/+, -1.5/+, +1.5/-, +4.5/-}
    \node at (\x, -2.2) [above] {\(\s\)};
  \foreach \x/\s in {-4.5/-, -1.5/+, +1.5/+, +4.5/+}
    \node at (\x, -3.3) [above] {\(\s\)};
  \foreach \x/\s in {-4.5/+, -1.5/-, +1.5/+, +4.5/-}
    \node at (\x, -4.4) [above] {\(\s\)};
}

\newcommand{\segnomistadisa}{% Segno del prodotto e quoziente di più funzioni
  \disegno{
    \trespolosegnidis{6}{1.1}
                     {4}{x-2, -2x+1, x+3, f(x)}
                     {3}{-3, +\frac{1}{2}, +2}
    \segnimistadisa
  }
}

\newcommand{\segnifratta}{% Segni del quoziente di due funzioni
  \foreach \x/\y in {+1.67/-.6, +1.67/-2.8}
    \cerchietto{\x}{\y}
  \foreach \x/\y in {-1.67/-1.7, -1.67/-2.8}
    \crocetta{\x}{\y}
  \foreach \x/\s in {-3.5/+, 0/+, 3.5/-}
    \node at (\x, -1.1) [above] {\(\s\)};
  \foreach \x/\s in {-3.5/-, 0/+, 3.5/+}
    \node at (\x, -2.2) [above] {\(\s\)};
  \foreach \x/\s in {-3.5/-, 0/+, 3.5/-}
    \node at (\x, -3.3) [above] {\(\s\)};
}

\newcommand{\segnofratta}{% Segno del quoziente di due funzioni
  \disegno{
    \trespolosegnidis{5}{1.1}
                     {3}{-3x+4, x+2, f(x)}
                     {2}{-2, +3}
    \segnifratta
  }
}

\newcommand{\soluzionefratta}{% Trespolo con soluzione di una dis. fratta
  \disegno{
    \trespolosegnidis{5}{1.1}
                     {3}{-3x+4, x+2, f(x)}
                     {2}{-2, +3}
    \segnifratta
    \rayl{-3.3}{5}{-1.67}{}{white}
    \rayr{-3.3}{5}{+1.67}{}{blue}
  }
}

\newcommand{\estremo}[1]{% estremo di un intervallo
  \def \intp{#1}
  \draw[fill=\intp] (p) circle (3pt);
}

\newcommand{\sisdisa}{
  \disegno{
    \rayrconasse{0}{6}{2}{0}{white}{x}
    \draw [opacity=0] (0, 0) -- (0, .6);
  }
}

\newcommand{\sisdisb}{
  \disegno{
    \raylconasse{0}{5}{2.5}{+2}{blue}{x}
    \draw [opacity=0] (0, 0) -- (0, .6);
  }
}

\newcommand{\trespolointervalli}[4]{% Grafo intervalli con enne assi
  % Esempio di chiamata:
% \disegno{
%   \trespolointervalli{9}{-6}
%                      {-1/\(A\), -2/\(B\), -3/\(C\), 
%                       -4/\(A \cup B\), -5/\(B \cap C\), -6/\(A - B\)}
%                      {1.5/-5, 3/-2, 4.5/3, 6/6, 7.5/9}
% }
  \def \dimx{#1}
  \def \dimy{#2}
  \def \eassi{#3}
  \def \epunti{#4}
   \foreach \posy/\ea in {#3}{
     \node at (0, \posy) [left] {\ea};
%      \assex{0}{\dimx}{\posy}          % Perché incricca il compilatore???
     \assev{0}{\posy}{\dimx}{\posy}
%      \asse{0}{\dimx}{\posy}{\(x\)}    % Perché incricca il compilatore???
   }
   \foreach \i/\e in {#4}
     \draw (\i, \dimy) -- (\i, -0.5) node [above] {\e};
}

\newcommand {\sistemaa}{% Grafo per operazioni tra intervalli
  \disegno{
    \trespolointervalli{6}{-3}
                       {-1/\(IS_1\), -2/\(IS_2\), 
                        -3/\(IS = IS_1 \cap IS_2\)}
                       {2/\(0\), 4/\(+2\)}
    \rayr{-1}{6}{2}{}{white}
    \rayl{-2}{0}{4}{}{blue}
    \inti{-3}{2}{4}{}{}{white}{blue}
  \def \taba{10}
  \def \tabb{15}
  \node at (\taba, -3) {\(0 < x \le +2\)}; 
  \node at (\tabb, -3) {\(\intervac{0}{+2}\)};
  }
}

\newcommand{\segnioss}{% Segni del quoziente di due funzioni
  \foreach \x/\y in {+1./-.6, +1./-2.8}
    \cerchietto{\x}{\y}
  \foreach \x/\y in {-1./-1.7, -1./-2.8}
    \crocetta{\x}{\y}
  \foreach \x/\s in {-2/+, 0/+, 2/-}
    \node at (\x, -1.1) [above] {\(\s\)};
  \foreach \x/\s in {-2/-, 0/+, 2/+}
    \node at (\x, -2.2) [above] {\(\s\)};
  \foreach \x/\s in {-2/-, 0/+, 2/-}
    \node at (\x, -3.3) [above] {\(\s\)};
}

\newcommand{\ossegni}{% Segno del quoziente di due funzioni
  \disegno{
    \trespolosegnidis{3}{1.1}
                     {3}{f(x), g(x), f(x)/g(x)}
                     {2}{a, b}
    \segnioss
    \draw [opacity=0] (0, 0) -- (5, 0);
  }
}

\newcommand {\ossist}{% Grafo per operazioni tra intervalli
  \disegno{
    \trespolointervalli{6}{-3}
                       {-1/\(IS_1\), -2/\(IS_2\), 
                        -3/\(IS = IS_1 \cap IS_2\)}
                       {2/\(a\), 4/\(b\)}
    \rayr{-1}{6}{2}{}{white}
    \rayl{-2}{0}{4}{}{blue}
    \inti{-3}{2}{4}{}{}{white}{blue}
  \def \taba{10}
  \def \tabb{15}
  }
}

\newcommand {\sistemab}{% Grafo per operazioni tra intervalli
  \disegno{
    \trespolointervalli{6}{-3}
                       {-1/\(IS_1\), -2/\(IS_2\), 
                        -3/\(IS = IS_1 \cap IS_2\)}
                       {2/\(-\frac{13}{2}\), 4/\(+\frac{2}{9}\)}
    \rayr{-1}{6}{2}{}{blue}
    \rayl{-2}{0}{4}{}{white}
    \inti{-3}{2}{4}{}{}{blue}{white}
  \def \taba{10}
  \def \tabb{17}
  \node at (\taba, -3) {\(-\dfrac{13}{2} \le x < +\dfrac{2}{9}\)}; 
  \node at (\tabb, -3) {\(\intervca{-\dfrac{13}{2}}{+\dfrac{2}{9}}\)};
  }
}
