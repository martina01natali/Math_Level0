% (c) 2012 -2014 Dimitrios Vrettos - d.vrettos@gmail.com
% (c) 2014 Daniele Zambelli - daniele.zambelli@gmail.com

\section{Esercizi}

\subsection{Esercizi dei singoli paragrafi}

%\subsubsection*{21.1 - Intervalli sulla retta reale}
\subsubsection*{\numnameref{sec:dis_intervalli}}

\begin{esercizio}
 \label{ese:dis_1}
 Rappresenta i seguenti intervalli usando i predicati e la notazione con le 
 parentesi.
 \begin{enumeratea}
\item .

\vspace{-18pt}
\input{\folder lbr/fig108_eser.pgf}
\item .

\vspace{-18pt}
\input{\folder lbr/fig109_eser.pgf}
\item .

\vspace{-18pt}
\input{\folder lbr/fig110_eser.pgf}
\item .

\vspace{-18pt}
\input{\folder lbr/fig111_eser.pgf}
\item .

\vspace{-18pt}
\input{\folder lbr/fig112_eser.pgf}
\item .

\vspace{-18pt}
\input{\folder lbr/fig106_ret.pgf}
\item .

\vspace{-18pt}
\input{\folder lbr/fig113_eser.pgf}
\end{enumeratea}
\end{esercizio}

\begin{esercizio}
 \label{ese:dis_2}
 Rappresenta graficamente i seguenti intervalli.
\begin{multicols}{4}
 \begin{enumeratea}
\item $x > 4$
\item $x \le 0$
\item $x < 6$
\item $x \ge 8$
\item $-5 < x \le 0$
\item $2 \le x < 0$
\item $x \ge -3$
\item $-1 \le x \le 1$
\item $3 < x \le 9$
\item $x > -3$
\item $x < 68$
\item $-0,3 \le x < 0,2$
\item $x \ge 457$
\item $2 \le x < -2$
\item $x \ge -631$
\item $-1,2 < x < 1,5$
 \end{enumeratea}
\end{multicols}
\end{esercizio}

\subsubsection*{\numnameref{sec:dis_binomio}}
\begin{esercizio}\label{ese:dis_3}
 Studia il segno dei seguenti binomi.
\begin{multicols}{4}
 \begin{enumeratea}
  \item  $-11 x -5$
  \item  $-2 x +4$
  \item  $-7 x +10$
  \item  $- x -7$
  \item  $6 x -6$
  \item  $-8 x $
  \item  $7 x +1$
  \item  $9 x -12$
  \item  $-3 x +4$
  \item  $-6 x +11$
  \item  $-2 x -10$
  \item  $6 x -8$
 \end{enumeratea}
\end{multicols}
\begin{flushright}
\vspace*{-8pt}

\begin{tabular}{llll}
\framebox(10,10){}\quad\(++\left(0\right)--\) \quad & 
  \framebox(10,10){}\quad\(++\left(2\right)--\) \quad &
  \framebox(10,10){}\quad\(++\left(\dfrac{10}{7}\right)--\) \quad & 
  \framebox(10,10){}\quad\(--\left(-\dfrac{1}{7}\right)++\) \\
\framebox(10,10){}\quad\(++\left(-5\right)--\) \quad & 
  \framebox(10,10){}\quad\(--\left(\dfrac{4}{3}\right)++\) \quad &
  \framebox(10,10){}\quad\(--\left(1\right)++\) \quad & 
  \framebox(10,10){}\quad\(++\left(-\dfrac{5}{11}\right)--\) \\
\framebox(10,10){}\quad\(++\left(\dfrac{11}{6}\right)--\) \quad & 
  \framebox(10,10){}\quad\(++\left(\dfrac{4}{3}\right)--\) \quad &
  \framebox(10,10){}\quad\(--\left(\dfrac{4}{3}\right)++\) \quad & 
  \framebox(10,10){}\quad\(++\left(-7\right)--\) 
\end{tabular}
\end{flushright}
\end{esercizio}

\begin{esercizio}\label{ese:dis_4}
 Studia il segno dei seguenti binomi.
\begin{multicols}{4}
 \begin{enumeratea}
  \item  $5 x +8$
  \item  $-12 x -10$
  \item  $11 x$
  \item  $-7$
  \item  $8 x +1$
  \item  $9 x +11$
  \item  $-8 x +8$
  \item  $-4 x$
  \item  $10 x -8$
  \item  $-2 x $
  \item  $8$
  \item  $-9 x -1$
 \end{enumeratea}
\end{multicols}
\begin{flushright}
\vspace*{-8pt}
\begin{tabular}{llll}
\framebox(10,10){}\quad\(++\left(-\dfrac{5}{6}\right)--\) \quad & 
  \framebox(10,10){}\quad\(----\) \quad &
  \framebox(10,10){}\quad\(++\left(1\right)--\) \quad & 
  \framebox(10,10){}\quad\(++\left(-\dfrac{1}{9}\right)--\) \\
\framebox(10,10){}\quad\(--\left(-\dfrac{1}{8}\right)++\) \quad & 
  \framebox(10,10){}\quad\(++++\) \quad &
  \framebox(10,10){}\quad\(--\left(0\right)++\) \quad & 
  \framebox(10,10){}\quad\(--\left(\dfrac{4}{5}\right)++\) \\
\framebox(10,10){}\quad\(++\left(0\right)--\) \quad & 
  \framebox(10,10){}\quad\(++\left(0\right)--\) \quad &
  \framebox(10,10){}\quad\(--\left(-\dfrac{8}{5}\right)++\) \quad & 
  \framebox(10,10){}\quad\(--\left(-\dfrac{11}{9}\right)++\)
\end{tabular}
\end{flushright}
\end{esercizio}

\subsubsection*{\numnameref{sec:dis_prodotto}}

\begin{esercizio}\label{ese:dis_5}
 Studia il segno dei seguenti prodotti.
 \begin{enumeratea}
  \item  $\left(-8 x +5\right)\left(10 x +11\right)$ \hfill 
  [$--\left [-\dfrac{11}{10} \right ]++\left [\dfrac{5}{8} \right ]--$]
  \item  $\left(-2 x -11\right)\left(- x +8\right)$ \hfill 
  [$++\left [-\dfrac{11}{2} \right ]--\left [8 \right ]++$]
  \item  $\left(x +7\right)\left(5 x -3\right)$ \hfill 
  [$++\left [-7 \right ]--\left [\dfrac{3}{5} \right ]++$]
  \item  $\left(7 x +9\right)\left(7 x +6\right)$ \hfill 
  [$++\left [-\dfrac{9}{7} \right ]--\left [-\dfrac{6}{7} \right ]++$]
  \item  $\left(-3 x +1\right)\left(10 x -5\right)$ \hfill 
  [$--\left [\dfrac{1}{3} \right ]++\left [\dfrac{1}{2} \right ]--$]
  \item  $\left(x +6\right)\left(x \right)$ \hfill 
  [$++\left [-6 \right ]--\left [0 \right ]++$]
  \item  $\left(6 x +1\right)\left(4 x -3\right)$ \hfill 
  [$++\left [-\dfrac{1}{6} \right ]--\left [\dfrac{3}{4} \right ]++$]
  \item  $\left(11 x +1\right)\left(-6 x -4\right)$ \hfill 
  [$--\left [-\dfrac{2}{3} \right ]++\left [-\dfrac{1}{11} \right ]--$]
%   \item  $\left(-2 x +9\right)\left(6 x -12\right)$ \hfill 
%   [$--\left [2 \right ]++\left [\dfrac{9}{2} \right ]--$]
%   \item  $\left(-2 x -10\right)\left(-11 x +8\right)$ \hfill 
%   [$++\left [-5 \right ]--\left [\dfrac{8}{11} \right ]++$]
%   \item  $\left(-12 x -11\right)\left(x +10\right)$ \hfill 
%   [$--\left [-10 \right ]++\left [-\dfrac{11}{12} \right ]--$]
%   \item  $\left(5 x +5\right)\left(2 x -11\right)$ \hfill 
%   [$++\left [-1 \right ]--\left [\dfrac{11}{2} \right ]++$]
 \end{enumeratea}
\end{esercizio}

\begin{esercizio}\label{ese:dis_6}
 Studia il segno dei seguenti prodotti.
 \begin{enumeratea}
  \item  $\left(5 x +2\right)\left(11 x +4\right)$ \hfill 
  [$++\left [-\dfrac{2}{5} \right ]--\left [-\dfrac{4}{11} \right ]++$]
  \item  $\left(-4 x -3\right)\left(10 x -4\right)$ \hfill 
  [$--\left [-\dfrac{3}{4} \right ]++\left [\dfrac{2}{5} \right ]--$]
  \item  $\left(-11 x -6\right)\left(-9 x +2\right)$ \hfill 
  [$++\left [-\dfrac{6}{11} \right ]--\left [\dfrac{2}{9} \right ]++$]
  \item  $\left(-3 x +5\right)\left(-4 x +10\right)$ \hfill 
  [$++\left [\dfrac{5}{3} \right ]--\left [\dfrac{5}{2} \right ]++$]
  \item  $\left(- x -11\right)\left(- x -12\right)$ \hfill 
  [$++\left [-12 \right ]--\left [-11 \right ]++$]
  \item  $\left(-12 x -12\right)\left(- x -1\right)$ \hfill 
  [$++\left [-1 \right ]--\left [-1 \right ]++$]
  \item  $\left(-6 x -3\right)\left(-3 x -6\right)$ \hfill 
  [$++\left [-2 \right ]--\left [-\dfrac{1}{2} \right ]++$]
%   \item  $\left(-3 x +9\right)\left(4 x -5\right)$ \hfill 
%   [$--\left [\dfrac{5}{4} \right ]++\left [3 \right ]--$]
%   \item  $\left(-10 x -8\right)\left(7 x -8\right)$ \hfill 
%   [$--\left [-\dfrac{4}{5} \right ]++\left [\dfrac{8}{7} \right ]--$]
%   \item  $\left(-11 x +4\right)\left(-3 x +5\right)$ \hfill 
%   [$++\left [\dfrac{4}{11} \right ]--\left [\dfrac{5}{3} \right ]++$]
%   \item  $\left(-10 x -2\right)\left(5 x -4\right)$ \hfill 
%   [$--\left [-\dfrac{1}{5} \right ]++\left [\dfrac{4}{5} \right ]--$]
%   \item  $\left(11 x +11\right)\left(-12 x +8\right)$ \hfill 
%   [$--\left [-1 \right ]++\left [\dfrac{2}{3} \right ]--$]
 \end{enumeratea}
\end{esercizio}

% \newpage

\subsubsection*{\numnameref{sec:dis_quoziente}}

\begin{esercizio}\label{ese:dis_7}
 Studia il segno delle seguenti frazioni.
\begin{multicols}{2}
 \begin{enumeratea}
  \item  $\dfrac{3 x +4}{-11 x +3}$ \hfill 
  [$--\left [-\dfrac{4}{3} \right ]++\left ]\dfrac{3}{11} \right [--$]
  \item  $\dfrac{- x -10}{x +4}$ \hfill 
  [$--\left [-10 \right ]++\left ]-4 \right [--$]
  \item  $\dfrac{-8 x -9}{4 x -5}$ \hfill 
  [$--\left [-\dfrac{9}{8} \right ]++\left ]\dfrac{5}{4} \right [--$]
  \item  $\dfrac{7 x -4}{5 x +11}$ \hfill 
  [$++\left ]-\dfrac{11}{5} \right [--\left [\dfrac{4}{7} \right ]++$]
  \item  $\dfrac{11 x -10}{-11 x -5}$ \hfill 
  [$--\left ]-\dfrac{5}{11} \right [++\left [\dfrac{10}{11} \right ]--$]
  \item  $\dfrac{-2 x +1}{-3 x }$ \hfill 
  [$++\left ]0 \right [--\left [\dfrac{1}{2} \right ]++$]
  \item  $\dfrac{8 x -4}{11 x +9}$ \hfill 
  [$++\left ]-\dfrac{9}{11} \right [--\left [\dfrac{1}{2} \right ]++$]
  \item  $\dfrac{-9 x -6}{-5 x +4}$ \hfill 
  [$++\left [-\dfrac{2}{3} \right ]--\left ]\dfrac{4}{5} \right [++$]
  \item  $\dfrac{7 x -12}{-5 x +8}$ \hfill 
  [$--\left ]\dfrac{8}{5} \right [++\left [\dfrac{12}{7} \right ]--$]
  \item  $\dfrac{-7 x +8}{7 x +8}$ \hfill 
  [$--\left ]-\dfrac{8}{7} \right [++\left [\dfrac{8}{7} \right ]--$]
%   \item  $\dfrac{7 x +9}{-11 x -12}$ \hfill 
%   [$--\left [-\dfrac{9}{7} \right ]++\left ]-\dfrac{12}{11} \right [--$]
%   \item  $\dfrac{-7 x }{- x +6}$ \hfill 
%   [$++\left [0 \right ]--\left ]6 \right [++$]
 \end{enumeratea}
\end{multicols}
\end{esercizio}

\begin{esercizio}\label{ese:dis_8}
 Studia il segno delle seguenti frazioni.
\begin{multicols}{2}
 \begin{enumeratea}
  \item  $\dfrac{3 x +5}{-9 x -12}$ \hfill 
  [$--\left [-\dfrac{5}{3} \right ]++\left ]-\dfrac{4}{3} \right [--$]
  \item  $\dfrac{-5 x +3}{-5 x -11}$ \hfill 
  [$++\left ]-\dfrac{11}{5} \right [--\left [\dfrac{3}{5} \right ]++$]
  \item  $\dfrac{-7 x +11}{x +4}$ \hfill 
  [$--\left ]-4 \right [++\left [\dfrac{11}{7} \right ]--$]
  \item  $\dfrac{9 x -2}{-7 x -10}$ \hfill 
  [$--\left ]-\dfrac{10}{7} \right [++\left [\dfrac{2}{9} \right ]--$]
  \item  $\dfrac{11 x +9}{6 x +5}$ \hfill 
  [$++\left ]-\dfrac{5}{6} \right [--\left [-\dfrac{9}{11} \right ]++$]
  \item  $\dfrac{9 x -5}{-3 x +10}$ \hfill 
  [$--\left [\dfrac{5}{9} \right ]++\left ]\dfrac{10}{3} \right [--$]
  \item  $\dfrac{-3 x +10}{-7 x +9}$ \hfill 
  [$++\left ]\dfrac{9}{7} \right [--\left [\dfrac{10}{3} \right ]++$]
  \item  $\dfrac{4 x +10}{-11 x +6}$ \hfill 
  [$--\left [-\dfrac{5}{2} \right ]++\left ]\dfrac{6}{11} \right [--$]
  \item  $\dfrac{5 x -10}{x -1}$ \hfill 
  [$++\left ]1 \right [--\left [2 \right ]++$]
  \item  $\dfrac{5 x +1}{-2 x -9}$ \hfill 
  [$--\left ]-\dfrac{9}{2} \right [++\left [-\dfrac{1}{5} \right ]--$]
%   \item  $\dfrac{-12 x +1}{8 x -4}$ \hfill 
%   [$--\left [\dfrac{1}{12} \right ]++\left ]\dfrac{1}{2} \right [--$]
%   \item  $\dfrac{-7 x +1}{5 x -3}$ \hfill 
%   [$--\left [\dfrac{1}{7} \right ]++\left ]\dfrac{3}{5} \right [--$]
 \end{enumeratea}
\end{multicols}
\end{esercizio}

%\subsubsection*{21.2 - Disequazioni numeriche}
\subsubsection*{\numnameref{sec:dis_numeriche}}

\begin{esercizio}
 \label{ese:dis_9}
Completa la seguente tabella indicando con una crocetta il tipo di
disuguaglianza o disequazione:

 \begin{tabularx}{.9\textwidth}{X|c|c|c|}
 \toprule
 Proposizione&\multicolumn{2}{c}{Disuguaglianza}& Disequazione\\
  & Vera & Falsa & \\
 \midrule
 Il doppio di un numero reale è minore del suo triplo aumentato di~1: & & & \\
 \midrule
 La somma del quadrato di~4 con~3 è maggiore della somma del quadrato di~3 
con~4: & & &\\
 \midrule
 Il quadrato della somma di~4 con~3 è minore o uguale a~49: & & & \\
 \midrule
 In~$\insZ:(5+8)-(2)^{4}>0$: & & & \\
 \midrule
 $-x^{2}>0$: & & & \\
 \midrule
 $(x+6)^{2}\cdot (1-9)\cdot (x+3-9)<0$: & & & \\
 \bottomrule
 \end{tabularx}
\end{esercizio}

\begin{esercizio}
 \label{ese:dis_10}
 Senza fare calcoli trova le soluzioni delle seguenti disequazioni.
\begin{multicols}{3}
 \begin{enumeratea}
\item $2x>3x$
\item $3x\ge2x$
\item $5>0$
\item $-1\le x$
\item $3>x$
\item $-4\le~x$
\item $x^2\ge~0$
\item $x^2+5\ge~0$
\item $-3x^2\le~0$
\item $x^{4}+x^{2}+10>0$
\item $x^{4}+x^{2}+100<0$
\item $x^3>x^2$
 \end{enumeratea}
\end{multicols}
\end{esercizio}

\newpage

\begin{esercizio}
 \label{ese:dis_11}
Trova l'insieme soluzione delle seguenti disequazioni.
 \begin{multicols}{3}
 \begin{enumeratea}
\item $x-2>0$
\item $-x+5\le~0$
\item $5x+15>0$
\item $-3x-5\ge~0$
\item $6x+3\le~0$
\item $x\ge~0$
\item $4x-1\le~0$
\item $3-x>0$
\item $-2x-3\ge~0$
\item $-3x-4\le~0$
\item $-x+3>0$
\item $-x-3\le~0$
 \end{enumeratea}
\end{multicols}
\end{esercizio}

% \begin{esercizio}[]
%  \label{ese:dis_12}
% Trova l'insieme soluzione delle seguenti disequazioni.
%  \begin{multicols}{3}
%  \begin{enumeratea}
%  \item $3+4x\ge~5x+2$
% \item $5x-4\le~8x-4$
% \item $-3x+2\ge~-x-8$
% \item $4x+7>~7x+8$
% \item $-5x+7\ge~7x+8$
% \item $4x+7<~7x+8$
% \item $8x+7\ge~10x+3$
% \item $4x+7\le~-6x+1$
% \item $-3x+7<~-2x+4$
% \item $x+7\ge~5x+3$
% \item $-2x+7\le~7x-9$
% \item $5x+7>~7x+4$
% \end{enumeratea}
% \end{multicols}
% \end{esercizio}


\begin{esercizio}\label{ese:dis_12}
 Trova l'insieme soluzione delle seguenti disequazioni.
\begin{multicols}{2}
 \begin{enumeratea}
  \item  $9 x  > -8 x +11$ \hfill [$x > \dfrac{11}{17}$]
  \item  $11 x +8 \ge -8 x -9$ \hfill [$x \ge -\dfrac{17}{19}$]
  \item  $-3 x +10 \le -8 x $ \hfill [$x \le -2$]
  \item  $7 x -7 \ge -3 x -3$ \hfill [$x \ge \dfrac{2}{5}$]
  \item  $9 x +8 < 4 x -8$ \hfill [$x < -\dfrac{16}{5}$]
  \item  $-6 x -3 \ge -5 x -11$ \hfill [$x \le 8$]
  \item  $6 x +7 > 6 x -1$ \hfill [$x {0} {1}$]
  \item  $-12 x -2 < 8 x -1$ \hfill [$x > -\dfrac{1}{20}$]
  \item  $x +10 \le 4 x +9$ \hfill [$x \ge \dfrac{1}{3}$]
  \item  $-9 x -4 < 7 x -12$ \hfill [$x > \dfrac{1}{2}$]
  \item  $-8 x -5 \le 5 x -3$ \hfill [$x \ge -\dfrac{2}{13}$]
  \item  $-5 x +5 < -10 x -10$ \hfill [$x < -3$]
 \end{enumeratea}
\end{multicols}
\end{esercizio}

\begin{esercizio}\label{ese:dis_13}
 Trova l'insieme soluzione delle seguenti disequazioni.
\begin{multicols}{2}
 \begin{enumeratea}
  \item  $-6 x +1 > -\dfrac{8}{3} x -11$ \hfill 
  [$x < \dfrac{18}{5}$]
  \item  $6 x +\dfrac{11}{3} \ge 2 x +1$ \hfill 
  [$x \ge -\dfrac{2}{3}$]
  \item  $3 x -11 < \dfrac{11}{2} x +\dfrac{11}{2}$ \hfill 
  [$x > -\dfrac{33}{5}$]
  \item  $\dfrac{10}{3} x -6 > -\dfrac{1}{2} x +\dfrac{9}{2}$ \hfill 
  [$x > \dfrac{63}{23}$]
  \item  $-\dfrac{8}{3} x +1 \ge -\dfrac{11}{2} x +\dfrac{7}{3}$ \hfill 
  [$x \ge \dfrac{8}{17}$]
  \item  $\dfrac{3}{2} x  < -\dfrac{11}{3} x +11$ \hfill 
  [$x < \dfrac{66}{31}$]
  \item  $-5 x +2 > \dfrac{9}{4} x -5$ \hfill 
  [$x < \dfrac{28}{29}$]
  \item  $\dfrac{9}{4} x +2 \le -6 x +\dfrac{3}{4}$ \hfill 
  [$x \le -\dfrac{5}{33}$]
  \item  $x +\dfrac{11}{3} \le -\dfrac{5}{4} x +1$ \hfill 
  [$x \le -\dfrac{32}{27}$]
  \item  $4 x +\dfrac{3}{2} > -\dfrac{7}{3} x +2$ \hfill 
  [$x > \dfrac{3}{38}$]
  \item  $-8 x +\dfrac{8}{3} \le -\dfrac{4}{3} x -\dfrac{11}{2}$ \hfill 
  [$x \ge \dfrac{49}{40}$]
  \item  $\dfrac{7}{4} x -4 \le -\dfrac{9}{2} x $ \hfill 
  [$x \le \dfrac{16}{25}$]
 \end{enumeratea}
\end{multicols}
\end{esercizio}

\begin{esercizio}[]
 \label{ese:dis_12}
Trova l'Insieme Soluzione delle seguenti disequazioni.
 \begin{multicols}{2}
 \begin{enumeratea}
\item $-3x-8 \ge~3(x+2)$
\item $4(x+1) < 2(2x+2)$
\item $x^{2}+4 > (x-3)^2$
\item $-4x^2 +3x+2 \le 2x(-2x+2)$
\item $3(2x+3)^2 \le~6x(2x+2)-1$
\item $x^{2}+3(x-4) < (x-1)^2$
\item $1-(2x-4)^{2} > -x(4x+1)+2$
\item $x^2-3x+5 \ge~(x+3)^2$
\item $x^2-2x-8 \le~(x-4)^2$
\item $(x+3)^{2} > (x-2)(x+2)$
\item $(x+1)^{2} \ge (x-1)^{2}$
\item $(x+1)^{2}-(2x +3)^2 \ge 5(x-2)^{2}$
\end{enumeratea}
\end{multicols}
\end{esercizio}

\begin{esercizio}[\Ast]
 \label{ese:dis_14}
Trova l'Insieme Soluzione delle seguenti disequazioni.
 \begin{multicols}{2}
 \begin{enumeratea}
\item $4x+4\ge~3\left(x+\dfrac{4}{3}\right)$
\item $\dfrac{x+5}{2}>-{\dfrac{1}{5}}$
\item $x^2+1\ge\dfrac{x^2+4x-1}{2}+3x$
\item $x+\dfrac{1}{2}<\dfrac{(x+3)}{3}-1$
\item $\dfrac{(x+5)}{3}+3+\dfrac{(x-1)}{3}\le x+4$
\item $\dfrac{3}{2}x+\dfrac{1}{4}<5\left(\dfrac{2}{3}x-\dfrac{1}{2}\right)$
\item $\dfrac{3}{2}(x+1)-\dfrac{1}{3}(1-x)<x+2$
\item $\dfrac{x+0,25}{2}<1,75+0,25x$
\end{enumeratea}
\end{multicols}
\end{esercizio}

\subsubsection*{\numnameref{sec:dis_prod_quo}}

% \begin{esercizio}
% \label{ese:21.54}
% Studia il segno della frazione
% \[f=\dfrac{x^{3}+11x^{2}+35x+25}{x^{2}-25}.\]
% \emph{Suggerimento}: scomponi in fattori numeratore e denominatore, otterrai
% \[ f=\dfrac{(x+5)^{2}(x+1)}{(x+5)(x-5)}.\]
% % Poniamo le~$\CE$ e semplifica la frazione: \dotfill
% % 
% % Studia separatamente il segno di tutti i fattori che vi compaiono. Verifica 
% che la tabella dei segni sia:
% % \begin{center}
% % \input{\folder lbr/fig031_seg.pgf}
% % \end{center}
% La frazione assegnata, con la~$\CE: x\neq -5\text{ e }x\neq~5$, si annulla 
% se~$x=-1$
% è positiva nell'insieme~$A^{+}=\left\{x\in \insR/-5<x<-1\vee x>5\right\}$, è 
% negativa in
% $A^{-}=\left\{x\in\insR/x<-5\vee -1<x<5\right\}$
% \end{esercizio}

\begin{esercizio}[]
\label{ese:dis_}
Determinate~$\IS$ delle seguenti disequazioni fratte.
\begin{multicols}{2}
\begin{enumeratea}
\spazielenx
\item $(x-2)(-3x+9)>0$
\item $(-3x+4)(-3x+5)<0$
\item $(-3x-4)(-3x-5)<0$
\item $(-7x-5)(-3x+9)\le0$
\item $(2x-1)(-7x+7)\ge0$
\item $(-6x-8)(-6x+2)\le0$
\item $(8x-3)(-5x+1)\ge0$
\end{enumeratea}
\end{multicols}
\end{esercizio}

\begin{esercizio}[]
\label{ese:21.55}
Determinate~$\IS$ delle seguenti disequazioni fratte.
\begin{multicols}{2}
\begin{enumeratea}
\spazielenx
\item $\dfrac{x-2}{3x-9}>0$ \hfill $\left[x<2\vee x>3\right]$
\item $\dfrac{x+2}{x-1}-\dfrac{2x+2}{x-1} < 0$ 
\hfill $\left[x< 0 \sor x>1\right]$
\item $\dfrac{4-3x}{6-5x}\geqslant \dfrac{-12+9x}{6-5x}$ 
\hfill $\left[x<1\vee x>4\right]$
\item $\dfrac{-x+8}{x-2}\ge~0$ \hfill $\left[...\right]$
\item $\dfrac{3x+4}{x^{2}+1}\ge\dfrac{6x+8}{x^{2}+1}$ 
 \hfill $\left[x<\dfrac{6}{5}\vee x\ge\dfrac{11}{9}\right]$
\item $\dfrac{-3x-8}{-x-3}\le~0$ \hfill $\left[...\right]$
\end{enumeratea}
\end{multicols}
\end{esercizio}

\begin{esercizio}[]
\label{ese:21.55}
Determinate~$\IS$ delle seguenti disequazioni fratte.
\begin{enumeratea}
\spazielenx
\item $\dfrac{3x+12}{(x-4)(6-3x)}\geqslant~0$
 \hfill $\left[x\le -4 \vee 2<x<4\right]$
\item $\dfrac{(x-4)(6-3x)}{3x+12}\geqslant~0$
 \hfill $\left[-{\dfrac{1}{2}}\le x\le~2\right]$
\item $\dfrac{4(x-3)+2(x+4)}{(x+4)(x-3)}\leqslant~0$
 \hfill $\left[x<-4\vee\dfrac{2}{3}\le x<3\right]$
\item $\dfrac{7(x+9)-6(x+3)}{(x+3)(x+9)}\geqslant~0$
 \hfill $\left[-45\le x<-9\vee x>-3\right]$
\item $\dfrac{4(-3x-3)+2(-6x+4)}{(-2x+4)(-x+3)}\leqslant~0$
 \hfill $\left[...\right]$
\item $\dfrac{(x+2)(-x-7)}{(-x+4)(3x-5)}\leqslant~0$
 \hfill $\left[...\right]$
\end{enumeratea}
\end{esercizio}

% \begin{esercizio}[\Ast]
% \label{ese:21.57}
% Determinate~$\IS$ delle seguenti disequazioni fratte.
% \begin{multicols}{2}
% \begin{enumeratea}
% \spazielenx
%  \item $\dfrac{3}{2-x}\leqslant \dfrac{1}{x-4}$
% \item $\dfrac{2}{4x-16}<\dfrac{2-6x}{x^{2}-8x+16}$
% \item $\dfrac{x-3}{x^{2}-4x+4}-1<\dfrac{3x-3}{6-3x}$
% \item $\dfrac{2}{x-2}>\dfrac{2x-2}{(x-2)(x+3)}$
% \end{enumeratea}
% \end{multicols}
% \end{esercizio}
% 
% \begin{esercizio}[\Ast]
% \label{ese:21.58}
% Determinate~$\IS$ delle seguenti disequazioni fratte.
% \begin{multicols}{2}
% \begin{enumeratea}
% \spazielenx
%  \item $\dfrac{5}{2x+6}\geqslant \dfrac{5x+4}{x^{2}+6x+9}$
% \item $\dfrac{x}{x+1}-\dfrac{1}{x^{3}+1}\le~0$
% \item $\dfrac{(x+3)(10x-5)}{x-2}<0$
% \item $\dfrac{4-3x}{x-2}<\dfrac{3x+1}{x-2}$
% \end{enumeratea}
% \end{multicols}
% \end{esercizio}
% 
% \begin{esercizio}[\Ast]
% \label{ese:21.59}
% Determinate~$\IS$ delle seguenti disequazioni fratte.
% \begin{multicols}{2}
% \begin{enumeratea}
% \spazielenx
%  \item $\dfrac{5x-4}{3x-12}\ge \dfrac{x-4}{4-x}$
% \item $\dfrac{2-x}{5x-15}\le \dfrac{5x-1}{2x-6}$
% \item $\dfrac{(3x-12)(6-x)}{(24-8x)(36-18x)}\leqslant~0$
% \item $\dfrac{(x-2)(5-2x)}{(5x-15)(24-6x)}\geqslant~0$
% \end{enumeratea}
% \end{multicols}
% \end{esercizio}

% \paragraph{\ref{ese:21.57}} a)~$2<x\le \dfrac{7}{2}\vee x>4$,\quad
% b)~$x<\dfrac{8}{13}$,\quad
% c)~$x<2\vee~2<x<\dfrac{5}{2}$,\quad d)~$x<-3\vee x>2$
% 
% \paragraph{\ref{ese:21.58}} a)~$x\le \dfrac{7}{5}\wedge x\neq-3$,\quad
% b)~$-1<x\le~1$,\quad
% \protect\\ c)~$x<-3\vee\dfrac{1}{2}<x<2$,\quad d)~$x<\dfrac{1}{2}\vee x>2$
% 
% \paragraph{\ref{ese:21.59}} a)~$x\le~2\vee x>4$,\quad b)~$x\le 
% \dfrac{1}{3}\vee 
% x>3$,\quad
% c)~$x<2\vee~3<x\le~4\vee x\ge~6$,\quad \protect\\ d)~$x\le~2\vee 
% \dfrac{5}{2}\le x<3\vee x>4$
% \end{multicols}
%\subsubsection*{21.3 - Sistemi di disequazioni}
\subsubsection*{\numnameref{sec:dis_sistemi}}

\begin{esercizio}[\Ast]
 \label{ese:21.37}
 Risolvi i seguenti sistemi di disequazioni.
 \begin{multicols}{2}
 \begin{enumeratea}
 \item $\left\{\begin{array}{l}
  3-x>x\\
  2x>3
        \end{array}\right.$
 \hfill $\left[\emptyset\right]$
\item $\left\{\begin{array}{l}
  3x\le~4\\
  5x\ge -4
   \end{array}\right.$
 \hfill $\left[-{\dfrac{4}{5}}\le x\le\dfrac{4}{3}\right]$
\item $\left\{\begin{array}{l}
  2x>3\\
  3x\le~4
        \end{array}\right.$
 \hfill $\left[\emptyset\right]$
\item $\left\{\begin{array}{l}
  3x-5<2\\
  x+7<-2x
   \end{array}\right.$
 \hfill $\left[x<-{\dfrac{7}{3}}\right]$
 \item $\left\{\begin{array}{l}
  3-x\ge x-3\\
  -x+3\ge~0
        \end{array}\right.$
 \hfill $\left[x\le~3\right]$
\item $\left\{\begin{array}{l}
  -x-3\le~3\\
  3+2x\ge~3x+2
   \end{array}\right.$
 \hfill $\left[-6\le x\le~1\right]$
\item $\left\{\begin{array}{l}
  2x-1>2x \\
  3x+3\le~3
        \end{array}\right.$
 \hfill $\left[\emptyset\right]$
\item $\left\{\begin{array}{l}
  2x+2<2x+3\\
  2(x+3)>2x+5
        \end{array}\right.$
 \hfill $\left[\insR\right]$
 \item $\left\{\begin{array}{l}
  -3x>0\\
  -3x+5\ge~0\\
  -3x\ge-2x
        \end{array}\right.$
 \hfill $\left[x<0\right]$
\item {\longarray $\left\{\begin{array}{l}
  -{\dfrac{4}{3}}x\ge\dfrac{2}{3}\\
  -{\dfrac{2}{3}}x\le\dfrac{1}{9}
        \end{array}\right.$}
 \hfill $\left[\emptyset\right]$
\item $\left\{\begin{array}{l}
  3+2x>3x+2 \\
  5x-4\le~6x-4\\
  -3x+2\ge -x-8
        \end{array}\right.$
 \hfill $\left[0\le x<1\right]$
\item $\left\{\begin{array}{l}
  4x+4\ge~3\cdot\left(x+\dfrac{4}{3}\right)\\
  4x+4\ge~2\cdot (2x+2)
        \end{array}\right.$
 \hfill $\left[x\ge~0\right]$
 \item $\left\{\begin{array}{l}
  3(x-1)<2(x+1)\\
  x-\dfrac{1}{2}+\dfrac{x+1}{2}>0
        \end{array}\right.$
 \hfill $\left[0<x<5\right]$
\item {\longarray $\left\{\begin{array}{l}
  16(x+1)-2+(x-3)^{2}\le(x+5)^{2}\\
        \dfrac{x+5}{3}+3+2\cdot\dfrac{x-1}{3}\le x+4
        \end{array}\right.$}
 \hfill $\left[\insR\right]$
\item $\left\{\begin{array}{l}
  x+\dfrac{1}{2}<\dfrac{1}{3}(x+3)-1\\
  (x+3)^{2}\ge (x-2)(x+2)
        \end{array}\right.$
 \hfill $\left[-{\dfrac{13}{6}}\le x<-{\dfrac{3}{4}}\right]$
\item {\longarray $\left\{\begin{array}{l}
        \dfrac{2x+3}{3}>x-1\\
        \dfrac{x-4}{5}<\dfrac{2x+1}{3}
        \end{array}\right.$}
 \hfill $\left[-{\dfrac{17}{7}}<x<6\right]$
\end{enumeratea}
\end{multicols}
\end{esercizio}

% \newpage

\subsubsection*{\numnameref{sec:dis_problemi}}

\begin{multicols}{2}

% \begin{esercizio}[]
%  \label{ese:dis_}
%  Sommando un numero con il triplo del suo successivo si deve ottenere
% un numero maggiore di~42. Quali numeri verificano questa
% condizione?
% \end{esercizio}
% 
% \begin{esercizio}[]
%  \label{ese:dis_}
%  La somma della metà di un numero più la sua terza parte deve essere minore 
% di 35. Quali numeri verificano questa condizione?
% \end{esercizio}
% 
% \begin{esercizio}[]
%  \label{ese:dis_}
%  Il triplo di un numero più il doppio del numero che lo precede non deve 
% essere minore di 56. Quali numeri verificano questa condizione?
% \end{esercizio}

\begin{esercizio}[]
 \label{ese:dis_}
 La differenza tra un numero e il triplo del numero che lo precede non deve 
essere maggiore della somma tra il doppio del numero e un quarto del numero 
che lo segue. Quali numeri verificano questa condizione?
\end{esercizio}

 \begin{esercizio}
 \label{ese:dis_}
 In una fabbrica, per produrre una certa merce, si ha
una spesa fissa settimanale di \officialeuro\ 413, ed un costo di produzione 
di \officialeuro\ 2,00 per ogni
kg di merce. Sapendo che la merce viene venduta a \officialeuro\ 4,00 al~kg, 
determinare la quantità minima da produrre
alla settimana perché l'impresa non sia in perdita.
 \end{esercizio}

 \begin{esercizio}[]
 \label{ese:dis_}
 Sommando due numeri pari consecutivi si deve ottenere un numero che
non supera la metà del numero più grande. Quali valori può
assumere il primo numero pari?
 \end{esercizio}

 \begin{esercizio}[]
 \label{ese:dis_}
 Il noleggio di una automobile costa \officialeuro\ 42,00 al giorno, più
\officialeuro\ 0,12 per ogni chilometro percorso. Qual è il massimo di
chilometri da percorrere giornalmente, per non spendere più di 
\officialeuro\ 75,00 al giorno?
 \end{esercizio}

 \begin{esercizio}[]
 \label{ese:dis_}
 Per telefonare in alcuni paesi esteri, una compagnia telefonica
propone due alternative di contratto:
\begin{enumeratea}
 \item \officialeuro\ 2,00 alla risposta più \officialeuro\ 0,25 per ogni 
minuto successivo;
\item \officialeuro\ 0,90 per ogni minuto di conversazione.
\end{enumeratea}
Quanti minuti deve durare una telefonata perché convenga la prima
alternativa?
 \end{esercizio}

\begin{esercizio}[]
 \label{ese:dis_}
 Il prezzo di un abbonamento mensile ferroviario è di \officialeuro\ 225,00.
 per la stessa tratta il prezzo di un singolo biglietto di A/R è 
di \officialeuro\ 15,50, trovare il numero minimo di viaggi per
cui l'abbonamento mensile risulta conveniente, e
rappresentare grafica-mente la soluzione.
 \end{esercizio}

%  \begin{esercizio}
%  \label{ese:21.22}
%  Al circolo tennis i soci pagano \officialeuro\ 12 a ora di gioco, i non
% soci pagano \officialeuro\ 15. Sapendo che la tessera annuale costa
% \officialeuro\ 150, dopo quante partite all'anno conviene
% fare la tessera di socio?
%  \end{esercizio}

\end{multicols}

\subsection{Esercizi riepilogativi}

\begin{esercizio}[\Ast]
 \label{ese:21.15}
Trova l'Insieme Soluzione delle seguenti disequazioni.
 \begin{enumeratea}
\item
 $\dfrac{1}{2}\left(3x-\dfrac{1}{3}\right)-\dfrac{1}{3}(1+x)(1-x)+
  3\left(\dfrac{1}{3}x-1\right)^{2}\ge~0$
 \hfill $\left[\insR\right]$
\item
 $3\dfrac{(x+1)}{2}-\dfrac{x+1}{3}-\dfrac{1}{9}>-5x+\dfrac{1}{2}$
 \hfill $\left[x>-{\dfrac{10}{111}}\right]$
\item
 $\left(\dfrac{x}{2}-1\right)\left(1+\dfrac{x}{2}\right)+x-\dfrac{1}{2}>
  x\dfrac{(x-1)}{4}+\dfrac{5x-6}{4}$
 \hfill $\left[\emptyset\right]$
\item
 $\dfrac{1}{2}\left(x-\dfrac{1}{2}\right)+\dfrac{1}{3}\left(x+
  \dfrac{1}{3}\right)>\dfrac{x-\dfrac{1}{2}}{3}+\dfrac{x-\dfrac{1}{3}}{2}$
 \hfill $\left[\insR\right]$
\end{enumeratea}
\end{esercizio}

\begin{esercizio}
 \label{ese:21.33}
Sulla retta reale rappresenta l'insieme soluzione~$S_{1}$
dell'equazione:
\[\dfrac{1}{6}+\dfrac{1}{4}\cdot (5x+3)=2+\dfrac{2}{3}\cdot (x+1)\]
e l'insieme soluzione~$S_{2}$ della disequazione:
\[\dfrac{1}{2}-2\cdot\left(\dfrac{1-x}{4}\right)\ge
~3-\dfrac{6-2x}{3}-\dfrac{x}{2}.\]

È vero che~$S_{1}\subset S_{2}$?
\end{esercizio}

\begin{esercizio}[\Ast]
 \label{ese:21.34}
 Determina i numeri reali che verificano il sistema:
 $\left\{%
  \begin{array}{l}
  x^{2}\le~0
  \\2-3x\ge~0
 \end{array}\right.$
 \hfill $\left[0\right]$
 \end{esercizio}

\begin{esercizio}
 \label{ese:21.35}
 L'insieme soluzione del sistema:
$\left\{\begin{array}{l}
  (x+3)^{3}-(x+3)\cdot (9x-2)>x^{3}+27\\
  \dfrac{x+5}{3}+3+\dfrac{2\cdot (x-1)}{3}<x+1
 \end{array}\right.$ è:
\begin{multicols}{2}
\boxA\quad~$\left\{x\in \insR/x>3\right\}$

\boxB\quad~$\left\{x\in \insR/x>-3\right\}$

\boxC\quad~$\left\{x\in \insR/x<-3\right\}$

\boxD\quad~$\IS=\emptyset $

\boxE\quad~$\left\{x\in\insR/x<3\right\}$
\end{multicols}

\end{esercizio}

\begin{esercizio}
 \label{ese:21.36}
 Attribuire il valore di verità alle seguenti proposizioni:

\begin{enumeratea}
\item il quadrato di un numero reale è sempre positivo;
\item l'insieme complementare di~$A=\{x\in\insR/x>-8\}\text{ è 
}B=\{x\in\insR/x<-8\}$
\item il monomio~$-6x^{3}y^{2}$ assume valore positivo per tutte le coppie 
dell'insieme~$\insR^{+}\times\insR^{+}$
\item nell'insieme~$\insZ$ degli interi relativi il 
sistema~$\left\{\begin{array}{l}x+1>0\\8x<0\end{array}\right.$ non ha soluzione;
\item l'intervallo~$\left[-1,\left.-{\dfrac{1}{2}}\right)\right.$ rappresenta 
l'$\IS$ del sistema~$\left\{\begin{array}{l}1+2x<0 \\\dfrac{x+3}{2}\le 
x+1\end{array}\right.$
\end{enumeratea}
\end{esercizio}

\begin{esercizio}[\Ast]
 \label{ese:21.41}
 Risolvi i seguenti sistemi di disequazioni.

 \begin{enumeratea}
 \item {\longarray $\left\{\begin{array}{l}
  2\left(x-\dfrac{1}{3}\right)+x>3x-2\\
	\dfrac{x}{3}-\dfrac{1}{2}\ge \dfrac{x}{4}-\dfrac{x}{6}
   \end{array}\right.;$}
 \hfill $\left[x\ge~2\right]$
\item $\left\{\begin{array}{l}
    \dfrac{3}{2}x+\dfrac{1}{4}<5\cdot\left(\dfrac{2}{3}x-\dfrac{1}{2}\right)\\
    x^2-2x+1\ge~0
   \end{array}\right.;$
 \hfill $\left[x>\dfrac{3}{2}\right]$
\item {\longarray $\left\{\begin{array}{l}
  3\left(x-\dfrac{4}{3}\right)+\dfrac{2-x}{3}+x-\dfrac{x-1}{3}>0\\
	
\left[1-\dfrac{1}{6}(2x+1)\right]+\left(x-\dfrac{1}{2}\right)^{2}<(x+1)^{2}
+\dfrac{1}{3}(1+2x)
   \end{array}\right.;$}
 \hfill $\left[x>\dfrac{9}{10}\right]$
\item {\longarray $\left\{\begin{array}{l}
	
\left(x-\dfrac{1}{2}\right)\left(x+\dfrac{1}{2}\right)>\left(x-\dfrac{1}{2}
\right)^{2}\\
  
2\left(x-\dfrac{1}{2}\right)\left(x+\dfrac{1}{2}\right)<\left(x-\dfrac{1}{2}
\right)^{2}+\left(x+\dfrac{1}{2}\right)^{2}
   \end{array}\right..$}
 \hfill $\left[x>\dfrac{1}{2}\right]$
 \end{enumeratea}
\end{esercizio}

%\subsubsection*{21.4 - Disequazioni polinomiali di grado superiore al primo}
% \subsubsection*{\numnameref{sec:fratta}}

% \newpage

\begin{esercizio}[\Ast]
 \label{ese:21.44}
Trova l'Insieme Soluzione delle seguenti disequazioni.
\begin{multicols}{1}
 \begin{enumeratea}
 \item $(x+2)(3-x)\le~0$ \hfill $\left[x\le -2\vee x\ge~3\right]$
\item $x(x-2)>0$ \hfill $\left[x<0\vee x>2\right]$
\item $(3x+2)(2-3x)<0$ \hfill $\left[x<-{\dfrac{2}{3}}\vee 
x>\dfrac{2}{3}\right]$
\item $-3x(2-x)(3-x)\ge~0$ \hfill $\left[x\ge~0\vee~2\le x\le~3\right]$
\item $(x+3)\cdot \left(\dfrac{1}{5}x+\dfrac{3}{2}\right)<0$ 
\item $\left(-{\dfrac{6}{11}}+2x\right)\cdot\left(-x+\dfrac{9}{2}\right)\le~0$
\item $\left(x+\dfrac{3}{2}\right)\cdot \left(5x+\dfrac{1}{5}\right)<0$
\item $\left(-{\dfrac{1}{10}}x+2\right)\cdot \left(-3x+9\right)\ge~0$
\item $(x-3)\cdot (2x-9)\cdot (4-5x)>0$
\item $(4x-2)\cdot (-3x+8)\cdot (-x+3)\ge0$
\item $\dfrac{(x-2)(x+4)(x+1)}{(x-1)(3x-9)(10-2x)}\leqslant~0$ 
 \hfill $\left[x\le -4\vee -1\le x<1\vee~2\le x<3\vee x>5\right]$
\item $\dfrac{(5-x)(3x+6)(x+3)}{(4-2x)(x-6)x}\leqslant~0$
 \hfill $\left[-3\le x\le -2\vee~0<x<2\vee~5\le x<6\right]$
\item $\dfrac{(x-5)(3x-6)(x-3)}{(4-2x)(x+6)x}\leqslant~0$
 \hfill $\left[x<-6\vee~0<x\le3\vee x\ge~5\text{ con }x\neq~2\right]$
\item $\dfrac{(x-3)(x+2)(15+5x)}{x^{2}-5x+4}\geqslant~0$
 \hfill $\left[-3\le x\le -2\vee~1<x\le~3\vee x>4\right]$
\end{enumeratea}
\end{multicols}
\end{esercizio}

% \begin{esercizio}[\Ast]
%  \label{ese:21.45}
% Trovare l'Insieme Soluzione delle seguenti disequazioni.
% \begin{multicols}{2}
%  \begin{enumeratea}
%  \item $(x+1)(1-x)\left(\dfrac{1}{2}x-2\right)\ge~0$
% \item $(x-1)(x-2)(x-3)(x-4)<0$
% \item $(x-4)(x+4)\le~0$
% \item $2x(2x-1)<0$
% \end{enumeratea}
% \end{multicols}
% \end{esercizio}
% 
% \begin{esercizio}[\Ast]
%  \label{ese:21.46}
% Trovare l'Insieme Soluzione delle seguenti disequazioni.
% \begin{multicols}{2}
%  \begin{enumeratea}
%  \item $x^{4}-81\ge~0$
% \item $x^{2}+17x+16\le~0$
% \item $16-x^{4}\le~0$
% \item $x^{2}+2x+1<0$
% \end{enumeratea}
% \end{multicols}
% \end{esercizio}
% 
% \begin{esercizio}[\Ast]
%  \label{ese:21.47}
% Trovare l'Insieme Soluzione delle seguenti disequazioni.
% \begin{multicols}{2}
%  \begin{enumeratea}
%  \item $x^{2}+6x+9\ge~0$
% \item $x^{2}-5x+6<0$
% \item $x^{2}+3x-4\le~0$
% \item $x^{3}>x^{2}$
% \end{enumeratea}
% \end{multicols}
% \end{esercizio}
% 
% \begin{esercizio}[\Ast]
%  \label{ese:21.48}
% Trovare l'Insieme Soluzione delle seguenti disequazioni.
% \begin{multicols}{2}
%  \begin{enumeratea}
%  \item $x^{2}(2x^{2}-x)-(2x^{2}-x)<0$
% \item $x^{2}-2x+1+x(x^{2}-2x+1)<0$
% \item $x^{3}-2x^{2}-x+2\ge~0$
% \item $x^{4}+4x^{3}+3x^{2}>0$
% \end{enumeratea}
% \end{multicols}
% \end{esercizio}
% 
% \begin{esercizio}[\Ast]
%  \label{ese:21.49}
% Trovare l'Insieme Soluzione delle seguenti disequazioni.
% \begin{multicols}{2}
%  \begin{enumeratea}
%  \item $(6x^{2}-24x)(x^{2}-6x+9)<0$
% \item $(x^{3}-8)(x+2)<(2-x)(x^{3}+8)$
% \item $(2a+1)(a^{4}-2a^{2}+1)<0$
% \item $x^{3}-6x^{2}+11>1-3x$
% \item $x^{6}-x^{2}+x^{5}-6x^{4}-x+6<0$
% \end{enumeratea}
% \end{multicols}
% \end{esercizio}
% 
% \begin{esercizio}[\Ast]
%  \label{ese:21.50}
%  Determinare i valori che attribuiti alla variabile~$y$ rendono positivi
% entrambi i polinomi
% seguenti:~$p_{1}=y^{4}-13y^{2}+36;\quad p_{2}=y^{3}-y^{2}-4y+4.$
% \end{esercizio}
% 
% \begin{esercizio}[\Ast]
%  \label{ese:21.51}
%  Determinare i valori di~$a$ che rendono~$p=a^{2}+1$ minore di~5.
% \end{esercizio}
% 
% \begin{esercizio}[\Ast]
%  \label{ese:21.52}
%  Determina~$\IS$ dei seguenti sistemi di disequazioni.
%  \begin{multicols}{3}
%  \begin{enumeratea}
%  \item $\left\{\begin{array}{l}
%   x^{2}-9\ge~0\\
%   x^{2}-7x+10<0
% 	   \end{array}\right.;$
% \item $\left\{\begin{array}{l}
%   x^{2}+3x-18\ge~0\\
%   12x^{2}+12x+3>0
% 	   \end{array}\right.;$
% \item $\left\{\begin{array}{l}
%   16x^{4}-1<0 \\
%   16x^{3}+8x^{2}\ge~0 \end{array}\right.. $
%  \end{enumeratea}
%  \end{multicols}
% \end{esercizio}
% 
% \begin{esercizio}[\Ast]
%  \label{ese:21.53}
%  Determina~$\IS$ dei seguenti sistemi di disequazioni.
%  \begin{multicols}{2}
%  \begin{enumeratea}
%  \item $\left\{\begin{array}{l}
%   49a^{2}-1\ge~0\\
%   9a^{2}<1\\
%   1-a>0
% 	   \end{array}\right.;$
% 
% \item $\left\{\begin{array}{l}
% 	  2x^{2}-13x+6<0\\
% 	  (2x^{2}-5x-3)(1-3x)>0\\
% 	  x^{2}+7>1
% 	   \end{array}\right..$
%  \end{enumeratea}
%  \end{multicols}
% \end{esercizio}

% \paragraph{\ref{ese:21.45}} a)~$x\le -1\vee~1\le x\le4$,\quad 
% b)~$1<x<2\vee~3<x<4$,\quad
% c)~$-4\le x\le~4$,\quad d)~$0<x<\dfrac{1}{2}$
% 
% \paragraph{\ref{ese:21.46}} a)~$x\le -3\vee x\ge~3$,\quad b)~$-16\le x\le 
% -1$,\quad
% c)~$x\le -2\vee x\ge~2$,\quad d)~$\emptyset $
% 
% \paragraph{\ref{ese:21.47}} a)~$\insR$,\quad b)~$2<x<3$,\quad
% c)~$-4\le x\le~1$,\quad d)~$x>1$
% 
% \paragraph{\ref{ese:21.48}} a)~$-1<x<0\vee \dfrac{1}{2}<x<1$,\quad \protect\\ 
% b)~$x<-1$,\quad
% c)~$-1\le x\le~1\vee x\ge~2$,\quad \protect\\ d)~$x<-3\vee x>-1\wedge x\neq~0$
% 
% \paragraph{\ref{ese:21.49}} a)~$0<x<4\wedge x\neq~3$,\quad b)~$-2<x<2$,\quad
% c)~$a<-{\dfrac{1}{2}}\wedge a\neq -1$,\quad d)~$-1<x<2\vee x>5$,\quad 
% e)~$-3<x<-1\vee~1<x<2$
% 
% \paragraph{\ref{ese:21.50}} $-2<y<1\vee y>3$
% 
% \paragraph{\ref{ese:21.51}} $-2<a<2$
% 
% \paragraph{\ref{ese:21.52}} a)~$3\le x<5$,\quad b)~$x\le -6\vee x\ge~3$,\quad
% c)~$-{\dfrac{1}{2}}<x<\dfrac{1}{2}$
% 
% \paragraph{\ref{ese:21.53}} a)~$-{\dfrac{1}{3}}<a\le -{\dfrac{1}{7}}\vee 
% \dfrac{1}{7}\le a<\dfrac{1}{3}$,\quad
% b)~$\dfrac{1}{2}<x<3$

% \newpage

% \begin{esercizio}[\Ast]
% \label{ese:21.61}
% Determinate~$\IS$ delle seguenti disequazioni fratte.
% \begin{multicols}{2}
% \begin{enumeratea}
% \spazielenx
% \item $\dfrac{\left(x-4\right)^{2}(x+3)}{x^{2}+5x+6}\geqslant~0$
% \item $\dfrac{x}{1-x^{2}}>\dfrac{1}{2x+2}-\dfrac{2}{4x-4}$
% \item $\dfrac{3-x}{x-2}<\dfrac{x-1}{x+3}+\dfrac{2}{x^{2}+x-6}$
% \item $\dfrac{2}{x+2}-\dfrac{1}{x+1}\ge \dfrac{3}{2x+2}$
% \end{enumeratea}
% \end{multicols}
% \end{esercizio}
% 
% \begin{esercizio}[\Ast]
% \label{ese:21.62}
% Determinate~$\IS$ delle seguenti disequazioni fratte.
% \begin{multicols}{2}
% \begin{enumeratea}
% \spazielenx
%  \item $\dfrac{3}{2x-1}\le \dfrac{2x^{2}}{2x^{2}-x}-\dfrac{x+1}{x}$
% \item $\dfrac{2x^{2}}{2x^{2}-x}>1$
% \item $\dfrac{2x}{2x-1}+\dfrac{x+2}{2x+1}>\dfrac{3}{2}$
% \item $\dfrac{x^{2}-5x+6}{x^{2}-7x+12}\le~1$
% \end{enumeratea}
% \end{multicols}
% \end{esercizio}
% 
% \begin{esercizio}[\Ast]
% \label{ese:21.63}
% Determinate~$\IS$ delle seguenti disequazioni fratte.
% \begin{multicols}{2}
% \begin{enumeratea}
% \spazielenx
%  \item $\dfrac{\dfrac{2}{x+1}}{x^{2}-1}<0$
% \item $\dfrac{x}{x+1}-\dfrac{4-x}{x+2}\ge \dfrac{2x+1}{x^{2}+3x+2}$
% \item $\dfrac{3}{2x^{2}-4x-6}-\dfrac{x-2}{3x+3}<\dfrac{x-1}{2x-6}$
% \item $\dfrac{1}{2-2x}\cdot 
% \left(\dfrac{x(x-2)}{x-1}-\dfrac{3}{3-3x}\right)>-1$
% \end{enumeratea}
% \end{multicols}
% \end{esercizio}
% 
% \begin{esercizio}[\Ast]
% \label{ese:21.64}
% Determinate~$\IS$ delle seguenti disequazioni fratte.
% 
% \begin{enumeratea}
%  \item 
% 
% $-{\dfrac{2}{27-3x^{2}}}-\dfrac{x+1}{2x-6}+\dfrac{3-2x}{6x-18}<-{\dfrac{3}{x^{2}
% -9}}+4\dfrac{x-3}{18-2x^{2}}$
% \item 
% $\dfrac{2}{x^{2}-3x+2}-\dfrac{x}{x-2}<\dfrac{x-1}{x-1}-\dfrac{1}{3x-x^{2}-2}
% +\dfrac{2-x}{4x-4}$
% \item 
% $\dfrac{(x-2)(x+4)(x^{2}+5x+6)}{(x^{2}-9)(-4-7x^{2})(x^{2}-6x+8)(x^{2}+4)}<0$
% \end{enumeratea}
% \end{esercizio}
% 
% \begin{esercizio}
% \label{ese:21.65}
% Dopo aver ridotto ai minimi termini la frazione
% $f=\dfrac{3x^{4}-2x^{3}+3x^{2}-2x}{6x^{2}-x-7}$, completa;
% 
%  \begin{enumeratea}
%  \item $f>0$ per~$x<-1$ oppure \dotfill
%  \item $f=0$ per \dotfill
%  \item $f<0$ per \dotfill
%  \end{enumeratea}
% \end{esercizio}
% 
% \begin{esercizio}
% \label{ese:21.66}
% Determinate il segno delle frazioni, dopo averle ridotte ai minimi termini.
% \[f_{1}=\dfrac{1-a^{2}}{2+3a};\quad 
% f_{2}=\dfrac{a^{3}-5a^{2}-3+7a}{9-6a+a^{2}};\quad 
% f_{3}=\dfrac{11m-m^{2}+26a}{(39-3m)(m^{2}+4m+4)}.\]
% \end{esercizio}
% 
% \begin{esercizio}[\Ast]
% \label{ese:21.67}
% Determinate~$\IS$ delle seguenti disequazioni fratte.
% \begin{multicols}{2}
% \begin{enumeratea}{\longarray
%  \item $\left\{\begin{array}{l}
%   \dfrac{2-x}{3x^{2}+x}\ge~0\\
%   x^{2}-x-6\ge~0\\
%   x^{2}-4\le~0
% 	\end{array}\right.;$
% \item $\left\{\begin{array}{l}
%         \dfrac{x^{2}-4x+4}{9-x^{2}>0}\\
%         x^{2}-3x\le~0
%        \end{array}\right.;$
% \item $\left\{\begin{array}{l}
% 	   \dfrac{1}{x-2}+\dfrac{3}{x+2}<0\\
% 	   \dfrac{2-x}{5x-15}\le\dfrac{5x-1}{2x-6}
% 	   \end{array}\right.;$
% \item $\left\{\begin{array}{l}
% 	   \dfrac{4}{8-4x}-\dfrac{6}{2x-4}<0\\
% 	   \dfrac{x}{x-2}-\dfrac{6}{x^{3}-8}>1
% 	   \end{array}\right.;$
% \item $\left\{\begin{array}{l}
% 	   
% \left(1+\dfrac{2}{x-2}\right)\left(1-\dfrac{2}{x-2}\right)<\dfrac{x-4}{2-x}\\
% 	   
% 
% \left(\dfrac{2-x}{x^{2}-6x+9}+\dfrac{2+x}{x^{2}-9}\right)\cdot{\dfrac{x^{3}-27}{
% 2x}}>0
% 	   \end{array}\right..$}
% \end{enumeratea}
% \end{multicols}
% \end{esercizio}
% 
% \begin{esercizio}[\Ast]
% \label{ese:21.68}
% Determinate~$\IS$ delle seguenti disequazioni fratte.
% \begin{multicols}{2}
% \begin{enumeratea}{\longarray
%  \item $\left\{\begin{array}{l}
%   \left(1-\dfrac{1}{x}\right)+3\left(\dfrac{2}{x}+1\right)>\dfrac{13}{2}\\
%   \dfrac{7+x}{2x}>\dfrac{2-x}{1-2x}
%    \end{array}\right.;$
% \item $\left\{\begin{array}{l}
%   \dfrac{x^{2}-2x-3}{2x^{2}-x-1}\ge~0\\
%   \dfrac{4x-1-3x^{2}}{x^{2}-4}\le~0
% 	\end{array}\right.;$
% \item $\left\{\begin{array}{l}
%   x^{2}-3x+2\le0\\
%   \dfrac{6}{2+x}-\dfrac{x+2}{x-2}>\dfrac{x^{2}}{4-x^{2}}
% 	\end{array}\right.;$
% \item $\left\{\begin{array}{l}
%   x^{2}+1\le -2x\\
%   3x-1<2\left(x-\dfrac{1}{2}\right)
%   \end{array}\right..$}
% \end{enumeratea}
% \end{multicols}
% \end{esercizio}
% 
% \paragraph{\ref{ese:21.61}} a)~$x>-2$,\quad b)~$x<-1$,\quad
% c)~$x<-3\vee -1<x<2\vee x>\dfrac{5}{2}$,\quad
% \protect\\ d)~$x\le -6\vee -2<x<-1$
% 
% \paragraph{\ref{ese:21.62}} a)~$x<0\vee\dfrac{1}{4}\le x<\dfrac{1}{2}$,\quad 
% b)~$x<\dfrac{1}{2}\wedge x\neq~0$,\quad
% c)~$-\dfrac{1}{2}<x<\dfrac{1}{10}\vee x>\dfrac{1}{2}$,\quad d)~$x<4\wedge 
% x\neq~3$
% 
% \paragraph{\ref{ese:21.63}} a)~$x<-1\vee -1<x<1$,\quad b)~$x<-2\vee x\ge 
% \dfrac{5}{2}$,\quad
% c)~$x<-1\vee~0<x<2\vee x>3$,\quad d)~$\insR-\{1\}$
% 
% \paragraph{\ref{ese:21.64}} a)~$x<-3\vee x>3$,\quad 
% b)~$x<0\vee~1<x<\dfrac{12}{7}\vee x>2$,\quad
% \protect\\ c)~$x<-4\vee -2<x<3\vee x>4\text{ con }x\neq2$
% 
% \paragraph{\ref{ese:21.67}} a)~$\left\{x\in\insR/x=-2\right\}$,\quad 
% b)~$\left\{x\in \insR/0\le x<3\text{ con }x\neq~2\right\}$,\quad
% c)~,$x<-2$\quad d)~$x>2$,\quad
% \protect\\ e)~$1<x<3\wedge x\neq~2$
% 
% \paragraph{\ref{ese:21.68}} a)~$0<x<\dfrac{7}{17}\vee\dfrac{1}{2}<x<2$,\quad 
% b)~$x<-2\vee \dfrac{1}{3}\le x<1\vee x\ge~3$,\quad
% \protect\\ c)~$1\le x<2$,\quad d)~$\emptyset $

% \begin{esercizio}
% \label{ese:21.69}
% Motivare la verità o la falsità delle seguenti
% proposizioni riferite alle frazioni.
% \begin{multicols}{3}
% \noindent\[f_{1}=\dfrac{a^{3}-81a}{81-a^{2}},\]
% \[f_{2}=\dfrac{7a^{2}+7}{3+3a^{4}+6a^{2}},\]
% \[f_{3}=\dfrac{20a-50a^{2}-2}{4a-20a^{2}},\]
% \[f_{4}=\dfrac{a^{4}}{2a^{4}+a^{2}},\]
% \[f_{5}=\dfrac{1-4a^{2}}{2-8a+8a^{2}},\]
% \[f_{6}=\dfrac{2a^{2}+a^{3}+a}{2a^{2}-a^{3}-a}.\]
% \end{multicols}
% \begin{enumeratea}
% \TabPositions{11cm}
% \item $f_{1}$ per qualunque valore positivo della variabile è negativa 
% \tab\boxV\quad\boxF
% \item $f_{2}$ è definita per qualunque valore attribuito alla variabile 
% \tab\boxV\quad\boxF
% \item $f_{3}$ è positiva nell'insieme~$\IS=\left\{a\in \insR/a<0\vee 
% a>\dfrac{1}{5}\right\}$ \tab\boxV\quad\boxF
% \item $f_{4}$ è positiva per qualunque valore reale attribuito alla variabile 
% \tab\boxV\quad\boxF
% \item nell'intervallo~${[}-\dfrac{1}{2},\dfrac{1}{2}{[}$, $f_{5}$ non si 
% annulla 
% \tab\boxV\quad\boxF
% \item $f_{6}$ è negativa per qualunque valore 
% dell'insieme~$K=\insR-\{-1,0,1\}$ \tab\boxV\quad\boxF
% \end{enumeratea}
% \end{esercizio}

\begin{multicols}{2}
 \begin{esercizio}[\Ast]
 \label{ese:21.23}
 \ In montagna l'abbonamento per due settimane allo
skipass costa \officialeuro\ 220 mentre il biglietto giornaliero costa
\officialeuro\ 20. Andando a sciare ogni giorno, dopo quanti giorni
conviene fare l'abbonamento? \hfill $\left[x>11\right]$
 \end{esercizio}

 \begin{esercizio}[\Ast]
 \label{ese:21.24}
 Marco ha preso alle prime tre prove di matematica i seguenti voti: 5;
5,5; 4,5. Quanto deve prendere alla quarta e ultima prova per avere almeno~6
di media? \hfill $\left[Almeno~9\right]$
 \end{esercizio}

 \begin{esercizio}
 \label{ese:21.25}
 Per produrre un tipo di frullatore un'azienda ha dei
costi fissi per \officialeuro\ 12\,000 a settimana e riesce a produrre~850
frullatori a settimana, ognuno dei quali ha un costo di produzione pari
a \officialeuro\ 34. L'azienda concorrente riesce a
vendere un frullatore analogo a \officialeuro\ 79. A quanto devono essere
venduti i frullatori in modo che l'azienda abbia un
utile e che il prezzo di vendita non sia superiore a quello del
prodotto concorrente?
 \end{esercizio}

 \begin{esercizio}[\Ast]
 \label{ese:21.26}
 Per noleggiare un'auto una compagnia propone
un'auto di tipo citycar al costo di \officialeuro\ 0,20 per km percorso e una 
quota fissa giornaliera
di \officialeuro\ 15,00,
un'auto di tipo economy al costo di \officialeuro\ 0,15
per km e una quota fissa giornaliera di \officialeuro\ 20,00. Dovendo
noleggiare l'auto per~3 giorni quanti km occorre fare
perché sia più conveniente l'auto di tipo economy?
 \hfill $\left[300\unit{km}\right]$
 \end{esercizio}

 \begin{esercizio}
 \label{ese:21.27}
 Alle~9.00 di mattina sono in autostrada e devo raggiungere una città
che dista~$740\unit{km}$ entro le~17.00 poiché ho un appuntamento di lavoro.
Prevedendo una sosta di mezzora per mangiare un panino, a quale
velocità devo viaggiare per arrivare in orario?
 \end{esercizio}

%  \begin{esercizio}[\Ast]
%  \label{ese:21.28}
%  Quanto deve essere lungo il lato di un triangolo equilatero il cui
% perimetro deve superare di~$900\unit{cm}$ il perimetro di un triangolo
% equilatero che ha il lato di~$10\unit{cm}$? 
% \hfill $\left[x>310\unit{cm}\right]$
%  \end{esercizio}

%  \begin{esercizio}[\Ast]
%  \label{ese:21.29}
%  I lati di un triangolo sono tali che il secondo è doppio del primo e
% il terzo è più lungo del secondo di~$3\unit{cm}$ Se il perimetro deve
% essere compreso tra~$10\unit{cm}$ e~$20\unit{cm}$, tra quali valori può variare 
% il lato
% più piccolo? \hfill 
% $\left[\dfrac{7}{5}\unit{cm}<x<\dfrac{17}{5}\unit{cm}\right]$
%  \end{esercizio}
% 
%  \begin{esercizio}[\Ast]
%  \label{ese:21.30}
%  In un triangolo isoscele l'angolo
% alla base deve essere minore della metà dell'angolo
% al vertice. Tra quali valori deve essere compresa la misura
% dell'angolo alla base? \hfill $\left[0^{\circ}<\alpha<45^{\circ}\right]$
%  \end{esercizio}
% 
%  \begin{esercizio}[\Ast]
%  \label{ese:21.31}
%  Un trapezio rettangolo l'altezza che è il triplo
% della base minore, mentre la base maggiore è~5 volte la base minore.
% Se il perimetro del trapezio non deve superare i~$100\unit{m}$, quali valori
% può assumere la lunghezza dell'altezza del
% trapezio? \hfill $\left[h\le \dfrac{150}{7}m\right]$
%  \end{esercizio}

%  \begin{esercizio}[\Ast]
%  \label{ese:21.32}
%  Un rettangolo ha le dimensioni una doppia dell'altra.
% Si sa che il perimetro non deve superare~$600\unit{m}$ e che
% l'area non deve essere inferiore a~$200\unit{m^2}$ Tra quali
% valori possono variare le dimensioni del rettangolo?
% \hfill [Il lato minore tra~$10\unit{m}$ e~$100\unit{m}$, 
%         il lato maggiore tra~$20\unit{m}$ e~$200\unit{m}$]
%  \end{esercizio}
\end{multicols}

% \subsection{Risposte}
% 
% \begin{multicols}{2}
% \paragraph{\ref{ese:21.10}} a)~$x<\dfrac{3}{2}$,\quad 
% b)~$x>\dfrac{3}{2}$,\quad
% c)~$x\le \dfrac{4}{3}$,\quad d)~$x\ge -{\dfrac{4}{5}}$,\quad
% e)~$\insR$,\quad f)~$\emptyset $,\quad
% g)~$x<3$,\quad \protect\\ h)~$x\ge -3$
% 
% \paragraph{\ref{ese:21.11}} a)~$x\le~1$,\quad b)~$x\le~0$,\quad
% c)~$x\le~5$,\quad d)~$\emptyset $,\quad
% e)~$\insR$,\quad f)~$\insR$,\quad
% g)~$\insR $,\quad h)~$\emptyset $
% 
% \paragraph{\ref{ese:21.12}} a)~$\emptyset $,\quad b)~$\insR$,\quad
% c)~$\emptyset $,\quad d)~$x\le -{\dfrac{10}{3}}$,\quad
% e)~$x<0$,\quad f)~$x\ge~0$,\quad
% g)~$x\le \dfrac{5}{3}$,\quad h)~$x\le -{\dfrac{8}{3}}$
% 
% \paragraph{\ref{ese:21.13}} a)~$x\ge~0$,\quad b)~$x\le -{\dfrac{3}{4}}$,\quad
% c)~$x\le~0$,\quad d)~$x\le -{\dfrac{1}{2}}$,\quad
% e)~$x\ge -{\dfrac{1}{6}}$,\quad f)~$x\ge -{\dfrac{27}{2}}$,\quad
% \protect\\g)~$x>-{\dfrac{27}{5}}$,\quad h)~$\insR$
% 
% \paragraph{\ref{ese:21.14}} a)~$x<-{\dfrac{3}{4}}$,\quad b)~$\insR$,\quad
% c)~$x\ge -{\dfrac{13}{6}}$,\quad d)~$x>\dfrac{3}{2}$,\quad
% e)~$x>1$,\quad f)~$x\ge~0$,\quad\protect\\
% g)~$\{x\in\insR/x<1\}=(-\infty,1)$,\quad h)~$x<\dfrac{13}{2}$

% \paragraph{\ref{ese:21.16}} $x>5$
% 
% \paragraph{\ref{ese:21.17}} $x\le -2/3$
% 
% \paragraph{\ref{ese:21.18}} Massimo~$294\unit{km}$
% 
% \paragraph{\ref{ese:21.20}} Meno di~3 minuti.
% 
% \paragraph{\ref{ese:21.21}} 14
% 
