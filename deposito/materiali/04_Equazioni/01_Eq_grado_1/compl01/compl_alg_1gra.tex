% (c) 2012-2013 Claudio Carboncini - claudio.carboncini@gmail.com
% (c) 2012-2014 Dimitrios Vrettos - d.vrettos@gmail.com
% (c) 2015 Daniele Zambelli daniele.zambelli@gmail.com

\input{\folder compl_alg_1gra_grafici}

\chapter{Complementi di algebra di primo grado}

\section{Equazioni di grado superiore al primo riducibili al primo grado}
\label{sec:compl1_eqgradosup}

Le equazioni di grado superiore al primo possono, in certi casi, essere 
ricondotte ad equazioni di primo grado, utilizzando la legge di annullamento 
del prodotto.

\begin{definizione}
 Nei numeri reali, un prodotto di più fattori è uguale a zero se e solo se 
 almeno uno dei fattori è uguale a zero.
\end{definizione}

Questa legge può essere usata per risolvere equazioni di questo tipo:
\[(x-4)(x+2)(3x-1)=0\]
Per ottenere gli zeri della funzione basta uguagliare a zero uno alla volta 
ogni fattore:
\begin{center} 
\zeriprodotto{(x-4)(x+2)(3x-1)=0}
                 {+1.6/{3x-1=0~ \sRarrow ~x_1=+\dfrac{1}{3}},
                  -0.8/{~~x+2=0~ \sRarrow ~x_2=-2},
                  -2.9/{~~x-4=0~ \sRarrow ~x_3=+4}}
                 {-1.3}
\end{center} 
Infatti il prodotto precedente vale zero se almeno uno dei fattori è zero; 
la verifica delle soluzioni è immediata.

% \begin{itemize} [nosep]
%  \item \(x-4=0\) cioè \(x=4\);
%  \item \(x+2=0\) cioè \(x=-2\);
%  \item \(3x-1=0\) cioè \(x=frac{1}{3}\);
% \end{itemize}
% Verifichiamo le tre soluzioni trovate:
% \begin{itemize} [nosep]
%  \item se \(x=4\) \quad allora \quad 
%  \((4-4)(4+2)(3 \cdot 4-1)=0 \cdot 6 \cdot 11=0\);
%  \item se \(x=-2\) \quad allora \quad 
%  \((-2-4)(-2+2)(3 \cdot (-2)-1)=-6 \cdot 0 \cdot (-7)=0\);
%  \item se \(x=\frac{1}{3}\) \quad allora \quad 
%  \((x-4)(x+2)(3x-1)=
%            (\frac{1}{3}-4)(\frac{1}{3}+2)(\frac{1}{3} \cdot 3 -1)=
%            -\frac{11}{3} \cdot \frac{7}{3} \cdot 0 =0\).
% \end{itemize}

È abbastanza facile convincersi che l'equazione non ha altre soluzioni, 
infatti qualunque altro numero messo al posto della \(x\) trasforma 
l'espressione a primo membro nel prodotto di tre numeri diversi da zero e nei
numeri reali il prodotto di numeri tutti diversi da zero non può essere zero.

In generale, se si ha un'equazione di grado~\(n\) scritta in forma 
normale~\(P(x)=0\) e se il polinomio~\(P(x)\) è fattorizzabile nel prodotto 
di~\(n\) fattori di primo grado, le soluzioni dell'equazione 
sono le soluzioni delle~\(n\) equazioni di primo grado che si ottengono 
uguagliando a zero ogni singolo fattore:
\begin{center} 
\zeriprodotto{(x+a_1)(x+a_2)\ldots (x+a_n)=0}
               {+2.1/{x+a_n=0~ \sRarrow ~x_1=-a_n},
                +0.5/{\qquad \dots \quad ~~~~ \sRarrow \qquad \dots},
                -1.4/{x+a_2~=0~ \sRarrow ~x_1=-a_2},
                -3.8/{x+a_1~=0~ \sRarrow ~x_n=-a_1}}
               {-1.0}
\end{center} 

La legge di annullamento del prodotto può essere utilizzata anche per 
equazioni che, apparentemente, non presentano un prodotto posto uguale a 
zero, vediamo un esempio.

\pagebreak %--------------------------------------

 \begin{esempio}
Risolvere~\(x^{2}-4=0\)
\begin{center} 
\zeriprodotto{x^2-4=(x-2)(x+2)=0}
               {+1.8/{x+2=0~ \sRarrow ~x_1=-2},
                -0.2/{x-2=0~ \sRarrow ~x_2=+2}}
               {-0.8}
\end{center} 
 \end{esempio}

% \begin{exrig}
 \begin{esempio}
Risolvere~\(x^2-x-2=0\)
\begin{center} 
\zeriprodotto{x^2-x-2=(x+1)(x-2)=0}
               {+2.4/{x-2=0~ \sRarrow ~x_1=+2},
                +0.4/{x+1=0~ \sRarrow ~x_2=-1}}
               {-0.8}
\end{center} 
 \end{esempio}

 \begin{esempio}
Risolvere~\(x^4-10x^2+9=0\)

Scomponendo in fattori il polinomio a primo membro, utilizzando la regola 
della scomposizione del particolare trinomio di secondo grado, si 
ottiene:~\((x^2-1)(x^2-9)=0\). Scomponendo ulteriormente in fattori si ha:
\((x-1)(x+1)(x-3)(x+3)=0\), da cui:
\begin{center} 
\zeriprodotto{(x-1)(x+1)(x-3)(x+3)=0}
                 {+2.5/{x+3=0~ \sRarrow ~x_1=-3},
                  +0.4/{x-3=0~ \sRarrow ~x_2=+3},
                  -1.7/{x+1=0~ \sRarrow ~x_2=-1},
                  -3.8/{x-1=0~ \sRarrow ~x_2=+1}}
                 {-0.8}
\end{center} 
 \end{esempio}

% \end{exrig}

% \ovalbox{\risolvii \ref{ese:20.1}, \ref{ese:20.2}, \ref{ese:20.3}, 
% \ref{ese:20.4}, \ref{ese:20.5}, \ref{ese:20.6}, \ref{ese:20.7}, 
% \ref{ese:20.8}, \ref{ese:20.9}, \ref{ese:20.10}}
% 
% \vspazio\ovalbox{\ref{ese:20.11}, \ref{ese:20.12}, \ref{ese:20.13}}

\section{Equazioni numeriche frazionarie}
\label{sec:compl1_eqfratte}

Nelle equazioni affrontate fin'ora abbiamo incontrato delle frazioni, ma 
l'incognita non è mai comparsa a denominatore. Una equazione in cui 
l'incognita si trova a denominatore si dice \emph{equazione fratta} o
\emph{equazione frazionaria}. Per essere risolte le equazioni fratte hanno 
bisogno di particolare attenzione infatti finché non abbiamo risolto 
l'equazione non possiamo dire se il denominatore è uguale o diverso da zero, 
ma noi sappiamo che in ogni frazione il denominatore \emph{deve} essere 
diverso da zero. 
% Perciò solo dopo aver risolto l'equazione possiamo dire se l'equazione 
% stessa aveva senso perciò se la soluzione trovata è accettabile o no.

\begin{definizione} 
Si chiama frazionaria o fratta un'equazione in cui l'incognita compare 
al denominatore.
\end{definizione}

\begin{procedura}
Per risolvere una equazione fratta dobbiamo:
\begin{enumerate*}
\item scomporre in fattori i denominatori;
\item eseguire il denominatore comune su tutta l'equazione;
\item scrivere le condizioni di accettabilità e eliminare i denominatori;
\item risolvere l'equazione intera così ottenuta;
\item controllare se le soluzioni trovate sono accettabili.
\end{enumerate*}
\end{procedura}

% \newpage %------------------------------------------------------------

% \begin{exrig}
 \begin{esempio}
Risolvere~\(\dfrac{x^{2}+x-2}{x^{2}-x}=1-\dfrac{5}{2x}\)

\begin{enumerate*}
\item scomporre in fattori i denominatori:~
\(\dfrac{x^{2}+x-2}{x(x-1)}=1-\dfrac{5}{2x}\)

\item eseguire il denominatore comune su tutta l'equazione
\[\frac{2(x^{2}+x-2)}{2x(x-1)}=\frac{2x(x-1)-5(x-1)}{2x(x-1)} \sRarrow
\frac{2x^{2}+2x-4}{2x(x-1)}=\frac{2x^2-2x-5x+5}{2x(x-1)}\]

\item scrivere le condizioni di accettabilità e eliminare i denominatori:
\[\sistema{x \neq 0 \\ 
           x \neq 1 \\ 
           \cancel{+2x^2}~\cancel{-2x^2} +2x +2x +5x = +4 +5 \sRarrow
             9x = 9 \sRarrow x=1}\]

% \item risolvere l'equazione intera così ottenuta:~
% \(9x = 9 \sRarrow x=1\)

\item controllare se le soluzioni trovate sono accettabili:\\
il valore~\(1\) è la soluzione dell'equazione intera ottenuta eliminando i 
denominatori, ma non è soluzione dell'equazione di partenza.
\end{enumerate*}
 \end{esempio}
 
%  \begin{esempio}
% Risolvere~.
% \end{esempio}
% 
% \begin{enumeratea}
%  \item Determiniamo il~\(\mcm\) dei denominatori, per fare questo dobbiamo 
% prima scomporli in fattori.
%     Riscriviamo:~\(\frac{x^{2}+x-3}{x\cdot (x-1)}=1-\frac{5}{2x}\) 
% con~\(\mcm=2x\cdot (x-1)\)
%  \item condizioni di esistenza: \[x-1\neq~0\wedge~2x\neq~0,\] 
% cioè~\(x\neq~1\wedge x\neq~0\), il dominio è~\(\Dom=\insR-\{1,0\}\)
%  \item trasportiamo al primo membro ed uguagliamo a zero 
% \[\frac{x^{2}+x-3}{x\cdot (x-1)}-1+\frac{5}{2x}=0\]
%     e riduciamo allo stesso denominatore (\(\mcm\)) ambo i membri 
% \[\frac{2x^{2}+2x-6-2x^{2}+2x+5x-5}{2x\cdot (x-1)}=0;\]
%  \item applichiamo il secondo principio di equivalenza moltiplicando ambo i 
% membri per il~\(\mcm\),
%     certamente diverso da zero per le condizioni poste in precedenza. 
% L'equazione diventa:~\(2x^{2}+2x-6-2x^{2}+2x+5x-5=0\)
%  \item riduciamo i monomi simili per portare l'equazione alla forma 
% canonica:~\(9x=11\)
%  \item dividiamo ambo i membri per~\(9\), otteniamo:~\(x=\frac{11}{9}\)
%  \item confrontando con le~\(\CE\), la soluzione appartiene all'insieme~\(\Dom\), 
% dunque è accettabile e l'insieme soluzione è:
%     \(\IS=\left\{\frac{11}{9}\right\}\).
% \end{enumeratea}
% 
% 
% In questa equazione l'incognita compare anche al denominatore.
% Ma i denominatori devono essere diversi da zero. Se \(x\) è uguale a \(-1\) 
% o \(x\) è uguale a \(+2\) l'espressione precedente non può essere calcolata.
% Questo ci dice che la soluzione che troveremo sarà accettabile solo se 
% \(x \neq -1\) e \(x \neq +2\).

% % \begin{exrig}
 \begin{esempio}
Risolvere~\(\dfrac{3x-2}{1+x}=\dfrac{3x}{x-2}\)
 
\begin{enumerate*}
\item scomporre in fattori i denominatori e eseguire il  denominatore comune 
su tutta l'equazione
\[\frac{(x-2)(3x-2)}{(1+x)(x-2)}=\frac{3x(x+1)}{(x+1)(x-2)}\Rightarrow
\frac{3x^{2}-2x-6x+4}{(1+x)(x-2)}=\frac{3x^2+3x}{(x+1)(x-2)}\]
\item scrivere le condizioni di accettabilità e eliminare i denominatori,
risolvere l'equazione intera così ottenuta e verificare che la soluzione sia 
accettabile:
\[\sistema{x \neq -1 \\ 
           x \neq +2 \\ 
           -11x = -4 \sRarrow x=\dfrac{4}{11} \quad 
             \text{Soluzione Accettabile.}}\]
\end{enumerate*}
\(\dfrac{4}{11}\) è soluzione dell'equazione intera ottenuta eliminando i 
denominatori, e è anche soluzione dell'equazione di partenza.
 \end{esempio}

% Per risolvere un'equazione frazionaria, prima di tutto dobbiamo renderla 
% nella forma
% \begin{equation*}
% \frac{F(x)}{G(x)}=0.
% \end{equation*}
% 
% \begin{enumeratea}
%  \item Determiniamo il~\(\mcm\) dei denominatori, \(\mcm=(1+x)\cdot (x-2)\).
%     Osserviamo che per~\(x = -1\) oppure per~\(x = 2\) le frazioni perdono di 
% significato, in quanto si annulla il denominatore;
%  \item imponiamo le condizioni di esistenza:~\(1+x\neq~0\) e~\(x-2\neq~0\) 
% cioè~\(\CE x\neq -1\wedge x\neq~2\). La ricerca dei valori
%     che risolvono l'equazione viene ristretta 
% all'insieme~\(\Dom=\insR-\{-1,2\}\) detto \emph{dominio} dell'equazione o
%     \emph{insieme di definizione};
%  \item applichiamo il primo principio d'equivalenza trasportando al primo 
% membro la frazione che si trova al secondo membro
%     e riduciamo allo stesso denominatore (\(\mcm\)),
%     \begin{equation*}
%       \frac{(3x-2)\cdot (x-2)-3x\cdot (1+x)}{(1+x)\cdot (x-2)}=0;
%     \end{equation*}
%  \item applichiamo il secondo principio di equivalenza moltiplicando ambo i 
% membri per il~\(\mcm\),
%     certamente diverso da zero per le condizioni poste precedentemente. 
% L'equazione diventa:~\((3x-2)\cdot (x-2)-3x\cdot (1+x)=0\)
%  \item eseguiamo le moltiplicazioni e sommiamo i monomi simili per portare 
% l'equazione alla forma canonica:
%     \(3x^{2}-6x-2x+4-3x-3x^{2}=0 \Rightarrow -11x=-4\)
%  \item dividiamo ambo i membri per~\(-11\), per il secondo principio di 
% equivalenza si ha:~\(x=\frac{4}{11}\)
%  \item confrontiamo il valore trovato con le~\(\CE\): in questo caso la 
% soluzione appartiene al dominio~\(\Dom\), quindi possiamo concludere
%     che è accettabile. L'insieme soluzione 
% è:~\(\IS=\left\{\frac{4}{11}\right\}\).
% \end{enumeratea}

% \end{exrig}

% \ovalbox{\risolvii \ref{ese:20.15}, \ref{ese:20.16}, \ref{ese:20.17}, 
% \ref{ese:20.18}, \ref{ese:20.19}, \ref{ese:20.20}, \ref{ese:20.21},
% \ref{ese:20.22}, \ref{ese:20.23}, \ref{ese:20.24}, \ref{ese:20.25}}
% 
% \vspazio\ovalbox{\ref{ese:20.26}, \ref{ese:20.27}}

\section{Equazioni letterali}
\label{sec:compl1_eqletterali}
Quando si risolvono problemi, ci si ritrova a dover tradurre nel linguaggio 
simbolico delle proposizioni del tipo:
<<Un lato di un triangolo scaleno ha lunghezza pari a~\(k\) volte la lunghezza 
dell'altro e la loro somma è pari a~\(2k\)>>.
Poiché la lunghezza del lato del triangolo non è nota, ad essa si attribuisce 
il valore incognito~\(x\) e quindi la proposizione
viene tradotta dalla seguente equazione:~\(x+kx=2k\).

È possibile notare che i coefficienti dell'equazione non sono solamente 
numerici, ma contengono una lettera dell'alfabeto diversa
dall'incognita. Qual è il ruolo della lettera~\(k\)?
Essa prende il nome di \emph{parametro} ed è una costante che rappresenta dei 
numeri fissi, quindi, può assumere dei valori prefissati.
Ogni volta che viene fissato un valore di~\(k\), l'equazione precedente assume 
una diversa forma. Infatti si ha:
\begin{center}
\begin{tabular}{ll}
\toprule
Valore di~\(K\) & Equazione corrispondente\\
\midrule
\(k=0\) & \(x=0\)\\
\(k=2\) & \(x+2x=4\)\\
\(k=-\frac{1}{2}\) & \(x-\frac{1}{2}x=-1\)\\
\bottomrule
\end{tabular}
\end{center}
Si può quindi dedurre che il parametro diventa una costante, all'interno 
dell'equazione nell'incognita~\(x\), ogni volta che se ne sceglie il valore.

Si supponga che il parametro~\(k\) assuma valori all'interno dell'insieme dei 
numeri reali. Lo scopo è quello di risolvere l'equazione,
facendo attenzione a rispettare le condizioni che permettono l'uso dei 
principi d'equivalenza e che permettono di ridurla in forma normale.

Riprendiamo l'equazione  \(x+kx=2k\), raccogliamo a fattore comune la~\(x\) si ha:
\begin{equation*}
 (k+1)x=2k.
\end{equation*}
Per determinare la soluzione di questa equazione di primo grado, è necessario 
utilizzare il secondo principio d'equivalenza e
dividere ambo i membri per il coefficiente~\(k+1\).
Si ricordi però che il secondo principio ci permette di moltiplicare o 
dividere i due membri dell'equazione per una stessa espressione,
purché questa sia diversa da zero.
Per questa ragione, nella risoluzione dell'equazione~\((k+1)x=2k\) è necessario 
distinguere i due casi:
\begin{itemize*}
\item se~\(k+1\neq~0\), cioè se~\(k\neq -1\), è possibile dividere per~\(k+1\) e si 
ha~\(x=\frac{2k}{k+1}\)
\item se~\(k+1=0\), cioè se~\(k=-1\), sostituendo tale valore all'equazione si 
ottiene l'equazione~\((-1+1)x=2\cdot (-1)\),
   cioè~\(0\cdot x=-2\) che risulta impossibile.
\end{itemize*}
Riassumendo si ha:
\begin{center}
\begin{tabular}{lcc}
\toprule
\multicolumn{3}{c} {\(x+kx=2k\) con~\(k \in \insR\)}\vspace{1.05ex}\\
Condizioni sul parametro & Soluzione & Equazione\\
\midrule
\(k=-1\) & nessuna & impossibile \\
\(k\neq-1\) & \(x=\frac{2k}{k+1}\) & determinata \\
\bottomrule
\end{tabular}
\end{center}
Ritorniamo ora al problema sul triangolo, spesso nell'enunciato del problema 
sono presenti delle limitazioni implicite
che bisogna trovare. Infatti, dovendo essere~\(x\) un lato del triangolo esso 
sarà un numero reale positivo.
Di conseguenza, dovendo essere l'altro lato uguale a~\(k\) volte~\(x\), il valore 
di~\(k\) deve necessariamente essere anch'esso positivo, ovvero~\(k>0\).
Di conseguenza il parametro~\(k\) non può mai assumere il valore~\(-1\) e quindi 
il problema geometrico ammette sempre una soluzione.

Questa analisi effettuata sui valori che può assumere il parametro~\(k\), prende 
il nome di \emph{discussione dell'equazione}.

\begin{procedura}
Stabilire quando una equazione è determinata, indeterminata, impossibile.

In generale, data l'equazione~\(ax+b=0\) si ha~\(ax=-b\) e quindi:
\begin{enumeratea}
\item se~\(a\neq~0\), l'equazione è determinata e ammette l'unica 
soluzione~\(x=-\frac{b}{a}\)
\item se~\(a=0\) e~\(b\neq~0\), l'equazione è impossibile;
\item se~\(a=0\) e~\(b=0\), l'equazione è soddisfatta da tutti i valori reali 
di~\(x\), ovvero è indeterminata.
\end{enumeratea}
\end{procedura}

% \newpage %------------------------------------------------------------

% \begin{exrig}
 \begin{esempio}
Risolvere e discutere~\(1+x+m=(x+1)^{2}-x(x+m)\).

Dopo aver fatto i calcoli si ottiene l'equazione~\((m-1)\cdot x=-m\) e quindi 
si ha:
\begin{itemize*}
 \item Se~\(m-1\neq~0\), cioè se~\(m\neq~1\), è possibile dividere ambo i membri 
 per~\(m-1\) e si ottiene l'unica soluzione~\(x=-{\frac{m}{m-1}}\)
 \item se~\(m-1=0\), cioè se~\(m=1\), sostituendo nell'equazione il valore~\(1\) si 
 ottiene~\(0\cdot x=-1\), che risulta impossibile.
\end{itemize*}
 \end{esempio}

 \begin{esempio}
Risolvere e discutere~\((k+3)x=k+4x(k+1)\).

Effettuando i prodotti si ottiene l'equazione:~\((3k+1)x=-k\) e quindi si ha:
\begin{itemize*}
 \item Se~\(3k+1\neq~0\), cioè se~\(k\neq -{\frac{1}{3}}\), è possibile dividere 
 ambo i membri per~\(3k+1\) e si ottiene l'unica soluzione~\(x=\frac{-k}{3k+1}\)
 \item se~\(k=-{\frac{1}{3}}\), sostituendo questo valore di~\(k\) nell'equazione 
 si ottiene~\(0\cdot x=\frac{1}{3}\), che risulta un'equazione impossibile.
\end{itemize*}
 \end{esempio}

 \begin{esempio}
Risolvere e discutere~\(a^{2}\cdot x=a+1+x\).

Portiamo al primo membro tutti i monomi che contengono 
l'incognita~\(a^{2}\cdot x-x=a+1\).
Raccogliamo a fattore comune l'incognita~\(x\cdot \left(a^{2}-1\right)=a+1\).
Scomponendo in fattori si ha 
l'equazione~\(x\cdot \left(a-1\right)\left(a+1\right)=a+1\).

I valori di~\(a\) che annullano il coefficiente dell'incognita 
sono~\(a=1\) e~\(a=-1\).
\begin{itemize*}
 \item Se nell'equazione sostituisco~\(a=1\), ottengo l'equazione~\(0x=2\) che è 
 impossibile;
 \item se sostituisco~\(a=-1\), ottengo l'equazione~\(0x=0\) che è indeterminata;
 \item escludendo i casi~\(a=1\) e~\(a=-1\), che annullano il coefficiente 
  della~\(x\) posso applicare il secondo principio
  di equivalenza delle equazione e dividere primo e secondo membro 
  per~\(a+1\), 
  ottengo~\(x=\frac{a+1}{\left(a+1\right)\cdot \left(a-1\right)}=\frac{1}{a-1}\).
\end{itemize*}
 \end{esempio}
Ricapitolando:
se~\(a=1\), 
allora~\(\IS=\emptyset\) se~\(a=-1\), allora~\(\IS=\insR\) 
se~\(a\neq +1\wedge a\neq -1\), 
allora~\(\IS=\left\{\frac{1}{a-1}\right\}\).\vspace*{1.05ex}
% \end{exrig}

% \ovalbox{\risolvii \ref{ese:20.34}, \ref{ese:20.35}, \ref{ese:20.36}, 
% \ref{ese:20.37}, \ref{ese:20.38}, \ref{ese:20.39}, \ref{ese:20.40}}
% 
% \subsection{Equazioni con due parametri}
% % \begin{exrig}
%  \begin{esempio}
% Risolvere e discutere~\((b+a)x-(b+2)(x+1)=-1\).
% 
% Mettiamo l'equazione in forma canonica:~\(bx+ax-bx-b-2x-2=-1\).
% Raccogliamo a fattore comune l'incognita~\((a-2)x=b+1\).
% \begin{itemize*}
%  \item Se~\(a-2=0\) l'equazione è impossibile o indeterminata. In questo caso:
%   \begin{itemize*}
%    \item se~\(b+1=0\) è indeterminata;
%    \item se~\(b+1\neq~0\) è impossibile;
%   \end{itemize*}
%  \item se~\(a-2\neq~0\) l'equazione è determinata e la sua soluzione 
% è~\(x=\frac{b+1}{a-2}\).
% \end{itemize*}
%  \end{esempio}
% Riassumendo:
% se~\(a=2\wedge b=-1\) allora~\(\IS=\insR\); se~\(a=2\wedge b\neq -1\) 
% allora~\(\IS=\emptyset\); se~\(a\neq~2\wedge b\neq -1\) 
% allora~\(\IS=\left\{\frac{b+1}{a-2}\right\}\).\vspace{1.05ex}
% % \end{exrig}
% 
% \ovalbox{\risolvii \ref{ese:20.41}, \ref{ese:20.42}, \ref{ese:20.43}}
% 
% \subsection{Equazioni letterali, caso in cui il denominatore contiene il 
% parametro}
% % \begin{exrig}
%  \begin{esempio}
% Risolvere e discutere~\(\frac{x+a}{2a-1}-\frac{1}{a-2a^{2}}=\frac{x}{a}\) 
% con~\(a\in \insR\).
% 
% Questa equazione è intera, pur presentando termini frazionari.
% Sappiamo che ogni volta che viene fissato un valore per il parametro, 
% l’equazione assume una forma diversa;
% la presenza del parametro al denominatore ci obbliga ad escludere 
% dall’insieme dei numeri reali quei valori che annullano il denominatore.
% 
% Per~\(a=0\vee a=\frac{1}{2}\) si annullano i denominatori quindi l’equazione è 
% priva di significato.
% Per risolvere l’equazione abbiamo bisogno delle condizioni di esistenza~\(\CE 
% a\neq~0\) et \(a\neq \frac{1}{2}\).
% 
% Procediamo nella risoluzione, riduciamo allo stesso denominatore ambo i 
% membri dell’equazione:
% \(\frac{a\cdot (x+a)+1}{a\cdot (2a-1)}=\frac{x\cdot (2a-1)}{a\cdot (2a-1)}\),
% applichiamo il secondo principio moltiplicando ambo i membri per il~\(\mcm\), 
% otteniamo:~\(ax+a^{2}+1=2ax-x\)
% che in forma canonica è
% \begin{equation*}
%  x\cdot (a-1)=a^{2}+1.
% \end{equation*}
% Il coefficiente dell’incognita dipende dal valore assegnato al parametro; 
% procediamo quindi alla discussione:
% \begin{itemize*}
%  \item se~\(a-1\neq~0\) cioè~\(a\neq~1\) possiamo applicare il secondo principio 
% e dividere ambo i membri per il coefficiente
%     \(a-1\) ottenendo~\(x=\frac{a^{2}+1}{a-1}\). L’equazione è determinata:
%     \[\IS=\left\{\frac{a^{2}+1}{a-1}\right\};\]
%  \item se~\(a-1=0\) cioè~\(a=1\) l’equazione diventa~\(0\cdot x=2\). L’equazione è 
% impossibile:~\(\IS=\emptyset\).
% \end{itemize*}
% 
% Riassumendo si ha:
% \begin{center}
% \begin{tabular}{lll}
% \toprule
% \multicolumn{3}{c} {\(\frac{x+a}{2a-1}-\frac{1}{a-2a^{2}}=\frac{x}{a}\) 
% con~\(a\in \insR\)}\vspace{1.05ex}\\
% Condizioni sul parametro & Insieme Soluzione & Equazione\\
% \midrule
% \(a=0\vee a=\frac{1}{2}\) & & priva di significato\\
% \(a=1\) & \(\IS=\emptyset\) & impossibile \\
% \(a\neq~0\wedge a\neq \frac{1}{2}\wedge a\neq~1\) & 
% \(\IS=\left\{\frac{a^{2}+1}{a-1}\right\}\) & determinata \\
% \bottomrule
% \end{tabular}
% \end{center}
%  \end{esempio}
% 
%  \begin{esempio}
% Risolvere e discutere~\(\frac{a-x}{a-2}+\frac{2ax}{a^{2}-4}-\frac{2-x}{a+2}=0\) 
% con~\(a\in \insR\).
% 
% Scomponendo i denominatori troviamo il~\(\mcm=a^2-4\).
% Pertanto se~\(a=2\) o~\(a=-2\) il denominatore si annulla e quindi l’equazione è 
% priva di significato.
% Per poter procedere nella risoluzione poni le~\(\CE a\neq -2\wedge a\neq~2\).
% 
% Riduci allo stesso 
% denominatore:~\(\frac{(a-x)(a+2)+2ax-(2-x)(a-2)}{(a+2)(a-2)}=0\).
% 
% Applica il secondo principio per eliminare il denominatore e svolgi i 
% calcoli. Arrivi alla forma canonica che è
%  \(2\cdot (a-2)\cdot x=a^{2}+4\).
% 
% Per le~\(\CE\) sul parametro il coefficiente dell’incognita è sempre diverso da 
% zero, pertanto puoi dividere per~\(2(a-2)\) e ottieni
% \(x=\frac{a^{2}+4}{2(a-2)}\).
% 
% Riassumendo si ha:
% \begin{center}
% \begin{tabular}{lll}
% \toprule
% \multicolumn{3}{c} {\(\frac{a-x}{a-2}+\frac{2ax}{a^{2}-4}-\frac{2-x}{a+2}=0\) 
% con~\(a\in \insR\)}\vspace{1.05ex}\\
% Condizioni sul parametro & Insieme Soluzione & Equazione\\
% \midrule
% \(a=-2\vee a=+2\) & & priva di significato\\
% \(a\neq -2\wedge a\neq +2\) & \(\IS=\left\{\frac{a^{2}+4}{2(a-2)}\right\}\) & 
% determinata \\
% \bottomrule
% \end{tabular}
% \end{center}
%  \end{esempio}
% % \end{exrig}
% 
% \subsection{Equazioni letterali frazionarie}
% \subsubsection{Caso in cui il denominatore contiene solo l’incognita}
% % \begin{exrig}
%  \begin{esempio}
% Risolvere e discutere~\(\frac{x+4a}{3x}=a-\frac{2x+2a}{6x}\) con~\(a\in \insR\).
% 
% Questa equazione è frazionaria o fratta perché nel denominatore compare 
% l’incognita.
% Sappiamo che risolvere un’equazione significa determinare i valori che 
% sostituiti all’incognita rendono vera
% l’uguaglianza tra il primo e il secondo membro. Non sappiamo determinare tale 
% valore solamente analizzando l’equazione,
% ma certamente possiamo dire che non dovrà essere~\(x = 0\) perché tale valore, 
% annullando i denominatori, rende privi di
% significato entrambi i membri dell’equazione.
% 
% Poniamo allora una condizione sull’incognita: la soluzione è accettabile 
% se~\(x\neq~0\).
% Non abbiamo invece nessuna condizione sul parametro.
% 
% Procediamo quindi con la riduzione allo stesso denominatore di ambo i membri 
% dell’equazione
% \(\frac{2x+8a}{6x}=\frac{6ax-2x-2a}{6x}\); eliminiamo il denominatore che per 
% la condizione posta è diverso da zero.
% Eseguiamo i calcoli al numeratore e otteniamo~\(4x-6ax=-10a\) da cui la forma 
% canonica:
% \begin{equation*}
%  x\cdot (3a-2)=5a.
% \end{equation*}
% Il coefficiente dell’incognita contiene il parametro quindi procediamo alla 
% discussione:
% \begin{enumeratea}
%  \item se~\(3a-2\neq~0\) cioè~\(a\neq \frac{2}{3}\) possiamo applicare il secondo 
% principio e dividere ambo i membri per il coefficiente
%       \(3a-2\) ottenendo~\(x=\frac{5a}{3a-2}\). L’equazione è 
% determinata:~\(\IS=\left\{\frac{5a}{3a-2}\right\}\).
%       A questo punto dobbiamo ricordare la condizione sull'incognita, 
% cioè~\(x\neq~0\),
%       quindi la soluzione è accettabile se~\(x=\frac{5a}{3a-2}\neq~0 
% \Rightarrow a\neq~0\);
%  \item se~\(3a-2=0\) cioè~\(a=\frac{2}{3}\) l’equazione diventa~\(0\cdot 
% x=\frac{10}{3}\), cioè l’equazione è impossibile:~\(\IS=\emptyset\).
% \end{enumeratea}
% Riassumendo si ha la tabella:
% \begin{center}
% \begin{tabular}{llll}
% \toprule
% \multicolumn{4}{c} {\(\frac{x+4a}{3x}=a-\frac{2x+2a}{6x}\) con~\(a\in 
% \insR\)}\vspace{1.05ex}\\
% \multicolumn{2}{c}{Condizioni} & &\\
% parametro & incognita & Insieme Soluzione & Equazione\\
% \midrule
%  &\(x\neq0\) & & \\
% \(a=\frac{2}{3}\) & & \(\IS=\emptyset\) & impossibile \\
% \(a\neq\frac{2}{3}\) & & \(\IS=\left\{\frac{5a}{3a-2}\right\}\) & determinata \\
% \(a\neq \frac{2}{3}\wedge a\neq0\) & accettabile &\(x=\frac{5a}{3a-2}\) & \\
% \bottomrule
% \end{tabular}
% \end{center}
%  \end{esempio}
% % \end{exrig}
% 
% \subsubsection{Caso in cui il denominatore contiene sia il parametro che 
% l’incognita}
% % \begin{exrig}
%  \begin{esempio}
% Risolvere e 
% discutere~\(\frac{2x+b}{x}+\frac{2x+1}{b-1}=\frac{2x^{2}+b^{2}+1}{bx-x}\) 
% con~\(b\in \insR\).
% \end{esempio}
% L’equazione è fratta; il suo denominatore contiene sia l’incognita che il 
% parametro.
% Scomponiamo in fattori i denominatori
% \[\frac{2x+b}{x}+\frac{2x+1}{b-1}=\frac{2x^{2}+b^{2}+1}{x\cdot (b-1)}.\]
% 
% Determiniamo le condizioni di esistenza che coinvolgono il parametro~\(\CE 
% b\neq~1\) e
% le condizioni sull’incognita: soluzione accettabile se~\(x\neq~0\).
% 
% Riduciamo allo stesso denominatore ed eliminiamolo in quanto per le 
% condizioni poste è diverso da zero.
% L'equazione canonica è~\(x\cdot (2b-1)=b+1.\)
% 
% Il coefficiente dell’incognita contiene il parametro quindi occorre fare la 
% discussione:
% 
% \begin{enumeratea}
%  \item se~\(2b-1\neq~0\) cioè~\(b\neq \frac{1}{2}\) possiamo dividere ambo i 
% membri per~\(2b-1\), otteniamo:
%     \(x=\frac{b+1}{2b-1}\). L’equazione è determinata, l'insieme delle 
% soluzioni è~\(\IS=\left\{\frac{b+1}{2b-1}\right\}\);
%     la soluzione è accettabile se verifica la condizione di 
% esistenza~\(x\neq~0\) da cui si ha
%     \[x=\frac{b+1}{2b-1}\neq~0 \Rightarrow b\neq -1,\]
%     cioè se~\(b=-1\) l'equazione ha una soluzione che non è accettabile, 
% pertanto è impossibile;
%  \item se~\(2b-1=0\) cioè~\(b=\frac{1}{2}\) l’equazione diventa~\(0\cdot 
% x=\frac{3}{2}\). L'equazione è impossibile, l'insieme delle soluzioni è vuoto:
%     \(\IS=\emptyset\).
% \end{enumeratea}
% La tabella che segue riassume tutti i casi:
% \begin{center}
% \begin{tabular}{llll}
% \toprule
% \multicolumn{4}{c} 
% {\(\frac{2x+b}{x}+\frac{2x+1}{b-1}=\frac{2x^{2}+b^{2}+1}{bx-x}\) con~\(b\in 
% \insR\)}\vspace{1.05ex}\\
% \multicolumn{2}{c}{Condizioni} & &\\
% parametro & incognita & Insieme Soluzione & Equazione\\
% \midrule
% \(b=1\) & & & priva di significato\\
% \(b\neq1\) &\(x\neq0\) & & \\
% \(b=\frac{1}{2}\vee b=-1\) & & \(\IS=\emptyset\) & impossibile \\
% \(b\neq~1\wedge b\neq \frac{1}{2}\) & & \(\IS=\left\{\frac{b+1}{2b-1}\right\}\) & 
% determinata \\
% \(b\neq~1\wedge b\neq \frac{1}{2}\wedge b\neq -1\) & accettabile 
% &\(x=\frac{b+1}{2b-1}\) & \\
% \bottomrule
% \end{tabular}
% \end{center}
% 
% % \end{exrig}
% 
% \ovalbox{\risolvii \ref{ese:20.44}, \ref{ese:20.45}, \ref{ese:20.46}, 
% \ref{ese:20.47}, \ref{ese:20.48}, \ref{ese:20.49}, \ref{ese:20.50}, 
% \ref{ese:20.51}}

\section{Equazioni letterali e formule inverse}
\label{sec:compl1_formuleinverse}

Le formule di geometria, di matematica finanziaria e di fisica possono essere 
viste come equazioni letterali.
I due principi di equivalenza delle equazioni permettono di ricavare le 
cosiddette formule inverse, ossia di risolvere
un'equazione letterale rispetto a una delle qualsiasi lettere incognite che vi 
compaiono.

% \begin{exrig}
 \begin{esempio}
Area del triangolo~\(A=\frac{b\cdot h}{2}\).

Questa equazione è stata risolta rispetto all'incognita~\(A\), ossia se sono 
note le misure della base~\(b\) e dell'altezza~\(h\)
è possibile ottenere il valore dell'area~\(A\).

È possibile risolvere l'equazione rispetto a un'altra lettera pensata come 
incognita.
Note le misure di~\(A\) e di~\(b\) ricaviamo~\(h\). Per il primo principio di 
equivalenza moltiplichiamo per~\(2\)
entrambi i membri dell'equazione
\[A=\frac{b\cdot h}{2}\Rightarrow~2A=b\cdot h,\]
dividiamo per~\(b\) entrambi i membri~\(\frac{2A}{b}=h\).
Ora basta invertire primo e secondo membro: \[h=\frac{2A}{b}.\]
 \end{esempio}

 \begin{esempio}
Formula del montante~\(M=C(1+it)\).

Depositando un capitale~\(C\) per un periodo di tempo~\(t\) in anni, a un tasso di 
interesse annuo~\(i\),
si ha diritto al montante~\(M\).

Risolviamo l'equazione rispetto al tasso di interesse~\(i\), ossia supponiamo di 
conoscere il capitale depositato~\(C\), il montante~\(M\)
ricevuto alla fine del periodo~\(t\) e ricaviamo il tasso di interesse che ci è 
stato applicato.
Partendo da~\(M=C(1+it)\), dividiamo primo e secondo membro per~\(C\), otteniamo 
\[\frac{M}{C}=1+it;\]
sottraiamo~\(1\) al primo e al secondo membro, otteniamo
\[\frac{M}{C}-1=it;\] dividiamo primo e secondo membro per~\(t\),
otteniamo
\[i=\frac{\left(\frac{M}{C}-1\right)}{t}\Rightarrow%
i=\frac{1}{t}\cdot \left(\frac{M}{C}-1\right)\Rightarrow 
i=\frac{M-C}{t\cdot C}.\]
 \end{esempio}

 \begin{esempio}
Formula del moto rettilineo uniforme~\(s=s_{0}+v\cdot t\).

Un corpo in una posizione~\(s_0\), viaggiando alla velocità costante~\(v\), 
raggiunge dopo un intervallo di tempo~\(t\) la posizione~\(s\).

Calcoliamo~\(v\) supponendo note le altre misure.
Partendo dalla formula~\(s=s_{0}+v\cdot t\) 
sottraiamo ad ambo i membri~\(s_0\), 
otteniamo~\(s-s_{0}=v\cdot t\)
dividiamo primo e secondo membro per~\(t\), otteniamo 
\[\frac{s-s_{0}}{t}=v.\]
 \end{esempio}

% \end{exrig}

% \ovalbox{\risolvii \ref{ese:20.53}, \ref{ese:20.54}, \ref{ese:20.55}, 
% \ref{ese:20.56}, \ref{ese:20.57}, \ref{ese:20.58}, \ref{ese:20.59}, 
% \ref{ese:20.60}, \ref{ese:20.61}}
% 
% \vspazio\ovalbox{\ref{ese:20.62}, \ref{ese:20.63}, \ref{ese:20.64}, 
% \ref{ese:20.65}}

% 
% \section{Sistemi fratti}
% \label{sec:compl1_sistemifratti}
% 
% Osservando il sistema
% 
% \[\left\{\begin{array}{l}\frac{2}{x+1}-\frac{3}{y-2}=
% \frac{2x-5y+4}{xy+y-2-2x}\\3y+2(x-y-1)=5x-8(-x-2y+1)\end{array}\right.\]
% 
% possiamo notare che la prima equazione è fratta.
% 
% \begin{definizione}
% Si chiama \emph{sistema fratto o frazionario} il sistema in cui almeno in una 
% delle equazioni che lo compongono compare l'incognita al denominatore.
% \end{definizione}
% 
% Poiché risolvere un sistema significa determinare tutte le coppie
% ordinate che verificano entrambe le equazioni, nel sistema fratto
% dovremo innanzi tutto definire il Dominio o Insieme di Definizione nel
% quale individuare le coppie soluzioni.
% 
% \begin{definizione}
% Si chiama \emph{Dominio}~(\(D\)) o \emph{Insieme di Definizione}~(\(ID\)) del 
% sistema fratto, l'insieme delle coppie ordinate che rendono diverso da zero i 
% denominatori che compaiono nelle equazioni.
% \end{definizione}
% 
% % \begin{exrig}
%  \begin{esempio}
% \(\longarray\left\{\begin{array}{l}\dfrac{2}{x+1}-\dfrac{3}{y-2}=
% \dfrac{2x-5y+4}{xy+y-2-2x}\\3y+2(x-y-1)=5x-8(-x-2y+1)\end{array}\right..\)
% 
% \paragraph{Passo I} Scomponiamo i denominatori nella prima equazione
% per determinare il~\(\mcm\).
% \[\longarray\left\{\begin{array}{l}{\dfrac{2}{x+1}-\dfrac{3}{y-2}=
% \dfrac{2x-5y+4}{(x+1)(y-2)}}\\{3y+2(x-y-1)=5x-8(-x-2y+1)}\end{array}\right.
% \Rightarrow\mcm=(x+1)(y-2).\]
% 
% \paragraph{Passo II} Poniamo le Condizioni di Esistenza da cui determineremo 
% il Dominio del sistema:
% \[\CE:\left\{\begin{array}{l}
%    x\neq -1\\y\neq~2
%    \end{array}\right.
%   \Rightarrow D=\IS=\left\{(x;y)\in \insR\times\insR\left|x\right.
%   \neq -1\text{ e }y\neq~2\right\}\]
% 
% \paragraph{Passo III} Riduciamo allo stesso denominatore la prima
% equazione, svolgiamo i calcoli nella seconda per ottenere la forma
% canonica:~\(\left\{\begin{array}{l}{-5x+7y=11}\\{11x+15y=6}\end{array}\right.\)
% 
% \paragraph{Passo IV} Risolviamo il sistema e otteniamo la coppia
% soluzione~\(\left(-{\frac{123}{152};\frac{151}{152}}\right)\) che è
% accettabile.
%  \end{esempio}
% 
%   \begin{esempio}
% 
% \(\longarray\left\{\begin{array}{l}{\dfrac{3x+y-1}{x}=3}\\
% {\dfrac{2x+3y}{y-1}=7}% \end{array}\right.\)
% 
% \paragraph{Passo I} Per la prima equazione si ha~\(\mcm=x\) per la 
% seconda~\(\mcm=y-1\).
% 
% \paragraph{Passo II} Poniamo le Condizioni di Esistenza da cui determineremo 
% il Dominio:
% \[\CE:\left\{\begin{array}{l}
%    x\neq~0\\y\neq~1
%    \end{array}\right.\rightarrow D=\IS=\left\{(x;y)\in 
%    \insR\times\insR |x\neq~0\text{ e }y\neq~1\right\}\]
% 
% \paragraph{Passo III} Riduciamo allo stesso denominatore sia la prima che la 
% seconda equazione:
% \(\left\{\begin{array}{l}{3x+y-1=3x}\\{2x+3y=7y-7}\end{array}\right..\)
% 
% \paragraph{Passo IV} Determiniamo la forma canonica:
% \(\left\{\begin{array}{l}{y-1=0}\\{2x-4y=-7}\end{array}\right..\)
% 
% \paragraph{Passo V} Determiniamo con un qualunque metodo la coppia
% soluzione:~\(\left(-{\frac{3}{2};1}\right)\) che non accettabile
% poiché contraddice la~\(\CE\) e quindi non appartiene al dominio. Il
% sistema assegnato è quindi impossibile~\(\IS=\emptyset \).
%  \end{esempio}
% % \end{exrig}

% \ovalbox{\risolvii \ref{ese:22.44}, \ref{ese:22.45}, \ref{ese:22.46}, 
% \ref{ese:22.47}, \ref{ese:22.48}, \ref{ese:22.49}}

% \section{Sistemi letterali}
% \label{sec:compl1_sistemiletterali}
% 
%  \begin{definizione}
%  Si chiama \emph{sistema letterale} il sistema in cui
% oltre alle incognite, solitamente indicate con~\(x\) e~\(y\), compaiono altre
% lettere dette parametri.
%  \end{definizione}
% 
% Noi affronteremo solo il caso in cui il parametro appaia solo al numeratore.
% % Distinguiamo tre casi distinti di discussione.
% 
% \subsection*{Equazioni lineari con parametro solo al numeratore}
% 
% % \begin{exrig}\vspace{1.10ex}
%  \begin{esempio}
%  \(\left\{\begin{array}{l}{2ax-(a-1)y=0}\\{-2x+3y=a}\end{array}\right..\)
% 
% 
% È un sistema letterale in quanto, reso in forma
% canonica, presenta un parametro nei suoi coefficienti. Esso è
% lineare, pertanto la coppia soluzione, se esiste, dipenderà dal
% valore del parametro.
% 
% Per \emph{discussione del sistema letterale} s'intende
% l'analisi e la ricerca dei valori che attribuiti al
% parametro rendono il sistema determinato (in tal caso si determina la
% soluzione) ma anche scartare i valori del parametro per cui il sistema
% è impossibile o indeterminato.
% Per discutere il sistema usiamo il metodo di Cramer.
% 
% \paragraph{Passo I} Calcoliamo il determinante del sistema:
% \[D=\left|\begin{array}{cc}{2a}&{-(a-1)}\\{-2}&{3}\end{array}\right|=4a+2.\]
% 
% \paragraph{Passo II} Determiniamo il valore del parametro che
% rende~\(D\) diverso da zero:~\(4a+2\neq~0\Rightarrow a\neq~0-\frac{1}{2}\). 
% Se~\(a\neq -{\frac{1}{2}}\) il sistema è determinato.
% 
% \paragraph{Passo III} Calcoliamo i determinanti~\(D_{x}\)
% e~\(D_{y}\) per trovare la coppia soluzione.
% \[D_{x}=\left|\begin{array}{cc}{0}&{-(a-1)}\\{a}&{3}\end{array}\right|=
% a\cdot (a-1);\quad
% D_{y}=\left|\begin{array}{cc}{2a}&{0}\\{-2}&{a}\end{array}\right|=2a^{2}.\]
% Quindi~\(x=\frac{a\cdot (a-1)}{4a+2}\) e~\(y=\frac{2a^{2}}{4a+2}\).
% 
% \paragraph{Passo IV} Il determinante è nullo se~\(a=-{\frac{1}{2}}\) poiché per 
% questo valore di~\(a\) i determinanti~\(D_{x}\) e~\(D_{y}\) sono diversi da zero si 
% ha che per~\(a=-{\frac{1}{2}}\) il sistema
% è impossibile.
% 
% Riassumendo si ha:
% \begin{center}
%  \begin{tabular}{lll}
% \toprule
% Condizioni sul parametro & Insieme Soluzione & Sistema\\
% \midrule
% \(a\neq -{\frac{1}{2}}\) & \(\left(\frac{a\cdot (a-1)}{4a+2};
% \frac{2a^{2}}{4a+2}\right)\) & determinato\\
% \(a=-{\frac{1}{2}}\) & \(\emptyset \) & impossibile\\
% \bottomrule
% \end{tabular}
% \end{center}
%  \end{esempio}
% % \end{exrig}

% \subsection*{Parametro al denominatore in almeno una equazione}
% 
% % \begin{exrig}\vspace{1.10ex}
% \begin{esempio}
%  \(\longarray\left\{\begin{array}{l}{\dfrac{y+a}{3}-\dfrac{a-x}{a-1}=a}\\
%  {\dfrac{x+2a}{a}-3=\dfrac{y}{2}-a}\end{array}\right..\)
% 
% Il sistema non è fratto pur presentando termini frazionari nelle sue
% equazioni; la presenza del parametro al denominatore ci obbliga ad
% escludere dall'insieme~\(\insR\) quei valori che annullano il
% denominatore.
% Se~\(a=1\) oppure~\(a=0\) ciascuna equazione del sistema è priva di
% significato, pertanto lo è anche il sistema.
% Con le condizioni di esistenza~\(\CE: a\neq~1\) e~\(a\neq~0\)
% possiamo ridurre allo stesso denominatore ciascuna equazione e condurre
% il sistema alla forma
% canonica:~\(\left\{\begin{array}{l}{3x+(a-1)y=2a^{2}+a}\\
% {2x-ay=2a-2a^{2}}\end{array}\right.\)
% 
% 
% \paragraph{Passo I} Calcoliamo il determinante del sistema:
% \(D=\left|\begin{array}{cc}{3}&{a-1}\\{2}&{-a}\end{array}\right|=2-5a.\)
% 
% \paragraph{Passo II} Determiniamo il valore del parametro che
% rende~\(D\) diverso da zero:~\(2-5a\neq~0\Rightarrow a\neq \frac{2}{5}\)
% Se~\(a\neq \frac{2}{5}\) il sistema è determinato.
% 
% \paragraph{Passo III} Calcoliamo i determinanti~\(D_{x}\)
% e~\(D_{y}\) per trovare la coppia soluzione
% 
% \[D_{x}=\left|\begin{array}{cc}{2a^{2}+a}&{a-1}\\
% {2a-2a^{2}}&{-a}\end{array}\right|=a\cdot (2a-5);\quad
% D_{y}=\left|\begin{array}{cc}{3}&{2a^{2}+a}\\
% {2}&{2a-2a^{2}}\end{array}\right|=2a\cdot (2-5a).\]
% Quindi~\(x=\frac{a\cdot (2-5a)}{2-5a}\) 
% e~\(y=\frac{2a\cdot (2-5a)}{2-5a}\) e semplificando~\((a;2a)\).
% 
% \paragraph{Passo IV} Il determinante è nullo se
% \(a=\frac{2}{5}\) poiché anche i determinanti~\(D_{x}\) e~\(D_{y}\) si annullano si 
% ha per~\(a=\frac{2}{5}\) sistema indeterminato.
% 
% Riassumendo si ha:
% 
% \begin{center}
% \begin{tabular}{lll}
% \toprule
% Condizioni sul parametro & Insieme Soluzione & Sistema\\
% \midrule
% \(a=0\vee a=1\) & \(\emptyset \) & privo di significato\\
% \(a\neq \frac{2}{5}\) e~\(a\neq~1\) e~\(a\neq~0\) & \(\left\{(a;2a)\right\}\) & 
% determinato\\
% \(a=\frac{2}{5}\) & \(\{\forall(x;y)\in\insR^2/3x-\frac{3}{5}y=\frac{18}{25}\}\) 
% & 
% indeterminato\\
% \bottomrule
% \end{tabular}
% \end{center}
% \end{esempio}
% % \end{exrig}

% \subsection*{Il sistema è frazionario}
% 
% % \begin{exrig}
%  \vspace{1.10ex}
%  \begin{esempio}
% \(\left\{\begin{array}{l}\frac{y-a}{x}=\frac{2}{a}\\{x+y=1}\end{array}\right..\)
% 
% Il sistema letterale è fratto e nel denominatore oltre al parametro
% compare l'incognita~\(x\). Se~\(a=0\) la prima equazione, e di conseguenza tutto 
% il sistema, è privo di significato. Per poter procedere alla ricerca
% dell'Insieme Soluzione poniamo sul
% parametro la condizione di esistenza:
% \begin{equation}
% \label{eq:23.1}
% \CE: a\neq~0.
% \end{equation}
% 
% Essendo fratto dobbiamo anche stabilire il Dominio del sistema:
% \begin{equation}
%  \label{eq:23.2}
% D=\{(x;y)\in \insR\times \insR | x\neq~0\}.
%  \end{equation}
% 
% 
% \paragraph{Passo I} Portiamo nella forma canonica:
% \(\left\{\begin{array}{l}-2x+ay=a^{2}\\
% x+y=1\end{array}\right.\text{ con }a\neq~0\text{ e }x\neq~0\).
% \paragraph{Passo II} Calcoliamo il determinante del sistema:
% \(D=\left|\begin{array}{cc}{-2}&{a}\\{1}&{1}\end{array}\right|=-2-a=-(2+a)\).
% 
% \paragraph{Passo III} Determiniamo il valore del parametro che
% rende~\(D\) diverso da zero:~\(-2-a\neq~0\Rightarrow a\neq -2\).
% Se~\(a\neq -2\) il sistema è determinato.
% \paragraph{Passo IV} calcoliamo i determinanti~\(D_{x}\)
% e~\(D_{y}\) per trovare la coppia soluzione
% 
% \[D_{x}=\left|\begin{array}{cc}a^{2}&{a}\\
% {1}&{1}\end{array}\right|=a\cdot (a-1);\quad
% D_{y}=\left|\begin{array}{cc}-2&a^{2}\\
% 1& 1\end{array}\right|=-2-a^{2}=-(2+a^{2}).\]
% Quindi~\(x=-{\frac{a\cdot (a-1)}{2+a}}\) e~\(y=\frac{a^{2}+2}{2+a}\) è la coppia 
% soluzione accettabile se~\(x=-{\frac{a\cdot (a-1)}{2+a}}\neq~0\) 
% per quanto stabilito in~\ref{eq:23.2}; essendo~\(a\neq~0\)
% per la~\ref{eq:23.1} la coppia soluzione è accettabile se~\(a\neq~1\).
% 
% \paragraph{Passo V} il determinante~\(D\) è nullo se~\(a=-2\)
% essendo i determinanti~\(D_{x}\) e~\(D_{y}\) diversi
% da zero si ha:
% se~\(a=-2\) il sistema è impossibile.
% Riassumendo si ha:
% 
% \begin{center}
% \begin{tabularx}{.9\textwidth}{XXll}
% \toprule
% Parametro & Incognite & Insieme Soluzione & Sistema\\
% \midrule
%  & \(x\neq~0\) & & \\
%  \(a=0\) & & & privo di significato\\
%  \(a \neq2,a\neq0\) & & \(\left(-{\frac{a\cdot (a-1)}{2+a}};
%  \frac{a^{2}+2}{2+a}\right)\) & determinato\\
%  \(a\neq -2\) e~\(a\neq~0\) e~\(a\neq~1\) & & accettabile & \\
% \(a=-2\) & & & impossibile\\
% \bottomrule
% \end{tabularx}
% \end{center}
%  \end{esempio}
% % \end{exrig}

% \ovalbox{\risolvii \ref{ese:22.50}, \ref{ese:22.51}, \ref{ese:22.52}, 
% \ref{ese:22.53}, \ref{ese:22.54}, \ref{ese:22.55}, \ref{ese:22.56}}
% 
% \section{Sistemi lineari di tre equazioni in tre incognite}
% \label{sec:compl1_sistemitreeq}
% 
% % \begin{problema}
% % Determinare tre numeri reali~\(x, y, z\) (nell'ordine) tali
% % che il doppio del primo uguagli l'opposto del secondo,
% % la differenza tra il primo e il triplo del terzo sia nulla e la somma
% % del secondo con il terzo superi il primo di~4 unità.
% % \end{problema}
% 
% \begin{problema}
% Determinare tre numeri reali~\(x, y, z\) (nell'ordine) tali
% che il doppio del primo uguagli l'opposto del secondo,
% il triplo del terzo sia uguale al primo aumentato~\(4\)
% e che la somma del secondo con il terzo sia inferiore al primo di~\(12\) unità.
% \end{problema}
% 
% \begin{soluzione}
% Formalizziamo le condizioni espresse nel testo attraverso equazioni
% lineari:
% 
% 
% \begin{enumeratea}
% \item il doppio del primo uguagli l'opposto del secondo:~\(2x=-y\)
% \item il triplo del terzo sia uguale al primo aumentato~\(4\):~\(3z=x+4\)
% \item la somma del secondo con il terzo sia inferiore al primo di~\(12\) 
% unità:~\(y+z=x-12\).
% \end{enumeratea}
% 
% Le tre condizioni devono essere vere contemporaneamente, quindi i tre
% numeri sono la terna soluzione del sistema di primo grado di tre equazioni in 
% tre incognite:
% \[\left\{\begin{array}{l}
%   2x=-y\\
%   3z=x+4\\
%   y+z=x-12
% \end{array}\right.\]
% 
% Per prima cosa scriviamo il sistema in forma normale:
% \[\left\{\begin{array}{l}
%   2x+y=0\\
%   -x+3z=4\\
%   -x+y+z=-12
% \end{array}\right.\]
% 
% Possiamo ora ricavare la~\(y\) dalla prima equazione e sostituirla nelle altre 
% due:
% \[\left\{\begin{array}{l}
%   y=-2x\\
%   -x+3z=4\\
%   -x -2x+z=-12
% \end{array}\right.
% \Rightarrow
% \left\{\begin{array}{l}
%   y=-2x\\
%   -x+3z=+4\\
%   -3x+z=-12
% \end{array}\right.\]
% 
% In questo modo abbiamo ottenuto un sottosistema formato da due equazioni in 
% due incognite:
% \[\left\{\begin{array}{l}
%   -x+3z=+4\\
%   -3x+z=-12
% \end{array}\right.\]
% 
% Possiamo risolverlo facilmente con il metodo di riduzione:
% 
% \[\left\{\begin{array}{l}
%   3x-9z=-12\\
%   -3x+z=-12
% \end{array}\right.
% \Rightarrow -8z=-24 \Rightarrow z=3
% \]
% 
% \[\left\{\begin{array}{l}
%   -x+3z=+4\\
%   9x-3z=+36
% \end{array}\right.
% \Rightarrow 8x=40 \Rightarrow x=5
% \]
% 
% Risolto così il sottosistema possiamo risolvere il sistema di partenza:
% \[\left\{\begin{array}{l}
%   x=5\\
%   y=-2x=-10\\
%   z=3
% \end{array}\right.\]
% 
% \end{soluzione}
% 
% % \begin{exrig}
%  \begin{esempio}
% \(\left\{\begin{array}{l}3x+y-z=7\\x+3y+z=5\\x+y-3z=3\end{array}\right.\)
% 
% Procediamo con il metodo di riduzione. Sommiamo le prime due 
% equazioni:~\(4x+4y=12\)
% Moltiplichiamo la seconda equazione per~3 e sommiamo con la 
% terza:~\(3(x+3y+z)+x+y=3\cdot 5+3=4x+10y=18\).
% Costruiamo il sistema di queste due equazioni
% nelle sole due incognite~\(x\) e~\(y\):
% \(\left\{\begin{array}{l}4x+4y=12\\4x+10y=18\end{array}\right..\)
% 
% Moltiplichiamo la seconda equazione per~\(-1\) e sommiamo le due equazioni:
% \begin{align*}
% \left\{\begin{array}{l}4x+4y=12 \\-4x-10y=-18
% \end{array}\right.&\Rightarrow
% \left\{\begin{array}{l}4x+4y=12
% \\-4x-10y+4x+4y=-18+12 \end{array}\right.\\
% &\Rightarrow
% \left\{\begin{array}{l}4x+4y=12 \\-6y=-6\Rightarrow
% y=1 \end{array}\right.\Rightarrow
% \left\{\begin{array}{l}x=2 \\y=1
% \end{array}\right..
% \end{align*}
% 
% Sostituendo nella prima equazione del sistema ricaviamo la terza
% incognita:~\(\left\{\begin{array}{l}x=2\\y=1\\z=0\end{array}\right.\).
% 
% La terna soluzione del sistema assegnato è~\((2;1;0)\).
%  \end{esempio}
% % \end{exrig}
% 
% % \ovalbox{\risolvii \ref{ese:22.57}, \ref{ese:22.58}, \ref{ese:22.59}, 
% % \ref{ese:22.60}, \ref{ese:22.61}, \ref{ese:22.62}, \ref{ese:22.63}}

\section{Sistemi da risolvere con sostituzioni delle variabili}
\label{sec:compl1_sistemisotituzionevariabili}

Alcuni sistemi possono essere ricondotti a sistemi lineari per mezzo di
sostituzioni nelle variabili.

% \begin{exrig}
 \begin{esempio}
\[\longarray\left\{\begin{array}{l}
  \dfrac{1}{x}+\dfrac{2}{y}=3\\
  \dfrac{2}{x}-\dfrac{4}{y}=-1 
\end{array}\right.\]
% \begin{multicolumn}{2}
Con la seguente sostituzione di variabili
\label{eq:compl1.sosti}
\[\longarray\left\{\begin{array}{l}
  u=\dfrac{1}{x}\\
  v=\dfrac{1}{y}
\end{array}\right.
 \text{ il sistema diventa }
\left\{\begin{array}{l}
  u+2v=3 \\
  2u-4v=-1 
\end{array}\right.\]
% \end{multicolumn}
Possiamo risolvere quest'ultimo con il metodo di riduzione, 
per ricavare l'inognita~\(u\):
\[\left\{\begin{array}{l}
2u+4v=6 \\
2u-4v=-1
\end{array}\right. \Rightarrow 4v=5 \Rightarrow u=\frac{5}{4}\]
Per ricavare l'incognita~\(v\):
\[\left\{\begin{array}{l}
-2u-4v=-6 \\
2u-4v=-1 \end{array}\right. \Rightarrow -8v=-7 \Rightarrow v=\frac{7}{8}\]
Avendo trovato i valori delle incognite~\(u\) e~\(v\) possiamo ricavare~\(x\) 
e~\(y\) sostituendo con i valori trovati nella~\ref{eq:compl1.sosti}:
\[\dfrac{1}{x}=\dfrac{5}{4} \sRarrow x=\dfrac{4}{5} \qquad
  \dfrac{1}{y}=\dfrac{7}{8} \sRarrow y=\dfrac{8}{7}\]
 \end{esempio}
% \end{exrig}

% \ovalbox{\risolvii \ref{ese:22.64}, \ref{ese:22.65}, \ref{ese:22.66}}
% 
% \section{Disequazioni polinomiali e fratte}
% \label{sec:compl1_disequazionipolinomialiefratte}
% 
% \subsection{Disequazioni di grado superiore al primo}
% 
% Quando una disequazione contiene un polinomio di grado superiore al primo
% possiamo:
% 
% \begin{procedura}
%  Per risolvere una disequazione polinomiale:
% \begin{enumeratea}
%  \item scomporre in fattori di primo grado il polinomio;
%  \item studiare il segno di ogni fattore;
%  \item rappresentare, con i diversi metodi visti, 
%   gli intervalli che risolvono la disequazione.
% \end{enumeratea}
% \end{procedura}
% 
% \begin{esempio}
%  \[p(x)=-2x^3+3x^2+8x-12 \le 0\]
% 
% In questo caso dobbiamo risolvere una disequazione di terzo grado. 
% Possiamo ridurre la difficoltà scomponendo in fattori il polinomio in modo da 
% ottenere il prodotto di più polinomi di primo grado: 
% 
% \begin{align*}
% P(x) = -2x^3+3x^2+8x-12 &= x^2(-2x+3)-4(-2x+3) = \\
%                         &= (x^2-4)(-2x+3) = \\
%                         &= (x-2)(x+2)(-2x+3) \le 0
% \end{align*}
% 
% \begin{minipage}{.65\textwidth}
% 
% A questo punto possiamo studiare il segno dei polinomi, applicando la regola 
% dei segni di un prodotto trovare il segno del polinomio e risolvere la 
% disequazione.
% 
% \end{minipage}
% \begin{minipage}{.30\textwidth}
% 
% \begin{inaccessibleblock}[Regola dei segni nella moltiplicazione]
% \begin{center}
%   \input{\folder lbr/fig022_tab.pgf}
% \end{center}
% \end{inaccessibleblock}
% 
% \end{minipage}
% 
% Studiamo il segno di ogni singolo fattore:
% \begin{itemize}
%  \item segno di \((x-2)\)
%  \subitem E.A.:~\(x-2=0 \Rightarrow x=2\)
%  \subitem
%   \begin{minipage}{.25\textwidth}
%    F.A.:~\(y=x-2 \rightarrow \)
%   \end{minipage}
%   \begin{minipage}{.30\textwidth}
% \begin{inaccessibleblock}[retta crescente con zero in~2]
%   % (c) 2014 Daniele Zambelli - daniele.zambelli@gmail.com

%%%
% Retta crescente zero in 2
%%%%
 
\begin{tikzpicture}[x=1.5mm, y=1.5mm, smooth]

\input{\folder lbr/lib_rettacre.pgf}
\node [above] {$+2$};

\end{tikzpicture}

% \end{inaccessibleblock}
%   \end{minipage}
%  \item segno di \((x+2)\)
%  \subitem E.A.:~\(x+2=0 \Rightarrow x=\frac{1}{2}\)
%  \subitem
%   \begin{minipage}{.25\textwidth}
%    F.A.:~\(y=x+2 \rightarrow \)
%   \end{minipage}
%   \begin{minipage}{.30\textwidth}
% \begin{inaccessibleblock}[retta crescente con zero in~-2]
%   \input{\folder lbr/fig061_retta21.pgf}
% \end{inaccessibleblock}
%   \end{minipage}
%  \item Segno del denominatore:
%  \subitem E.A.:~\(-2x+3=0 \Rightarrow x=\frac{3}{2}\)
%  \subitem
%   \begin{minipage}{.25\textwidth}
%    F.A.:~\(y=-2x+3 \rightarrow \)
%   \end{minipage}
%   \begin{minipage}{.30\textwidth}
% \begin{inaccessibleblock}[retta decrescente con zero in~3/2]
%   \input{\folder lbr/fig062_retta22.pgf}
% \end{inaccessibleblock}
%   \end{minipage}
%  \item Con la regola dei segni calcolo il segno della frazione 
% \begin{inaccessibleblock}[Grafo con i segni]
%   % (c) 2014 Daniele Zambelli - daniele.zambelli@gmail.com

%%%
% Studio dei segni di un prodotto
%%%%
 
\begin{tikzpicture}[x=2.5mm, y=5mm, smooth]

\coordinate (a_top) at (-5, 1);
\coordinate (a_bottom) at (-5, -3);
\coordinate (b_top) at (0, 1);
\coordinate (b_bottom) at (0, -3);
\coordinate (c_top) at (5, 1);
\coordinate (c_bottom) at (5, -3);

% (c) 2014 Daniele Zambelli - daniele.zambelli@gmail.com

%%%
% Grafo per il calcolo del segno con tre assi
%%%%
 
% (c) 2014 Daniele Zambelli - daniele.zambelli@gmail.com

%%%
% Asse cartesiano x
%%%%

\input{\magdir assiepiani/asse10.pgf}
\node [below] at (10, 0)  {$x$};

\begin{scope}[yshift= -.5cm]
  % (c) 2014 Daniele Zambelli - daniele.zambelli@gmail.com

%%%
% Asse cartesiano x
%%%%

\input{\magdir assiepiani/asse10.pgf}
\node [below] at (10, 0)  {$x$};

  \begin{scope}[yshift= -.5cm]
    % (c) 2014 Daniele Zambelli - daniele.zambelli@gmail.com

%%%
% Asse cartesiano x
%%%%

\input{\magdir assiepiani/asse10.pgf}
\node [below] at (10, 0)  {$x$};

    \begin{scope}[yshift= -.5cm]
      % (c) 2014 Daniele Zambelli - daniele.zambelli@gmail.com

%%%
% Asse cartesiano x
%%%%

\input{\magdir assiepiani/asse10.pgf}
\node [below] at (10, 0)  {$x$};

    \end{scope}
  \end{scope}
\end{scope}

\draw [-] [] (a_top) -- (a_bottom);
\draw [-] [] (b_top) -- (b_bottom);
\draw [-] [] (c_top) -- (c_bottom);

\node [above] at (-5, 1) {$-2$};
\node [above] at (0, 1) {$+\frac{3}{2}$};
\node [above] at (5, 1) {$+2$};

\node [above left] at (-10, 0) {$x-2$};
\node [above] at (-7.5, 0) {$-$};
\node [above] at (-2.5, 0) {$-$};
\node [above] at (2.5, 0) {$-$};
\draw (5, .5) circle (3pt);
\node [above] at (7.5, 0) {$+$};

\node [above left] at (-10, -1) {$x+2$};
\node [above] at (-7.5, -1) {$-$};
\draw (-5, -.5) circle (3pt);
\node [above] at (-2.5, -1) {$+$};
\node [above] at (2.5, -1) {$+$};
\node [above] at (7.5, -1) {$+$};

\node [above left] at (-10, -2) {$-2x+3$};
\node [above] at (-7.5, -2) {$+$};
% \draw (0 -.4, -1.5 -.2) -- (0 +.4, -1.5 +.2) 
%       (0 -.4, -1.5 +.2) -- (0 +.4, -1.5 -.2);
\node [above] at (-2.5, -2) {$+$};
\draw (0, -1.5) circle (3pt);
\node [above] at (2.5, -2) {$-$};
\node [above] at (7.5, -2) {$-$};

\node [above left] at (-10, -3.15) {$P(x)$};
\node [above] at (-7.5, -3) {$+$};
\draw (-5, -2.5) circle (3pt);
\node [above] at (-2.5, -3) {$-$};
\draw (0, -2.5) circle (3pt);
\node [above] at (2.5, -3) {$+$};
\draw (5, -2.5) circle (3pt);
\node [above] at (7.5, -3) {$-$};

\end{tikzpicture}
 
% \end{inaccessibleblock}
%  \item Quindi i valori di~\(x\) che rendono vera la disequazione, cioè i valori
%   che rendono~\(f(x)\) non positivo, sono quelli 
%   che si trovano a sinistra di~\(-2\) oppure che si trovano a destra di~\(+3\). 
%  \subitem 
%   \begin{minipage}{.35\textwidth}
%    rappresentazione grafica: 
%   \end{minipage}
%   \begin{minipage}{.30\textwidth}
% \begin{inaccessibleblock}[Retta con gli intervalli evidenziati]
%    % (c) 2014 Daniele Zambelli - daniele.zambelli@gmail.com

%%%
% Valori esterni all'intervallo -2; 3
%%%%
 
\begin{tikzpicture}[x=1.5mm, y=1.5mm, smooth]

% \clip (-7.5, -5.5) rectangle (10.9, 10.9);

\coordinate (m_i) at (-10, 0);
\coordinate (a) at (-5, 0);
\coordinate (b) at (0, 0);
\coordinate (c) at (5, 0);
\coordinate (p_i) at (10, 0);

% (c) 2014 Daniele Zambelli - daniele.zambelli@gmail.com

%%%
% Asse cartesiano x
%%%%

% (c) 2014 Daniele Zambelli - daniele.zambelli@gmail.com

%%%
% Asse cartesiano generico
%%%%

\draw [-{Stealth[length=2mm, open, round]}] (-10, 0) -- (10, 0);

\node [below] at (10, 0)  {$x$};


\begin{scope}[blue,thick]
\draw [-,decorate,decoration=snake] (a) -- (b);
% \draw[fill=white] (a) circle (2pt) node [above] {$-3$};
\draw[fill] (a) circle (2pt) node [above] {$-2$};
% \draw [-,decorate,decoration=snake] (b) -- (c);
\draw[fill] (b) circle (2pt) node [above] {$+\frac{3}{2}$};
\draw [decorate,decoration=snake] (c) -- (p_i);
\draw[fill] (c) circle (2pt) node [above] {$+2$};
\end{scope}

\end{tikzpicture}

% \end{inaccessibleblock}
%   \end{minipage}
%  \subitem rappresentazione con i 
%    predicati:~\(-2 \le x \le \frac{3}{2} \quad \lor \quad x \ge 2\) 
%  \subitem rappresentazione con le 
%   parentesi:~\(]-2;~-3[ \quad \cup \quad [2;~+\infty]\). 
% \end{itemize}
% 
% \end{esempio}
% 
% 
% \subsection{Soluzione di disequazioni fratte}
% 
% Quando una disequazione contiene la variabile dobbiamo seguire la procedura:
% 
% \begin{procedura}
%  Per risolvere una disequazione fratta:
% \begin{enumeratea}
%  \item spostare tutti i termini a primo membro e sommarli in modo da ottenere 
%  una sola frazione e a secondo membro solo lo zero;
%  \item studiare il segno della frazione;
%  \item rappresentare, con i diversi metodi visti, 
%   gli intervalli che risolvono la disequazione.
% \end{enumeratea}
% \end{procedura}
% 
%  \begin{esempio}
% \[\frac{2}{x-7} \ge \frac{3}{x+4}\]
% \begin{itemize}
%  \item Scrivere la disequazione in forma normale:
%  \[\frac{2}{x-7} - \frac{3}{x+4} \ge 0 \Rightarrow 
%    \frac{2 x +8 -3x +21}{(x-7)(x+4)} \ge 0 \Rightarrow
%    \frac{-x +29}{(x-7)(x+4)} \ge 0\]
%  \item Segno del numeratore:
%  \subitem E.A.:~\(-x +29=0 \Rightarrow x=29\)
%  \subitem
%   \begin{minipage}{.25\textwidth}
%    F.A.:~\(y=-x +29 \rightarrow \)
%   \end{minipage}
%   \begin{minipage}{.30\textwidth}
%   \input{\folder lbr/fig065_retta_n1.pgf}
%   \end{minipage}
%  \item Segno del denominatore 1:
%  \subitem E.A.:~\(x -7=0 \Rightarrow x=+7\)
%  \subitem
%   \begin{minipage}{.25\textwidth}
%    F.A.:~\(y=x -7 \rightarrow \)
%   \end{minipage}
%   \begin{minipage}{.30\textwidth}
%   \input{\folder lbr/fig066_retta_d1.pgf}
%   \end{minipage}
%  \item Segno del denominatore 2:
%  \subitem E.A.:~\(x +4=0 \Rightarrow x=-4\)
%  \subitem
%   \begin{minipage}{.25\textwidth}
%    F.A.:~\(y=x +4 \rightarrow \)
%   \end{minipage}
%   \begin{minipage}{.30\textwidth}
%   \input{\folder lbr/fig067_retta_d2.pgf}
%   \end{minipage}
%  \item Con la regola dei segni calcolo il segno della frazione 
%   % (c) 2014 Daniele Zambelli - daniele.zambelli@gmail.com

%%%
% Studio dei segni di una frazione
%%%%
 
\begin{tikzpicture}[x=2.5mm, y=5mm, smooth]

\coordinate (a_top) at (-5, 1);
\coordinate (a_bottom) at (-5, -3);
\coordinate (b_top) at (0, 1);
\coordinate (b_bottom) at (0, -3);
\coordinate (c_top) at (5, 1);
\coordinate (c_bottom) at (5, -3);

% (c) 2014 Daniele Zambelli - daniele.zambelli@gmail.com

%%%
% Grafo per il calcolo del segno con tre assi
%%%%
 
% (c) 2014 Daniele Zambelli - daniele.zambelli@gmail.com

%%%
% Asse cartesiano x
%%%%

\input{\magdir assiepiani/asse10.pgf}
\node [below] at (10, 0)  {$x$};

\begin{scope}[yshift= -.5cm]
  % (c) 2014 Daniele Zambelli - daniele.zambelli@gmail.com

%%%
% Asse cartesiano x
%%%%

\input{\magdir assiepiani/asse10.pgf}
\node [below] at (10, 0)  {$x$};

  \begin{scope}[yshift= -.5cm]
    % (c) 2014 Daniele Zambelli - daniele.zambelli@gmail.com

%%%
% Asse cartesiano x
%%%%

\input{\magdir assiepiani/asse10.pgf}
\node [below] at (10, 0)  {$x$};

    \begin{scope}[yshift= -.5cm]
      % (c) 2014 Daniele Zambelli - daniele.zambelli@gmail.com

%%%
% Asse cartesiano x
%%%%

\input{\magdir assiepiani/asse10.pgf}
\node [below] at (10, 0)  {$x$};

    \end{scope}
  \end{scope}
\end{scope}

\draw [-] [] (a_top) -- (a_bottom);
\draw [-] [] (b_top) -- (b_bottom);
\draw [-] [] (c_top) -- (c_bottom);

\node [above] at (-5, 1) {$-4$};
\node [above] at (0, 1) {$+7$};
\node [above] at (5, 1) {$+29$};

\node [above left] at (-10, 0) {$-x+29$};
\node [above] at (-7.5, 0) {$+$};
\node [above] at (-2.5, 0) {$+$};
\node [above] at (2.5, 0) {$+$};
\draw (5, .5) circle (3pt);
\node [above] at (7.5, 0) {$-$};

\node [above left] at (-10, -1) {$x-7$};
\node [above] at (-7.5, -1) {$-$};
\node [above] at (-2.5, -1) {$-$};
\draw (0 -.4, -0.5 -.2) -- (0 +.4, -0.5 +.2) 
      (0 -.4, -0.5 +.2) -- (0 +.4, -0.5 -.2);
\node [above] at (2.5, -1) {$+$};
\node [above] at (7.5, -1) {$+$};

\node [above left] at (-10, -2) {$x+4$};
\node [above] at (-7.5, -2) {$-$};
\draw (-5 -.4, -1.5 -.2) -- (-5 +.4, -1.5 +.2) 
      (-5 -.4, -1.5 +.2) -- (-5 +.4, -1.5 -.2);
\node [above] at (-2.5, -2) {$+$};
\node [above] at (2.5, -2) {$+$};
\node [above] at (7.5, -2) {$+$};

\node [above left] at (-10, -3.15) {$F(x)$};
\node [above] at (-7.5, -3) {$+$};
\draw (-5 -.4, -2.5 -.2) -- (-5 +.4, -2.5 +.2) 
      (-5 -.4, -2.5 +.2) -- (-5 +.4, -2.5 -.2);
\node [above] at (-2.5, -3) {$-$};
\draw (0 -.4, -2.5 -.2) -- (0 +.4, -2.5 +.2) 
      (0 -.4, -2.5 +.2) -- (0 +.4, -2.5 -.2);
\node [above] at (2.5, -3) {$+$};
\draw (5, -2.5) circle (3pt);
\node [above] at (7.5, -3) {$-$};

\end{tikzpicture}
 
%  \item Quindi i valori di~\(x\) che rendono vera la disequazione, cioè i valori
%   che rendono~\(f(x)\) negativo, sono quelli 
%   che si trovano a sinistra di~\(-2\) oppure che si trovano a destra di~\(+3\). 
%  \subitem 
%   \begin{minipage}{.35\textwidth}
%    rappresentazione grafica: 
%   \end{minipage}
%   \begin{minipage}{.30\textwidth}
%    % (c) 2014 Daniele Zambelli - daniele.zambelli@gmail.com

%%%
% Valori x < -4  or  7 < x <= 29
%%%%
 
\begin{tikzpicture}[x=1.5mm, y=1.5mm, smooth]

% \clip (-7.5, -5.5) rectangle (10.9, 10.9);

\coordinate (m_i) at (-10, 0);
\coordinate (a) at (-5, 0);
\coordinate (b) at (0, 0);
\coordinate (c) at (5, 0);
\coordinate (p_i) at (10, 0);

% (c) 2014 Daniele Zambelli - daniele.zambelli@gmail.com

%%%
% Asse cartesiano x
%%%%

% (c) 2014 Daniele Zambelli - daniele.zambelli@gmail.com

%%%
% Asse cartesiano generico
%%%%

\draw [-{Stealth[length=2mm, open, round]}] (-10, 0) -- (10, 0);

\node [below] at (10, 0)  {$x$};


\begin{scope}[blue,thick]
\draw [-,decorate,decoration=snake] (m_i) -- (a);
% \draw[fill=white] (a) circle (2pt) node [above] {$-3$};
\draw[fill=white] (a) circle (2pt) node [above] {$-4$};
% \draw [-,decorate,decoration=snake] (b) -- (c);
\draw[fill=white] (b) circle (2pt) node [above] {$+7$};
\draw [decorate,decoration=snake] (b) -- (c);
\draw[fill] (c) circle (2pt) node [above] {$+29$};
\end{scope}

\end{tikzpicture}

%   \end{minipage}
%  \subitem rappresentazione con i 
%   predicati:~\(x < -4 \quad \lor \quad +7 < x \le 29\) 
%  \subitem rappresentazione con le 
%   parentesi:~\(]-\infty;~-4[ \quad \cup \quad ]+7;~+\infty[\). 
% \end{itemize}
%  \end{esempio}



% 
% 
% 
% 
% 
% % \begin{exrig}
%  \begin{esempio}
%  \((3x-7)\cdot (2-x)>0.\)
% 
%  La disequazione equivale a determinare i valori che attribuiti alla
% variabile~\(x\) rendono positivo il polinomio~\(p=(3x-7)\cdot (2-x)\).
% 
% Studiamo separatamente il segno dei due fattori:
% 
% \[F_{1}:3x-7>0\Rightarrow x>\frac{7}{3},\quad
% F_{2}: 2-x>0\Rightarrow x<2.\]
% 
% Per risolvere la disequazione iniziale ci è di particolare aiuto un
% grafico che sintetizzi la situazione.
% \begin{center}
%  \input{\folder lbr/fig025_seg.pgf}
% \end{center}
% 
% Applicando poi la regola dei
% segni otteniamo il segno del polinomio~\(p=(3x-7)\cdot (2-x)\).
% 
% Ricordiamo che la disequazione che stiamo risolvendo~\((3x-7)\cdot(2-x)>0\)
% è verificata quando il polinomio~\(p=(3x-7)\cdot (2-x)\) è
% positivo, cioè nell'intervallo in cui abbiamo
% ottenuto il segno ``\(+\)''. Possiamo
% concludere~\(\IS=\left\{x\in\insR/2<x<\frac{7}{3}\right\}\).
%  \end{esempio}
% 
%  \begin{esempio}
% \((x-3)\cdot (2x-9)\cdot (4-5x)>0.\)
% 
% Determiniamo il segno di ciascuno dei suoi tre fattori:
% 
% \[
% F_{1}: x-3>0\Rightarrow x>3;\quad
% F_{2}:2x-9>0\Rightarrow x>\dfrac{9}{2};\quad
% F_{3}:4-5x>0\Rightarrow x<\dfrac{4}{5}.
% \]
% Costruiamo la tabella dei segni:
% \begin{center}
%  \input{\folder lbr/fig026_seg.pgf}
% \end{center}
% La disequazione è verificata negli intervalli dove è presente il
% segno ``\(+\)''.
% \[\IS=\left\{x\in\insR/x<\frac{4}{5}\vee~3<x<\frac{9}{2}\right\}.\]
% \end{esempio}
% 
% \begin{esempio}
% \(4x^{3}+4x^{2}\le~1+x.\)
% La disequazione è di terzo grado; trasportiamo al primo membro tutti i
% monomi:
% \[4x^{3}+4x^{2}-1-x\le~0.\]
% 
% Possiamo risolverla se riusciamo a scomporre in fattori di primo grado
% il polinomio al primo membro:
% \[4x^{3}+4x^{2}-1-x=4x^{2}(x+1)-(x+1)=(x+1)(4x^2-1) \Rightarrow 
% (x+1)(2x-1)(2x+1)\le~0.\]
% 
% Studiamo ora il segno di ciascun fattore, tenendo conto che sono
% richiesti anche i valori che annullano ogni singolo fattore (legge di
% annullamento del prodotto):
% 
% \[ F_{1}:x+1\ge~0\Rightarrow x\ge -1;\quad F_{2}:2x-1\ge~0\Rightarrow 
% x\ge \dfrac{1}{2},\quad F_{3}:2x+1\ge~0\Rightarrow x\ge -{\dfrac{1}{2}}.\]
% Possiamo ora costruire la tabella dei segni.
% Ricordiamo che la disequazione di partenza~\(4x^{3}+4x^{2}\le~1+x\) è
% verificata dove compare il segno~``\(-\)'':
% 
% \begin{center}
% \input{\folder lbr/fig027_seg.pgf}
% \end{center}
% \[\IS=\left\{x\in \insR/x\le-1\text{ oppure }-\frac{1}{2}\le 
% x\le\frac{1}{2}\right\}.\]
% \end{esempio}
% % \end{exrig}
% 
% \begin{procedura}
%  Determinare l'\(\IS\) Di una disequazione polinomiale di grado
% superiore al primo:
% 
% \begin{enumeratea}
%  \item scrivere la disequazione nella forma~\(p\leq0\), \(p\geq~0\),
% \(p<0\), \(p>0\)
% \item scomporre in fattori irriducibili il polinomio;
% \item determinare il segno di ciascun fattore, ponendolo sempre maggiore
% di zero, o maggiore uguale a zero a seconda della richiesta del
% problema;
% \item costruire la tabella dei segni, segnando con un punto ingrossato
% gli zeri del polinomio;
% \item determinare gli intervalli in cui il polinomio assume il segno
% richiesto.
% \end{enumeratea}
% \end{procedura}
% 
% % \ovalbox{\risolvii \ref{ese:21.44}, \ref{ese:21.45}, \ref{ese:21.46}, 
% % \ref{ese:21.47}, \ref{ese:21.48}, \ref{ese:21.49}, \ref{ese:21.50}, 
% % \ref{ese:21.51},
% % \ref{ese:21.52}, \ref{ese:21.53}}
% 
% \section{Disequazioni frazionarie}
% \label{sec:compl1_disequazionifratte}
% 
% Un'espressione contenente operazioni tra frazioni
% algebriche ha come risultato una frazione algebrica. Con la condizione
% di esistenza che il denominatore della frazione sia diverso da zero la
% ricerca del segno di una frazione algebrica viene effettuata con la
% stessa procedura seguita per il prodotto di due o più fattori.
% 
% % \begin{exrig}
%  \begin{esempio}
%  \(p=\dfrac{3x-7}{2-x}\ge~0\).
% 
%  Poniamo innanzi tutto la~\(\CE: 2-x\neq~0\)
%  cioè~\(x\neq~2\) e procediamo studiando il segno del
% numeratore e del denominatore. Terremo conto della~\(\CE\) ponendo il
% denominatore semplicemente maggiore di zero e non maggiore uguale.
% \[N\ge~0\Rightarrow~3x-7\ge~0\Rightarrow x\ge \frac{7}{3},\]
% \[D>0\Rightarrow2-x>0\Rightarrow \ x<2.\]
% \begin{center}
% \input{\folder lbr/fig028_seg.pgf}
% \end{center}
% Analogamente a quanto fatto per il prodotto, dalla tabella dei segni otteniamo
% \[\IS=\left\{x\in \insR/2<x\le \frac{7}{3}\right\}\]
% in cui vediamo già compresa la~\(\CE\) che inizialmente avevamo posto.
%  \end{esempio}
% % \end{exrig}
% 
% \begin{procedura}
%  Procedura per determinare~\(\IS\) di una disequazione frazionaria:
% 
% \begin{enumeratea}
%  \item applicare il primo principio e trasportare tutti i termini al primo 
%   membro;
%  \item eseguire i calcoli dell'espressione al primo membro per arrivare a una 
%   disequazione nella forma:
%  \subitem \(\dfrac{N(x)}{D(x)}>0\) oppure~\(\dfrac{N(x)}{D(x)}\ge~0\) 
%   oppure~\(\dfrac{N(x)}{D(x)}<0\) oppure~\(\dfrac{N(x)}{D(x)}\le~0\)
%  \item studiare il segno del numeratore e del denominatore, ponendo~\(N(x)>0\) 
%   (oppure~\(N(x)\geq~0\) a secondo della richiesta) e~\(D(x)>0\)
%  \item costruire la tabella dei segni, segnando con un punto in grassetto gli 
%   zeri del numeratore;
%  \item determinare gli intervalli in cui il polinomio assume il segno 
%   richiesto.
% \end{enumeratea}
% \end{procedura}
% 
% % \begin{exrig}
%  \begin{esempio}
%  \(\dfrac{x-1}{2x+2}+\dfrac{2x+1}{4x-2}>
%  \dfrac{4x^{2}(2x+1)+1}{8x^{3}+8x^{2}-2x-2}.\)
% 
% Trasportiamo tutti i termini al primo 
% membro~\(\dfrac{x-1}{2x+2}+\dfrac{2x+1}{4x-2}-
%         \dfrac{4x^{2}(2x+1)+1}{8x^{3}+8x^{2}-2x-2}>0\).
% 
% Scomponiamo in fattori i denominatori, determiniamo il minimo comune
% multiplo e sommiamo le frazioni per arrivare alla forma~\(\frac{N(x)}{D(x)}>0\):
% 
% \begin{align}
% &\frac{x-1}{2(x+1)}+\frac{2x+1}{2(2x-1)}-
%  \frac{4x^{2}(2x+1)+1}{2(x+1)(2x-1)(2x+1)}>0 \notag\\
% \Rightarrow & \frac{(x-1)(2x-1)(2x+1)+(2x+1)(2x+1)(x+1)-
%  4x^{2}(2x+1)+1}{2(x+1)(2x-1)(2x+1)}>0 \notag\\
% \Rightarrow & \frac{4x+1}{2(x+1)(2x-1)(2x+1)}>0. \label{eq:21.1}
% \end{align}
% Studiamo separatamente il segno di tutti i fattori che compaiono nella
% frazione, sia quelli al numeratore sia quelli al denominatore e
% costruiamo la tabella dei segni:
%  \[\begin{gathered}
%  N>0\Rightarrow~4x+1>0\Rightarrow x>-{\frac{1}{4}},\\
%  D>0\Rightarrow\left\{\begin{array}{l}
%   x+1>0\Rightarrow x>-1 \\
%   2x-1>0\Rightarrow x>\frac{1}{2}\\
%   2x+1>0\Rightarrow x>-{\frac{1}{2}}
% \end{array}\right..
% \end{gathered}\]
% \begin{center}
% \input{\folder lbr/fig029_seg.pgf}
% \end{center}
% Non abbiamo posto le~\(\CE\) in quanto già rispettate dalle disequazioni
% del denominatore.
% Prendiamo gli intervalli in cui il segno della frazione è positivo
% come richiesto dalla disequazione~\ref{eq:21.1}:
%  \[\IS=\left\{x\in \insR/x<-1\vee -{\frac{1}{2}}<x<-{\frac{1}{4}}\vee 
%  x>\frac{1}{2}\right\}.\]
% \end{esempio}
% 
%  \begin{esempio}
% \(\dfrac{x}{2}-\dfrac{2}{3}\cdot {\dfrac{2x-3}{x-1}}+\dfrac{10x-3}{6x-6}\le
% \dfrac{3}{2}\cdot {\dfrac{x^{2}+2}{3x-2}}+\dfrac{1}{3(x-1)(3x-2)}.\)
% 
% Trasportiamo tutti i termini al primo membro:
% \[\frac{x}{2}-\frac{2}{3}\cdot\frac{2x-3}{x-1}+\frac{10x-3}{6x-6}-
% \frac{3}{2}\cdot\frac{x^{2}+2}{3x-2}-\frac{1}{3(x-1)(3x-2)}\le~0.\]
% 
% Eseguiamo le operazioni per semplificare la frazione e ridurla alla
% forma~\(\frac{N(x)}{D(x)}\le~0\):
% 
% \begin{align}
%   &\frac{x}{2}-\frac{4x-6}{3(x-1)}+\frac{10x-3}{6(x-1)}-
%    \frac{3x^{2}+6}{2(3x-2)}-\frac{1}{3(x-1)(3x-2)}\le~0\notag\\
%   \Rightarrow &\frac{3x(x-1)(3x-2)-2(4x-6)(3x-2)+(10x-3)(3x-2)-
%   3(3x^{2}+6)(x-1)-2}{6(x-1)(3x-2)}\le~0\notag\\
%   \Rightarrow &\frac{11x-2}{6(x-1)(3x-2)}\le~0. \label{eq:22.2}
% \end{align}
% 
% Studiamo il segno del numeratore e dei fattori del denominatore:
%  \[\begin{gathered}N\ge~0\Rightarrow~11x-2\ge~0\Rightarrow x\ge\frac{2}{11},\\
%   D>0\Rightarrow\left\{\begin{array}{l}
%   d_{1}>0\Rightarrow x-1>0\Rightarrow x>1\\
%   d_{2}>0\Rightarrow~3x-2>0\Rightarrow x>\dfrac{2}{3}
% \end{array}\right.. \end{gathered}\]
% \begin{center}
% \input{\folder lbr/fig030_seg.pgf}
% \end{center}
% Non abbiamo posto le~\(\CE\) in quanto già rispettate dalle disequazioni
% del denominatore. Prendiamo gli intervalli in cui il segno della frazione è 
% positivo o
% nullo come dalla disequazione~\ref{eq:22.2}:
% \[\IS=\left\{x\in \insR/x\le\frac{2}{11}\vee \frac{2}{3}<x<1\right\}.\]
%  \end{esempio}
% % \end{exrig}
% % \ovalbox{\risolvii \ref{ese:21.54}, \ref{ese:21.55}, \ref{ese:21.56}, 
% \ref{ese:21.57}, \ref{ese:21.58}, \ref{ese:21.59}, \ref{ese:21.60}, 
% \ref{ese:21.61}, \ref{ese:21.62}, \ref{ese:21.63}}
% % 
% % \vspazio\ovalbox{\ref{ese:21.64}, \ref{ese:21.65}, \ref{ese:21.66}, 
% \ref{ese:21.67}, \ref{ese:21.68}, \ref{ese:21.69}}
% % 
% % \cleardoublepage
