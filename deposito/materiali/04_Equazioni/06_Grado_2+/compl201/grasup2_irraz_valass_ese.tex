% (c) 2016 Elisabetta Campana
% (c) 2016 Daniele Zambelli daniele.zambelli@gmail.com

\section{Esercizi}

\subsection{Esercizi dei singoli paragrafi}

\subsubsection*{\numnameref{sec:irvalass_supsec}}

% \begin{esercizio}
% \label{ese:D.19}
% testo esercizio
% \end{esercizio}

\begin{esercizio}\label{ese:03.1}
Risolvi le seguenti equazioni:
\begin{enumeratea}
\item \(9\,{x}^{8}-108\,{x}^{4}+360=0\) 
\hfill [\(\not\exists x \in \mathbb{R}\)]
\item \(9\,{x}^{8}-72\,{x}^{4}+144=324\) 
\hfill [\(-\sqrt [4]{10};~\sqrt [4]{10}\)]
\item \(-9=-64\, \left( 5\,x-2 \right) ^{6}\) 
\hfill [\({\dfrac{\sqrt [3]{3}}{10}}+{\dfrac{2}{5}};~-{\dfrac
{\sqrt [3]{3}}{10}}+{\dfrac{2}{5}}\)]
\item \(729\, \left( x+1 \right) ^{6}+2=0\) 
\hfill [\(\not\exists x \in \mathbb{R}\)]
\item \(243\,{x}^{5}=-9\) 
\hfill [\(-{\dfrac{{3}^{{\frac{2}{5}}}}{3}}\)]
% \item \(0= \left( 9\,x-7 \right) ^{5}-9\) 
% \hfill [\{\dfrac{{3}^{{\frac{2}{5}}}}{9}}+{\dfrac{7}{9}}\)]
\item \(6\,{x}^{10}-96\,{x}^{5}+234=0\) 
\hfill [\(\sqrt [5]{3});~\sqrt [5]{13}\)]
\item \(-8\,{x}^{8}-392=112\,{x}^{4}-72\) 
\hfill [\(\not\exists x \in \mathbb{R}\)]
\item \(78=-3\,{x}^{10}+6\,{x}^{5}\) 
\hfill [\(\not\exists x \in \mathbb{R}\)]
\item \(-70\,{x}^{4}+175=-7\,{x}^{8}-448\) 
\hfill [\(\not\exists x \in \mathbb{R}\)]
% \item \(- \left( 5\,x-9 \right) ^{7}=-9\) 
% \hfill [\{\dfrac{{3}^{{\frac{2}{7}}}}{5}}+{\dfrac{9}{5}}\)]
% \item \(-7\,{x}^{2}=-x \left( {x}^{3}-10\,{x}^{2}+70 \right) \) 
% \hfill [\(0;~10;~\sqrt {7};~-\sqrt {7}\)]
\item \(-8\,{x}^{10}-32\,{x}^{5}=160\) 
\hfill [\(\not\exists x \in \mathbb{R}\)]
\item \(-7\,{x}^{10}+140\,{x}^{5}-700=0\) 
\hfill [\(\sqrt[5]{10}\)]
\item \(-56\,{x}^{4}-679=7\,{x}^{8}\) 
\hfill [\(\not\exists x \in \mathbb{R}\)]
\item \({x}^{4}-13\,{x}^{3}-13\,x+40=-41\,{x}^{2}\) 
\hfill [\(5;~8\)]
\item \(1024\,{x}^{5}-8=0\) 
\hfill [\(\dfrac{{2}^{\frac{3}{5}}}{4}\)]
% \item \(16\,{x}^{5}+36={x}^{10}+64\) 
% \hfill [\(\sqrt [5]{2};~\sqrt [5]{14}\)]
\item \(x \left( {x}^{3}+6 \right) =-6\,{x}^{3}-9\,{x}^{2}-8\) 
\hfill [\(-4;~-2\)]
\item \({x}^{4}-72\,x+120=2\,{x}^{2} \left( 6\,x-13 \right) \) 
\hfill [\(2;~10\)]
\item \(0=-6\,{x}^{8}-24\,{x}^{4}+462\) 
\hfill [\(-\sqrt [4]{7};~\sqrt [4]{7}\)]
\item \( \left( x+9 \right) ^{6}-4=0\) 
\hfill [\(\sqrt [3]{2}-9;~-\sqrt [3]{2}-9\)]
\item \(128\,{x}^{3}+800=-8\,{x}^{6}\) 
\hfill [\(\not\exists x \in \mathbb{R}\)]
% \item \(5\,{x}^{8}-40=0\) 
% \hfill [\(-{2}^{{\frac{3}{8}}};~{2}^{{\frac{3}{8}}}\)]
% \item \(0=2\,{x}^{10}-8\,{x}^{5}+8\) 
% \hfill [\\sqrt [5]{2}\)]
% \item \(108\,{x}^{4}=9\,{x}^{8}+324\) 
% \hfill [\(-\sqrt [4]{6};~\sqrt [4]{6}\)]
\end{enumeratea}
\end{esercizio}

\begin{esercizio}\label{ese:03.1}
Risolvi le seguenti equazioni:
\begin{enumeratea}
\item \(-{\dfrac{ \left( x+15 \right) ^{5}}{9^5}}= \left( 4\,x-5 \right) 
^{5}\) 
\hfill [\(\dfrac{30}{37}\)]
\item \({\dfrac{{x}^{10}}{5}}=-{\dfrac{3\,{x}^{5}}{4}}-{\dfrac{65}{64}}\) 
\hfill [\(\not\exists x \in \mathbb{R}\)]
\item \({x}^{3}-{\dfrac{5}{28}}={\dfrac{7\,{x}^{6}}{5}}\) 
\hfill [\(\dfrac{\sqrt [3]{980}}{14}\)]
% \item \({\dfrac{7\,{x}^{8}}{10}}=-{\dfrac{{x}^{4}}{9}}+{\dfrac
% {1615}{1134}}\) 
% \hfill [\(-{\dfrac{{7}^{{\frac{3}{4}}}\sqrt {3}\sqrt 
% [4]{85}}{21}};~
% {\dfrac{{7}^{{\frac{3}{4}}}\sqrt {3}\sqrt [4]{85}}{21}}\)]
\item \({\dfrac{ \left( 9\,x-32 \right) ^{4}}{12^4}}=-2\) 
\hfill [\(\not\exists x \in \mathbb{R}\)]
% \item \({x}^{4}+{\dfrac{62\,{x}^{2}}{7}}-{\dfrac{88\,x}{7}}-{\dfrac
% {80}{7}}=-11\,{x}^{3}\) 
% \hfill [\(-10;~-1;~{\dfrac{2 \sqrt {14}}{7}};~
% -{\dfrac{2 \sqrt {14}}{7}}\)]
% \item \({\dfrac{x \left( 4\,{x}^{3}+31\,x+63 \right) 
% }{4}}=7\,{x}^{3}+{\dfrac{45}{2}}\) 
% \hfill [\(5)\sor-{\dfrac{3}{2}};~2)\sor{\dfrac{3}{2}}\)]
\item \(-{\dfrac{4}{5}}={\dfrac{9\,{x}^{6}}{5}}\) 
\hfill [\(\not\exists x \in \mathbb{R}\)]
\item \(0={\dfrac{{x}^{10}}{2}}-{\dfrac{{x}^{5}}{2}}+{\dfrac{401}{8}}\) 
\hfill [\(\not\exists x \in \mathbb{R}\)]
\item \(0={\dfrac{ \left( 35\,x-18 \right) ^{5}}{24}}\) 
\hfill [\(\dfrac{18}{35}\)]
\item \({x}^{4}-{\dfrac{601\,{x}^{2}}{10}}-{\dfrac
{2\,x}{5}}+6=-4\,{x}^{3}\) 
\hfill [\(-10;~6;~{\dfrac{\sqrt {10}}{10}};~
-{\dfrac{\sqrt {10}}{10}}\)]
\item \(-90=-{\dfrac{x \left( 3\,{x}^{3}+45\,{x}^{2}+157\,x-75 \right) 
}{3}}\) 
\hfill 
[\(-9;~x=-6;~x={\dfrac{\sqrt {15}}{3}};~x=-{\dfrac{\sqrt{15}}{3}}\)]
\item \(0={\dfrac{ \left( 35\,x-16 \right) ^{7}}{40^7}}-{\dfrac{1}{2}}\) 
\hfill [\({\dfrac{4 {2}^{6/7}}{7}}+{\dfrac{16}{35}}\)]
\item \({\dfrac{2}{9}}=-{\dfrac{ \left( 9\,x-56 \right) ^{8}}{5}}\) 
\hfill [\(\not\exists x \in \mathbb{R}\)]
\item \(-{\dfrac{512\, \left( 25\,x-14 \right) ^{9}}{35^9}}+4=0\) 
\hfill [\({\dfrac{7 {2}^{2/9}}{10}}+{\dfrac{14}{25}}\)]
\item \({\dfrac{5\,{x}^{8}}{7}}+{\dfrac{3\,{x}^{4}}{2}}=-{\dfrac
{91}{80}}\) 
\hfill [\(\not\exists x \in \mathbb{R}\)]
\item \({\dfrac{ \left( 81\,x+80 \right) ^{7}}{72^7}}=- \left( x-1 
\right) ^{7}\) 
\hfill [\(-{\dfrac{8}{153}}\)]
\item \(0=-{\dfrac{ \left( 9\,x-5 \right) ^{4}}{6561}}\) 
\hfill [\({\dfrac{5}{9}}\)]
\item \({\dfrac{ \left( 9\,x-10 \right) ^{3}}{15^3}}=- \left( 9\,x+7 
\right) ^{3}\) 
\hfill [\(-{\dfrac{95}{144}}\)]
\item \({\dfrac{3\,{x}^{3}}{7}}={\dfrac{4\,{x}^{6}}{5}}+{\dfrac
{24545}{28^2}}\) 
\hfill [\(\not\exists x \in \mathbb{R}\)]
\item \({\dfrac{1}{486}}=-6\,{x}^{10}-{\dfrac{2\,{x}^{5}}{9}}\) 
\hfill [\(-{\dfrac{{12}^{{\frac{2}{5}}}}{6}}\)]
% \item \(-{\dfrac{ \left( 3\,x-2 \right)  \left( 4\,{x}^{2}-x-2 \right) 
% }{3}}=-{x}^{4}\) 
% \hfill [\({\dfrac{\sqrt {3}}{3}};~-{\dfrac{\sqrt {3}}{3}};~2\)]
\item \({\dfrac{4\,{x}^{3}}{3}}-{\dfrac{16}{81}}={\dfrac
{9\,{x}^{6}}{4}}-{\dfrac{100}{9}}\) 
\hfill [\({\dfrac{\sqrt [3]{68}}{3}};~-{\dfrac{\sqrt[3]{52}}{3}}\)]
\end{enumeratea}
\end{esercizio}

\begin{esercizio}\label{ese:03.1}
Risolvi le seguenti disequazioni:
\begin{enumeratea}
% \item \(5^9\, \left( x-1 \right)^{9}+512\, \left( 2\,x-1 \right)^{9}>0\) 
% \hfill [\({\dfrac{7}{9}}<x\)]
% \item \(-168\geq 3\,{x}^{6}+54\,{x}^{3}\) 
% \hfill [\(x\leq -{2}^{{\frac{2}{3}}} \sand -\sqrt [3]{14}\leq x\)]
\item \(4\,{x}^{6}-48\,{x}^{3}+144\leq 324\) 
\hfill [\(x\leq \sqrt [3]{15} \sand -\sqrt [3]{3}\leq x\)]
\item \(-8\, \left( 3\,x-1 \right) ^{3}> \left( 7\,x+4 \right) ^{3}\) 
\hfill [\(x<-{\dfrac{2}{13}}\)]
\item \(0\geq  \left( 10\,x-7 \right) ^{5}-1\) 
\hfill [\(x\leq {\dfrac{4}{5}}\)]
\item \(-{x}^{3}\geq -1\) 
\hfill [\(x\leq 1\)]
\item \(-5\,{x}^{9}+40\geq 0\) 
\hfill [\(x\leq \sqrt [3]{2}\)]
\item \(9\,{x}^{10}+900\geq 180\,{x}^{5}\) 
\hfill [\(\forall x \in \mathbb{R}\)]
\item \({x}^{3} \left( x+8 \right) >-15\,{x}^{2}\) 
\hfill [\((x<-5)\sor ((x<0) \sand (-3<x))\sor (0<x)\)]
\item \(3\,{x}^{8}-147<-6\,{x}^{4}-3\) 
\hfill [\(x<\sqrt [4]{6} \sand -\sqrt [4]{6}<x\)]
% \item \(14\,{x}^{3}<-{x}^{4}-46\,{x}^{2}+28\,x+96\) 
% \hfill [\(((x<-6) \sand (-8<x))\sor ((x<\sqrt {2}) \sand 
% (-\sqrt {2}<x))\)]
\item \(-10\,{x}^{6}\geq 90\) 
\hfill [\(\not\exists x \in \mathbb{R}\)]
\item \(126\,{x}^{3}+133<7\,{x}^{6}\) 
\hfill [\((x<-1)\sor (\sqrt [3]{19}<x)\)]
\item \(-64\, \left( 2\,x+1 \right) ^{3}> \left( 2\,x-3 \right) ^{3}\) 
\hfill [\(x<-{\dfrac{1}{10}}\)]
% \item \(162>-2\,{x}^{8}+36\,{x}^{4}\) 
% \hfill [\((x<-\sqrt {3})\sor ((x<\sqrt {3}) \sand (-\sqrt {3}<x))\sor 
% (\sqrt {3}<x)\)]
\item \({x}^{10}+81>-18\,{x}^{5}\) 
\hfill [\(\not\exists x \in \mathbb{R}\)]
\item \( \left( 8\,x+7 \right) ^{4}+6\leq 0\) 
\hfill [\(\not\exists x \in \mathbb{R}\)]
% \item \(-1000000000\, \left(x+1 \right)^{9}+\left(9\,x-8 \right)^{9}>0\) 
% \hfill [\(x<-18\)]
\item \(-390625\, \left( 2\,x-1 \right) ^{8}\geq - \left( 2\,x-5 \right) 
^{8}\) 
\hfill [\(x\leq {\dfrac{5}{6}} \sand 0\leq x\)]
\item \(-8\,{x}^{10}<16\,{x}^{5}+8\) 
\hfill [\((x<-1)\sor (-1<x)\)]
\item \(- \left( x-9 \right) ^{7}<-128\, \left( 4\,x-5 \right) ^{7}\) 
\hfill [\(x<{\dfrac{1}{7}}\)]
\item \(-42\,{x}^{5}+700>-7\,{x}^{10}-63\) 
\hfill [\(\forall x \in \mathbb{R}\)]
\item \(6\,{x}^{4}<30\) 
\hfill [\(x<\sqrt [4]{5} \sand -\sqrt [4]{5}<x\)]
\item \(-60\,{x}^{4}+490>-10\,{x}^{8}-90\) 
\hfill [\(\forall x \in \mathbb{R}\)]
% \item \(- \left( 8\,x+5 \right) ^{7}\geq -3\) 
% \hfill [\(x\leq -{\dfrac{5}{8}}+{\dfrac{\sqrt [7]{3}}{8}}\)]
% \item \(-7\,{x}^{10}-28\geq -28\,{x}^{5}\) 
% \hfill [\(x=\sqrt [5]{2}\)]
\end{enumeratea}
\end{esercizio}

\begin{esercizio}\label{ese:03.1}
Risolvi le seguenti disequazioni:
\begin{enumeratea}
\item \(0\geq -{\dfrac{7\,{x}^{8}}{5}}\) 
\hfill [\(\forall x \in \mathbb{R}\)]
\item \({\dfrac{5\,{x}^{3}}{3}}+{\dfrac{25}{36}}\geq -{x}^{6}\) 
\hfill [\(\forall x \in \mathbb{R}\)]
% \item \({\dfrac{x \left( 7\,{x}^{3}+70\,{x}^{2}+60\,x-30 \right) 
% }{7}}<{\dfrac{27}{7}}\) 
% \hfill [\(((x<-1) \sand (-9<x))\sor ((x<{\dfrac{\sqrt {21}}{7}}) \sand 
% (-{\dfrac{\sqrt {21}}{7}}<x))\)]
\item \({\dfrac{3}{14}}\leq -{\dfrac{2\,{x}^{8}}{7}}\) 
\hfill [\(\not\exists x \in \mathbb{R}\)]
\item \( \left( 3\,x-2 \right) ^{4}-{\dfrac{ \left( 3\,x+10 \right) 
^{4}}{15^4}}\geq 0\) 
\hfill [\((x\leq {\dfrac{5}{12}})\sor ({\dfrac{20}{21}}\leq x)\)]
% \item \({x}^{2} \left( {x}^{2}+83 \right) \geq 19\,{x}^{3}-133\,x+630\) 
% \hfill [\((x\leq -\sqrt {7})\sor ((x\leq 9) \sand (\sqrt {7}\leq x))\sor 
% (10\leq x)\)]
\item \(729\,{x}^{6}+3\leq 0\) 
\hfill [\(\not\exists x \in \mathbb{R}\)]
% \item \(-3\,{x}^{2} \left( 2\,x+5 \right) <-{x}^{4}-48\,x-56\) 
% \hfill [\(((x<-1) \sand (-2 \sqrt {2}<x))\sor ((x<7) \sand (2 \sqrt 
% {2}<x))\)]
\item \({\dfrac{8}{9}}\geq - \left( 5\,x+3 \right) ^{9}\) 
\hfill [\(-{\dfrac{\sqrt [3]{2}{3}^{{\frac{7}{9}}}}{15}}-{\dfrac
{3}{5}}\leq x\)]
\item \(-55\,{x}^{2}>-{x}^{4}-3\,{x}^{3}+3\,x-54\) 
\hfill [\((x<-9)\sor ((x<1) \sand (-1<x))\sor (6<x)\)]
\item \(-{\dfrac{7\,{x}^{4}}{9}}-{\dfrac{7}{12}}>0\) 
\hfill [\(\not\exists x \in \mathbb{R}\)]
\item \(-{\dfrac{10}{3}}\geq {\dfrac{5\,{x}^{6}}{4}}\) 
\hfill [\(\not\exists x \in \mathbb{R}\)]
\item \(0\leq -{\dfrac{ \left( 5\,x+42 \right) ^{8}}{30^8}}-{\dfrac
{1}{8}}\) 
\hfill [\(\not\exists x \in \mathbb{R}\)]
\item \({\dfrac{2}{7}}\geq -{x}^{6}\) 
\hfill [\(\forall x \in \mathbb{R}\)]
\item \(-6\,{x}^{3}-47\,{x}^{2}+42\,x+280\geq -{x}^{4}\) 
\hfill [\((x\leq -4)\sor ((x\leq \sqrt {7}) \sand (-\sqrt {7}\leq x))\sor 
(10\leq x)\)]
\item \(0>-{x}^{10}-1\) 
\hfill [\(\forall x \in \mathbb{R}\)]
\item \({\dfrac{5\,{x}^{6}}{3}}\geq -{\dfrac{9\,{x}^{3}}{8}}-{\dfrac
{1971}{2^8  cdot 5}}\) 
\hfill [\(\forall x \in \mathbb{R}\)]
\item \(1\leq {\dfrac{ \left( 21\,x-1 \right) ^{3}}{343}}\) 
\hfill [\({\dfrac{8}{21}}\leq x\)]
\item \(0\geq -{\dfrac{ \left( 35\,x-36 \right) ^{6}}{4}}-{\dfrac{ 
\left( 50\,x-27 \right) ^{6}}{5}}\) 
\hfill [\(\forall x \in \mathbb{R}\)]
% \item \({\dfrac{5}{98}}\leq {\dfrac{5\,{x}^{6}}{2}}-{\dfrac
% {4\,{x}^{3}}{5}}+{\dfrac{8}{125}}\) 
% \hfill [\((x\leq {\dfrac{\sqrt [3]{735}}{35}})\sor ({\dfrac{\sqrt 
% [3]{12985}}{35}}\leq x)\)]
% \item \(9\,{x}^{8}\leq -{\dfrac{4\,{x}^{4}}{3}}-{\dfrac{241}{324}}\) 
% \hfill [\(\not\exists x \in \mathbb{R}\)]
\item \({\dfrac{21}{50}}\geq -{\dfrac{7\,{x}^{9}}{10}}\) 
\hfill [\(-{\dfrac{\sqrt [9]{3}{5}^{{\frac{8}{9}}}}{5}}\leq x\)]
\item \({x}^{8}-1\leq -{\dfrac{6\,{x}^{4}}{7}}-{\dfrac{9}{49}}\) 
\hfill [\(x\leq {\dfrac{{7}^{{\frac{3}{4}}}\sqrt {2}}{7}} \sand -{\dfrac
{{7}^{{\frac{3}{4}}}\sqrt {2}}{7}}\leq x\)]
\item \(-{\dfrac{72}{49}}\geq {\dfrac{2\,{x}^{6}}{9}}\) 
\hfill [\(\not\exists x \in \mathbb{R}\)]
% \item \({\dfrac{4}{5}}\geq -{\dfrac{ \left( x+3 \right) ^{9}}{3^9}}\) 
% \hfill [\(-{\dfrac{3 {2}^{2/9}{5}^{{\frac{8}{9}}}}{5}}-3\leq x\)]
% \item \({\dfrac{ \left( 8\,{x}^{2}-5 \right)  \left( {x}^{2}+2 \right) 
% }{8}}\leq -{\dfrac{3\,x \left( 8\,{x}^{2}-5 \right) }{8}}\) 
% \hfill [\(((x\leq -1) \sand (-2\leq x))\sor ((x\leq {\dfrac{\sqrt 
% {10}}{4}}) \sand (-{\dfrac{\sqrt {10}}{4}}\leq x))\)]
\end{enumeratea}
\end{esercizio}

\subsubsection*{\numnameref{sec:irvalass_irraz}}

\begin{esercizio}\label{ese:03.1}
Una sola delle seguenti equazioni è irrazionale, perché?
\begin{multicols}{4}
\begin{enumeratea}
\item \(\sqrt{3} x = 3\)
\item \(\sqrt{5} -2x = 5\)
\item \(\sqrt{5x} -2x = 5\)
\item \(\sqrt{5}x^2 +\sqrt{2} = 0\)
\end{enumeratea}
\end{multicols}
\end{esercizio}

\begin{esercizio}\label{ese:03.1}
Risolvi le seguenti equazioni irrazionali:
\begin{multicols}{4}
\begin{enumeratea}
\item \(\sqrt{3 -x} = -3\)
\item \(\sqrt{x +1} +2 = 0 \)
\item \(\sqrt{-x^2 -1} = 2 \)
\item \(\sqrt{1 -x} = \sqrt{2}\)
\end{enumeratea}
\end{multicols}
\end{esercizio}

\begin{esercizio}\label{ese:03.1}
Risolvi le seguenti equazioni irrazionali del tipo:
\(\sqrt[3]{A(x)} = B(x)\)
\begin{multicols}{2}
\begin{enumeratea}
\item \(\sqrt[3]{x^2 -1} = 2\) \hfill [\(\mp 3\)]
\item \(\sqrt[3]{x -1} = -2\) \hfill [\(-7\)]
\item \(2\sqrt[3]{x^2 -1} = 1\) \hfill [\(\mp \dfrac{3 \sqrt{2}}{4}\)]
\item \(\sqrt[3]{x^3 -2x^2 -4} = x -2\) \hfill [\(\dfrac{3 \mp 
\sqrt{5}}{2}\)]
\item \(\sqrt[3]{x^3 -6x^2} +2 = x\) \hfill [\(\dfrac{2}{3}\)]
\item \(\sqrt[3]{2 +x^3} = 1 +x\) \hfill [\(\dfrac{3 \mp \sqrt{21}}{6}\)]
\item \(\sqrt[3]{x^3 -4x} -x +2 = 0\) \hfill [\(\dfrac{2}{3};~2\)]
\item \(\sqrt[3]{x -1} = \sqrt[3]{2x -1}\) \hfill [\(0\)]
\item \(\sqrt[3]{x^3 +4x^2 -3} = x +1\)
  \hfill [\(4;~-1\)]
\item \(\sqrt[3]{x^3 -7} = x -1\) \hfill [\(-1;~2\)]
\end{enumeratea}
\end{multicols}
\end{esercizio}

\begin{esercizio}\label{ese:03.1}
Risolvi le seguenti equazioni irrazionali del tipo:
\(\sqrt{A(x)} = B(x)\)
\begin{multicols}{2}
\begin{enumeratea}
\item \(\sqrt{3 -x^2} = 1\) \hfill [\(\mp \sqrt{2}\)]
\item \(\sqrt{2x +5} = 3x -3\) \hfill [\(2\)]
\item \(\sqrt{x} = x-2\) \hfill [\(4\)]
\item \(\sqrt{x -6} = -x\) \hfill [\(4\)]
\item \(\sqrt{4 -3x} = x\) \hfill [\(1\)]
\item \(x = 6 +\sqrt{x^2 -12}\) \hfill [N.H.S.R.]
\item \(\sqrt{7x^2 +1} = 1 +x\) \hfill [\(0;~\dfrac{1}{3}\)]
\item \(\sqrt{x^2 -4x} = 3 +x\) \hfill [\(-\dfrac{9}{10}\)]
\item \(\sqrt{25 -x^2} +2 = 2x\) \hfill [\(3\)]
\item \(4x -2 = \sqrt{x^2 -1}\) \hfill [\(\emptyset\)]
% \item \(\sqrt{3x^2 +6x +1} = x^2 +3x +1\) \hfill [\(-4;~0\)]
\item \(\sqrt{9 -x^2} = x +3\) \hfill [\(-3;~0\)]
\item \(\sqrt{9 -x^2} = -x -3\) \hfill [\(-3\)]
\item \(\sqrt{12 -x\tonda{x +2} +x} = x -3\) \hfill [\(3\)]
\end{enumeratea}
\end{multicols}
\end{esercizio}

\begin{esercizio}\label{ese:03.1}
Risolvi le seguenti disequazioni irrazionali
\begin{multicols}{2}
\begin{enumeratea}
\item \(\sqrt{x+3} > 0\)
\item \(\sqrt{-x} \geq 0 \)
\item \(\sqrt{x^2 +1} > -2 \)
\item \(\sqrt{1 -x} < 0\)
\item \(\sqrt[3]{x-2} \geq 0\)
\item \(\sqrt[3]{-x^2-1} \leq 0\)
\item \(\sqrt[3]{-x^2} \geq 0\)
\item \(\sqrt[4]{-2x +4} \geq 2\)
\end{enumeratea}
\end{multicols}
\end{esercizio}

\begin{esercizio}\label{ese:03.1}
Risolvi le seguenti disequazioni irrazionali del tipo:
\(\sqrt[3]{A(x)} \lessgtr B(x)\)

Esercizio guida: \(\sqrt[3]{x +5} < 3\)
\begin{center}
\begin{tabular}{ll}
eleviamo alla terza entrambi i membri: & \(x +5 < 27\)\\
risolviamo & \(x < 22\)
\end{tabular}
\end{center}

% \begin{multicols}{2}
\begin{enumeratea}
\item \(\sqrt[3]{x^3 -8} < x +2 \)
\hfill [\(x<-2 \sor x>0\)]
\item \(\sqrt[3]{x^3 -8} > x +2 \)
\hfill [\(\forall x \in \R\)]
\item \(\sqrt[3]{x^3 -3x} > -2x \)
\hfill [\(-\dfrac{1}{\sqrt{3}}<x<0 \sor x>\dfrac{1}{\sqrt{3}}\)]
\item \(\sqrt[3]{x^3 -6x} < x -2 \)
\hfill [\(x< \dfrac{2}{3}\)]
\end{enumeratea}
% \end{multicols}
\end{esercizio}

\begin{esercizio}\label{ese:03.1}
Risolvi le seguenti disequazioni irrazionali del tipo:
\(\sqrt{A(x)} \leq B(x)\)

Esercizio guida: \(x -1 \geq \sqrt{1 +2x}\)
\begin{center}
\begin{tabular}{ll}
forma canonica: & \(\sqrt{2x +1} \leq x -1\)\\
sistema equivalente & 
\(\sistema{x-1 \geq 0 \\ 2x +1 \geq 0  \\ 2x +1 \leq x^2 -2x +1}\)\\
\(\sistema{x \geq 1 \\x \geq -\dfrac{1}{2} \\ x^2 -4x \geq 0}\) &
\(\sistema{x \geq 1 \\ x \leq 0 \sor x \geq 4}\)\\
soluzioni del sistema: & \(x \geq 4\)
\end{tabular}
\end{center}

\begin{multicols}{2}
\begin{enumeratea}
\item \(\sqrt{x^2 -1} -3 < x\)
\hfill [\(-\dfrac{5}{3} < x \leq -1 \sor x \geq 1\)]
\item \(\sqrt{x^2 -x} < x +1\)
\hfill [\(-\dfrac{1}{3} < x \leq 0 \sor x \geq 1\)]
\item \(4 + \sqrt{2x -2} < 2x\)
\hfill [\(x > 3\)]
\item \(\sqrt{x^2 -4} +x < 4\)
\hfill [\(x \leq -2 \sor 2 \leq x <\dfrac{5}{2}\)]
\item \(\sqrt{x^2 +16} \leq x -3\)
\hfill [\(\emptyset\)]
% \item \(\sqrt{x^2 -3} < 2\)
% \hfill [\(-\sqrt{7} < x < -\sqrt{3} \sor \sqrt{3} \leq x < \sqrt{7}\)]
\item \(\sqrt{-2x +4} -2x \leq 8\)
\hfill [\(-\dfrac{5}{3} \leq x \leq 2\)]
\item \(5 - \sqrt{x +1} > x\)
\hfill [\(-1 \leq x < 3\)]
\item \(\sqrt{\dfrac{x}{x -1}} < 1 - \dfrac{2}{x}\)
\hfill [\(x < 0\)]
\item \(\sqrt{x^2 -5x +4} < x -3\)
\hfill [\(4 \leq x < 5\)]
\end{enumeratea}
\end{multicols}
\end{esercizio}

\begin{esercizio}\label{ese:03.1}
Risolvi le seguenti disequazioni irrazionali del tipo:
\(\sqrt{A(x)} \geq B(x)\)

Esercizio guida: \(\sqrt{-1 +x} +x > 3\)
\begin{center}
\begin{tabular}{ll}
forma canonica: & \(\sqrt{x -1} > -x +3\)\\
la disequazione è equivalente a & 
\(\sistema{-x+3 < 0 \\ x -1 \geq 0} \sor 
  \sistema{-x+3 \geq 0 \\ x -1 \geq \tonda{-x +3}^2}\)\\
\(\sistema{x > 3 \\ x \geq 1} \sor 
  \sistema{x \leq 3 \\ x^2 -7x +10 < 0}\) &
\(\sistema{x > 3 \\ x \geq 1} \sor 
  \sistema{x \leq 3 \\ 2 < x < 5}\)\\
soluzioni dei sistemi: & \(x > 3 \sor 2 < x \leq 3\) \\
unione delle soluzioni: & \(x > 2\)
\end{tabular}
\end{center}

\begin{multicols}{2}
\begin{enumeratea}
\item \(\sqrt{2x^2 -2} +2 > -3x\)
\hfill [\(x \geq 1 \)]
\item \(2x -1 \leq \sqrt{1 -x^2}\)
\hfill [\(-1 \leq x \leq \dfrac{4}{5}\)]
\item \(\sqrt{x^2 +2x -3} +x > 0\)
\hfill [\(x \geq 1\)]
\item \(\sqrt{2 -2x} > -2x +1\)
\hfill [\(\dfrac{1 - \sqrt{5}}{4} < x \leq 1 <\)]
\item \(\sqrt{x-5} -3x > -2\)
\hfill [\(\emptyset\)]
\item \(\sqrt{4x +20} > x +2\)
\hfill [\(-5 \leq x < 4\)]
\item \(\sqrt{x^2 -5x} > 2x\)
\hfill [\(x < 0\)]
\item \(\sqrt{x +2} > x\)
\hfill [\(-2 \leq x < 2\)]
\item \(\sqrt{\dfrac{x}{x -1}} < 1 - \dfrac{2}{x}\)
\hfill [\(x < 0\)]
\item \(x +4 < \sqrt{x^2 -4}\)
\hfill [\(x < -\dfrac{5}{2}\)]
\end{enumeratea}
\end{multicols}
\end{esercizio}

\subsubsection*{\numnameref{sec:irvalass_valass}}

\begin{esercizio}\label{ese:03.1}
Calcola il valore dei seguenti valori assoluti:
\begin{multicols}{3}
\begin{enumeratea}
\item \(|-2|=\dots\)
\item \(|+7|=\dots\)
\item \(|3-7|=\dots\)
\item \(|5-\dfrac{1}{3}|=\dots\)
\item \(|\sqrt{2}-\sqrt{6}|=\dots\)
\item \(|\sqrt{3}-2|=\dots\)
\end{enumeratea}
\end{multicols}
\end{esercizio}

\begin{esercizio}\label{ese:03.1}
Completa come nell'esempio:
\[
|2x|=
        \left\lbrace 
        \begin{array}{lcl}
        2x & \text{se}& x\geq 0\\
        -2x & \text{se}& x< 0\\
        \end{array}
        \right.
\]

\begin{enumerate}
\item 
\(
|4x|=
\left\lbrace 
\begin{array}{lcl}
\dots & \text{se}& x\geq\dots\\
\dots & \text{se}& x<\dots\\
\end{array}
\right.
\)
\item 
\(
|-2x|=
\left\lbrace 
\begin{array}{lcl}
\dots & \text{se}& x\geq\dots\\
\dots & \text{se}& x<\dots\\
\end{array}
\right.
\)
\item 
\(
|x-3|=
\left\lbrace 
\begin{array}{lcl}
x-3 & \text{se}& x\geq\dots\\
\dots & \text{se}& x<\dots\\
\end{array}
\right.
\)
\item 
\(
|x^2-2x|=
\left\lbrace 
\begin{array}{lcl}
\dots & \text{se}& x\leq\dots \sor x\geq\dots\\
\dots & \text{se}& \dots <x<\dots\\
\end{array}
\right.
\)
\item 
\(
|7-2x|=
\left\lbrace 
\begin{array}{lcl}
\dots & \text{se}& x\geq\dots\\
\dots & \text{se}& x<\dots\\
\end{array}
\right.
\)
\item 
\(
|x^2-6x+8|=
\left\lbrace 
\begin{array}{lcl}
\dots & \text{se}& x\leq\dots \sor x\geq\dots\\
\dots & \text{se}& \dots <x<\dots\\
\end{array}
\right.
\)
\item 
\(
|\dfrac{x-1}{2x}|=
\left\lbrace 
\begin{array}{lcl}
\dfrac{x-1}{2x} & \text{se}& x\leq\dots \sor x\geq\dots\\
\dots & \text{se}& \dots <x<\dots\\
\end{array}
\right.
\)
\end{enumerate}
\end{esercizio}

\begin{esercizio}\label{ese:03.1}
Quale delle seguenti figure rappresenta il grafico della funzione 
\(y=|x-4|\)?

\begin{figure}[h]
\begin{inaccessibleblock}[TODO]
\centering
\includegraphics[width=0.9\linewidth]{img/imm6} 
%[scale=0.35]{img/fig001.png}
\end{inaccessibleblock}
% \caption{Retta}
\label{fig:abs_imm6}
\end{figure}
% \begin{figure}[h]
%         \centering
%         \includegraphics[width=0.9\linewidth]{imm6}
%         %\caption{}
%         \label{fig:imm1}
% \end{figure}
\end{esercizio}

\begin{esercizio}\label{ese:03.1}
Traccia il grafico delle seguenti funzioni come nell'esempio:

\noindent\begin{minipage}{.40\textwidth}
\[y=|x^2-4|\]
\begin{tabular}{|c|c|}
        \hline
        x & y \\
        \hline
        0 & 4 \\
        \hline  
        -1 & 3 \\
        \hline
        1 & 3 \\
        \hline
        -2 & 0 \\
        \hline
        2 & 0 \\
        \hline
        -3 & 5 \\
        \hline
        3 & 5 \\
        \hline                                                  
\end{tabular} 
\end{minipage}
\hfill
\begin{minipage}{.58\textwidth}
\begin{inaccessibleblock}[TODO]
\centering
\includegraphics[width=0.9\linewidth]{img/imm7} 
%[scale=0.35]{img/fig001.png}
\end{inaccessibleblock}
%       \includegraphics[width=0.5\linewidth]{imm7}
\end{minipage}

\begin{multicols}{4}
\begin{enumeratea}
        \item \(y=|2x-1|\)
        \item \(y=|x+2|\)
        \item \(y=|x-1|\)
        \item \(y=2|x-1|\)
        \item \(y=\dfrac{|x-2|}{2}\)
        \item \(y=|3x|\)
        \item \(y=|x^2-3x+2|\)
        \item \(y=|x^2-1|\)
\end{enumeratea}
\end{multicols}
\end{esercizio}

\begin{esercizio}\label{ese:03.1}
Equazioni del tipo \(|P(x)|=k\) esempi:
\begin{enumeratea}
\item[a)] \(|x^2-3|=0\), ricordando che \(|x^2-3|=0\) se e solo se 
\(x^2-3=0\), 
l'equazione ha come soluzioni \(x=\pm\sqrt{3}\).
\item[b)] \(|x^2-3|=-2\), impossibile perché il valore assoluto di 
un'espressione 
algebrica è sempre un numero non negativo.
\item[c)] \(|2x-3|=2\), l'equazione equivale a:
\[2x-3=2 \sor 2x-3=-2\]
e quindi
\[x=\dfrac{5}{2}\sor x=\dfrac{1}{2}\]
\end{enumeratea}

\noindent Risolvi le seguenti equazioni:

\begin{multicols}{2}
\begin{enumeratea}
\item \(|x-3|=2\)
 \hfill \(\left[ 1;~5\right] \)
\item \(|x+1|=3\)
 \hfill \(\left[ -4;~2\right] \)
\item \(|x^2-6x+8|=0\)
 \hfill \(\left[ 2;~4\right] \)
\item \(\left| \dfrac{x-1}{2x}\right| =\dfrac{1}{4}\)
 \hfill \(\left[ 
\dfrac{2}{3};~2\right] \)
\item \(|x^2-9|=-3\)
 \hfill \(\left[impossibile \right] \)
\item \(|x^4-x^2|=0\)
 \hfill \(\left[ 0;~\pm 1\right] \)
\item \(|-x+4|=2\)
 \hfill \(\left[ 2;~6\right] \)
\item \(|4x+3|=2\)
 \hfill \(\left[ -\dfrac{3}{2};~-\dfrac{1}{4} \right] \)
\item \(|x^2-6x+4|=4\)
 \hfill \(\left[ 0;~2;~4 ;~6 \right] \)
\item \(|x^2-2x|=1\)
 \hfill \(\left[ 1;~1+\sqrt{2} \right] \)
\item \(|2x^3+6x-5|=-2\)
 \hfill \(\left[ impossibile \right] \)
\item \(\left| \dfrac{x^2-3x}{x+2}\right| =1\)
 \hfill \(\left[ 2\pm 
\sqrt{6} \right] \)
\item \(|x+3|=2\)
 \hfill \(\left[ -1;~-5 \right] \)
\end{enumeratea}
\end{multicols}
\end{esercizio}

\begin{esercizio}\label{ese:03.1}
Equazioni del tipo \(|A(x)|=|B(x)|\) esempi:
\begin{enumeratea}
\item[a)] \(|x^2-4|=|x-2|\), 

l'equazione equivale a \quad \(x^2-4=x-2 \sor x^2-4=-(x-2)\)

cioè: \quad \(x^2-x-2=0 \sor x^2+x+6=0\)

la prima equazione ha soluzioni \([-1;~2]\), 

la seconda \([-3;~2]\), 

pertanto le soluzioni dell'equazione di partenza sono 
\(S=\left\lbrace -3;~-1;~2\right\rbrace \).

\item[b)] \(|x^2-3|=-2\), 

impossibile perché il valore assoluto di 
un'espressione algebrica è sempre un numero non negativo.
\item[c)] \(|2x-3|=2\), 

l'equazione equivale a: \quad \(2x-3=2 \sor 2x-3=-2\)

e quindi \quad \(x=\dfrac{5}{2} \sor x=\dfrac{1}{2}\)
\end{enumeratea}

\noindent Risolvi le seguenti equazioni:

\begin{multicols}{2}
\begin{enumeratea}
\item \(\left| x-1\right| =\left| 2x-3\right| \) 
\hfill \(\left[\dfrac{4}{3};~2\right] \)
\item \(\left| x+1\right| =\left| 2x-1\right| \) 
\hfill \(\left[ 0;~2\right]\)
\item \(\left| x^2-2x+3 \right| =\left| x-4 \right| \) 
\hfill \(\left[\dfrac{1\pm\sqrt{5}}{2}\right] \)
\item \(\left| x^2-x-5\right| =\left| x-2\right| \) 
\hfill \(\left[-1;~3;~\pm \sqrt{7}\right] \)
\item \(\left| x+3\right| =\left| x\right| \) 
\hfill \(\left[ -\dfrac{3}{2} \right] \)
\item \(\left| x^2-5x \right| =\left| x^2+2x \right| \) 
\hfill \(\left[ 0;~\dfrac{3}{2} \right] \)
\item \(\left| 3x+5\right| =\left| 2x+3\right| \) 
\hfill \(\left[-2;~-\dfrac{8}{5} \right] \)
\item \(\left| x+1\right| =\left| 2x\right| \) 
\hfill \(\left[-\dfrac{1}{3};~1 \right] \)
\item \(\left| x^3-6x\right| =\left| x^3-2x\right| \) 
\hfill \(\left[ 0;~\pm 2 \right] \)
\item \(\left|\dfrac{x^2+1}{x}\right| =\left| 2x\right| \) 
\hfill \(\left[\pm 1 \right] \)
\item \(\left| x^2-3x-10\right| =\left| x^2-4\right| \) 
\hfill \(\left[-2;~\dfrac{7}{2} \right] \)
\item \(\left| \dfrac{1}{2}x+2\right| =\left|-x-1\right| \) 
\hfill \(\left[-2;~+2 \right] \)
\end{enumeratea}
\end{multicols}
\end{esercizio}

\begin{esercizio}\label{ese:03.1}
\noindent Risolvi le equazioni del tipo \(|A(x)|=B(x)\)

Esempio:
% \begin{enumeratea}
\(|x-3|=2x+2\), l'equazione equivale a risolvere:
\[
\left\lbrace 
\begin{array}{l}
x-3\geq 0 \\
x-3=2x+2\\
\end{array}
\right.
\sor
\left\lbrace 
\begin{array}{l}
x-3< 0 \\
-(x-3)=2x+2\\
\end{array}
\right.
\]      

cioè:
\[
\left\lbrace 
\begin{array}{l}
x\geq 3 \\
x=-52\\
\end{array}
\right.
\sor
\left\lbrace 
\begin{array}{l}
x< 3 \\
x=\dfrac{1}{3}\\
\end{array}
\right.
\]
il primo sistema non ammette soluzione, pertanto la soluzione 
dell'equazione 
di partenza è \(    x=\dfrac{1}{3}\).
% \end{enumeratea}
\begin{multicols}{2}
\begin{enumeratea}
\item \(\left| x+3 \right| =5x-2 \) 
\hfill \(\left[ \dfrac{5}{4}\right] \)
\item \(\left| x-1 \right| =2x \)
 \hfill \(\left[ \dfrac{1}{3}\right] \)
\item \(\left| x-4 \right| =-6+2x \)
 \hfill \(\left[ \dfrac{10}{3}\right] \)
\item \(\left| x+1 \right| =\dfrac{1}{2}x^2-3 \)
 \hfill \(\left[-1-\sqrt{5}\right] \)
\item \(x^2-2= \left| x \right| \)
 \hfill \(\left[ -2;~2\right] \)
\item \(2\left| x+1 \right| =x^2-2 \)
 \hfill \(\left[1+\sqrt{5};~-2\right] \)
\item \(\left| x-7 \right| =x-8 \)
 \hfill \(\left[ impossibile\right] \)
\item \(\left| \dfrac{x^2+1}{x} \right| =2x \)
 \hfill \(\left[ 1 \right] \)
\item \(\left| x^2-4x-12 \right| =x^2 \)
 \hfill \(\left[ -3;~1\pm \sqrt{7}\right] \)
\item \(\left| x^2-3x+2 \right| =-4+2x \)
 \hfill \(\left[ 2;~3\right] \)
\item \(\left| x^2-4 \right| -x=8 \)
 \hfill \(\left[ -3;~4\right] \)
\end{enumeratea}
\end{multicols}
\end{esercizio}

\begin{esercizio}\label{ese:03.1}
Disequazioni esempi:
\begin{enumeratea}
        \item[a)] \(|x-2|\geq -2\), il valore assoluto di un numero é 
sempre 
positivo o nullo, perciò la disequazione è verificata per ogni \(x\in 
\mathbb{R}\).

        \item[b)] \(|5x-2|\leq 0\), il valore assoluto di un numero é 
sempre 
positivo o nullo, perciò la disequazione è verificata se e solo se 
\(5x-2=0\) 
quindi \(x=\dfrac{2}{5}\).
        \item[c)] \(|x-3|>5\), la disequazione è soddisfatta se \(x-3<-5 
\sor 
x-3>5\) quindi quando \(x<-2 \sor x>8\).
        \item[d)] \(|x-4|\leq 5\), la disequazione è equivalente a \(-5\leq 
x-4 
\leq 5\) e quindi \(-1\leq x \leq 9\).
\end{enumeratea}

\noindent Risolvi le seguenti disequazioni:

\begin{multicols}{2}
\begin{enumeratea}
\item \(\left| 5x-2\right| \geq -2 \)
 \hfill \(\left[ \dots \right] \)
\item \(\left| 8x+2\right| < 0 \)
 \hfill \(\left[ \dots \right] \)
\item \(\left| x-5\right| \geq 3 \)
 \hfill \(\left[ x\leq 2 \sor x\geq 8 \right] \)
\item \(\left| x-3\right| >0 \)
 \hfill \(\left[ x\neq 3 \right] \)
\item \(\left| 2x-5\right| \leq 7 \)
 \hfill \(\left[ -1\leq x \leq 6 \right] \)
\item \(\left| x^2+3x\right| >4 \)
 \hfill \(\left[ x<-4 \sor x>1 \right] \)
\item \(\left| x^2-3x+2\right| >0 \)
 \hfill \(\left[ x\neq 1 \sand x\neq 2 \right] \)
\item \(\left| x^2-4x+4\right| \leq 0 \)
 \hfill \(\left[ x=2 \right] \)
\item \(-\left| 2x-5\right| <3 \)
 \hfill \(\left[ \forall x \in \mathbb{R} \right] \)
\item \(\left| x^4+16\right| \leq 0 \)
 \hfill \(\left[ impossibile \right] \)
\item \(\left| 3x+2\right| \geq 5 \)
 \hfill \(\left[ x\leq -\dfrac{7}{3} \sor x\geq 1 \right] \)
\item \(\left| x^2-4\right| <-4 \)
 \hfill \(\left[ impossibile \right] \)
\item \(-\left| x^3+2\right| <3 \)
 \hfill \(\left[ \forall x \in \mathbb{R} \right] \)
\item \(\left| \dfrac{x^2-4}{x}\right| <3 \)
 \hfill \(\left[ -4<x<-1 \sor 1<x<4 \right] \)
\item \(\left| \dfrac{x-5}{x+3}\right| >\dfrac{1}{2} \)
 \hfill \(\quadra{\left(x<\dfrac{7}{3} \sor x>13 \right) \sand x \neq -3}\)
\item \(\left| \dfrac{x-2}{x-4}\right| >1 \)
 \hfill \(\left[x>3 \sand x \neq 4 \right] \)
\end{enumeratea}
\end{multicols}
\end{esercizio}

\subsection{Esercizi di riepilogo}

\begin{esercizio}\label{ese:03.1}
Determina per quali valori di \(a\) e di \(n\) l’equazione seguente non ha
soluzioni in \(\R\):
\(x^n+3a-12=0\)
\end{esercizio}

\begin{esercizio}\label{ese:03.1}
Risolvi le seguenti equazioni e disequazioni di grado superiore al 
secondo:
\begin{multicols}{2}
\begin{enumeratea}
\item \(\tonda{x^2-1}^3=-8\) \hfill [\(NHSR\)]
\item \(x^4+2x^3-x-2=0\) \hfill [\(-2;~1\)]
\item \(\dfrac{16}{x^2}=\dfrac{x^2}{16}\) \hfill [\(\mp 4\)]
\item \(\dfrac{1}{2}\tonda{x-6}^3+4=0\) \hfill [\(4\)]
\item \(x^5-3x^3+x^2-3=0\) \hfill [\(\mp \sqrt{3};~-1\)]
\item \(\dfrac{1}{2}\tonda{2x-1}^3=500\) \hfill [\(\dfrac{11}{2}\)]
\item \(x^{10}-31x^5-32=0\) \hfill [\(1;~2\)]
\item \(x^3+x>0\) \hfill [\(x>0\)]
\item \(\tonda{x^2+4}\tonda{x^2-9}\leqslant 0\) 
\hfill [\(-3\leqslant x \leqslant +3\)]
\item \(5 \tonda{x-2}^4 +3> 2\) \hfill [\(\emptyset\)]
\item \(\dfrac{\tonda{x^2-1}^2\tonda{x^2-4x}^3}{2x^2+1} \geqslant 0\) 
\hfill [\(x \leqslant 0 \vee x \geqslant 0\)]
\item \(\dfrac{\tonda{x^2-1}^4\tonda{x-4}^3}{x^2+1} \geqslant 0\) 
\hfill [\(x \geqslant 4\)]
\end{enumeratea}
\end{multicols}
\end{esercizio}

\begin{esercizio}\label{ese:03.1}
Risolvi le seguenti equazioni e disequazioni irrazionali:
\begin{multicols}{2}
\begin{enumeratea}
\item \(\sqrt{x} + \sqrt{2x+1}+1 = 0\) \hfill [\(NHSR\)]
\item \(\sqrt{x^2-1} = x+3\) \hfill [\(-4 / 3\)]
\item \(\sqrt{x-3}+5 = x\) \hfill [\(4;~7\)]
\item \(\sqrt{x^2+1}+1 = 0\) \hfill [\(NHSR\)]
\item \(\sqrt{x^2-1} = 2x\) \hfill [\(NHSR\)]
\item \(\sqrt{x-1} > 3-x\) \hfill [\(1 \leqslant x < 3\)]
\item \(\sqrt{x^2 + 1} < -2\) \hfill [\(\emptyset\)]
\item \(\sqrt{x^2-4} < x+1\) \hfill [\(x \geqslant 2\)]
\item \(\sqrt{x^2+4}>-1\) \hfill [\(\R\)]
\end{enumeratea}
\end{multicols}
\end{esercizio}

\begin{esercizio}\label{ese:03.1}
Risolvi le seguenti equazioni e disequazioni con valori assoluti:
\begin{multicols}{2}
\begin{enumeratea}
\item \(\abs{x^2+2} = 3x\) \hfill [\(1;~2\)]
\item \(\abs{x-1}+2 = 0\) \hfill [\(\emptyset\)]
\item \(\abs{2x-5} < 3\) \hfill [\(1 < x < 4\)]
\item \(\abs{x+2} = 2x+3\) \hfill [\(-1\)]
\item \(\abs{x^2-3} > 6\) \hfill [\(x < -3 \sor x > +3\)]
\item \(\abs{1-x} \leqslant 4 - \sqrt{17}\) \hfill [\(\emptyset\)]
\item \(\abs{x+4} > 0\) \hfill [\(\R\)]
\item \(\abs{1-x} \leqslant 3\) \hfill [\(-2 \leqslant x \leqslant 4\)]
\item \(\dfrac{\abs{x-1}-2}{4-x} < 0 \) \hfill [\(-1 < x < 3 \sor x > 4\)]
\end{enumeratea}
\end{multicols}
\end{esercizio}
