% (c) 2016 Daniele Zambelli daniele.zambelli@gmail.com
% (c) 2016 Elisabetta Campana

% \begin{wrapfloat}{figure}{r}{0pt}
% \includegraphics[scale=0.35]{img/fig000_.png}
% \caption{...}
% \label{fig:...}
% \end{wrapfloat}
% 
% \begin{center} \input{\folder lbr/fig000_.pgf} \end{center}

% \subsection{Funzioni, Equazioni e disequazioni con valore assoluto}
% \label{sec:irvalass_}

\chapter{Complementi di algebra}

\section{Equazioni di grado superiore al secondo}
\label{sec:irvalass_supsec}

A questo punto siamo in grado di risolvere equazioni di primo e secondo 
grado. Impareremo ora come comportarci nel caso, più generale, di 
equazioni polinomiali di grado superiore.

\subsection{Equazioni che si possono risolvere tramite scomposizione}
% \label{subsec:irvalass_supsec_scomp}

Pensiamo, ad esempio, di dover risolvere un'equazione polinomiale del 
tipo \(P(x)=0\), lo sappiamo già fare? Certo, se possiamo applicare la 
tecnica della scomposizione (raccoglimenti totali e parziali, prodotti 
notevoli, regola di Ruffini...). Scomponiamo il polinomio \(P(x)\)  
scrivendolo come  prodotto di più polinomi di grado minore e poi, 
mediante la legge dell'annullamento del prodotto, risolviamo le equazioni 
che abbiamo trovato.

Prima di procedere ricordiamo la legge dell'annullamento del prodotto:

\begin{definizione}
Legge dell'annullamento del prodotto: il prodotto di due o più fattori è 
uguale a zero quando almeno uno dei fattori è nullo.   
\end{definizione}

\begin{esempio}
\(x^3-4x=0\)
\begin{center}
\begin{tabular}{ll}
raccogliamo a fattore comune la \(x\): & \(x(x^2-4)=0\)\\
applichiamo la legge dell'annullamento del prodotto: & \(x=0 \sor (x^2-4)=0\)\\
risolviamo le due equazioni ottenute: & \(x=0 \sor x=\pm 2\)
\end{tabular}
\end{center}
\end{esempio}

\begin{esempio}
\(3x^3-2x^2-3x+2=0\)
\begin{center}
\begin{tabular}{ll}
facciamo un primo raccoglimento parziale: & \(3x(x^2-1)-2(x^2-1)=0\)\\
raccogliamo a fattore comune la \(x\): & \((x^2-1)(3x-2)=0\)\\
applichiamo la legge dell'annullamento del prodotto: & \(x^2-1=0 \sor 3x-2=0\)\\
risolviamo le due equazioni ottenute: \(x=\pm 1\sor x=\frac{2}{3}\)& 
\end{tabular}
\end{center}
\end{esempio}

\begin{esempio}
\(x^4-3x^3+2x^2=0\)
\begin{center}
\begin{tabular}{ll}
raccogliamo a fattore comune la \(x\): & \(x^2(x^2-3x+2)=0\)\\
applichiamo la legge dell'annullamento del prodotto: & \(x^2=0 \sor x^2-3x+2=0\)\\
risolviamo le due equazioni ottenute: & \(x=0 \sor x=1 \sor x=2\)
\end{tabular}
\end{center}
\end{esempio}

\subsection{Equazioni monomie}

\begin{definizione}[Equazione monomia]
Un'equazione si dice \textbf{monomia} se può essere scritta nella forma:
\[ax^n=0\]      
\end{definizione}

Ricordando che \[x^n=0\]
equivale a \[\underbrace{x\cdot x\cdot x\cdot x \dots \cdot x}_{\text{\(n\) 
volte}}=0\]
e per la legge dell'annullamento del prodotto, abbiamo \[x=0 \sor x=0 
\sor x=0 \sor x=0 \sor \cdots \sor x=0 \]
Si può dire quindi che l'equazione \(ax^n=0\) ha \(n\) soluzioni 
\emph{coincidenti} uguali a 0.

\subsection{Equazioni binomie}

\begin{definizione}[Equazione binomia]
Un'equazione si dice \textbf{binomia} se può essere scritta nella forma:
\[ax^n+b=0\]
dove \(n\) è un numero \emph{intero positivo} e \(a\) e \(b\)  \emph{numeri 
reali}  non nulli.  
\end{definizione}

Il numero delle soluzioni dipende da \(n\) e dal segno di \(a\) e \(b\).
Infatti se riscriviamo l'equazione
\[ax^n+b=0\]
e risolviamo rispetto a \(x^n\), otteniamo l'equazione equivalente:
\[x^n=-\frac{b}{a}\]

\begin{itemize}
\item se \(n\) è \textbf{pari} l'equazione ammette soluzioni reali 
solo se  \(-\frac{b}{a}>0\) e le soluzioni saranno date da:
\[x=\pm \sqrt[n]{-\frac{b}{a}}\]
se \(-\frac{b}{a}<0\) l'equazione non ammette radici reali in 
quanto non esiste la radice di indice pari di un numero negativo.
\item se \(n\) \textbf{dispari} l'equazione ammette sempre 
una sola soluzione reale data da  \[x= \sqrt[n]{-\frac{b}{a}}\] 
\end{itemize}

\begin{esempio}
\(8x^3+1=0\)
\begin{center}
\begin{tabular}{ll}
Risolviamo rispetto a \(x^3\) e otteniamo l'equazione equivalente: & 
\(x^3=-\frac{1}{8}\)\\
estraiamo quindi la radice cubica: & \(x=\sqrt[3]{-\frac{1}{8}} = -\frac{1}{2}\)
\end{tabular}
\end{center}
\end{esempio}

\begin{esempio}
\(4x^2-9=0\)
\begin{center}
\begin{tabular}{ll}
Risolviamo rispetto a \(x^2\) e otteniamo l'equazione equivalente: & 
\(x^2=\frac{9}{4}\)\\
le soluzioni di questa equazione sono due: & 
\(x=\pm \sqrt{\frac{9}{4}}=\pm\frac{3}{2}\)
\end{tabular}
\end{center}
\end{esempio}    

\begin{esempio}
\(2x^2 +50=0\)
\\[4pt]
Risolviamo rispetto a \(x^2\) e otteniamo l'equazione equivalente: \(x^2=-\frac{50}{2}=-25\)
\\[4pt]
Poiché nessun numero reale elevato alla seconda dà un risultato 
negativo: \emph{no soluzioni reali}
\end{esempio}    

\begin{esempio}
\(-\frac{2}{3}x^6+2=0\)
\\[4pt]
Risolviamo rispetto a \(x^6\) e otteniamo l'equazione equivalente: \quad 
\(x^6=\frac{-2}{-\frac{2}{3}}=3\)\\[4pt]
le soluzioni di questa equazione sono perciò: \(x=\pm \sqrt[6]{3}\)
\end{esempio}

\subsection{Equazioni trinomie particolari}

\begin{definizione}[Equazione trinomia particolare]
Un'equazione si dice \textbf{trinomia particolare} se può essere scritta nella 
forma:
\[ax^{2n}+bx^n+c=0\]
dove \(n\) è un numero intero positivo e \(a\), \(b\) e \(c\)  numeri reali  non 
nulli. 
\end{definizione}

Possiamo distinguere tre casi:
\begin{itemize}
\item se \(n=1\) l'equazione diventa \(ax^{2}+bx+c=0\), si riduce 
quindi a un'equazione di \(2^\circ\) grado.
\item se \(n=2\) l'equazione diventa \(ax^{4}+bx^2+c=0\), con \(a\), 
\(b\) e \(c\) numeri reali  non nulli e viene chiamata \textbf{equazione 
biquadratica}.
\item se \(n\geq 2\) le equazioni trinomie si possono ricondurre a 
equazioni di secondo grado tramite una semplice sostituzione:\\[4pt]
ponendo \(x^n=z\), e quindi \(x^{2n}=(x^n)^2=z^2\), si ottiene:
\[ax^{2n}+bx^n+c=0 \quad \longrightarrow \quad az^{2}+bz+c=0\]
ora non resta che risolvere questa equazione. Se essa non ammette  
soluzioni reali, neppure quella di partenza ne avrà, mentre se ammette soluzioni reali, ad esempio \(z_1\) e \(z_2\), le soluzioni 
dell'equazione originaria saranno le soluzioni delle due equazioni 
binomie:
\begin{center}
  \(x^n=z_1\) e \(x^n=z_2\).
\end{center}
\end{itemize}


\begin{comment}
\begin{esempio}

\begin{center}
\begin{tabular}{ll}
: & \\
: & \\
: & 
\end{tabular}
\end{center}
\end{esempio}
\end{comment}

\begin{esempio}
  \(x^6+9x^3+8=0\)
\\[4pt]
  Ponendo \(x^3=z\), l'equazione diventa: \(z^2+9z+8=0\)
\\[4pt]
  Risolviamo questa equazione: 
  \(\tonda{z+8}\tonda{z+1}=0 \sRarrow z_1=-8 \sor z_2=-1\)
\\[4pt]
  Ritorniamo ora alla variabile \(x\):
  \begin{itemize}
    \item \(x^3=z_1=-8 \sRarrow x_1=\sqrt[3]{-8}=-2\) 
    \item \(x^3=z_2=-1 \sRarrow x_2=\sqrt[3]{-1}=-1\)
  \end{itemize}
\end{esempio}

\begin{esempio}
  \(x^4+x^2-6=0\)
  \\[4pt]
  Ponendo \(x^2=z\), l'equazione diventa \(z^2+z-6=0\) 
  \\[4pt]
  Questa equazione ha soluzioni:
  \(\tonda{z+3}\tonda{z-2}=0 \sRarrow z_1=-3 \sor z_2=+2\)
  \\[4pt]
  Ritorniamo ora alla variabile \(x\):
  \begin{itemize}
    \item \(x^2=z_1 \sRarrow x^2=-3\) che non ha soluzioni reali
    \item \(x^2=z_2 \sRarrow x^2=+2\) che soluzioni \(x_{1,2}=\pm \sqrt[]{2}\)
  \end{itemize}
\end{esempio}

\begin{esempio}
  \(x^{14}-10x^7+25=0\) 
  \\[4pt]
  Ponendo \(x^7=z\), l'equazione diventa \(z^2-10z+25=0\)
  \\[4pt]
  Questa equazione ha un'unica soluzione: \(z_1=5\). Ritornando in \(x\) si ottiene: \(x = \sqrt[7]{5}\).
\end{esempio}


\section{Equazioni e disequazioni irrazionali}
\label{sec:irvalass_irraz}

\subsection{Equazioni irrazionali}

Consideriamo le seguenti equazioni:
\begin{enumerate}
 \item \(\sqrt{x-4} = \dfrac{x-4}{2}\)
 \item \(\sqrt[3]{x^{2}+1} -2=0\)
\end{enumerate}

come si può osservare tali equazioni contengono un radicale nel cui radicando 
compare  l'incognita.
Queste equazioni si dicono irrazionali.
\begin{definizione}[Equazione irrazionale]
 Un'\textbf{equazione irrazionale} è un'equazione algebrica in cui l'incognita 
compare all'interno del radicando di uno o più radicali.
\[\sqrt[n]{A(x)} = B(x)\]
\end{definizione}

\vspace{5pt}

\noindent Sono quindi equazioni irrazionali: 
\(\sqrt{2x+5} = 3\tonda{x-1}\) \quad e \quad \(\sqrt{2x} = \tonda{3x+4}\)  \\
mentre non lo sono: \(\sqrt{2} = \tonda{3x-2}\) \quad e \quad 
\(x^2 + \sqrt{3} = 3\) 
\\[4pt]
Per risolvere un'equazione irrazionale si cerca, tramite opportuni elevamenti a 
potenza, di ricondursi ad un'equazione razionale equivalente.
\\[4pt]
Ricordiamo che data un'equazione \(A(x)=B(x)\), se eleviamo ambo i membri alla 
\(n\), dove \(n\) è un numero intero positivo, e consideriamo l'equazione 
\(\tonda{(A(x)}^n=\tonda{B(x)}^n\) si possono verificare i seguenti casi:
\begin{itemize}
 \item 
 se n è dispari, essa è equivalente a quella data.
Infatti nel caso n sia un numero dispari è sufficiente elevare entrambi i 
membri 
dell'equazione allo stesso indice, ottenendo  un'equazione razionale che 
ammette 
le stesse soluzioni di quella di partenza.
 \item 
 se n è pari, essa ha come soluzioni, oltre a quelle di \(A(x)=B(x)\), anche 
quelle di \(A(x) = -B(x)\)
Quindi, per risolvere equazioni di questo tipo è sufficiente tenere presente il 
fatto che, elevando ambo i membri alla \(n\) si ottiene un'equazione che 
oltre alle soluzioni di quella data può ammetterne anche altre.
\end{itemize}

\begin{esempio}
 \(\sqrt[3]{x^3-x^2-x+25} = x+1\)

\vspace{4pt}
 
 elevando ambo i membri al cubo si ottiene: \(x^3-x^2-x+25 = x^3+3x^2+3x+1\)
 
 semplificando si ottiene: 
 \(x^2+x-6=0\) 
 
 e quindi, scomponendo il polinomio:
 \(\tonda{x+3}\tonda{x-2}=0 \sRarrow x_1 = 2 \sor x_2=-3\)
 
 possiamo verificare le soluzioni trovate:
 
 \(\sqrt[3]{2^3-2^2-2+25} = 2+1 \sRarrow \sqrt[3]{8-4-2+25} = 3 \sRarrow 
 \sqrt[3]{27} = 3\)
 
 \(\sqrt[3]{\tonda{-3}^3-\tonda{-3}^2-\tonda{-3}+25} = \tonda{-3}+1 \sRarrow 
 \sqrt[3]{-27-9+3+25} = -2 \sRarrow \sqrt[3]{-8} = -2\)
\end{esempio}

\begin{esempio}
 \(\sqrt{x+4} = 3\)

\vspace{4pt}
  
 elevando ambo i membri al quadrato si ottiene: \(x+4 = 9\)
 
 che ha come soluzione: \(x=5\)
 
 possiamo verificare la soluzione:
 \(\sqrt{5+4} = 3 \sRarrow \sqrt{9} = 3\)
\end{esempio}

\begin{esempio}
 \(\sqrt{x-3} = x-5\)
 
\vspace{4pt}
 
 elevando ambo i membri al quadrato si ottiene: \(x-3 = x^2-10x+25\)
 
 semplificando si ottiene: \(x^2-11x+28=0\)
 
 scomponendo il polinomio si ottiene:
 \(\tonda{x-7}\tonda{x-4}=0 \sRarrow x_1 = 7 \sor x_2 = 4\)
 
 ora verifichiamo le soluzioni trovate:
 \(\sqrt{7-3} = 7-5 \sRarrow \sqrt{4} = 2\)
 
 \(\sqrt{4-3} = 4-5 \sRarrow \sqrt{1} = -1\) (soluzione non accettabile)\\
 
 Quindi l'equazione ha una sola soluzione. Infatti \(7\) è una soluzione accettabile,
 mentre \(4\) non è accettabile, ovvero è soluzione dell'equazione razionale ottenuta,
 ma non dell'equazione irrazionale di partenza.
\end{esempio}

\begin{esempio}
 \(\sqrt{2x+5} = 3 \tonda{x-1}\)
 
\vspace{4pt}
 
 elevando ambo i membri al quadrato si ottiene: \(2x+5 = 9\tonda{x-1}^2\)
 
 semplificando si ottiene: \(2x+5 = 9x^2-18x+9 \sRarrow 9x^2-20x+4=0\)
 
 che dà come soluzioni:
 \(x_{1,2} = \dfrac{10 \pm \sqrt{100-36}}{9} = \dfrac{10 \pm 8}{9} \sRarrow 
 x_1 = \dfrac{2}{9} \sor x_2 = 2\)
 
 ora verifichiamo le soluzioni trovate:
 \(\sqrt{2\cdot 2+5} = 3 \tonda{2-1} \sRarrow \sqrt{9} = 3\) 
 
 \(\sqrt{2\cdot \dfrac{2}{9}+5} = 3 \tonda{\dfrac{2}{9}-1} \sRarrow 
   \sqrt{\dfrac{49}{9}} = -\dfrac{7}{9}\) (soluzione non accettabile)
   
\end{esempio}

% Per riconoscere quali soluzioni siano valide abbiamo due possibilità:
% \begin{enumerate}
%  \item Verificare una per una le soluzioni dell'equazione razionale ottenuta 
% elevando i membri dell'equazione irrazionale e vedere quali di queste 
% verificano anche l'equazione di partenza.
%  \item porre delle condizioni e accettare solo le soluzioni che le soddisfano.
% \end{enumerate}

Quando eleviamo entrambi i membri ad un esponente \emph{pari} otteniamo 
un'equazione che può avere delle soluzioni che non sono soluzioni 
dell'equazione di partenza (soluzioni non accettabili). Per individuare le soluzioni dell'equazione data 
possiamo verificare una per una tutte le soluzioni dell'equazione razionale e 
vedere quali di queste sono anche soluzioni di quella irrazionale, come abbiamo 
fatto negli esempi precedenti.

\vspace{5pt}

Un altro metodo consiste nel porre delle condizioni prima di eliminare le radici 
e accettare poi solo le soluzioni che le soddisfano.\\

È importante ricordare che:
\begin{enumerate}
 \item la radice pari di un numero negativo non è un numero reale;
 \item Se esiste, la radice pari di un numero è sempre positiva.
\end{enumerate}

Perciò, quando passiamo dall'equazione \(\sqrt[n]{A(x)} = B(x)\) 
all'equazione \(A(x) = \tonda{B(x)}^n\)  dobbiamo aggiungere le due 
informazioni che abbiamo perduto in questo passaggio, quindi l'equazione di 
partenza è equivalente al seguente sistema:
\[\sistema{A(x) \geqslant 0 & \text{condizione di realtà} \\
           B(x) \geqslant 0 & \text{condizione di positività} \\
           A(x) = \tonda{B(x)}^n}
\]
Osservando il sistema precedente si può notare che la 
prima condizione può essere tralasciata perché è una conseguenza 
dell'ultima, infatti se \(A(x)\) è uguale a 
\(\tonda{B(x)}^n\) con \(n\) pari, allora senz'altro \(A(x)\) è positivo.

\vspace{5pt}

In conclusione vale la seguente identità:
\[\sqrt[n]{A(x)} = B(x) \qquad \sLRarrow \qquad \sistema{B(x) \geqslant 0 \\
           A(x) = \tonda{B(x)}^n}
\]

\begin{esempio}
Riprendiamo l'equazione \(\sqrt{2x+5} = 3 \tonda{x-1}\)
 \\[4pt]
 essa è equivalente al sistema: \(\sistema{3 \tonda{x-1} \geqslant 0 \\
           2x+5 = 9\tonda{x-1}^2} \quad \Rightarrow \quad \sistema{x \geqslant 1 \\
           9x^2-20x+4=0}\) \\[4pt]
 che dà come soluzioni:
 \(x_{1,2} = \dfrac{10 \pm \sqrt{100-36}}{9} = \dfrac{10 \pm 8}{9} \sRarrow 
 x_1 = \dfrac{2}{9} \;\text{(S.N.A.)} \quad  x_2 = 2 \;\text{(S.A.)}\)\\
 
quindi l'equazione ha una soluzione accettabile (S.A.) e una non accettabile (S.N.A.)
 
\end{esempio}

Certi casi semplici possono essere risolti al volo senza particolari calcoli;
tutte le seguenti equazioni irrazionali non hanno soluzioni reali.

\begin{enumerate}
 \item \(\sqrt{5x +8} = -7\) 
 \hfill perché: . . . . . . . . . . . . . . . . . . . . . . . . . . . . . . . .
 \item \(\sqrt{x-4} = -x^2-3\) 
 \hfill perché: . . . . . . . . . . . . . . . . . . . . . . . . . . . . . . . .
 \item \(\sqrt{-x^2} = 9\)  
 \hfill perché: . . . . . . . . . . . . . . . . . . . . . . . . . . . . . . . .
 \item \(\sqrt{4x-5} = -\tonda{x-7}^2\) 
 \hfill perché: . . . . . . . . . . . . . . . . . . . . . . . . . . . . . . . .
 \item \(\sqrt{-\tonda{2x+8}^2} = 14\) 
 \hfill perché: . . . . . . . . . . . . . . . . . . . . . . . . . . . . . . . .
 \item \(\sqrt{-2x^2 + 3x -4} = -x-4\) 
 \hfill perché: . . . . . . . . . . . . . . . . . . . . . . . . . . . . . . . .
\end{enumerate}

\subsection{Disequazioni irrazionali}
\label{sec:irvalass_irrazionali}

Se le disequazioni contengono l'incognita sotto una radice si dicono 
disequazioni \emph{irrazionali}.

Vediamo i casi che si possono presentare.

\subsubsection{Radici con indice dispari}

Le disequazioni irrazionali del tipo:

\[\sqrt[n]{A(x)} \leqslant B(x) \text{ o } \sqrt[n]{A(x)} \geqslant B(x)\]

con \(n\) dispari si risolvono semplicemente elevando ambo i membri della 
disequazione allo stesso indice, ottenendo una disequazione razionale che 
ammette le stesse soluzioni di quella di partenza.

\begin{esempio}
\(\sqrt[3]{-6x^2 +12x +1} \leqslant x -2\)
\begin{center} \begin{tabular}{rl}
elevando ambo i membri al cubo si ottiene: & 
\(-6x^2 +12x +1 \leqslant x^3 -6x^2 +12x -8\)\\
semplificando si ottiene: & \(x^3 -9 \geqslant 0\)\\
che dà come soluzioni: & \(x \geqslant \sqrt[3]{9}\)
\end{tabular} \end{center}
\end{esempio}

\begin{esempio}
\(\sqrt[3]{x^3 -3x +2} > x -1\)
\begin{center} \begin{tabular}{rl}
elevando ambo i membri al cubo si ottiene: & 
\(x^3 -3x +2 > x^3 -3x^2 +3x -1\)\\
semplificando si ottiene: & \(3 \tonda{x -1}^2 > 0\)\\
che dà come soluzione: & \( x \neq 1\) 
\end{tabular} \end{center} 
\end{esempio}

\subsubsection{Radici con indice pari}

In questo testo ci limiteremo al caso \(n=2\), ma il caso più 
generale si affronta in modo analogo. Le disequazioni irrazionali del tipo:

\[\sqrt[n]{A(x)} \leqslant B(x) \;\text{ o }\; \sqrt[n]{A(x)} \geqslant B(x)\]

sono equivalenti a un sistema di disequazioni. Possiamo 
distinguere due casi.

\paragraph{Primo caso: \(\sqrt{A(x)} \leqslant B(x)\)}
~

Osservazioni: 
\begin{enumerate} 
 \item Il radicando, deve sempre essere maggiore o uguale a zero. 
Per la condizione di realtà (C.R.) dovrà quindi essere \({A(x)} \geqslant 0\).
 \item Se il radicando è positivo, anche la radice è definita e sarà 
positiva, quindi anche \({B(x)} \geqslant 0\)
 \item Se i membri della disequazione sono entrambi positivi e il primo è 
minore 
del secondo allora anche il quadrato del primo membro deve essere minore del 
quadrato del secondo membro: 
\(\quadra{\sqrt{A(x)}}^2 \leqslant \quadra{B(x)}^2\).
\end{enumerate}

Tradotte in simboli, queste osservazioni producono la seguente equivalenza tra 
la disequazione irrazionale e un sistema di disequazioni razionali:

\[\sqrt{A(x)} \leqslant B(x) \sLRarrow 
  \sistema{A(x) \geqslant 0 \\ 
           B(x) \geqslant 0 \\ 
           A(x) \leqslant \quadra{B(x)}^2}\]

% \newpage %-----------------------------------------

\begin{esempio}
 \(\sqrt{x^2 -4} -4 \leqslant x\)
\begin{center} \begin{tabular}{rl}
riducendo in forma normale: & \(\sqrt{x^2 -4} \leqslant x +4\) \\ [12pt]
equivalente al sistema razionale: &  
\(\sistema{x^2 -4 \geqslant 0 \\ 
           x +4 \geqslant 0 \\ 
           x^2 -4 \leqslant x^2 + 8x +16}\) \\ \\
che si riduce a: &  
\(\sistema{x^2 -4 \geqslant 0 \\ 
           x +4 \geqslant 0 \\ 
           -8x -20 \leqslant 0} \sRarrow 
  \sistema{x \leqslant -2 \sor x \geqslant +2 \\ 
           x \geqslant -4 \\ 
           x \geqslant -\frac{5}{2}}\) \\ \\
che dà come soluzioni: & 
\(-\dfrac{5}{2} \leqslant x \leqslant -2 \sor x \geqslant 2\)
\end{tabular} \end{center}
\end{esempio}


\paragraph{Secondo caso: \(\sqrt{A(x)} \geqslant B(x)\)}
~

Abbiamo le seguenti due possibilità: 
\begin{enumerate} 
 \item Se \(B(x) < 0\) per verificare la disequazione basta che la 
radice esista, perché essendo positiva sarà senz'altro maggiore di un numero 
negativo.
 \item Se invece \(B(x)\geqslant 0\) allora il radicando deve 
essere maggiore o uguale al suo quadrato e, in questo caso, verifica anche la 
condizione di esistenza della radice.
\end{enumerate}
Tradotte in simboli, queste osservazioni producono i seguenti due sistemi di 
disequazioni razionali:

\[\sqrt[n]{A(x)} \geqslant B(x) \sLRarrow 
  \sistema{B(x) < 0 \\ 
           A(x) \geqslant 0} \sor 
  \sistema{B(x) \geqslant 0 \\ 
           A(x) \geqslant \quadra{B(x)}^2}\]

\begin{esempio}
 \(\sqrt{4x^2 +3x -1} -2x > -3\)
\begin{center} \begin{tabular}{rl}
riducendo in forma normale: & \(\sqrt{4x^2 +3x -1} > 2x -3\) \\ [12pt]
equivalente all'unione dei sistemi razionali: &  
\(\sistema{2x -3 < 0 \\ 
           4x^2 +3x -1 \geqslant 0} \sor 
  \sistema{2x -3 \geqslant 0 \\ 
           4x^2 +3x -1 > 4x^2 -12x +9}\) \\ \\
che si riduce a: &    
\(\sistema{2x -3 < 0 \\ 
           4x^2 +3x -1 \geqslant 0} \sor 
  \sistema{2x -3 \geqslant 0 \\ 
           15x -10 > 0}\) \\ \\
la soluzione del primo sistema è: & 
\(x \leqslant -1 \sor \dfrac{1}{4} \leqslant x < \dfrac{3}{2}\) \\ \\
la soluzione del secondo sistema è: & 
\(x \geqslant \dfrac{3}{2}\)
\end{tabular} \end{center}
La soluzione della disequazione è data dall'unione delle soluzioni,cioé: 
\( x \leqslant -1 \sor x \geqslant -\dfrac{1}{4}\)
\end{esempio}


\section{Equazioni con valori assoluti}
\label{sec:irvalass_valass}

Per risolvere un'equazione nella quale compare il valore assoluto di qualche 
termine , dobbiamo aver chiaro cosa significa \textbf{valore assoluto di un 
numero reale}.

\subsection{Definizione di valore assoluto}

Si definisce valore assoluto (o modulo) di un numero \(x\), e si indica con 
\(|x|\), 
una funzione che associa a \(x\) un numero reale non negativo. Infatti se \(x\) è 
un 
numero reale, il suo valore assoluto è \(x\) stesso se \(x\) è non negativo, è \(-x\) 
se \(x\) è negativo.\\
In simboli:
\[|x|=\left\lbrace 
\begin{array}{lcl}
x & \text{se} & x\geq 0 \\
-x & \text{se} & x< 0 \\
\end{array}
\right. 
\]
se riportiamo il numero \(x\) sulla retta dei numeri reali, il valore assoluto di 
\(x\), \(|x|\), non è altro che la distanza del punto che rappresenta \(x\), 
dall'origine 0.

\begin{figure}[h]
\begin{center}
\begin{inaccessibleblock}[TODO]
\includegraphics[width=0.8\linewidth]{img/imm1} %[scale=0.35]{img/fig001.png}
\end{inaccessibleblock}
\caption{Retta}
\label{fig:abs_imm1}
\end{center}
\end{figure}

\begin{comment}
%figura:
\begin{figure}[h]
\centering
\includegraphics[width=0.9\linewidth]{imm1}
%\caption{}
\label{fig:imm1}
\end{figure}


% esempio
\begin{esempio}
 
\begin{center} \begin{tabular}{rl}
elevando ambo i membri al cubo si ottiene: & 
 \\
semplificando si ottiene: &  
che dà come soluzioni: &   
\end{tabular} \end{center}
\end{esempio}
\end{comment}

È logico che, per come l'abbiamo definito,\textit{ il valore assoluto di un 
numero reale è sempre non negativo}.

Esempi:
\begin{enumerate}
\item[a.]
\(|-2|=-(-2)=2\)
\item[b.]
\(|+4|=+4\)
\item[b.]
\(|-2+\sqrt{3}|=-(-2+\sqrt{3})=2-\sqrt{3}\)
\end{enumerate} 
Nel caso in cui al posto di \(x\) ci fosse una espressione algebrica  \(P(x)\) si 
definisce il valore assoluto di \(P(x)\), nel seguente modo:
\[|P(x)|=\left\lbrace 
\begin{array}{lcl}
P(x) & \text{se} & P(x)\geq 0 \\
-P(x) & \text{se} & P(x)< 0 \\
\end{array}
\right. 
\]
Vediamo subito alcuni esempi:
\begin{enumerate}
\item
\(|x-2|=\left\lbrace 
\begin{array}{lcl}
x-2 & \text{se} & x-2\geq 0 \\
-(x-2) & \text{se} & x-2< 0 \\
\end{array}
\right.
\text{ossia: }
|x-2|=\left\lbrace 
\begin{array}{lcl}
x-2 & \text{se} & x\geq 2 \\
-x+2 & \text{se} & x<2 \\
\end{array}
\right.
\)
\item
\(|x^2-3x+2|=\left\lbrace 
\begin{array}{lcl}
x^2-3x+2 & \text{se} & x^2-3x+2\geq 0 \\
-(x^2-3x+2) & \text{se} & x^2-3x+2< 0 \\
\end{array}
\right.\)
che si può risolvere e scrivere:\\[0.2cm]
\(
|x^2-3x+2|=\left\lbrace 
\begin{array}{lcl}
x^2-3x+2 & \text{se} & x\leq 1 \vee x\geq 2 \\
-x^2+3x-2 & \text{se} & 1<x<2 \\
\end{array}
\right.
\)
\end{enumerate}

\subsection{La funzione ``valore assoluto''}

La funzione \(y=|x|\) si chiama \emph{funzione valore assoluto} ed ha il grafico di Figura \ref{fig:abs_imm2}.
\begin{figure}[h]
\begin{center}
\begin{inaccessibleblock}[TODO]
\includegraphics[width=0.5\linewidth]{img/imm2} %[scale=0.35]{img/fig001.png}
\caption{Grafico della funzione \(y = |x|\)}
\label{fig:abs_imm2}
\end{inaccessibleblock}
% \caption{Retta}
\end{center}
\end{figure}
% \begin{figure}[h]
%         \centering
%         \includegraphics[width=0.9\linewidth]{imm2}
%         %\caption{}
%         \label{fig:imm1}
% \end{figure}
%\begin{figure}[h]
%\begin{center}
%\begin{inaccessibleblock}[TODO]
%\centering
%\includegraphics[width=0.5\linewidth]{img/imm3} %[scale=0.35]{img/fig001.png}
%\end{inaccessibleblock}
% % \caption{Retta}
% \label{fig:abs_imm3}
% \end{center}
% \end{figure}
% % \begin{figure}[h]
% %         \centering
% %         \includegraphics[width=0.9\linewidth]{imm3}
% %         %\caption{}
% %         \label{fig:imm1}
% % \end{figure}
Si tratta di una funzione pari, ovvero simmetrica rispetto all'asse delle \(y\). In pratica non è altro che
la retta bisettrice del primo e terzo quadrante, a cui viene applicata la trasformazione \(y = f\tonda{|x|}\)

\subsection{Proprietà del valore assoluto}
Il valore assoluto gode delle seguenti proprietà:
\begin{itemize}
        \item \(|x+y|\leq |x|+|y| \quad \forall x,y \in \mathbb{R}\) \quad (\emph{disuguaglianza triangolare})
        \item \(|x\cdot y|=|x|\cdot |y| \quad  \forall x,y \in \mathbb{R}\)
        \item \(\left|\frac{x}{y} \right| =\frac{|x|}{|y|} \quad \forall x \in \mathbb{R}, 
\forall y \in \mathbb{R}_0\)
        \item \(|x|=|y| \longleftrightarrow x=\pm y \quad  \forall x,y \in \mathbb{R}\)
        \item \(|x|=|-x|, \forall x \in \mathbb{R}\)
        \item \(|x|^2=x^2, \forall x \in \mathbb{R}\)
        \item \(\sqrt{x^2}=|x|, \forall x \in \mathbb{R}\)
\end{itemize}

\subsection{Equazioni con il valore assoluto}
Come facciamo a risolvere equazioni nelle quali l'incognita compare all'interno 
di qualche valore assoluto? Vediamo alcuni esempi:
\paragraph{1\textdegree~caso} L'equazione si presenta nella forma:  
\fbox{\(|P(x)|=k\)} \quad \((k\in \mathbb{R})\)
\begin{esempio} 
\(|x-5|=-2\) \\[4pt]
L'equazione non ha soluzione perché il valore assoluto di un polinomio è sempre non negativo.
\end{esempio}
\begin{esempio}  
\(|x-5|=0\) \\[4pt]
In questo caso possiamo risolvere l'equazione ricordando che il valore assoluto di un numero è zero se e solo se 
lo è il numero stesso, quindi \(x=5\).
\end{esempio}
\begin{esempio}  
\(|x-5|=2\) \\[4pt]
Possiamo risolvere l'equazione ricordando che  il valore assoluto di un numero è uguale 
ad un numero positivo, \(|P(x)|=k>0\), se e solo se \(P(x)=\pm k\) (proprietà 4 del 
valore assoluto) l'equazione è equivalente a:
\[x-5=\pm 2\]
che corrisponde a scrivere:
\[x-5= 2 \vee x-5=-2\]
che ha come soluzioni:
\[x=7 \vee x=3\]
\end{esempio}

\noindent \textbf{In generale}: data un'equazione del tipo \(|P(x)|=k\) si ha che:
\begin{itemize}[nosep]
 \item se \(k < 0\)  l'equazione non ha alcuna soluzione reale
 \item se \(k = 0\)  l'equazione è equivalente a  \(P(x) = 0\)
 \item se \(k > 0\)   l'equazione è equivalente a  \(P(x)=k \vee P(x)=-k\)
\end{itemize}

\begin{esempio} 
\(|2x^2+5x|=0,\) \\[4pt]
L'equazione è equivalente a: \(2x^2+5x=0\)
che ha come soluzioni:
\(x=0 \vee x=-\frac{5}{2}\)
\end{esempio}

\begin{esempio} 
\(|2x^2+5x|=-3\)\\[4pt] Siamo nel caso \(k < 0\) 
quindi l'equazione è \emph{impossibile}
\end{esempio}
        
\begin{esempio} 
\(|2x^2+5x|=3\)\\[4pt] L'equazione è equivalente a:
\(2x^2+5x=3 \vee 2x^2+5x=-3\) \\[4pt]
La prima ha soluzioni: \(x=\frac{1}{2} \vee x=-3\) mentre la seconda: \(x=-\frac{3}{2} \vee x=-1\)\\[4pt]
Pertanto l'insieme delle soluzioni dell'equazione di partenza è: 
\(S=\left\lbrace \frac{1}{2},-\frac{3}{2},-1,-3\right\rbrace \)
\end{esempio}

\paragraph{2\textdegree~caso} L'equazione si presenta nella forma:  
\fbox{\(|A(x)|=|B(x)|\)}\\[4pt]
Ricordando la proprietà 4 dei valori assoluti tale equazione è equivalente a:
\[A(x)=B(x) \quad\vee\quad  A(x)=-B(x)\]

\begin{esempio} \(|3x+1|=|x-1|\), l'equazione è equivalente a
\[3x+1=x-1 \vee 3x+1=-x+1\]
la prima ha soluzione: \(x=-1\)\\
la seconda ha soluzione: \(x=0\)\\
Pertanto l'insieme delle soluzioni dell'equazione di partenza è: 
\(S=\left\lbrace -1,0 \right\rbrace \) 
\end{esempio}
                
\begin{esempio}
E se dovessimo risolvere un'equazione di questo tipo? \(|2x-3|=-|x+1|\) \\[4pt]
Niente paura, è chiaro che l'equazione è impossibile perché \dots
\end{esempio}

\paragraph{3\textdegree~caso} L'equazione si presenta nella forma:  
\fbox{\(|A(x)|=B(x)\)}\\[4pt]
Cosa dobbiamo fare in questo caso? Dobbiamo semplicemente \textquotedblleft 
sciogliere\textquotedblright il valore assoluto!
Se ricordiamo la definizione di \(|A(x)|\) capiamo subito che l'insieme delle 
soluzioni di questa equazione è l'unione degli insiemi delle soluzioni dei 
seguenti sistemi misti:
\[
\left\lbrace 
\begin{array}{l}
A(x)=B(x)\\
A(x)\geq 0\\
\end{array}
\right.
\vee
\left\lbrace 
\begin{array}{l}
-A(x)=B(x)\\
A(x)< 0\\
\end{array}
\right.
\]
Facciamo subito un esempio per capire meglio:
\begin{esempio}  \(|2x-1|=x+3\)\\[4pt] L'equazione è equivalente a:
\(
\left\lbrace 
\begin{array}{l}
2x-1=x+3\\
2x-1\geq 0\\
\end{array}
\right.
\vee
\left\lbrace 
\begin{array}{l}
-(2x-1)=x+3\\
2x-1< 0\\
\end{array}
\right.
\) \\
Il primo sistema ha soluzione \(x=4\) e il secondo \(x=-\frac{2}{3}\)\\[4pt]
L'insieme delle soluzioni dell'equazione di partenza è pertanto: 
\(S=\left\lbrace 4,-\frac{2}{3} \right\rbrace \).
\end{esempio}

\subsection{Disequazioni con i valori assoluti}

Vogliamo risolvere alcune semplici disequazioni con valori assoluti, 
riconducibili ai casi:
\[|P(x)|>k \quad; \quad |P(x)|<k\quad; \quad |P(x)|\geq k\quad; \quad |P(x)|\leq k\]


\paragraph{1\textdegree~caso} La disequazione si presenta nella forma:  
\fbox{\(|P(x)|\leq k\)} o \fbox{\(|P(x)|\geq k\)} con \(k\leq 0\)

\vspace{4pt}

La soluzione della disequazione \(|P(x)|\leq k\) con \(k\leq 0\) è molto semplice 
se si ricorda che il valore assoluto di una espressione algebrica è sempre non 
negativo e quindi non potrà mai essere minore di un numero negativo. Quindi, ad esempio:

\begin{esempio}  
\(|2x-1|\leq -3\) \\[4pt] La disequazione è impossibile: un numero non negativo non può essere minore di uno negativo.
\end{esempio}
\begin{esempio} \(|3x-2|\leq 0\) \\[4pt] In questo caso poiché il 
valore assoluto di un numero non è mai negativo, la disequazione è verificata 
se e solo se \(|3x-2|=0\), cioè \(x=\frac{2}{3}\).
\end{esempio}

La disequazione \(|P(x)|\geq k\) con \(k\leq 0\), sarà invece verificata \(\forall x 
\in \mathbb{R}\).

\begin{esempio} \(|x^2-1|\geq -3\)\\[4pt] La disequazione è 
verificata \(\forall x \in \mathbb{R}\): un numero non negativo è sempre maggiore di uno negativo.
\end{esempio}

\paragraph{2\textdegree~caso} La disequazione si presenta nella forma:  
\fbox{\(|P(x)|\leq k\)} con \(k> 0\)

\begin{esempio}  \(|x-1|\leq 2\), ricordando che il valore 
assoluto di \(x\) non è altro che la distanza del punto che rappresenta 
\(x\) dall'origine 0 della retta dei numeri reali, allora il valore assoluto di 
un 
numero è \(\leq 2\) quando quel numero è compreso fra -2 e +2.\\
        Nella figura sono evidenziati tutti i numeri reali la cui distanza 
dall'origine è \(\leq 2\).

\begin{figure}[h]
\begin{center}
\begin{inaccessibleblock}[TODO]
\centering
\includegraphics[width=0.5\linewidth]{img/imm4} %[scale=0.35]{img/fig001.png}
\end{inaccessibleblock}
% \caption{Retta}
\label{fig:abs_imm4}
\end{center}
\end{figure}
%         \begin{figure}[h]
%                 \centering
%                 \includegraphics[width=0.9\linewidth]{imm4}
%                 %\caption{}
%                 \label{fig:imm1}
%         \end{figure}\\

La disequazione precedente è equivalente alla:
\[-2\leq x-1 \leq 2\]
questa doppia disequazione può essere risolta:
\begin{enumerate}
  \item [a)] tramite un sistema:
    \[
    \left\lbrace 
    \begin{array}{l}
    x-1\leq 2\\
    x-1\geq -2\\
    \end{array}
    \right.
    \]
    le cui soluzioni sono:
    \[-1\leq x \leq 3\]
  \item [b)] semplicemente sommando +1 a ciascun anello della catena:
    \[-2+1\leq x-1+1 \leq 2+1\]
    e quindi:
    \[-1\leq x \leq 3.\]
\end{enumerate}
\end{esempio}

\paragraph{3\textdegree~caso} La disequazione si presenta nella forma:  
\fbox{\(|P(x)|\geq k\)} con \(k> 0\)\\

\begin{esempio} \(|x-1|\geq 2\)\\[4pt] Ricordando che il valore 
assoluto di \(x\), \(|x|\), non è altro che la distanza del punto che rappresenta 
\(x\) dall'origine 0 della retta dei numeri reali, allora il valore assoluto di 
un numero è \(\geq 2\) quando quel numero è \(\leq -2\) oppure \(\geq 2\).\\
        Nella figura sono evidenziati tutti i numeri reali la cui distanza 
dall'origine è \(\geq 2\).
\begin{figure}[h]
\begin{center}
\begin{inaccessibleblock}[TODO]
\centering
\includegraphics[width=0.7\linewidth]{img/imm5} %[scale=0.35]{img/fig001.png}
\end{inaccessibleblock}
% \caption{Retta}
\label{fig:abs_imm5}
\end{center}
\end{figure}
%         \begin{figure}[h]
%                 \centering
%                 \includegraphics[width=0.9\linewidth]{imm5}
%                 %\caption{}
%                 \label{fig:imm1}
%         \end{figure}\\
La disequazione precedente è equivalente a: \(x-1\leq -2 \vee x-1 \geq 2\) \\[4pt]
e quindi la soluzione è: \(x\leq -1 \vee x\geq 3.\)
\end{esempio}

\paragraph{4\textdegree~caso} La disequazione si presenta nella forma:  
\fbox{\(|A(x)|\geq B(x)\)} o \fbox{\(|A(x)|\leq B(x)\)}\\[4pt]
Prendiamo il primo caso come riferimento, anche se la stessa trattazione vale in modo simile anche nel secondo.
In questo caso è sufficiente ``sciogliere'' il valore assoluto, come nel caso delle equazioni viste in precedenza.
Utilizzando ancora una volta la definizione di \(|A(x)|\), si può tradurre la disequazione
nell'unione degli insiemi delle soluzioni dei due sistemi seguenti:
\[
\left\lbrace 
\begin{array}{l}
A(x)\geq B(x)\\
A(x)\geq 0\\
\end{array}
\right.
\vee \;
\left\lbrace 
\begin{array}{l}
-A(x)\geq B(x)\\
A(x)< 0\\
\end{array}
\right.
\]

\begin{esempio} \(|2x-4|\geq x+2\)\\[4pt] La disequazione è equivalente a:
\(\left\lbrace 
\begin{array}{l}
2x-4\geq x+2\\
2x-4\geq 0\\
\end{array}
\right.
\vee \;
\left\lbrace 
\begin{array}{l}
-\tonda{2x-4}\geq x+2\\
2x-4< 0\\
\end{array}
\right.\) \\[4pt]
Semplificando i due sistemi si ottiene:
\(\left\lbrace 
\begin{array}{l}
x\geq 6\\
x\geq 2\\
\end{array}
\right.
\vee \;
\left\lbrace 
\begin{array}{l}
x\leq \dfrac{2}{3}\\
x< 2\\
\end{array}
\right.\) \\[4pt]
le cui soluzioni sono, rispettivamente, \(x \geq 6\) e \(x\leq \dfrac{2}{3}\). La soluzione finale della disequazione di partenza
è quindi l'unione delle due soluzioni, ovvero: \(x\leq \dfrac{2}{3} \;\vee \;x \geq 6\).
\end{esempio}

\begin{esempio} \(|x^2-5x+4|\leq x^2-1\)\\[4pt] La disequazione è equivalente a:
\(\left\lbrace 
\begin{array}{l}
x^2-5x+4\leq x^2-1\\
x^2-5x+4\geq 0\\
\end{array}
\right.
\vee \;
\left\lbrace 
\begin{array}{l}
-\tonda{x^2-5x+4}\leq x^2-1\\
x^2-5x+4< 0\\
\end{array}
\right.\) \\[4pt]
Semplificando i due sistemi si ottiene:
\(\left\lbrace 
\begin{array}{l}
x\geq 1\\
x \leq 1 \; \vee \; x \geq 4\\
\end{array}
\right.
\vee \;
\left\lbrace 
\begin{array}{l}
x\leq 1 \; \vee \;x \geq \dfrac{3}{2}\\[6pt]
1< x < 4\\
\end{array}
\right.\) \\[4pt]
La soluzione del primo sistema è \(x \geq 4\), mentre quella del secondo è \(\dfrac{3}{2}\leq x <4\).
La soluzione finale della disequazione di partenza, unione delle due soluzioni trovate, è \(x \geq \dfrac{3}{2}\)
\end{esempio}









