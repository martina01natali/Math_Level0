% (c)~2012 Dimitrios Vrettos - d.vrettos@gmail.com
% (c)~2014 Claudio.carboncinii - claudio.carboncini@gmail.com
% (c) 2014 Daniele Zambelli - daniele.zambelli@gmail.com

\chapter{Radicali}



\section{Dai numeri naturali ai numeri irrazionali}
\label{sec:radicali_irrazionali}
Nel volume Algebra 1 abbiamo presentato i diversi insiemi numerici. Li 
riprendiamo brevemente per poi approfondire i numeri reali e le loro 
proprietà.

L'insieme dei \emph{numeri naturali} racchiude i numeri che utilizziamo per 
contare; si indica nel seguente modo:
\[\insN=\{0,1,2,3,4,5,6,7,8,9,10,11,\ldots\}\]

Su questi numeri sono definite le seguenti operazioni:
\begin{itemize*}
 \item \emph{addizione}: $n+m$ è il numero che si ottiene partendo da $n$ e 
  continuando a contare per altre $m$ unità;
 \item \emph{sottrazione}: $n-m$ è il numero, se esiste, che 
  addizionato a $m$ dà come risultato $n$;
 \item \emph{moltiplicazione}: $n \cdot m$ è il numero che si ottiene 
sommando 
  $n$ volte $m$, o meglio sommando $n$ addendi tutti uguali a $m$
 \item \emph{divisione}: $n:m$ è il numero, se esiste, che 
  moltiplicato per $m$ dà come risultato $n$;
 \item \emph{potenza}: $n^{m}$ è il numero che si ottiene moltiplicando $m$ 
  fattori tutti uguali a $n$ con $m \ge 2$, ponendo $n^{1}=n$ e $n^{0}=1$;
 \item \emph{radice}: $\sqrt[{n}]{m}$ con $n\ge 2$ è il numero, se esiste, 
  che elevato a $n$ dà come risultato $m$.
\end{itemize*}

L'addizione, la moltiplicazione e la potenza sono definite su tutto 
l'insieme dei numeri naturali, cioè dati due numeri naturali qualsiasi, $n$ 
ed $m$, la somma $n+m$ e il loro prodotto $n \cdot m$ e la loro potenza 
$n^{m}$, escluso il caso $0^{0}$, danno come risultato un numero naturale. 
Non sempre invece, la differenza $n-m$, il quoziente $n:m$ o la 
radice $\sqrt[{n}]{m}$ di due numeri naturali danno come risultato un 
numero naturale.

Tuttavia in molte situazioni vogliamo poter eseguire anche le sottrazioni, 
le divisioni e le radici. Per poter eseguire sempre le sottrazioni abbiamo 
inventato un nuovo insieme numerico: i numeri interi \(\Z\), per poter 
eseguire sempre (quasi) le divisioni abbiamo inventato i numeri razionali 
\(\R\), per poter eseguire (almeno in certi casi) la radice, dovremo 
inventare un nuovo insieme numerico: l'insieme dei numeri reali \(\R\).

Con i numeri razionali, infatti, non è sempre possibile eseguire 
l'estrazione di radice. Consideriamo, ad esempio, $\sqrt{2}$, cioè il numero 
che elevato al quadrato dà 2. La radice di due è un esempio particolarmente 
importante perché rappresenta la lunghezza della diagonale di un quadrato 
di lato uno.

\begin{center}
 \input{\folder lbr//fig001_iraz.pgf}
\end{center}

La radice di due, non è un numero razionale, cioè non può essere scritto né 
sotto forma di frazione né sotto forma di numero decimale finito o 
periodico. 
Supponiamo che la radice di due sia uguale ad un numero razionale scritto 
come emme fratto enne con emme e enne \emph{primi fra di loro}: ad esempio 
\[\sqrt{2}=\dfrac{m}{n}\]
Se questo è vero, allora saranno vere anche le seguenti uguaglianze:
\[2= \dfrac{m^2}{n^2} \quad \text{ e } \quad 2n^2 = m^2\]
Questo vuol dire che \(m^2\) è un numero pari ma allora anche \(m\) deve 
essere pari quindi può essere scritto come il doppio di un altro numero:
\(m=2k\). sostituendolo nell'uguaglianza precedente otteniamo:
\[2n^2 = \tonda{2k}^2 \sRarrow 2n^2 = 4k^2 \sRarrow n^2 = 2k^2\]
Ma allora anche \(n^2\) è un numero pari. Quindi se la radice di due fosse 
esprimibile come rapporto tra due numeri, questi due numeri sarebbero 
contemporaneamente primi tra di loro e pari e quindi: supporre che 
\(\sqrt{2}\) sia un numero razionale porta a una contraddizione.

% Si può dimostrare che~$\sqrt{2}$ non è un numero razionale con una elegante 
% dimostrazione per assurdo. 

% \begin{itemize*}
%  \item 
% Supponiamo che la radice di due sia un numero razionale scritto sotto 
% forma 
% di frazione~$\sqrt{2}= \frac{a}{b}$, Certamente possiamo ridurre la 
% frazione 
% ai minimi termini.
%  \item 
% Se si elevano al quadrato entrambi i membri dell'equazione precedente, si 
% ottiene:~$2=\frac{a^{2}}{b^{2}}$.
%  \item
% La precedente uguaglianza può anche essere scritta come $2 \cdot b^2 = 
% a^2$, 
% da cui si deduce che $a^2$ è un numero pari. Ma se $a^2$ è pari lo deve 
% essere anche $a$.
% quindi $a$ può essere visto come il doppio di un numero:~$a=2 \cdot k$.
%  \item 
% L'uguaglianza precedente può essere vista come:~$2 \cdot b^2 = (2 \cdot 
% k)^2$.
% Elevando al quadrato il monomio di destra si 
% ottiene:~$2 \cdot b^2 = 4 \cdot k^2$.
%  \item 
% Dividendo per 2 entrambi i membri dell'uguaglianza 
% otteniamo:~$b^2 = 2 \cdot k^2$. Da questo discende che anche~$b^2$ è pari 
% e 
% quindi anche $b$, ma se sono entrambi pari allora la frazione di partenza 
% è
% riducibile contraddicendo che la frazione di partenza fosse ridotta ai 
% minimi 
% termini.
% \end{itemize*}

% Elevare un numero al quadrato significa elevare al quadrato le
% singole potenze dei fattori primi in cui questo si scompone. I fattori
% primi di~$a^{2}$ e di~$b^{2}$ sono gli stessi di~$a$ e di~$b$ con
% gli esponenti raddoppiati, Se~$a$ e~$b$ non hanno fattori in comune, 
% anche~$a^{2}$ e~$b^{2}$ non li avranno. Quindi ~$a^{2}$ non può essere il 
% doppio di~$b^{2}$.
% Perciò~$2\ne\frac{a^{2}}{b^{2}}$ e~$\sqrt{2}\ne\frac{a}{b}$.
% 
% Oltre a $\sqrt{2}$ vi sono altri infiniti numeri che non possono essere 
% scritti come frazione. Molte radici e alcuni numeri particolari come $\pi$, 
% che corrisponde alla misura della circonferenza di diametro $1$.
% 
% Questi numeri sono detti \emph{numeri irrazionali} e costituiscono 
% l'insieme $\insJ$ dei numeri irrazionali.
% L'unione degli insiemi $\insQ$ e $\insJ$ è l'insieme $\insR$ dei numeri 
% reali.

La radice di due non è un numero razionale, numeri di questo tipo si dicono 
\emph{numeri irrazionali} l'insieme di questi numeri è indicato con il 
simbolo \(\insJ\).

Tornando al problema di calcolare della \(\sqrt{2}\) possiamo dire 
che~$1<\sqrt{2}<2$. 
Il valore cercato evidentemente non è un numero intero. 
Abbiamo dimostrato sopra che non può neppure essere
numero decimale finito.
Compiliamo una tabella che contenga nella prima
riga i numeri con una sola cifra decimale compresi tra~1 e~2 e nella
seconda riga i rispettivi quadrati:

\begin{center}
\begin{tabular}{ccccccc}
\toprule
$x$ & 1,1 & 1,2 & 1,3 & 1,4 & 1,5 & 1,6\\
$x^{2}$ & 1,21 & 1,44 & 1,69 & 1,96 & 2,25 & 2,89\\
\bottomrule
\end{tabular}
\end{center}

Osserviamo che il numero~2 è compreso tra~$1,4^{2}$ e~$1,5^{2}$,
di conseguenza~$1,4<\sqrt{2}<1,5$, ma ancora
non possiamo precisare il suo valore, anche se abbiamo ristretto
l'intervallo in cui si trova il punto~$K$. Diciamo che~1,4 è un valore 
approssimato per
difetto di~$\sqrt{2}$ mentre~1,5
è un valore approssimato per eccesso; scrivendo~$\sqrt{2}=1,4$
oppure~$\sqrt{2}=1,5$ commettiamo un errore minore di~1/10.

Per migliorare l'approssimazione e tentare di ottenere~$\sqrt{2}$
come numero razionale costruiamo la tabella dei numeri
decimali con due cifre compresi tra~1,4 e~1,5:

\begin{center}
\begin{tabular}{ccccc}
\toprule
$x$ &1,41 &1,42 &1,43 &1,44\\
$x^{2}$ & 1,9881 & 2,0164 & 2,0049 & 2,0776\\
\bottomrule
\end{tabular}
\end{center}

Ora possiamo dire che~1,41 è un valore approssimato per difetto 
di~$\sqrt{2}$ mentre~1,42 è un valore approssimato
per eccesso, con un errore dell'ordine di~1/100. Abbiamo quindi migliorato
l'approssimazione e di conseguenza abbiamo ristretto l'intervallo in cui 
cade il punto~$K$, ma ancora non abbiamo trovato un numero razionale che 
sia uguale a~$\sqrt{2}$.

Continuando con lo stesso procedimento costruiamo due classi di numeri 
razionali che approssimano una per difetto e
una per eccesso il numero cercato, restringendo ogni volta l'ampiezza 
dell'intervallo in cui cade il punto~$K$.
Il procedimento può continuare all'infinito e le cifre decimali che 
troviamo non si ripetono periodicamente.

\begin{center}
 \begin{tabular}{rcll}
\toprule
Valore per difetto & Numero &Valore per eccesso & Ordine dell'errore\\
\midrule
1 & $\sqrt{2}$  & 2  & $10^{0}$\\
1,4 & $\sqrt{2}$ & 1,5  & $10^{-1}$\\
1,41 & $\sqrt{2}$ & 1,42 & $10^{-2}$\\
1,414 & $\sqrt{2}$ &1,415 & $10^{-3}$\\
1,4142 & $\sqrt{2}$ & 1,4143 & $10^{-4}$\\
\ldots & $\sqrt{2}$ & \ldots & \ldots\\
\bottomrule
\end{tabular}
\end{center}

Se andiamo avanti all'infinito, l'approssimazione per difetto diverrà 
infinitamente vicina all'approssimazione per eccesso. 
Possiamo pensare che tra questi due valori ci sia un numero. Questo è un 
numero di un nuovo tipo, un numero \emph{reale}: \(\R\).

\section{I numeri reali}
\label{sec:radicali_reali}

L'unione dei nueri razionali e dei numeri irrazionali dà l'insieme dei 
numeri reali: $\insR=\insQ \cup \insJ$.
I numeri reali sono tutti quei numeri che si possono scrivere in forma 
decimale con un numero finito o infinito di cifre, non necessariamente 
periodiche.
Per esempio, la frazione $\dfrac{17}{16}$ è uguale al numero decimale 
finito~1,0625.
La frazione $\dfrac{16}{17}$ è uguale al numero decimale periodico 
$0,\overline{9411764705882352}$.

Il numero $\pi$ è invece un numero decimale a infinite cifre non periodico. 
Riportiamo alcune cifre:
$\pi $ = 3, 141 592 653 589 793 238 462 643 383 279 502 884 197 169 399 375 
105 820 974 944 592 307 816 406 286
% 208 998 628 034 825 342 117 067 982 148 086 513 282 306 647 093 844 609 
% 550 
% 582 231 725 359 408 128 481 117 450 284 102
% 701 938 521 105 559 644 622 948 954 930 381 964 428 810 975 665 933 446 
% 128 
% 475 648 233 786 783 165 271 201 909 145 648
% 566 923 460 348 610 454 326 648 213 393 607 260 
\ldots 
Nonostante i numeri irrazionali siano stati scoperti dallo stesso Pitagora o 
dai suoi allievi nel $IV$~secolo~$\aC$, solo nel $XIX$~secolo Augustin-Louis 
Cauchy e Richard Dedekind sono giunti a una formulazione rigorosa di numeri 
reali.

In effetti, assumere che i numeri reali sono tutti quelli che si possono 
scrivere in forma decimale finita o infinita,  comporta dei problemi. 
Per esempio, gli algoritmi per addizionare, 
sottrarre e moltiplicare due numeri richiedono di cominciare dalla cifra 
più a destra, cosa che non è possibile per i numeri decimali che non 
finiscono mai. 

\begin{comment}
È possibile costruire l'insieme dei numeri reali a 
partire dall'insieme dei numeri razionali dividendoli in due insiemi~$A$ 
e~$B$ 
con particolari caratteristiche:
\begin{enumerate*}
 \item $A \cap B=\emptyset$
 \item $A \cup B=\insQ$
 \item $\forall a \in A, \forall b \in B, a<b$
\end{enumerate*}

Una coppia di insiemi con queste caratteristiche venne chiamato da 
Dedekind~(1831-1916) una \emph{sezione}, o \emph{partizione} di $\insQ$.

Dato che $A$ e $B$ devono avere intersezione nulla:

\begin{itemize*}
 \item se $A$ ha un massimo $B$ non può avere un minimo;
 \item se $A$ non ha un massimo $B$ può avere un minimo;
 \item $A$ può non avere un massimo $B$ può non avere un minimo;
\end{itemize*}

% Nei primi due casi La sezione individua un numero Razionale, nel terzo 
% caso,
% individua un numero irrazionale.

\begin{exrig}
 \begin{esempio}
 Sezioni
 \begin{itemize}
 \item I due insiemi $A$ e $B$ così definiti: 
 $A=\left\{x\in \insQ |\, x<3\right\}$ e
$B=\left\{x\in \insQ |\, x \ge 3\right\}$ 
definiscono una sezione di $\insQ$, 
infatti $A \cap B=\emptyset$ $A \cup B=\insQ$ e ogni elemento di A è minore 
di ogni elemento di B; 
inoltre possiamo osservare che $A$ non ammette massimo,
non essendoci in esso un numero che sia maggiore di tutti gli altri, 
mentre $B$ ammette il minimo che è 3;
 \item siano $A=\left\{x\in \insQ |\, x<-1\right\}$, $B=\left\{x \in \insQ 
|\, 
x>0\right\}$ la coppia $(A,B)$ non è una sezione di $\insQ$ perché pur 
essendo 
$A\cap B=\emptyset$ non è $A\cup B=\insQ$
 \item siano $A=\left\{x\in \insQ |\, x \le \frac{2}{7}\right\}$, 
$B=\left\{x 
\in Q |\, x \ge \frac{2}{7}\right\}$, anche in questo caso la coppia 
$(A,B)$ 
non 
è una sezione di $\insQ$ poiché $A\cap B=\left\{\frac{2}{7}\right\}$
 \item costruiamo gli insiemi $A$ e $B$ nel seguente modo: $A$ sia l'unione 
tra 
l'insieme dei numeri razionali negativi e tutti i razionali il cui quadrato 
è 
minore di 2, in $B$ mettiamo tutti i razionali il cui quadrato è maggiore di 
2. 
$A=\insQ^{-} \cup \left\{x \in \insQ |\, x^{2}<2\right\}$, $B=\left\{x \in 
\insQ 
|\, x^{2}>2\right\}$. Si ha $A \cap B=\emptyset$ $A\cup B=\insQ$, inoltre 
ogni 
elemento di $A$ è minore di ogni elemento di $B$, dunque $(A,B)$ è una 
sezione 
di $\insQ$, ma $A$ non possiede il massimo e $B$ non possiede il minimo, in 
quanto abbiamo già dimostrato che non esiste un numero razionale che ha $2$ 
come 
quadrato. Questa sezione individua un buco nell'insieme $\insQ$.
 \end{itemize}
 \end{esempio}
\end{exrig}

\begin{definizione}
Si chiama \emph{elemento separatore} di una partizione $(A,B)$ di $\insQ$ 
il 
massimo di $A$ o il minimo di $B$, nel caso in cui almeno uno di questi 
elementi 
esista.
\end{definizione}

Nel primo esempio, poiché esiste il minimo di $B$, la partizione $(A,B)$ 
ammette 
un elemento separatore e identifica il numero razionale $3$.
Nel quarto esempio non esiste un numero razionale che fa da elemento 
separatore, 
la sezione $(A,B)$ identifica un numero irrazionale.

\begin{definizione}
L'insieme $\insR$ dei numeri reali è l'insieme di tutte le partizioni di 
$\insQ$. Chiamiamo
numero razionale le partizioni che ammettono elemento separatore, chiamiamo 
\emph{numero irrazionale} le sezioni che non ammettono elemento separatore.
\end{definizione}

Ogni numero reale è individuato da due insiemi di numeri razionali che 
contengono, nel primo tutte le approssimazioni per difetto e il secondo
tutte le approssimazioni per eccesso.

% Ritornando all'esempio precedente, il numero $\sqrt{2}$ è individuato 
% dalla 
% sezione costituita dagli insiemi $A=\left\{x\in \insQ |\, x<0\right\}$ 
% oppure 
% $x^{2}<2$ e $B=\left\{x\in \insQ |\, x^{2}>2\right\}$.
% Nell'insieme $A$ ci sono tutti i numeri razionali negativi oltre quelli 
% che 
% approssimano $\sqrt{2}$ per difetto: 
% \[A=\{1;1,4;1,41;1,414;1,4142;1,414213;\ldots\}.\]
% Nell'insieme B ci sono tutti i numeri razionali che approssimano 
% $\sqrt{2}$ 
% per eccesso:
% \[B=\{2;1,5;1,42;1,415;1,4143;1,41422;1,414214;\ldots\}.\]

Questa costruzione dell'insieme dei numeri reali $\insR$ a partire 
dall'insieme dei numeri razionali $\insQ$ è puramente astratta e formale, 
non serve al calcolo, ma
permette di collegare i nuovi numeri all'insieme dei numeri naturali 
$\insN$. 

Nell'insieme delle partizioni di $\insQ$ è possibile definire in modo 
rigoroso 
l'ordinamento e le operazioni, nella pratica si usano sempre delle 
approssimazioni, magari molto elevate.

\begin{definizione}
Un insieme $X$ si dice \emph{continuo} se ogni partizione $(X', X'')$ di 
$X$ 
ammette uno e un solo elemento separatore, cioè se esiste un elemento $x$ 
appartenente a $X$ tale che per ogni $x'$ di $X'$ e per ogni $x''$ di $X''$ 
si 
ha $x'{\leq}x{\leq}x''$.
\end{definizione}

\begin{teorema}[di Dedekind]
Ogni partizione dell'insieme $\insR$ di numeri reali ammette uno e uno solo 
elemento separatore.
\end{teorema}

Da questo teorema segue che il numero reale è definito come l'elemento 
separatore di una sezione $(A,B)$ di numeri reali.

\end{comment}
Anche per i numeri reali, la retta può rimanere un comodo modello visuale.

\begin{postulato}[di continuità della retta]
Esiste una corrispondenza biunivoca tra l'insieme dei punti della retta 
(reale) e l'insieme $\insR$ dei numeri reali.
\end{postulato}

Da questo postulato segue la possibilità di definire sulla retta un sistema 
di coordinate: ad ogni punto corrisponde un numero reale (la sua ascissa) e 
viceversa ad ogni numero reale è associato uno e un solo punto sulla retta. 
Analogamente nel piano il sistema di assi cartesiano permette di 
realizzare una corrispondenza biunivoca tra coppie ordinate di numeri reali 
(ascissa e ordinata) e punti. 
Così nello spazio a tre dimensioni possiamo associare i punti alle terne 
ordinate di numeri \dots

% \subsection{Confronto fra numeri reali}
% Per confrontare due numeri reali, osserviamo prima di tutto i segni. Se i 
% segni dei numeri sono
% discordi, il numero negativo è minore del numero positivo. Se i segni dei 
% numeri sono concordi si valuta la parte intera del numero: se sono 
% positivi è 
% più grande quello che ha la parte intera maggiore, viceversa se sono 
% negativi 
% è più grande quello che ha la parte intera minore. A parità di parte 
% intera 
% bisogna confrontare la parte decimale partendo dalle cifre più a sinistra 
% finché non si 
% trova la prima cifra decimale diversa: se i numeri sono positivi è 
% maggiore 
% quello che ha la cifra maggiore; se sono negativi è maggiore quello che ha 
% la 
% cifra minore.
% 
% \begin{exrig}
%  \begin{esempio}
%  Confrontare i seguenti numeri reali
%  \begin{itemize}
%  \item $\sqrt{2}<\sqrt{3}$ per verificarlo ci si può aiutare con la 
% calcolatrice 
% per calcolare le prime cifre decimali dei due numeri 
% $\sqrt{2}=1,4142\ldots$, 
% $\sqrt{3}=1,7320\ldots$ oppure ci si arriva osservando che il numero che 
% elevato 
% al quadrato dà 2 deve essere minore del numero che elevato al quadrato dà 
% 3;
%  \item $\sqrt{99}<10$ per verificarlo è sufficiente osservare che 
% $\sqrt{100}=10$.
%  \end{itemize}
%  \end{esempio}
% \end{exrig}
% 
% \ovalbox{\risolvii \ref{ese:1.3}, \ref{ese:1.4}, \ref{ese:1.5}, 
% \ref{ese:1.6}, 
% \ref{ese:1.7}}\vspazio

\section{Valore assoluto}
\label{sec:radicali_valass}

Si definisce \emph{valore assoluto} di un numero reale $a$, indicato con 
$\valass{a}$, il numero stesso se $a$ è positivo o nullo, il suo opposto se 
$a$ è negativo.

\[
|a|=\left\{\begin{array}{lcc}
 a & \text{, se } & a\ge~0\\
-a & \text{, se } & a<0\end{array}.\right.
\]

Il numero $a$ si dice argomento del valore assoluto.
\begin{multicols}{3}
$\valass{-3}=3$\\$\valass{+5}=5$\\$\valass{0}=0$.
\end{multicols}

\subsection{Proprietà del valore assoluto}
$\valass{x+y}\le \valass{x}+\valass{y}$: il valore assoluto della somma di 
due numeri è minore o uguale della somma dei valori assoluti dei due numeri. 
Si ha l'uguaglianza solo quando i due numeri reali hanno lo stesso segno, 
oppure quando almeno uno dei due numeri è nullo.

$\valass{x-y}\le \valass{x}+\valass{y}$: il valore assoluto della differenza 
di due numeri è minore o uguale della somma dei valori assoluti dei due 
numeri.

$\valass{x\cdot y}=\valass{x}\cdot \valass{y}$: il valore assoluto del 
prodotto di due numeriè uguale al prodotto dei valori assoluti dei due 
numeri.

$\left |{\dfrac{x}{y}}\right|=\dfrac{\valass{x}}{\valass{y}}$: il valore 
assoluto del rapporto di due numeri è uguale al rapporto dei valori 
assoluti dei due numeri.

In generale, se l'argomento del valore assoluto è una funzione $f(x)$ si ha:
\[
\valass{f(x)}=\left\{\begin{array}{lcc}
 f(x) & \text{, se } & f(x)\ge~0\\
-f(x) & \text{, se } & f(x)<0\end{array}.\right.
\]
\begin{exrig}
 \begin{esempio}
 Valore assoluto di numeri reali
 \begin{itemize}
 \item $\valass{5+3}=\valass{5}+\valass{3}$ in entrambi i casi si ottiene 
$8$
 \item $\valass{5+(-3)}=2$ mentre $\valass{5}+\valass{-3}=8$, pertanto 
$\valass{5+(-3)}<\valass{5}+\valass{-3}$.
 \end{itemize}
 \end{esempio}
\end{exrig}

Nelle espressioni contenenti valori assoluti di argomento letterale si deve 
cercare di eliminare il valore assoluto.

% \begin{exrig}
 \begin{esempio}
 Valore assoluto di argomento letterale
 \begin{itemize}
 \item $\left|{x^{2}}\right|=x^{2}$ infatti $x^{2}$ è una quantità sempre 
non 
negativa;
 \item $\left|{a^{2}+1}\right|=a^{2}+1$ infatti $a^{2}$ è sempre positivo, 
aumentato di $1$ sarà sempre $>0$
 \item $\valass{x-1}=\left\{\begin{array}{l}
x-1\text{, se }x\ge~1\\
-x+1\text{, se }x<1\end{array}\right.$ una funzione di questo tipo si dice 
\emph{definita per casi};
 \item $f(a)=\valass{a+1}-3a+1$ acquista due significati a seconda che 
l'argomento del valore assoluto sia non negativo o negativo. La sua 
espressione 
algebrica è:
\[
f(a)=\valass{a+1}-3a+1=\left\{\begin{array}{l}
a+1-3a+1=-2a+2 \text{, se }a+1\ge~0\Rightarrow a\ge-1\\
-(a+1)-3a+1=-4a \text{, se }a+1<~0\Rightarrow a<-1\end{array}.\right.
\]
 \end{itemize}
 \end{esempio}
% \begin{esempio}
%  $f(x)=\valass{x-5}+\valass{x+2}$.
% 
%  La presenza di due valori assoluti ci obbliga a studiare i casi generati 
% dal 
% segno dei singoli argomenti.
%  Pertanto poiché l'argomento del primo valore assoluto è non negativo per 
% $x\ge 
% 5$ e l'argomento del secondo valore assoluto è non negativo
%  per $x\ge -2$, possiamo porre la reciproca situazione nel seguente 
% grafico:
% \begin{center}
% \input{\folder lbr//fig002_rad2.pgf}
% \end{center}
% 
% \begin{enumerate}[label={(\Alph*)}]
%  \item $x<-2$: in questo intervallo entrambi gli argomenti sono negativi, 
% pertanto 
% \[f(x)=\valass{x-5}+\valass{x+2}=-x+5-x-2=-2x+3.\]
% Se $x=-2$ si ha $f(-2)=\valass{-2-5}+0=7$
%  \item $-2<x<5$ il primo argomento è negativo e il secondo è positivo, 
% pertanto 
% \[f(x)=\valass{x-5}+\valass{x+2}=-x+5+x+2=7.\]
% Se $x=5$ si ha $f(5)=0+\valass{5+2}=7$
%  \item $x>5$ entrambi gli argomenti positivi, pertanto 
% \[f(x)=\valass{x-5}+\valass{x+2}=x-5+x+2=2x-3.\]
% \end{enumerate}
% Possiamo allora sintetizzare in questo modo
% \[
% \valass{x-5}+\valass{x+2}=\left\{\begin{array}{l}
% -2x+3 \text{, se }x<-2\\
% 7 \text{, se }-2\le x<5\\
% 2x-3 \text{, se }x\ge 5\end{array}.\right.
% \]
% \end{esempio}
% \end{exrig}
% \ovalbox{\risolvii \ref{ese:1.7}, \ref{ese:1.8}, \ref{ese:1.9}, 
% \ref{ese:1.10}, \ref{ese:1.11}}















\section{Dalle potenze alle radici}
\label{sec:radicali_potenze_radici}

\subsection{Osservazioni sulle potenze}

% \subsubsection{Operazioni inverse}

Quando abbiamo parlato delle operazioni, abbiamo anche accennato all'idea
delle operazioni inverse. Un'operazione inversa è un'operazione che permette
di ``tornare indietro'' rispetto all'effetto prodotto da un'operazione.
Ad esempio se ad un numero qualunque aggiungo~7 e poi tolgo~7 ottengo ancora il 
numero di partenza. Possiamo dire che la sottrazione è l'operazione inversa 
dell'addizione. Così, se parto da un numero~$a$ e lo moltiplico per~$b$ poi lo 
divido per~$b$ ottengo ancora il numero~$a$:

$142 + 7 - 7 = 149 -7 = 142$

$a * b \div b= a$

L'operazione inversa può anche essere vista come quell'operazione he permette 
di trovare un operando conoscendo il risultato:

% \begin{multicols}{2}
$\dots + 3 = 7 => \dots = 7 - 3$

$5 + \dots = 9 => \dots = 9 - 5$

$\dots \times 3 = 6 => \dots = 6 \div 3$

$15 \times \dots = 5 => \dots = 15 \div 5$
% \end{multicols}

Le cose si complicano un po' se consideriamo le altre due operazioni 
aritmetiche:

% \begin{multicols}{2}
$\dots - 3 = 7 => \dots = 7 \overset{?}{\dots} 3$

$5 - \dots = 9 => \dots = 9 \overset{?}{\dots} 5$

$\dots \div 3 = 6 => \dots = 6 \overset{?}{\dots} 3$

$15 \div \dots = 5 => \dots = 15 \overset{?}{\dots} 5$
% \end{multicols}

Perché Sottrazione e divisione si comportano in modo diverso da addizione e 
moltiplicazione? Come si comporterà la potenza?

Anche la potenza ha due operazioni inverse. Se dobbiamo trovare la base:

$\dots ^3 = 8 \Rightarrow \dots = \sqrt[3]{8} = 2$ (Radice)

mentre per trovare l'esponente:

$2 ^{\dots} = 8 \Rightarrow \dots = \log_{2}{8} = 3$ (Logaritmo)

\begin{definizione}
La \emph{radice} è l'operazione che permette di calcolare la base conoscendo
la potenza e l'esponente.
\end{definizione}

\begin{definizione}
Il \emph{logaritmo} è l'operazione che permette di calcolare l'esponente
conoscendo la potenza e la base.
\end{definizione}

Ne seguito del capitolo ci concentreremo sulle radici rimandando lo studio
dei logaritmi a tempi migliori. 

\section{Definizioni}
\label{sec:radicali_definizioni}

Prima di tutto mettiamoci d'accordo su alcuni nomi:
\large
\[b=\sqrt[n]{a}\]
\begin{center} 
\(\sqrt[]{~}\) \quad \emph{radice}; \quad
\(n\) \quad \emph{indice}; \quad
\(a\) \quad \emph{radicando}; \quad
\(b\) \quad \emph{radice}.
\end{center}
\normalsize
Osserviamo che, come nelle potenze usiamo la stessa parola per indicare 
l'operazione e il risultato dell'operazione, si deve capire dal contesto il 
significato della parola ``radice''.

Per quanto detto, potremmo, in prima approssimazione,  dare la seguente 
definizione: 

\begin{definizione}
Chiamiamo \textbf{radice ennesima} del numero~$a$ quel numero che 
elevato alla enne dà come risultato $a$:
 \[b=\sqrt[n]{a} \sLRarrow b^n=a\]
\end{definizione}

Questa definizione rende l'idea di cosa sia la radice, ma, purtroppo, la 
realtà è più complicata\dots 
Vediamo alcuni esempi:

\begin{esempio}
 Usando la definizione precedente calcola le seguenti radici:
 \begin{multicols}{4}
  \begin{enumerate}
   \item \(\sqrt[4]{-16}\);
   \item \(\sqrt[4]{+16}\);
   \item \(\sqrt[3]{-64}\);
   \item \(\sqrt[3]{+64}\);
  \end{enumerate}
 \end{multicols}
Affrontiamo con un po' di giusta pignoleria i quattro problemi:
\begin{enumerate}
 \item Vogliamo trovare il numero che elevato alla 
quarta dia come risultato~$-16$: \(x^4=-16\)

Se ci ricordiamo le riflessioni fatte sulle potenze, dovrebbe tornarci alla 
mente che una potenza con esponente pari è sempre positiva: nessun numero 
reale elevato alla quarta può dare un risultato negativo.

Quindi: \(\sqrt[4]{-16} \quad \text{Non Ha Soluzioni Reali}\)
 \item Il secondo esempio apparentemente è più semplice: è facile trovare 
il numero che elevato alla quarta dia~16: è~2 \dots

Ma ricordandoci l'osservazione sulle potenze a qualcuno può venire in mente 
che anche~$-2$ elevato alla quarta fa~16! 
Quindi in questo caso l'operazione di radice darebbe due risultati. Quando 
useremo i numeri \emph{Complessi} accetteremo anche operazioni che danno 
più risultati, ma fin che lavoriamo con i ``tranquilli'' numeri reali, 
vogliamo operzioni che diano per risultato un solo numero. I matematici 
hanno deciso di scartare il valore negativo.

Quindi \(\sqrt[4]{+16} = +2\)
 \item Nel terzo esempio dovremo trovare quel numero che elevato alla terza 
faccia~$-64$ in questo caso il numero esiste ed è solo~$-4$.

Quindi: \(\sqrt[3]{-64} = -4\)
 \item Nel quarto esempio dovremo trovare quel numero che elevato alla 
terza faccia~$+64$ in questo caso il numero esiste ed è solo~$+4$.

Quindi: \(\sqrt[3]{+64} = +4\)
\end{enumerate}
\end{esempio}

Osservando gli esempi vediamo che è importante distinguere tra radici con 
indice pari e radici con indice dispari e possiamo così dare la 
definizione di radice ennesima.
\begin{definizione}
di \textbf{radice}:

 \begin{itemize}
  \item Se l'indice è \textbf{dispari} la radice di un qualunque numero~$a$ 
è quel numero che elevato alla enne dà come risultato~$a$:
 \[b=\sqrt[n]{a} \sLRarrow b^n=a\]
  \item Se l'indice è \textbf{pari} abbiamo due casi:
  \begin{itemize}
   \item se il \textbf{radicando~$a$ è negativo} la radice 
           Non Ha Soluzioni Reali;
   \item se il \textbf{radicando~$a$ è positivo} la radice è quel numero 
positivo che elevato alla enne dà come risultato~$a$:
 \[\text{se} \quad a \geqslant 0 \quad \text{allora} \quad 
    b=\sqrt[n]{a} ~\sLRarrow ~ b^n=a \quad \text{con} \quad b \geqslant 0\]
  \end{itemize}
 \end{itemize}
\end{definizione}

Vediamo alcuni casi.

% \begin{exrig}
\begin{esempio}
Alcune radici di numeri reali:
 \begin{multicols}{2}
\begin{itemize}
\item $\sqrt 4=2$ infatti $2^2=4 \text{ e } 2 \geqslant 0$
\item $\sqrt{\dfrac 9{16}}=\dfrac 3 4$ infatti 
$\left(\dfrac 3 4\right)^2=\dfrac 9{16} \text{ e } \dfrac{3}{4} \geqslant 0$
\item $\sqrt{0,01}=0,1$ infatti $0,1^2=0,01 \text{ e } \dots$
\item $\sqrt 1=1$ infatti $1^2=1 \text{ e } \dots$
\item $\sqrt 0=0$ infatti $0^2=0 \text{ e } \dots$
\item $\sqrt{-16}$ non esiste, radicando negativo;
\item $\sqrt{11}=3,3166247\dots$ è un numero irrazionale;
\item $\sqrt[3]{-8}=-2$ infatti $\left(-2\right)^3=-8$
\item $\sqrt[3]{125}=5$ infatti $5^3=125$
\item $\sqrt[3]1=1$ infatti $1^3=1$
\item $\sqrt[3]0=0$ infatti $0^3=0$
\item $\sqrt[3]{-1000}=-10$ infatti $\left(-10\right)^3=-1000$
\item $\sqrt[3]{\dfrac 1 8}=\dfrac 1 2$ infatti 
  $\left(\dfrac 1 2\right)^3=\dfrac 1 8$
\item $\sqrt[3]{0,125}=0,5$ infatti $(0,5)^3=0,125$
 \item $\sqrt[4]{16}=2$ infatti $2^4=16 \text{ e } \dots$
 \item $\sqrt[4]{-16}$ N.H.S.R. infatti $(-2)^4=+16$
 \item $\sqrt[5]{32}=2$ infatti $2^5=16$
 \item $\sqrt[4]1=1$ infatti $1^4=1 \text{ e } \dots$
 \item $\sqrt[n]0=0 \text{ infatti } 0^n=0 \text{ e } 0 \geqslant 0$
 \item $\sqrt[5]{-1}=-1 \text{ infatti } (-1)^5=-1$
\end{itemize}
\end{multicols}
\end{esempio}
% \end{exrig}

\begin{comment}

\subsection{Radici quadrate}

\begin{definizione}
Si dice \emph{radice quadrata} di un numero reale positivo o nullo quel numero 
reale positivo o nullo che elevato al quadrato dà come risultato il numero dato.
In simboli~$\sqrt a=b \Leftrightarrow b^2=a$ dove $a,b\in \insR^{+} \cup \{0\}$.
\end{definizione}

Il simbolo $\sqrt{\quad}$ è il simbolo della radice quadrata; 
il numero $a$ è detto \emph{radicando}, 
il numero $b$ è detto \emph{radice quadrata} di $a$.

Dalla definizione $\sqrt{a^2}=a$ con $a\ge 0$, quindi: 

$$\sqrt{81}=9 \text{ perché } 9^2=81 \quad 
\sqrt{\frac {9}{64}}=\frac{3}{8}
\text{ perché } \left(\frac 3 8\right)^2=\frac 9{64}$$

\osservazione $\sqrt{81}=\sqrt{(-9)^2}$, ma non è vero che 
$\sqrt{(-9)^2}=-9$ perché nella definizione di radice quadrata abbiamo imposto 
che il risultato dell'operazione di radice quadrata sia sempre un numero 
positivo o nullo.
Questa osservazione ci induce a porre molta attenzione quando il radicando è 
un'espressione letterale: in questo caso $\sqrt{a^2}=a$ non è del tutto 
corretto poiché $a$ può assumere sia valori positivi sia valori negativi
mentre, nei numeri reali, il risultato della radice quadrata non è mai un 
numero negativo. 
Scriveremo correttamente~$\sqrt{a^2}=\valass{a}$.

% \begin{exrig}
\begin{esempio}
Radici quadrate
 \begin{multicols}{2}
\begin{itemize}
\item $\sqrt 4=2$ infatti $2^2=4$
\item $\sqrt{\dfrac 9{16}}=\dfrac 3 4$ infatti $\left(\dfrac 3 4\right)^2=
  \dfrac 9{16}$
\item $\sqrt{0,01}=0,1$ infatti $0,1^2=0,01$
\item $\sqrt 1=1$ infatti $1^2=1$
\item $\sqrt 0=0$ infatti $0^2=0$
\item $\sqrt{-16}$ non esiste, radicando negativo;
\item $\sqrt{11}$ esiste ma non è un numero intero né razionale, 
  è un numero irrazionale;
\item $\sqrt{x^2}=\left|x\right|$ dobbiamo mettere il valore assoluto 
  al risultato perché non conoscendo il segno di $x$ dobbiamo imporre che 
  il risultato sia sicuramente positivo;
\item $\sqrt{a^2-4a+4}=\sqrt{(a-2)^2}=\left|a-2\right|$ dobbiamo mettere 
  il valore assoluto perché $a-2$ può anche essere negativo;
\item $\sqrt{9(x+1)^2}=3\left|x+1\right|$.
\end{itemize}
\end{multicols}
\end{esempio}
% \end{exrig}

\subsection{Radici cubiche}

\begin{definizione}
 Si dice \emph{radice cubica} di un numero reale $a$ quel numero che, 
 elevato al cubo, dà come risultato $a$. 
 In simboli $\sqrt[3]a=b \Leftrightarrow b^3=a$ dove $a,b\in \insR$.
\end{definizione}

Puoi notare che la radice cubica di un numero reale esiste sempre sia per 
i numeri positivi o nulli, sia per i numeri negativi.

% \begin{exrig}
\begin{esempio}
Radici cubiche
 \begin{multicols}{2}
 \begin{itemize}
\item $\sqrt[3]{-8}=-2$ infatti $\left(-2\right)^3=-8$
\item $\sqrt[3]{125}=5$ infatti $5^3=125$
\item $\sqrt[3]1=1$ infatti $1^3=1$
\item $\sqrt[3]0=0$ infatti $0^3=0$
\item $\sqrt[3]{-1000}=-10$ infatti $\left(-10\right)^3=-1000$
\item $\sqrt[3]{\dfrac 1 8}=\dfrac 1 2$ infatti 
  $\left(\dfrac 1 2\right)^3=\dfrac 1 8$
\item $\sqrt[3]{0,125}=0,5$ infatti $(0,5)^3=0,125$
\item $\sqrt[3]{x^3}=x$ per le radici cubiche non si deve mettere 
  il valore assoluto;
\item $\sqrt[3]{x^3+3x^2+3x+1}=\sqrt[3]{(x+1)^3}=x+1$ non si deve mettere 
  il valore assoluto.
\end{itemize}
\end{multicols}
\end{esempio}
% \end{exrig}

Osserva che la radice cubica di un numero mantiene sempre lo stesso segno del 
numero in quanto il cubo di un numero reale conserva sempre il segno della 
base.

\subsection{Radici n-esime}
Oltre alle radici quadrate e cubiche si possono considerare radici di indice 
qualsiasi. 
Si parla in generale di radice \emph{n-esima} per indicare una radice con un 
qualsiasi indice $n$.

\begin{definizione}
Si dice \emph{radice n-esima} di un numero reale~$a$ quel numero~$b$ che 
elevato alla~$n$ dà come risultato~$a$. Se~$n$ è pari ~$b$ è positivo.
In simboli~$\sqrt[n]a=b \Leftrightarrow b^n=a$ con $n\in \insN, n > 0$.

$a$ si dice \emph{radicando}, 

$n$ si dice \emph{indice}, 

$b$ si dice \emph{radice}

Non si definisce la radice di indice $0$ e la scrittura $\sqrt[0]a$ è priva 
di significato. Alla scrittura~$\sqrt[1]a$ si dà il valore $a$.
\end{definizione}

Quando si tratta con le radici n-esime di un numero reale, bisogna fare 
attenzione se l'indice della radice è pari o dispari. 
Si presentano infatti i seguenti casi:
\begin{itemize}
 \item se l'indice $n$ è dispari $\sqrt[n]a$ è definita per qualsiasi valore 
  di $a\in \insR$, inoltre è negativa se~$a<0$, positiva se $a>0$ e 
  nulla se $a=0$
 \item se l'indice $n$ è pari $\sqrt[n]a$ è definita solo per i valori 
  di~$a\geq 0$ e si ha che $\sqrt[n]a \ge 0$.
\end{itemize}

% \begin{exrig}
\begin{esempio}
Radici n-esime
\begin{multicols}{2}
 \begin{itemize}
 \item $\sqrt[4]{16}=2$ infatti $2^4=16$
 \item $\sqrt[4]{-16}$ non esiste infatti $(-2)^4=+16$
 \item $\sqrt[5]{32}=2$ infatti $2^5=16$
 \item $\sqrt[4]1=1$ infatti $1^4=1$
 \item $\sqrt[n]0=0$
 \item $\sqrt[5]{-1}=-1$ infatti $(-1)^5=-1$
 \item $\sqrt[4]{x^4}=\left|x\right|$ va messo il valore assoluto perché 
   l'indice della radice è pari;
 \item $\sqrt[5]{x^5}=x$ non va messo il valore assoluto perché l'indice 
   della radice è dispari.
\end{itemize}
\end{multicols}
\end{esempio}
% \end{exrig}

% \ovalbox{\risolvii \ref{ese:2.1}, \ref{ese:2.2}, \ref{ese:2.3}, 
% \ref{ese:2.4},\ref{ese:2.5}, \ref{ese:2.6},\ref{ese:2.7}, 
% \ref{ese:2.8},\ref{ese:2.9}, \ref{ese:2.10}}

\section{Condizioni di esistenza}
\label{sec:radicali_condizioni_esistenza}

Riprendendo le conoscenze sulle potenze, ricordiamo che il quadrato di un 
numero non può essere negativo perché un numero negativo per se stesso dà come 
risultato un numero positivo. 
In generale una potenza con esponente pari è sempre positiva.  
Mentre una potenza con esponente dispari ha lo stesso segno della base.

\newpage

\begin{esempio}
Alcune potenze:

\begin{multicols}{4}
 $(+5)^2=+25$
 
 $(-5)^2=+25$
 
 $(+2)^6=+64$
 
 $(-2)^6=+64$
 
 $(+2)^3=+8$
 
 $(-2)^3=-8$
 
 $(+2)^7=+128$
 
 $(-2)^7=-128$
\end{multicols}

\end{esempio}

Possiamo affermare che nessun numero reale può essere la radice quadrata di 
un numero negativo, poiché non esiste nessun numero reale che elevato al 
quadrato dia come risultato un numero negativo. 

E, più in generale, nessun numero reale può essere la radice di indice pari 
di un numero negativo.

Finché abbiamo a che fare con numeri, le cose risultano abbastanza semplici, 
ma quando il radicando è un'espressione letterale dobbiamo fare molta 
attenzione a operare su di esso.

Non è detto che $sqrt{a}$ esista, perché non possiamo sapere se~$a$ 
rappresenta un numero positivo o negativo. 
Quindi $sqrt{a}$ è un numero reale solo se~$a \ge 0$.

Le \emph{condizioni di esistenza} (in breve si può scrivere $\CE$) 
di un radicale sono le condizioni cui devono 
soddisfare le variabili che compaiono nel radicando affinché la radice 
sia un numero reale.

Supponiamo di avere $\sqrt[n]{A(x)}$ con $A(x)$ espressione nella
variabile~$x$, dobbiamo distinguere i seguenti casi:
\begin{itemize*}
\item se $n$ è pari la radice è un numero reale solo per i valori 
  di~$x$ che rendono non negativo il radicando, cioè $\CE: A(x)\ge 0$
\item se $n$ è dispari la radice è un numero reale per qualsiasi valore 
  della variabile~$x$ che permette di calcolare il radicando.
\end{itemize*}

% \begin{exrig}
\begin{esempio}
Condizioni di esistenza
 \begin{itemize}
 \item $\sqrt x$:\quad $\CE x\ge 0$
 \item $\sqrt[3]x$:\quad $\CE \forall x\in \insR$
 \item $\sqrt{-x}$:\quad $\CE x\le 0$
 \item $\sqrt[3]{-x}$:\quad $\CE \forall x\in \insR$
 \item $\sqrt{x-1}$:\quad $\CE x-1\ge 0 \Rightarrow x\ge 1$
 \item $\sqrt{a^2+1}$:\quad $\CE \forall a\in \insR$, infatti $a^2$ è sempre 
   positivo pertanto $a^2+1>0, \forall a\in \insR$
 \item $\sqrt[3]{\frac 1{x+1}}$:\quad la radice cubica è definita per valori 
   sia positivi sia negativi del radicando, tuttavia bisogna comunque porre la 
   condizione che il denominatore della frazione non sia nullo, 
   quindi $\CE x+1\neq 0 \Rightarrow x\neq -1$
 \item $\sqrt[4]{xy}$:\quad $\CE xy\ge 0$
 \item $\sqrt[5]{a^2(a-3)}$: poiché la radice ha indice dispari non occorre 
   porre alcuna condizione di esistenza.
\end{itemize}
\end{esempio}

\begin{esempio}
 Determina le condizioni di esistenza della seguente 
 espressione: $\sqrt x+\sqrt{x+1}$.

C.E. $\sqrt x$ esiste per $x\ge 0$, $\sqrt{x+1}$ 
esiste per $x+1\ge 0$, quindi per individuare le condizioni di esistenza 
dell'espressione occorre risolvere il sistema 
$\left\{\begin{array}{l} x\ge0\\ x+1\ge0\end{array}\right.
\Rightarrow\left\{\begin{array}{l}x\ge0\\x\ge-1\end{array}\right.$.

\begin{center}
 \input{\folder lbr//fig001_is.pgf}
\end{center}

In definitiva $\CE x\ge 0$.
\end{esempio}

\begin{esempio}
 Determina le condizioni di esistenza della radice 
 $\sqrt[4]{\dfrac{x-1}{x+1}}$.

C.E. $\dfrac{x-1}{x+1}\ge 0$. 
Occorre discutere il segno della frazione $f$, combinando il segno del 
numeratore $N$ e del denominatore $D$:

\begin{center}
 \input{\folder lbr//fig002_seg.pgf}
\end{center}
Pertanto C.E. $x<-1\vee x\ge 1$.
\end{esempio}
% \end{exrig}

% \vspazio\ovalbox{\risolvii \ref{ese:2.11}, \ref{ese:2.12}, \ref{ese:2.13}, 
% \ref{ese:2.14},\ref{ese:2.15}}

\end{comment}

\section{Potenze ad esponente razionale}
\label{sec:radicali_esp_razionale}

Ritorniamo al problema di partenza, vogliamo trovare l'operazione che, data 
una potenza e l'esponente dia come risultato la base cioè dati: \(b^n\) e 
\(n\) dia come risultato \(b\).

Per ottenere questo non avevamo bisogno di creare una nuova operazione, la 
radice, bastava la potenza:
\[\tonda{b^n}^\frac{1}{n}=b\]
In generale:
\[\sqrt[n]{a^m}=a^\frac{m}{n}\]
\begin{osservazione}
 Dobbiamo prestare attenzione al fatto che, se l'esponente è negativo, la 
base deve essere diversa da zero.
\end{osservazione}

\begin{osservazione}
Dato che abbiamo visto che se l'indice è pari il radicando deve essere 
maggiore di zero, in generale la potenza che ha peresponente un numero 
razionale è definita solo se la base è maggiore di zero.
\end{osservazione}

\begin{definizione}
Si dice \emph{potenza a esponente razionale} $\frac m n$ di un numero reale 
positivo $a$ l'espressione:
 $a^{\frac m n}=\sqrt[n]{a^m}=\left(\sqrt[n]a\right)^m$ con 
 $\frac m n\in \insQ$.
\end{definizione}

\begin{esempio}
 Calcola le seguenti radici usando le potenze a esponente razionale.
 \begin{itemize*}
 \item~$\tonda{\sqrt[3]{12}}^2=12^{\frac{2}{3}}$
 \item~$\tonda{\sqrt[3]{8}}^2=\tonda{2^3}^{\frac{2}{3}}=2^{3\frac{2}{3}}=4$
 \item~$\tonda{\sqrt{25}}^3=\tonda{5^2}^{\frac{3}{2}}=5^3=125$
 \item $\sqrt[3]{125^{-2}}=\sqrt[3]{(5^3)^{-2}}=
        \tonda{5^3}^{-\frac{2}{3}}=5^{-2}=\dfrac 1{25}$
\end{itemize*}
\end{esempio}

\begin{esempio}
 Calcola le seguenti potenze a esponente razionale.
 \begin{itemize*}
 \item $64^{\frac{2}{3}}=\tonda{4^3}^{\frac{2}{3}}=4^2=16$
 \item $64^{-\frac{2}{3}}=\dfrac{1}{\tonda{4^3}^{\frac{2}{3}}}=
   \dfrac{1}{4^2}=\dfrac{1}{16}$
 \item $\left(\dfrac 1{49}\right)^{-\frac{1}{2}}=
        \tonda{49}^{\frac{1}{2}}=\tonda{7^2}^{\frac{1}{2}}=7$
 \item $\left(\dfrac{1}{8}\right)^{-\frac{3}{2}}=
   \tonda{2^3}^{\frac{3}{2}}=
   \tonda{2^9}^\frac{1}{2}=
   \tonda{2 \cdot 2^8}^\frac{1}{2}=
   \tonda{2^8}^{\frac{1}{2}}2^{\frac{1}{2}}=2^4 \sqrt{2}$
\end{itemize*}
\end{esempio}

\begin{comment}

In questo paragrafo ci proponiamo di scrivere la radice n-esima di un 
numero 
reale $a\geq0$ sotto forma di potenza di $a$, vogliamo cioè che sia:
$\sqrt[n]a=a^x$.

\paragraph {Caso con esponente positivo}
Elevando ambo i membri dell'uguaglianza alla potenza~$n$ otteniamo: 
$\left(\sqrt[n]a\right)^n=\left(a^x\right)^n$ da cui si ottiene 
$a=a^{n\cdot x}$.
Trattandosi di due potenze con base~$a{\geq}0$, l'uguaglianza è resa 
possibile 
solo se sono uguali gli esponenti. 
In altre parole, deve essere: $1=n\cdot x \Rightarrow x=\dfrac 1 n$, 
quindi: $\sqrt[n]a=a^{\frac 1 n}$.

Vediamo ora di generalizzare la formula. Sia $m$ un numero intero positivo, 
possiamo scrivere $a^{\frac m n}=\left(a^{\frac 1 n}\right)^m$ e 
quindi $a^{\frac m n}=\left(\sqrt[n]a\right)^m$.

% \begin{exrig}
\begin{esempio}
 Calcola le seguenti potenze a esponente razionale positivo.
 \begin{itemize*}
 \item~$27^{\frac{2}{3}}$: si ha che 
   $27^{\frac{2}{3}}=\left(\sqrt[3]{27}\right)^2=3^2=9$
 \item~$25^{\frac 3 2}$: si ha che 
   $25^{\frac 3 2}=\left(\sqrt[2]{25}\right)^3=5^3=125$.
\end{itemize*}
\end{esempio}
% \end{exrig}

\paragraph{Caso con esponente negativo}
Per definire la potenza ad esponente razionale negativo è necessario 
imporre 
la restrizione $a{\neq}0$, infatti risulta:
$a^{-\frac m n}=\dfrac 1{a^{\frac m n}}=\left(\dfrac 1 a\right)^{\frac m n}$

% \begin{exrig}
\begin{esempio}
 Calcola le seguenti potenze a esponente razionale negativo.
 \begin{itemize*}
 \item $27^{-\frac{2}{3}}=\dfrac 1{\left(\sqrt[3]{27}\right)^2}=
   \dfrac 1{3^2}=\dfrac 1 9$
 \item $125^{-\frac{2}{3}}=\sqrt[3]{125^{-2}}=\sqrt[3]{(5^3)^{-2}}=
   \sqrt[3]{(5^{-2})^3}=5^{-2}=\dfrac 1{25}$
 \item $\left(\dfrac 1 8\right)^{-\frac 3 2}=
   \sqrt{\left(\dfrac 1 8\right)^{-3}}=\sqrt{8^3}=\sqrt{(2^3)^3}=\sqrt{2^9}$
 \item $\left(\dfrac 1{49}\right)^{-\frac 1 2}=(49)^{\frac 1 2}=\sqrt{49}=7$
\end{itemize*}
\end{esempio}
% \end{exrig}

In generale si dà la seguente
\begin{definizione}
Si dice \emph{potenza a esponente razionale} $\frac m n$ di un numero reale 
positivo $a$ l'espressione:
 $a^{\frac m n}=\sqrt[n]{a^m}=\left(\sqrt[n]a\right)^m$ con 
 $\frac m n\in \insQ$.
\end{definizione}

Perché abbiamo dovuto imporre la condizione che~$a$ sia un numero positivo?
Partiamo dall'espressione $a^{\frac 1 n}$ con $n\in \insN-\{0\}$, 
se $n$ è dispari la potenza $a^{\frac 1 n}$ è sempre definita per ogni 
valore 
della base $a$, mentre se è pari $a^{\frac 1 n}$ è definita solo 
per $a{\geq}0$.

Nel caso generale $a^{\frac m n}$ con $\frac m n\in \insQ$ 
la formula $a^{\frac m n}=\left(\sqrt[n]a\right)^m$ è falsa se $a<0$.

Consideriamo il seguente esempio:
$(-2)^{\frac 6 6}=\left[(-2)^{\frac 1 
6}\right]^6=\left(\sqrt[6]{-2}\right)^6$ 
non è definita nei numeri reali perché non esiste la radice sesta di un 
numero 
negativo.
Tuttavia possiamo anche scrivere 

\[(-2)^{\frac 6 6}=\left[(-2)^6\right]^{\frac 1 6}=(64)^{\frac 1 6}=
\sqrt[6]{64}=2.\]

Arriviamo pertanto a due risultati differenti.

Per estendere la definizione al caso di basi negative sarebbe necessario 
stabilire un ordine di priorità delle operazioni ma ciò andrebbe contro la 
proprietà commutativa del prodotto degli esponenti di una potenza di 
potenza.

% \vspazio\ovalbox{\risolvii \ref{ese:2.16}, \ref{ese:2.17}, 
\ref{ese:2.18}, 
% \ref{ese:2.19},\ref{ese:2.20}}

\end{comment}

\section{Semplificazione di radici}
\label{sec:radicali_semplificazione}

\begin{proposizione}
Il valore di una radice in $\insR^+\cup \{0\}$ non cambia se moltiplichiamo 
o dividiamo l'indice della radice e l'esponente del radicando per uno 
stesso 
numero positivo. 
In simboli 
\[\sqrt[n]{a^m}=a^{\frac{m}{n}}=a^{\frac{mt}{nt}}=\sqrt[nt]{a^{mt}}\]
\[\sqrt[nt]{a^{mt}}=a^{\frac{mt}{nt}}=\sqrt[n]{a^m}=a^{\frac{m}{n}}\]
con $a\ge 0$ e $m,n,t\in \insN-\{0\}$.
\end{proposizione}

% \newpage
% % \begin{exrig}
%  \begin{esempio}
%  Radici equivalenti.
%  \begin{itemize}
%  \item $\sqrt{2}=\sqrt[4]{2^2}$ abbiamo moltiplicato per $2$ indice della 
%   radice ed esponente del radicando;
%  \item $\sqrt[3]a=\sqrt[9]{a^3}$ abbiamo moltiplicato per $3$ indice 
% della 
%   radice ed esponente del radicando.
% \end{itemize}
%  \end{esempio}
% % \end{exrig}
% 
% \begin{proposizione}
% Il valore di una radice in $\insR^+\cup \{0\}$ non cambia se dividiamo 
% l'indice della radice e l'esponente del radicando per un loro divisore 
% comune. 
% In simboli $\sqrt[nt]{a^{mt}}=\sqrt[n]{a^m}$ con $a\ge 0$ e 
% $m,n,t\in \insN-\{0\}$.
% \end{proposizione}

% \begin{exrig}
 \begin{esempio}
Semplificazione di radici
\begin{itemize}
 \item $\sqrt[4]{5^2}=5^{\frac{2}{4}}=5^{\frac{1}{2}}=\sqrt 5$: 
  abbiamo semplificato per $2$ indice della radice ed esponente del 
radicando;
 \item 
$\sqrt[4]{(-3)^2}=\sqrt[4]{3^2}=3^{\frac{2}{4}}=
 3^{\frac{1}{2}}=\sqrt 3$
 \item $\sqrt[10]{4^{15}}=4^{\frac{15}{10}}=4^{\frac{3}{2}}=\sqrt{4^3}$: 
  abbiamo semplificato per $5$
 \item $\sqrt[7]{3^9}=3^{\frac{9}{7}}$: 
  non è riducibile perché indice della radice ed esponente non hanno 
divisori comuni;
 \item $\sqrt[8]{2^6}=2^{\frac{6}{8}}=2^{\frac{3}{4}}=\sqrt[4]{2^3}$
 \item $\sqrt[6]{\left(\dfrac{1}{5}\right)^{-9}}=\sqrt[6]{5^9}=
        5^{\frac{9}{6}}=5^{\frac{3}{2}}=\sqrt[2]{5^3}$
 \item $\sqrt{10^{-4}}=10^{-\frac{4}{2}}=10^{-2}=\frac 1{100}$
 \item $\sqrt{30\cdot 27\cdot 10}$: scomponendo in fattori primi otteniamo 
  \[\sqrt{30\cdot 27\cdot 10}=
    \sqrt{2\cdot 3\cdot 5\cdot 3^3\cdot 2\cdot 5}=
    \sqrt{2^2\cdot 3^4\cdot 5^2}.\] 
    Osserviamo che tutti gli esponenti del radicando e l'indice della 
radice hanno un divisore, quindi 
    $\sqrt{2^2\cdot 3^4\cdot 5^2}=
     2^{\frac{2}{2}} \cdot 3^{\frac{4}{2}} \cdot 5^{\frac{2}{2}}=
     2\cdot 3^2\cdot 5=90$
\end{itemize}
\end{esempio}
% \end{exrig}

La proprietà invariantiva si può applicare per semplificare i radicali se 
la base del radicando è positiva o nulla, se fosse negativa si potrebbe 
perdere la concordanza del segno. 
Per esempio~$\sqrt[10]{(-2)^6}\neq \sqrt[5]{(-2)^3}$, infatti il primo 
radicando è positivo mentre il secondo è negativo.

Invece $\sqrt[9]{(-2)^3}=\sqrt[3]{-2}$ perché in questo caso la 
concordanza del segno è conservata, infatti pur essendo la base negativa, 
l'esponente resta dispari, conservando il segno della base.

Se il radicando ha base negativa e nella semplificazione il suo esponente 
passa da pari a dispari è necessario mettere il radicando in valore 
assoluto: 
$\sqrt[10]{(-2)^6}=\sqrt[5]{\left|-2^3\right|}$.

Se il radicando è letterale si segue la stessa procedura: ogni volta che 
studiando il segno del radicando si trova che la base può essere 
negativa, se l'esponente del radicando passa da pari a dispari, si mette il 
modulo per garantire la concordanza del segno:
$\sqrt[10]{x^6}=\sqrt[5]{\left|x^3\right|}$, $\CE \forall x \in \insR$.

\section{Moltiplicazione e divisione di radici}
\label{sec:radicali_moltiplicazione}

Tutte queste regole si riducono all'applicazione delle proprietà delle 
potenze dopo aver tradotto le radici in potenze con esponente razionale.

\paragraph{Moltiplicazione e divisione di radici con lo stesso radicando}

Il prodotto di due radici che hanno lo stesso radicando lo si ottiene 
applicando le proprietà delle potenze:
\[\sqrt[m]a \cdot \sqrt[n]a=a^\frac{1}{m} \cdot a^\frac{1}{n} =
  a^\frac{1}{m} + a^\frac{1}{n}=
  a^\frac{n+m}{mn}\]

Analogamente per  il quoziente 
\[\sqrt[m]a \div \sqrt[n]a=a^\frac{1}{m} \div a^\frac{1}{n} =
  a^\frac{1}{m} - a^\frac{1}{n}=
  a^\frac{n-m}{mn}\]

% \begin{exrig}
 \begin{esempio}
Moltiplicazione e divisione di radici con lo stesso radicando.
\begin{itemize}
\item $\sqrt[4]6 \cdot \sqrt[3]6=6^{\frac 1 4}\cdot 6^{\frac 1 3}=
       6^{\frac 1 4+\frac 1 3}=6^{\frac 7{12}}=\sqrt[12]{6^7}$
\item $\sqrt[4]6:\sqrt[3]6=6^{\frac 1 4}:6^{\frac 1 3}=
       6^{\frac 1 4-\frac 1 3}=6^{-\frac 1{12}}=\frac 1{\sqrt[12]6}$.
\end{itemize}
 \end{esempio}
% \end{exrig}

\paragraph{Moltiplicazione e divisione di radici con lo stesso indice}

Il prodotto di due radici che hanno lo stesso indice è una radice che ha 
per indice lo stesso indice e per radicando il prodotto dei radicandi:
\[\sqrt[n]a \cdot \sqrt[n]b=
  a^\frac{1}{n} \cdot b^\frac{1}{n} =
  \tonda{ab}^\frac{1}{n} =
  \sqrt[n]{ab}\]

Allo stesso modo, il quoziente di due radici che hanno lo stesso indice è 
una radice che ha per indice lo stesso indice e per radicando il 
quoziente dei radicandi:
\[\frac{\sqrt[n]a}{\sqrt[n]b}=
  \frac{a^\frac{1}{n}}{b^\frac{1}{n}} =
  \tonda{\frac{a}{b}}^\frac{1}{n} =
  \sqrt[n]{\frac{a}{b}}\]

\emph{Nota}: 
 Se $a$ e $b$ sono entrambi negativi e l'indice è pari, l'espressione
 $\sqrt[n]a \cdot \sqrt[n]b$ (o $\sqrt[n]a \div \sqrt[n]b$) non ha valori 
 reali mentre $\sqrt[n]{ab}$ (o $\sqrt[n]{a \div b}$) ha il
 radicando positivo (meno per meno) e quindi ha valori reali. 
 Moltiplicando in questo modo  possiamo ottenere un risultato reale che 
 l'espressione di partenza non ha.
 
 Il prodotto di due numeri non reali può essere un numero reale!

% Per rendersi conto di questa proprietà si possono trasformare le radici 
% in 
% potenze ad esponenti razionali e applicare le proprietà delle potenze:
%  \[\sqrt[n]a\cdot \sqrt[n]b=a^{\frac 1 n}\cdot b^{\frac 1 n}=(ab)^{\frac 
% 1 
% n}=
%    \sqrt[n]{ab},\quad \sqrt[n]a:\sqrt[n]b=a^{\frac 1 n}:b^{\frac 1 n}=
%    \left(\dfrac a b\right)^{\frac 1 n}=\sqrt[n]{\dfrac a b}.\]

% \begin{exrig}
 \begin{esempio}
Moltiplicazione e divisione di radici con lo stesso indice.
\begin{itemize}
\item $\sqrt{2} \cdot \sqrt{3}=\sqrt{2 \cdot 3}=\sqrt 6$
\item $\sqrt{2} \cdot \sqrt{3}=
       2^{\frac{1}{2}} \cdot 3^{\frac{1}{2}}=(2  \cdot 3)^{\frac{1}{2}}=
       6^{\frac{1}{2}}=\sqrt 6$
\item $\frac{\sqrt[3]9}{\sqrt[3]{72}}=\sqrt[3]{\frac 9{72}}=
       \sqrt[3]{\frac 1 8}=\frac 1 2$
\end{itemize}
 \end{esempio}
% \end{exrig}

\paragraph{Moltiplicazione e divisione di radici con indici diversi}

Nel caso siano diversi sia i radicandi, sia gli indici, possiamo:
\begin{procedura}
\begin{enumeratea}
 \item scomporre in fattori irriducibili tutti i radicandi;
%  \item porre le condizioni di esistenza;
 \item scrivere le radici sotto forma di potenza;
 \item applicare le proprietà delle potenze.
\end{enumeratea}
\end{procedura}

% \begin{exrig}
 \begin{esempio}
Moltiplicazione e divisione di radici con indice diverso.
\begin{itemize}
\item $\sqrt{2}\cdot \sqrt[3]2=
       2^\frac{1}{2} \cdot 2^\frac{1}{3} = 2^{\frac{1}{2} + \frac{1}{3}} =
       2^{\frac{3+2}{6}} = 2^{\frac{5}{6}} =\sqrt[6]{2^5}$
\item 
  $\sqrt[3]{\frac{3}{2}} \cdot \sqrt[4]{\frac{8}{27}} : 
\sqrt[6]{\frac{2}{3}}=
   \left(\frac{3}{2}\right)^{\frac{1}{3}} \cdot 
   \left(\frac{2^3}{3^3}\right)^{\frac{1}{4}} : 
   \left(\frac{2}{3}\right)^{\frac{1}{6}}=
   \left(\frac{3}{2}\right)^{\frac{4}{12}} \cdot 
   \left(\frac{2^3}{3^3}\right)^{\frac{3}{12}} : 
   \left(\frac{2}{3}\right)^{\frac{2}{12}}=
   \left(\frac{3^4}{2^4} \cdot 
   \frac{2^9}{3^9} \cdot 
   \frac{3^2}{2^2}\right)^{\frac{1}{12}}=
   \left(\frac{2^3}{3^3}\right)^{\frac{1}{12}}=
   \left(\frac{2}{3}\right)^{\frac{3}{12}}=
   \left(\frac{2}{3}\right)^{\frac{1}{4}}=\sqrt[4]{\frac{2}{3}}$
\end{itemize}
 \end{esempio}

\section{Portare un fattore sotto il segno di radice}
\label{sec:radicali_portare_dentro}

Per portare un fattore dentro il segno di radice bisogna elevarlo 
all'indice 
della radice:
\begin{itemize*}
 \item $a\sqrt[n]b=\sqrt[n]{a^n\cdot b}$ se $n$ è pari e $a\ge 0$
 \item $a\sqrt[n]b=-\sqrt[n]{a^n\cdot b}$ se $n$ è pari e $a<0$
 \item $a\sqrt[n]b=\sqrt[n]{a^n\cdot b}$ se $n$ è dispari.
\end{itemize*}

Ricordando che abbiamo posto $\sqrt[1]a=a$, portare un fattore sotto radice 
equivale a svolgere la moltiplicazione tra una radice di indice $1$ e una 
radice di indice qualsiasi.
% \begin{exrig}
 \begin{esempio}
 Portare un numero reale dentro il segno di radice.
 \begin{itemize}
 \item $2\cdot \sqrt[3]7=\sqrt[3]{2^3\cdot 7}=\sqrt[3]{56}$
 \item $3\cdot \sqrt{\frac 2{21}}=\sqrt{3^2\cdot \frac 2{21}}=
        \sqrt{9\cdot \frac 2{21}}=\sqrt{\frac 6 7}$
 \item $-\frac 1 2\sqrt{3}$\, lasciamo fuori dalla radice il segno meno 
       $-\frac 1 2\sqrt{3}=-\sqrt{\left(\frac 1 2\right)^2\cdot 3}=
        -\sqrt{\frac 3 4}$
 \item $-\frac 1 3\cdot \sqrt{12}=-\sqrt{\left(\frac 1 3\right)^2\cdot 12}=
        -\sqrt{\frac 1 9\cdot 12}=-\sqrt{\frac 4 3}$
 \item $(1-\sqrt{2})\cdot \sqrt{3}=-(\sqrt{2}-1)\cdot \sqrt{3}=
        -\sqrt{(\sqrt{2}-1)^2\cdot 3}$
 \item $-2\sqrt[3]5=\sqrt[3]{(-2)^3\cdot 5}=\sqrt[3]{-40}$.
 \end{itemize}
 \end{esempio}
 
\section{Portare un fattore fuori dal segno di radice}
\label{sec:radicali_portare_fuori}

È possibile portare fuori dal segno di radice quei fattori aventi come 
esponente un numero che sia maggiore o uguale all'indice della radice. 
In generale si inizia scomponendo in fattori irriducibili il radicando, 
ottenendo un radicale del tipo $\sqrt[n]{a^m}$ con $m\ge n$.

\paragraph{I° modo:} si esegue la divisione intera $m:n$ ottenendo un 
quoziente $q$ e un resto $r$. Per la proprietà della divisione si 
ha $m=n\cdot q+r$ quindi $\sqrt[n]{a^m}=\sqrt[n]{a^{n\cdot q+r}}$ 
e per le proprietà delle potenze 
$\sqrt[n]{a^{n\cdot q+r}}=\sqrt[n]{(a^q)^n\cdot a^r}$ 
e per la regola del prodotto di due radici con medesimo indice si ottiene:

\[\sqrt[n]{a^{n\cdot q+r}}= \sqrt[n]{(a^q)^n\cdot a^r}=
  \sqrt[n]{(a^q)^n}\cdot \sqrt[n]{a^r}=
  a^q\cdot \sqrt[n]{a^r}\text{ con } r<n.\]
Notiamo che il fattore ``fuori`` dalla radice ha per esponente il quoziente 
della divisione intera, mentre il fattore che rimane ``dentro`` ha per 
esponente il resto della divisione stessa.

 $\sqrt[3]{a^8}=\ldots $ eseguiamo la divisione $8:3$ con $q=2$ e $r=2$, 
 otteniamo $\sqrt[3]{a^8}=a^2\cdot \sqrt[3]{a^2}$.

\paragraph{II° modo:} si può trasformare la potenza del radicando nel 
prodotto di due potenze con la stessa base; una avente esponente multiplo 
dell'indice della radice e l'altra avente per esponente la differenza tra 
l'esponente iniziale e il multiplo trovato. Consideriamo il seguente 
esempio:

 $\sqrt[3]{a^8}=\ldots $ il multiplo di $3$ più vicino a $8$ è $6$ quindi, 
 otteniamo 
\[\sqrt[3]{a^8}=\sqrt[3]{a^6\cdot a^2}=\sqrt[3]{a^6}\cdot \sqrt[3]{a^2}=
  a^2\cdot \sqrt[3]{a^2}.\]

% \begin{exrig}
 \begin{esempio}
 Portare un numero reale fuori dal segno di radice.
\begin{itemize}
 \item $\sqrt{1200}$ Si scompone in fattori primi il radicando 
       $1200=2^4\cdot 5^2\cdot 3$ ne segue allora che 
       $\sqrt{1200}=\sqrt{2^4\cdot 5^2\cdot 3}=2^2\cdot 
5\sqrt{3}=20\sqrt{3}$
 \item $\sqrt{75}=\sqrt{5^2\cdot 3}=5\sqrt{3}$
 \item $\sqrt{720}=\sqrt{2^4\cdot 3^2\cdot 5}=
        2^2\cdot 3\cdot \sqrt 5=12\sqrt 5$
\end{itemize}
 \end{esempio}
% \end{exrig}

\section{Potenza di radice e radice di radice}
\label{sec:radicali_potenza}

Per elevare a potenza una radice si eleva a quella potenza il radicando: 
$\left(\sqrt[n]a\right)^m=\sqrt[n]{a^m}$.
Si capisce il perché di questa proprietà trasformando, come negli altri 
casi, la radice in potenza con esponente frazionario:
$\left(\sqrt[n]a\right)^m=\left(a^{\frac 1 n}\right)^m=a^{\frac m n}=
 \sqrt[n]{a^m}$.
% % \newpage
% % \begin{exrig}
%  \begin{esempio}
%  Potenza di radice.
%  \begin{multicols}{2}
%  \begin{itemize}
%  \item $\left(\sqrt{2}\right)^2=\sqrt{2^2}=2$
%  \item $\left(\sqrt[3]{2ab^2c^3}\right)^2=\sqrt[3]{4a^2b^4c^6}$.
%  \end{itemize}
%  \end{multicols}
%  \end{esempio}
% % \end{exrig}

La radice di un'altra radice è uguale a una radice con lo stesso radicando 
e con indice il prodotto degli indici delle radici: 
$\sqrt[m]{\sqrt[n]a}=\sqrt[m\cdot n]a$. 
Anche questa proprietà si può spiegare con le proprietà delle potenze 
trasformando la radice in potenza con esponente frazionario: 
$\sqrt[m]{\sqrt[n]a}=\left(a^{\frac 1 n}\right)^{\frac 1 m}=
 a^{\frac 1{mn}}=\sqrt[m\cdot n]a$

% \begin{exrig}
 \begin{esempio}
 Radice di radice.
 \begin{multicols}{2}
 \begin{itemize}
 \item $\left(\sqrt{2}\right)^2=\sqrt{2^2}=2$
 \item $\sqrt{\sqrt{2}}=\sqrt[2\cdot 2]2=\sqrt[4]2$
 \end{itemize}
 \end{multicols}
 \end{esempio}

 \begin{esempio}
 Data l'espressione $E=\sqrt[5]{3\cdot\sqrt{2}}$ ridurla ad unico radicale.

 In questo caso non possiamo subito applicare la regola annunciata, ma 
dobbiamo  portare il fattore esterno dentro la radice più interna ottenendo 
 $\sqrt[5]{\sqrt{3^2 \cdot 2}}=\sqrt[10]{18}$.

Osserviamo che l'espressione $E=\sqrt[5]{3+\sqrt{2}}$ non si può ridurre 
ad unico radicale, se non sotto determinate condizioni che analizzeremo in 
seguito.
 \end{esempio}
% \end{exrig}
% \vspazio\ovalbox{\risolvii \ref{ese:2.46}, \ref{ese:2.47}, 
% \ref{ese:2.48}, 
% \ref{ese:2.49}}

\section{Somma di radicali}
\label{sec:radicali_somma}

Si dice \emph{radicale} un'espressione del tipo $a\sqrt[n]b$ con $a$ e $b$ 
numeri reali, $b{\geq}0$ ed $n\in \insN$. Il numero $a$ prende il nome di 
\emph{coefficiente} del radicale.

Operare con i radicali è simile al modo di operare con i monomi. Infatti è 
possibile effettuare somme algebriche soltanto se i radicali hanno lo 
stesso 
indice e lo stesso radicando, mentre si possono sempre effettuare 
moltiplicazioni e divisioni dopo averli ridotti allo stesso indice.
\begin{definizione}
 Due radicali si dicono \emph{simili} se hanno lo stesso indice e lo stesso 
 radicando.
\end{definizione}

È possibile effettuare somme algebriche soltanto se i radicali sono simili, 
si eseguono le somme allo stesso modo in cui si eseguono le somme 
algebriche 
dei monomi.

Attenzione l'operazione $\sqrt{2}+\sqrt{3}=\sqrt 5$\, è errata in quanto i 
radicali addendi non sono simili.

\newpage %-----------------------------------------------

% \begin{exrig}
 \begin{esempio}
Esegui le seguenti somme di radicali.
\begin{itemize}
 \item $\sqrt 8+\sqrt{2}=\sqrt{2^3}+\sqrt{2}=2\sqrt{2}+\sqrt{2}=3\sqrt{2}$
 \item $2\sqrt{45}-\sqrt{80}=2\sqrt{3^2\cdot 5}-\sqrt{2^4\cdot 5}=
        2\cdot 3\cdot \sqrt 5-2^2\sqrt 5=6\sqrt 5-4\sqrt 5=2\sqrt 5$
 \item $\sqrt{2}+\sqrt{3}$
  non si può eseguire perché i radicali non sono simili;
 \item $\sqrt[3]2+\sqrt{2}$
  non si può eseguire perché i radicali non sono simili;
 \item $\sqrt{3}+\sqrt{3}=2\sqrt{3}$
 \item $2\sqrt 5-\sqrt 5=\sqrt 5$
 \item $\frac 1 2\sqrt 7-\frac 4 3\sqrt 7=
        \left(\frac 1 2-\frac 4 3\right)\sqrt 7=\frac{3-8} 6\sqrt 7=
        -\frac 5 6\sqrt 7$
 \item $3\sqrt{2}+2\sqrt{3}-2\sqrt{2}+3\sqrt{3}=(3-2)\sqrt{2}+(2+3)\sqrt{3}=
        \sqrt{2}+5\sqrt{3}$ abbiamo sommato i radicali simili;
%  \item $2\sqrt a+3\sqrt a=5\sqrt a,\,\CE a\ge 0$
%  \item $\sqrt[4]{a^5}+\sqrt[4]{a^3}\cdot \sqrt 
% a+\sqrt[4]{a^6}:\sqrt[4]a$. 
%   Poniamo le condizioni di esistenza $a>0$ e svolgiamo i calcoli: 
%   $\sqrt[4]{a^5}+\sqrt[4]{a^3\cdot a^2}+\sqrt[4]{a^6:a}=
%    \sqrt[4]{a^5}+\sqrt[4]{a^5}+\sqrt[4]{a^5}=3\sqrt[4]{a^5}=
%    3\sqrt[4]{a^4\cdot a}=3a\sqrt[4]a$.
\end{itemize}
 \end{esempio}
% \end{exrig}

Per semplificare le espressioni che seguono, useremo le procedure di 
calcolo dei polinomi.

% \newpage

% \begin{exrig}
 \begin{esempio}
Esegui le seguenti operazioni con i radicali.
\begin{itemize}
 \item $(1+\sqrt{2})(3\sqrt{2}-1)=3\sqrt{2}-1+3\sqrt {2^2}-\sqrt{2}=
        3\sqrt{2}-1+3\cdot 2-\sqrt{2}=2\sqrt{2}+5$
 \item $(\sqrt{3}+1)^2=(\sqrt{3})^2+(1)^2+2\cdot \sqrt{3}\cdot 1=
        3+1+2\sqrt{3}=4+2\sqrt{3}$
 \item $(\sqrt{3}-\sqrt{2})^2=(\sqrt{3})^2+(\sqrt{2})^2+2 \sqrt{3} 
(-\sqrt{2})=
        3+2-2\sqrt 6=5-2\sqrt 6$
 \item $(3+\sqrt{2}+\sqrt{3})^2=
        (3)^2+(\sqrt{2})^2+(\sqrt{3})^2+6 \sqrt{2}+6 \sqrt{3}+2 \sqrt{2} 
\sqrt{3}=
        14+6\sqrt{2}+6\sqrt{3}+2\sqrt 6$
 \item $(\sqrt{2}+4)(3-\sqrt{2})=3\sqrt{2} 
+\sqrt{2}(-\sqrt{2})+12+4(-\sqrt{2})=
        3\sqrt{2}-2+12-4\sqrt{2}=10-\sqrt{2}$
 \item $(\sqrt{2}-3)^3=(\sqrt{2})^3-9(\sqrt{2})^2+27\sqrt{2}+(-3)^3=
        2\sqrt{2}-18+27\sqrt{2}-27=29\sqrt{2}-45$.
\end{itemize}
 \end{esempio}
% \end{exrig}

% Le espressioni con radicali possono essere trasformate in potenze con 
% esponente frazionario per poi applicare le proprietà delle potenze:
% 
% % \begin{exrig}
% Trasforma i radicali in potenze con esponente frazionario applicando le 
% proprietà delle potenze.
%  \begin{esempio}
% % \begin{itemize}
% %  \item
%   \[\dfrac{\sqrt a\cdot \sqrt[3]{a^2\cdot b}}{\sqrt[6]{a^5\cdot b}}=
%    \dfrac{a^{\frac 1 2}\cdot a^{\frac{2}{3}}\cdot 
%           b^{\frac 1 3}}{a^{\frac 5 6}\cdot b^{\frac 1 6}}=
%    a^{\frac 1 2+\frac{2}{3}-\frac 5 6}\cdot b^{\frac 1 3-\frac 1 6}=
%    a^{\frac 2 6}\cdot b^{\frac 1 6}=\sqrt[6]{a^2b}\]
%  \end{esempio}
%  \begin{esempio}
% %  \item
% %   $\sqrt{\dfrac{\sqrt[3]{a^2}\cdot \sqrt b}{\sqrt[5]{a^2}}}\cdot 
% %    \sqrt[3]{\dfrac{\sqrt[4]{a^6b}}{a\sqrt[3]b}}$.
%  \begin{align*}
%  \sqrt{\frac{\sqrt[3]{a^2}\cdot \sqrt b}{\sqrt[5]{a^2}}}\cdot
%  \sqrt[3]{\frac{\sqrt[4]{a^6b}}{a\sqrt[3]b}}&=
%  \left(\frac{a^{\frac{2}{3}}\cdot b^{\frac 1 2}}
%  {a^{\frac 2 5}}\right)^{\frac 1 2}\left(\frac{a^{\frac 3 2}\cdot 
%   b^{\frac 1 4}}{ab^{\frac 1 3}}\right)^{\frac 1 3}\\ &=
%   \frac{a^{\frac 1 3}\cdot b^{\frac 1 4}}{a^{\frac 1 5}}\cdot 
%   \frac{a^{\frac 1 2}\cdot b^{\frac 1{12}}}{a^{\frac 1 3}\cdot b^{\frac 1 
% 9}}\\
%  &=a^{\frac 1 3-\frac 1 5\;+\frac 1 2-\frac 1 3}\cdot 
%    b^{\frac 1 4+\frac 1{12}-\frac 1 9}\\
%  &=a^{\frac 3{10}}\cdot b^{\frac 2 9}\\
%  &=\sqrt[10]{a^3}\cdot \sqrt[9]{b^2};
%  \end{align*}
%  \end{esempio}
% %  \begin{esempio}
% % %  \item 
% % %    $\sqrt[6]{\dfrac{x^3\cdot \sqrt[3]{xy^2}}{x^2-\sqrt{xy}}}$
% %  \begin{align*}
% %  \sqrt[6]{\frac{x^3\cdot \sqrt[3]{xy^2}}{x^2-\sqrt{xy}}}&=
% %  \left(\frac{x^3\cdot (xy^2)^{\frac 1 3}}{x^2-(xy)^{\frac 1 2}}\right)^
% %    {\frac 1 6}\\
% %  &=\left(\frac{x^3\cdot x^{\frac 1 3}\cdot y^{\frac{2}{3}}}
% %               {x^2-x^{\frac 1 2}\cdot y^{\frac 1 2}}\right)^{\frac 1 
% 6}\\
% %  &=\left(\frac{x^{\frac{10} 3}\cdot y^{\frac{2}{3}}}
% %               {x^{\frac 1 2}\cdot \left(x^{\frac 3 2}-y^{\frac 1 
% 2}\right)}
% %               \right)^{\frac 1 6}\\
% %  &=\left[x^{\frac{17} 6}\cdot y^{\frac{2}{3}}\cdot
% %    \left(x^{\frac 3 2}-y^{\frac 1 2}\right)^{-1}\right]^{\frac 1 6}\\
% %  &=x^{\frac{17}{36}}\cdot y^{\frac 1 9}\cdot 
% %       \left(x^{\frac 3 2}-y^{\frac 1 2}\right)^{-\frac 1 6}.
% %  \end{align*}
% % % \end{itemize}
% %  \end{esempio}
% % \end{exrig}
% \vspazio\ovalbox{\risolvii \ref{ese:2.50}, \ref{ese:2.51}, 
% \ref{ese:2.52}, 
% \ref{ese:2.53}, \ref{ese:2.54}, \ref{ese:2.55}, \ref{ese:2.56}, 
% \ref{ese:2.57}, \ref{ese:2.58}, \ref{ese:2.59}, \ref{ese:2.60}, 
% \ref{ese:2.61}, \ref{ese:2.62},}
% 
% \vspazio\ovalbox{\ref{ese:2.63}, \ref{ese:2.64}, \ref{ese:2.65}, 
% \ref{ese:2.66}, \ref{ese:2.67}}

\section{Razionalizzazione del denominatore di una frazione}
\label{sec:radicali_razionalizzazione}

Nel calcolo di espressioni che contengono radicali può capitare che al 
denominatore compaiano dei radicali. 
Per migliorare l'approssimazione si cerca di evitare questa situazione e 
operare affinché non compaiano radicali al denominatore. 
Questa operazione prende il nome di \emph{razionalizzazione del 
denominatore}.

Razionalizzare il denominatore di una frazione vuol dire trasformare una 
frazione in una frazione equivalente avente per denominatore un'espressione 
nella quale non compaiano radici.

\paragraph{I° Caso:} la frazione è del tipo $\dfrac a{\sqrt b}$ con $b>0$.

Per razionalizzare il denominatore di una frazione di questo tipo basta 
moltiplicare numeratore e denominatore per $\sqrt b$, che prende il nome di 
fattore razionalizzante: 

\[\dfrac {a} {\sqrt b} = \dfrac{a\sqrt b}{\sqrt b\cdot \sqrt b}=
  \dfrac{a\sqrt b}b.\]

\newpage %-----------------------------------------------

% \begin{exrig}
 \begin{esempio}
Razionalizza il denominatore delle seguenti espressioni.
\begin{itemize}
 \item $\dfrac 1{\sqrt{2}}=\dfrac{1\cdot \sqrt{2}}{\sqrt{2}\cdot \sqrt{2}}=
        \dfrac{\sqrt{2}} 2$
 \item $\dfrac 3{2\sqrt{3}}=\dfrac{3\sqrt{3}}{2\sqrt{3}\sqrt{3}}=
        \dfrac{3\sqrt{3}}{2\cdot 3}=\dfrac{\sqrt{3}} 2$
%  \item $\dfrac{a^2-1}{\sqrt{a-1}}=
%         \dfrac{(a^2-1)\sqrt{a-1}}{\sqrt{a-1}\sqrt{a-1}}=
%         \dfrac{(a^2-1)\sqrt{a-1}}{a-1}=
%         \dfrac{(a-1)(a+1)\sqrt{a-1}}{a-1}=(a+1)\sqrt{a-1}$.
\end{itemize}
 \end{esempio}
% \end{exrig}

\paragraph{II° Caso:}
 la frazione è del tipo $\dfrac a{\sqrt[n]{b^m}}$ con $b>0 \wedge n>m$.

In questo caso il fattore razionalizzante è $\sqrt[n]{b^{n-m}}$. Infatti si 
ha:
\begin{equation*}
\dfrac a{\sqrt[n]{b^m}}=
\dfrac{a\sqrt[n]{b^{n-m}}}{\sqrt[n]{b^m}\cdot \sqrt[n]{b^{(n-m)}}}=
\dfrac{a\sqrt[n]{b^{n-m}}}{\sqrt[n]{b^m\cdot b^{n-m}}}=
\dfrac{a\sqrt[n]{b^{n-m}}}{\sqrt[n]{b^n}}=\dfrac{a\sqrt[n]{b^{n-m}}} b
\end{equation*}
Se abbiamo un'espressione in cui l'esponente del radicando è maggiore o 
uguale all'indice della radice, prima di razionalizzare, possiamo portare 
fuori dalla radice un fattore.

% \begin{exrig}
 \begin{esempio}
Razionalizza il denominatore delle seguenti espressioni.
\begin{itemize}
 \item $\dfrac 1{\sqrt[3]2}$:\, il fattore razionalizzante è 
$\sqrt[3]{2^2}$ 
  quindi:
  \[\dfrac 1{\sqrt[3]2}=
  \dfrac{1\cdot \sqrt[3]{2^2}}{\sqrt[3]2\cdot \sqrt[3]{2^2}}=
  \dfrac{\sqrt[3]4}{\sqrt[3]{2^3}}=\dfrac{\sqrt[3]4} 2;\]
%  \item $\dfrac{ab}{\sqrt[4]{xa^2b^3}}$:
%  il fattore razionalizzante è $\sqrt[4]{x^3a^2b}$
%  quindi: 
%  \[\dfrac{ab}{\sqrt[4]{xa^2b^3}}=
%  \dfrac{ab\cdot \sqrt[4]{x^3a^2b}}{\sqrt[4]{xa^2b^3}\cdot 
% \sqrt[4]{x^3a^2b}}=
%  \dfrac{ab\sqrt[4]{x^3a^2b}}{\sqrt[4]{x^4a^4b^4}}=
%  \dfrac{ab\sqrt[4]{x^3a^2b}}{xab}=\dfrac{\sqrt[4]{x^3a^2b}} x;\]
%  \item $\dfrac 1{\sqrt[3]{b^5}}=\dfrac 1{b\sqrt[3]{b^2}}=
%         \dfrac{1\cdot \sqrt[3]b}{b\sqrt[3]{b^2}\cdot \sqrt[3]b}=
%         \dfrac{\sqrt[3]b}{b^2}$.
\end{itemize}
 \end{esempio}
% \end{exrig}


\paragraph{III° Caso:} 
la frazione è del tipo $\dfrac x{\sqrt a+\sqrt b}$ 
oppure $\dfrac x{\sqrt a-\sqrt b}$ con $a>0 \wedge b>0$.

Per questo tipo di frazione occorre sfruttare il prodotto notevole 
$(a+b)(a-b)=a^2-b^2$. Il fattore razionalizzante nel primo caso è $\sqrt 
a-\sqrt b$, nel secondo è $\sqrt a+\sqrt b$.
Sviluppiamo solo il primo caso, poiché il secondo è del tutto analogo:
\begin{equation*}
\dfrac x{\sqrt a+\sqrt b}=
\dfrac{x\cdot (\sqrt a-\sqrt b)}{(\sqrt a+\sqrt b)\cdot (\sqrt a-\sqrt b)}=
\dfrac{x(\sqrt a-\sqrt b)}{\sqrt{a^2}-\sqrt{b^2}}=
\dfrac{x(\sqrt a-\sqrt b)}{a-b}
\end{equation*}

% \begin{exrig}
 \begin{esempio}
Razionalizza il denominatore delle seguenti espressioni.
\begin{itemize}
 \item $\dfrac 2{\sqrt{3}-\sqrt 5}=
 \dfrac{2\cdot (\sqrt{3}+\sqrt 5)}{(\sqrt{3}-\sqrt 5)\cdot (\sqrt{3}+\sqrt 
5)}=
 \dfrac{2(\sqrt{3}+\sqrt 5)}{\sqrt{3^2}-\sqrt{5^2}}=
 \dfrac{2(\sqrt{3}+\sqrt 5)}{-2}=-(\sqrt{3}+\sqrt 5)$
 \item $\dfrac{\sqrt{2}}{3-\sqrt{2}}=
 \dfrac{\sqrt{2}\cdot (3+\sqrt{2})}{(3-\sqrt{2})\cdot (3+\sqrt{2})}=
 \dfrac{\sqrt{2}(3+\sqrt{2})}{3^2-\sqrt{2^2}}=
 \dfrac{\sqrt{2}(3+\sqrt{2})}{9-2}= \dfrac{\sqrt{2}(3+\sqrt{2})} 7$
%  \item $\dfrac{1+\sqrt a}{1-\sqrt a}=
%  \dfrac{(1+\sqrt a)\cdot (1+\sqrt a)}{(1-\sqrt a)(1+\sqrt a)}=
%  \dfrac{(1+\sqrt a)^2}{1-\sqrt{a^2}}=\dfrac{1+2\sqrt a+a}{1-a}$ 
%  con $a\ge 0\wedge a\neq 1$.
\end{itemize}
 \end{esempio}
% \end{exrig}

% \paragraph{IV° Caso:}
% la frazione è del tipo $\dfrac x{\sqrt a+\sqrt b+\sqrt c}$
% 
% Anche in questo caso si utilizza il prodotto notevole della differenza di 
% quadrati, solo che va ripetuto più volte.
% 
% % \begin{exrig}
%  \begin{esempio}
% Razionalizza $\dfrac 1{\sqrt{2}+\sqrt{3}+\sqrt 5}$.
% 
% Il fattore di razionalizzazione è in questo caso $\sqrt{2}+\sqrt{3}-\sqrt 
% 5$ 
% quindi:
%  \[\dfrac 1{\sqrt{2}+\sqrt{3}+\sqrt 5}\cdot 
%    \dfrac{\sqrt{2}+\sqrt{3}-\sqrt 5}{\sqrt{2}+\sqrt{3}-\sqrt 5}=
%    \dfrac{\sqrt{2}+\sqrt{3}-\sqrt 5}{(\sqrt{2}+\sqrt{3})^2-5}=
%    \dfrac{\sqrt{2}+\sqrt{3}-\sqrt 5}{2+3+2\sqrt 6-5}=
%    \dfrac{\sqrt{2}+\sqrt{3}-\sqrt 5}{2\sqrt 6};\]
%  ora il fattore razionalizzante di questa frazione è $\sqrt 6$:
%  \[\dfrac{\sqrt{2}+\sqrt{3}-\sqrt 5}{2\sqrt 6}\cdot \dfrac{\sqrt 6}{\sqrt 
% 6}=
%  \dfrac{\sqrt{12}+\sqrt{18}-\sqrt{30}}{2\cdot 6}=
%  \dfrac{2\sqrt{3}+3\sqrt{2}-\sqrt{30}}{12}.\]
%  \end{esempio}
% % \end{exrig}
% 
% \paragraph{V° Caso:} 
% la frazione è del tipo $\dfrac x{\sqrt[3]a+\sqrt[3]b}$.
% 
% In questo caso si utilizza il prodotto notevole 
% $(a+b)(a^2-ab+b^2)=a^3+b^3$ e quello analogo~$(a-b)(a^2+ab+b^2)=a^3-b^3$.
% \begin{align*}
% \frac x{\sqrt[3]a+\sqrt[3]b}=
% \frac x{\sqrt[3]a+\sqrt[3]b}\cdot 
% \frac{\sqrt[3]{a^2}-\sqrt[3]{ab}+\sqrt[3]{b^2}}
%      {\sqrt[3]{a^2}-\sqrt[3]{ab}+\sqrt[3]{b^2}}=&
%      \frac{x\left(\sqrt[3]{a^2}-\sqrt[3]{ab}+\sqrt[3]{b^2}\right)}
%           {(\sqrt[3]a)^3+(\sqrt[3]b)^3}\\
% &=\frac{x\left(\sqrt[3]{a^2}-\sqrt[3]{ab}+\sqrt[3]{b^2}\right)}{a+b}.
% \end{align*}
% 
% % \begin{exrig}
%  \begin{esempio}
% Razionalizza $\dfrac 1{\sqrt[3]{2}-\sqrt[3]{3}}$.
% 
% Il fattore di razionalizzazione è in questo caso 
% $\sqrt[3]{2^2}+\sqrt[3]{2\cdot 3}+\sqrt[3]{3^2}$ quindi:
%  \[\dfrac{1\cdot \left(\sqrt[3]{2^2}+\sqrt[3]{2\cdot 
% 3}+\sqrt[3]{3^2}\right)}
%          {\left(\sqrt[3]2-\sqrt[3]3\right)\cdot \left(\sqrt[3]{2^2}+
%           \sqrt[3]{2\cdot 3}+\sqrt[3]{3^2}\right)}=
%    \dfrac{\sqrt[3]{2^2}+\sqrt[3]{2\cdot 3}+\sqrt[3]{3^2}}{2-3}=
%    -\left(\sqrt[3]4+\sqrt[3]6+\sqrt[3]9\right).\]
%  \end{esempio}
% % \end{exrig}
% \vspazio\ovalbox{\risolvii \ref{ese:2.68}, \ref{ese:2.69}, 
% \ref{ese:2.70}, 
% \ref{ese:2.71}, \ref{ese:2.72}, \ref{ese:2.73}, \ref{ese:2.74}, 
% \ref{ese:2.75}, \ref{ese:2.76}}

% \section{Radicali doppi}
% \label{sec:radicali_radicali_doppi}
% 
% Si dice radicale doppio un'espressione del tipo $\sqrt{a+\sqrt b}$ 
% oppure $\sqrt{a-\sqrt b}$.
% 
% I radicali doppi possono essere trasformati nella somma algebrica di due 
% radicali semplici se l'espressione $a^2-b$ è un quadrato perfetto. 
% La formula per ottenere la trasformazione in radicali semplici è:
% 
% \begin{equation*}
% \sqrt{a\pm \sqrt b}=
% \sqrt{\dfrac{a+\sqrt{a^2-b}} 2}\pm \sqrt{\dfrac{a-\sqrt{a^2-b}} 2}
% \end{equation*}
% 
% % \begin{exrig}
%  \begin{esempio}
% Trasforma, se possibile, i seguenti radicali doppi in radicali semplici.
% \begin{itemize}
%  \item $\sqrt{7-\sqrt{40}}=
%  \sqrt{\dfrac{7+\sqrt{49-40}} 2}-\sqrt{\dfrac{7-\sqrt{49-40}} 2}=
%  \sqrt{\dfrac{7+3} 2}-\sqrt{\dfrac{7-3} 2}=\sqrt 5-\sqrt{2}$
%  \item $\sqrt{2-\sqrt{3}}=
%  \sqrt{\dfrac{2+\sqrt{2^2-3}} 2}-\sqrt{\dfrac{2-\sqrt{2^2-3}} 2}=
%  \sqrt{\dfrac 3 2}-\sqrt{\dfrac 1 
% 2}=\dfrac{\sqrt{3}-\sqrt{2}}{\sqrt{2}}$, 
%  razionalizzando il denominatore si ottiene: 
%  $\dfrac{\sqrt{3}-\sqrt{2}}{\sqrt{2}}=
%  \dfrac{(\sqrt{3}-\sqrt{2})\cdot \sqrt{2}}{\sqrt{2}\cdot \sqrt{2}}=
%  \dfrac{\sqrt 6-\sqrt{2}} 2$
%  \item $\sqrt{7+2\sqrt 6}=\sqrt{7+\sqrt{24}}$
%  per applicare la formula abbiamo portato il fattore $2$ dentro la 
% radice: 
%  $\sqrt{7+\sqrt{24}}=
%  \sqrt{\dfrac{7-\sqrt{49-24}} 2}+\sqrt{\dfrac{7-\sqrt{49-24}} 2}=
%  \sqrt{\dfrac{7+5} 2}+\sqrt{\dfrac{7-5} 2}=\sqrt 6+1$
%  \item $\sqrt{5+\sqrt{3}}=
%  \sqrt{\dfrac{5+\sqrt{25-3}} 2}+\sqrt{\dfrac{5-\sqrt{25-3}} 2}=
%  \sqrt{\dfrac{5+\sqrt{22}} 2}+\sqrt{\dfrac{5-\sqrt{22}} 2}$
%  la formula non è stata di alcuna utilità in quanto il radicale doppio 
% non 
%  è stato eliminato.
% \end{itemize}
%  \end{esempio}
% % \end{exrig}
% % \vspazio\ovalbox{\risolvii \ref{ese:2.77}, \ref{ese:2.78}, 
% \ref{ese:2.79}}

\section{Equazioni, disequazioni e sistemi a coefficienti irrazionali}
\label{sec:radicali_equazioni}

Avendo imparato come operare con i radicali puoi risolvere equazioni, 
sistemi 
e disequazioni con coefficienti irrazionali.

\subsection{Equazioni di primo grado}
% \begin{exrig}
\begin{esempio}
Risolvi le seguenti equazioni.
\begin{itemize}
 \item 
  $\sqrt{3}x=9\Rightarrow \ x=\dfrac 9{\sqrt{3}} \Rightarrow 
  x=\dfrac 9{\sqrt{3}}\cdot \dfrac{\sqrt{3}}{\sqrt{3}}=
   \dfrac{9\sqrt{3}} 3=3\sqrt{3}$
 \item
  $(\sqrt{3}-1)x-\sqrt 6=2x-\sqrt{2}(3\sqrt{2}+1)+1$.
 \begin{align*}
&(\sqrt{3}-1)x-\sqrt 6=2x-\sqrt{2}(3\sqrt{2}+1)+1\\
 \Rightarrow&\sqrt{3}x-x-\sqrt 6=2x-3\cdot 2-\sqrt{2}+1\\
 \Rightarrow&\sqrt{3}x-3x=\sqrt 6-\sqrt{2}-5\\
 \Rightarrow &x(\sqrt{3}-3)=\sqrt 6-\sqrt{2}-5\\
 \Rightarrow &x=\dfrac{\sqrt 6-\sqrt{2}-5}{\sqrt{3}-3}.
 \end{align*}
Razionalizziamo ora il denominatore:
 \[x=\dfrac{\sqrt 6-\sqrt{2}-5}{\sqrt{3}-3}\cdot 
     \dfrac{\sqrt{3}+3}{\sqrt{3}+3}=
     \dfrac{3\sqrt{2}+3\sqrt{6}-\sqrt{6}-3\sqrt{2}-5\sqrt{3}-15}{3-9}=
 \dfrac{2\sqrt 6-5\sqrt{3}-15}{-6}=\]
%  -\dfrac{\sqrt 6} 3+\dfrac{5\sqrt{3}} 6+\dfrac 5 2\]
\end{itemize}
\end{esempio}
% \end{exrig}

\subsection{Disequazioni di primo grado}
% \begin{exrig}
\begin{esempio}
Risolvi le seguenti disequazioni.
 \begin{itemize}
 \item $(\sqrt{3}-1)x\le \sqrt{3}$ il coefficiente dell'incognita è 
positivo, 
 quindi: $x\le \dfrac{\sqrt{3}}{\sqrt{3}-1}$ e poi razionalizzando 
 $x\le \dfrac{3+\sqrt{3}} 2$
 \item $2x\cdot(1-\sqrt{2})\ge -3\sqrt{2}$ il coefficiente dell'incognita è 
 negativo, quindi $x\le \dfrac{-3\sqrt{2}}{2(1-\sqrt{2})}$ e poi 
 razionalizzando $x\le 3+\dfrac 3 2\sqrt{2}$.
 \end{itemize}

\end{esempio}
% \end{exrig}

\subsection{Sistemi di primo grado}
% \begin{exrig}
\begin{esempio}
Risolvi $\left\{\begin{array}{l}
 {x(2+\sqrt{2})+y=\sqrt{2}(2+x)}\\
 {x-(\sqrt{2}+1)y=-\dfrac{\sqrt{2}} 2(1+2y)}
 \end{array}
\right..$

Eseguiamo i calcoli per ottenere la forma canonica:

\[\left\{\begin{array}{l}
 {2x+x\sqrt{2}+y=2\sqrt{2}+x\sqrt{2}}\\
 {x-y\sqrt{2}-y=-\dfrac{\sqrt{2}} 2-y\sqrt{2}}
 \end{array}
\right.\Rightarrow
\left\{\begin{array}{l}
 {2x+y=2\sqrt{2}}\\
 {x-y=-\dfrac{\sqrt{2}} 2}
 \end{array}
\right.\]
e con il metodo di riduzione, sommando le due equazioni otteniamo:
\[\left\{\begin{array}{l}
 {3x=2\sqrt{2}-\dfrac{\sqrt{2}} 2}\\
 {y=2\sqrt{2}-2x}
 \end{array}
\right.\Rightarrow
\left\{\begin{array}{l}
 {x=\dfrac{\sqrt{2}} 2}\\
 {y=2\sqrt{2}-2\dfrac{\sqrt{2}} 2}
 \end{array}
\right.\Rightarrow
\left\{\begin{array}{l}
 {x=\dfrac{\sqrt{2}} 2}\\
 {y=\sqrt{2}}
 \end{array}
\right..\]
\end{esempio}
% \end{exrig}
% \vspazio\ovalbox{\risolvii \ref{ese:2.80}, \ref{ese:2.81}, 
% \ref{ese:2.82}, 
% \ref{ese:2.83}, \ref{ese:2.84}, \ref{ese:2.85}, \ref{ese:2.86}, 
% \ref{ese:2.87}, \ref{ese:2.88}, \ref{ese:2.89}}

\section{Radicandi letterali}
\label{sec:radicali_letterali}

\subsection{Condizioni di esistenza}
\label{subsec:radicali_condizioni_esistenza}

Le stesse regole che valgono per le radici con radicali numerici valgono 
anche se il radicando contiene delle lettere, ma qui abbiamo una 
complicazione.
Osserviamo i seguenti esempi:

\begin{multicols}{4}
 \begin{enumerate}
  \item \(\sqrt{-7}=\dots\)
  \item \(\sqrt{+7}=\dots\) 
  \item \(\sqrt[3]{-7}=\dots\)
  \item \(\sqrt[3]{+7}=\dots\)
  \item \(\sqrt{-a}=\dots\)
  \item \(\sqrt{+a}=\dots\)
  \item \(\sqrt[3]{-a}=\dots\)
  \item \(\sqrt[3]{+a}=\dots\)
 \end{enumerate}
\end{multicols}

Discutiamo i diversi casi presentati:

 \begin{enumerate}
  \item Questa operazione Non Ha Risultato Reale.
  \item Questa operazione ha per risultato il numero reale che elevato al 
quadrato dà come risultato 7, un numero reale che è maggiore 
di~2,6457513 e minore di~2,6457514
  \item Questa operazione ha per risultato il numero reale che elevato al 
cubo dà come risultato $-7$, un numero reale che è maggiore 
di~$-1,912931183$ e minore di~$-1,912931182$
  \item Questa operazione ha per risultato il numero reale che elevato al 
cubo dà come risultato $+7$, un numero reale che è maggiore 
di~$+1,912931182$ e minore di~$+1,912931183$
  \item Questo caso assomiglia al primo solo superficialmente. Siamo sicuri 
che $-a$ sia un numero negativo? No! Dipende da qual è il numero 
rappresentato da $a$. Se $a$ rappresenta un numero negativo allora $-a$ 
rappresenta un numero positivo. Quindi non possiamo dire che $-a$ non dia 
un risultato reale. \(\sqrt{-a}\) rappresenta un numero reale se $a$ è 
minore o uguale a zero.
  \item Per le considerazioni precedenti in questo caso~\(\sqrt{+a}\) 
rappresenta un numero reale solo se $a$ è maggiore o uguale a zero.
  \item In questo caso e nel seguente l'espressione rappresenta sempre un 
numero reale.
 \end{enumerate}

 Conclusioni:
 \begin{itemize}
  \item Se l'indice della frazione è dispari non sorge alcun problema.
  \item Se l'indice della frazione è pari:
  \begin{itemize}
   \item Se il radicando è un'espressione numerica, basta svolgerla per 
vedere se la radice dà un risultato reale;
   \item Se il radicando è un'espressione letterale bisogna studiare il 
segno di questa espressione per capire quando la radice dà un risultato 
reale.
  \end{itemize}
 \end{itemize}

Vediamo alcuni esempi.

% \begin{exrig}
\begin{esempio}
Condizioni di esistenza
 \begin{itemize}
 \item $\sqrt x$:\quad $\CE x\ge 0$
 \item $\sqrt[3]x$:\quad $\CE \forall x\in \insR$
 \item $\sqrt{-x}$:\quad $\CE x\le 0$
 \item $\sqrt[3]{-x}$:\quad $\CE \forall x\in \insR$
 \item $\sqrt{x-1}$:\quad $\CE x-1\ge 0 \Rightarrow x\ge 1$
 \item $\sqrt{a^2+1}$:\quad $\CE \forall a\in \insR$, infatti $a^2$ è 
sempre positivo pertanto $a^2+1>0, \forall a\in \insR$
 \item $\sqrt[3]{\frac 1{x+1}}$:\quad la radice cubica è definita per 
valori sia positivi sia negativi del radicando, tuttavia bisogna comunque 
porre la condizione che il denominatore della frazione non sia nullo, 
quindi $\CE x+1\neq 0 \Rightarrow x\neq -1$
 \item $\sqrt[4]{xy}$:\quad $\CE xy\ge 0$
 \item $\sqrt[5]{a^2(a-3)}$: poiché la radice ha indice dispari non occorre 
   porre alcuna condizione di esistenza.
\end{itemize}
\end{esempio}

\begin{esempio}
 Determina le condizioni di esistenza della seguente 
 espressione: $\sqrt x+\sqrt{x+1}$.

C.E. $\sqrt x$ esiste per $x\ge 0$, $\sqrt{x+1}$ 
esiste per $x+1\ge 0$, quindi per individuare le condizioni di esistenza 
dell'espressione occorre risolvere il sistema 
$\left\{\begin{array}{l} x\ge0\\ x+1\ge0\end{array}\right.
\Rightarrow\left\{\begin{array}{l}x\ge0\\x\ge-1\end{array}\right.$.

\begin{center}
 \input{\folder lbr//fig001_is.pgf}
\end{center}

In definitiva $\CE x\ge 0$.
\end{esempio}

\begin{esempio}
 Determina le condizioni di esistenza della radice 
 $\sqrt[4]{\dfrac{x-1}{x+1}}$.

C.E. $\dfrac{x-1}{x+1}\ge 0$. 
Occorre discutere il segno della frazione $f$, combinando il segno del 
numeratore $N$ e del denominatore $D$:

\begin{center}
 \input{\folder lbr//fig002_seg.pgf}
\end{center}
Pertanto C.E. $x<-1\vee x\ge 1$.
\end{esempio}
% \end{exrig}

\subsection{Operazioni con radicali letterali}
\label{subsec:radicali_operazioni_rad_lett}

Di seguito vengono presentati, sotto forma di esercizi svolti, alcuni 
argomenti già presentati per i radicandi aritmetici.

\paragraph{Semplificazione di radici}

\begin{esempio}
 Se abbiamo la radice:~$\sqrt[4]{a^2}$ potremmo pensare di semplificarla in:
 
 \[\sqrt[4]{a^2}=a^{\frac{2}{4}}=a^{\frac{1}{2}}=\sqrt{a}\]
 
Ora, se $a=+5$ ci ritroviamo esattamente come nel primo degli esempi 
precedenti. Se $a=-5$ lo svolgimento precedente ci darebbe: $\sqrt {-5}$ 
che non è un numero reale perché la radice ha indice pari e il radicando 
è negativo, mentre l'espressione iniziale equivale ad un numero reale dato 
che il radicando, essendo elevato alla seconda è senz'altro positivo.
Possiamo risolvere questo errore utilizzando il valore assoluto.
\end{esempio}

Nel caso di radicandi letterali la regola della semplificazione delle 
radici diventa:

\begin{proposizione}
 \[\sqrt[nt]{a^{mt}}=\left\{\begin{array}{l}
   \sqrt[n]{a^m} \text{ se }t \text{ è dispari}\\
   \sqrt[n]{\left|a^m\right|}\text{, se }t\text{ è pari}
 \end{array}\right.\]
\end{proposizione}

% \begin{exrig}
 \begin{esempio}
Semplificazione di radici con espressione letterale come radicando.
\begin{itemize}
\item $\sqrt{4x^4y^2a^6}=\sqrt{2^2x^4y^2a^6}=2x^2\left|ya^3\right|$: 
 abbiamo semplificato per $2$ sia l'indice della radice che l'esponente del 
 radicando;
\item $\sqrt[12]{a^2+2a+1}=\sqrt[12]{(a+1)^2}=\sqrt[6]{\left|a+1\right|}$: 
 dopo aver riconosciuto che il radicando è il quadrato del binomio, 
 abbiamo semplificato per $2$ indice ed esponente;
\item $\sqrt{x^2y^2}=\valass{xy}$
\item $\sqrt{x^2+2xy+y^2}=\sqrt{(x+y)^2}=\valass{x+y}$
\item $\sqrt{x^2+y^2}$ non è semplificabile perché il radicando non può 
essere 
 espresso sotto forma di potenza;
\item $\sqrt[6]{(x-1)^2}=\sqrt[3]{\valass{x-1}}$
\end{itemize}
 \end{esempio}
% \end{exrig}

\paragraph{Moltiplicazione e divisione di radici}

Se abbiamo radicali letterali, prima di operare, è necessario determinare 
le condizioni di esistenza: il prodotto di due radicali esiste là dove sono 
soddisfatte le condizioni di esistenza di tutti i fattori; 
il quoziente esiste là dove sono soddisfatte le condizioni di esistenza di 
dividendo e divisore, con il divisore diverso da zero.

Una volta verificate le condizioni di esistenza, possiamo trasformare le
radici in potenze con esponente frazionario, operare su queste espressioni 
utilizzando le proprietà delle potenze e ritrasformare il risultato in 
radici.

\begin{esempio}
$\sqrt{2a}\cdot \sqrt{\frac a b}:\sqrt{\frac {2b} 9}=
       \sqrt{2a}\cdot \sqrt{\frac a b}:\sqrt{\frac {2b} 9}=
       \sqrt{2a\cdot \frac a b\cdot \frac 9{2b}}=\sqrt{\frac{9a^2}{b^2}}=
       \frac {3a} b 
      (\CE a\ge 0\wedge b>0)$.
\end{esempio}

\begin{esempio}
Prodotto di radici con radicandi letterali.
 $\dfrac{\sqrt[3]{x^2y}\cdot \sqrt{xy}}{\sqrt[6]{x^2y^3}}$
 
 Il campo di esistenza è: $\CE x>0\wedge y>0$.
 
\begin{align*}
\dfrac{\sqrt[3]{x^2y}\cdot \sqrt{xy}}{\sqrt[6]{x^2y^3}} &=
\frac{{x^\frac{2}{3} \cdot y^\frac{1}{3}} \cdot 
       x^\frac{1}{2} \cdot y^\frac{1}{2}}
     {x^\frac{2}{6} \cdot y^\frac{3}{6}} =
         & \mbox{scrittura sotto forma di potenze}\\
    &=x^\frac{2}{3} \cdot x^\frac{1}{2} : x^\frac{2}{6} \cdot 
      y^\frac{1}{3} \cdot y^\frac{1}{2} : y^\frac{3}{6}=
         & \mbox{proprietà commutativa} \\
    &=x^{\frac{2}{3} + \frac{1}{2} - \frac{2}{6}} \cdot 
      y^{\frac{1}{3} + \frac{1}{2} - \frac{3}{6}}= 
         & \mbox{proprietà delle potenze}\\
    &=x^{\frac{4+3-2}{6}} \cdot y^{\frac{2+3-3}{6}}=
         & \mbox{calcolo degli esponenti} \\
    &=x^{\frac{5}{6}} \cdot y^{\frac{2}{6}}=
    \sqrt[6]{x^5y^2}
\end{align*}

% \[\frac{\sqrt[3]{x^2y}\cdot \sqrt{\mathit{xy}}}{\sqrt[6]{x^2y^3}}=
%         \sqrt[6]{\frac{\left(x^2y\right)^2\cdot (xy)^3}{x^2y^3}}=
%         \sqrt[6]{\frac{x^4y^2x^3y^3}{x^2y^3}}=
%         \sqrt[6]{\frac{x^7y^5}{x^2y^3}}=
%         \sqrt[6]{x^5y^2}.
% \]
\end{esempio}

% \begin{esempio}
%  $\sqrt[3]{\dfrac{ax+a}{x^2+2x+1}}\cdot \sqrt{\dfrac{x^2-2x+1}{ax-a}}$
%  
% \begin{enumeratea}
% \item Scomponiamo in fattori i radicandi 
%  $\sqrt[3]{\dfrac{a(x+1)}{(x+1)^2}}\cdot \sqrt{\dfrac{(x-1)^2}{a(x-1)}}$
% \item $\CE$\, $x+1\neq 0\wedge a(x-1)>0\Rightarrow x\neq -1\wedge 
%  ((a>0\wedge x>1)\vee (a<0\wedge x<1))$
% \item Semplifichiamo le frazioni di ciascun radicando 
%  $\sqrt[3]{\dfrac a{x+1}}\cdot \sqrt{\dfrac{x-1} a}$
% \item Trasformiamo nello stesso indice: il~$\mcm$ degli indici è~$6$, 
% quindi:
% \[
% \sqrt[6]{\left(\dfrac a{x+1}\right)^2}\cdot 
% \sqrt[6]{\left(\dfrac{x-1} a\right)^3}=
%  \sqrt[6]{\dfrac{a^2}{(x+1)^2}\cdot \dfrac{(x-1)^3}{a^3}}=
%  \sqrt[6]{\dfrac{(x-1)^3}{a(x+1)^2}}
% \]
% \end{enumeratea}
% \end{esempio}
% 
% \begin{esempio}
%  $\sqrt[3]{\dfrac{x^2}{x^2-2x+1}}:\sqrt[4]{\dfrac{x^4-2x^2+1}{x^2-1}}$.
% 
%  \begin{enumeratea}
% \item Scomponiamo in fattori i radicandi
%  $\sqrt[3]{\dfrac{x^2}{(x-1)^2}}:
%  \sqrt[4]{\dfrac{(x-1)^2\cdot (x+1)^2}{(x+1)(x-1)}}$
% \item $\CE$\, $(x-1)(x+1)>0\Rightarrow x<-1\vee x>1$. 
%  L'operazione che dobbiamo eseguire è una divisione e dunque il divisore 
% deve 
%  essere diverso da zero, quindi $x\neq -1\wedge x\neq 1$, comunque già 
%  implicite nelle C.E. trovate;
%  
% \begin{center}
%  \input{\folder lbr//fig003_seg.pgf}
% \end{center}
% \item Semplifichiamo i radicandi 
%  $\sqrt[3]{\dfrac{x^2}{(x-1)^2}}:\sqrt[4]{(x-1)\cdot (x+1)}$
% \item Riduciamo allo stesso indice: il $\mcm$ degli indici è $12$, quindi:
% 
% $\sqrt[12]{\left[\frac{x^2}{(x-1)^2}\right]^4}:\sqrt[12]{(x-1)^3 (x+1)^3}
% \Rightarrow 
% \sqrt[12]{\frac{x^8}{(x-1)^8}\cdot \frac 1{(x-1)^3 (x+1)^3}}=
% \sqrt[12]{\frac{x^8}{(x-1)^{11}(x+1)^3}}$.
% \end{enumeratea}
% \end{esempio}
% \end{exrig}

\paragraph{Portare un fattore sotto il segno di radice}

 \begin{esempio}
 Portare una espressione letterale dentro il segno di radice.
 \begin{itemize}
 \item $a\cdot \sqrt[3]b=\sqrt[3]{a^3b}$\, l'indice della radice è dispari 
  pertanto si porta sotto radice senza alcuna condizione;
 \item $(x-1)\cdot \sqrt[3]x=\sqrt[3]{(x-1)^3\cdot x}$\, l'indice della 
radice 
  è dispari, non sono necessarie condizioni sulla $x$
 \item $(x-2)\sqrt y$\, osserviamo che il radicale esiste per $y\ge 0$.
  Per portare dentro il segno di radice il coefficiente $(x-2)$ bisogna 
fare 
  la distinzione:
 \[
 (x-2)\sqrt y=\left\{\begin{array}{l}\sqrt{(x-2)^2y}\text{, se }x\ge 2\\
 -(2-x)\sqrt y=-\sqrt{(2-x)^2y}\text{, se }x<2;\end{array}\right.
 \]
 \item $(x-1)\sqrt{x-2}$.\, Il radicale esiste per $x-2\ge 0\ \to \ x\ge 
2$, 
 per questi valori il coefficiente esterno $(x-1)$ è positivo e può essere 
 portato dentro la radice: \[(x-1)\sqrt{x-2}=\sqrt{(x-1)^2(x-2)};\]
 \item $\frac{a-1}{a+3}\cdot \sqrt{\frac{a+2}{(a-1)^2}}$. Determiniamo le 
 condizioni di esistenza del radicale: per l'esistenza della frazione 
 $\frac{a+2}{(a-1)^2}$ deve essere $(a-1)^2\neq 0$, ovvero $a\neq 1$. 
 Affinché il radicando sia positivo o nullo, essendo il denominatore 
sempre positivo (ovviamente per $a\neq 1$) è sufficiente che sia 
$a+2\geqslant 0$ ovvero $a\geqslant -2$. 
 Pertanto le condizioni di esistenza sono~ $a\geqslant -2$ e $a\neq 1$.

 Studiamo ora il segno della frazione algebrica da portare sotto radice: 
 tale frazione è positiva o nulla per $a<-3\vee a\geqslant 1$, è negativa 
 per $-3<a\leqslant 1$.

 Se $a>1$ si ha 
 $\frac{a-1}{a+3}\cdot \sqrt{\frac{a+2}{(a-1)^2}}=
  \sqrt{\frac{(a-1)^2}{(a+3)^2}\cdot \frac{a+2}{(a-1)^2}}=
  \sqrt{\frac{a+2}{(a+3)^2}}$.

 Se $-2<a<1$ il fattore da portare sotto radice è negativo, quindi:
 \[-\left(-\frac{a-1}{a+3}\right)\cdot \sqrt{\frac{a+2}{(a-1)^2}}=
   -\sqrt{\frac{[-(a-1)]^2}{(a+3)^2}\cdot \frac{a+2}{(a-1)^2}}=
   -\sqrt{\frac{a+2}{(a+3)^2}}\]

 Se $a=-2$ l'espressione da calcolare vale zero mentre il caso $a=1$ è 
escluso  dalla condizione di esistenza.
 \end{itemize}
 \end{esempio}
% \end{exrig}

\paragraph{Portare un fattore fuori dal segno di radice}

Quando portiamo fuori dalla radice un termine letterale dobbiamo verificare 
se l'indice della radice è pari o dispari e se il termine che portiamo 
fuori è positivo o negativo. In particolare:
\[\sqrt[n]{a^nb}=\left\{\begin{array}{l}
 a\sqrt[n]b\text{, se }n\text{ dispari;}\\
 \valass a\sqrt[n]b\text{, se }n\text{ pari.}\end{array}\right.\]

% \begin{exrig}
 \begin{esempio}
 Portare una espressione letterale fuori dal segno di radice.
\begin{itemize}
 \item $\sqrt{2a^2}=\valass a\sqrt{2}$ bisogna mettere $a$ in valore 
assoluto perché sotto radice poteva essere sia negativo che positivo, la 
radice invece deve essere sempre positiva; se $a<0$ la 
relazione~$\sqrt{2a^2}=a\sqrt{2}$ è  errata;
 \item $\sqrt[3]{a^5b^7cd^3}$ occorre eseguire le divisioni intere tra gli 
 esponenti e l'indice della radice. Cominciamo da $a^5$ risulta $5:3 = 1$ 
con resto uguale a $2$ per $b^7$ si ha $7:3$ con quoziente~$2$ e resto~$1$ 
 l'esponente di $c$ è minore dell'indice; per $d^3$ si ha $3:3$ con 
quoziente  $1$ e resto~$0$. In definitiva 
 $\sqrt[3]{a^5b^7cd^3}={ab}^2 d\sqrt[3]{a^2{bc}}$, o anche:
\[\sqrt[3]{a^5b^7cd^3}=\sqrt[3]{(a^3a^2)(b^6b)cd^3}=
  \sqrt[3]{a^3b^6d^3}\cdot \sqrt[3]{a^2bc}=ab^2d^3\sqrt[3]{a^2bc}\]
In questo caso non c'è da mettere il valore assoluto perché l'indice della 
radice è dispari;
 \item $\sqrt[3]{\dfrac{3^3x^3y}{z^6}}$,\, 
       $\CE z\neq 0\,\sqrt[3]{\dfrac{3^3x^3y}{z^6}}=3\dfrac x{z^2}\sqrt[3]y$
 \item \(\sqrt[4]{4x^4(1-x)}\)
  \[\sqrt[4]{4x^4-4x^5}=\sqrt[4]{4x^4(1-x)}=\valass x\sqrt[4]{4(1-x)}=
    \left\{\begin{array}{l}
    x\sqrt[4]{4(1-x)}\text{, se }0\le x\le 1\\
    -x\sqrt[4]{4(1-x)}\text{, se }x<0.\end{array}\right.\]
 \item $\sqrt{3(a-1)^2}=\valass{a-1}\sqrt{3}=\left\{\begin{array}{l}
   -(a-1)\sqrt{3}\text{, se }a < 1 \\
   (a-1)\sqrt{3}\text{, se }a \ge 1
 \end{array}\right.$
\end{itemize}
 \end{esempio}
% \end{exrig}


\paragraph{Potenza di radice e radice di radice}

% \begin{exrig}
 \begin{esempio}
 Potenza di radice.
 \begin{multicols}{2}
 \begin{itemize}
 \item $\left(\sqrt[3]{2ab^2c^3}\right)^2=\sqrt[3]{4a^2b^4c^6}$
 \item $\sqrt[3]{\sqrt[4]{2x}}=\sqrt[12]{2x}$
 \end{itemize}
 \end{multicols}
 \end{esempio}
% \end{exrig}

\paragraph{Somma di radicali}

% \begin{exrig}
 \begin{esempio}
Esegui le seguenti somme di radicali.
\begin{itemize}
 \item $2\sqrt a+3\sqrt a=5\sqrt a,\,\CE a\ge 0$
 \item $\sqrt[4]{a^5}+\sqrt[4]{a^3}\cdot \sqrt 
a+\sqrt[4]{a^6}:\sqrt[4]a$. 
  Poniamo le condizioni di esistenza $a>0$ e svolgiamo i calcoli: 
  $\sqrt[4]{a^5}+\sqrt[4]{a^3\cdot a^2}+\sqrt[4]{a^6:a}=
   \sqrt[4]{a^5}+\sqrt[4]{a^5}+\sqrt[4]{a^5}=3\sqrt[4]{a^5}=
   3\sqrt[4]{a^4\cdot a}=3a\sqrt[4]a$.
\end{itemize}
 \end{esempio}
% \end{exrig}

\paragraph{Razionalizzazione del denominatore di una frazione}

% \begin{exrig}
 \begin{esempio}
Razionalizza il denominatore della seguente espressione.

$\dfrac{a^2-1}{\sqrt{a-1}}=
        \dfrac{(a^2-1)\sqrt{a-1}}{\sqrt{a-1}\sqrt{a-1}}=
        \dfrac{(a^2-1)\sqrt{a-1}}{a-1}=
        \dfrac{(a-1)(a+1)\sqrt{a-1}}{a-1}=(a+1)\sqrt{a-1}$.
\ \end{esempio}

 \begin{esempio}
Razionalizza il denominatore delle seguenti espressioni.
\begin{itemize}
 \item $\dfrac{ab}{\sqrt[4]{xa^2b^3}}$:
 il fattore razionalizzante è $\sqrt[4]{x^3a^2b}$
 quindi: 
 \[\dfrac{ab}{\sqrt[4]{xa^2b^3}}=
 \dfrac{ab\cdot \sqrt[4]{x^3a^2b}}{\sqrt[4]{xa^2b^3}\cdot 
\sqrt[4]{x^3a^2b}}=
 \dfrac{ab\sqrt[4]{x^3a^2b}}{\sqrt[4]{x^4a^4b^4}}=
 \dfrac{ab\sqrt[4]{x^3a^2b}}{xab}=\dfrac{\sqrt[4]{x^3a^2b}} x;\]
 \item $\dfrac 1{\sqrt[3]{b^5}}=\dfrac 1{b\sqrt[3]{b^2}}=
        \dfrac{1\cdot \sqrt[3]b}{b\sqrt[3]{b^2}\cdot \sqrt[3]b}=
        \dfrac{\sqrt[3]b}{b^2}$.
\end{itemize}
 \end{esempio}
 
 \begin{esempio}
Razionalizza il denominatore della seguente espressione.

$\dfrac{1+\sqrt a}{1-\sqrt a}=
 \dfrac{(1+\sqrt a)\cdot (1+\sqrt a)}{(1-\sqrt a)(1+\sqrt a)}=
 \dfrac{(1+\sqrt a)^2}{1-\sqrt{a^2}}=\dfrac{1+2\sqrt a+a}{1-a}$ 
 con $a\ge 0\wedge a\neq 1$
 \end{esempio}
% \end{exrig}
